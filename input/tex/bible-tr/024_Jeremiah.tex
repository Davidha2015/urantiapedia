\begin{document}

\title{Yeremya}


\chapter{1}

\par 1 Benyamin topraklarinda Anatot Kenti'ndeki kâhinlerden* Hilkiya oglu Yeremya'nin sözleri.
\par 2 RAB, Yahuda Krali Amon oglu Yosiya'nin kralliginin on üçüncü yilinda Yeremya'ya seslendi.
\par 3 RAB'bin Yeremya'ya seslenisi Yahuda Krali Yosiya oglu Yehoyakim'in döneminden, Yahuda Krali Yosiya oglu Sidkiya'nin kralliginin on birinci yilinin besinci ayina* dek, yani Yerusalim halkinin sürgüne gönderilmesine dek sürdü.
\par 4 RAB bana söyle seslendi:
\par 5 "Ana rahminde sana biçim vermeden önce tanidim seni. Dogmadan önce seni ayirdim, Uluslara peygamber atadim."
\par 6 Bunun üzerine, "Ah, Egemen RAB, konusmayi bilmiyorum, çünkü gencim" diye karsi çiktim.
\par 7 RAB, "'Gencim deme" dedi, "Seni gönderecegim herkese gidecek, sana buyuracagim her seyi söyleyeceksin.
\par 8 Onlardan korkma, çünkü seni kurtarmak için ben seninleyim." Böyle diyor RAB.
\par 9 Sonra RAB elini uzatip agzima dokundu, "Iste sözlerimi agzina koydum" dedi,
\par 10 "Bak, uluslarin ve ülkelerin kökünden sökülmesi, yikilip yok olmasi, yerle bir edilmesi, kurulup dikilmesi için bugün sana yetki verdim."
\par 11 RAB, "Yeremya, ne görüyorsun?" diye seslendi. "Bir badem dali görüyorum" diye yanitladim.
\par 12 RAB, "Dogru gördün" dedi, "Çünkü sözümü yerine getirmek için gözlemekteyim*fa*."
\par 13 RAB yine, "Ne görüyorsun?" diye seslendi. "Kuzeyden bu yöne bakan, kaynayan bir kazan görüyorum" diye yanitladim.
\par 14 RAB söyle dedi: "Ülkede yasayanlarin tümü üzerine Kuzeyden felaket saliverilecek*fb*.
\par 15 Çünkü kuzey kralliklarinin bütün halklarini çagiriyorum" diyor RAB. "Krallari gelip Yerusalim surlarinda, Bütün Yahuda kentlerinin karsisinda, Yerusalim'in kapi girislerinde Tahtlarini kuracaklar.
\par 16 Yaptiklari kötülükten ötürü Halkimin cezasini bildirecegim: Beni biraktilar, Baska ilahlara buhur yakip Elleriyle yaptiklarina tapindilar.
\par 17 "Sen kalk, hazirlan! Sana buyuracagim her seyi onlara söyle. Onlardan yilma! Yoksa onlarin önünde ben seni yildiririm.
\par 18 Iste, bütün ülkeye -Yahuda krallarina, önderlerine, kâhinlerine, ülke halkina- karsi bugün seni surlu bir kent, demir bir direk, tunç* bir duvar kildim.
\par 19 Sana savas açacak, ama seni yenemeyecekler. Çünkü seni kurtarmak için ben seninleyim." Böyle diyor RAB.

\chapter{2}

\par 1 RAB bana söyle seslendi:
\par 2 "Git, sunlari Yerusalim halkina duyur. RAB diyor ki, "'Gençligindeki bagliligini, Gelinligindeki sevgini, Çölde, ekilmemis toprakta Beni nasil izledigini animsiyorum.
\par 3 Israil RAB için kutsal bir halk, Hasadinin ilk ürünüydü. Onu yeren herkes suçlu sayilir, Basina felaket gelirdi" diyor RAB.
\par 4 RAB'bin sözünü dinleyin, Ey Yakup soyu, Israil'in bütün boylari!
\par 5 RAB diyor ki, "Atalariniz bende ne haksizlik buldular da Benden uzaklastilar? Degersiz putlari izleyerek Kendileri de degersiz oldular.
\par 6 'Misir'dan bizi çikaran, Çölde, çukurlarla dolu çorak toprakta, Koyu karanlikta kalan kurak toprakta, Kimsenin geçmedigi, Kimsenin yasamadigi toprakta Bize yol gösteren RAB nerede? diye sormadilar.
\par 7 Meyvesini, en iyi ürününü yiyesiniz diye Sizi verimli bir ülkeye getirdim. Oysa siz gelir gelmez ülkemi kirlettiniz, Mülkümü igrenç bir yere çevirdiniz.
\par 8 Kâhinler, 'RAB nerede? diye sormadilar, Kutsal Yasa uzmanlari beni tanimadilar, Yöneticiler bana baskaldirdilar; Peygamberler Baal* adina peygamberlik edip Ise yaramaz putlarin ardinca gittiler.
\par 9 "Bu yüzden sizden yine davaci olacagim" diyor RAB, "Torunlarinizdan da davaci olacagim.
\par 10 Gidin de Kittim kiyilarina bakin! Kedar ülkesine adam gönderip iyice inceleyin, Hiç böyle bir sey oldu mu, olmadi mi görün.
\par 11 Hiçbir ulus ilahlarini degistirdi mi? -Ki onlar zaten tanri degildirler- Ama benim halkim görkemini Ise yaramaz putlara degisti.
\par 12 Ey gökler, sasin buna, Tir tir titreyin, sasakalin" diyor RAB.
\par 13 "Çünkü halkim iki kötülük yapti: Beni, diri sularin pinarini birakti, Kendilerine sarniçlar, Su tutmayan çatlak sarniçlar kazdilar.
\par 14 Israil usak mi? Köle olarak mi dogdu? Öyleyse neden gümbür gümbür kükreyen Genç aslanlara av oldu? Ülkeyi viraneye çevirdiler, Kentler yerle bir edildi, kimsesiz birakildi!
\par 16 Nof ve Tahpanhes halki Kafani kirdi.
\par 17 Seni yolda yürüten Tanrin RAB'bi birakmakla Basina bunlari getirdin.
\par 18 Simdi Sihor suyundan içmek için Misir'a gitmek size yarar saglar mi? Firat suyundan içmek için Asur'a gitmek size ne saglar?
\par 19 Seni kendi kötülügün yola getirecek, Dönekligin seni paylayacak. Tanrin RAB'bi birakmanin, Benden korkmamanin Ne kadar kötü, ne kadar aci oldugunu gör de anla." Her Seye Egemen Egemen RAB böyle diyor.
\par 20 "Boyundurugunu çok önce kirdin, Baglarini kopardin. 'Kulluk etmeyecegim dedin. Gerçekten de her yüksek tepede, Her bol yaprakli agacin altinda Fahise gibi yatip kalktin.
\par 21 Oysa ben seni en iyi cinsten Seçme bir asma olarak dikmistim. Nasil oldu da yozlasip yabanil asmaya döndün?
\par 22 Çamasir sodasiyla yikansan, Bol kül suyu kullansan bile, Suçun önümde yine leke gibi duruyor" Diyor Egemen RAB.
\par 23 "Öyleyken nasil, 'Ben kirlenmedim, Baallar'i* izlemedim diyebilirsin? Vadide nasil davrandigina bak da Ne yaptigini anla. Sen orada burada dolasan Ayagi tez bir disi devesin.
\par 24 Kösnüyüp havayi koklayan Kira aliskin yaban esegisin. Azginken kim tutabilir onu? Pesine düsenlerin yorulmasi gerekmez, Çiftlesme zamani gelince onu bulurlar.
\par 25 Yalinayak kosmaktan sakin, Susuzluktan bogazini koru. Ama sen, 'Bos ver! Ben baska ilahlari seviyorum, Onlari izleyecegim dedin.
\par 26 "Hirsiz yakalandiginda nasil utanirsa, Israil'in halki, krallari, önderleri, Kâhinleri, peygamberleri de öyle utanacak.
\par 27 Onlar agaca, 'Babamsin, Tasa, 'Bizi sen dogurdun derler. Çünkü bana yüzlerini degil, Sirtlarini çevirdiler. Ama felakete ugrayinca, 'Kalk da bizi kurtar diye yakarirlar.
\par 28 Hani nerede kendiniz için yaptiginiz ilahlar? Felakete ugradiginizda kurtarabiliyorlarsa, Kalkip gelsinler. Kentlerinin sayisi kadar Ilahlarin var, ey Yahuda halki."
\par 29 "Neden bana dava açiyorsunuz? Hepiniz bana baskaldirdiniz" diyor RAB.
\par 30 "Halkinizi bosuna cezalandirdim, yola gelmediler. Kiliciniz yirtici aslan gibi öldürdü peygamberlerinizi.
\par 31 "Ey siz, bu kusagin çocuklari, RAB'bin sözünü anlayin! Ben Israil için bir çöl, Kapkaranlik bir ülke mi oldum? Öyleyse halkim neden, 'Basimiza buyruguz, Artik sana dönmeyecegiz diyor?
\par 32 Erden kiz takilarini, Gelin çeyizini unutabilir mi? Ama halkim sayisiz günlerce unuttu beni.
\par 33 Aski kovalamakta Ne kadar beceriklisin! Kötü kadinlara bile kendi yöntemlerini ögretebildin.
\par 34 Etegin suçsuz yoksullarin kaniyla lekelenmis, Oysa ev soyarken yakalamadin onlari. Bütün bunlara karsin,
\par 35 'Ben suçsuzum, Kuskusuz RAB'bin bana öfkesi dindi diyorsun. Ama 'Günah islemedim dedigin için Yargilayacagim seni.
\par 36 Neden boyuna döneklik yapip duruyorsun? Asur'da düskirikligina ugradigin gibi, Misir'da da düskirikligina ugrayacaksin.
\par 37 Oradan da ellerin basinda çikacaksin, Çünkü RAB senin güvendiklerini reddetti; Onlardan yarar saglamayacaksin."

\chapter{3}

\par 1 "Diyelim ki, bir adam karisini bosar, Kadin da onu birakip baska biriyle evlenir. Adam bir daha o kadina döner mi? Bu davranis ülkeyi büsbütün kirletmez mi? Oysa sen pek çok oynasla fahiselik ettin, Yine bana mi dönmek istiyorsun?" diyor RAB.
\par 2 "Çiplak tepelere bak da gör. Sevismedigin yer mi kaldi? Çölde yasayan bedevi gibi Yol kenarlarinda oynaslarini bekleyip durdun. Fahiseliginle, kötülüklerinle ülkeyi kirlettin.
\par 3 Bu yüzden yagmurlarin ardi kesildi, Son yagmur yagmadi. Yüzsüz bir fahiseye benzedin, Utanç duymak istemedin.
\par 4 'Baba, gençligimden beri Benim dostumsun diye az önce bana seslenmedin mi?
\par 5 'Sonsuza dek kizgin mi kalacaksin? Öfken sonsuza dek mi sürecek? Evet, böyle konusuyor, Ama elinden gelen her kötülügü yapiyorsun."
\par 6 Kral Yosiya döneminde RAB bana, "Dönek Israil'in yaptigini gördün mü?" dedi, "Her yüksek tepenin üzerine, her bol yaprakli agacin altina gidip fahiselik etti.
\par 7 Bütün bunlari yaptiktan sonra bana geri dönecegini düsündüm, ama dönmedi. Hain kizkardesi Yahuda da gördü bunlari.
\par 8 Fahiseligi yüzünden dönek Israil'i bosayip ona bosanma belgesini verdigim halde, kizkardesi hain Yahuda'nin hiç korkmadigini, gidip fahiselik ettigini gördüm.
\par 9 Hiç umursamadan fahiseligiyle ülkeyi kirletti; tasla, agaçla zina etti.
\par 10 Bütün bunlara karsin, hain kizkardesi Yahuda içtenlikle degil, göstermelik olarak bana döndü." Böyle diyor RAB.
\par 11 RAB bana, "Dönek Israil hain Yahuda'dan daha dogru oldugunu gösterdi" dedi,
\par 12 "Git, bu sözleri kuzeye duyur. De ki, "'Ey dönek Israil, geri dön diyor RAB. 'Size artik öfkeyle bakmayacagim, Çünkü ben sevecenim diyor RAB. 'Öfkemi sonsuza dek sürdürmem.
\par 13 Ancak suçunu kabul et: Tanrin RAB'be baskaldirdin, Her bol yaprakli agacin altinda Sevgini yabanci ilahlarla paylastin, Beni dinlemedin." Böyle diyor RAB.
\par 14 "Geri dön, ey dönek halk" diyor RAB, "Çünkü kocan benim. Birinizi kentten, ikinizi bir boydan alip Siyon'a geri getirecegim.
\par 15 Size gönlüme göre çobanlar verecegim; sizi bilgiyle, sagduyuyla güdecekler.
\par 16 Ülkede büyüyüp sayica çogaldiginiz günlerde" diyor RAB, "Halk artik, 'RAB'bin Antlasma Sandigi* demeyecek. Sandik bir daha kimsenin aklina gelmeyecek; animsanmayacak, özlenmeyecek, bir yenisi de yapilmayacak.
\par 17 O zaman Yerusalim'e, 'RAB'bin Tahti diyecekler. RAB'bin adini onurlandirmak için bütün uluslar Yerusalim'de toplanacak. Bundan böyle kötü yüreklerinin inadi uyarinca davranmayacaklar.
\par 18 O günlerde Yahuda halkiyla Israil halki kuzeyde bir ülkeden birlikte yürüyecek, atalarina mülk olarak vermis oldugum ülkede bir araya gelecekler.
\par 19 "Ben RAB, demistim ki, 'Ne kadar isterdim Seni çocuklarimdan saymayi; Sana güzel ülkeyi, Uluslarin en güzel mülkünü vermeyi! Bana baba diyecegini, Benden hiç ayrilmayacagini sandim.
\par 20 Ama bir kadin kocasina nasil ihanet ederse, Sen de bana öyle ihanet ettin, ey Israil halki!" Böyle diyor RAB.
\par 21 Çiplak tepelerde bir ses duyuluyor, Israil halkinin aglayisi ve yakarisi. Çünkü dogru yoldan saptilar, Tanrilari RAB'bi unuttular.
\par 22 "Geri dönün, ey dönek çocuklar, Dönekliginizi iyilestireyim." Halk, "Iste buradayiz, sana geliyoruz!" diyor, "Çünkü Tanrimiz RAB sensin.
\par 23 Kuskusuz daglardan, Tepelerden gelen tapinma sesleri aldaticidir. Kuskusuz Israil'in kurtulusu Tanrimiz RAB'dedir.
\par 24 Gençligimizden bu yana Atalarimizin emeginin ürününü, Davarlarini, sigirlarini, Ogullarini, kizlarini Utanilasi putlar yedi.
\par 25 Utanç içinde yatalim, Rezilligimiz bizi örtsün! Çünkü biz de atalarimiz da Gençligimizden bu yana Tanrimiz RAB'be karsi günah isledik, Tanrimiz RAB'bin sesine kulak asmadik."

\chapter{4}

\par 1 "Eger geri dönersen, ey Israil, Eger bana geri dönersen" diyor RAB, "Igrenç putlarini gözümün önünden uzaklastirir, Bir daha yoldan sapmazsan;
\par 2 'RAB'bin varligi hakki için diyerek Sadakatle, adaletle, dogrulukla ant içersen, Uluslar O'nun araciligiyla kutsanacak, O'nunla övünecekler."
\par 3 RAB Yahuda ve Yerusalim halkina söyle diyor: "Isletilmemis topraginizi sürün, Dikenler arasina ekmeyin.
\par 4 Ey sizler, Yahuda halki ve Yerusalim'de yasayanlar, Kendinizi RAB'be adayin, Bunu engelleyen her seyi yüreginizden uzaklastirin. Yoksa yaptiginiz kötülüklerden ötürü Öfkem ates gibi yagacak, Her seyi yiyip bitirecek Ve söndüren olmayacak."
\par 5 "Yahuda'da duyurun, Yerusalim'de ilan edin, 'Ülkede boru çalin! deyin, 'Toplanin diye haykirin, 'Surlu kentlere kaçalim!
\par 6 Siyon'a giden yolu gösteren Bir isaret koyun! Güvenliginiz için kaçin! Durmayin! Üzerinize kuzeyden felaket, Büyük yikim getirmek üzereyim."
\par 7 Aslan ininden çikti, Uluslari yok eden yola koyuldu. Ülkenizi viran etmek için Yerinden ayrildi. Kentleriniz yerle bir edilecek, Içlerinde yasayan kalmayacak.
\par 8 Onun için çula sarinin, Dövünüp haykirin, Çünkü RAB'bin kizgin öfkesi üzerimizden kalkmadi.
\par 9 "O gün" diyor RAB, "Kral da önderler de yilacak, Kâhinler saskina dönecek, Peygamberler donakalacak."
\par 10 O zaman, "Ah, Egemen RAB" dedim, "'Esenlikte olacaksiniz diyerek bu halki da Yerusalim'i de tam anlamiyla aldattin. Çünkü kiliç bogazimiza dayandi."
\par 11 O zaman bu halka ve Yerusalim'e, "Çöldeki çiplak tepelerden halkima dogru sicak bir rüzgar esiyor, ama harman savurmak ya da ayirmak için degil" denecek,
\par 12 "Benden gelen bu rüzgar çok daha güçlü olacak. Simdi bu halka yargilarimi bildiriyorum."
\par 13 Iste düsman bulut gibi ilerliyor; Savas arabalari kasirga sanki, Atlari kartallardan daha çevik. Vay basimiza! Mahvolduk!
\par 14 Ey Yerusalim, yüregini kötülükten arindir ki, Kurtulasin. Ne zamana dek yüreginde kötü düsünceler barindiracaksin?
\par 15 Dan'dan bir ses bildiriyor, Efrayim daglarindan kötü haber duyuruyor!
\par 16 "Uluslara duyurun, Yerusalim'e bildirin: 'Uzak bir ülkeden gelen ordu çevresini kusatacak, Yahuda kentlerine karsi Savas naralari atacaklar.
\par 17 Bir tarlayi koruyanlar gibi Kusatacaklar Yerusalim'i. Çünkü Yerusalim bana baskaldirdi" diyor RAB.
\par 18 "Kendi davranislarin, kendi yaptiklarin Basina gelmesine neden oldu bunlarin. Cezan bu. Ne aci! Nasil da yüregine isliyor!"
\par 19 Ah, içim, içim! Acidan kivraniyorum. Ah, yüregim, yüregim çarpiyor. Sessiz duramiyorum! Çünkü boru sesini, savas naralarini isittim!
\par 20 Felaket felaketi izliyor, Bütün ülke viran oldu. Bir anda çadirlarim, Perdelerim yok oldu.
\par 21 Ne zamana dek düsman sancagini görmek, Boru sesini duymak zorunda kalacagim?
\par 22 "Halkim akilsizdir, Beni tanimiyor. Aptal çocuklardir, Akillari yok. Kötülük etmeyi iyi bilir, Iyilik etmeyi bilmezler" diyor RAB.
\par 23 Ben Yeremya yere baktim, sekilsizdi, bostu, Göge baktim, isik yoktu.
\par 24 Daglara baktim, titriyorlardi, Bütün tepeler sarsiliyordu.
\par 25 Baktim, insan yoktu, Gökte uçan bütün kuslar kaçmisti.
\par 26 Baktim, verimli toprak çöle dönmüs, Bütün kentler yikilmisti. Bütün bunlar RAB'bin yüzünden, O'nun kizgin öfkesi yüzünden olmustu.
\par 27 RAB diyor ki, "Bütün ülke viran olacak, Ama onu büsbütün yok etmeyecegim.
\par 28 Bu yüzden yeryüzü yasa gömülecek, Gök kararacak; Çünkü ben söyledim, ben tasarladim. Fikrimi degistirmeyecek, Verdigim karardan dönmeyecegim."
\par 29 Her kentin halki, Atlilarla okçularin gürültüsünden kaçiyor. Kimi çaliliklara giriyor, Kimi kayaliklara tirmaniyor. Bütün kentler terk edildi, Oralarda kimse yasamiyor.
\par 30 Ey sen, viran olmus kent, Kirmizi giysiler giymekle, Altin süsler bezenmekle, Gözüne sürme çekmekle ne elde edeceksin? Kendini böyle güzellestirmen bosuna. Oynaslarin seni küçümsüyor, Canini almak istiyorlar.
\par 31 Sanci çeken kadinin haykirisini, Ilk çocugunu doguran kadinin çektigi aciyi, Ellerini uzatmis, solugu kesilmis Siyon kizinin*, "Eyvah! Katillerin karsisinda bayiliyorum" Diye haykirdigini isitir gibi oldum.

\chapter{5}

\par 1 "Yerusalim sokaklarinda dolasin, Çevrenize bakip düsünün, Kent meydanlarini arastirin. Eger adil davranan, Gerçegi arayan bir kisi bulursaniz, Bu kenti bagislayacagim.
\par 2 'RAB'bin varligi hakki için deseler de, Aslinda yalan yere ant içiyorlar."
\par 3 Ya RAB, gözlerin gerçegi ariyor. Onlari vurdun, ama incinmediler, Onlari yiyip bitirdin, Ama yola gelmeyi reddettiler. Yüzlerini kayadan çok sertlestirdiler, Geri dönmek istemediler.
\par 4 "Bunlar sadece yoksul kisiler, Akilsizlar" dedim, "Çünkü RAB'bin yolunu, Tanrilari'nin buyruklarini bilmiyorlar.
\par 5 Büyüklere gidip onlarla konusayim. RAB'bin yolunu, Tanrilari'nin buyruklarini bilirler kuskusuz." Gelgelelim onlar da boyundurugu kirmis, Baglari koparmisti.
\par 6 Bu yüzden ormandan bir aslan çikip onlara saldiracak, Çölden gelen bir kurt onlari parça parça edecek, Bir pars kentlerinin önünde pusu kuracak, Oradan çikan herkes parçalanacak. Çünkü isyanlari çok, Döneklikleri sayisizdir.
\par 7 "Yaptiklarindan ötürü neden bagislayayim seni? Çocuklarin beni terk etti, Tanri olmayan ilahlarin adiyla ant içtiler. Onlari doyurdugumda zina ettiler, Fahiselerin evlerine dolustular.
\par 8 Sehvet düskünü, besili aygirlar! Her biri komsusunun karisina kisniyor.
\par 9 Bu yüzden onlari cezalandirmayayim mi?" diyor RAB, "Böyle bir ulustan öcümü almayayim mi?
\par 10 "Baglarini dolasip Asmalarini kesin, Ama büsbütün yok etmeyin. Dallarini koparip atin, Çünkü onlar RAB'be ait degil.
\par 11 Israil ve Yahuda halki Bana sürekli ihanet etti" diyor RAB.
\par 12 RAB için yalan söyleyerek, "O bir sey yapmaz. Felaket bize ugramayacak, Kiliç da kitlik da görmeyecegiz" dediler.
\par 13 Peygamberler lafebesidir, Tanri'nin sözü onlarda degil. Onlara böyle yapilacak.
\par 14 Bu yüzden, Her Seye Egemen RAB Tanri diyor ki, "Madem böyle seyler konusuyorsunuz, Ben de sözümü agziniza ates, Bu halki da odun edecegim; Ates onlari yakip yok edecek.
\par 15 Ey Israil halki, Uzaktan gelecek bir ulusu Üzerinize saldirtacagim" diyor RAB, "Köklü, eski bir ulus; Sen onlarin dilini bilmez, Ne dediklerini anlamazsin.
\par 16 Oklarinin kilifi açik bir mezar gibidir, Hepsi birer yigittir.
\par 17 Ürününü, yiyeceklerini tüketecek, Ogullarini, kizlarini öldürecekler; Davarlarini, sigirlarini, Asmalarinin, incir agaçlarinin meyvesini yiyecek, Güvendigin surlu kentlerini Kiliçla yerle bir edecekler.
\par 18 "Ama o günlerde bile sizi büsbütün yok etmeyecegim" diyor RAB.
\par 19 "'Tanrimiz RAB neden bize bütün bunlari yapti? Diye sorduklarinda, söyle yanitlayacaksin: 'Beni nasil biraktiniz, ülkenizde yabanci ilahlara nasil kulluk ettinizse, siz de kendinize ait olmayan bir ülkede yabancilara öyle kulluk edeceksiniz.
\par 20 "Yakup soyuna bildirin, Yahuda halkina duyurun:
\par 21 Ey gözleri olan ama görmeyen, Kulaklari olan ama isitmeyen, Sagduyudan yoksun akilsiz halk, Sunu dinle:
\par 22 Benden korkman gerekmez mi?" diyor RAB, "Huzurumda titremen gerekmez mi? Ben ki, sonsuza dek geçerli bir kuralla Denize sinir olarak kumu koydum. Deniz siniri geçemez; Dalgalar kabarsa da üstün gelemez, Kükrese de siniri asamaz.
\par 23 Ama bu halkin yüregi asi ve inatçi. Sapmislar, kendi yollarina gitmisler.
\par 24 Içlerinden, 'Ilk ve son yagmurlari zamaninda yagdiran, Belli ürün biçme haftalarini bizim için koruyan Tanrimiz RAB'den korkalim demiyorlar.
\par 25 Bunlari uzaklastiran suçlarinizdi, Bu iyilikten sizi yoksun birakan günahlarinizdi.
\par 26 "Halkim arasinda kötü kisiler var. Kus avlamak için pusuya yatanlar gibi Tuzak kuruyor, insan yakaliyorlar.
\par 27 Kus dolu bir kafes nasilsa, Onlarin evleri de hileyle dolu. Bu sayede güçlenip zengin oldular,
\par 28 Semirip parladilar, Yaptiklari kötülüklerle siniri astilar. Kazanabilecekleri halde öksüzün davasina bakmiyor, Yoksulun hakkini savunmuyorlar.
\par 29 Bu yüzden onlari cezalandirmayayim mi?" diyor RAB, "Böyle bir ulustan öcümü almayayim mi?
\par 30 "Ülkede korkunç, dehset verici bir sey oldu:
\par 31 Peygamberler yalan peygamberlik ediyor, Halki basina buyruk kâhinler yönetiyor, Halkim da bunu benimsiyor. Ama bunun sonunda ne yapacaksiniz?"

\chapter{6}

\par 1 "Güvenliginiz için kaçin, ey Benyamin halki! Yerusalim'den kaçin! Tekoa'da boru çalin! Beythakkerem'e bir isaret koyun. Çünkü kuzeyden bir felaket, Büyük bir yikim gelecek gibi görünüyor.
\par 2 Siyon kizini*, o güzel, narin kizi yok edecegim.
\par 3 Çobanlar sürüleriyle ona geliyor, Çevresinde çadirlarini kuracaklar. Herkes kendi sürüsünü otlatacak."
\par 4 "Yerusalim'e karsi savas hazirligi yapin! Kalkin, ögleyin saldiriya geçelim! Vay halimize, gün karariyor! Aksamin gölgeleri gitgide uzuyor.
\par 5 Haydi, gece saldiriya geçelim, Kentin kalelerini yerle bir edelim."
\par 6 Her Seye Egemen RAB diyor ki, "Agaçlari kesin, Yerusalim'e karsi kusatma rampalari yapin. Bu kent cezalandirilmali, Içinde zorbaliktan baska bir sey yok.
\par 7 Kuyu suyunu nasil taze tutuyorsa, Yerusalim de kötülügünü öyle taze tutuyor. Siddet ve yikim yankilaniyor orada, Karsimda hep hastalik ve yaralar var.
\par 8 Uyarilara kulak ver, ey Yerusalim! Yoksa seni birakacagim, Seni bir viraneye, Oturulmaz bir ülkeye çevirecegim."
\par 9 Her Seye Egemen RAB diyor ki, "Asmadan nasil üzüm toplanirsa, Israil halkindan geride kalanlari da öyle toplayacaklar. Üzüm toplayan biri gibi Elini yine asma dallarina uzat."
\par 10 Isitsinler diye kiminle konusayim, Kimi uyarayim? Kulaklari tikali, isitemiyorlar. RAB'bin sözünü asagiliyor, Ondan hoslanmiyorlar.
\par 11 Bu yüzden RAB'bin öfkesiyle doluyum, Kendimi tutmaktan yoruldum. "Sokaktaki çocuklarin, Toplanan gençlerin üzerine bosalt öfkeni. Nasil olsa kari da koca da, Yasli da yillarca yasamis olan da kurtulamayacak.
\par 12 Evleri, tarlalari, karilari Baskalarina verilecek, Çünkü ülkede yasayanlara karsi Elimi kaldiracagim" diyor RAB.
\par 13 "Küçük büyük herkes kazanç pesinde, Peygamberler, kâhinler, hepsi halki aldatiyor.
\par 14 Esenlik yokken, 'Esenlik, esenlik diyerek Halkimin yarasini sözde iyilestirdiler.
\par 15 Yaptiklari igrençliklerden utandilar mi? Hayir, ne utanmasi? Kizarip bozarmanin ne oldugunu bile bilmiyorlar. Bu yüzden onlar da düsenlerin arasinda yer alacak, Onlari cezalandirdigimda sendeleyip düsecekler" diyor RAB.
\par 16 RAB diyor ki, "Yol kavsaklarinda durup bakin, Eski yollari sorun, Iyi yol nerede, ögrenin, O yolda yürüyün, Canlariniz rahata kavusur. Ama onlar, 'O yolda yürümeyiz dediler.
\par 17 Size bekçiler atayip, 'Boru sesini dinleyin dedim, Ama onlar, 'Dinlemeyiz dediler.
\par 18 Bundan ötürü, ey uluslar, Baslarina neler gelecegini isitin! Sen de anla, ey topluluk!
\par 19 Dinle, ey yeryüzü! Bu halkin üzerine felaket, Kendi kurdugu düzenin sonucunu getirmek üzereyim. Çünkü sözlerime kulak asmadilar, Kutsal Yasam'i reddettiler.
\par 20 Neden bana Saba'dan günnük, Uzak bir ülkeden güzel kokulu kamis getiriliyor? Yakmalik sunularinizi* kabul etmiyorum, Kurbanlarinizdan hosnut degilim."
\par 21 Bu yüzden RAB diyor ki, "Bu halkin önüne tökezler koyacagim, Babalar da ogullar da Tökezleyip birlikte düsecek, Komsu dostuyla birlikte yok olacak."
\par 22 RAB diyor ki, "Iste kuzeyden bir ordu geliyor. Dünyanin uçlarindan Büyük bir ulus harekete geçiyor.
\par 23 Yay, pala kusanmislar, Gaddar ve acimasizlar. Atlara binmis gelirken, Kükreyen denizi andiriyor sesleri. Savasa hazir savasçilar Karsina dizilecekler, ey Siyon kizi!"
\par 24 Haberlerini aldik, Ellerimizde derman kalmadi. Doguran kadin gibi Üzüntü, sanci sardi bizi.
\par 25 Kirlara çikmayin, Yolda yürümeyin! Düsmanin kilici orada, Her yer dehset içinde.
\par 26 Ey halkim, çula sarin, Kül içinde yuvarlan. Biricik ogul için yas tutar gibi Aci aci dövün. Çünkü yok edici ansizin gelecek üzerimize.
\par 27 "Seni halkimi deneyesin diye atadim, Öyle ki, onlari taniyip yollarini sinayasin.
\par 28 Hepsi de çok dikbasli, Onu bunu çekistirerek dolasan insanlardir, Tunç* kadar, demir kadar katidirlar. Hepsi bastan çikmistir.
\par 29 Körük üfürdükçe üfürüyor, Kursunu ateste eritiyor, Ama bosunadir yapilan islem, Çünkü kötüler arinmiyor.
\par 30 Onlara gümüs artigi denecek, Çünkü RAB onlari reddetti."

\chapter{7}

\par 1 RAB Yeremya'ya söyle seslendi:
\par 2 "RAB'bin Tapinagi'nin kapisinda durup su sözü duyur. De ki, "'RAB'bin sözünü dinleyin, ey RAB'be tapinmak için bu kapilardan giren Yahuda halki!
\par 3 Israil'in Tanrisi, Her Seye Egemen RAB diyor ki: Yasantinizi ve uygulamalarinizi düzeltin. O zaman burada kalmanizi saglarim.
\par 4 "RAB'bin Tapinagi, RAB'bin Tapinagi, RAB'bin Tapinagi buradadir!" gibi aldatici sözlere güvenmeyin.
\par 5 Eger yasantinizi ve uygulamalarinizi gerçekten düzeltir, birbirinize karsi adil davranir,
\par 6 yabanciya, öksüze, dula haksizlik etmez, burada suçsuz kani akitmaz, sizi yikima götüren baska ilahlarin ardinca gitmezseniz,
\par 7 burada, sonsuza dek atalariniza vermis oldugum ülkede kalmanizi saglarim*fi*.
\par 8 Ne var ki, sizler ise yaramaz aldatici sözlere güveniyorsunuz.
\par 9 "'Çalmak, adam öldürmek, zina etmek, yalan yere ant içmek, Baal'a* buhur yakmak, tanimadiginiz baska ilahlarin ardinca gitmek, bütün bu igrençlikleri yapmak için mi bana ait olan tapinaga gelip önümde duruyor, güvenlikteyiz diyorsunuz?
\par 11 Bana ait olan bu tapinak sizin için bir haydut ini mi oldu? Ama ben görüyorum neler yaptiginizi! diyor RAB.
\par 12 "'Daha önce adimi yerlestirmis oldugum Silo'daki yerime gidin. Halkim Israil'in kötülügü yüzünden ona ne yaptigimi görün.
\par 13 Bütün bunlari yaptiniz, diyor RAB, size defalarca seslendim ama dinlemediniz; sizi çagirdim ama yanit vermediniz.
\par 14 Bu yüzden Silo'ya ne yaptimsa, bana ait olan, güvendiginiz bu tapinaga da -sizlere, atalariniza vermis oldugum bu yere de aynisini yapacagim.
\par 15 Kardeslerinizi, bütün Efrayim soyunu nasil attiysam, sizleri de öyle atacagim huzurumdan.
\par 16 "Sana gelince, ey Yeremya, bu halk için yalvarma; onlar için ne yakar ne de dilekte bulun; bana yalvarip yakarma, çünkü seni dinlemeyecegim.
\par 17 Onlarin Yahuda kentlerinde, Yerusalim sokaklarinda neler yaptiklarini görmüyor musun?
\par 18 Çocuklar odun topluyor, babalar ates yakiyor, kadinlar Gök Kraliçesi'ne pide pisirmek için hamur yoguruyor. Beni öfkelendirmek için baska ilahlara dökmelik sunular sunuyorlar.
\par 19 Incittikleri ben miyim, diyor RAB. Hayir, kendilerini inciterek utanca boguyorlar.
\par 20 "Bu yüzden Egemen RAB diyor ki, 'Buranin üzerine, insanin, hayvanin, kirdaki agaçlarin, topragin ürününün üzerine kizgin öfkemi yagdiracagim. Yakip yok edecek her seyi, sönmeyecek.
\par 21 "Israil'in Tanrisi, Her Seye Egemen RAB diyor ki, 'Yakmalik sunularinizi* öbür kurbanlariniza ekleyin de et yiyin.
\par 22 Çünkü atalarinizi Misir'dan çikardigimda, yakmalik sunularla kurbanlar hakkinda onlara seslenip buyruk vermedim.
\par 23 Onlara sunu buyurdum: Sözümü dinlerseniz, ben sizin Tanriniz, siz de benim halkim olursunuz. Iyilik bulmaniz için her konuda size buyurdugum yolda yürüyün.
\par 24 Ne var ki, dinlemediler, kulak asmadilar; kendi isteklerinin, kötü yüreklerinin inadi dogrultusunda yürüdüler. Ileri degil, geri gittiler.
\par 25 Atalarinizin Misir'dan çiktigi günden bu yana, size her gün defalarca peygamber kullarimi gönderdim.
\par 26 Ama beni dinlemediniz, kulak asmadiniz. Inat ederek atalarinizdan daha çok kötülük yaptiniz.
\par 27 "Onlara bütün bunlari söyleyeceksin ama seni dinlemeyecekler. Onlari çagiracaksin ama yanit vermeyecekler.
\par 28 Bunun için onlara de ki, 'Tanrisi RAB'bin sözünü dinlemeyen, ders almayan ulus iste budur. Bana bagliliklari yok oldu, bagliliktan söz etmez oldular.
\par 29 "'Saçini kes ve at, ey Yerusalim, Çiplak tepeler üzerinde agit yak. Çünkü RAB, öfkesine ugramis kusagi Reddedip terk etti."
\par 30 "'Yahuda halki gözümde kötü olani yapti, diyor RAB. Bana ait olan bu tapinaga igrenç putlarini yerlestirerek onu kirlettiler.
\par 31 Ogullarini, kizlarini ateste kurban etmek için Ben-Hinnom Vadisi'nde, Tofet'te* puta tapilan yerler kurdular. Böyle bir seyi ne buyurdum ne de aklimdan geçirdim.
\par 32 Bundan ötürü oraya artik Tofet ya da Ben-Hinnom Vadisi degil, Kiyim Vadisi denecegi günler geliyor, diyor RAB. Tofet'te yer kalmayana dek gömecekler ölüleri.
\par 33 Bu halkin ölüleri yirtici kuslara, yabanil hayvanlara yem olacak; onlari korkutup kaçiran kimse olmayacak.
\par 34 Yahuda kentlerinde, Yerusalim sokaklarinda sevinç ve nese sesine, gelin güvey sesine son verecegim; ülke viraneye dönecek.

\chapter{8}

\par 1 "'O zaman, diyor RAB, Yahuda krallariyla önderlerinin, kâhinlerin, peygamberlerin, Yerusalim'de yasamis olanlarin kemikleri mezarlarindan çikarilacak.
\par 2 Toplanmayacak, gömülmeyecek kemikler, topragin üzerinde gübre gibi olacaklar. Yerusalim halkinin sevdigi, kulluk ettigi, izledigi, danistigi, taptigi günesin, ayin, gök cisimlerinin önüne serilecekler.
\par 3 Bu kötü ulustan bütün sag kalanlar, kendilerini sürdügüm yerlerde yasayanlar, ölümü yasama yegleyecekler. Her Seye Egemen RAB böyle diyor.
\par 4 "Onlara de ki, 'RAB söyle diyor: "'Insan yere düser de kalkmaz mi, Yoldan sapar da geri dönmez mi?
\par 5 Öyleyse neden bu halk yoldan sapti? Neden Yerusalim sürekli döneklik ediyor? Hileye yapisiyor, Geri dönmeyi reddediyorlar.
\par 6 Dikkatle dinledim, Ama dogru söylemiyorlar. Kimse, ne yaptim, diyerek kötülügünden pismanlik duymuyor. Savasta segirten at gibi Herkes kendi yoluna gidiyor.
\par 7 Gökteki leylek bile Belli mevsimlerini bilir. Kumru da kirlangiç da turna da Göç etme zamanini gözetir. Oysa halkim buyruklarimi bilmez.
\par 8 "'Nasil, biz bilge kisileriz, RAB'bin Yasasi bizdedir, diyebiliyorsunuz? Iste, bilginlerin yalanci kalemi Yasayi yalana çevirmis.
\par 9 Bilgeler utandirildi, Yildirilip ele geçirildi. RAB'bin sözünü reddettiler. Nasil bir bilgelikmis onlarinki?
\par 10 Bundan ötürü karilarini baskalarina, Tarlalarini sahiplenecek yeni kisilere verecegim. Küçük büyük herkes kazanç pesinde, Peygamberler, kâhinler, hepsi halki aldatiyor.
\par 11 Esenlik yokken, Esenlik, esenlik, diyerek Halkimin yarasini sözde iyilestirdiler.
\par 12 Yaptiklari igrençliklerden utandilar mi? Hayir, ne utanmasi? Kizarip bozarmanin ne oldugunu bile bilmiyorlar. Bu yüzden onlar da düsenlerin arasinda yer alacak, Cezalandirildiklarinda sendeleyip düsecekler diyor RAB.
\par 13 "'Onlari büsbütün yok edecegim, diyor RAB, Ne asmada üzüm kalacak, Ne incir agacinda incir. Yapraklari solup kuruyacak. Onlara ne verdiysem, Ellerinden alinacak."
\par 14 "Neden burada oturup duruyoruz? Toplanalim da surlu kentlere kaçalim, Orada ölelim! Tanrimiz RAB bizi ölüme terk etti, Bize zehirli su içirdi. Çünkü O'na karsi günah isledik.
\par 15 Esenlik bekledik, iyilik gelmedi. Sifa umduk, yilginlik bulduk.
\par 16 Düsman atlarinin hiriltisi Dan bölgesinden duyuluyor, Aygirlarinin kisnemesinden Bütün ülke titriyor. Ülkeyi ve içindeki her seyi, Kenti ve orada yasayanlari Yok etmeye geliyorlar."
\par 17 "Bakin, araniza yilanlar, Büyüden etkilenmeyen engerekler gönderecegim, Sizi sokacaklar" diyor RAB.
\par 18 Üzüntüm avutulamaz, Yüregim baygin,
\par 19 Ülkenin en uzak köselerinden Halkimin feryadini dinleyin: "RAB Siyon'da degil mi? Krali orada degil mi?" RAB, "Putlariyla, Ise yaramaz yabanci ilahlariyla Neden öfkelendiriyorlar beni?" diyor.
\par 20 "Ürün biçme zamani geçti, Yaz sona erdi, Biz ise kurtulmadik" diye haykiriyorlar.
\par 21 Halkimin yarasindan ben de yaralandim. Yasa büründüm, dehsete düstüm.
\par 22 Gilat'ta merhem yok mu, Hekim yok mu? Öyleyse halkimin yarasi neden iyi edilmedi?

\chapter{9}

\par 1 Keske basim bir pinar, Gözlerim bir gözyasi kaynagi olsa! Halkimin öldürülenleri için Aglasam gece gündüz!
\par 2 Keske halkimi birakabilmem, Onlardan uzaklasabilmem için Çölde konaklayacak bir yerim olsa! Hepsi zina ediyor, Hain bir topluluk!
\par 3 "Yalan söylemek için ülkede Dillerini yay gibi geriyor, Güçlerini gerçek yolunda kullanmiyorlar. Kötülük üstüne kötülük yapiyor, Beni tanimiyorlar" diyor RAB.
\par 4 "Herkes dostundan sakinsin, Kardeslerinizin hiçbirine güvenmeyin. Çünkü her kardes Yakup gibi aldatici, Her dost iftiracidir.
\par 5 Dost dostu aldatiyor, Kimse gerçegi söylemiyor. Dillerine yalan söylemeyi ögrettiler, Suç isleye isleye yorgun düstüler.
\par 6 Sen, ey Yeremya, Aldaticiligin ortasinda yasiyorsun. Aldaticiliklari yüzünden Beni tanimak istemiyorlar." Böyle diyor RAB.
\par 7 Bundan ötürü Her Seye Egemen RAB diyor ki, "Iste, onlari aritip sinayacagim, Halkimin günahi yüzünden Baska ne yapabilirim ki?
\par 8 Dilleri öldürücü bir ok, Hep aldatiyor. Komsusuna esenlik diliyor, Ama içinden ona tuzak kuruyor.
\par 9 Bu yüzden onlari cezalandirmayayim mi?" diyor RAB, "Böyle bir ulustan öcümü almayayim mi?"
\par 10 Daglar için aglayip yas tutacagim, Otlaklar için agit yakacagim. Çöle dönüstüler, Kimse geçmiyor oralardan. Sigirlarin bögürmesi duyulmuyor, Kuslar, yabanil hayvanlar kaçip gitti.
\par 11 "Yerusalim'i tas yigini, Çakallarin barinagi haline getirecegim. Yahuda kentlerini Kimsenin yasayamayacagi bir viraneye döndürecegim."
\par 12 Hangi bilge kisi buna akil erdirecek? RAB'bin seslendigi kisi kim ki, sözünü açiklayabilsin? Ülke neden yikildi? Neden kimsenin geçemedigi bir çöle dönüstü?
\par 13 RAB, "Kendilerine verdigim yasayi biraktilar, sözümü dinlemediler, yasami izlemediler" diyor,
\par 14 "Onun yerine yüreklerinin inadini, atalarinin ögrettigi gibi Baallar'i* izlediler."
\par 15 Bunun için Israil'in Tanrisi, Her Seye Egemen RAB diyor ki, "Bu halka pelinotu yedirecek, zehirli su içirecegim.
\par 16 Onlari kendilerinin de atalarinin da tanimadigi uluslarin arasina dagitacak, tümünü yok edene dek peslerine kilici salacagim."
\par 17 Her Seye Egemen RAB diyor ki, "Iyi düsünün! Agit yakan kadinlari çagirin gelsinler. En iyilerini çagirin gelsinler.
\par 18 Hemen gelip bizim için agit yaksinlar; Gözlerimiz gözyasi döksün, Gözkapaklarimizdan sular aksin.
\par 19 Siyon'dan aglama sesi duyuluyor: 'Yikima ugradik! Büyük utanç içindeyiz, Çünkü ülkemizi terk ettik, Evlerimiz yerle bir oldu."
\par 20 Ey kadinlar, RAB'bin sözünü dinleyin! Agzindan çikan her söze kulak verin. Kizlariniza yas tutmayi, Komsunuza agit yakmayi ögretin.
\par 21 Ölüm pencerelerimize tirmandi, Kalelerimize girdi; Sokaklari çocuksuz, Meydanlari gençsiz birakti.
\par 22 Onlara de ki, "RAB söyle diyor: 'Insan cesetleri gübre gibi, Biçicinin ardindaki demetler gibi topraga serilecek. Onlari toplayacak kimse olmayacak."
\par 23 RAB söyle diyor: "Bilge kisi bilgeligiyle, Güçlü kisi gücüyle, Zengin kisi zenginligiyle övünmesin.
\par 24 Dünyada iyilik yapanin, Adaleti, dogrulugu saglayanin Ben RAB oldugumu anlamakla Ve beni tanimakla övünsün övünen. Çünkü ben bunlardan hoslanirim" diyor RAB.
\par 25 "Yalniz bedence sünnetli olanlari cezalandiracagim günler geliyor" diyor RAB.
\par 26 "Misir'i, Yahuda'yi, Edom'u, Ammon'u, Moav'i, çölde yasayan ve zülüflerini kesenlerin hepsini cezalandiracagim. Çünkü bütün bu uluslar gerçekte sünnetsiz*, bütün Israil halki da yürekte sünnetsizdir."

\chapter{10}

\par 1 RAB'bin sana ne söyledigini dinle, ey Israil halki!
\par 2 RAB söyle diyor: "Uluslarin yolunu ögrenmeyin, Gök belirtilerinden yilmayin; Bu belirtilerden uluslar yilsa bile.
\par 3 Uluslarin töreleri yararsizdir. Ormandan agaç keserler, Usta keskisiyle ona biçim verir.
\par 4 Altinla, gümüsle süsler, Çekiçle, çivilerle saglamlastirirlar; Yerinden kimildamasin diye.
\par 5 Salatalik bostanindaki korkuluk gibidir putlari, Konusamazlar; Onlari tasimak gerek, çünkü yürüyemezler. Onlardan korkmayin, zarar veremezler; Iyilik de edemezler."
\par 6 Senin gibisi yok, ya RAB, Sen büyüksün, Adin da büyüktür gücün sayesinde.
\par 7 Senden kim korkmaz, Ey uluslarin krali? Bu sana yakisir. Uluslarin bilgeleri arasinda, Bütün ülkelerinde Senin gibisi yok.
\par 8 Hepsi budala ve akilsiz. Yararsiz putlardan ne ögrenilebilir ki? Agaçtan yapilmis onlar!
\par 9 Tarsis'ten dövme gümüs, Ufaz'dan altin getirilir. Ustayla kuyumcunun yaptigi nesnenin üzerine Lacivert, mor giydirilir, Hepsi usta isidir.
\par 10 Ama gerçek Tanri RAB'dir. O yasayan Tanri'dir, Sonsuza dek kral O'dur. O öfkelenince yeryüzü titrer, Uluslar dayanamaz gazabina.
\par 11 "Onlara sunu diyeceksin, 'Yeri, gögü yaratmayan bu ilahlar, Yerden de gögün altindan da yok olacaklar."
\par 12 Gücüyle yeryüzünü yaratan, Bilgeligiyle dünyayi kuran, Akliyla gökleri yayan RAB'dir.
\par 13 O gürleyince gökteki sular çagildar, Yeryüzünün dört bucagindan bulutlar yükseltir, Yagmur için simsek çaktirir, Ambarlarindan rüzgar estirir.
\par 14 Hepsi budala, bilgisiz, Her kuyumcu yaptigi puttan utanacak. O putlar yapmaciktir, Soluk yoktur onlarda.
\par 15 Yararsiz, alay edilesi nesnelerdir, Cezalandirilinca yok olacaklar.
\par 16 Yakup'un Payi onlara benzemez. Her seye biçim veren O'dur, O'nun mirasidir Israil oymagi, Her Seye Egemen RAB'dir adi.
\par 17 Kusatma altinda olan sizler, Esyalarinizi toplayin yerden.
\par 18 RAB diyor ki, "Iste bu kez bu ülkede yasayanlari Firlatip atacagim; Ele geçirilmeleri için Onlari sikistiracagim."
\par 19 Yaramdan ötürü vay basima gelen! Derdim iyilesmez! Ama, 'Dert benim derdim, Dayanmaliyim dedim.
\par 20 Çadirim yikildi, ipleri koptu. Çocuklarim benden ayrildi, Yok artik onlar. Çadirimi kuracak, Perdelerimi takacak kimse kalmadi.
\par 21 Çobanlar budala, RAB'be danismiyorlar. Bu yüzden isleri yolunda gitmiyor, Bütün sürüleri dagildi.
\par 22 Dinle! Haber geliyor! Kuzey ülkesinden büyük patirti geliyor! Yahuda kentlerini viraneye çevirecek, Çakallara barinak edecek.
\par 23 Insanin yasaminin kendi elinde olmadigini, Adimlarina yön vermenin ona düsmedigini Biliyorum, ya RAB.
\par 24 Beni öfkenle degil, Yalniz adaletinle yola getir, ya RAB, Yoksa beni hiçe indirirsin.
\par 25 Öfkeni seni tanimayan uluslarin, Adini anmayan topluluklarin üzerine dök. Çünkü onlar Yakup soyunu yiyip bitirdiler, Onu tümüyle yok ettiler, Yurdunu viraneye çevirdiler.

\chapter{11}

\par 1 RAB Yeremya'ya söyle seslendi:
\par 2 "Bu antlasmanin kosullarini dinle. Yahuda halkina ve Yerusalim'de yasayanlara açikla.
\par 3 Onlara diyeceksin ki, 'Israil'in Tanrisi RAB söyle diyor: Bu antlasmanin kosullarina uymayan lanet altindadir!
\par 4 Atalarinizi Misir'dan, demir eritme ocagindan çikardigimda bu antlasmaya bagli kalmalarini buyurdum. Onlara dedim ki: Sözümü dinleyin, buyurdugum her seyi yerine getirin. Böylece siz benim halkim olursunuz, ben de sizin Tanriniz olurum.
\par 5 Iste o zaman süt ve bal akan ülkeyi -bugün sizin olan ülkeyi- atalariniza verecegime iliskin içtigim andi yerine getirmis olacagim." "Amin, ya RAB" diye karsilik verdim.
\par 6 RAB söyle dedi: "Söyleyecegim her seyi Yahuda kentlerinde, Yerusalim sokaklarinda duyur: 'Bu antlasmanin kosullarini dinleyin, onlara uyun.
\par 7 Atalarinizi Misir'dan çikardigim günden bu yana sözümü dinlemeleri için onlari defalarca uyardim.
\par 8 Ama dinlemediler, kulak asmadilar. Bunun yerine kötü yüreklerinin inadi uyarinca davrandilar. Ben de uymalarini buyurdugum, ama uymadiklari bu antlasmada açiklanan bütün lanetleri baslarina getirdim."
\par 9 RAB bana dedi ki, "Yahuda halkiyla Yerusalim'de yasayanlar bana düzen kuruyorlar.
\par 10 Sözlerimi dinlemek istemeyen atalarinin suçlarina döndüler. Baska ilahlarin ardinca gidip onlara taptilar. Israil halkiyla Yahuda halki, atalariyla yaptigim antlasmayi bozdu.
\par 11 Bu yüzden RAB, 'Kaçip kurtulamayacaklari bir yikim getirecegim baslarina diyor, 'Bana yakarsalar da onlari dinlemeyecegim.
\par 12 Yahuda kentlerinde oturan halk da Yerusalim'de yasayanlar da gidip buhur yaktiklari ilahlara yalvaracaklar. Ama yikim geldiginde, bu ilahlar onlara yardim edemez.
\par 13 Kentlerinin sayisi kadar ilahin var, ey Yahuda! O utanilasi ilaha, Baal'a* buhur yakmak için Yerusalim sokaklarinin sayisi kadar sunak kurdunuz.
\par 14 "Sana gelince, ey Yeremya, bu halk için yalvarma; ne yakar ne de dilekte bulun. Sikintili zamanlarinda beni çagirdiklarinda onlari dinlemeyecegim.
\par 15 "Sevgilim kötü düzenler kuruyor, Öyleyse tapinagimda isi ne? Adaklar ve kutsanmis et ugrayacagin felaketi önleyebilir mi? Felaket gelince sevinecek misin?"
\par 16 RAB sana meyvesi ve biçimi güzel, Yapragi bol zeytin agaci adini vermisti. Ama güçlü firtina koptugunda Agaci tutusturacak; Dallari kirilacak.
\par 17 Seni dikmis olan Her Seye Egemen RAB, Basina felaket getirmeye karar verdi. Çünkü Israil ve Yahuda halklari Kötülük yapti, Baal'a buhur yakarak beni öfkelendirdiler.
\par 18 Benim için kurduklari düzeni RAB bana açikladi. Haberim vardi, çünkü ne yaptiklarini bana gösterdi.
\par 19 Kesime götürülen uysal bir kuzu gibiydim. Bana düzen kurduklarini anlamamistim. Söyle diyorlardi: "Agaci da meyvesini de yok edelim, Bir daha adi anilmasin diye Onu yasayanlar diyarindan kesip atalim."
\par 20 Adaletle yargilayan, Yüregi ve düsünceyi sinayan, Her Seye Egemen RAB, Davami senin eline birakiyorum. Onlardan alacagin öcü göreyim!
\par 21 "Seni öldürmek isteyen Anatot halki için RAB diyor ki, 'Onlar, RAB'bin adina peygamberlik etme, yoksa seni öldürürüz diyorlardi.
\par 22 Her Seye Egemen RAB, 'Onlari cezalandiracagim diyor, 'Gençleri kiliçtan geçirilecek, ogullariyla kizlari kitliktan ölecek.
\par 23 Sag kalan olmayacak. Cezalandirilacaklari yil Anatot halkinin basina felaket getirecegim."

\chapter{12}

\par 1 Davami önüne getirsem, Hakli çikarsin, ya RAB. Ama adalet konusunda Seninle tartismak istiyorum. Neden kötülerin isi iyi gidiyor? Neden hainler tasasizca yasiyor?
\par 2 Onlari sen diktin, kök saldilar, Büyüyüp ürün verdiler. Adin agizlarindan düsmüyor, Yürekleriyse senden uzak.
\par 3 Beni tanirsin, ya RAB, Beni görür, yüregimin seninle oldugunu bilirsin. Kasaplik koyun gibi ayir onlari, Kesim gününe hazirla!
\par 4 Içinde yasayanlarin kötülügü yüzünden, Ülke ne zamana dek yas tutacak, Otlar ne zamana dek sararip solacak? Hayvanlarla kuslar yok oldu. Çünkü bu halk, "O basimiza neler gelecegini görmüyor" dedi.
\par 5 "Ey Yeremya, Insanlarla yarisa girip yoruldunsa, Atlarla nasil yarisacaksin? Güvenli bir ülkede sendelersen, Seria çaliliklariyla nasil basa çikacaksin?
\par 6 Kardeslerin, öz ailen bile sana ihanet etti, Arkandan seslerini yükselttiler. Yüzüne karsi olumlu konussalar bile onlara güvenme.
\par 7 Evimi terk ettim, Mirasimi reddettim, Sevgilimi düsmanlarinin eline verdim.
\par 8 Mirasim karsimda Ormandaki aslan gibi oldu; Kükreyip üzerime saldirdi. Bu yüzden ondan nefret ediyorum.
\par 9 Mirasim sirtlan ya da yirtici kus mu oldu karsimda? Çevresindeki yirtici kuslar saldiriyor ona. Gidin, bütün yabanil hayvanlari toplayip getirin, Yiyip bitirsinler onu.
\par 10 Pek çok çoban bagimi bozdu, Tarlami çignedi, Güzelim tarlami issiz çöle döndürdü.
\par 11 Onu viraneye çevirdiler, Önümde viran olmus agliyor; Bütün ülke viran olmus, Yine de aldiran yok.
\par 12 Çöldeki çiplak tepelere Yikicilar geldi. RAB'bin kilici ülkeyi Bir uçtan bir uca yiyip bitiriyor. Kimse kavusmayacak esenlige.
\par 13 Halkim bugday ekip diken biçti, Emek verip yarar görmedi. RAB'bin kizgin öfkesi yüzünden Ürününüzden utanacaksiniz."
\par 14 RAB diyor ki, "Halkim Israil'e verdigim mülke el koyan bütün kötü komsularimi ülkelerinden söküp atacak, Yahuda halkini da atacagim.
\par 15 Hepsini söküp attiktan sonra Yahuda'ya yine aciyacak, her birini kendi mülküne, kendi ülkesine geri getirecegim.
\par 16 Halkima Baal'in* adiyla ant içmeyi ögrettiler. Bunun gibi, halkimin yolunda yürümeyi ve 'RAB'bin varligi hakki için diyerek benim adimla ant içmeyi de iyice ögrenirlerse, halkimin arasinda saglam yerleri olacak.
\par 17 Ama kulak asmayan her ulusu kökünden söküp atacak, yok edecegim" diyor RAB.

\chapter{13}

\par 1 RAB bana, "Git, kendine keten bir kusak satin alip beline sar, ama suya sokma" dedi.
\par 2 RAB'bin buyrugu uyarinca bir kusak satin alip belime sardim.
\par 3 RAB bana ikinci kez seslendi:
\par 4 "Satin aldigin belindeki kusagi al, Perat'a git. Kusagi orada bir kaya kovuguna gizle."
\par 5 RAB'bin buyrugu uyarinca gidip kusagi Perat'a yakin bir yere gizledim.
\par 6 Uzun süre sonra RAB bana, "Kalk, Perat'a git, gizlemeni buyurdugum kusagi al" dedi.
\par 7 Bunun üzerine Perat'a gittim, gizledigim yeri kazip kusagi aldim. Ancak kusak çürümüstü, hiçbir ise yaramazdi.
\par 8 RAB bana söyle seslendi:
\par 9 "RAB diyor ki, 'Iste Yahuda'nin gururunu da Yerusalim'in büyük gururunu da böyle çürütecegim.
\par 10 Sözümü dinlemek istemeyen, yüreklerinin inadi uyarinca davranan, baska ilahlari izleyip onlara kulluk eden, tapan bu kötü halk, bu ise yaramaz kusak gibi olacak.
\par 11 Kusak insanin beline nasil yapisirsa, ben de Israil ve Yahuda halklarini kendime öyle yapistirdim diyor RAB, 'Öyle ki, bana ün, övgü, onur getirecek bir halk olsunlar. Ama dinlemediler."
\par 12 "Onlara de ki, 'Israil'in Tanrisi RAB, Her tulum sarapla dolacak, diyor. Eger sana, 'Her tulumun sarapla dolacagini bilmiyor muyuz sanki? derlerse,
\par 13 onlara de ki, 'Bu ülkede yasayan herkesi -Davut'un tahtinda oturan krallari, kâhinleri, peygamberleri, Yerusalim'de yasayanlarin tümünü- sarhos olana dek sarapla dolduracagim diyor RAB.
\par 14 'Onlari -babalarla çocuklari- birbirlerine çarpacagim. Acimadan, esirgemeden, sevecenlik göstermeden hepsini yok edecegim diyor RAB."
\par 15 Dinleyin, kulak verin, Gururlanmayin, Çünkü RAB konustu.
\par 16 Karanlik basmadan, Kararan daglarda Ayaklariniz tökezlemeden Tanriniz RAB'bi onurlandirin. Siz isik beklerken, RAB onu kopkoyu, zifiri karanliga çevirecek.
\par 17 Ama bu uyariyi dinlemezseniz, Gururunuz yüzünden aglayacagim gizlice, Gözlerim aci aci gözyasi dökecek, Gözyaslarim sel gibi akacak. Çünkü RAB'bin sürüsü sürgüne gönderilecek.
\par 18 Krala ve ana kraliçeye söyle: "Tahtlarinizdan inin, Çünkü görkemli taçlariniz basinizdan düstü."
\par 19 Negev'deki kentler kapanacak, Onlari açan olmayacak. Sürgüne gönderilecek Yahuda, Tamami sürgüne gönderilecek.
\par 20 Gözlerinizi kaldirip bakin, Kuzeyden gelenleri görün. Nerede sana emanet edilen sürü? Övündügün kuzular nerede?
\par 21 Sana dost olmasi için yetistirdigin kisileri RAB basina yönetici atayinca ne diyeceksin? Doguran kadinin çektigi sanci gibi Seni de agri tutmayacak mi?
\par 22 "Neden bütün bunlar basima geldi?" dersen, Günahlarinin çoklugu yüzünden eteklerin açildi, Tecavüze ugradin.
\par 23 Kûslu* derisinin rengini, Pars beneklerini degistirebilir mi? Kötülük etmeye alismis olan sizler de iyilik edemezsiniz.
\par 24 "Çöl rüzgarinin savurdugu saman çöpü gibi Dagitacagim sizleri.
\par 25 Payin, sana ayirdigim pay bu olacak" diyor RAB. "Çünkü beni unuttun, Sahte ilahlara güvendin.
\par 26 Ayibin ortaya çiksin diye Eteklerini yüzüne dek kaldiracagim.
\par 27 Kirdaki tepeler üzerinde Yaptigin igrençlikleri -zinalarini, Çapkin çapkin kisneyisini, yüzsüz fahiseliklerini- gördüm. Vay basina geleceklere, ey Yerusalim! Ne zamana dek böyle kirli kalacaksin?"

\chapter{14}

\par 1 RAB kurakliga iliskin Yeremya'ya söyle seslendi:
\par 2 "Yahuda yas tutuyor, Kentleri bitkin; Halki karalar giymis, yerlere oturmus, Yerusalim'in haykirisi yükseliyor.
\par 3 Soylular usaklarini suya gönderiyorlar. Sarniçlara gidiyor, ama su bulamiyor, Kaplari bos dönüyorlar. Asagilanmis, utanç içinde, Baslarini örtüyorlar.
\par 4 Ülke yagmursuz, toprak çatlamis, Irgatlar utanç içinde baslarini örtüyorlar.
\par 5 Kirdaki geyik bile Yeni dogmus yavrusunu birakiyor, Çünkü ot yok.
\par 6 Yaban esekleri çiplak tepelerde durmus, Çakal gibi soluyorlar; Gözlerinin feri sönmüs, Çünkü otlak yok."
\par 7 Suçlarimiz bize karsi taniklik etse de, Adin ugruna bir seyler yap, ya RAB! Pek çok döneklik ettik, Sana karsi günah isledik.
\par 8 Ey Israil'in Umudu, Sikinti anlarindaki Kurtaricisi! Neden ülkede bir yabanci, Ancak bir gece konaklayan yolcu gibisin?
\par 9 Neden sasirmis biri gibi, Kurtarmaya gücü yetmeyen savasçi gibisin? Aramizdasin sen, ya RAB, Seniniz, birakma bizi!
\par 10 Bu halk için RAB diyor ki, "Gezip tozmayi pek sever, Ayaklarini dolasmaktan esirgemezler. Bu yüzden RAB onlardan hosnut degil, Simdi animsayacak suçlarini, Günahlari için onlari cezalandiracak."
\par 11 Sonra RAB bana, "Bu halkin iyiligi için yalvarma" dedi,
\par 12 "Oruç* tutsalar bile feryatlarina kulak vermeyecegim. Yakmalik sunu*, tahil sunusu* sunsalar bile kabul etmeyecegim. Tersine, kiliçla, kitlikla, salgin hastalikla yok edecegim onlari."
\par 13 Bunun üzerine, "Ah, Egemen RAB, peygamberler bu halka, 'Kiliç yüzü görmeyecek, kitlik çekmeyeceksiniz; burada size kalici esenlik saglayacagim diyorlar" dedim.
\par 14 RAB, "Peygamberler benim adimla yalan peygamberlik ediyorlar" dedi, "Onlari ne gönderdim, ne onlara buyruk verdim, ne de seslendim. Size uydurma görümlerden, falciliktan, bos seylerden, akillarindan geçen hayallerden söz ediyorlar.
\par 15 Adimla konusan peygamberler için ben RAB diyorum ki, onlari göndermedigim halde, 'Bu ülkede kiliç da kitlik da olmayacak diyorlar. Ama kendileri de kiliçla, kitlikla yok olacaklar.
\par 16 Peygamberlik ettikleri halk da kitlik ve kiliç yüzünden Yerusalim sokaklarina atilacak. Onlari da karilarini, ogullarini, kizlarini da gömecek kimse olmayacak. Yaptiklari kötülügü kendi baslarina getirecegim.
\par 17 "Onlara de ki, "'Gözlerim gece gündüz Durmadan gözyasi döksün, Çünkü erden kizim, halkim Agir bir yara aldi, Ezici bir darbe yedi.
\par 18 Kira çiksam, kiliçtan geçirilenleri, Kente girsem, kitliktan kirilanlari görüyorum. Olup bitenden habersiz peygamberlerle kâhinlerse Ülkeyi dolasip duruyorlar."
\par 19 Yahuda'yi büsbütün mü reddettin? Siyon'dan tiksiniyor musun? Neden sifa bulmayacak kadar yaraladin bizi? Esenlik bekledik, iyilik gelmedi. Sifa umduk, yilginlik bulduk.
\par 20 Yaptigimiz kötülükleri, Atalarimizin suçlarini biliyoruz, ya RAB; Gerçekten sana karsi günah isledik.
\par 21 Adin ugruna bizi küçümseme, Görkemli tahtinin hor görülmesine izin verme. Bizimle yaptigin antlasmayi animsa, Bozma onu.
\par 22 Uluslarin degersiz putlarindan herhangi biri Yagmur yagdirabilir mi? Gökler kendiliginden Saganak yagdirabilir mi? Bunu yalniz sen yapabilirsin, Ya RAB Tanrimiz. Umudumuz sende, Çünkü bütün bunlari yapan sensin.

\chapter{15}

\par 1 RAB bana dedi ki, "Musa'yla Samuel önümde durup yalvarsalar bile, bu halka acimayacagim; kov onlari önümden, gitsinler!
\par 2 Sana, 'Nereye gidelim? diye sorarlarsa de ki, 'RAB söyle diyor: "'Ölüm için ayrilanlar ölüme, Kiliç için ayrilanlar kilica, Kitlik için ayrilanlar kitliga, Sürgün için ayrilanlar sürgüne.
\par 3 "Onlarin basina dört tür yikim getirmeye karar verdim" diyor RAB, "Öldürmek için kilici, paralamak için köpekleri, yiyip bitirmek, yok etmek için yirtici kuslarla yabanil hayvanlari salacagim üzerlerine.
\par 4 Yahuda Krali Hizkiya oglu Manasse'nin Yerusalim'de yaptiklari yüzünden bütün yeryüzü kralliklarini dehsete düsürecegim.
\par 5 "Kim aciyacak sana, ey Yerusalim? Kim yas tutacak senin için? Hal hatir sormak için Kim yolundan dönüp sana gelecek?
\par 6 Sen beni reddettin" diyor RAB, "Gerisingeri gidiyorsun. Ben de elimi sana karsi kaldiracak, Seni yok edecegim; Merhamet ede ede yoruldum.
\par 7 Ülkenin kapilarinda, Halkimi yabayla savuracak, Çocuksuz birakacak, yok edecegim; Çünkü yollarindan dönmediler.
\par 8 Dul kadinlarinin sayisi denizin kumundan çok olacak. Gençlerinin annelerine Ögle vakti yok ediciyi gönderecegim; Üzerlerine ansizin aci, dehset salacagim.
\par 9 Yedi çocuklu kadin Bayilip son solugunu verecek; Daha gündüzken günesi batacak, Utandirilip alçaltilacak. Sag kalanlari düsmanlarinin önünde Kilica teslim edecegim." Böyle diyor RAB.
\par 10 Vay basima! Herkesle çekisip davaci olayim diye Dogurmussun beni, ey annem! Ne ödünç aldim, ne de verdim, Yine de herkes lanet okuyor bana.
\par 11 RAB söyle dedi: "Kuskun olmasin, iyilik için seni özgür kilacagim, Yikim ve sikinti zamaninda Düsmanlarini sana yalvartacagim.
\par 12 "Demiri, kuzeyden gelen demiri Ya da tuncu* kimse kirabilir mi?
\par 13 Ülkende islenen günahlar yüzünden Servetini de hazinelerini de karsiliksiz, Çapul mali olarak verecegim.
\par 14 Bilmedigin bir ülkede Düsmanlarina köle edecegim seni. Çünkü size karsi öfkem Ates gibi tutusup yanacak."
\par 15 Sen bilirsin, ya RAB, Beni animsa, beni kolla. Bana eziyet edenlerden öcümü al. Sabrinla beni canimdan etme, Senin ugruna asagilandigimi unutma.
\par 16 Sözlerini bulur bulmaz yuttum, Bana nese, yüregime sevinç oldu. Çünkü seninim ben, Ya RAB, Her Seye Egemen Tanri!
\par 17 Eglenenlerin arasinda oturmadim, Onlarla sevinip cosmadim. Elin üzerimde oldugu için Tek basima oturdum, Çünkü beni öfkeyle doldurmustun.
\par 18 Neden sürekli aci çekiyorum? Neden yaram agir ve umarsiz? Benim için aldatici bir dere, Güvenilmez bir pinar mi olacaksin?
\par 19 Bu yüzden RAB diyor ki, "Eger dönersen seni yine hizmetime alirim; Ise yaramaz sözler degil, Degerli sözler söylersen, Benim sözcüm olursun. Bu halk sana dönecek, Ama sen onlara dönmemelisin.
\par 20 Bu halkin karsisinda Saglamlastirilmis tunç bir duvar kilacagim seni; Seninle savasacak ama yenemeyecekler, Çünkü yardim etmek, kurtarmak için Ben seninleyim" diyor RAB.
\par 21 "Seni kötünün elinden kurtaracak, Acimasizin avucundan kurtaracagim."

\chapter{16}

\par 1 RAB bana söyle seslendi:
\par 2 "Kendine kari alma, burada ogullarin, kizlarin olmasin."
\par 3 Bu ülkede dogan ogullarla kizlar ve anne babalari için RAB diyor ki,
\par 4 "Ölümcül hastaliklardan ölecekler. Onlar için yas tutulmayacak, gömülmeyecekler. Cesetleri topragin üzerinde gübre gibi kalacak. Kiliçla, kitlikla yok olacaklar; cesetleri yirtici kuslara, yabanil hayvanlara yem olacak."
\par 5 Çünkü RAB diyor ki, "Cenaze yemeginin verildigi eve gitme, dövünmek için gitme, bassagligi dileme. Çünkü ben bu halktan esenligimi, sevgimi, sevecenligimi geri çektim" diyor RAB.
\par 6 "Bu ülkede büyükler de küçükler de ölecek, gömülmeyecekler. Onlar için yas tutan, dövünüp bedenini yaralayan, basini tiras eden olmayacak.
\par 7 Ölene yas tutani avutmak için kimse onunla yemek yemeyecek. Anne babasini yitirene kimse avunç kâsesini sunmayacak.
\par 8 "Sölen evine de gitme, onlarla oturma, yiyip içme.
\par 9 Çünkü Israil'in Tanrisi, Her Seye Egemen RAB diyor ki, 'Burada sevinç ve nese sesine, gelin güvey sesine senin günlerinde gözünün önünde son verecegim.
\par 10 "Bütün bunlari bu halka bildirdiginde, 'RAB neden basimiza bu büyük felaketi getirecegini bildirdi? Suçumuz ne? Tanrimiz RAB'be karsi isledigimiz günah ne? diye sorarlarsa,
\par 11 de ki, 'Atalariniz beni terk etti diyor RAB, 'Baska ilahlarin ardinca gittiler, onlara kulluk edip taptilar. Beni terk ettiler, Kutsal Yasam'a uymadilar.
\par 12 Sizse atalarinizdan daha çok kötülük yaptiniz. Beni dinleyeceginize, kötü yüreginizin inadi uyarinca davrandiniz.
\par 13 Bu yüzden sizi bu ülkeden sizin de atalarinizin da bilmedigi bir ülkeye atacagim. Orada gece gündüz baska ilahlara kulluk edeceksiniz, çünkü size lütfetmeyecegim.
\par 14 "Artik insanlarin, 'Israil halkini Misir'dan çikaran RAB'bin varligi hakki için demeyecegi günler geliyor" diyor RAB.
\par 15 "Bunun yerine, 'Israil halkini kuzey ülkesinden ve sürdügü bütün öbür ülkelerden geri getiren RAB'bin varligi hakki için diyecekler. Çünkü atalarina vermis oldugum topraklara onlari geri getirecegim.
\par 16 "Birçok balikçi çagirmak üzereyim. Onlari yakalayacaklar" diyor RAB, "Ardindan birçok avci çagiracagim. Her dagin, her tepenin üzerinden, kaya kovuklarindan avlayacaklar onlari.
\par 17 Bütün yaptiklarini görüyorum; hiçbiri benden gizli degil. Günahlari da gözümden kaçmiyor.
\par 18 Ilkin suçlarini, günahlarini iki katiyla onlara ödetecegim. Çünkü tiksindirici cansiz ilahlariyla ülkemi kirlettiler, mülkümü igrenç putlariyla doldurdular.
\par 19 "Ya RAB, sen benim gücüm, Kalem, sikinti gününde siginagimsin. Dünyanin dört bucagindan Uluslar sana gelip, 'Atalarimiz yalniz yalanlari, Kendilerine hiçbir yarari olmayan Degersiz putlari miras aldilar diyecekler,
\par 20 'Insan kendine ilah yapar mi? Onlar ilah degil ki!
\par 21 "Onun için bu kez onlara Gücümü, kudretimi tanitacagim. O zaman adimin RAB oldugunu anlayacaklar."

\chapter{17}

\par 1 "Yahuda'nin günahi demir kalemle yazildi; Yüreklerinin levhalari, Sunaklarinin boynuzlari üzerine Elmas uçlu aletle oyuldu.
\par 2 Bol yaprakli her agacin yaninda, Her yüksek tepedeki sunaklarla, Asera* putlariyla Çocuklariymis gibi ilgileniyorlar.
\par 3 Ey kirdaki dagim, ülkende islenen günahlar yüzünden Servetini, bütün hazinelerini Ve puta tapilan yerlerini birakacagim, yagmalansin.
\par 4 Sana verdigim mülkü kendi suçunla yitireceksin. Bilmedigin bir ülkede Düsmanlarina köle edecegim seni. Çünkü öfkemi alevlendirdiniz, Tutusup sonsuza dek yanacak."
\par 5 RAB diyor ki, "Insana güvenen, Insanin gücüne dayanan, Yüregi RAB'den uzaklasan kisi lanetlidir.
\par 6 Böylesi bozkirdaki çali gibidir, Iyilik geldigi zaman görmeyecek; Kurak çöle, Kimsenin yasamadigi tuzlaya yerlesecek.
\par 7 "Ne mutlu RAB'be güvenen insana, Güveni yalniz RAB olana!
\par 8 Böylesi su kiyilarina dikilmis agaca benzer, Köklerini akarsulara salar. Sicak gelince korkmaz, Yapraklari hep yesildir. Kuraklik yilinda kaygilanmaz, Meyve vermekten geri durmaz."
\par 9 Yürek her seyden daha aldaticidir, iyilesmez, Onu kim anlayabilir?
\par 10 "Ben RAB, herkesi davranislarina, Yaptiklarinin sonucuna göre ödüllendirmek için Yüregi yoklar, düsünceyi denerim."
\par 11 Yumurtlamadigi yumurtalarin üzerinde oturan keklik nasilsa, Haksiz servet edinen kisi de öyledir. Yasaminin ortasinda serveti onu birakir, Yasaminin sonunda kendisi aptal çikar.
\par 12 Tapinagimizin yeri Baslangiçtan yüceltilmis görkemli bir tahttir.
\par 13 Ey Israil'in umudu RAB, Seni birakanlarin hepsi Utanilacak duruma düsecek. Sana sirtini dönenler topraga yazilacak, Çünkü RAB'bi, diri su pinarini biraktilar.
\par 14 Sifa ver bana, ya RAB, O zaman iyi olurum; Kurtar beni, kurtulus bulurum, Çünkü övgüm sensin.
\par 15 Bana, "Hani, RAB'bin sözü nerede? Haydi, gelsin yerine bakalim" deyip duruyorlar.
\par 16 Senin hizmetinde çoban olmaktan kaçinmadim, Felaket gününü de ben istemedim. Dudaklarimdan çikan her sözü bilirsin, ya RAB. O söz zaten senin agzindaydi.
\par 17 Dehset verme bana, Felaket gününde siginagim sensin.
\par 18 Bana eziyet edenler utandirilsin, Ama beni utandirma; Onlari yilginliga düsür, Ama beni düsürme. Felaket gününü getir üzerlerine, Onlari iki kat yikimla ez.
\par 19 RAB bana söyle dedi: "Yahuda krallarinin girip çiktigi Halk Kapisi'na ve Yerusalim'in öbür kapilarina git, orada dur.
\par 20 Halka de ki, 'Ey Yahuda krallari, Yahuda halki, Yerusalim'de oturup bu kapilardan girenler, RAB'bin sözünü dinleyin!
\par 21 RAB diyor ki, Sabat Günü* yük tasimamaya, Yerusalim kapilarindan içeri bir sey sokmamaya dikkat edin.
\par 22 Sabat Günü evinizden yük çikarmayin, hiç is yapmayin. Atalariniza buyurdugum gibi Sabat Günü'nü kutsal sayacaksiniz.
\par 23 Ne var ki, onlar sözümü dinlemediler, kulak asmadilar. Dikbaslilik ederek beni dinlemediler, yola gelmek istemediler.
\par 24 Beni iyi dinlerseniz, diyor RAB, Sabat Günü bu kentin kapilarindan yük tasimayip hiç is yapmayarak Sabat Günü'nü kutsal sayarsaniz,
\par 25 Davut'un tahtinda oturan krallarla önderler savas arabalarina, atlara binip Yahuda halki ve Yerusalim'de yasayanlarla birlikte bu kentin kapilarindan girecekler. Bu kentte sonsuza dek insanlar yasayacak.
\par 26 Yahuda kentlerinden, Yerusalim çevresinden, Benyamin topraklarindan, Sefela'dan, daglik bölgeden, Negev'den gelip RAB'bin Tapinagi'na yakmalik sunular*, kurbanlar, tahil sunulari*, günnük ve sükran sunulari getirecekler.
\par 27 Ancak beni dinlemez, Sabat Günü Yerusalim kapilarindan yük tasiyarak girer, o günü kutsal saymazsaniz, kentin kapilarini atese verecegim. Yerusalim saraylarini yakip yok edecek, hiç sönmeyecek ates."

\chapter{18}

\par 1 RAB Yeremya'ya söyle seslendi:
\par 2 "Kalk, çömlekçinin isligine git; orada sana seslenecegim."
\par 3 Bunun üzerine çömlekçinin isligine gittim. Çark üzerinde çalisiyordu.
\par 4 Yaptigi balçiktan kap elinde bozulunca çömlekçi balçiga istedigi biçimi vererek baska bir kap yapti.
\par 5 RAB bana yine seslendi:
\par 6 "Bu çömlekçinin yaptigini ben de size yapamaz miyim, ey Israil halki? diyor RAB. Çömlekçinin elinde balçik neyse, siz de benim elimde öylesiniz, ey Israil halki!
\par 7 Bir ulusun ya da kralligin kökünden sökülecegini, yikilip yok edilecegini duyururum da,
\par 8 uyardigim ulus kötülügünden dönerse, basina felaket getirme kararimdan vazgeçerim.
\par 9 Öte yandan, bir ulusun ya da kralligin kurulup dikilecegini duyururum da,
\par 10 o ulus sözümü dinlemeyip gözümde kötü olani yaparsa, ona söz verdigim iyiligi yapmaktan vazgeçerim.
\par 11 "Bu nedenle Yahuda halkiyla Yerusalim'de yasayanlara de ki, 'RAB söyle diyor: Iste size bir felaket tasarliyor, size karsi bir düzen kuruyorum. Onun için her biriniz kötü yolundan dönsün, yasantinizi da davranislarinizi da düzeltin.
\par 12 Ama onlar, 'Bos ver! Biz kendi tasarlarimizi sürdürecegiz; her birimiz kötü yüreginin inadi uyarinca davranacak diyecekler."
\par 13 Bu yüzden RAB diyor ki, "Uluslar arasinda sorusturun: Böylesini kim duydu? Erden kiz Israil Çok korkunç bir sey yapti.
\par 14 Kayalik bayirlardan Lübnan'in kari hiç eksik olur mu? Uzaktan akan soguk sular hiç kesilir mi?
\par 15 Oysa halkim beni unuttu, Degersiz ilahlara buhur yakti. Bu ilahlar gidecekleri yollarda, Eski yollarda sendelemelerine neden oldu; Onlari sapa, bitmemis yollarda yürüttü.
\par 16 Ülkeleri viran edilecek, Sürekli alay konusu olacak; Oradan her geçen saskin saskin Basini sallayacak.
\par 17 Onlari düsmanlarinin önünde Dogu rüzgari gibi dagitacagim; Yikim günü yüzümü degil, Sirtimi çevirecegim onlara."
\par 18 Bunun üzerine, "Haydi, Yeremya'ya karsi bir düzen kuralim!" dediler, "Çünkü yasayi ögretecek kâhin, ögüt verecek bilge, Tanri sözünü bildirecek peygamber hiç eksik olmayacak. Gelin, ona sözle saldiralim, söylediklerini de dinlemeyelim."
\par 19 Dinle beni, ya RAB, Beni suçlayanlarin dediklerini isit!
\par 20 Iyilige karsi kötülük mü yapmali? Ama onlar bana çukur kazdilar. Onlara duydugun öfkeyi yatistirmak, Onlarin iyiligini dilemek için Senin önünde nasil durdugumu animsa.
\par 21 Bu yüzden çocuklarini kitliga ver, Kilicin agzina at. Karilari çocuksuz, dul kalsin, Erkeklerini ölüm alip götürsün, Gençleri savasta kiliçtan geçirilsin.
\par 22 Sen üzerlerine ansizin akincilar gönderdiginde, Evlerinden çigliklar duyulsun. Çünkü beni yakalamak için çukur kazdilar, Ayaklarima gizli tuzak kurdular.
\par 23 Beni öldürmek için kurduklari düzenlerin hepsini Biliyorsun, ya RAB. Bagislama suçlarini, Günahlarini önünden silme. Yigilip kalsinlar senin önünde. Öfkeliyken ugras onlarla.

\chapter{19}

\par 1 RAB bana söyle dedi: "Git, çömlekçiden bir çömlek satin al. Halkin ve kâhinlerin ileri gelenlerinden birkaçini yanina alip
\par 2 Harsit Kapisi'na yakin Ben-Hinnom Vadisi'ne git. Sana söyleyeceklerimi orada duyur.
\par 3 De ki, 'RAB'bin sözünü dinleyin, ey Yahuda krallari ve Yerusalim'de yasayanlar! Israil'in Tanrisi, Her Seye Egemen RAB söyle diyor: Dinleyin! Buraya, her duyani saskina çevirecek bir felaket göndermek üzereyim.
\par 4 Çünkü beni terk ettiler, burayi yabanci bir ülke haline getirdiler. Kendilerinin de atalariyla Yahuda krallarinin da tanimadigi baska ilahlara burada buhur yaktilar, burayi döktükleri suçsuz kaniyla doldurdular.
\par 5 Çocuklarini ateste Baal'a* kurban etmek için tapinma yerleri kurdular. Böyle bir sey ne buyurdum ne sözünü ettim ne de aklimdan geçirdim.
\par 6 Bundan ötürü buranin artik Tofet* ya da Ben-Hinnom Vadisi degil, Kiyim Vadisi diye anilacagi günler geliyor, diyor RAB.
\par 7 Yahuda ve Yerusalim'in tasarilarini burada bosa çikaracagim. Onlari canlarina susayanlarin eline verecek, düsmanlarinin önünde kiliçla düsürecegim. Cesetlerini yem olarak yirtici kuslara, yabanil hayvanlara verecegim.
\par 8 Bu kenti viraneye çevirecek, alay konusu edecegim; oradan her geçen saskin saskin bakip basina gelen belalardan ötürü onunla alay edecek.
\par 9 Onlara ogullarinin, kizlarinin etini yedirecegim. Canlarina susamis düsmanlari onlari kusattiginda sikintidan birbirlerini yiyecekler.
\par 10 "O zaman seninle gidenlerin önünde çömlegi kir
\par 11 ve onlara de ki, 'Her Seye Egemen RAB söyle diyor: Çömlekçinin çömlegi nasil kirilip bir daha onarilamazsa, ben de bu halki ve bu kenti öyle kiracagim. Ölüleri yer kalmayana dek Tofet'te gömecekler.
\par 12 Bu kente de içinde yasayanlara da böyle davranacagim, diyor RAB. Bu kenti Tofet gibi yapacagim.
\par 13 Yerusalim'in evleri de Yahuda krallarinin saraylari da Tofet gibi kirli sayilacak. Çünkü bu evlerin damlarinda gök cisimlerine buhur yaktilar, baska ilahlara dökmelik sunular sundular."
\par 14 Peygamberlik etmesi için RAB'bin Tofet'e gönderdigi Yeremya oradan döndü. RAB'bin Tapinagi'nin avlusunda durup halka söyle dedi:
\par 15 "Israil'in Tanrisi, Her Seye Egemen RAB diyor ki, 'Iste bu kente ve çevresindeki köylere sözünü ettigim bütün felaketleri getirecegim. Çünkü dikbaslilik edip sözümü dinlemediler."

\chapter{20}

\par 1 RAB'bin Tapinagi'nin bas görevlisi Immer oglu Kâhin Pashur, Yeremya'nin böyle peygamberlik ettigini duyunca,
\par 2 onun dövülüp RAB'bin Tapinagi'nin Yukari Benyamin Kapisi'ndaki tomruga vurulmasini buyurdu.
\par 3 Ertesi gün Pashur kendisini tomruktan saliverince, Yeremya ona, "RAB sana Pashur degil, Magor-Missaviv*fo* adini verdi" dedi,
\par 4 "RAB diyor ki, 'Seni de dostlarini da yildiracagim. Dostlarinin düsman kiliciyla düstügünü gözlerinle göreceksin. Bütün Yahuda'yi Babil Krali'nin eline teslim edecegim; onlari Babil'e sürecek ya da kiliçtan geçirecek.
\par 5 Bu kentin bütün zenginligini -ürününü, degerli esyalarini, Yahuda krallarinin hazinelerini- düsmanlarinin eline verecegim. Hepsini yagmalayip Babil'e götürecekler.
\par 6 Sana gelince, ey Pashur, sen de evinde yasayanlarin hepsi de Babil'e sürüleceksiniz. Sen de kendilerine yalan peygamberlik ettigin bütün dostlarin da orada ölüp gömüleceksiniz."
\par 7 Beni kandirdin, ya RAB, Ben de kandim. Bana üstün geldin, beni yendin. Bütün gün alay konusu oluyorum, Herkes benimle egleniyor.
\par 8 Çünkü konustukça feryat ediyor, Siddet diye, yikim diye haykiriyorum. RAB'bin sözü yüzünden bütün gün yeriliyor, Gülünç duruma düsüyorum.
\par 9 "Bir daha onu anmayacak, O'nun adina konusmayacagim" desem, Sözü kemiklerimin içine hapsedilmis, Yüregimde yanan bir ates sanki. Onu içimde tutmaktan yoruldum, Yapamiyorum artik.
\par 10 Birçogunun, "Her yer dehset içinde! Suçlayin! Suçlayalim onu!" diye fisildastigini duydum. Bütün güvendigim insanlar düsmemi gözlüyor, "Belki kanar, onu yeneriz, Sonra da öcümüzü aliriz" diyorlar.
\par 11 Ama RAB güçlü bir savasçi gibi benimledir. Bu yüzden bana eziyet edenler tökezleyecek, Üstün gelemeyecek, Basarisizliga ugrayip büyük utanca düsecekler; Onursuzluklari sonsuza dek unutulmayacak.
\par 12 Ey dogru kisiyi sinayan, Yüregi ve düsünceyi gören Her Seye Egemen RAB! Davami senin eline birakiyorum. Onlardan alacagin öcü göreyim!
\par 13 Ezgiler okuyun RAB'be! Övün RAB'bi! Çünkü yoksulun canini kötülerin elinden O kurtardi.
\par 14 Lanet olsun dogdugum güne! Kutlu olmasin annemin beni dogurdugu gün!
\par 15 "Bir oglun oldu!" diyerek babama haber getiren, Onu sevince bogan adama lanet olsun!
\par 16 RAB'bin acimadan yerle bir ettigi Kentler gibi olsun o adam! Sabah feryatlar, Öglen savas naralari duysun!
\par 17 Çünkü beni annemin rahminde öldürmedi; Annem mezarim olur, Rahmi hep gebe kalirdi.
\par 18 Neden ana rahminden çiktim? Dert, üzüntü görmek, Ömrümü utanç içinde geçirmek için mi?

\chapter{21}

\par 1 Kral Sidkiya Malkiya oglu Pashur'la Maaseya oglu Kâhin Sefanya'yi Yeremya'ya gönderince, RAB Yeremya'ya seslendi. Pashur'la Sefanya ona söyle demisti:
\par 2 "Lütfen bizim için RAB'be danis. Çünkü Babil Krali Nebukadnessar bize saldiriyor. Belki RAB bizim için sasilacak islerinden birini yapar da Nebukadnessar ülkemizden çekilir."
\par 3 Yeremya su karsiligi verdi: "Sidkiya'ya deyin ki,
\par 4 'Israil'in Tanrisi RAB söyle diyor: Surlarin disinda sizi kusatan Babil Krali ve Kildaniler'le* savasmakta kullandiginiz silahlari size karsi çevirecegim; hepsini bu kentin ortasina toplayacagim.
\par 5 Ben de elimi size karsi kaldiracagim; kudretle, kizginlikla, gazapla, büyük öfkeyle sizinle savasacagim.
\par 6 Bu kentte yasayanlari yok edecegim; insan da, hayvan da korkunç bir salgin hastaliktan ölecek.
\par 7 Ondan sonra da, diyor RAB, Yahuda Krali Sidkiya'yla görevlilerini, bu kentte salgindan, kiliçtan, kitliktan sag çikan halki Babil Krali Nebukadnessar'in ve canlarina susamis düsmanlarinin eline teslim edecegim. Hepsini kiliçtan geçirecek, canlarini bagislamayacak, merhamet etmeyecek, acimayacak.
\par 8 "Bunun yanisira halka sunlari da söyle: 'RAB diyor ki: Iste yasama giden yolu da ölüme giden yolu da önünüze koyuyorum.
\par 9 Bu kentte kalan kiliçtan, kitliktan, salgindan ölecek; disari çikip kenti kusatan Kildaniler'e teslim olansa yasayacak, hiç degilse canini kurtarmis olacak.
\par 10 Bu kente iyilik degil, kötülük etmeye karar verdim, diyor RAB. Bu kenti Babil Krali ele geçirip atese verecek."
\par 11 "Yahuda Krali'nin ailesine de ki, 'RAB'bin sözünü dinleyin:
\par 12 RAB söyle diyor, ey Davut soyu: "'Her sabah adaleti uygulayin, Soyguna ugramis kisiyi zorbanin elinden kurtarin. Yoksa yaptiginiz kötülük yüzünden Öfkem ates gibi tutusup yanacak, Söndüren olmayacak.
\par 13 Ey vadinin üstünde, Kayalik ovada oturan Yerusalim, Sana karsiyim diyor RAB, Siz ki, kim bize saldirabilir? Siginagimiza kim girebilir, diyorsunuz.
\par 14 Sizi yaptiklariniza göre cezalandiracagim, diyor RAB. Bütün çevresini yakip yok edecek Bir ates tutusturacagim kentin ormaninda."

\chapter{22}

\par 1 RAB bana dedi ki, "Yahuda Krali'nin sarayina gidip su haberi bildir:
\par 2 'RAB'bin sözünü dinleyin, ey Davut'un tahtinda oturan Yahuda Krali'yla görevlileri ve bu kapilardan giren halk!
\par 3 RAB diyor ki: Adil ve dogru olani yapin. Soyguna ugrayani zorbanin elinden kurtarin. Yabanciya, öksüze, dula haksizlik etmeyin, siddete basvurmayin. Burada suçsuz kani dökmeyin.
\par 4 Bu buyruklari özenle yerine getirirseniz, Davut'un tahtinda oturan krallar savas arabalariyla, atlariyla bu sarayin kapilarindan girecekler; görevlileriyle halklari da onlari izleyecek.
\par 5 Ancak bu buyruklara uymazsaniz, diyor RAB, adim üzerine ant içerim ki, bu saray viraneye dönecek."
\par 6 Çünkü Yahuda Krali'nin sarayi için RAB diyor ki, "Sen benim için Gilat gibisin, Lübnan'in dorugu gibi. Ama hiç kuskun olmasin, seni çöle döndürecek, Kimsenin yasamadigi kentlere çevirecegim.
\par 7 Eli silahli yok ediciler görevlendirecegim sana karsi. En iyi sedir agaçlarini kesecek, Atese atacaklar.
\par 8 "Bu kentten geçen birçok ulus birbirlerine, 'RAB bu büyük kente neden bunu yapti? diye soracaklar.
\par 9 "Yanit söyle olacak: 'Çünkü Tanrilari RAB'bin antlasmasini biraktilar, baska ilahlara tapip kulluk ettiler."
\par 10 Ölen için aglamayin, yasa bürünmeyin; Ancak sürgüne giden için aglayin aci aci. Çünkü bir daha dönmeyecek, Anayurdunu görmeyecek.
\par 11 Babasi Yosiya'nin yerine Yahuda Krali olan ve buradan çikip giden Yosiya oglu Sallum için RAB diyor ki, "Bir daha dönmeyecek buraya.
\par 12 Sürgüne gönderildigi yerde ölecek, bir daha bu ülkeyi görmeyecek."
\par 13 "Sarayini haksizlikla, Yukari odalarini adaletsizlikle yapan, Komsusunu parasiz çalistiran, Ücretini ödemeyen adamin vay basina!
\par 14 'Kendim için yukari odalari havadar, Genis bir saray yapacagim diyenin vay basina! Sarayina büyük pencereler açar, Sedir agaciyla kaplar, Kirmiziya boyar.
\par 15 "Bol bol sedir agaci kullandin diye Kral mi oldun sanirsin? Baban doyasiya yiyip içti, Ama iyi ve dogru olani yapti; Onun için de isleri iyi gitti.
\par 16 Ezilenin, yoksulun davasini savundu, Onun için de isleri iyi gitti. Beni tanimak bu degil midir?" diyor RAB.
\par 17 "Seninse gözlerin de yüregin de yalniz kazanca, Suçsuz kani dökmeye, Baski, zorbalik yapmaya yönelik."
\par 18 Bu yüzden RAB Yahuda Krali Yosiya oglu Yehoyakim için diyor ki, "Onun için kimse, 'Ah kardesim! Vah kizkardesim! diye dövünmeyecek. Onun için kimse, 'Ah efendim! Vah onun görkemi! diye dövünmeyecek.
\par 19 Sürüklenip Yerusalim kapilarindan disari atilacak, Esek gömülür gibi gömülecek o."
\par 20 "Lübnan'a git, feryat et, Sesin Basan'dan duyulsun, Haykir Avarim'den, Çünkü bütün oynaslarin ezildi.
\par 21 Kendini güvenlikte sandiginda seni uyardim. Ama, 'Dinlemem dedin. Gençliginden bu yana böyleydi tutumun, Sözümü hiç dinlemedin.
\par 22 Rüzgar bütün çobanlarini alip götürecek, Oynaslarin sürgüne gidecek. Iste o zaman yaptigin kötülükler yüzünden Utanacak, asagilanacaksin.
\par 23 Ey sen, Lübnan'da yasayan, Yuvasini sedir agacindan kuran adam! Sana doguran kadin gibi acilar, sancilar geldiginde, Nasil da inleyeceksin!"
\par 24 "Varligim hakki için derim ki" diyor RAB, "Ey Yahuda Krali Yehoyakim oglu Yehoyakin, sag elimdeki mühür yüzügü olsan bile, çikarip atardim seni.
\par 25 Seni can düsmanlarinin, korktugun kisilerin, Babil Krali Nebukadnessar'la Kildaniler'in* eline teslim edecegim.
\par 26 Seni de seni doguran anneni de dogmadiginiz bir ülkeye atacagim; orada öleceksiniz.
\par 27 Dönmeye can attiginiz ülkeye bir daha dönemeyeceksiniz."
\par 28 Bu mu Yehoyakin? Bu hor görülmüs kirik çömlek, Kimsenin istemedigi kap? Neden kendisi de çocuklari da Bilmedikleri bir ülkeye atildilar?
\par 29 Ülke, ey ülke, RAB'bin sözünü dinle, ey ülke!
\par 30 RAB diyor ki, "Bu adami çocuksuz, Ömrünce basarisiz biri olarak yazin. Çünkü soyundan gelen hiç kimse basarili olmayacak, Soyundan gelen hiç kimse Davut'un tahtinda oturamayacak, Yahuda'da bir daha krallik etmeyecek."

\chapter{23}

\par 1 "Otlagimin koyunlarini yok edip dagitan çobanlarin vay basina!" diyor RAB.
\par 2 Halkimi güden çobanlar için Israil'in Tanrisi RAB söyle diyor: "Sürümü dagitip sürdünüz, onlarla ilgilenmediniz. Simdi ben sizinle ilgilenecegim, yaptiginiz kötülük yüzünden sizi cezalandiracagim." RAB böyle diyor.
\par 3 "Sürmüs oldugum bütün ülkelerden sürümün sag kalanlarini toplayip otlaklarina geri getirecegim; orada verimli olup çogalacaklar.
\par 4 Onlari güdecek çobanlar koyacagim baslarina. Bundan böyle korkmayacak, yilginliga düsmeyecekler. Bir tanesi bile eksilmeyecek" diyor RAB.
\par 5 "Iste Davut için dogru bir dal Çikaracagim günler geliyor" diyor RAB. "Bu kral bilgece egemenlik sürecek, Ülkede adil ve dogru olani yapacak.
\par 6 Onun döneminde Yahuda kurtulacak, Israil güvenlik içinde yasayacak. O, 'RAB dogrulugumuzdur adiyla anilacak.
\par 7 "Artik insanlarin, 'Israil halkini Misir'dan çikaran RAB'bin varligi hakki için demeyecekleri günler geliyor" diyor RAB.
\par 8 "Bunun yerine, 'Israil soyunu kuzey ülkesinden ve sürdügü bütün öbür ülkelerden geri getiren RAB'bin varligi hakki için diyecekler. Böylece kendi topraklarinda yasayacaklar."
\par 9 Peygamberlere gelince, Yüregim paramparça, Bütün kemiklerim titriyor. RAB'bin yüzünden, O'nun kutsal sözleri yüzünden Sarhos gibi, Saraba yenik düsen bir adam gibiyim.
\par 10 Çünkü ülke zina edenlerle dolu, Lanet yüzünden yas tutuyor. Otlaklar kurumus. Izledikleri yol kötü, Güçlerini haksizca kullaniyorlar.
\par 11 "Peygamber de kâhin de tanrisiz; Tapinagimda bile kötülüklerini gördüm" diyor RAB.
\par 12 "Bu yüzden izledikleri yol Onlar için kaygan olacak; Karanliga sürülecek, Orada tökezleyip düsecekler. Çünkü cezalandirilacaklari yil Baslarina felaket getirecegim" diyor RAB.
\par 13 "Samiriye peygamberleri arasinda Su igrençligi gördüm: Baal* adina peygamberlik ederek Halkim Israil'i bastan çikariyorlar.
\par 14 Yerusalim peygamberleri arasinda Su korkunç seyi gördüm: Zina ediyorlar, yalan pesindeler. Kötülük edenleri güçlendirdiklerinden, Kimse kötülügünden dönmüyor. Benim için hepsi Sodom gibi, Yerusalim halki Gomora gibi oldu."
\par 15 Bu nedenle Her Seye Egemen RAB peygamberler için söyle diyor: "Onlara pelinotu yedirecek, Zehirli su içirecegim. Çünkü Yerusalim peygamberleri Tanrisizligin bütün ülkeye yayilmasina neden oldular."
\par 16 Her Seye Egemen RAB diyor ki, "Size peygamberlik eden peygamberlerin Dediklerine kulak asmayin, Onlar sizi aldatiyor. RAB'bin agzindan çikanlari degil, Kendi hayal ettikleri görümleri anlatiyorlar.
\par 17 Beni küçümseyenlere sürekli, 'RAB diyor ki: Size esenlik olacak! diyorlar. Yüreklerinin inatçiligi dogrultusunda davrananlara, 'Basiniza felaket gelmeyecek diyorlar.
\par 18 RAB'bin sözünü duyup anlamak için RAB'bin meclisinde kim bulundu ki? O'nun sözüne kulak verip duyan kim?
\par 19 Iste, RAB'bin firtinasi öfkeyle kopacak, Kasirgasi döne döne kötülerin basina patlayacak.
\par 20 Aklinin tasarladigini tümüyle yapana dek RAB'bin öfkesi dinmeyecek. Son günlerde açikça anlayacaksiniz bunu.
\par 21 Bu peygamberleri ben göndermedim, Ama çabucak ortaya çiktilar. Onlara hiç seslenmedim, Yine de peygamberlik ettiler.
\par 22 Ama meclisimde dursalardi, Sözlerimi halkima bildirir, Onlari kötü yollarindan ve davranislarindan Döndürürlerdi.
\par 23 Ben yalnizca yakindaki Tanri miyim? Uzaktaki Tanri da degil miyim?" diyor RAB,
\par 24 "Kim gizli yere saklanir da Onu görmem?" diyor RAB, "Yeri gögü doldurmuyor muyum?" diyor RAB.
\par 25 "Adimla yalanci peygamberlik edenlerin ne dediklerini duydum. 'Bir düs gördüm! Bir düs! diyorlar.
\par 26 Kafalarindan uydurduklari hileleri aktaran bu yalanci peygamberler ne zamana dek sürdürecekler bunu?
\par 27 Atalari nasil Baal* yüzünden adimi unuttuysa, onlar da birbirlerine düslerini anlatarak halkima adimi unutturmayi tasarliyorlar.
\par 28 Düsü olan peygamber düsünü anlatsin; ama sözümü alan onu sadakatle bildirsin. Bugdayin yaninda saman nedir ki?" diyor RAB.
\par 29 "Benim sözüm ates gibi degil mi? Kayalari paramparça eden balyoz gibi degil mi?" RAB böyle diyor.
\par 30 "Iste bunun için sözlerimi birbirlerinden çalan peygamberlere karsiyim" diyor RAB.
\par 31 "Evet, kendi sözlerini söyleyip, 'RAB böyle diyor diyen peygamberlere karsiyim" diyor RAB.
\par 32 "Uydurma düsler gören peygamberlere karsiyim" diyor RAB. "Bu düsleri anlatiyor, yalanlarla, bos övünmelerle halkimi bastan çikariyorlar. Ben onlari ne gönderdim, ne de atadim. Bu halka hiç mi hiç yararlari yok" diyor RAB.
\par 33 "Halktan biri, bir peygamber ya da kâhin, 'RAB'bin bildirisi nedir? diye sorarsa, 'Ne bildirisi? diye karsilik vereceksin. Sizi basimdan atacagim" diyor RAB.
\par 34 "Eger bir peygamber, kâhin ya da baska biri, 'Bu RAB'bin bildirisidir derse, onu da ailesini de cezalandiracagim.
\par 35 Her biriniz komsunuza ve kardesinize, 'RAB ne yanit verdi? ya da, 'RAB ne söyledi? demelisiniz.
\par 36 Bundan böyle, 'RAB'bin bildirisi lafini agziniza almayacaksiniz. Herkesin sözü kendi bildirisi olacak. Yasayan Tanri'nin, Her Seye Egemen RAB'bin, Tanrimiz'in sözlerini çarpitiyorsunuz siz.
\par 37 Bir peygambere, 'RAB sana ne yanit verdi? ya da, 'RAB ne söyledi? demelisiniz.
\par 38 Ama, 'RAB'bin bildirisidir derseniz, RAB diyor ki, 'RAB'bin bildirisidir diyorsunuz. Oysa, 'RAB'bin bildirisidir demeyeceksiniz diye sizi uyarmistim.
\par 39 Bu yüzden sizi büsbütün unutacagim, sizi de size ve atalariniza verdigim kenti de önümden söküp atacagim.
\par 40 Sizi hiç unutulmayacak bir utanca düsürecek, sürekli alay konusu edecegim."

\chapter{24}

\par 1 Babil Krali Nebukadnessar Yahuda Krali Yehoyakim oglu Yehoyakin'le Yahuda önderlerini, zanaatçilari, demircileri Yerusalim'den Babil'e sürdükten sonra, RAB bana tapinaginin önüne konmus iki sepet incir gösterdi.
\par 2 Sepetlerin birinde ilk ürüne benzer çok iyi incirler vardi; ötekindeyse çok kötü, yenmeyecek kadar çürük incirler vardi.
\par 3 RAB, "Yeremya, ne görüyorsun?" diye sordu. "Incir" diye yanitladim, "Iyi incirler çok iyi, öbürleriyse çok kötü, yenmeyecek kadar çürük."
\par 4 Bunun üzerine RAB bana söyle seslendi:
\par 5 "Israil'in Tanrisi RAB diyor ki, 'Buradan Kildan* ülkesine sürgüne gönderdigim Yahuda sürgünlerini bu iyi incirler gibi iyi sayacagim.
\par 6 Iyilik bulmalari için onlari gözetecek, bu ülkeye geri getirecegim. Onlari bina edecegim, yikmayacagim; onlari dikecegim, kökünden sökmeyecegim.
\par 7 Benim RAB oldugumu anlayacak bir yürek verecegim onlara. Onlar benim halkim olacaklar, ben de onlarin Tanrisi olacagim. Çünkü bütün yürekleriyle bana dönecekler.
\par 8 "'Ama Yahuda Krali Sidkiya'yla önderlerini, Yerusalim'den sag çikip da bu ülkede ya da Misir'da yasayanlari yenmeyecek kadar çürük incir gibi yapacagim diyor RAB,
\par 9 'Onlari bütün ülkelerin gözünde igrenç, korkunç bir duruma düsürecegim. Onlari sürdügüm her yerde ayiplanacak, ibret olacak, alaya alinacak, lanetlenecekler.
\par 10 Kendilerine ve atalarina verdigim topraktan yok olana dek üzerlerine kiliç, kitlik, salgin hastalik salacagim."

\chapter{25}

\par 1 Yahuda Krali Yosiya oglu Yehoyakim'in döneminin dördüncü yilinda, RAB Yahuda halkiyla ilgili olarak Yeremya'ya seslendi. Nebukadnessar'in Babil Krali olusunun birinci yiliydi bu.
\par 2 Peygamber Yeremya Yahuda halkina ve Yerusalim'de yasayanlara RAB'bin sözünü aktararak söyle dedi:
\par 3 Yirmi üç yildir, Yahuda Krali Amon oglu Yosiya'nin döneminin on üçüncü yilindan bugüne dek, RAB bana sesleniyor. Ben de RAB'bin sözünü defalarca size aktardim, ama dinlemediniz.
\par 4 RAB peygamber kullarini defalarca size gönderdi, ama dinlemediniz, kulak asmadiniz.
\par 5 Sizi uyardilar, "Simdi herkes kötü yolundan, kötü islerinden dönsün ki, RAB'bin sonsuza dek size ve atalariniza verdigi toprakta yasayasiniz" dediler,
\par 6 "Kulluk etmek, tapinmak için baska ilahlarin ardinca gitmeyin; elinizle yaptiginiz putlarla beni öfkelendirmeyin ki, ben de size zarar vermeyeyim."
\par 7 "Ama beni dinlemediniz" diyor RAB, "Sonuç olarak elinizle yaptiginiz putlarla beni öfkelendirip kendinizi zarara soktunuz."
\par 8 Bunun için Her Seye Egemen RAB diyor ki, "Madem sözlerimi dinlemediniz,
\par 9 ben de bütün kuzeydeki halklari ve kulum Babil Krali Nebukadnessar'i çagirtacagim" diyor RAB, "Onlari bu ülkeye de burada yasayanlarla çevresindeki bütün uluslara da karsi getirecegim. Bu halki tamamen yok edecegim, ülkelerini dehset ve alay konusu edip sonsuz bir viraneye çevirecegim.
\par 10 Sevinç ve nese sesini, gelin güvey sesini, degirmen taslarinin sesini, kandil isigini onlardan uzaklastiracagim.
\par 11 Bütün ülke bir virane, dehset verici bir yer olacak. Bu uluslar Babil Krali'na yetmis yil kulluk edecekler.
\par 12 "Ama yetmis yil dolunca" diyor RAB, "Suçlari yüzünden Babil Krali'yla ulusunu, Kildan* ülkesini cezalandiracak, sonsuz bir viranelik haline getirecegim.
\par 13 O ülke için söylediklerimin hepsini, Yeremya'nin uluslara ettigi bu kitapta yazili bütün peygamberlik sözlerini ülkenin basina getirecegim.
\par 14 Pek çok ulus, büyük krallar onlari köle edinecek. Yaptiklarina ve ellerinden çikan ise göre karsilik verecegim onlara."
\par 15 Israil'in Tanrisi RAB bana söyle dedi: "Elimdeki öfke sarabiyla dolu kâseyi* al, seni gönderecegim bütün uluslara içir.
\par 16 Sarabi içince sendeleyecek, üzerlerine gönderecegim kiliç yüzünden çildiracaklar."
\par 17 Böylece kâseyi RAB'bin elinden alip beni gönderdigi bütün uluslara içirdim:
\par 18 Bugün oldugu gibi viranelik, dehset ve alay konusu, lanetlik olsunlar diye Yerusalim'e, Yahuda kentlerine, krallariyla önderlerine;
\par 19 Firavunla görevlilerine, önderlerine ve halkina,
\par 20 Misir'da yasayan bütün yabancilara; Ûs krallarina, Filist krallarina -Askelon'a, Gazze'ye, Ekron'a, Asdot'tan sag kalanlara-
\par 21 Edom'a, Moav'a, Ammon'a;
\par 22 bütün Sur ve Sayda krallarina, deniz asiri ülkelerin krallarina;
\par 23 Dedan'a, Tema'ya, Bûz'a, zülüflerini kesen bütün halklara;
\par 24 Arabistan krallarina, çölde yasayan yabanci halkin krallarina;
\par 25 Zimri, Elam, Med krallarina;
\par 26 sirasiyla uzak yakin bütün kuzey krallarina, yeryüzündeki bütün uluslarin krallarina içirdim. Hepsinden sonra Sesak Krali da içecektir.
\par 27 "Sonra onlara de ki, 'Israil'in Tanrisi, Her Seye Egemen RAB söyle diyor: Üzerinize salacagim kiliç yüzünden sarhos olana dek için, kusun, düsüp kalkmayin.
\par 28 Eger kâseyi elinden alir, içmek istemezlerse, de ki, 'Her Seye Egemen RAB söyle diyor: Kesinlikle içeceksiniz!
\par 29 Bana ait olan kentin üzerine felaket getirmeye basliyorum. Cezasiz kalacaginizi mi saniyorsunuz? Sizi cezasiz birakmayacagim. Iste dünyada yasayan herkesin üzerine kilici çagiriyorum. Her Seye Egemen RAB böyle diyor.
\par 30 "Bütün bu peygamberlik sözlerini onlara ilet: "'Yükseklerden kükreyecek RAB, Kutsal konutundan gürleyecek, Agilina siddetle kükreyecek. Dünyada yasayanlarin tümüne Üzüm ezenler gibi bagiracak.
\par 31 Gürültü yeryüzünün dört yaninda yankilanacak. Çünkü RAB uluslara dava açacak; Herkesi yargilayacak Ve kötüleri kilica teslim edecek" diyor RAB.
\par 32 Her Seye Egemen RAB diyor ki, "Iste felaket ulustan ulusa yayiliyor! Dünyanin dört bucagindan büyük kasirga kopuyor."
\par 33 O gün RAB dünyayi bir uçtan bir uca öldürülenlerle dolduracak. Onlar için yas tutulmayacak, toplanip gömülmeyecekler. Topragin üzerinde gübre gibi kalacaklar.
\par 34 Haykirin, ey çobanlar, Aci aci bagirin! Toprakta yuvarlanin, ey sürü baslari! Çünkü bogazlanma zamaniniz doldu, Degerli bir kap gibi düsüp parçalanacaksiniz.
\par 35 Çobanlar kaçamayacak, Sürü baslari kurtulamayacak!
\par 36 Duy çobanlarin haykirisini, Sürü baslarinin bagirisini! RAB onlarin otlaklarini yok ediyor.
\par 37 RAB'bin kizgin öfkesi yüzünden Güvenlikte olduklari agillar yok oldu.
\par 38 Inini terk eden genç aslan gibi, RAB inini birakti. Zorbanin kilici Ve RAB'bin kizgin öfkesi yüzünden Ülkeleri viraneye döndü.

\chapter{26}

\par 1 Yahuda Krali Yosiya oglu Yehoyakim'in kralliginin baslangicinda RAB söyle seslendi:
\par 2 "RAB diyor ki, RAB'bin Tapinagi'nin avlusunda dur, tapinmak için Yahuda kentlerinden oraya gelen herkese seslen. Sana buyurdugum her seyi tek söz eksiltmeden onlara bildir.
\par 3 Belki dinler de kötü yollarindan dönerler. O zaman ben de yaptiklari kötülükler yüzünden baslarina getirmeyi tasarladigim felaketten vazgeçerim.
\par 4 Onlara de ki, 'RAB söyle diyor: Size verdigim yasa uyarinca yürümez, beni dinlemez,
\par 5 size defalarca gönderdigim kullarim peygamberlerin sözlerine kulak vermezseniz, ki kulak vermiyorsunuz,
\par 6 bu tapinaga Silo'dakine yaptigimin aynisini yapar, bu kenti bütün dünya uluslari arasinda lanetlik ederim."
\par 7 Kâhinler, peygamberler ve bütün halk Yeremya'nin RAB'bin Tapinagi'nda söyledigi bu sözleri duydular.
\par 8 Yeremya Tanri'nin halka iletmesini buyurdugu sözleri bitirince, kâhinlerle peygamberler ve halk onu yakalayip, "Ölmen gerek!" dediler,
\par 9 "Neden bu tapinak Silo'daki gibi olacak, bu kent de içinde kimsenin yasamayacagi bir viraneye dönecek diyerek RAB'bin adiyla peygamberlik ediyorsun?" Bütün halk RAB'bin Tapinagi'nda Yeremya'nin çevresinde toplanmisti.
\par 10 Yahuda önderleri olup bitenleri duyunca, saraydan RAB'bin Tapinagi'na gidip tapinagin Yeni Kapi girisinde yerlerini aldilar.
\par 11 Bunun üzerine kâhinlerle peygamberler, önderlere ve halka, "Bu adam ölüm cezasina çarptirilmali" dediler, "Çünkü bu kente karsi peygamberlik etti. Kendi kulaklarinizla isittiniz bunu."
\par 12 Bunun üzerine Yeremya önderlerle halka, "Bu tapinaga ve kente karsi isittiginiz peygamberlik sözlerini iletmem için beni RAB gönderdi" dedi,
\par 13 "Simdi yollarinizi, davranislarinizi düzeltin, Tanriniz RAB'bin sözüne kulak verin. O zaman RAB basiniza getirecegini söyledigi felaketten vazgeçecek.
\par 14 Bana gelince, iste elinizdeyim! Gözünüzde iyi ve dogru olan neyse, bana öyle yapin.
\par 15 Ancak sunu kesinlikle bilin ki, eger beni öldürürseniz, siz de bu kent ve içinde yasayanlar da suçsuz birinin kanini dökmekten sorumlu tutulacaksiniz. Çünkü bütün bu sözleri bildirmem için beni gerçekten RAB size gönderdi."
\par 16 Bunun üzerine önderlerle halk, kâhinlerle peygamberlere, "Bu adam ölüm cezasina çarptirilmamali" dediler, "Çünkü bizimle Tanrimiz RAB'bin adina konustu."
\par 17 Halkin ileri gelenlerinden birkaçi öne çikip orada toplanmis halka,
\par 18 "Moresetli Mika Yahuda Krali Hizkiya döneminde peygamberlik etti" dediler, "Yahuda halkina dedi ki, Her Seye Egemen RAB söyle diyor, "'Siyon tarla gibi sürülecek, Tas yiginina dönecek Yerusalim, Tapinagin kuruldugu dag Çalilarla kaplanacak.
\par 19 "Yahuda Krali Hizkiya ya da Yahuda halkindan biri onu öldürdü mü? Bunun yerine Hizkiya RAB'den korkarak O'nun lütfunu diledi. RAB de onlara bildirdigi felaketten vazgeçti. Bizse, üzerimize büyük bir yikim getirmek üzereyiz."
\par 20 Kiryat-Yearimli Semaya oglu Uriya adinda peygamberlik eden bir adam daha vardi. Tipki Yeremya gibi o da RAB'bin adina bu kente ve ülkeye karsi peygamberlik etti.
\par 21 Kral Yehoyakim'le askerleri ve komutanlari Uriya'nin sözlerini duydular. Kral onu öldürmek istedi. Bunu duyan Uriya korkuya kapilarak kaçip Misir'a gitti.
\par 22 Bunun üzerine Kral Yehoyakim pesinden adamlarini -Akbor oglu Elnatan'la baskalarini- Misir'a gönderdi.
\par 23 Uriya'yi Misir'dan çikarip Kral Yehoyakim'e getirdiler. Kral onu kiliçla öldürtüp cesedini siradan halk mezarligina attirdi.
\par 24 Ancak Safan oglu Ahikam Yeremya'yi korudu. Böylece Yeremya öldürülmek üzere halkin eline teslim edilmedi.

\chapter{27}

\par 1 Yahuda Krali Yosiya oglu Sidkiya'nin kralliginin baslangicinda RAB Yeremya'ya seslendi:
\par 2 RAB bana dedi ki, "Kendine sirimla baglanmis tahta bir boyunduruk yap, boynuna geçir.
\par 3 Sonra Yerusalim'e, Yahuda Krali Sidkiya'ya gelen ulaklar araciligiyla Edom, Moav, Ammon, Sur ve Sayda krallarina haber gönder.
\par 4 Efendilerine sunu bildirmelerini buyur: Israil'in Tanrisi Her Seye Egemen RAB diyor ki, 'Efendilerinize söyleyin:
\par 5 Yeryüzünü de üstünde yasayan insanlarla hayvanlari da büyük gücümle, kudretli elimle ben yarattim. Onu uygun gördügüm kisiye veririm.
\par 6 Simdi bütün bu ülkeleri Babil Krali kulum Nebukadnessar'a verecegim. Yabanil hayvanlari da kulluk etsinler diye ona verecegim.
\par 7 Ülkesi için saptanan zaman gelinceye dek bütün uluslar ona, ogluna, torununa kulluk edecek. Sonra birçok ulus, büyük krallar onu köle edecekler.
\par 8 "'Hangi ulus ya da krallik Babil Krali Nebukadnessar'a kulluk edip boyunduruguna girmezse, o ulusu onun eline teslim edene dek kiliçla, kitlikla, salgin hastalikla cezalandiracagim, diyor RAB.
\par 9 Size gelince, peygamberlerinizi, falcilarinizi, düs görenlerinizi, medyumlarinizi, büyücülerinizi dinlemeyin! Onlar size, Babil Krali'na kulluk etmeyeceksiniz diyorlar.
\par 10 Size yalan peygamberlik ediyorlar. Bunun sonucu sizi ülkenizden uzaklastirmak oluyor. Sizi sürecegim, yok olacaksiniz.
\par 11 Ama Babil Krali'nin boyunduruguna girip ona kulluk eden ulusu kendi topraginda birakacagim, diyor RAB. O ulus topragini isleyecek, orada yasayacak."
\par 12 Yahuda Krali Sidkiya'ya bütün bu sözleri ilettim. Dedim ki, "Boyunlarinizi Babil Krali'nin boyundurugu altina koyun. Ona ve halkina kulluk edin ki sag kalasiniz.
\par 13 RAB'bin Babil Krali'na kulluk etmeyen her ulus için dedigi gibi, niçin sen ve halkin kiliç, kitlik, salgin hastalik yüzünden ölesiniz?
\par 14 Size, 'Babil Krali'na kulluk etmeyeceksiniz diyen peygamberlerin sözlerine kulak asmayin. Onlar size yalan peygamberlik ediyorlar.
\par 15 'Onlari ben göndermedim diyor RAB, 'Adimla yalan peygamberlik ediyorlar. Bu yüzden sizi de size peygamberlik eden peygamberleri de sürecegim, hepiniz yok olacaksiniz."
\par 16 Sonra kâhinlerle halka söyle dedim: "RAB diyor ki, 'Iste RAB'bin Tapinagi'nin esyalari yakinda Babil'den geri getirilecek diyen peygamberlerinize kulak asmayin. Onlar size yalan peygamberlik ediyorlar.
\par 17 Onlari dinlemeyin. Sag kalmak için Babil Krali'na kulluk edin. Bu kent neden viraneye çevrilsin?
\par 18 Eger bunlar peygamberse ve RAB'bin sözü onlardaysa, RAB'bin Tapinagi'nda, Yahuda Krali'nin sarayinda ve Yerusalim'de kalan esyalar Babil'e götürülmesin diye Her Seye Egemen RAB'be yalvarsinlar.
\par 19 "Çünkü Babil Krali Nebukadnessar Yahuda Krali Yehoyakim oglu Yehoyakin'i Yahuda ve Yerusalim soylulariyla birlikte Yerusalim'den Babil'e sürdügünde alip götürmedigi sütunlar, havuz, ayakliklar ve kentte kalan öbür esyalar için Her Seye Egemen RAB söyle diyor.
\par 21 Evet, RAB'bin Tapinagi'nda, Yahuda Krali'nin sarayinda ve Yerusalim'de kalan esyalar için Israil'in Tanrisi, Her Seye Egemen RAB söyle diyor:
\par 22 'Bunlar Babil'e tasinacak, aldiracagim güne dek de orada kalacaklar diyor RAB, 'O zaman onlari alacak, yeniden buraya yerlestirecegim."

\chapter{28}

\par 1 Ayni yil Yahuda Krali Sidkiya'nin kralliginin baslangicinda, dördüncü yilinin besinci ayinda* Givonlu Azzur oglu Peygamber Hananya RAB'bin Tapinagi'nda kâhinlerle bütün halkin önünde bana söyle dedi:
\par 2 "Israil'in Tanrisi, Her Seye Egemen RAB diyor ki, 'Babil Krali'nin boyundurugunu kiracagim.
\par 3 Babil Krali Nebukadnessar'in buradan alip Babil'e götürdügü RAB'bin Tapinagi'na ait bütün esyalari iki yil içinde buraya geri getirecegim.
\par 4 Yahuda Krali Yehoyakim oglu Yehoyakin'le Babil'e sürgüne giden bütün Yahudalilar'i buraya geri getirecegim diyor RAB, 'Çünkü Babil Krali'nin boyundurugunu kiracagim."
\par 5 Bunun üzerine Peygamber Yeremya, kâhinlerin ve RAB'bin Tapinagi'ndaki halkin önünde Peygamber Hananya'yi yanitladi.
\par 6 Yeremya söyle dedi: "Amin! RAB aynisini yapsin! RAB, tapinagina ait esyalarla bütün sürgünleri Babil'den buraya geri getirerek ettigin peygamberlik sözlerini gerçeklestirsin!
\par 7 Yalniz simdi sana ve halka söyleyecegim su sözü dinle:
\par 8 Çok önce, benden de senden de önce yasamis peygamberler birçok ülke ve büyük kralligin basina savas, felaket, salgin hastalik gelecek diye peygamberlik ettiler.
\par 9 Ancak esenlik olacagini söyleyen peygamberin sözü yerine gelirse, onun gerçekten RAB'bin gönderdigi peygamber oldugu anlasilir."
\par 10 Bunun üzerine Peygamber Hananya, Peygamber Yeremya'nin boynundan boyundurugu çikarip kirdi.
\par 11 Sonra halkin önünde söyle dedi: "RAB diyor ki, 'Babil Krali Nebukadnessar'in bütün uluslarin boynuna taktigi boyundurugu iki yil içinde iste böyle kiracagim." Böylece Peygamber Yeremya yoluna gitti.
\par 12 Peygamber Hananya Peygamber Yeremya'nin boynundaki boyundurugu kirdiktan sonra RAB Yeremya'ya söyle seslendi:
\par 13 "Git, Hananya'ya de ki, 'RAB söyle diyor: Sen tahtadan yapilmis boyundurugu kirdin, ama yerine demir boyunduruk yapacaksin!
\par 14 Çünkü Israil'in Tanrisi, Her Seye Egemen RAB diyor ki, Babil Krali Nebukadnessar'a kulluk etmeleri için bütün bu uluslarin boynuna demir boyunduruk geçirdim, ona kulluk edecekler. Yabanil hayvanlari da onun denetimine verecegim."
\par 15 Peygamber Yeremya Peygamber Hananya'ya, "Dinle, ey Hananya!" dedi, "Seni RAB göndermedi. Ama sen bu ulusu yalana inandirdin.
\par 16 Bu nedenle RAB diyor ki, 'Seni yeryüzünden silip atacagim. Bu yil öleceksin. Çünkü halki RAB'be karsi kiskirttin."
\par 17 Peygamber Hananya o yilin yedinci ayinda öldü.

\chapter{29}

\par 1 Peygamber Yeremya'nin sürgünde sag kalan halkin ileri gelenlerine, kâhinlerle peygamberlere ve Nebukadnessar'in Yerusalim'den Babil'e sürdügü bütün halka Yerusalim'den gönderdigi mektubun metni asagida yazilidir.
\par 2 Bu mektup Kral Yehoyakin'in, ana kraliçenin, saray görevlilerinin, Yahuda ve Yerusalim önderlerinin, zanaatçilarla demircilerin Yerusalim'den sürgüne gitmelerinden sonra,
\par 3 Yahuda Krali Sidkiya'nin Babil Krali Nebukadnessar'a gönderdigi Safan oglu Elasa ve Hilkiya oglu Gemarya eliyle gönderildi:
\par 4 Israil'in Tanrisi, Her Seye Egemen RAB Yerusalim'den Babil'e sürdügü herkese söyle diyor:
\par 5 "Evler yapip içinde oturun, bahçe dikip ürününü yiyin.
\par 6 Evlenin, ogullariniz, kizlariniz olsun; ogullarinizi, kizlarinizi evlendirin. Onlarin da ogullari, kizlari olsun. Orada çogalin, azalmayin.
\par 7 Sizi sürmüs oldugum kentin esenligi için ugrasin. O kent için RAB'be dua edin. Çünkü esenliginiz onunkine baglidir."
\par 8 Evet, Israil'in Tanrisi, Her Seye Egemen RAB diyor ki, "Aranizdaki peygamberlerle falcilara aldanmayin. Düs görmeye özendirdiginiz kisilere kulak asmayin.
\par 9 Çünkü onlar adimi kullanarak size yalan peygamberlik ediyorlar. Onlari ben göndermedim." RAB böyle diyor.
\par 10 RAB diyor ki, "Babil'de yetmis yiliniz dolunca sizinle ilgilenecek, buraya sizi geri getirmek için verdigim iyi sözü tutacagim.
\par 11 Çünkü sizin için düsündügüm tasarilari biliyorum" diyor RAB. "Kötü tasarilar degil, size umutlu bir gelecek saglayan esenlik tasarilari bunlar.
\par 12 O zaman beni çagiracak, gelip bana yakaracaksiniz. Ben de sizi isitecegim.
\par 13 Beni arayacaksiniz, bütün yüreginizle arayinca beni bulacaksiniz.
\par 14 Kendimi size buldurtacagim" diyor RAB. "Sizi eski gönencinize kavusturacagim. Sizi sürdügüm bütün yerlerden ve uluslardan toplayacagim" diyor RAB. "Ve sizi sürgün ettigim yerden geri getirecegim."
\par 15 Madem, "RAB bizim için Babil'de peygamberler çikardi" diyorsunuz,
\par 16 Davut'un tahtinda oturan kral, bu kentte kalan halk ve sizinle sürgüne gitmeyen yurttaslariniz için RAB söyle diyor.
\par 17 Evet, Her Seye Egemen RAB diyor ki, "Üzerlerine kiliç, kitlik, salgin hastalik gönderecegim. Onlari yenilmeyecek kadar çürük, bozuk incir gibi edecegim.
\par 18 Kiliçla, kitlikla, salgin hastalikla peslerine düsecegim. Yeryüzündeki bütün kralliklara dehset saçacak, kendilerini sürdügüm bütün ülkeler arasinda lanetlenecek, dehset ve alay konusu olacak, asagilanacaklar.
\par 19 Çünkü sözlerime kulak asmadilar" diyor RAB. "Kullarim peygamberler araciligiyla sözlerimi defalarca gönderdim, ama dinlemediniz." RAB böyle diyor.
\par 20 Bunun için, ey sizler, Yerusalim'den Babil'e gönderdigim sürgünler, RAB'bin sözüne kulak verin!
\par 21 Adimi kullanarak size yalan peygamberlik eden Kolaya oglu Ahav'la Maaseya oglu Sidkiya için Israil'in Tanrisi, Her Seye Egemen RAB diyor ki, "Onlari Babil Krali Nebukadnessar'in eline teslim edecegim, gözünüzün önünde ikisini de öldürecek.
\par 22 Onlardan ötürü Babil'deki bütün Yahuda sürgünleri, 'RAB seni Babil Krali'nin diri diri yaktirdigi Sidkiya ve Ahav'la ayni duruma düsürsün diye lanet edecek.
\par 23 Çünkü Israil'de çirkin seyler yaptilar; komsularinin karilariyla zina ettiler, onlara buyurmadigim halde adimla yalan sözler söylediler. Bunu biliyorum ve buna tanigim" diyor RAB.
\par 24 Nehelamli Semaya'ya diyeceksin ki,
\par 25 "Israil'in Tanrisi, Her Seye Egemen RAB söyle diyor: Sen Yerusalim halkina, Maaseya oglu Kâhin Sefanya'yla bütün öbür kâhinlere kendi adina mektuplar göndererek söyle yazmistin:
\par 26 "'RAB tapinagin sorumlusu olarak Yehoyada yerine seni kâhin atadi. Peygamber gibi davranan her deliyi tomruga, demir boyunduruga vurmak görevindir.
\par 27 Öyleyse aranizda kendini peygamber ilan eden Anatotlu Yeremya'yi neden azarlamadin?
\par 28 Çünkü Yeremya biz Babil'dekilere su haberi gönderdi: Sürgün uzun olacak. Onun için evler yapip içinde oturun; bahçe dikip ürününü yiyin."
\par 29 Kâhin Sefanya mektubu Peygamber Yeremya'ya okuyunca,
\par 30 RAB Yeremya'ya söyle seslendi:
\par 31 "Bütün sürgünlere su haberi gönder: 'Nehelamli Semaya için RAB diyor ki: Madem ben göndermedigim halde Semaya peygamberlik edip sizleri yalana inandirdi,
\par 32 ben de Nehelamli Semaya'yi ve bütün soyunu cezalandiracagim: Bu halkin arasinda soyundan kimse sag kalmayacak, halkima yapacagim iyiligi görmeyecek, diyor RAB. Çünkü o halki bana karsi kiskirtti."

\chapter{30}

\par 1 RAB Yeremya'ya söyle seslendi:
\par 2 "Israil'in Tanrisi RAB diyor ki, 'Sana bildirdigim bütün sözleri bir kitaba yaz.
\par 3 Iste halkim Israil'i ve Yahuda'yi eski gönençlerine kavusturacagim günler yaklasiyor diyor RAB, 'Onlari atalarina verdigim topraklara geri getirecegim, orayi yurt edinecekler diyor RAB."
\par 4 Israil ve Yahuda için RAB'bin bildirdigi sözler sunlardir:
\par 5 "RAB diyor ki, 'Korku sesi duyduk, Esenlik degil, dehset sesi.
\par 6 Sorun da görün: Erkek, çocuk dogurur mu? Öyleyse neden doguran kadin gibi Her erkegin ellerini belinde görüyorum? Neden her yüz solmus?
\par 7 Ah, ne korkunç gün! Onun gibisi olmayacak. Yakup soyu için sikinti dönemi olacak, Yine de sikintidan kurtulacak.
\par 8 "'O gün diyor Her Seye Egemen RAB, 'Boyunlarindaki boyundurugu kiracak, Baglarini koparacagim. Bundan böyle yabancilar onlari Kendilerine köle etmeyecekler.
\par 9 Onun yerine Tanrilari RAB'be Ve baslarina atayacagim krallari Davut'a Kulluk edecekler.
\par 10 "'Korkma, ey kulum Yakup, Yilma, ey Israil diyor RAB. 'Çünkü seni uzak yerlerden, Soyunu sürgün edildigi ülkeden kurtaracagim. Yakup yine huzur ve güvenlik içinde olacak, Kimse onu korkutmayacak.
\par 11 Çünkü ben seninleyim, Seni kurtaracagim diyor RAB. 'Seni aralarina dagittigim bütün uluslari Tümüyle yok etsem de, Seni büsbütün yok etmeyecek, Adaletle yola getirecek, Hiç cezasiz birakmayacagim.
\par 12 "RAB diyor ki, 'Senin yaran sifa bulmaz, Beren iyilesmez.
\par 13 Davani görecek kimse yok, Yaran umarsiz, sifa bulmaz.
\par 14 Bütün oynaslarin unuttu seni, Arayip sormuyorlar. Seni düsman vururcasina vurdum, Acimasizca cezalandirdim. Çünkü suçun çok, Günahlarin sayisiz.
\par 15 Neden haykiriyorsun yarandan ötürü? Yaran sifa bulmaz. Suçlarinin çoklugu ve sayisiz günahin yüzünden Getirdim bunlari basina.
\par 16 Ama seni yiyenlerin hepsi yem olacak, Bütün düsmanlarin sürgüne gidecek. Seni soyanlar soyulacak, Yagmalayanlar yagmalanacak.
\par 17 Ama ben seni sagligina kavusturacak, Yaralarini iyilestirecegim diyor RAB, 'Çünkü Siyon itilmis, Onu arayan soran yok diyorlar.
\par 18 "RAB diyor ki, 'Yakup'un çadirlarini eski gönencine kavusturacagim, Konutlarina aciyacagim. Yerusalim höyük üzerinde yeniden kurulacak, Saray kendi yerinde duracak.
\par 19 Oralardan sükran ve sevinç sesleri duyulacak. Sayilarini çogaltacagim, azalmayacaklar, Onlari onurlandiracagim, küçümsenmeyecekler.
\par 20 Çocuklari eskisi gibi olacak, Topluluklari önümde saglam duracak; Onlara baski yapanlarin hepsini cezalandiracagim.
\par 21 Önderleri kendilerinden biri olacak, Yöneticileri kendi aralarindan çikacak. Onu kendime yaklastiracagim, Bana yaklasacak. Kim cani pahasina yaklasabilir bana? diyor RAB.
\par 22 "'Böylece siz benim halkim olacaksiniz, Ben de sizin Tanriniz olacagim.
\par 23 Iste RAB'bin firtinasi öfkeyle kopacak, Siddetli kasirgasi kötülerin basinda patlayacak.
\par 24 Aklinin tasarladigini tümüyle yapana dek, RAB'bin kizgin öfkesi dinmeyecek. Son günlerde bunu anlayacaksiniz."

\chapter{31}

\par 1 "O zaman" diyor RAB, "Bütün Israil boylarinin Tanrisi olacagim, onlar da benim halkim olacaklar."
\par 2 RAB diyor ki, "Kiliçtan kaçip kurtulan halk çölde lütuf buldu. Ben Israil'i rahata kavusturmaya gelirken,
\par 3 Ona uzaktan görünüp söyle dedim: Seni sonsuz bir sevgiyle sevdim, Bu nedenle sevecenlikle seni kendime çektim.
\par 4 Seni yeniden bina edecegim, Yeniden bina edileceksin, ey erden kiz Israil! Yine teflerini alacak, Sevinçle cosup oynayanlara katilacaksin.
\par 5 Samiriye daglarinda yine bag dikeceksin; Bag dikenler üzümünü yiyecekler.
\par 6 Efrayim'in daglik bölgesindeki bekçilerin, 'Haydi, Siyon'a, Tanrimiz RAB'be çikalim' Diye bagiracaklari bir gün var."
\par 7 RAB diyor ki, "Yakup için sevinçle haykirin! Uluslarin basi olan için bagirin! Övgülerinizi duyurun! 'Ya RAB, halkini, Israil'den sag kalanlari kurtar deyin.
\par 8 Iste, onlari kuzey ülkesinden Geri getirmek üzereyim; Onlari dünyanin dört bucagindan toplayacagim. Aralarinda kör, topal, Gebe kadin da, doguran kadin da olacak. Büyük bir topluluk olarak buraya dönecekler.
\par 9 Aglaya aglaya gelecekler, Benden yardim dileyenleri geri getirecegim. Akarsular boyunca tökezlemeyecekleri Düz bir yolda yürütecegim onlari. Çünkü ben Israil'in babasiyim, Efrayim de ilk oglumdur.
\par 10 "RAB'bin sözünü dinleyin, ey uluslar! Uzaktaki kiyilara duyurun: 'Israil'i dagitan onu toplayacak, Sürüsünü kollayan çoban gibi kollayacak onu deyin.
\par 11 Çünkü RAB Yakup'u kurtaracak, Onu kendisinden güçlü olanin elinden özgür kilacak.
\par 12 Siyon'un yüksek tepelerine gelip Sevinçle haykiracaklar. RAB'bin verdigi iyilikler karsisinda -Tahil, yeni sarap, zeytinyagi, Davar ve sigir yavrulari karsisinda- Yüzleri sevinçle parlayacak. Sulanmis bahçe gibi olacak, Bir daha solmayacaklar.
\par 13 O zaman erden kizlar, genç yasli erkekler Hep birlikte oynayip sevinecek. Yaslarini coskuya çevirecek, Üzüntülerini avutup onlari sevindirecegim.
\par 14 Kâhinleri bol yiyecekle doyuracagim, Halkim iyiliklerimle doyacak" diyor RAB.
\par 15 RAB diyor ki, "Rama'da bir ses duyuldu, Aglayis ve aci feryat sesleri! Çocuklari için aglayan Rahel avutulmak istemiyor. Çünkü onlar yok artik!"
\par 16 RAB diyor ki, "Sesini aglamaktan, Gözlerini yas dökmekten alikoy. Çünkü verdigin emek ödüllendirilecek" diyor RAB. "Halkim düsman ülkesinden geri dönecek.
\par 17 Gelecegin için umut var" diyor RAB. "Çocuklarin yurtlarina dönecekler.
\par 18 "Efrayim'in inlemelerini kuskusuz duydum: 'Beni egitilmemis dana gibi yola getirdin Ve yola geldim. Beni geri getir, döneyim. Çünkü RAB Tanrim sensin.
\par 19 Yanlis yola saptiktan sonra pisman oldum. Aklim basima gelince bagrimi dövdüm. Gençligimdeki ayiplarimdan utandim, Rezil oldum.
\par 20 "Efrayim degerli oglum degil mi? Hosnut oldugum çocuk degil mi? Kendisi için ne dersem diyeyim, Onu hiç unutmuyorum. Bu yüzden yüregim sizliyor, Çok aciyorum ona" diyor RAB.
\par 21 "Kendin için yol isaretleri koy, Direkler dik. Yolunu, gittigin yolu iyi düsün. Geri dön, ey erden kiz Israil, kentlerine dön!
\par 22 Ne zamana dek bocalayip duracaksin, ey dönek kiz? RAB dünyada yeni bir sey yaratti: Kadin erkegi koruyacak."
\par 23 Israil'in Tanrisi, Her Seye Egemen RAB söyle diyor: "Yahuda ve kentlerindeki halki eski gönençlerine kavusturdugum zaman yine su sözleri söyleyecekler: 'RAB sizi kutsasin, Ey dogruluk yurdu, ey kutsal dag!
\par 24 Halk, irgatlar, sürüleriyle dolasan çobanlar Yahuda'da ve kentlerinde birlikte yasayacak.
\par 25 Yorgun cana kana kana içirecek, bitkin cani doyuracagim."
\par 26 Bunun üzerine uyanip baktim. Uykum bana tatli geldi.
\par 27 "Israil ve Yahuda'da insan ve hayvan tohumu ekecegim günler yaklasiyor" diyor RAB,
\par 28 "Kökünden söküp yok etmek, yerle bir edip yikmak, yikima ugratmak için onlari nasil gözledimse, kurup dikmek için de gözleyecegim" diyor RAB.
\par 29 "O günler insanlar artik, 'Babalar koruk yedi, Çocuklarin disleri kamasti' demeyecekler.
\par 30 Herkes kendi suçu yüzünden ölecek. Koruk yiyenin disleri kamasacak.
\par 31 "Israil halkiyla ve Yahuda halkiyla Yeni bir antlasma yapacagim günler geliyor" diyor RAB,
\par 32 "Atalarini Misir'dan çikarmak için Ellerinden tuttugum gün Onlarla yaptigim antlasmaya benzemeyecek. Onlarin kocasi olmama karsin, Bozdular o antlasmami" diyor RAB.
\par 33 "Ama o günlerden sonra Israil halkiyla Yapacagim antlasma sudur" diyor RAB, "Yasami içlerine yerlestirecek, Yüreklerine yazacagim. Ben onlarin Tanrisi olacagim, Onlar da benim halkim olacak.
\par 34 Bundan böyle kimse komsusunu ya da kardesini, 'RAB'bi taniyin diye egitmeyecek. Çünkü küçük büyük hepsi Taniyacak beni" diyor RAB. "Çünkü suçlarini bagislayacagim, Günahlarini artik anmayacagim."
\par 35 Gündüz isik olsun diye günesi saglayan, Gece isik olsun diye ayi, yildizlari düzene koyan, Dalgalari kükresin diye denizi kabartan RAB -O'nun adi Her Seye Egemen RAB'dir- diyor ki,
\par 36 "Eger kurulan bu düzen önümden kalkarsa, Israil soyu sonsuza dek Önümde ulus olmaktan çikar" diyor RAB.
\par 37 RAB söyle diyor: "Gökler ölçülebilse, Dünyanin temelleri incelenip anlasilabilse, Israil soyunu bütün yaptiklari yüzünden Reddederim" diyor RAB.
\par 38 "Yerusalim Kenti'nin Hananel Kulesi'nden Köse Kapisi'na dek benim için yeniden kurulacagi günler geliyor" diyor RAB,
\par 39 "Ölçü ipi oradan Garev Tepesi'ne dogru uzayip Goa'ya dönecek.
\par 40 Ölülerle küllerin atildigi bütün vadi, Kidron Vadisi'ne dek uzanan tarlalar, doguda At Kapisi'nin kösesine dek RAB için kutsal olacak. Kent bir daha kökünden sökülmeyecek, sonsuza dek yikilmayacak."

\chapter{32}

\par 1 Yahuda Krali Sidkiya'nin onuncu, Nebukadnessar'in on sekizinci yilinda RAB Yeremya'ya seslendi.
\par 2 O sirada Babil Krali'nin ordusu Yerusalim'i kusatmaktaydi. Peygamber Yeremya Yahuda Krali'nin sarayindaki muhafiz avlusunda tutukluydu.
\par 3 Yahuda Krali Sidkiya onu orada tutuklatmisti. "Neden böyle peygamberlik ediyorsun?" demisti, "Sen diyorsun ki, 'RAB söyle diyor: Bu kenti Babil Krali'nin eline teslim etmek üzereyim, onu ele geçirecek.
\par 4 Yahuda Krali Sidkiya Kildaniler'in* elinden kaçip kurtulamayacak, kesinlikle Babil Krali'nin eline teslim edilecek; onunla yüzyüze konusacak, onu gözleriyle görecek.
\par 5 Sidkiya Babil'e götürülecek, ben onunla ilgilenene dek orada kalacak, Kildaniler'le savassaniz bile basarili olamayacaksiniz diyor RAB."
\par 6 Yeremya, "RAB bana söyle seslendi" diye yanitladi,
\par 7 "Amcan Sallum oglu Hanamel sana gelip, 'Anatot'taki tarlami satin al. Çünkü en yakin akrabam olarak tarlayi satin alma hakki senindir diyecek.
\par 8 "Sonra RAB'bin sözü uyarinca amcamin oglu Hanamel muhafiz avlusunda yanima gelip, 'Benyamin bölgesinde, Anatot'taki tarlami satin al dedi, 'Çünkü miras hakki da en yakin akrabalik hakki da senindir. Onu kendin için satin al. "O zaman RAB'bin sözünün yerine geldigini anladim.
\par 9 Böylece Anatot'taki tarlayi amcamin oglu Hanamel'den satin aldim. Tarlaya karsilik kendisine on yedi sekel gümüs tartip ödedim.
\par 10 Satis belgesini çagirdigim taniklarin önünde imzalayip mühürledim, gümüsü terazide tarttim.
\par 11 Satis belgesini -kural ve kosullari içeren mühürlenmis kâgidi ve açik sözlesme belgesini- aldim.
\par 12 Amcamin oglu Hanamel'in, satis belgesini imzalayan taniklarin ve muhafiz avlusunda oturan bütün Yahudiler'in gözü önünde satis belgesini Mahseya oglu Neriya oglu Baruk'a verdim.
\par 13 "Hepsinin gözü önünde Baruk'a su buyruklari verdim:
\par 14 'Israil'in Tanrisi, Her Seye Egemen RAB diyor ki, Bu satis belgesini -mühürlenmis, açik olanini- al, uzun süre durmak üzere bir çömlege koy.
\par 15 Çünkü Israil'in Tanrisi, Her Seye Egemen RAB söz veriyor, bu ülkede yine evler, tarlalar, baglar satin alinacak diyor.
\par 16 "Tarlanin satis belgesini Neriya oglu Baruk'a verdikten sonra RAB'be söyle yakardim:
\par 17 "Ey Egemen RAB! Büyük gücünle, kudretinle yeri gögü yarattin. Yapamayacagin hiçbir sey yok.
\par 18 Binlerce insana sevgi gösterir, ama babalarin isledigi günahlarin karsiligini çocuklarina ödetirsin. Ey büyük ve güçlü Tanri! Her Seye Egemen RAB'dir senin adin.
\par 19 Tasarilarin ne büyük, islerin ne güçlü! Gözlerin insanlarin bütün yaptiklarina açiktir. Herkese davranislarina, yaptiklarinin sonucuna göre karsiligini verirsin.
\par 20 Sen ki, Misir'da, Israil'de, bütün insanlar arasinda bugüne dek mucizeler, harikalar yarattin. Bugün oldugu gibi ün kazandin.
\par 21 Halkin Israil'i belirtilerle, sasilasi islerle, güçlü, kudretli elinle, büyük korku saçarak Misir'dan çikardin.
\par 22 Atalarina verecegine ant içtigin bu topraklari, süt ve bal akan ülkeyi onlara verdin.
\par 23 Gelip ülkeyi mülk edindiler, ama senin sözünü dinlemediler, Kutsal Yasan uyarinca yürümediler. Yapmalarini buyurdugun seylerin hiçbirini yapmadilar. Bu yüzden bütün bu felaketleri getirdin baslarina.
\par 24 "Iste, kenti ele geçirmek için kusatma rampalari yapildi. Kiliç, kitlik, salgin hastalik yüzünden kent saldiran Kildaniler'e teslim edilecek. Söylediklerin yerine geldi, sen de görüyorsun!
\par 25 Yine de, Egemen RAB, kent Kildaniler'e teslim edilecegi halde sen bana, 'Tarlayi çagirdigin taniklar önünde gümüsle satin al dedin."
\par 26 Bunun üzerine RAB Yeremya'ya söyle seslendi:
\par 27 "Bütün insanligin Tanrisi RAB benim. Var mi yapamayacagim bir sey?
\par 28 Bu yüzden RAB diyor ki: Bak, bu kenti Kildaniler'le Babil Krali Nebukadnessar'in eline vermek üzereyim; onu ele geçirsin.
\par 29 Kente saldiran Kildaniler gelip onu atese verecekler. Kenti de damlarinda Baal'in* onuruna buhur yakip baska ilahlara dökmelik sunular sunarak beni öfkelendirdikleri evleri de yakacaklar.
\par 30 "Çünkü Israil ve Yahuda halklari gençliklerinden beri hep gözümde kötü olani yapiyor; Israil halki ellerinin yaptiklariyla beni sürekli öfkelendiriyor, diyor RAB.
\par 31 Evet, bu kent kuruldugundan bu yana beni öyle öfkelendirdi, kizdirdi ki onu önümden söküp atacagim.
\par 32 Çünkü Israil ve Yahuda halklarinin -kendilerinin, krallarinin, önderlerinin, kâhinlerinin, peygamberlerinin, Yahuda ve Yerusalim'de yasayanlarin- beni öfkelendirmek için yaptiklari kötülüklerin haddi hesabi yok.
\par 33 Bana yüzlerini degil, sirtlarini çevirdiler. Onlari defalarca uyarmama karsin dinlemediler, yola gelmediler.
\par 34 Bana ait olan bu tapinaga igrenç putlarini yerlestirerek onu kirlettiler.
\par 35 Ben-Hinnom Vadisi'nde ilah Molek'e* sunu olarak ogullarini, kizlarini ateste kurban etmek için Baal'in tapinma yerlerini kurdular. Böyle igrenç seyler yaparak Yahuda'yi günaha sürüklemelerini ne buyurdum, ne de aklimdan geçirdim.
\par 36 "Siz bu kent için, 'Kiliçla, kitlikla, salgin hastalikla Babil Krali'nin eline veriliyor diyorsunuz. Ama simdi Israil'in Tanrisi RAB diyor ki:
\par 37 Kizginlikla, gazapla, büyük öfkeyle onlari sürdügüm ülkelerden hepsini toplayacagim. Onlari buraya geri getirip güvenlik içinde yasamalarini saglayacagim.
\par 38 Onlar benim halkim olacak, ben de onlarin Tanrisi olacagim.
\par 39 Tek bir yürek, tek bir yasam tarzi verecegim onlara; gerek kendilerinin gerekse çocuklarinin iyiligi için benden hep korksunlar.
\par 40 Onlarla kalici bir antlasma yapacagim: Onlara iyilik etmekten vazgeçmeyecek, benden hiç ayrilmasinlar diye yüreklerine Tanri korkusu salacagim.
\par 41 Onlara iyilik etmekten sevinç duyacagim; gerçekten bütün yüregimle, bütün canimla onlari bu ülkede dikecegim.
\par 42 "RAB diyor ki: Bu halkin basina bütün bu büyük felaketleri nasil getirdiysem, onlara söz verdigim bütün iyilikleri de öyle saglayacagim.
\par 43 Sizlerin, 'Viran olmus, insansiz, hayvansiz, Kildaniler'in eline verilmis dediginiz bu ülkede yine tarlalar satin alinacak.
\par 44 Benyamin bölgesinde, Yerusalim çevresindeki köylerde, Yahuda kentlerinde, daglik bölgenin, Sefela'nin ve Negev'in kentlerinde gümüsle tarlalar satin alinacak, satis belgeleri taniklarin önünde imzalanip mühürlenecek. Çünkü eski gönençlerine kavusturacagim onlari" diyor RAB.

\chapter{33}

\par 1 Yeremya muhafiz avlusunda tutukluyken, RAB ona ikinci kez seslendi:
\par 2 "Dünyayi yaratan, yerini alsin diye ona biçim veren, adi RAB olan söyle diyor:
\par 3 'Bana yakar da seni yanitlayayim; bilmedigin büyük, akil almaz seyleri sana bildireyim.
\par 4 Kusatma rampalarina, kilica karsi siper olsun diye bu kentin yikilmis olan evleriyle Yahuda krallarinin saraylari için Israil'in Tanrisi RAB söyle diyor:
\par 5 'Kildaniler savasmak, evleri öfke ve kizginlikla vurdugum insanlarin cesetleriyle doldurmak üzere gelecekler. O insanlar ki, yaptiklari kötülükler yüzünden bu kentten yüzümü çevirdim.
\par 6 "'Yine de bu kenti iyilestirip sagliga kavusturacagim. Halkina sifa verecek, bol esenlik, güvenlik içinde yasamalarini saglayacagim.
\par 7 Yahuda'yi ve Israil'i eski gönencine kavusturacak, önceden oldugu gibi bina edecegim.
\par 8 Onlari bana karsi isledikleri bütün günahlardan arindiracak, bana karsi isledikleri günahlari da isyanlarini da bagislayacagim.
\par 9 Dünyadaki bütün uluslarin önünde bu kent benim için sevinç, övgü ve onur kaynagi olacak. Bu uluslar Yerusalim halkina yaptigim iyilikleri, sagladigim gönenci duyunca, korkuya kapilip titreyecekler.
\par 10 "RAB söyle diyor: 'Bu kent viran olmus, insansiz, hayvansiz kalmis diyorsunuz. Ne var ki, terk edilmis, insansiz, hayvansiz Yahuda kentlerinde, Yerusalim sokaklarinda sevinç ve nese sesi, gelin güvey sesi, RAB'bin Tapinagi'na sükran sunulari getirenlerin sesi yine duyulacak: 'Her Seye Egemen RAB'be sükredin, Çünkü O iyidir, Sevgisi sonsuza dek kalicidir. Çünkü ülkeyi eski gönencine kavusturacagim diyor RAB.
\par 12 "Her Seye Egemen RAB diyor ki, 'Viran olmus, insansiz, hayvansiz kalmis bu ülkenin bütün kentlerinde çobanlarin sürülerini dinlendirecegi otlaklar olacak yeniden.
\par 13 Daglik bölgede, Sefela, Negev, Benyamin bölgesindeki kentlerde, Yerusalim'in çevresindeki köylerde, Yahuda kentlerinde çoban degneginin altindan geçen sürüler olacak diyor RAB.
\par 14 "'Israil ve Yahuda halkina verdigim güzel sözü yerine getirecegim günler geliyor diyor RAB.
\par 15 "'O günlerde, o zamanda, Davut için dogru bir dal yetistirecegim; Ülkede adil ve dogru olani yapacak.
\par 16 O günlerde Yahuda kurtulacak, Yerusalim güvenlik içinde yasayacak. O, RAB dogrulugumuzdur adiyla anilacak.
\par 17 RAB söyle diyor: 'Israil tahti üzerinde oturan Davut soyunun ardi arkasi kesilmeyecek.
\par 18 Levili kâhinlerden önümde yakmalik sunu* sunacak, tahil sunusu* yakacak, kurban kesecek biri hiç eksik olmayacak."
\par 19 RAB Yeremya'ya söyle seslendi:
\par 20 "RAB diyor ki, 'Eger belirlenmis zamanlarda gece ve gündüz olmasi için gece ve gündüzle yaptigim antlasma bozulabilirse, tahtinda oturan ogullari krallik yapsin diye kulum Davut'la ve bana hizmet eden Levili kâhinlerle yaptigim antlasma da ancak o zaman bozulabilir.
\par 22 Kulum Davut'un soyunu ve bana hizmet eden Levililer'i sayilamaz gök cisimleri kadar, ölçülemez deniz kumu kadar çogaltacagim."
\par 23 RAB Yeremya'ya söyle seslendi:
\par 24 "Bu halkin, 'RAB seçtigi iki aileyi de reddetti dedigini görmüyor musun? Halkimi öyle küçümsüyorlar ki, artik bir ulus saymiyorlar onu.
\par 25 RAB diyor ki, 'Gece ve gündüzle bir antlasma yapip yerin, gögün kurallarini saptamasaydim,
\par 26 Yakup soyuyla kulum Davut'u da reddeder, Davut'un ogullarindan birinin Ibrahim, Ishak, Yakup soyuna krallik etmesini saglamazdim. Ama ben onlari eski gönençlerine kavusturacak, onlara aciyacagim."

\chapter{34}

\par 1 Babil Krali Nebukadnessar'la bütün ordusu, kralligi altindaki bütün uluslarla halklar, Yerusalim ve çevresindeki kentlere karsi savasirken RAB Yeremya'ya söyle seslendi:
\par 2 "Israil'in Tanrisi RAB diyor ki, 'Git, Yahuda Krali Sidkiya'ya RAB söyle diyor de: Bu kenti Babil Krali'nin eline teslim etmek üzereyim, onu atese verecek.
\par 3 Ve sen Sidkiya, onun elinden kaçip kurtulamayacaksin; kesinlikle yakalanacak, onun eline teslim edileceksin. Babil Krali'ni gözünle görecek, onunla yüzyüze konusacaksin. Sonra Babil'e götürüleceksin.
\par 4 "'Ancak, ey Yahuda Krali Sidkiya, RAB'bin sözünü dinle! RAB senin için söyle diyor: Kiliçla ölmeyeceksin,
\par 5 esenlikle öleceksin. Atalarin olan senden önceki krallarin onuruna ates yaktiklari gibi, senin onuruna da ates yakip senin için ah efendimiz diyerek agit tutacaklar. Ben RAB söylüyorum bunu."
\par 6 Peygamber Yeremya bütün bunlari Yerusalim'de Yahuda Krali Sidkiya'ya söyledi.
\par 7 O sirada Babil Krali'nin ordusu Yerusalim'e ve Yahuda'nin henüz ele geçirilmemis kentlerine -Lakis'e, Azeka'ya saldirmaktaydi. Yahuda'da surlu kent olarak yalniz bunlar kalmisti.
\par 8 Kral Sidkiya Yerusalim'deki halkla kölelerin özgürlügünü ilan eden bir antlasma yaptiktan sonra RAB Yeremya'ya seslendi.
\par 9 Bu antlasmaya göre herkes kadin, erkek Ibrani kölelerini özgür birakacak, hiç kimse Yahudi kardesini yaninda köle olarak tutmayacakti.
\par 10 Böylece bu antlasmanin yükümlülügü altina giren bütün önderlerle halk kadin, erkek kölelerini özgür birakarak antlasmaya uydular. Artik kimseyi köle olarak tutmadilar. Antlasmaya uyarak köleleri özgür biraktilar.
\par 11 Ama sonra düsüncelerini degistirerek özgür biraktiklari kadin, erkek köleleri geri alip zorla kölelestirdiler.
\par 12 Bunun üzerine RAB Yeremya'ya söyle seslendi:
\par 13 "Israil'in Tanrisi RAB diyor ki: Atalarinizi Misir'dan, köle olduklari ülkeden çikardigimda onlarla bir antlasma yaptim. Onlara dedim ki,
\par 14 'Size satilip alti yil kölelik eden Ibrani kardeslerinizi yedinci yil özgür birakacaksiniz. Ama atalariniz beni dinlemediler, kulak asmadilar.
\par 15 Sizse sonradan yola gelip gözümde dogru olani yaptiniz: Hepiniz Ibrani kardeslerinizin özgürlügünü ilan ettiniz. Önümde, bana ait olan tapinakta bu dogrultuda bir antlasma yapmistiniz.
\par 16 Ama düsüncenizi degistirerek adima saygisizlik ettiniz. Kendi isteginizle özgür biraktiginiz kadin, erkek kölelerinizi geri alip zorla kölelestirdiniz.
\par 17 "Bu nedenle RAB diyor ki, Ibrani köle kardeslerinizi, yurttaslarinizi özgür birakmayarak beni dinlemediniz. Simdi ben size 'özgürlük -kiliç, kitlik ve salgin hastalikla yok olmaniz için 'özgürlük- ilan edecegim, diyor RAB. Sizi dünyadaki bütün kralliklara dehset verici bir örnek yapacagim.
\par 18 Antlasmami bozan, danayi ikiye ayirip parçalari arasindan geçerek önümde yaptiklari antlasmanin kosullarini yerine getirmeyen bu adamlari -Yahuda ve Yerusalim önderlerini, saray görevlilerini, kâhinleri ve dana parçalarinin arasindan geçen bütün ülke halkini-
\par 20 can düsmanlarinin eline teslim edecegim. Cesetleri yirtici kuslara, yabanil hayvanlara yem olacak.
\par 21 "Yahuda Krali Sidkiya'yla önderlerini de can düsmanlarinin eline, üzerinizden çekilen Babil ordusunun eline teslim edecegim.
\par 22 Buyrugu ben verecegim diyor RAB. Babilliler'i bu kente geri getirecegim. Saldirip kenti ele geçirecek, atese verecekler. Yahuda kentlerini içinde kimsenin yasamayacagi bir viraneye çevirecegim."

\chapter{35}

\par 1 Yahuda Krali Yosiya oglu Yehoyakim döneminde RAB Yeremya'ya söyle seslendi:
\par 2 "Rekavlilar'in evine gidip onlarla konus. Onlari RAB'bin Tapinagi'nin odalarindan birine götürüp sarap içir."
\par 3 Bunun üzerine Havassinya oglu Yirmeya oglu Yaazanya'yi, kardeslerini, bütün çocuklarini ve Rekav ailesinin öbür üyelerini yanima alip
\par 4 Tanri adami Yigdalya oglu Hanan'in ogullarinin RAB'bin Tapinagi'ndaki odasina götürdüm. Bu oda önderlerin odasinin bitisiginde, kapi görevlisi Sallum oglu Maaseya'nin odasinin üstündeydi.
\par 5 Rekav ailesinin üyelerinin önüne sarap dolu testiler, kâseler koyarak, "Buyrun, sarap için" dedim.
\par 6 Ne var ki, "Biz sarap içmeyiz" diye karsilik verdiler, "Çünkü atamiz Rekav oglu Yehonadav bize su buyrugu verdi: 'Siz de soyunuzdan gelenler de asla sarap içmeyeceksiniz!
\par 7 Ayrica ev yapmayacak, tohum ekmeyecek, bag dikmeyeceksiniz. Böyle seyler edinmeyecek, ömür boyu çadirlarda yasayacaksiniz. Öyle ki, göç ettiginiz topraklarda uzun süre yasayasiniz.
\par 8 Atamiz Rekav oglu Yehonadav'in bize buyurdugu her seyi yaptik. Kendimiz de karilarimiz, ogullarimiz, kizlarimiz da hiç sarap içmedik.
\par 9 Içinde oturmak için evler yapmadik, baglar, tarlalar, ekinler edinmedik.
\par 10 Çadirlarda yasadik; atamiz Yehonadav ne buyurduysa hepsini yaptik.
\par 11 Ama Babil Krali Nebukadnessar bu ülkeye saldirinca, 'Haydi, Kildan* ve Aram ordusundan kaçmak için Yerusalim'e gidelim dedik. Bunun için Yerusalim'de kaldik."
\par 12 Bundan sonra RAB Yeremya'ya söyle seslendi:
\par 13 "Israil'in Tanrisi, Her Seye Egemen RAB söyle diyor: 'Git, Yahuda halkina ve Yerusalim'de yasayanlara sunlari söyle: Sözlerimi dinleyerek hiç ders almayacak misiniz, diyor RAB.
\par 14 Rekav oglu Yehonadav, soyuna sarap içmemelerini buyurdu; buyruguna uyuldu. Bugüne dek sarap içmediler. Çünkü atalarinin buyruguna uydular. Bense size defalarca seslendigim halde beni dinlemediniz.
\par 15 Defalarca size kullarim peygamberleri gönderdim. Kötü yolunuzdan dönmeniz, davranislarinizi düzeltmeniz, baska ilahlarin ardinca gidip onlara tapinmamaniz için hepinizi uyardilar. Ancak o zaman size ve atalariniza verdigim toprakta yasayacaksiniz. Ama kulak verip beni dinlemediniz.
\par 16 Rekav oglu Yehonadav'in soyu atalarinin verdigi buyrugu tuttu, ama bu halk beni dinlemedi.
\par 17 "Bu yüzden Israil'in Tanrisi, Her Seye Egemen RAB diyor ki, 'Iste, Yahuda ve Yerusalim'de yasayan herkesin basina sözünü ettigim her felaketi getirmek üzereyim. Çünkü onlari uyardim, ama dinlemediler; onlari çagirdim, ama yanit vermediler."
\par 18 Yeremya Rekav ailesine söyle dedi: "Israil'in Tanrisi, Her Seye Egemen RAB diyor ki, 'Ataniz Yehonadav'in buyruguna uydunuz, onun bütün uyarilarini dikkate aldiniz, size buyurdugu her seyi yaptiniz.
\par 19 Bunun için Israil'in Tanrisi, Her Seye Egemen RAB diyor ki, 'Rekav oglu Yehonadav'in soyundan önümde hizmet edecek olanlar hiçbir zaman eksilmeyecek."

\chapter{36}

\par 1 Yahuda Krali Yosiya oglu Yehoyakim'in kralliginin dördüncü yilinda RAB Yeremya'ya söyle seslendi:
\par 2 "Bir tomar al, Yosiya'nin döneminden bu yana Israil, Yahuda ve öteki uluslarla ilgili sana söyledigim her seyi yaz.
\par 3 Belki Yahuda halki basina getirmeyi tasarladigim bütün felaketleri duyar da kötü yolundan döner; ben de suçlarini, günahlarini bagislarim."
\par 4 Yeremya Neriya oglu Baruk'u çagirip RAB'bin kendisine söyledigi bütün sözleri tomara yazdirdi.
\par 5 Sonra Baruk'a su buyrugu verdi: "Ben tutukluyum, RAB'bin Tapinagi'na gidemem.
\par 6 Sen oruç* günü RAB'bin Tapinagi'na git. Oradaki halka sana yazdirdigim RAB'bin sözlerini tomardan oku. Yahuda kentlerinden gelen halka da oku.
\par 7 Belki RAB'be yalvarir, kötü yollarindan dönerler. Çünkü RAB'bin bu halka karsi sözünü ettigi öfke ve kizginligi büyüktür."
\par 8 Neriya oglu Baruk, Peygamber Yeremya'nin buyurdugu her seyi yapti. RAB'bin tomarda yazili sözlerini tapinakta okudu.
\par 9 Yahuda Krali Yosiya oglu Yehoyakim'in kralliginin besinci yilinin dokuzuncu ayi*, Yerusalim'de yasayan bütün halk ve Yahuda kentlerinden Yerusalim'e gelen herkes için RAB'bin önünde oruç ilan edildi.
\par 10 Safan oglu Yazman Gemarya'nin yukari avluda, tapinagin Yeni Kapisi'nin girisindeki odasinda Baruk, Yeremya'nin sözlerini tomardan RAB'bin Tapinagi'ndaki bütün halka okudu.
\par 11 Safan oglu Gemarya oglu Mikaya tomardan okunan RAB'bin sözlerini duyunca,
\par 12 asagiya inip saraydaki yazman odasina gitti. Bütün önderler orada toplanti halindeydi: Yazman Elisama, Semaya oglu Delaya, Akbor oglu Elnatan, Safan oglu Gemarya, Hananya oglu Sidkiya ve öbürleri.
\par 13 Mikaya Baruk'un halka okudugu tomardan duydugu her seyi onlara iletti.
\par 14 Bunun üzerine bütün önderler Kusi oglu Selemya oglu Netanya oglu Yehudi'yi Baruk'a gönderdiler. Yehudi Baruk'a, "Halka okudugun tomari al, gel" dedi. Neriya oglu Baruk tomari alip yanlarina gitti.
\par 15 Önderler, "Lütfen otur, bize de oku" dediler. Baruk da tomari onlara okudu.
\par 16 Bütün bu sözleri duyar duymaz, korkuyla birbirlerine bakip Baruk'a, "Bütün bu sözleri kesinlikle krala bildirmemiz gerek" dediler.
\par 17 Sonra Baruk'a, "Söyle bize, bütün bunlari nasil yazdin? Yeremya mi yazdirdi sana?" diye sordular.
\par 18 Baruk, "Evet" diye yanitladi, "Bütün bu sözleri o söyledi. Ben de hepsini mürekkeple tomara yazdim."
\par 19 Bunun üzerine Baruk'a, "Git, sen de Yeremya da gizlenin. Nerede oldugunuzu kimse bilmesin" dediler.
\par 20 Tomari Yazman Elisama'nin odasinda birakip avluda bulunan kralin yanina gittiler. Ona duyduklari her sözü ilettiler.
\par 21 Kral tomari alip getirmesi için Yehudi'yi gönderdi. Yehudi tomari Yazman Elisama'nin odasindan getirdi. Krala ve yanindaki önderlere okudu.
\par 22 Dokuzuncu aydi, kral kislik sarayindaydi. Önünde mangalda ates yaniyordu.
\par 23 Yehudi tomardan üç dört sütun okur okumaz kral yazman çakisiyla onlari kesip mangaldaki atese atti. Tomarin tümü yanip bitene dek bu isi sürdürdü.
\par 24 Tomardaki sözleri duyan kralla görevlileri ne korktular ne de giysilerini yirttilar.
\par 25 Elnatan, Delaya ve Gemarya tomari yakmamasi için yalvardilarsa da kral onlari dinlemedi.
\par 26 Yazman Baruk'u ve Peygamber Yeremya'yi tutuklamalari için oglu Yerahmeel'e, Azriel oglu Seraya'ya ve Avdeel oglu Selemya'ya buyruk verdi. Oysa RAB onlari gizlemisti.
\par 27 Kral, Yeremya'nin Baruk'a yazdirdigi sözleri içeren tomari yaktiktan sonra, RAB Yeremya'ya söyle seslendi:
\par 28 "Baska bir tomar al, Yahuda Krali Yehoyakim'in yaktigi ilk tomardaki bütün sözleri yazdir.
\par 29 Yahuda Krali Yehoyakim için de ki, 'RAB söyle diyor: Babil Krali'nin kesinlikle gelip bu ülkeyi viraneye çevirecegini, içindeki insani da hayvani da yok edecegini neden tomara yazdin diye sorup tomari yaktin.
\par 30 Bu yüzden RAB Yahuda Krali Yehoyakim için söyle diyor: Davut'un tahtinda oturan kimsesi olmayacak; ölüsü gündüzün sicaga, gece ayaza birakilacak.
\par 31 Isledikleri suçlar yüzünden kendisini de çocuklariyla görevlilerini de cezalandiracagim. Onlarin, Yerusalim'de yasayanlarin ve Yahuda halkinin basina sözünü ettigim bütün felaketleri getirecegim. Çünkü beni dinlemediler."
\par 32 Bunun üzerine Yeremya baska bir tomar alip Neriya oglu Yazman Baruk'a verdi. Yahuda Krali Yehoyakim'in atese atip yaktigi tomardaki bütün sözleri Baruk Yeremya'nin agzindan tomara yazdi. Bu sözlere, benzer birçok söz daha eklendi.

\chapter{37}

\par 1 Babil Krali Nebukadnessar'in Yahuda'ya atadigi Yosiya oglu Sidkiya, Yehoyakim oglu Yehoyakin'in yerine kral oldu.
\par 2 Ama kendisi de görevlileriyle ülke halki da RAB'bin Peygamber Yeremya araciligiyla söyledigi sözleri dikkate almadilar.
\par 3 Kral Sidkiya, Selemya oglu Yehukal'la Maaseya oglu Kâhin Sefanya'yi su haberle Peygamber Yeremya'ya gönderdi: "Lütfen bizim için Tanrimiz RAB'be yalvar."
\par 4 O sirada Yeremya daha cezaevine konmamisti, halk arasinda dolasiyordu.
\par 5 Firavunun ordusu Misir'dan çikmisti. Yerusalim'i kusatma altinda tutan Kildaniler* bu haberi duyunca Yerusalim'den çekildiler.
\par 6 Derken RAB Peygamber Yeremya'ya söyle seslendi:
\par 7 "Israil'in Tanrisi RAB diyor ki: Danismak için sizi bana gönderen Yahuda Krali'na söyle deyin: 'Size yardim etmek için Misir'dan çikip gelen firavunun ordusu ülkesine dönecek.
\par 8 Kildaniler de dönecek; bu kentle savasip onu ele geçirecek, atese verecekler.
\par 9 "RAB diyor ki, 'Kildaniler üzerimizden çekilip gidecek diyerek kendinizi aldatmayin. Çünkü çekilip gitmeyecekler!
\par 10 Sizinle savasan bütün Babil ordusunu bozguna ugratsaniz, çadirlarinda yalniz yaralilar kalsa bile, bunlar çadirlardan çikip bu kenti atese verecekler."
\par 11 Firavunun ordusu yüzünden Babil ordusu Yerusalim'den çekilince,
\par 12 Peygamber Yeremya Benyamin topraklarindaki halkin arasinda payina düsen mirasi almak üzere Yerusalim'den gitmek istedi. Benyamin Kapisi'na vardiginda, Hananya oglu Selemya oglu bekçibasi Yiriya, "Sen Kildaniler'in tarafina geçiyorsun!" diyerek onu tutukladi.
\par 14 Yeremya, "Yalan!" dedi, "Ben Kildaniler'in tarafina geçmiyorum." Ama Yiriya onu dinlemedi. Yeremya'yi tutuklayip önderlere götürdü.
\par 15 Yeremya'ya öfkelenen önderler onu dövdürtüp cezaevine çevirdikleri Yazman Yonatan'in evine kapattilar.
\par 16 Böylece bodrumda bir hücreye kapatilan Yeremya uzun süre orada kaldi.
\par 17 Sonra Kral Sidkiya Yeremya'yi sarayina getirtti. Orada kendisine gizlice, "RAB'den bir söz var mi?" diye sordu. "Evet" diye yanitladi Yeremya, "Babil Krali'nin eline verileceksin."
\par 18 Sonra Kral Sidkiya'ya söyle dedi: "Sana, görevlilerine ve bu halka karsi ne günah isledim ki beni cezaevine kapattiniz?
\par 19 'Babil Krali sana da bu ülkeye de saldirmayacak diyen peygamberlerin hani nerede?
\par 20 Simdi lütfen beni dinle, ey efendim kral! Lütfen dilegimi kabul et. Beni Yazman Yonatan'in evine geri gönderme. Orada ölmek istemiyorum."
\par 21 Bunun üzerine Kral Sidkiya Yeremya'nin muhafiz avlusuna kapatilmasini, kentteki ekmek bitene dek her gün firincilar sokagindan kendisine bir ekmek verilmesini buyurdu. Böylece Yeremya muhafiz avlusunda kaldi.

\chapter{38}

\par 1 Mattan oglu Sefatya, Pashur oglu Gedalya, Selemya oglu Yehukal ve Malkiya oglu Pashur Yeremya'nin halka söyledigi su sözleri duydular:
\par 2 "RAB diyor ki, 'Bu kentte kalan kiliçtan, kitliktan, salgindan ölecek. Kildaniler'e* gidense sag kalacak, canini kurtarip yasayacak.
\par 3 RAB diyor ki, 'Bu kent kesinlikle Babil Krali'nin ordusuna teslim edilecek, Babil Krali onu ele geçirecek."
\par 4 Önderler krala, "Bu adam öldürülmeli" dediler, "Çünkü söyledigi bu sözlerle kentte kalan askerlerin ve halkin cesaretini kiriyor. Bu adam halkin yararini degil, zararini istiyor."
\par 5 Kral Sidkiya, "Iste o sizin elinizde" diye yanitladi, "Kral size engel olamaz ki."
\par 6 Böylece Yeremya'yi alip kralin oglu Malkiya'nin muhafiz avlusundaki sarnicina halatlarla sarkitarak indirdiler. Sarniçta su yoktu, yalniz çamur vardi. Yeremya çamura batti.
\par 7 Sarayda görevli hadim Kûslu* Ebet-Melek Yeremya'nin sarnica atildigini duydu. Kral Benyamin Kapisi'nda otururken,
\par 8 Ebet-Melek saraydan çikip kralin yanina gitti ve ona söyle dedi:
\par 9 "Efendim kral, bu adamlarin Peygamber Yeremya'ya yaptiklari kötüdür. Onu sarnica attilar, orada açliktan ölecek. Çünkü kentte ekmek kalmadi."
\par 10 Bunun üzerine kral, "Buradan yanina üç adam al, Peygamber Yeremya'yi ölmeden sarniçtan çikarin" diye ona buyruk verdi.
\par 11 Ebet-Melek yanina adamlari alarak saray hazinesinin alt odasina gitti. Oradan eski bezler, yirtik pirtik giysiler alip halatlarla sarnica, Yeremya'ya sarkitti.
\par 12 Sonra Yeremya'ya, "Bu eski bezleri, yirtik giysileri halatlarla baglayip koltuklarinin altina geçir" diye seslendi. Yeremya söyleneni yapti.
\par 13 Onu halatlarla çekip sarniçtan çikardilar. Yeremya muhafiz avlusunda kaldi.
\par 14 Kral Sidkiya Peygamber Yeremya'yi RAB'bin Tapinagi'nin üçüncü girisine getirterek, "Sana bir sey soracagim" dedi, "Benden bir sey gizleme."
\par 15 Yeremya, "Sana bir sey bildirirsem, beni öldürmeyecek misin?" diye karsilik verdi, "Üstelik ögüt versem bile beni dinlemeyeceksin."
\par 16 Kral Sidkiya, "Bize yasam veren RAB'bin varligi hakki için seni öldürmeyecegim, caninin pesinde olan bu adamlarin eline seni teslim etmeyecegim" diyerek gizlice ant içti.
\par 17 Bunun üzerine Yeremya Sidkiya'ya su karsiligi verdi: "Israil'in Tanrisi, Her Seye Egemen RAB Tanri diyor ki, 'Babil Krali'nin komutanlarina teslim olursan, canin bagislanacak, bu kent de atese verilmeyecek. Sen de ailen de sag kalacaksiniz.
\par 18 Ama Babil Krali'nin komutanlarina teslim olmazsan, kent Kildaniler'e* teslim edilecek, onu atese verecekler. Sen de onlardan kaçip kurtulamayacaksin."
\par 19 Kral Sidkiya, "Kildaniler'in tarafina geçen Yahudiler'den korkuyorum" dedi, "Kildaniler beni onlarin eline verebilir, onlar da bana kötü davranirlar."
\par 20 "Vermezler" diye yanitladi Yeremya, "Lütfen sana aktardigim RAB'bin sözünü isit. O zaman sag kalir, iyilik görürsün.
\par 21 Ama teslim olmak istemezsen, RAB bana sunu açikladi:
\par 22 Yahuda Krali'nin sarayinda kalan bütün kadinlar Babil Krali'nin komutanlarina çikarilacak. O kadinlar sana, "'Güvendigin insanlar Seni aldatip yenilgiye ugratti; Çamura batti ayaklarin, Güvendigin insanlar seni birakip gitti diyecekler.
\par 23 "Bütün karilarin, çocuklarin Kildaniler'e teslim edilecek. Sen de onlardan kaçip kurtulamayacak, Babil Krali'nin eliyle yakalanacaksin. Bu kent atese verilecek."
\par 24 Sidkiya, "Ölmek istemiyorsan, konustuklarimizi kimse duymasin" dedi,
\par 25 "Görevliler seninle konustugumu duyup da gelir, 'Krala ne söyledin, kral sana ne dedi, açikla bize, bizden gizleme! Yoksa seni öldürürüz derlerse,
\par 26 'Beni Yonatan'in evine geri gönderme, yoksa orada ölürüm diye krala yalvardim dersin."
\par 27 Bütün görevliler gelip Yeremya'yi sorguya çektiler. Yeremya kralin kendisine söylemesini buyurdugu her seyi onlara anlatti. Sorguyu biraktilar. Çünkü kralla yaptigi konusma duyulmamisti.
\par 28 Yeremya Yerusalim'in ele geçirildigi güne dek muhafiz avlusunda kaldi.

\chapter{39}

\par 1 Yahuda Krali Sidkiya'nin dokuzuncu yilinin onuncu ayinda* Babil Krali Nebukadnessar bütün ordusuyla Yerusalim önlerine gelerek kenti kusatti.
\par 2 Sidkiya'nin kralliginin onbirinci yilinda, dördüncü ayin dokuzuncu günü kent surlarinda gedik açildi.
\par 3 Yerusalim ele geçirilince Babil Krali'nin bütün komutanlari -Samgarli Nergal-Sareser, askeri danisman Nebo- Sarsekim, bas görevli Nergal-Sareser ve bütün öteki görevliler- içeri girip Orta Kapi'da oturdular.
\par 4 Yahuda Krali Sidkiya'yla askerler onlari görünce kaçtilar. Gece kral bahçesinin yolundan iki duvarin arasindaki kapidan kaçarak Arava yoluna çiktilar.
\par 5 Ama artlarina düsen Kildani* ordusu Eriha ovalarinda Sidkiya'ya yetisti, onu yakalayip Hama topraklarinda, Rivla'da Babil Krali Nebukadnessar'in huzuruna çikardilar. Nebukadnessar onun hakkinda karar verdi:
\par 6 Rivla'da Sidkiya'nin gözü önünde ogullarini, sonra da bütün Yahuda ileri gelenlerini öldürttü.
\par 7 Sidkiya'nin gözlerini oydu, zincire vurup Babil'e götürdü.
\par 8 Kildaniler sarayla halkin evlerini atese verdiler, Yerusalim surlarini yiktilar.
\par 9 Komutan Nebuzaradan kentte sag kalanlari, kendi safina geçen kaçaklari ve geri kalan halki Babil'e sürgün etti.
\par 10 Ancak hiçbir seyi olmayan bazi yoksullari Yahuda'da birakti, onlara bag ve tarla verdi.
\par 11 Babil Krali Nebukadnessar, muhafiz birligi komutani Nebuzaradan araciligiyla Yeremya'yla ilgili su buyrugu verdi:
\par 12 "Onu sorumlulugun altina al, ona iyi bak, hiç zarar verme, senden ne dilerse yap."
\par 13 Bunun üzerine muhafiz birligi komutani Nebuzaradan, askeri danisman Nebusazban, bas görevli Nergal-Sareser ve Babil Krali'nin öbür görevlileri
\par 14 adam gönderip Yeremya'yi muhafiz avlusundan getirttiler. Evine geri götürmesi için Safan oglu Ahikam oglu Gedalya'nin koruyuculuguna verdiler. Böylece Yeremya halki arasinda yasamini sürdürdü.
\par 15 Yeremya daha muhafiz avlusunda tutukluyken RAB ona söyle seslenmisti:
\par 16 "Git, Kûslu* Ebet-Melek'e de ki, 'Israil'in Tanrisi, Her Seye Egemen RAB söyle diyor: Bu kent üzerine yarar degil, zarar verecek sözlerimi yerine getirmek üzereyim. O gün olanlari sen de göreceksin.
\par 17 Ama o gün seni kurtaracagim diyor RAB. Korktugun adamlarin eline teslim edilmeyeceksin.
\par 18 Seni kesinlikle kurtaracagim, kiliçla öldürülmeyeceksin. Hiç degilse canini kurtarmis olacaksin. Çünkü bana güvendin, diyor RAB."

\chapter{40}

\par 1 Muhafiz birligi komutani Nebuzaradan Yeremya'yi Rama'da saliverdikten sonra RAB Yeremya'ya seslendi. Nebuzaradan onu Babil'e sürülen Yerusalim ve Yahuda halkiyla birlikte zincire vurulmus olarak Rama'ya götürmüstü.
\par 2 Muhafiz birligi komutani Yeremya'yi yanina çagirtip, "Tanrin RAB buraya karsi bu felaketi belirledi" dedi,
\par 3 "Simdi dedigini yapti, yapacagini söyledigi her seyi yerine getirdi. Çünkü RAB'be karsi günah islediniz, O'nun sözünü dinlemediniz. Bütün bunlar bu yüzden basiniza geldi.
\par 4 Iste ellerindeki zincirleri çözüyorum. Benimle Babil'e gelmeyi yeglersen gel, sana iyi bakarim; eger benimle Babil'e gelmek istemezsen de sorun yok. Bak, bütün ülke önünde! Iyi ve dogru bildigin yere git.
\par 5 Ama burada kalirsan Babil Krali'nin Yahuda kentlerine vali atadigi Safan oglu Ahikam oglu Gedalya'nin yanina dönüp onunla birlikte halkin arasinda yasa ya da istedigin yere git." Muhafiz birligi komutani Yeremya'ya yiyecek ve armagan verip yoluna gönderdi.
\par 6 Yeremya Mispa'ya, Ahikam oglu Gedalya'nin yanina gitti. Onunla ve ülkede kalan halkla birlikte orada yasamaya basladi.
\par 7 Kirdaki ordu komutanlariyla adamlari, Babil Krali'nin Ahikam oglu Gedalya'yi ülkeye vali atadigini, Babil'e sürülmemis yoksul kadin, erkek ve çocuklari ona emanet ettigini duyunca,
\par 8 Mispa'ya, Gedalya'nin yanina geldiler. Gelenler Netanya oglu Ismail, Kareah'in ogullari Yohanan ve Yonatan, Tanhumet oglu Seraya, Netofali Efay'in ogullari, Maakali oglu Yaazanya ve adamlariydi.
\par 9 Safan oglu Ahikam oglu Gedalya onlara ve adamlarina ant içerek, "Kildaniler'e* kulluk etmekten korkmayin" dedi, "Ülkeye yerlesip Babil Krali'na hizmet edin. Böylesi sizin için daha iyi olur.
\par 10 Bana gelince, Mispa'da kalacagim, gelecek Kildaniler'in önünde sizi temsil edecegim. Siz sarap, yaz meyveleri, zeytinyagi toplayip kaplarinizda depolayin ve aldiginiz kentlerde yasayin."
\par 11 Moav, Ammon, Edom ve öbür ülkelerde yasayan Yahudiler'in hepsi, Babil Krali'nin Yahuda'da bir kesim halki sag biraktigini ve Safan oglu Ahikam oglu Gedalya'yi onlara vali atadigini duyunca,
\par 12 sürülmüs olduklari ülkelerden geri dönüp Yahuda'ya, Mispa'da bulunan Gedalya'nin yanina geldiler. Sonra bol bol sarap ve yaz meyvesi topladilar.
\par 13 Kareah oglu Yohanan'la kirdaki bütün ordu komutanlari da Mispa'da bulunan Gedalya'nin yanina geldiler.
\par 14 Ona, "Ammon Krali Baalis'in seni öldürmek için Netanya oglu Ismail'i gönderdigini bilmiyor musun?" dediler. Gelgelelim Ahikam oglu Gedalya onlara inanmadi.
\par 15 Kareah oglu Yohanan Mispa'da Gedalya'ya gizlice, "Izin ver de gidip Netanya oglu Ismail'i öldüreyim" dedi, "Kimse bilmeyecek! Neden seni öldürsün de çevrende toplanan bütün Yahudiler dagilsin, Yahuda'da sag kalmis olanlar yok olsun?"
\par 16 Ama Ahikam oglu Gedalya, "Böyle birsey yapma! Ismail'le ilgili söylediklerin yalan" dedi.

\chapter{41}

\par 1 O yilin yedinci ayinda* kral soyundan ve kralin bas görevlilerinden Elisama oglu Netanya oglu Ismail, on adamiyla birlikte Mispa'ya, Ahikam oglu Gedalya'nin yanina gitti. Orada, Mispa'da birlikte yemek yerlerken,
\par 2 Netanya oglu Ismail'le yanindaki on adam ayaga kalkip Babil Krali'nin ülkeye vali atadigi Safan oglu Ahikam oglu Gedalya'yi kiliçla öldürdüler.
\par 3 Ismail Mispa'da Gedalya'yla birlikte olan bütün Yahudiler'i ve oradaki Kildan* askerlerini de öldürdü.
\par 4 Gedalya öldürüldükten bir gün sonra, ölüm haberi duyulmadan önce
\par 5 Sekem'den, Silo'dan, Samiriye'den sakallarini tiras etmis, giysilerini yirtmis, bedenlerinde yaralar açmis seksen adam geldi. RAB'bin Tapinagi'nda sunmak için yanlarinda tahil, günnük getirmislerdi.
\par 6 Netanya oglu Ismail Mispa'dan aglaya aglaya onlari karsilamaya çikti. Onlari görünce, "Ahikam oglu Gedalya'ya gelin" dedi.
\par 7 Kente girince, Netanya oglu Ismail ve yanindakiler onlari öldürüp bir sarnica attilar.
\par 8 Ancak onlardan on kisi Ismail'e, "Bizi öldürme!" dediler, "Tarlada sakli bugdayimiz, arpamiz, zeytinyagimiz ve balimiz var." Böylece Ismail vazgeçip onlari öbürleriyle birlikte öldürmedi.
\par 9 Ismail'in öldürdügü adamlarin bedenlerini atmis oldugu sarniç büyüktü. Kral Asa, bu sarnici Israil Krali Baasa'dan korunmak için kazmisti. Netanya oglu Ismail orayi ölülerle doldurdu.
\par 10 Ismail, muhafiz birligi komutani Nebuzaradan'in Ahikam oglu Gedalya'nin sorumluluguna biraktigi Mispa'daki bütün halki ve kral kizlarini tutsak aldi. Netanya oglu Ismail tümünü tutsak alip Ammonlular'a siginmak üzere yola çikti.
\par 11 Kareah oglu Yohanan ve yanindaki ordu komutanlari, Netanya oglu Ismail'in isledigi cinayetleri duyunca,
\par 12 bütün adamlarini alip Netanya oglu Ismail'le savasmaya gittiler. Givon'daki büyük havuzun yakininda ona yetistiler.
\par 13 Ismail'in yanindaki adamlar, Kareah oglu Yohanan ve yanindaki ordu komutanlarini görünce sevindiler.
\par 14 Ismail'in Mispa'dan tutsak olarak götürdügü herkes geri dönüp Kareah oglu Yohanan'a katildi.
\par 15 Netanya oglu Ismail ve sekiz adamiysa Yohanan'dan kaçip Ammonlular'a sigindilar.
\par 16 Kareah oglu Yohanan'la yanindaki ordu komutanlari sag kalanlarin hepsini -Ahikam oglu Gedalya'yi öldüren Netanya oglu Ismail'den kurtarip Givon'dan geri getirdigi yigit askerleri, kadinlari, çocuklari, saray görevlilerini- Mispa'dan alip götürdüler.
\par 17 Kildaniler'den* kaçmak için Misir'a dogru yola çiktilar. Beytlehem yakininda, Gerut-Kimham'da durdular. Kildaniler'den korkuyorlardi. Çünkü Netanya oglu Ismail, Babil Krali'nin ülkeye vali atadigi Ahikam oglu Gedalya'yi öldürmüstü.

\chapter{42}

\par 1 Ordu komutanlari, Kareah oglu Yohanan, Hosaya oglu Azarya ve küçük büyük bütün halk yaklasip
\par 2 Peygamber Yeremya'ya söyle dediler: "Lütfen dilegimizi kabul et! Bizim için, bütün sag kalan bu halk için Tanrin RAB'be yakar. Çünkü bir zamanlar sayica çok olan bizler gördügün gibi simdi azinlikta kaldik.
\par 3 Tanrin RAB nereye gidecegimizi, ne yapacagimizi bize bildirsin."
\par 4 Peygamber Yeremya, "Olur" dedi, "Isteginiz uyarinca Tanriniz RAB'be yakaracagim. RAB bana ne yanit verirse, bir sey saklamadan size bildirecegim."
\par 5 Bunun üzerine, "Tanrin RAB'bin senin araciliginla bize bildirecegi her sözü yerine getirmezsek, RAB aramizda gerçek ve güvenilir tanik olsun" dediler,
\par 6 "Seni kendisine gönderdigimiz Tanrimiz RAB'bin sözünü begensek de begenmesek de dinleyecegiz ki, üzerimize iyilik gelsin. Evet, Tanrimiz RAB'bin sözünü dinleyecegiz."
\par 7 On gün sonra RAB Yeremya'ya seslendi.
\par 8 Yeremya, Kareah oglu Yohanan'la yanindaki ordu komutanlarini ve küçük büyük bütün halki çagirdi.
\par 9 Onlara söyle dedi: "Dileginizi önüne sunmam için beni kendisine gönderdiginiz Israil'in Tanrisi RAB diyor ki,
\par 10 'Bu ülkede kalirsaniz, sizi bina ederim, yikmam; dikerim, sökmem. Çünkü basiniza getirdigim felakete üzülüyorum.
\par 11 Korktugunuz Babil Krali'ndan artik korkmayin, ondan korkmayin diyor RAB. Çünkü ben sizinleyim, sizi kurtaracak, onun elinden özgür kilacagim.
\par 12 Size sevecenlik gösterecegim. Söyle ki, Babil Krali size aciyacak, sizi topraklariniza geri gönderecek.
\par 13 "Ama, 'Bu ülkede kalmayacagiz der, Tanriniz RAB'bin sözünü dinlemezseniz,
\par 14 'Savas görmeyecegimiz, boru sesi duymayacagimiz, açlik çekmeyecegimiz Misir'a gidip orada yasayacagiz derseniz,
\par 15 RAB'bin sözünü dinleyin, ey Yahuda'dan sag kalanlar! Israil'in Tanrisi, Her Seye Egemen RAB söyle diyor: 'Eger Misir'a gidip orada yerlesmeye kesin kararliysaniz,
\par 16 korktugunuz kiliç size orada yetisecek, tasalandiginiz kitlik Misir'da yakaniza yapisacak, orada öleceksiniz.
\par 17 Yerlesmek üzere Misir'a gitmeye kararli olan herkes kiliçtan, kitliktan, salgin hastaliktan ölecek. Baslarina getirecegim felaketten kurtulup sag kalan olmayacak.
\par 18 "Israil'in Tanrisi, Her Seye Egemen RAB diyor ki, 'Öfkem, kizginligim Yerusalim'de yasayanlarin üzerine döküldügü gibi, siz Misir'a gidenlerin üzerine de dökülecek. Siz lanetlik, dehset konusu olacak, asagilanacak, yerileceksiniz. Burayi bir daha görmeyeceksiniz.
\par 19 Ey Yahuda'dan sag kalanlar, RAB size, 'Misir'a gitmeyin! diye buyurmustur. Bunu iyi bilin. Bugün sizi uyariyorum:
\par 20 Beni Tanriniz RAB'be gönderip, 'Bizim için Tanrimiz RAB'be yakar. O'nun bize söyleyecegi her seyi bildir, yapacagiz demekle kendinizi aldatiyorsunuz!
\par 21 Bugün size bildirdim, ama Tanriniz RAB'bin benim araciligimla size ilettigi sözlerin hiçbirini dinlemediniz.
\par 22 Simdi iyi bilin ki, yerlesmek üzere gitmeye can attiginiz yerde kiliçtan, kitliktan, salgin hastaliktan öleceksiniz."

\chapter{43}

\par 1 Yeremya Tanrilari RAB'bin bütün bu sözlerini -Tanrilari RAB'bin onun araciligiyla kendilerine ilettigi her seyi- halka bildirmeyi bitirince
\par 2 Hosaya oglu Azarya, Kareah oglu Yohanan ve bütün küstah adamlar ona, "Yalan söylüyorsun!" dediler, "Tanrimiz RAB, 'Yerlesmek üzere Misir'a gitmeyin demek için göndermedi seni bize.
\par 3 Bizi öldürsünler, Babil'e sürsünler diye Kildaniler'in* eline teslim etmek için Neriya oglu Baruk seni bize karsi kiskirtiyor."
\par 4 Böylece Kareah oglu Yohanan, bütün ordu komutanlari ve halk RAB'bin Yahuda'da kalmalarina iliskin buyruguna karsi geldiler.
\par 5 Kareah oglu Yohanan'la bütün ordu komutanlari, sürüldükleri uluslardan yerlesmek üzere Yahuda'ya geri dönen Yahuda halkini alip götürdüler.
\par 6 Muhafiz birligi komutani Nebuzaradan'in Safan oglu Ahikam oglu Gedalya'nin sorumluluguna birakmis oldugu bütün kadinlari, erkekleri, çocuklari, kral kizlarini da götürdüler. Peygamber Yeremya'yla Neriya oglu Baruk'u da alip
\par 7 RAB'bin sözünü dinlemeyerek Misir'a gittiler. Tahpanhes'e vardilar.
\par 8 Tahpanhes'te RAB Yeremya'ya söyle seslendi:
\par 9 "Yahudiler'in gözü önünde eline büyük taslar al, Tahpanhes'te firavun sarayinin girisindeki tugla kaldirimin harcina göm.
\par 10 Onlara de ki, 'Israil'in Tanrisi, Her Seye Egemen RAB söyle diyor: Iste kulum Babil Krali Nebukadnessar'i buraya getirtip tahtini harca gömdügüm bu taslarin üzerine kuracagim. Nebukadnessar otagini bu taslarin üzerine kuracak.
\par 11 Gelip Misir'i bozguna ugratacak. Ölüm için ayrilanlar ölüme, Sürgün için ayrilanlar sürgüne, Kiliç için ayrilanlar kilica gidecek.
\par 12 Misir ilahlarinin tapinaklarini atese verip yakacak, ilahlari alip götürecek. Çoban giysisiyle kendisini nasil örterse, o da Misir'i öyle örtecek. Sonra oradan sag salim çikacak.
\par 13 Misir'daki Günes Tapinagi'nin dikili taslarini kiracak, Misir ilahlarinin tapinaklarini atese verecek."

\chapter{44}

\par 1 Misir'in Migdol, Tahpanhes, Nof kentlerinde ve Patros bölgesinde yasayan Yahudiler'e iliskin RAB Yeremya'ya söyle seslendi:
\par 2 "Israil'in Tanrisi, Her Seye Egemen RAB diyor ki, Yerusalim ve Yahuda kentlerine getirdigim bütün felaketleri gördünüz. Iste yaptiklari kötülük yüzünden kentler bugün yikik; içlerinde oturan yok. Sizin de kendilerinin ve atalarinin da önceden tanimadiginiz baska ilahlara buhur yakip taparak beni öfkelendirdiler.
\par 4 Peygamber kullarimi defalarca gönderip, 'Nefret ettigim bu igrençlikleri yapmayin! diyerek onlari uyardim.
\par 5 Ama dinlemediler, kulak asmadilar. Kötülüklerinden dönmediler, baska ilahlara buhur yakmaktan vazgeçmediler.
\par 6 Bu yüzden kizgin öfkemi döktüm; Yahuda kentlerine, Yerusalim sokaklarina karsi öfkem giderek siddetlendi. Onlar bugün oldugu gibi yikik ve issiz birakildi.
\par 7 "Israil'in Tanrisi RAB, Her Seye Egemen Tanri söyle diyor: Neden bu büyük felaketi basiniza getiriyorsunuz? Kadin erkek, çoluk çocuk Yahuda halkindan kesilip atilacak, sizden sag kalan olmayacak.
\par 8 Yerlesmek üzere geldiginiz Misir'da ellerinizin yaptiklariyla, baska ilahlara buhur yakmakla beni öfkelendiriyorsunuz. Basiniza felaket getiriyorsunuz. Dünyadaki uluslarca asagilanacak, yerileceksiniz.
\par 9 Yahuda'da, Yerusalim sokaklarinda atalarinizin, Yahuda krallariyla karilarinin, kendinizin, karilarinizin yaptiginiz kötülükleri unuttunuz mu?
\par 10 Bugüne dek pismanlik duymadilar, benden korkmadilar. Size ve atalariniza verdigim yasa ve kurallar uyarinca yasamadilar.
\par 11 "Bu yüzden Israil'in Tanrisi, Her Seye Egemen RAB diyor ki: Basiniza yikim getirmeye, bütün Yahuda halkini yok etmeye kararliyim.
\par 12 Yerlesmek üzere Misir'a gelmeye kararli olan Yahuda'nin sag kalanlarini ele alacagim. Hepsi Misir'da yok olacak; kiliçtan geçirilecek ya da kitliktan ölecek. Küçük büyük hepsi kiliçtan, kitliktan ölecek. Lanetlenecek, dehset konusu olacak, asagilanacak, yerilecekler.
\par 13 Yerusalim'i cezalandirdigim gibi, Misir'da yasayanlari da kiliçla, kitlikla, salgin hastalikla cezalandiracagim.
\par 14 Yerlesmek için Misir'a gelen Yahuda halkinin sag kalanlarindan hiçbiri kurtulmayacak, hiç kimse sag kalip Yahuda'ya dönmeyecek. Yerlesmek üzere oraya dönmek isteseler de, kaçip kurtulan birkaç kisi disinda dönen olmayacak."
\par 15 Karilarinin baska ilahlara buhur yaktigini bilen erkekler, orada duran kadinlar, Misir'in Patros bölgesinde yasayan bütün halk -ki büyük bir topluluktu- Yeremya'ya su karsiligi verdi:
\par 16 "RAB'bin adiyla bize söylediklerini dinlemeyecegiz!
\par 17 Tersine, yapacagimizi söyledigimiz her seyi kesinlikle yapacagiz: Gök Kraliçesi'ne buhur yakacak, atalarimizin, krallarimizin, önderlerimizin ve kendimizin Yahuda kentlerinde, Yerusalim sokaklarinda yaptigimiz gibi ona dökmelik sunular dökecegiz. O zamanlar bol yiyecegimiz vardi, her isimiz yolundaydi, sikinti çekmiyorduk.
\par 18 Oysa Gök Kraliçesi'ne buhur yakmayi, dökmelik sunular dökmeyi biraktigimiz günden bu yana her yönden yokluk çekiyoruz; kiliçtan, kitliktan yok oluyoruz."
\par 19 Kadinlar, "Evet, Gök Kraliçesi'ne buhur yakip dökmelik sunular dökecegiz! Ona benzer pideler pisirip kendisine dökmelik sunular döktügümüzü kocalarimiz bilmiyor muydu sanki?" diye eklediler.
\par 20 Bunun üzerine Yeremya ona karsilik veren kadin erkek bütün halka söyle dedi:
\par 21 "Sizin, atalarinizin, krallarinizin, önderlerinizin, ülke halkinin Yahuda kentlerinde, Yerusalim sokaklarinda yaktiginiz buhuru RAB unuttu mu? Haberi yok muydu?
\par 22 RAB yaptiginiz kötülüklere, igrençliklere artik dayanamadigi için, bugün oldugu gibi ülkeniz asagilanip yerildi, kimsenin yasamadigi dehset verici bir viranelik oldu.
\par 23 Siz baska ilahlara buhur yaktiniz, RAB'be karsi günah islediniz; O'nun sözünü dinlemediniz, yasasina, kurallarina, antlasma kosullarina uymadiniz. Bu yüzden bugün oldugu gibi basiniza felaket geldi."
\par 24 Yeremya bütün halka, özellikle de kadinlara, "RAB'bin sözüne kulak verin, ey Misir'da yasayan Yahudalilar" dedi,
\par 25 "Israil'in Tanrisi, Her Seye Egemen RAB diyor ki, 'Gök Kraliçesi'ne buhur yakacagiz, dökmelik sunular dökecegiz, adaklarimizi kesinlikle yerine getirecegiz diyerek siz de karilariniz da verdiginiz sözü yerine getirdiniz. "Öyleyse verdiginiz sözü tutun! Adadiginiz adaklari tümüyle yerine getirin!
\par 26 Misir'da yasayan Yahudiler, RAB'bin sözünü dinleyin! 'Büyük adim üzerine ant içiyorum ki diyor RAB, 'Misir'da yasayan Yahudiler'den hiçbiri bundan böyle adimi agzina alip Egemen RAB'bin varligi hakki için diye ant içmeyecek.
\par 27 Çünkü onlarin yararini degil, zararini gözlüyorum; Misir'da yasayan Yahudiler yok olana dek kiliçtan, kitliktan ölecek.
\par 28 Kiliçtan kurtulup da Misir'dan Yahuda'ya dönenlerin sayisi pek az olacak. Misir'a yerlesmeye gelen Yahuda halkindan sag kalanlar o zaman kimin sözünün yerine geldigini anlayacak: Benim sözümün mü, yoksa onlarinkinin mi?
\par 29 "'Basiniza yikim getirecegim; sözümün yerine gelecegini bilesiniz diye diyor RAB, 'Sizi burada cezalandiracagima iliskin belirti su olacak.
\par 30 RAB diyor ki, 'Yahuda Krali Sidkiya'yi can düsmani Babil Krali Nebukadnessar'in eline nasil teslim ettimse, Misir Firavunu Hofra'yi da can düsmanlarinin eline öyle teslim edecegim."

\chapter{45}

\par 1 Yahuda Krali Yosiya oglu Yehoyakim'in dördüncü yilinda Neriya oglu Baruk, Peygamber Yeremya'nin kendisine söyledigi sözleri tomara yazdiktan sonra Yeremya ona sunlari söyledi:
\par 2 "Ey Baruk, Israil'in Tanrisi RAB sana söyle diyor:
\par 3 Sen, 'Vay basima! Çünkü RAB acima aci katti. Inlemekten bitkin düstüm, bana rahat yok dedin.
\par 4 "RAB bana, 'Ona söyle diyeceksin dedi: 'RAB diyor ki, bütün ülkeyi yikacagim; bina ettigimi yikacak, diktigimi sökecegim.
\par 5 Sana gelince, büyük seyler pesinde mi kosuyorsun? Sakin kosma! Çünkü bütün halkin üzerine felaket getirmek üzereyim diyor RAB, 'Ama sen nereye gidersen git, canini bagislayacagim."

\chapter{46}

\par 1 RAB uluslara iliskin Peygamber Yeremya'ya söyle seslendi:
\par 2 Misir'a iliskin: Yahuda Krali Yosiya oglu Yehoyakim'in dördüncü yilinda, Babil Krali Nebukadnessar'in Firat kiyisinda, Karkamis'ta yenilgiye ugrattigi Firavun Neko'nun ordusuyla ilgili bildiri:
\par 3 "Küçük büyük kalkanlari dizin, Savasmak için ilerleyin!
\par 4 Atlari kosun, beygirlere binin! Migferlerinizi takin, yerinizi alin! Mizraklarinizi cilalayin, Zirhlarinizi kusanin!
\par 5 Ne görüyorum? Dehsete düstüler, geri çekiliyorlar! Yigitleri bozguna ugramis, Arkalarina bakmadan kaçisiyorlar. Her yer dehset içinde" diyor RAB.
\par 6 "Ayagi tez olan kaçamiyor, Yigit kaçip kurtulamiyor. Kuzeyde, Firat kiyisinda Tökezleyip düstüler.
\par 7 Nil gibi yükselen, Irmak gibi sulari çalkalanan kim?
\par 8 Misir'dir Nil gibi yükselen, Irmak gibi sulari çalkalanan. 'Yükselip yeryüzünü kaplayacagim; Kentleri de içlerinde oturanlari da Yok edecegim diyor Misir.
\par 9 Sahlanin, ey atlar! Çilginca saldirin, ey savas arabalari! Ey kalkan tasiyan Kûslu*, Pûtlu yigitler, Yay çeken Ludlular, ilerleyin!
\par 10 "Çünkü o gün Her Seye Egemen Egemen RAB'bin günüdür. Düsmanlarindan öç almasi için Öç günüdür. Kiliç doyana dek yiyecek, Kanlarini kana kana içecek. Çünkü Rab, Her Seye Egemen RAB Kuzeyde, Firat kiyisinda kurban hazirliyor.
\par 11 "Ey erden kiz Misir, Gilat'a git de merhem al! Ama bosuna çok ilaç kullaniyorsun, Senin için sifa yok.
\par 12 Uluslar utancini duydu, Feryadinla doldu yeryüzü. Yigit yigide tökezleyip Ikisi birlikte yere seriliyor."
\par 13 Babil Krali Nebukadnessar'in gelip Misir'a saldiracagina iliskin RAB'bin Peygamber Yeremya'ya bildirdigi söz sudur:
\par 14 "Misir'da bildirin, Migdol'da duyurun, Nof'ta, Tahpanhes'te duyurun: 'Yerini al, hazirlan, Çünkü çevrendekileri yiyip bitiriyor kiliç!
\par 15 Ilahin Apis neden kaçti? Bogan neden ayakta kalamadi? Çünkü RAB onu yere serdi!
\par 16 Boyuna tökezleyip birbirlerinin üzerine düsecekler. 'Kalkin, acimasizlarin kilici yüzünden halkimiza, Yurdumuza dönelim diyecekler.
\par 17 'Firavun yaygaracinin biri, Firsati kaçirdi diyecekler.
\par 18 "Varligim hakki için" diyor Kral, Adi Her Seye Egemen RAB, "Daglar arasinda Tavor Dagi nasilsa, Karmel Dagi deniz kiyisinda nasilsa, Size saldiracak kisi de öyledir.
\par 19 Ey sizler, Misir'da yasayanlar, Toplayin esyanizi, sürgüne gideceksiniz! Nof*fj* öyle viran olup yanacak ki, Kimse oturmayacak içinde.
\par 20 "Misir güzel bir düve, Ama kuzeyden atsinegi geliyor ona.
\par 21 Ücretli askerleri besili danalar gibi. Onlar da geri dönüp birlikte kaçacak, Yerlerinde durmayacaklar. Çünkü üzerlerine yikim günü, Cezalandirilacaklari an gelecek.
\par 22 Düsman ordusu ilerleyince, Misir yilan gibi tislayarak kaçacak. Agaç kesen adamlar gibi Baltalarla ona saldiracaklar.
\par 23 Gür olsa bile kesecekler ormanini" diyor RAB, "Çünkü çekirgelerden daha çok onlar, Sayiya vurulamazlar.
\par 24 Misir utandirilacak, Kuzey halkinin eline teslim edilecek."
\par 25 Israil'in Tanrisi, Her Seye Egemen RAB diyor ki, "Iste No Kenti'nin ilahi Amon'u, firavunu, Misir'la ilahlarini, krallarini ve firavuna güvenenleri cezalandirmak üzereyim.
\par 26 Hepsini can düsmanlari Babil Krali Nebukadnessar'la görevlilerinin eline teslim edecegim. Ama sonra, eskiden oldugu gibi insanlar yine Misir'da yasayacak" diyor RAB.
\par 27 "Korkma, ey kulum Yakup, Yilma, ey Israil. Çünkü seni uzak yerlerden, Soyunu sürgün edildigi ülkeden kurtaracagim. Yakup yine huzur ve güvenlik içinde olacak, Kimse onu korkutmayacak.
\par 28 Korkma, ey kulum Yakup, Çünkü ben seninleyim" diyor RAB. "Seni aralarina sürdügüm uluslarin hepsini Tümüyle yok etsem de, Seni büsbütün yok etmeyecegim. Adaletle yola getirecek, Hiç cezasiz birakmayacagim seni."

\chapter{47}

\par 1 Firavun Gazze'ye saldirmadan önce RAB'bin Peygamber Yeremya'ya bildirdigi Filistliler'e iliskin söz sudur:
\par 2 RAB diyor ki, "Bakin sular kuzeyden nasil yükseliyor! Taskin bir irmak olacak, Ülkeyi ve içindeki her seyi, Kentleri ve içinde yasayanlari kaplayacak. Insanlar yakaracak, Ülkede yasayan herkes feryat edecek.
\par 3 Dörtnala kosan aygirlarin Toynak seslerinden, Savas arabalarinin takirtisindan, Tekerleklerin gürültüsünden Babalar dönüp çocuklarina bakmayacak; Ellerinde derman kalmayacak.
\par 4 Çünkü Filistliler'in yok edilecegi gün geliyor. Sur ve Sayda'ya yardim edebilecek Sag kalan herkes kesilip yok edilecek. RAB Kaftor kiyisindan gelen Filistliler'in Sag kalanlarini yok edecek.
\par 5 Gazze yastan saçini yolacak, Askelon susturulacak. Ey ovada sag kalanlar, Ne zamana dek bedenlerinizi yaralayacaksiniz?
\par 6 Ah, RAB'bin kilici! Yatismana daha ne kadar zaman var? Dön kinina! Dur ve sessiz ol!
\par 7 Ama RAB ona buyruk vermisken, Askelon'a, deniz kiyisina Saldirmak üzere görevlendirmisken Kiliç nasil yatisabilir?"

\chapter{48}

\par 1 Moav'a iliskin: Israil'in Tanrisi, Her Seye Egemen RAB söyle diyor: "Vay Nevo'nun basina gelenlere! Çünkü viraneye çevrilecek. Kiryatayim utandirilacak, ele geçirilecek. Misgav utandirilacak, kirilip dökülecek.
\par 2 Moav artik övülmeyecek, Hesbon'da onun yikimi için düzen kuracak, 'Haydi, su Moav ulusuna son verelim diyecekler. Ey Madmen, sen de susturulacaksin, Kiliç kovalayacak seni.
\par 3 Horonayim'den feryat duyulacak: 'Kent mahvoldu, büyük yikima ugradi!
\par 4 "Moav yikilacak, Yavrularinin aglayisi duyulacak.
\par 5 Aglaya aglaya çikiyorlar Luhit Yokusu'ndan, Horonayim inisinde Yikimin neden oldugu aci feryatlar duyuluyor.
\par 6 Kaçin, caninizi kurtarin! Çölde yaban esegi gibi kosun!
\par 7 "Evet, basarilarina, mal varligina güvendigin için Sen de ele geçirileceksin. Ilahin Kemos da kâhinleri ve görevlileriyle birlikte Sürgün edilecek.
\par 8 Yok edici her kente ugrayacak, Tek kent kurtulmayacak. Vadi yerle bir olacak, Yayla altüst edilecek" diyor RAB.
\par 9 "Moav topragina tuz dökün, kisirlassin, Kentleri öyle viran olacak ki, Kimse yasamayacak oralarda.
\par 10 Lanet olsun RAB'bin isini savsaklayana! Kilicini kan dökmekten alikoyana lanet olsun!
\par 11 Moav gençliginden bu yana güvenlikteydi, Sarap tortusu gibi durgun kaldi, Bir kaptan öbürüne bosaltilmadi, Sürgüne gönderilmedi. O yüzden tadini yitirmedi, kokusu bozulmadi.
\par 12 "Ama onu bosaltacak adamlari gönderecegim günler geliyor" diyor RAB, "Onu bosaltacaklar. Kaplarini bosaltacak, küplerini paramparça edecekler.
\par 13 Israil halki güvendigi Beytel'den nasil utandiysa, Moav da Kemos ilahindan öyle utanacak.
\par 14 "Nasil, 'Biz yigidiz, Savasa hazir askerleriz dersiniz?
\par 15 Moav ve kentlerini yerle bir eden, Saldiriya geçti. En seçkin gençleri kesime gidecek. Adi Her Seye Egemen RAB olan Kral böyle diyor.
\par 16 Moav'in yikimi yakinda geliyor, Ugrayacagi felaket hizla yaklasiyor.
\par 17 Dövünün onun için, Ey çevresinde yasayan, ününü bilen sizler! 'Kudret asasi, Görkemli degnek nasil da kirildi! deyin.
\par 18 "Ey Divon Kenti'nde yasayan halk, Görkeminden in, Kuru toprak üstünde otur. Çünkü Moav'i yerle bir eden sana da saldiracak, Kalelerini yikacak.
\par 19 Ey sen, Aroer'de oturan, Yol kenarinda dur da gözle! Kaçan adama, kurtulan kadina, 'Ne oldu? diye sor.
\par 20 Moav utandirildi, darmadagin oldu. Feryat et, haykir! Moav'in yikildigini Arnon Vadisi'nde duyur.
\par 21 "Yayladaki kentler -Holan, Yahas, Mefaat, Divon, Nevo, Beytdivlatayim, Kiryatayim, Beytgamul, Beytmeon, Keriyot, Bosra, uzak yakin bütün Moav kentleri- yargilanacak.
\par 25 Moav'in boynuzu kesildi, kolu kirildi" diyor RAB.
\par 26 "Moav'i sarhos edin, Çünkü RAB'be büyüklük tasladi. Moav kendi kusmugunda yuvarlanacak, Alay konusu olacak.
\par 27 Israil senin için gülünesi bir ulus mu oldu? Hirsizlar arasinda mi yakalandi ki, Ondan söz ettikçe bas salliyorsun?
\par 28 "Ey Moav'da yasayanlar, Kentlerinizi terk edip kayalara siginin. Uçurumun agzinda yuvasini yapan Güvercin gibi olun.
\par 29 Moav'in ne denli gururlanip büyüklendigini, Kendini ne denli begendigini, Kibirlenip küstahlastigini, Övünüp kabardigini duyduk.
\par 30 Küstahligini biliyorum" diyor RAB, "Övünmesi bosunadir, yaptiklari da.
\par 31 Bu yüzden Moav için haykiracak, Bütün Moav için feryat edecegim. Aglayacagim Kîr-Hereset halki için.
\par 32 Ey Sivma asmasi, Senin için Yazer halkindan çok aglayacagim. Filizlerin gölü asip Yazer'e ulasti. Yok edici yaz meyvelerini, üzümünü yok etti.
\par 33 Moav'in meyve bahçelerinden, tarlalarindan Sevinç ve nese yok oldu. Üzüm sikma çukurlarindan sarap akisini durdurdum; Kimse sevinç çigliklariyla üzüm ezmiyor, Çigliklar var, ama sevinç çigliklari degil.
\par 34 "Hesbon ve Elale'nin haykirislari Yahas'a ulasiyor. Soar'dan Horonayim'e, Eglat-Selisiya'ya dek çigliklar yükseliyor. Çünkü Nimrim sulari bile kurudu.
\par 35 Moav'da puta tapilan yerlerde Sunu sunanlari, Ilahlarina buhur yakanlari Yok edecegim" diyor RAB.
\par 36 "Bu yüzden yüregim ney gibi Inliyor Moav için; Kîr-Hereset halki için ney gibi Inliyor yüregim. Çünkü elde ettikleri zenginlik uçup gitti.
\par 37 "Herkes saçini sakalini kesecek, Elini yaralayacak, Beline çul saracak.
\par 38 Moav damlarinda, meydanlarinda Yalniz aglayis var. Çünkü Moav'i kimsenin begenmedigi Bir kap gibi kirdim" diyor RAB.
\par 39 "Nasil da darmadagin oldu Moav! Nasil aciyla feryat ediyor! Nasil da sirtini dönüyor utançtan! Moav çevresindekilere alay konusu, Dehset verici bir örnek oldu."
\par 40 RAB diyor ki, "Bakin! Düsman birden çullanan bir kartal gibi Kanatlarini Moav'in üzerine açacak.
\par 41 Keriyot ele geçirilecek, Kaleler alinacak. O gün Moavli askerlerin yüregi, Dogum sancisi çeken kadinin yüregi gibi olacak.
\par 42 Moav yikima ugrayacak, Halk olmaktan çikacak; Çünkü RAB'be karsi büyüklük tasladi.
\par 43 Önünde dehset, çukur ve tuzak var, Ey Moav halki!" diyor RAB.
\par 44 "Dehsetten kaçan çukura düsecek, Çukurdan çikan tuzaga yakalanacak; Çünkü Moav'in üzerine Cezalandirma yilini getirecegim" diyor RAB.
\par 45 "Hesbon'un gölgesinde Bitkin düsmüs kaçkinlar. Çünkü Hesbon'dan ates, Sihon'un ortasindan alev çikti; Moavlilar'in alinlarini, Kargasa çikaranlarin baslarini yakip yok etti.
\par 46 Vay sana, ey Moav! Ilah Kemos'un halki yok oldu, Ogullarin sürgüne gönderildi, Kizlarin tutsak alindi.
\par 47 Ama son günlerde Yine eski gönencine kavusturacagim Moav'i" diyor RAB. Moav'in yargisi burada sona eriyor.

\chapter{49}

\par 1 RAB Ammonlular'a iliskin söyle diyor: "Israil'in çocuklari yok mu? Yok mu mirasçisi? Öyleyse neden ilah Molek* Gad'i mülk edindi? Neden onun halki Gad kentlerinde oturuyor?
\par 2 Iste bu nedenle" diyor RAB, "Ammonlular'in Rabba Kenti'ne karsi Savas narasini isittirecegim günler geliyor. Rabba issiz bir höyük olacak, Köyleri atese verilecek. Böylece Israil, kendisini mülk edinenleri Mülk edinecek" diyor RAB.
\par 3 "Haykir, ey Hesbon! Ay Kenti yikildi! Feryat edin, ey Rabba kizlari! Çul sarinip yas tutun. Duvarlarin arasinda oraya buraya kosusun. Çünkü Molek kâhinleri ve görevlileriyle birlikte Sürgüne gönderilecek.
\par 4 Verimli vadilerinle ne kadar övünüyorsun, Ey dönek kiz! Servetine güvenerek, 'Bana kim saldirabilir? diyorsun.
\par 5 Bütün çevrenden Dehset saçacagim üzerine" Diyor Her Seye Egemen Egemen RAB. "Her biriniz apar topar sürülecek, Kaçkinlari toplayan olmayacak.
\par 6 Ama sonra Ammonlular'i Eski gönencine kavusturacagim" diyor RAB.
\par 7 Her Seye Egemen RAB Edom'a iliskin söyle diyor: "Teman Kenti'nde bilgelik kalmadi mi artik? Akilli kisilerde ögüt tükendi mi? Bilgelikleri yozlasti mi?
\par 8 Kaçin, geri dönün, derinliklere siginin, Ey Dedan'da yasayanlar! Çünkü Esav'i cezalandirdigimda Basina felaket getirecegim.
\par 9 Üzüm toplayanlar bagina girseydi, Birkaç salkim birakmazlar miydi? Gece hirsizlar gelselerdi, Yalnizca gereksindiklerini çalmazlar miydi?
\par 10 Oysa ben Esav'i çirilçiplak soyacak, Gizli yerlerini açiga çikaracagim, Gizlenemeyecek. Çocuklari, akrabalari, komsulari Yikima ugrayacak. Kendisi de yok olacak!
\par 11 Öksüz çocuklarini birak, Ben yasatirim onlari. Dul kadinlarin da bana güvensinler."
\par 12 RAB diyor ki, "Hak etmeyenler bile kâseyi* içmek zorundayken, Sen mi cezasiz kalacaksin? Hayir, cezasiz kalmayacaksin, Kesinlikle içeceksin kâseyi.
\par 13 Adim üzerine ant içerim ki" diyor RAB, "Bosra dehset konusu olacak, yerilecek, Viraneye dönecek, asagilanacak. Bütün kentleri sonsuza dek yikik kalacak."
\par 14 RAB'den bir haber aldim: Uluslara gönderdigi haberci, "Edom'a saldirmak için toplanin, Savasa hazirlanin!" diyor.
\par 15 "Bak, seni uluslar arasinda küçük düsürecegim, Insanlar seni hor görecek.
\par 16 Saçtigin dehset ve yüregindeki gurur Seni aldatti. Sen ki, kaya kovuklarinda yasiyor, Tepenin dorugunu elinde tutuyorsun. Yuvani kartal gibi yükseklerde kursan da, Oradan indirecegim seni" diyor RAB.
\par 17 "Edom dehset konusu olacak, Oradan geçen herkes saskin saskin bakip Basina gelen belalardan ötürü Onunla alay edecek.
\par 18 Sodom'la Gomora'yi ve çevredeki köyleri Nasil yerle bir ettimse" diyor RAB, "Orada da kimse oturmayacak, Insan oraya yerlesmeyecek.
\par 19 "Seria çaliliklarindan Sulak otlaga çikan aslan gibi Edom'u bir anda yurdundan kovacagim. Seçecegim kisiyi ona yönetici atayacagim. Var mi benim gibisi? Var mi bana dava açacak biri, Bana karsi duracak çoban?"
\par 20 Bu yüzden RAB'bin Edom'a karsi ne tasarladigini, Teman'da yasayanlara karsi ne amaçladigini isitin: "Sürünün küçükleri bile sürülecek, Halki yüzünden Edom otlaklari çöle dönüstürülecek.
\par 21 Yikilislarinin gürültüsünden yeryüzü titreyecek, Çigliklari Kizildeniz'e* dek duyulacak.
\par 22 Düsman kartal gibi üzerlerine çullanacak, Kanatlarini Bosra'ya karsi açacak. O gün Edomlu askerlerin yüregi, Dogum sancisi çeken kadinin yüregi gibi olacak."
\par 23 Sam'a iliskin: "Hama ve Arpat utanacak, Çünkü kötü haber isittiler. Korkudan eridiler, Sessiz duramayan deniz gibi Kaygiyla sarsildilar.
\par 24 "Sam güçsüz düstü, Kaçmak için döndü; Telasa kapildi, Doguran kadin gibi Sanci ve acilar sardi onu.
\par 25 Nasil oldu da sevinç buldugum ünlü kent Terk edilmedi?
\par 26 Bu yüzden gençleri meydanlarda düsecek, Bütün savasçilari susturulacak o gün" Diyor Her Seye Egemen RAB.
\par 27 "Sam surlarini atese verecegim, Yakip yok edecek Ben-Hadat'in saraylarini."
\par 28 Babil Krali Nebukadnessar'in bozguna ugrattigi Kedar ve Hasor kralliklarina iliskin RAB söyle diyor: "Kalkin, Kedar'a saldirin, Dogu halkini yok edin.
\par 29 Çadirlariyla sürüleri alinacak, Çadir perdeleri, Esyalariyla develeri alinip götürülecek. Insanlar, 'Her yer dehset içinde! diye bagiracaklar onlara.
\par 30 Kaçin, uzaklasin! Derinliklere siginin, Ey Hasor'da oturanlar!" diyor RAB. "Çünkü Babil Krali Nebukadnessar Size düzen kurdu; Sizin için bir tasarisi var.
\par 31 Kalkin, tasasiz ve güvenlik içinde Yasayan ulusa saldirin" diyor RAB. "Onun kent kapilari, sürgüleri yok, Halki tek basina yasiyor.
\par 32 Develeri yagma edilecek, Sayisiz sürüleri çapul mali olacak. Zülüflerini kesenleri Dört yana dagitacagim, Her yandan felaket getirecegim baslarina" diyor RAB.
\par 33 "Çakallarin ugragi Hasor, Sonsuza dek viran kalacak, Orada kimse oturmayacak, Insan oraya yerlesmeyecek."
\par 34 Yahuda Krali Sidkiya'nin kralliginin baslangicinda RAB'bin Peygamber Yeremya'ya bildirdigi Elam'a iliskin söz sudur:
\par 35 Her Seye Egemen RAB diyor ki, "Bakin, Elam'in yayini, Asil gücünü kiracagim.
\par 36 Üzerine gögün dört ucundan Dört rüzgari gönderecek, Halkini bu rüzgarlara dagitacagim. Elam sürgünlerinin gitmedigi Bir ulus kalmayacak.
\par 37 Düsmanlarinin önünde, Can düsmanlarinin önünde Elam'i darmadagin edecegim. Baslarina felaket gönderecek, Siddetli öfkemi yagdiracagim" diyor RAB, "Onlari büsbütün yok edene dek Peslerine kilici salacagim.
\par 38 Elam'da tahtimi kuracak, Elam Krali'yla önderlerini Yok edecegim" diyor RAB.
\par 39 "Ama son günlerde Elam'i eski gönencine kavusturacagim" diyor RAB.

\chapter{50}

\par 1 RAB'bin Babil ve Kildan* ülkesine iliskin Peygamber Yeremya araciligiyla bildirdigi söz sudur:
\par 2 "Uluslara duyurun, haberi bildirin! Sancak dikip duyurun, hiçbir sey gizlemeyin! 'Babil ele geçirilecek deyin, 'Ilahi Bel utandirilacak, Ilahi Marduk paramparça olacak. Putlari utandirilacak, Ilahlari paramparça olacak.
\par 3 Çünkü kuzeyden gelen bir ulus ona saldiracak, Ülkesini viran edecek. Orada kimse yasamayacak, Insan da hayvan da kaçip gidecek.
\par 4 O günlerde, o zamanda" diyor RAB, "Israil halkiyla Yahuda halki birlikte gelecek; Tanrilari RAB'bi aramak için Aglaya aglaya gelecekler.
\par 5 Yüzleri Siyon'a dönük, Oraya giden yolu soracak, Kalici, unutulmaz bir antlasmayla RAB'be baglanmak için gelecekler.
\par 6 "Halkim yitik koyunlardir, Çobanlari onlari bastan çikardi. Daglarda basibos dolandirdilar onlari, Dag, tepe avare dolastilar, Kendi agillarini unuttular.
\par 7 Kim bulduysa yedi onlari. Düsmanlari, 'Biz suçlu degiliz dediler, 'Çünkü onlar gerçek otlaklari olan RAB'be, Atalarinin umudu RAB'be karsi günah islediler.
\par 8 "Babil'den kaçip kurtulun! Kildan ülkesini terk edin, Sürüye yön veren teke gibi olun!
\par 9 Çünkü birbiriyle anlasmis büyük uluslari Kuzeydeki topraklardan kiskirtip Babil'in karsisina çikaracagim. Babil'le savasmak üzere karsisina dizilecek, Onu kuzeyden ele geçirecekler. Oklari usta savasçi oku gibidir, Hiçbiri bos dönmeyecek.
\par 10 Kildan ülkesi yagmaya ugrayacak, Onu yagmalayanlar mala doyacak" diyor RAB.
\par 11 "Ey mirasimi yagmalayan sizler! Madem sevinip cosuyorsunuz, Harman döven düve gibi siçriyor, Aygir gibi kisniyorsunuz;
\par 12 Anneniz büyük utanca bogulacak, Sizi doguranin yüzü kizaracak. Uluslarin en önemsizi, Kurak, bozkir, çöl olacak.
\par 13 RAB'bin öfkesi yüzünden kimse yasamayacak orada, Büsbütün issiz kalacak. Her geçen, Babil'in aldigi yaralari görünce sasacak, Hayrete düsecek.
\par 14 Babil'in çevresinde savasmak üzere dizilin, Ey bütün yay çekenler! Oklarla saldirin ona, oklarinizi esirgemeyin! Çünkü o RAB'be karsi günah isledi.
\par 15 Her yandan ona karsi savas narasi yükseltin! Teslim oldu, kuleleri düstü, Surlari yerle bir oldu. Çünkü RAB'bin öcüdür bu. Ondan öç alin. Yaptiginin aynisini yapin ona.
\par 16 Ekin ekeni biçim vakti orakçiyla birlikte Babil'den atin. Zorbanin kilici yüzünden Herkes halkina dönsün, Ülkesine kaçsin."
\par 17 "Israil aslanlarin kovaladigi Dagilmis bir sürüdür. Önce Asur Krali yedi onu. Sonra Babil Krali Nebukadnessar kemiklerini ezdi."
\par 18 Bu yüzden Israil'in Tanrisi, Her Seye Egemen RAB diyor ki, "Asur Krali'ni nasil cezalandirdiysam, Babil Krali'yla ülkesini de öyle cezalandiracagim.
\par 19 Israil'i yeniden otlagina kavusturacagim, Karmel'de, Basan'da otlayacak; Efrayim ve Gilat daglik bölgelerinde Istedigi kadar yiyip doyacak.
\par 20 O günlerde, o zamanda" diyor RAB, "Israil'in suçu arastirilacak Ama bulunamayacak; Yahuda'nin günahlari da arastirilacak Ama bulunamayacak. Çünkü sag biraktiklarimi bagislayacagim."
\par 21 "Meratayim ülkesine, Pekot'ta yasayanlara saldir. Onlari öldür, tümüyle yok et" diyor RAB, "Sana ne buyurduysam hepsini yap.
\par 22 Ülkede savas, büyük yikim Gürültüsü duyuluyor.
\par 23 Dünyanin balyozu Nasil da kirilip paramparça oldu! Babil uluslar arasinda nasil dehset oldu!
\par 24 Senin için tuzak kurdum, ey Babil, Bilmeden tuzagima düstün. Bulunup yakalandin, Çünkü RAB'be karsi çiktin.
\par 25 RAB silahhanesini açti, Öfkesinin silahlarini çikardi. Her Seye Egemen Egemen RAB'bin Kildan* ülkesinde yapacagi is var.
\par 26 Uzaktan ona saldirin. Ambarlarini açin, Mallarini tahil gibi küme küme yigin. Tamamen yok edin onu, Geriye hiçbir sey kalmasin.
\par 27 Genç bogalarini öldürün, Kesime gitsinler! Vay baslarina! Çünkü onlarin günü, Cezalandirilma zamani geldi.
\par 28 Dinleyin! Tanrimiz RAB'bin öç aldigini, Tapinaginin öcünü aldigini Babil'den kaçip kurtulanlar Siyon'da duyuruyorlar.
\par 29 "Okçulari, yay gerenlerin hepsini çagirin Babil'e karsi, Çevresini kusatin, kaçip kurtulan olmasin. Yaptiklarina göre karsilik verin ona, Yaptiklarinin aynisini yapin. Çünkü RAB'be, Israil'in Kutsali'na Küstahlik etti.
\par 30 Bu yüzden gençleri meydanlarda düsecek, Bütün savasçilari susturulacak o gün" diyor RAB.
\par 31 "Iste, sana karsiyim, ey küstah!" Diyor Her Seye Egemen Egemen RAB. "Çünkü senin günün, Seni cezalandiracagim zaman geldi.
\par 32 Küstah tökezleyip düsecek, Onu kaldiran olmayacak. Kentlerini atese verecegim, Bütün çevresini yakip yok edecek."
\par 33 Her Seye Egemen RAB söyle diyor: "Israil halki da Yahuda halki da Eziyet çekiyor. Onlari tutsak edenler siki tutmus, Salivermek istemiyorlar.
\par 34 Ama onlarin Kurtaricisi güçlüdür, O'nun adi Her Seye Egemen RAB'dir. Onlarin ülkesine huzur, Babil'de yasayanlaraysa kargasalik getirmek için Davalarini hararetle savunacak.
\par 35 "Kildaniler'e karsi kiliç!" diyor RAB, "Babil'de yasayanlara, Babil önderlerine, Bilgelerine karsi kiliç!
\par 36 Sahte peygamberlere karsi kiliç! Aptalliklari ortaya çikacak. Yigitlerine karsi kiliç! Saskina dönecek onlar.
\par 37 Atlarina, savas arabalarina Aralarindaki yabancilara karsi kiliç! Hepsi kadin gibi ürkek olacak. Hazinelerine karsi kiliç! Yagma edilecek onlar.
\par 38 Sularina kuraklik! Kuruyacak sular. Çünkü Babil putlar ülkesidir, Korkunç putlar yüzünden halki çildirmis.
\par 39 "Bu yüzden yabanil hayvanlar, çakallar, Baykuslar yasayacak orada, Artik insan yasamayacak, Kusaklar boyu kimse oturmayacak.
\par 40 Sodom'la Gomora'yi ve çevredeki köyleri Nasil yerle bir ettimse" diyor RAB, "Orada da kimse oturmayacak, Insan oraya yerlesmeyecek.
\par 41 Iste kuzeyden bir ordu geliyor. Dünyanin uçlarindan Büyük bir ulus Ve birçok kral harekete geçiyor.
\par 42 Yay, pala kusanmislar, Gaddar ve acimasizlar. Atlara binmis gelirken, Kükreyen denizi andiriyor sesleri. Savasa hazir savasçilar Karsina dizilecekler, ey Babil kizi!
\par 43 Babil Krali onlarin haberini aldi, Ellerinde derman kalmadi. Doguran kadin gibi Üzüntü, sanci sardi onu.
\par 44 Seria çaliliklarindan Sulak otlaga çikan aslan gibi Kildaniler'i bir anda yurdundan kovacagim. Seçecegim kisiyi oraya yönetici atayacagim. Var mi benim gibisi? Var mi bana dava açacak biri, Bana karsi duracak çoban?"
\par 45 Bu yüzden RAB'bin Babil'e karsi ne tasarladigini, Kildan ülkesine karsi ne amaçladigini isitin: "Sürünün küçükleri bile sürülecek, Halki yüzünden otlaklari çöle dönüstürülecek.
\par 46 'Babil düstü sesiyle yeryüzü titreyecek, Çigligi uluslar arasinda duyulacak."

\chapter{51}

\par 1 RAB diyor ki, "Iste Babil'e ve Lev-Kamay'da yasayanlara karsi Yok edici bir rüzgar çikaracagim.
\par 2 Tahil savuranlari gönderecegim Babil'e; Onu savurup ayiklasinlar, Ülkesini bosaltsinlar diye. Yikim günü her yandan saldiracaklar ona.
\par 3 Okçu yayini germesin, Zirhini kusanmasin. Onun gençlerini esirgemeyin! Ordusunu tümüyle yok edin.
\par 4 Kildan ülkesinde ölüler, Babil sokaklarinda yaralilar serilecek yere.
\par 5 Israil'in Kutsali'na karsi Ülkeleri suçla dolu olmasina karsin, Tanrilari Her Seye Egemen RAB Israil ve Yahuda halklarini birakmadi.
\par 6 Babil'den kaçin! Herkes canini kurtarsin! Babil'in suçu yüzünden yok olmayin! Çünkü RAB'bin öç alma zamanidir, Ona hakkettigini verecek.
\par 7 Babil RAB'bin elinde bir altin kâseydi*, Bütün dünyayi sarhos etti. Uluslar sarabini içtiler, Bu yüzden çildirdilar.
\par 8 Ansizin düsüp paramparça olacak Babil, Yas tutun onun için! Yarasina merhem sürün, belki iyilesir.
\par 9 'Babil'i iyilestirmek istedik, ama iyilesmedi. Birakalim onu, Hepimiz kendi ülkemize dönelim. Çünkü onun yargisi göklere erisiyor, Bulutlara kadar yükseliyor.
\par 10 "'RAB hakli oldugumuzu gösterdi, Gelin, Tanrimiz RAB'bin neler yaptigini Siyon'da anlatalim.
\par 11 "Oklari bileyin, Ok kiliflarini doldurun! RAB Med krallarini harekete geçirdi, Amaci Babil'i yok etmek. RAB öcünü, tapinaginin öcünü alacak.
\par 12 Babil surlarina karsi sancak kaldirin! Muhafizlari pekistirin, Nöbetçileri yerlestirin, Pusu kurun! Çünkü RAB Babil halki için söylediklerini Hem tasarladi hem de yerine getirdi.
\par 13 Ey sizler, akarsularin kiyisinda yasayan, Hazinesi bol olanlar, Sonunuz geldi, zamaniniz doldu.
\par 14 Her Seye Egemen RAB varligi hakki için ant içti: Seni çekirge sürüsüyle doldurur gibi Askerlerle dolduracagim. Sana karsi zafer çigliklari atacaklar."
\par 15 "Gücüyle yeryüzünü yaratan, Bilgeligiyle dünyayi kuran, Akliyla gökleri yayan RAB'dir.
\par 16 O gürleyince gökteki sular çagildar, Yeryüzünün dört bucagindan bulutlar yükseltir, Yagmur için simsek çaktirir, Ambarlarindan rüzgar estirir.
\par 17 Hepsi budala, bilgisiz. Her kuyumcu yaptigi puttan utanacak. O putlar yapmaciktir, Soluk yoktur onlarda.
\par 18 Yararsiz, alay edilesi nesnelerdir, Cezalandirilinca yok olacaklar.
\par 19 Yakup'un Payi onlara benzemez. Mirasi olan oymak dahil Her seye biçim veren O'dur, Her Seye Egemen RAB'dir adi.
\par 20 "Sen benim savas çomagim, Savas silahimsin. Uluslari parçalayacak, Kralliklari yok edecegim seninle.
\par 21 Seninle atlarla binicilerini, Savas arabalariyla sürücülerini kirip ezecegim.
\par 22 Erkeklerle kadinlari, Gençlerle yaslilari, Delikanlilarla genç kizlari,
\par 23 Çobanla sürüsünü, Çiftçiyle öküzlerini, Valilerle yardimcilarini darmadagin edecegim.
\par 24 "Babil'de ve Kildan* ülkesinde yasayanlara Siyon'da yaptiklari bütün kötülügün karsiligini Gözlerinizin önünde ödetecegim" diyor RAB.
\par 25 "Ey yikici dag, sana karsiyim, Ey bütün dünyayi yikan" diyor RAB, "Elimi sana karsi kaldirip Seni uçuruma yuvarlayacak, Yanik bir daga çevirecegim.
\par 26 Senden köse tasi, temel tasi olmayacak, Çünkü sonsuza dek viran kalacaksin" diyor RAB.
\par 27 "Ülkeye sancak dikin! Uluslar arasinda boru çalin! Uluslari Babil'le savasmaya hazirlayin. Ararat, Minni, Askenaz kralliklarini Ona karsi toplayin. Ona karsi bir komutan atayin, Çekirge sürüsü kadar at gönderin üzerine.
\par 28 Uluslari -Med krallarini, valilerini, Bütün yardimcilarini, Yönetimi altindaki bütün ülkeleri- Onunla savasmaya hazirlayin.
\par 29 Ülke titreyip kivraniyor! Çünkü RAB'bin Babil diyarini Issiz bir viraneye çevirme amaci Yerine gelmeli.
\par 30 Babil yigitleri savastan vazgeçti, Kalelerinde oturuyorlar. Güçleri tükendi, Ürkek kadinlara döndüler. Oturduklari yerler atese verildi, Kapi sürgüleri kirildi.
\par 31 Babil Krali'na ulak üstüne ulak, Haberci üstüne haberci geldi. 'Kent bütünüyle düstü, Irmak geçitleri tutuldu, Batakliklar atese verildi, Askerler dehsete kapildi diye haber verdiler."
\par 33 Israil'in Tanrisi, Her Seye Egemen RAB diyor ki, "Zamani gelince harman yeri nasil çignenirse, Babil kizi da öyle olacak. Kisa süre sonra onun da Biçim zamani gelecek."
\par 34 Siyon halki, "Babil Krali Nebukadnessar yuttu bizi, ezdi, Bos bir kaba çevirdi" diyecek, "Canavar gibi yuttu bizi, Güzel yemeklerimizle karnini doyurdu, Sonra bizi kustu. Bize ve yurttaslarimiza yapilan zorbalik Babil'in basina gelsin." Yerusalim, "Dökülen kanimizin hesabi Kildaniler'den sorulsun" diyecek. Rab Israil'e Yardim Edecek
\par 36 Bunun için RAB diyor ki, "Iste davanizi ben savunacagim, Öcünüzü ben alacagim; Onun irmagini kurutacak, Kaynagini kesecegim.
\par 37 Babil tas yiginina, çakal yuvasina dönecek, Dehset ve alay konusu olacak. Kimse yasamayacak orada.
\par 38 Halki genç aslanlar gibi kükreyecek, Aslan yavrulari gibi homurdanacak.
\par 39 Ama kizistiklarinda onlara sölen verip Hepsini sarhos edecegim; Keyiflensinler, Uyanmayacaklari sonsuz bir uykuya Dalsinlar diye" diyor RAB.
\par 40 "Onlari kuzu gibi, koç ve teke gibi Bogazlanmaya götürecegim."
\par 41 "Sesak nasil alindi! Bütün dünyanin övünç kaynagi nasil ele geçirildi! Uluslar arasinda Babil nasil dehset oldu!
\par 42 Deniz basacak Babil'i, Kabaran dalgalar örtecek.
\par 43 Kentleri viran olacak, Topraklari kimsenin yasamadigi, geçmedigi Kurak bir çöle dönecek.
\par 44 Babil ilahi Bel'i orada cezalandiracak, Yuttugunu ona kusturacagim. Artik akin akin uluslar gelmeyecek ona. Babil surlari yikilacak.
\par 45 "Oradan çik, ey halkim! Hepiniz caninizi kurtarin! Kaçin RAB'bin kizgin öfkesinden!
\par 46 Ülkede duyacaginiz söylentiler yüzünden Cesaretinizi yitirmeyin, korkmayin. Bir yil bir söylenti duyulur, ertesi yil bir baskasi; Ülkedeki zorbalikla, Önderin öndere karsi çiktigiyla Ilgili söylentiler yayilir.
\par 47 Iste bu yüzden Babil'in putlarini Cezalandiracagim günler geliyor. Bütün ülke utandirilacak, Öldürülenler ülkenin ortasinda yere serilecek.
\par 48 O zaman yer, gök ve onlardaki her sey Babil'in basina gelenlere sevinecek. Çünkü kuzeyden gelen yok ediciler Saldiracaklar ona" diyor RAB.
\par 49 Yeremya söyle diyor: "Israil'in öldürülenleri yüzünden düsmelidir Babil. Yeryüzünde öldürülen herkes Babil yüzünden düstü.
\par 50 Ey sizler, kiliçtan kurtulanlar, Kaçin, oyalanmayin! RAB'bi anin uzaktan, Yerusalim'i düsünün!"
\par 51 "Rezil olduk, çünkü asagilandik, Yüzümüz utanç içinde. Çünkü yabancilar RAB'bin Tapinagi'nin Kutsal yerlerine girmisler."
\par 52 "Bu yüzden" diyor RAB, "Putlarini cezalandiracagim günler geliyor, Yaralilar inleyecek bütün ülkede.
\par 53 Babil göklere çiksa, Yüksekteki kalesini pekistirse de, Yok edicileri gönderecegim üzerine" diyor RAB.
\par 54 "Babil'den çiglik, Kildan* ülkesinden büyük yikim sesi duyuluyor.
\par 55 Çünkü RAB Babil'i yikima ugratiyor; Samatasini susturuyor. Düsman engin sular gibi kükrüyor, Seslerinin gürültüsü yankilaniyor.
\par 56 Çünkü Babil'e karsi bir yok edici çikacak; Yigitleri tutsak olacak, Yaylari paramparça edilecek. Çünkü RAB karsilik veren bir Tanri'dir, Her seyin tam karsiligini verir.
\par 57 Babil önderlerini, bilgelerini, valilerini, Yardimcilarini, yigitlerini öyle sarhos edecegim ki, Sonsuz bir uykuya dalacak, hiç uyanmayacaklar" Diyor adi Her Seye Egemen RAB olan Kral.
\par 58 Her Seye Egemen RAB diyor ki, "Babil'in kalin surlari yerle bir edilecek, Yüksek kapilari atese verilecek. Halklarin çektigi emek bosuna, Uluslarin didinmesi atese yarayacak."
\par 59 Yahuda Krali Sidkiya'nin kralliginin dördüncü yilinda, bas görevli Mahseya oglu Neriya oglu Seraya Sidkiya'yla birlikte Babil'e gittiginde Peygamber Yeremya ona su buyrugu verdi.
\par 60 Yeremya Babil'in basina gelecek bütün felaketleri, Babil'e iliskin bütün bu sözleri bir tomara yazmisti.
\par 61 Yeremya Seraya'ya söyle dedi: "Babil'e varir varmaz bütün bu sözleri okumayi unutma.
\par 62 De ki, 'Ya RAB, burayi yikacagini, içinde insan da hayvan da yasamayacagini, ülkenin sonsuza dek viran kalacagini söyledin.
\par 63 Okumayi bitirince tomari bir tasa baglayip Firat'a firlat.
\par 64 Sonra de ki, 'Babil basina getirecegim felaket yüzünden batacak, bir daha kalkamayacak. Bitkin düsecekler." Yeremya'nin sözleri burada son buluyor.

\chapter{52}

\par 1 Sidkiya yirmi bir yasinda kral oldu ve Yerusalim'de on bir yil krallik yapti. Annesi Livnali Yeremya'nin kizi Hamutal'di.
\par 2 Yehoyakim gibi Sidkiya da RAB'bin gözünde kötü olani yapti.
\par 3 RAB Yerusalim'le Yahuda'ya öfkelendigi için onlari huzurundan atti. Sidkiya Babil Krali'na karsi ayaklandi.
\par 4 Sidkiya'nin kralliginin dokuzuncu yilinda, onuncu ayin* onuncu günü, Babil Krali Nebukadnessar bütün ordusuyla Yerusalim önlerine gelip ordugah kurdu. Kentin çevresine rampa yaptilar.
\par 5 Kral Sidkiya'nin kralliginin on birinci yilina kadar kent kusatma altinda kaldi.
\par 6 Dördüncü ayin dokuzuncu günü kentte kitlik öyle siddetlendi ki, halk bir lokma ekmek bulamaz oldu.
\par 7 Sonunda kentin surlarinda bir gedik açildi. Kildaniler* kenti çepeçevre kusatmis olmasina karsin, bütün askerler gece kral bahçesinin yolundan iki duvarin arasindaki kapidan kaçarak Arava yoluna çiktilar.
\par 8 Ama Kildani ordusu Kral Sidkiya'nin ardina düserek Eriha ovalarinda ona yetisti. Sidkiya'nin bütün ordusu dagildi.
\par 9 Kral Sidkiya yakalanip Hama topraklarinda, Rivla'da Babil Krali'nin huzuruna çikarildi. Babil Krali onun hakkinda karar verdi.
\par 10 Sidkiya'nin gözü önünde ogullarini, sonra da bütün Yahuda önderlerini öldürttü.
\par 11 Sidkiya'nin gözlerini oydu, zincire vurup Babil'e götürdü. Sidkiya öldügü güne dek cezaevinde tutuldu.
\par 12 Babil Krali Nebukadnessar'in kralliginin on dokuzuncu yilinda, besinci ayin onuncu günü muhafiz birligi komutani, Babil Krali'nin görevlisi Nebuzaradan Yerusalim'e girdi.
\par 13 RAB'bin Tapinagi'ni, sarayi ve Yerusalim'deki bütün evleri atese verip önemli yapilari yakti.
\par 14 Muhafiz birligi komutani önderligindeki Kildani ordusu Yerusalim'i çevreleyen bütün surlari yikti.
\par 15 Komutan Nebuzaradan yoksullardan bazilarini, kentte sag kalanlari, Babil Krali'nin safina geçen kaçaklari ve zanaatçilari sürgün etti.
\par 16 Ancak bagcilik, çiftçilik yapsinlar diye bazi yoksullari orada birakti.
\par 17 Kildaniler RAB'bin Tapinagi'ndaki tunç* sütunlari, ayakliklari, tunç havuzu parçalayip tunçlari Babil'e götürdüler.
\par 18 Tapinak törenlerinde kullanilan kovalari, kürekleri, fitil masalarini, çanaklari, tabaklari, bütün tunç esyalari aldilar.
\par 19 Muhafiz birligi komutani saf altin ve gümüs taslari, buhurdanlari, çanaklari, kovalari, kandillikleri, tabaklari, dökmelik sunu taslarini alip götürdü.
\par 20 RAB'bin Tapinagi için Kral Süleyman'in yaptirmis oldugu iki sütun, havuz ve altindaki on iki tunç boga heykeliyle ayakliklar için hesapsiz tunç harcanmisti.
\par 21 Her sütun on sekiz arsin yüksekligindeydi, çevresi on iki arsindi. Her birinin kalinligi dört parmakti, içi bostu.
\par 22 Üzerinde tunç bir baslik vardi. Basligin yüksekligi bes arsindi, çevresi tunçtan ag ve nar motifleriyle bezenmisti. Öbür sütun da nar motifleriyle süslenmisti ve ötekine benziyordu.
\par 23 Yanlarda doksan alti nar motifi vardi. Basligi çevreleyen ag motifinin üzerinde toplam yüz nar motifi bulunuyordu.
\par 24 Muhafiz birligi komutani Nebuzaradan Baskâhin Seraya'yi, Baskâhin Yardimcisi Sefanya'yi ve üç kapi nöbetçisini tutsak aldi.
\par 25 Kentte kalan askerlerin komutanini, kralin yedi danismanini, ayrica ülke halkini askere yazan ordu komutaninin yazmanini ve ülke halkindan kentte bulunan altmis kisiyi tutsak etti.
\par 26 Hepsini Rivla'ya, Babil Krali'nin yanina götürdü.
\par 27 Babil Krali Hama ülkesinde, Rivla'da onlari idam etti. Böylece Yahuda halki ülkesinden sürülmüs oldu.
\par 28 Nebukadnessar'in sürgüne götürdügü halkin sayisi sudur: Yedinci yil 3 023 Yahudi;
\par 29 Nebukadnessar'in on sekizinci yilinda Yerusalim'den 832 kisi;
\par 30 yirmi üçüncü yilinda, muhafiz birligi komutani Nebuzaradan'in sürdügü 745 Yahudi. Hepsi 4 600 kisiydi.
\par 31 Yahuda Krali Yehoyakin'in sürgündeki otuz yedinci yili Evil-Merodak Babil Krali oldu. Evil-Merodak o yilin on ikinci ayinin* yirmi besinci günü, Yahuda Krali Yehoyakin'e lütfederek onu cezaevinden çikardi.
\par 32 Kendisiyle tatli tatli konustu ve ona Babil'deki öteki sürgün krallardan daha üstün bir yer verdi.
\par 33 Yehoyakin cezaevi giysilerini üstünden çikardi. Yasadigi sürece Babil Krali'nin sofrasinda yer aldi.
\par 34 Yasami boyunca Babil Krali tarafindan günlük yiyecegi sürekli karsilandi.


\end{document}