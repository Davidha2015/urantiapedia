\begin{document}

\title{Hezekiel}


\chapter{1}

\par 1 Otuzuncu yilda, dördüncü ayin* besinci günü Kevar Irmagi kiyisinda sürgünde yasayanlar arasindayken gökler açildi, Tanri'dan gelen görümler gördüm.
\par 2 Kral Yehoyakin'in sürgünlügünün besinci yilinda, ayin besinci günü,
\par 3 Kildan* ülkesinde, Kevar Irmagi kiyisinda RAB Buzi oglu Kâhin* Hezekiel'e seslendi. RAB'bin eli orada onun üzerindeydi.
\par 4 Kuzeyden esen kasirganin göz alici bir isikla çevrelenmis, ates saçan büyük bir bulutla geldigini gördüm. Atesin ortasi isildayan madeni andiriyordu.
\par 5 En ortasinda insana benzer dört canli yaratik duruyordu;
\par 6 her birinin dört yüzü, dört kanadi vardi.
\par 7 Bacaklari dimdikti, ayaklari buzagi ayagina benziyor ve cilali tunç* gibi parliyordu.
\par 8 Dört yanlarinda, kanatlarin altinda insan elleri vardi. Dördünün de yüzleri, kanatlari vardi.
\par 9 Kanatlari birbirine degerek dosdogru ilerliyor, ilerlerken saga sola dönmüyordu.
\par 10 Her yaratigin dört yüzü vardi: Önde dördünün yüzü insan yüzüne, sagda dördünün aslan yüzüne, solda dördünün öküz yüzüne, arkada dördünün kartal yüzüne benzer bir yüzü vardi.
\par 11 Yüzleri böyleydi. Kanatlari yukariya dogru açilmisti. Her yaratigin iki kanadi yanda öbür yaratiklarin kanadina degiyor, iki kanatla da bedenlerini örtüyordu.
\par 12 Her biri dosdogru ilerliyordu. Ruhlari onlari nereye yönlendirirse, saga sola sapmadan oraya gidiyorlardi.
\par 13 Canli yaratiklarin görünüsü yanan ates közleri ya da mesale gibiydi. Ates yaratiklarin ortasinda hareket ediyordu; isik saçiyor ve içinden simsekler çakiyordu.
\par 14 Yaratiklar simsek çakar gibi hizla ileri geri gidip geliyorlardi.
\par 15 Bu dört yüzlü yaratiklara bakarken, her birinin yaninda, yere degen bir tekerlek gördüm.
\par 16 Tekerleklerin görünüsü ve yapisi söyleydi: Sari yakut gibi parliyorlardi ve dördü de birbirine benziyordu. Görünüsleri ve yapilislari iç içe girmis bir tekerlek gibiydi.
\par 17 Hareket edince yaratiklarin baktiklari dört yönden birine dogru saga sola sapmadan ilerliyordu.
\par 18 Tekerleklerin kenari yüksek ve korkunçtu; hepsi çepeçevre gözlerle doluydu.
\par 19 Canli yaratiklar hareket edince, yanlarindaki tekerlekler de hareket ediyordu; yaratiklar yerden yükseldikçe, tekerlekler de onlarla birlikte yükseliyordu.
\par 20 Ruhlari onlari nereye yönlendirirse oraya gidiyorlardi. Tekerlekler de onlarla birlikte yükseliyordu. Çünkü yaratiklarin ruhu tekerleklerdeydi.
\par 21 Yaratiklar hareket ettiginde onlar da hareket ediyor, yaratiklar durdugunda onlar da duruyor, yaratiklar yerden yükseldiginde onlar da yükseliyordu. Çünkü yaratiklarin ruhu tekerleklerdeydi.
\par 22 Kubbeye benzer, billur gibi parlak ve korkunç bir sey canli yaratiklarin baslari üzerine yayilmisti.
\par 23 Kubbenin altinda kanatlarinin biri öbürünün kanatlarina dogru açilmisti. Her birinin bedenini örten baska iki kanadi vardi.
\par 24 Yaratiklar hareket edince, kanatlarinin çikardigi sesi duydum. Gürül gürül akan sularin çagiltisini, Her Seye Gücü Yeten'in sesini, bir ordunun gürültüsünü ansitiyordu. Durunca kanatlarini indiriyorlardi.
\par 25 Kanatlari inik dururken, baslari üzerindeki kubbeden bir ses duyuldu.
\par 26 Baslari üzerindeki kubbenin üstünde laciverttasindan yapilmis tahta benzer bir nesne vardi. Yüksekte, tahti andiran nesnede insana benzer biri oturuyordu.
\par 27 Gördüm ki, beli andiran kisminin yukarisi içi ates dolu maden gibi isildiyordu, belden asagisi atese benziyordu ve çevresi göz alici bir isikla kusatilmisti.
\par 28 Görünüsü yagmurlu bir gün bulutlarin arasinda olusan gökkusagina benziyordu. Öyleydi çevresini saran parlaklik. RAB'bin görkemini andiran olayin görünüsü böyleydi. Görünce, yüzüstü yere yigildim, birinin konustugunu duydum.

\chapter{2}

\par 1 Bana, "Ey insanoglu, ayaga kalk, seninle konusacagim" dedi.
\par 2 O benimle konusur konusmaz Ruh içime girdi, beni ayaklarimin üzerinde durdurdu; benimle konusani duydum.
\par 3 Bana, "Ey insanoglu, seni Israil halkina, bana baskaldiran o asi ulusa gönderiyorum" dedi, "Onlar ve atalari bugüne kadar bana karsi geldiler.
\par 4 Bu halk dikbasli ve inatçidir. Seni onlara gönderiyorum. Onlara, 'Egemen RAB söyle diyor diyeceksin.
\par 5 Bu asi halk seni ister dinlesin, ister dinlemesin, yine de aralarinda bir peygamber oldugunu bilecektir.
\par 6 Sen, ey insanoglu, onlardan ve sözlerinden korkma! Çevrende çalilar, dikenler olsa, akrepler arasinda yasasan bile korkma. Asi bir halk olsalar bile, onlarin söyleyeceklerinden korkma, onlar yüzünden yilginliga düsme.
\par 7 Seni ister dinlesinler, ister dinlemesinler, onlara sözlerimi söyleyeceksin. Çünkü onlar asi bir halktir.
\par 8 Sen, ey insanoglu, sana söyleyecegimi dinle! Bu baskaldiran halk gibi asi olma! Agzini aç, sana verecegimi ye!"
\par 9 Baktim, bana dogru uzanmis bir el gördüm; içinde tomar halinde bir kitap vardi.
\par 10 Tomari önümde açti, her iki yani da yaziliydi. Orada agitlar, iniltiler, figanlar yaziliydi.

\chapter{3}

\par 1 Bana, "Ey insanoglu, sana verileni ye. Bu tomari yedikten sonra git, Israil halkina seslen" dedi.
\par 2 Böylece agzimi açtim, yemem için tomari bana verdi.
\par 3 Bana, "Ey insanoglu, sana verdigim tomari ye, mideni onunla doldur" dedi. Bunun üzerine tomari yedim. Bal gibi tatli geldi bana.
\par 4 Sonra söyle dedi: "Ey insanoglu, Israil halkina git, onlara sözlerimi ilet.
\par 5 Çünkü seni konusmasi anlasilmaz, dili zor bir halka degil, Israil halkina gönderiyorum.
\par 6 Evet, seni konusmasi anlasilmaz, dili zor, dediklerini anlamadigin halklara göndermiyorum. Onlara gönderseydim, seni dinlerlerdi.
\par 7 Israil halki seni dinlemek istemeyecektir, çünkü o beni dinlemek istemiyor. Bütün Israil halki dikbasli ve inatçidir.
\par 8 Seni onlar kadar inatçi yapacagim, senin alnini onlarinki kadar katilastiracagim.
\par 9 Alnini çakmak tasindan daha sert bir kaya gibi yapacagim. Her ne kadar asi bir halksalar da onlardan korkma, yilma."
\par 10 Bana, "Ey insanoglu, iyice dinle ve sana söyleyeceklerimi yüregine yerlestir" dedi,
\par 11 "Simdi sürgünde yasayan halkina git ve seni ister dinlesinler, ister dinlemesinler, onlara, 'Egemen RAB söyle diyor de."
\par 12 Sonra Ruh beni kaldirdi ve arkamda, "RAB'bin görkemine kendi yerinde övgüler olsun!" diye büyük bir gürleme duydum.
\par 13 Canli yaratiklarin birbirine çarpan kanatlarinin çikardigi sesi, yanlarindaki tekerleklerin gürültüsünü, büyük bir gürleme duydum.
\par 14 Ruh beni kaldirip götürdü. RAB'bin güçlü eli üzerimde oldugu halde, üzüntüyle, öfkeyle gittim.
\par 15 Kevar Irmagi kiyisindaki Tel-Abib'de yasayan sürgünlerin yanina geldim. Orada, yasadiklari yerde onlarin arasinda saskinlik içinde yedi gün kaldim.
\par 16 Yedi gün sonra RAB bana söyle seslendi:
\par 17 "Insanoglu, seni Israil halkina bekçi atadim. Benden bir söz duyar duymaz onlari benim yerime uyaracaksin.
\par 18 Kötü kisiye, 'Kesinlikle öleceksin dedigim zaman onu uyarmaz, yasamini kurtarmak amaciyla onu kötü yolundan döndürmek için konusmazsan, o kisi günahi içinde ölecek; ama onun kanindan seni sorumlu tutacagim.
\par 19 Ancak kötü kisiyi uyardigin halde kötülügünden ve kötü yolundan dönmezse, o günahi içinde ölecek. Ama sen canini kurtarmis olacaksin.
\par 20 "Dogru kisi dogrulugundan döner de kötülük yaparsa, onu yikima ugratacagim, o da ölecek. Onu uyarmadigin için günahi içinde ölecek, yaptigi dogru isler anilmayacak. Ancak onun kanindan seni sorumlu tutacagim.
\par 21 Ama dogru kisiyi günah islemesin diye uyarirsan, o da günah islemezse, kesinlikle yasayacak. Çünkü o uyarilara kulak vermistir; sen de canini kurtarmis olacaksin."
\par 22 RAB'bin eli orada üzerimdeydi. Bana, "Kalk, ovaya git" dedi, "Orada seninle konusacagim."
\par 23 Böylece kalkip ovaya gittim. RAB'bin görkemi tipki Kevar Irmagi kiyisinda gördügüm gibi orada durmaktaydi. Yüzüstü yere yigildim.
\par 24 Ruh içime girdi, beni ayaklarimin üzerinde durdurdu. Benimle söyle konustu: "Git, evine kapan.
\par 25 Halkin arasina çikmaman için seni halatlarla baglayacaklar, ey insanoglu.
\par 26 Dilini damagina yapistiracagim; konusmayacak, onlari paylayamayacaksin. Çünkü bu halk asidir.
\par 27 Ama seninle konustugumda dilini çözecegim. Onlara, 'Egemen RAB söyle diyor diyeceksin. Dinleyen dinlesin, dinlemeyen dinlemesin. Çünkü bu halk asidir."

\chapter{4}

\par 1 "Sen, ey insanoglu, bir tugla al, önüne koy, üzerine Yerusalim Kenti'ni çiz.
\par 2 Kenti kusat, duvarla çevir. Kente karsi toprak rampalar yap, ordugah kur, çevresine kütükler yerlestir.
\par 3 Sonra demir bir sac al; demirden bir duvar gibi kendinle kentin arasina koy. Yüzünü ona dogru çevir. Kent kusatma altinda tutulacak, onu sen kusatacaksin. Bu Israil halki için bir belirti olacak.
\par 4 "Sonra sol yanina uzan, Israil halkinin günahini yüklen. Sol yanina uzanacagin günler kadar onlarin suçunun cezasini çekeceksin.
\par 5 Suçlarinin yil sayisi kadar sana gün ayirdim. Böylece üç yüz doksan gün Israil halkinin suçunun cezasini çekeceksin.
\par 6 "Bunu yaptiktan sonra, bu kez sag yanina uzan, Yahuda halkinin suçunun cezasini çek. Sana kirk gün, her yil için bir gün ayirdim.
\par 7 Yüzünü Yerusalim kusatmasina çevir, çiplak kollarini kaldirip Yerusalim'e karsi peygamberlik et.
\par 8 Kusatma günlerini bitirinceye dek bir yandan öbür yana dönmemen için seni halatlarla baglayacagim.
\par 9 "Bugday, arpa, bakla, mercimek, dari, kizil bugday al, bir kaba koy. Bunlardan kendine ekmek yap. Bir yanina uzanacagin üç yüz doksan gün boyunca bu ekmekten yiyeceksin.
\par 10 Her gün belirli zamanda yemen için yirmi sekel ekmek tartacaksin.
\par 11 Bunun gibi suyu da belirli zamanda, ölçüyle, bir hinin altida biri kadar içeceksin.
\par 12 Yiyecegini arpa pidesi yer gibi ye ve insan diskisindan ates yakip üzerinde halkin gözü önünde pisir."
\par 13 RAB, "Uluslar arasina dagitacagim Israil halki böylelikle kirli sayilan yiyecekleri yiyecek" dedi.
\par 14 Ben, "Eyvah, ey Egemen RAB!" diye karsilik verdim, "Hiçbir zaman kirli sayilan bir seye dokunmadim. Gençligimden bu yana kendiliginden ölmüs ya da yabanil bir hayvan tarafindan öldürülmüs bir hayvanin etini yemedim, agzima kirli sayilan et koymadim."
\par 15 "Peki" dedi, "Ekmegini insan diskisi yerine tezek yakip üzerinde pisirmene izin verecegim."
\par 16 Sonra, "Insanoglu, Yerusalim'i her türlü yiyecekten yoksun birakacagim" dedi, "Bu halk yiyecegini tartiyla ve kaygi içinde yiyecek, suyunu ölçüyle ve saskinlik içinde içecek.
\par 17 Yiyecegi de suyu da azalacak. Hepsi saskinliga düsecek, günahlari içinde eriyip yok olacak.

\chapter{5}

\par 1 "Ey insanoglu, keskin bir kiliç al, berber usturasi gibi kullanarak basini, sakalini tiras et. Sonra bir terazi getir, killari bölümlere ayir.
\par 2 Yerusalim'in kusatilmasi bitince, killarin üçte birini kentin ortasinda yakacaksin. Üçte birini kiliçla kentin çevresine firlatacak, kalan üçte birini de rüzgara savuracaksin. Ben de yalin kiliç onlarin pesine düsecegim.
\par 3 Birkaç tel kil birak, giysinin kivrimlarina tak.
\par 4 Yine birkaçini alip atese at, yansin. O killardan bütün Israil halkina ates yayilacak.
\par 5 "Egemen RAB diyor ki: Bu Yerusalim'i uluslarin ortasina yerlestirdim, çevresini ülkelerle kusattim.
\par 6 Öyleyken Yerusalim çevresindeki bütün uluslardan ve ülkelerden daha çok kötülük yaparak ilkelerimi, kurallarimi çignedi. Ilkelerime karsi geldi, kurallarim uyarinca davranmadi.
\par 7 Bundan ötürü Egemen RAB diyor ki: Çevrenizde yasayan uluslardan daha azgindiniz, kurallarimi izlemediniz, ilkelerime uymadiniz. Çevrenizde yasayan uluslarin ilkelerine de uymadiniz.
\par 8 "Bundan ötürü Egemen RAB diyor ki: Iste ben size karsiyim, uluslarin gözü önünde sizi cezalandiracagim.
\par 9 Yaptiginiz bütün igrençlikler yüzünden önceden yapmadigimi, bir daha yapmayacagimi size yapacagim.
\par 10 Böylece aranizda babalar çocuklarini, çocuklar da babalarini yiyecekler. Sizi cezalandiracagim, sag kalanlarinizi her yana dagitacagim.
\par 11 Egemen RAB varligim hakki için diyor, madem tapinagimi igrenç put ve uygulamalarinizla kirlettiniz, ben de sizi esirgemeyecek, size acimayacak, sizi kayirmayacagim.
\par 12 Kentte yasayanlarinizin üçte biri salgin hastalik ya da kitlik yüzünden yok olacak; üçte biriniz çevrede kiliçtan geçirilecek; üçte birinizi de her yana dagitip yalin kiliç pesinize düsecegim.
\par 13 "Böylece kizginligim son bulacak, onlara karsi öfkemi yatistiracagim. O zaman ben de rahata kavusacagim. Öfkemi onlarin üzerine bosaltinca, ben RAB'bin kiskançligimdan onlarla konustugumu anlayacaklar.
\par 14 "Çevrenizdeki uluslar arasinda, yoldan her geçenin gözü önünde sizi yikima ugratacak, asagilayacagim.
\par 15 Öfke, kizginlik ve aci paylamalarla sizi cezalandirdigimda çevrenizdeki uluslar arasinda alay konusu olacak, asagilanacaksiniz; ders alinacak, sasilacak bir duruma düseceksiniz. Ben, RAB bunu söyledim.
\par 16 Sizi yok etmek için üzerinize öldürücü, yikici kitlik oklarini salacagim. Üzerinize salacagim kitligi daha da artiracak, sizi her türlü yiyecekten yoksun birakacagim.
\par 17 Üzerinize kitlik ve yabanil hayvanlar salacagim, sizi çocuklarinizdan edecekler. Salgin hastalik ve dökülen kan sizi süpürüp yok edecek; basiniza da kiliç getirecegim. Ben, RAB böyle söyledim."

\chapter{6}

\par 1 RAB bana söyle seslendi:
\par 2 "Ey insanoglu, yüzünü Israil daglarina dogru çevir ve onlara karsi peygamberlik et.
\par 3 De ki, 'Ey Israil daglari, Egemen RAB'bin sözünü dinleyin. Egemen RAB daglara, tepelere, vadilere, derelere söyle diyor: Üzerinize kiliç gönderecegim, tapinma yerlerinizi yikacagim.
\par 4 Sunaklarinizi devirecek, buhur sunaklarinizi paramparça edecegim. Kiliçtan geçirilmis halkinizi putlarinizin önüne düsürecegim.
\par 5 Israilliler'in cesetlerini putlarinin önüne atacak, kemiklerini sunaklarinin çevresine dagitacagim.
\par 6 Yasadiginiz her yerde kentleriniz yakilip yikilacak, tapinma yerleriniz yerle bir edilecek. Öyle ki, sunaklariniz devrilip yikilsin, putlariniz ezilip paramparça olsun, buhur sunaklariniz yok edilsin, el emeginiz bosa çiksin.
\par 7 Halkiniz her yerde öldürülecek. O zaman benim RAB oldugumu anlayacaksiniz.
\par 8 "'Birkaç kisiyi ölümden kurtaracagim. Ülkelere, uluslar arasina dagilan bazilariniz kiliçtan kurtulacak.
\par 9 Kurtulanlar tutsak alindiklari uluslarda beni animsayacaklar. Benden dönen sadakatsiz yüreklerinden, putlari ardinca sehvete sürükleyen gözlerinden derin aci duydum. Yaptiklari kötülükler ve igrenç uygulamalar yüzünden kendilerinden tiksinecekler.
\par 10 Benim RAB oldugumu, baslarina bu felaketi getirecegimi bosuna söylemedigimi anlayacaklar.
\par 11 "'Egemen RAB söyle diyor: Ellerinizi çirpin, ayaklarinizi yere vurun, Israil halkinin bütün kötü ve igrenç uygulamalarindan ötürü inleyin! Çünkü kiliçla, kitlikla, salgin hastalikla yok olacaklar.
\par 12 Uzaktakiler salgin hastaliktan ölecek, yakindakiler kiliçtan geçirilecek, kusatma sirasinda sag kalanlar kitliktan ölecek. Böylece onlara duydugum öfkeye son verecegim.
\par 13 Putlarinin arasina, sunaklarinin çevresine, her yüksek tepeye, dag doruguna, her yeseren bol yaprakli agacin altina cesetleri serilince, benim RAB oldugumu anlayacaklar. Oralarda putlarina güzel kokulu buhur sundular.
\par 14 Elimi onlara karsi uzatacak, çölden Rivla'ya kadar yasadiklari ülkeyi yerle bir edip issiz birakacagim. O zaman benim RAB oldugumu anlayacaklar."

\chapter{7}

\par 1 RAB bana söyle seslendi:
\par 2 "Ey insanoglu, Egemen RAB Israil ülkesine söyle diyor: Son yaklasti! Ülkenin dört kösesinin sonu geldi.
\par 3 Senin de sonun geldi! Senin üzerine öfkemi yagdiracagim. Yaptiklarina göre seni yargilayacak, bütün igrenç uygulamalarinin karsiligini verecegim.
\par 4 Sana acimayacak, seni esirgemeyecegim. Yaptiklarinin ve sendeki igrenç uygulamalarin karsiligini verecegim. O zaman benim RAB oldugumu anlayacaksiniz.
\par 5 "Egemen RAB söyle diyor: Yikim! Iste duyulmamis bir yikim geliyor.
\par 6 Sonun geldi! Evet, sonun geldi! Sana karsi uyaniyor. Iste geliyor.
\par 7 Ey ülkede yasayan halk, yikima ugrayacaksin. Yikim zamani yaklasti! Gün yakin! Daglarin üzerinden sevinç sesi yerine kargasa sesi geliyor.
\par 8 Çok yakinda kizginligimi üzerine bosaltacak, sana duydugum öfkeyi üzerine dökecegim. Yaptiklarina göre seni yargilayacak, bütün igrenç uygulamalarinin karsiligini verecegim.
\par 9 Sana acimayacak, seni esirgemeyecegim. Yaptiklarinin ve sendeki igrenç uygulamalarin karsiligini verecegim. O zaman seni cezalandiranin ben RAB oldugumu anlayacaksin.
\par 10 "Iste o gün! Gün yaklasti! Yikim hazir. Degnek çiçeklendi, gurur tomurcuklandi.
\par 11 Zorbalik ayaklanip kötülügün sopasi oldu. Halktan, o kalabaliktan kimse kalmayacak; mallarindan, görkemlerinden bir sey kalmayacak.
\par 12 "Son yaklasti! Gün geldi! Alici sevinmesin, satici üzülmesin. Çünkü öfkem bütün halkin üzerine yagacak.
\par 13 Satici yasadigi sürece sattigini geri alamayacak. Çünkü herkesi ilgilendiren bu görüm degistirilmeyecek. Isledigi günahlar yüzünden kimse canini koruyamayacak.
\par 14 Borazan çalindi, herkes hazir, ama kimse savasa gitmeyecek. Çünkü öfkem bütün halkin üzerindedir.
\par 15 "Iste disarda kiliç, içerde salgin hastalik ve kitlik. Kentin disindakiler kiliçla öldürülecek, kenttekilerse kitliktan, salgin hastaliktan yok olacak.
\par 16 Sag kalanlar vadilerdeki güvercinler gibi daglara kaçacak; her biri günahindan ötürü inleyecek.
\par 17 Eller gevseyecek, dizler titreyecek.
\par 18 Çul kusanacak, dehsete düsecekler. Yüzleri utançtan kizaracak, baslari tiras edilecek.
\par 19 Gümüslerini sokaga atacaklar. Altinlari kirli sayilacak. RAB'bin öfkesini bosalttigi gün onlari ne altinlari, ne gümüsleri kurtarabilir. Bunlarla ne açliklarini giderebilir, ne karinlarini doyurabilirler. Altin ve gümüs onlari suça sürükledi.
\par 20 Mücevherlerinin güzelligiyle gururlanirlardi. Igrenç, tiksindirici putlarini bunlardan yaptilar. Bu yüzden mücevherlerini kirli bir nesneye çevirecegim.
\par 21 Hepsini yagma mal olarak yabanci uluslara, ganimet olarak dünyadaki kötülere verecegim; onlari kirletecekler.
\par 22 Yüzümü onlardan çevirecegim. Degerli tapinagimi kirletecekler; zorbalar içeri girip orayi kirletecekler.
\par 23 "Kendinize zincirler hazirlayin! Ülkede kan akitiliyor, kent zorbalik dolu.
\par 24 Uluslarin en kötülerini buraya getirecegim; evlerinizi mülk edinecekler. Güçlülerin gururuna son verecegim. Kutsal yerleri kirletilecek.
\par 25 Korku gelince esenlik arayacak, ama bulamayacaklar.
\par 26 Yikim üstüne yikim gelecek. Kötü haberler birbirini kovalayacak. Peygamberden görüm isteyecekler; kâhin Kutsal Yasa'yi ögretemeyecek, ileri gelenler ögüt veremeyecek.
\par 27 Kral yas tutacak, önder umutsuzluga düsecek, ülkedeki halkin korkudan elleri titreyecek. Onlari yaptiklarina göre cezalandiracak, yargiladiklari gibi yargilayacagim. O zaman benim RAB oldugumu anlayacaklar."

\chapter{8}

\par 1 Sürgünlügün altinci yili, altinci ayin* besinci günü evde Yahuda'nin ileri gelenleriyle otururken Egemen RAB'bin eli bana dokundu.
\par 2 Baktim, insana benzer birini gördüm: Görünüsü, belinden asagisi atesi andiriyor, belinden yukarisi maden gibi isildiyordu.
\par 3 Eli andiran bir sey uzatip beni saçlarimdan tuttu. Ruh beni yerle gök arasina kaldirdi ve Tanri'dan gelen görümlerde Yerusalim'e, iç avlunun kuzeye bakan kapisinin giris bölümüne götürdü. Tanri'nin kiskançligini uyandiran kiskançlik putu orada dikiliydi.
\par 4 Ovada gördügüm görümdeki gibi, Israil'in Tanrisi'nin görkemi oradaydi.
\par 5 Sonra bana, "Ey insanoglu, kuzeye bak!" dedi. Baktim, sunak kapisinin kuzeye bakan giris bölümünde duran kiskançlik putunu gördüm.
\par 6 Bana, "Insanoglu, ne yaptiklarini görüyor musun?" dedi, "Tapinagimdan uzaklasayim diye Israil halki çok igrenç seyler yapiyor. Bundan daha igrenç seyler göreceksin."
\par 7 Beni avlunun giris bölümüne getirdi. Baktim, duvarda bir delik gördüm.
\par 8 Bana, "Haydi duvari del, insanoglu" dedi. Duvari deldim, orada bir kapi gördüm.
\par 9 Bana, "Içeri gir de burada yaptiklari kötü ve igrenç seyleri gör" dedi.
\par 10 Böylece içeriye girip baktim. Duvarin her yanina çesit çesit sürüngen*, igrenç hayvan sekilleri ve Israil halkinin bütün putlari oyulmustu.
\par 11 Israil ileri gelenlerinden yetmis kisiyle Safan oglu Yaazanya orada, putlarin önünde duruyordu. Her birinin elinde bir buhurdan vardi; buhurun kokusu bulut gibi yükseliyordu.
\par 12 "Insanoglu, Israil halkinin ileri gelenlerinin kendi putlarinin odalarinda, karanlikta neler yaptiklarini gördün mü?" dedi, "Onlar, 'RAB bizi görmüyor, RAB ülkeyi birakti diyorlar."
\par 13 Bana yine, "Daha igrenç seyler yaptiklarini da göreceksin" dedi.
\par 14 Bundan sonra beni RAB'bin Tapinagi'nin kuzeye bakan kapisinin giris bölümüne götürdü. Orada oturup Tammuz için aglayan kadinlari gördüm.
\par 15 Bana, "Insanoglu, bunu gördün mü? Bundan daha igrenç seyler de göreceksin" dedi.
\par 16 Beni RAB'bin Tapinagi'nin iç avlusuna götürdü. Tapinagin girisinde, eyvanla sunak arasinda yirmi bes kadar adam vardi. Sirtlarini RAB'bin Tapinagi'na, yüzlerini doguya dönmüs, günese tapiniyorlardi.
\par 17 Bana, "Insanoglu, bunlari gördün mü?" dedi, "Yahuda halki burada yaptigi igrenç seyler yetmiyormus gibi, ülkeyi zorbalikla doldurup beni sürekli öfkelendiriyor. Bak, dali nasil burunlarina uzatiyorlar!
\par 18 Bundan ötürü onlara öfkeyle davranacak, acimayacagim, onlari esirgemeyecegim. Yüksek sesle beni çagirsalar bile onlari dinlemeyecegim."

\chapter{9}

\par 1 Sonra yüksek sesle, "Kenti cezalandiracak olanlar, ellerinde yok edici silahlariyla buraya gelsin" diye seslendigini duydum.
\par 2 Kuzeye bakan yukari kapi yolundan alti kisinin geldigini gördüm. Her birinin elinde ölümcül bir silah vardi. Aralarinda keten giysili, belinde yazi takimi olan bir adam vardi. Içeriye girip tunç* sunagin yaninda durdular.
\par 3 Israil Tanrisi'nin görkemi bulundugu yerden, Keruvlar'in* üzerinden ayrilip tapinagin esigine gitti. RAB keten giysili, belinde yazi takimi olan adama seslendi:
\par 4 "Yerusalim Kenti'nin içinden geç, orada yapilan igrenç seylerden ötürü dövünüp aglayanlarin alinlarina isaret koy" dedi.
\par 5 Öbürlerine, "Kent boyunca onu izleyin ve kimseye acimadan, kimseyi esirgemeden öldürün" dedigini duydum.
\par 6 "Yasliyi, genci, genç kizi, kadini, çocuklari öldürün. Yalniz alinlarinda isaret olanlara dokunmayin. Ise tapinagimdan baslayin." Onlar da tapinagin önünde duran Israil ileri gelenlerinden ise basladilar.
\par 7 Onlara, "Tapinagi kirletin, avlularini cesetlerle doldurun. Haydi baslayin!" dedi. Bunun üzerine onlar gidip kenttekileri öldürmeye basladilar.
\par 8 Onlar halki öldürürken ben tek basima kaldim. Yüzüstü yere kapanip, "Ah, ey Egemen RAB! Öfkeni Yerusalim üzerine bosaltirken, geri kalan bütün Israilliler'i de mi yok edeceksin?" diye haykirdim.
\par 9 "Israil ve Yahuda halkinin günahi pek büyük" diye karsilik verdi, "Ülke kan, kent haksizlik dolu. Onlar, 'RAB ülkeyi birakti, RAB görmüyor diyorlar.
\par 10 Ben de onlara acimayacak, onlari esirgemeyecegim. Yaptiklarini kendi baslarina getirecegim."
\par 11 Derken keten giysili, belinde yazi takimi olan adam, "Buyruklarini yerine getirdim" diye haber verdi.

\chapter{10}

\par 1 Baktim, Keruvlar'in* basi üzerindeki kubbenin üzerinde laciverttasindan tahta benzer bir nesne gördüm.
\par 2 RAB keten giysili adama, "Keruvlar'in altindaki tekerleklerin arasina gir. Avuçlarini Keruvlar'in arasindaki ates közleriyle doldurup kentin üzerine közleri saç" dedi. Adamin oraya girdigini gördüm.
\par 3 Adam oraya girdiginde, Keruvlar tapinagin güney tarafinda duruyordu. Bulut tapinagin iç avlusunu doldurdu.
\par 4 RAB'bin görkemi Keruvlar'in üzerinden ayrilip tapinagin esigine gitti. Tapinak bulutla doldu. Avlu RAB'bin görkeminin pariltisiyla doluydu.
\par 5 Keruvlar'in kanatlarinin sesi dis avludan bile duyuluyordu; tipki Her Seye Gücü Yeten Tanri'nin sesi gibiydi.
\par 6 RAB keten giysili adama, "Keruvlar'dan ve tekerleklerin arasindan ates al" diye buyurunca, adam oraya girip bir tekerlegin yaninda durdu.
\par 7 Sonra Keruvlar'dan biri aralarindaki atese elini uzatti, biraz ates alip keten giysili adamin avuçlarina koydu. Adam atesi alip oradan ayrildi.
\par 8 Keruvlar'in kanatlari altinda insan eline benzer bir sekil göründü.
\par 9 Baktim, her Keruv'un yaninda birer tane olmak üzere dört tekerlek gördüm. Tekerlekler sari yakut gibi parildiyordu.
\par 10 Dördü de birbirine benziyor, iç içe girmis bir tekerlegi andiriyordu.
\par 11 Hareket edince Keruvlar'in baktiklari dört yönden birine dogru, saga sola dönmeden ilerliyordu. Ön tekerlek nereye yönelirse, öbür tekerlekler de onun ardinca gidiyordu.
\par 12 Keruvlar'in bedenleri -sirtlari, elleri, kanatlari- ve dördünün de tekerlekleri çepeçevre gözlerle doluydu.
\par 13 Tekerleklere "Dönen tekerlekler" dendigini duydum.
\par 14 Her Keruv'un dört yüzü vardi: Birinci yüz öküz yüzüne, ikincisi insan yüzüne, üçüncüsü aslan yüzüne, dördüncüsü kartal yüzüne benziyordu.
\par 15 Keruvlar yukariya dogru yükseldi. Bunlar daha önce Kevar Irmagi kiyisinda gördügüm canli yaratiklardi.
\par 16 Keruvlar hareket edince, yanlarindaki tekerlekler de hareket ediyor, Keruvlar yerden yükselmek için kanatlarini açinca, tekerlekler de yanlarindan ayrilmiyordu.
\par 17 Keruvlar durdugunda onlar da duruyor, Keruvlar yerden yükseldiginde onlar da yükseliyordu. Çünkü yaratiklarin ruhu tekerleklerdeydi.
\par 18 RAB'bin görkemi tapinagin esiginden ayrilip Keruvlar'in üzerinde durdu.
\par 19 Ben bakarken Keruvlar kanatlarini açip yerden yükseldi, tekerlekler de onlarla yükseldi. RAB'bin Tapinagi'nin Dogu Kapisi'nin girisinde durdular. Israil Tanrisi'nin görkemi onlarin üzerindeydi.
\par 20 Kevar Irmagi kiyisinda, Israil Tanrisi'nin altinda gördügüm ve Keruvlar oldugunu anladigim canli yaratiklar bunlardi.
\par 21 Her birinin dört yüzü, dört kanadi vardi. Kanatlarinin altinda insan elini andiran bir sey vardi.
\par 22 Yüzleri Kevar Irmagi kiyisinda gördügüm yüzlere benziyordu. Her biri dosdogru ilerliyordu.

\chapter{11}

\par 1 Ruh beni yine yukariya kaldirip RAB'bin Tapinagi'nin Dogu Kapisi'na götürdü. Kapinin giris bölümünde yirmi bes adam vardi. Aralarinda halkin önderlerinden Azzur oglu Yaazanya'yi, Benaya oglu Pelatya'yi gördüm.
\par 2 RAB bana, "Insanoglu, bunlar kötülük tasarlayan ve bu kentte kötü ögüt veren adamlardir" dedi,
\par 3 "Onlar, 'Yikim yakin degil, ev yapmanin zamanidir. Bu kent kazan, biz de etiz diyorlar.
\par 4 Bundan ötürü onlari uyar, ey insanoglu, onlari uyar."
\par 5 Sonra RAB'bin Ruhu üzerime inip sunlari söylememi buyurdu: "RAB söyle diyor: Ey Israil halki, neler söylediginizi ve neler düsündügünüzü bilirim.
\par 6 Bu kentte birçok kisi öldürdünüz, kentin sokaklarini ölülerle doldurdunuz.
\par 7 "Bundan ötürü Egemen RAB söyle diyor: Oraya attiginiz ölüler et, kent de kazandir. Ama sizi kentin disina sürecegim.
\par 8 Kiliçtan korktunuz, ama ben üzerinize kiliç gönderecegim. Egemen RAB böyle diyor.
\par 9 Sizi kentten çikarip yabancilarin eline teslim edecegim. Sizi cezalandiracagim.
\par 10 Kiliçla öldürüleceksiniz. Sizi Israil sinirinda cezalandiracagim. O zaman benim RAB oldugumu anlayacaksiniz.
\par 11 Bu kent sizin için kazan olmayacak, siz de onun içinde et olmayacaksiniz. Sizi Israil sinirinda cezalandiracagim.
\par 12 O zaman benim RAB oldugumu anlayacaksiniz. Kurallarimi izlemediniz, ilkelerime uymadiniz; çevrenizdeki uluslarin ilkelerine uydunuz."
\par 13 Ben peygamberlikte bulunurken Benaya oglu Pelatya öldü. Yüzüstü yere kapanip, "Ah, ey Egemen RAB! Geri kalan Israilliler'i büsbütün mü yok edeceksin?" diye yüksek sesle haykirdim.
\par 14 RAB bana söyle seslendi:
\par 15 "Ey insanoglu, Yerusalim'de yasayanlar senin kardeslerin, akrabalarin ve öbür Israilliler için, 'Onlar RAB'den uzaklar, bu ülke mülk olarak bize verildi demisler."
\par 16 "Bu yüzden de ki, 'Egemen RAB söyle diyor: Onlari uzaktaki uluslar arasina gönderdim, ülkeler arasina dagittim. Öyleyken gittikleri ülkelerde kisa süre için onlara barinak oldum.
\par 17 "De ki, 'Egemen RAB söyle diyor: Sizi uluslar arasindan toplayacak, dagilmis oldugunuz ülkelerden geri getirecek, Israil ülkesini yeniden size verecegim.
\par 18 "Ülkeye dönecek, tiksindirici, igrenç putlari oradan söküp atacaklar.
\par 19 Onlara tek bir yürek verecegim, içlerine yeni bir ruh koyacagim. Içlerindeki tas yüregi çikarip onlara etten bir yürek verecegim.
\par 20 O zaman kurallarimi izleyecek, ilkelerime uymaya özen gösterecekler. Onlar halkim olacak, ben de onlarin Tanrisi olacagim.
\par 21 Tiksindirici, igrenç putlara gönülden yönelenlere gelince, yaptiklarinin aynisini baslarina getirecegim. Böyle diyor Egemen RAB."
\par 22 Keruvlar* kanatlarini açti, tekerlekler yanlarinda duruyordu. Israil Tanrisi'nin görkemi onlarin üzerindeydi.
\par 23 RAB'bin görkemi kentin ortasindan yükselip kentin dogusundaki daga kondu.
\par 24 Görümde Tanri'nin Ruhu beni yukari kaldirip Kildan* ülkesindeki sürgünlerin yanina götürdü. Sonra gördügüm görüm kayboldu.
\par 25 Ben de RAB'bin bana gösterdigi her seyi sürgündekilere anlattim.

\chapter{12}

\par 1 RAB bana söyle seslendi:
\par 2 "Insanoglu, asi bir halkin arasinda yasiyorsun. Gözleri varken görmüyor, kulaklari varken isitmiyorlar. Çünkü bu halk asidir.
\par 3 "Sen, insanoglu, sürgüne gidecekmis gibi esyani topla, onlarin gözü önünde, gündüzün yola çik, bulundugun yerden baska bir yere git. Kim bilir, asi bir halk olmalarina karsin seni görüp anlayabilirler.
\par 4 Gündüzün, halkin gözü önünde topladigin sürgün esyani çikar. Aksam yine onlarin gözü önünde sürgüne giden biri gibi yola çik.
\par 5 Onlar seni izlerken duvari delip esyani çikar.
\par 6 Seni izlerlerken esyani sirtlayip karanlikta tasi. Ülkeyi görmemek için yüzünü ört. Çünkü yapacaklarin Israil halki için bir uyari olacaktir."
\par 7 Bana verilen buyruk uyarinca davrandim. Gündüzün sürgüne gidecekmis gibi esyalarimi çikardim. Aksam elimle duvari deldim. Esyalarimi karanlikta çikarip onlar izlerken sirtimda tasidim.
\par 8 Ertesi sabah RAB bana seslendi:
\par 9 "Insanoglu, o asi Israil halki sana, 'Ne yapiyorsun? Diye sormadi mi?
\par 10 "Onlara de ki, 'Egemen RAB söyle diyor: Yerusalim'deki önder ve orada yasayan bütün Israil halkina iliskin bir bildiridir bu.
\par 11 Ben sizin için bir uyariyim de. Sana yaptigimin tipkisi onlara da yapilacak. Tutsak olarak sürgüne gidecekler.
\par 12 "Onlarin önderi karanlikta esyasini sirtinda tasiyarak yola koyulacak. Esyasini çikarmak için duvarda bir gedik açacak. Ülkeyi görmemek için yüzünü örtecek.
\par 13 Onun üzerine agimi atacagim, kurdugum tuzaga düsecek. Onu Babil'e, Kildan* ülkesine götürecegim, ama ülkeyi göremeden orada ölecek.
\par 14 Çevresindekilerin tümünü -yardimcilarini, ordusunu dünyanin dört bucagina dagitacagim. Yalin kiliç onlarin peslerine düsecegim.
\par 15 Onlari uluslar arasina dagitip ülkelere sürdügümde, benim RAB oldugumu anlayacaklar.
\par 16 Gittikleri uluslarda yaptiklari bütün igrenç uygulamalari anlatmalari için aralarindan birkaç kisiyi kiliçtan, kitliktan, salgin hastaliktan sag birakacagim. Böylece benim RAB oldugumu anlayacaklar."
\par 17 RAB bana söyle seslendi:
\par 18 "Insanoglu, yiyecegini titreyerek ye, suyunu korkudan ürpererek iç.
\par 19 Ülkede yasayan halka de ki, 'Egemen RAB Israil ve Yerusalim'de yasayanlar için söyle diyor: Yiyeceklerini umutsuzluk içinde yiyecek, sularini saskinlik içinde içecekler. Orada yasayanlarin yaptigi zorbalik yüzünden ülke issiz birakilacak.
\par 20 Halkin içinde yasadigi kentler yakilacak, ülke çöle dönüsecek. O zaman benim RAB oldugumu anlayacaksiniz."
\par 21 RAB bana söyle seslendi:
\par 22 "Insanoglu, Israil'de yaygin olan, 'Günler geçiyor, her görüm bosa çikiyor deyisinin anlami nedir?
\par 23 Onlara de ki, 'Egemen RAB söyle diyor: Ben bu deyise son verecegim. Bundan böyle Israil'de bir daha söylenmeyecek. Yine onlara de ki, 'Her görümün yerine gelecegi günler yaklasti.
\par 24 Artik Israil halki arasinda yalan görüm ya da aldatici falcilik olmayacak.
\par 25 Ama ben RAB, ne dersem gecikmeden olacak. Siz, ey asi Israil halki, söylediklerimin tümünü sizin günlerinizde yerine getirecegim. Böyle diyor Egemen RAB."
\par 26 RAB bana söyle seslendi:
\par 27 "Insanoglu, Israil halki, 'Onun gördügü görüm uzak günler için, peygamberlik sözleri de uzak gelecekle ilgili diyor.
\par 28 "Bundan ötürü onlara de ki, 'Egemen RAB söyle diyor: Söyledigim sözlerden hiçbiri artik gecikmeyecek, ne dersem olacak. Böyle diyor Egemen RAB."

\chapter{13}

\par 1 RAB bana söyle seslendi:
\par 2 "Insanoglu, peygamberlikte bulunan Israil peygamberlerine karsi sen peygamberlik et. Kendiliginden peygamberlik eden o peygamberlere de ki, 'RAB'bin sözüne kulak verin!
\par 3 Egemen RAB söyle diyor: Hiçbir görüm görmemis ama kurduklari hayaller uyarinca davranan akilsiz peygamberlerin vay basina!
\par 4 Ey Israil, peygamberlerin yikintilar arasindaki çakallara benziyor.
\par 5 RAB'bin gününde Israil halkinin savasta direnmesi için gidip duvardaki gedikleri onarmadiniz.
\par 6 Onlarin görümleri uydurmadir. Yaptiklari yalan peygamberliklere RAB'bin sözüdür diyorlar. Oysa onlari ben göndermedim. Yine de söylediklerinin yerine gelecegini umuyorlar.
\par 7 Ben söylemedigim halde, RAB'bin sözüdür diyorsunuz. Oysa gördügünüz görümler uydurma, yaptiginiz falcilik yalan degil mi?
\par 8 "'Bu yüzden Egemen RAB söyle diyor: Söylediginiz bos sözler, gördügünüz yalan görümlerden ötürü size karsiyim. Böyle diyor Egemen RAB.
\par 9 Elim uydurma görüm gören, yalan yere falcilik eden peygamberlere karsi olacak. Onlar halkimin toplulugunda bulunmayacak, Israil halkinin kütügüne yazilmayacak, Israil ülkesine girmeyecekler. O zaman benim Egemen RAB oldugumu anlayacaksiniz.
\par 10 "'Esenlik yokken esenlik diyerek halkimi aldatiyorlar. Biri dayaniksiz bir duvar yapinca, sahte peygamberler üzerine siva vuruyorlar.
\par 11 Duvari sivayanlara de ki: Duvar yikilacak; saganak yagmur yagacak, ardindan dolu yagdiracagim. Siddetli bir rüzgar çikip duvara karsi esecek.
\par 12 Duvar çökünce size, nerede duvara vurdugunuz siva demeyecekler mi?
\par 13 "'Onun için Egemen RAB söyle diyor: Öfkemden duvari yerle bir etmek için siddetli bir rüzgar gönderecegim; kizginligimdan saganak yagmur ve dolu yagdiracagim.
\par 14 Siva vurdugunuz duvari yikip yerle bir edecegim. Temeli açilip ortaya çikacak. Yikilacak ve altinda yok olacaksiniz. O zaman benim RAB oldugumu anlayacaksiniz.
\par 15 Böylece öfkemi duvarin ve duvara siva vuranlarin üzerine bosaltacagim. Size duvar da duvara siva vuran da Yerusalim'de esenlik yokken esenlik görümleri gören Israilli peygamberler de yok oldu diyecegim. Egemen RAB böyle diyor."
\par 17 "Sen, ey insanoglu, kendiliginden peygamberlik eden halkinin kizlarina yüzünü çevir. Onlara karsi peygamberlik et.
\par 18 De ki, 'Egemen RAB söyle diyor: Insanlari tuzaga düsürmek için herkese bilek bagi diken, her boyda bas örtüsü yapan kadinlarin vay basina! Kendi caninizi korurken halkimin canini mi tuzaga düsüreceksiniz?
\par 19 Birkaç avuç arpayla birkaç dilim ekmek için halkimin arasinda beni küçük düsürdünüz. Yalana kulak veren halkima yalan söyleyerek ölümü hak etmemis canlari öldürdünüz, ölümü hak etmis canlari yasattiniz.
\par 20 "'Bundan ötürü Egemen RAB söyle diyor: Insanlari kus gibi tuzaga düsüren sihirli bilek baglariniza karsiyim. Onlari bileklerinizden koparacagim. Kus gibi tuzaga düsürdügünüz insanlari özgür kilacagim.
\par 21 Örtülerinizi yirtacak, halkimi elinizden kurtaracagim. Bir daha tuzaginiza düsmeyecekler. O zaman benim RAB oldugumu anlayacaksiniz.
\par 22 Madem incitmek istemedigim dogru kisinin cesaretini yalanlarinizla kirdiniz ve canini kurtarmak için kötü kisiyi kötü yolundan dönmemeye yüreklendirdiniz,
\par 23 bir daha uydurma görümler görmeyecek, falcilik etmeyeceksiniz. Halkimi elinizden kurtaracagim. O zaman benim RAB oldugumu anlayacaksiniz."

\chapter{14}

\par 1 Israil ileri gelenlerinden kimisi gelip yanima oturdu.
\par 2 O sirada RAB bana söyle seslendi:
\par 3 "Insanoglu, bu adamlarin yüregi putlara bagli. Diktikleri putlarin kendilerini günaha sokmasina olanak veriyorlar. Öyleyse onlarin bana danismasina izin vermeli miyim?
\par 4 Bunun için onlarla konus ve de ki, 'Egemen RAB söyle diyor: Yüregini puta baglayan, diktigi putun kendisini günaha sokmasina olanak veren, sonra da peygambere danismaya gelen her Israilli'ye putlarinin çokluguna göre ben RAB kendim karsilik verecegim.
\par 5 Bunu, putlari yüzünden bana sirt çeviren Israil halkinin yüregini yeniden kendime çekmek için yapacagim.
\par 6 "Bu yüzden Israil halkina de ki, 'Egemen RAB söyle diyor: Geri dönün! Putlarinizdan vazgeçin, igrenç uygulamalarinizi birakin!
\par 7 "'Israil halkindan biri ya da Israil'de yasayan bir yabanci benden ayrilir, yüregini putlara baglar, diktigi putlarin kendisini günaha sokmasina olanak verir, sonra da bana danismak üzere bir peygambere giderse, ben RAB kendim ona karsilik verecegim.
\par 8 O kisiye karsi çikacagim. Onu bir belirti, bir alay konusu yapip halkimin arasindan atacagim. O zaman benim RAB oldugumu anlayacaksiniz.
\par 9 "'Bir peygamber ayartilir da bir söz söylerse, onu ayartan benim. Elimi ona karsi uzatacagim, onu halkim Israil'in arasindan çikarip yok edecegim.
\par 10 Suçlarinin cezasini çekecekler. Peygamber de ona danisan da ayni sekilde cezalandirilacak.
\par 11 Böylece Israil halki bir daha benden ayrilmayacak, günahlariyla kendilerini kirletmeyecekler. Onlar halkim olacaklar, ben de onlarin Tanrisi olacagim. Egemen RAB böyle diyor."
\par 12 RAB bana söyle seslendi:
\par 13 "Insanoglu, eger bir ülke bana sadakatsizlik eder, günah islerse, ben de o ülkeye karsi elimi uzatir, onu her türlü yiyecekten yoksun birakir, üzerine kitlik gönderir, insanlari ve hayvanlari yok edersem;
\par 14 su üç adam Nuh, Daniel, Eyüp- orada olsalar bile, dogruluklariyla ancak kendi canlarini kurtarabilirler. Egemen RAB böyle diyor.
\par 15 "Ya da ülkeye yabanil hayvanlar gönderirsem ve ülkeyi kimsesiz birakirlarsa, ülke viraneye döner, hayvanlar yüzünden kimse içinden geçemezse;
\par 16 Varligim hakki için diyor Egemen RAB, bu üç kisi o ülkede yasasa bile, ne ogullarini ne de kizlarini kurtarabilirler. Ancak kendi canlarini kurtarabilirler. Ülke ise viraneye döner.
\par 17 "Ya da o ülkeye kiliç gönderir, 'Kiliç ülkeyi yarsin der, oradaki insanlari ve hayvanlari yok edersem;
\par 18 varligim hakki için diyor Egemen RAB, bu üç kisi orada olsa bile, ne ogullarini ne de kizlarini kurtarabilirler. Ancak kendi canlarini kurtarabilirler.
\par 19 "O ülkeye salgin hastalik gönderir, kan dökerek öfkemi yagdirir, oradaki insanlari ve hayvanlari yok edersem;
\par 20 varligim hakki için diyor Egemen RAB, Nuh, Daniel*fh* ve Eyüp orada olsa bile, ne ogullarini ne de kizlarini kurtarabilirler. Dogruluklariyla ancak kendi canlarini kurtarabilirler.
\par 21 "Egemen RAB söyle diyor: Yerusalim'deki insanlari ve hayvanlari yok etmek için üzerine dört agir yargimi kilici, kitligi, yabanil hayvanlari, salgin hastaligi- gönderdigimde daha neler neler olacak!
\par 22 Orada sag birakilacak kimi ogullariniz, kizlariniz olacak, çikip yaniniza gelecekler. Onlarin davranislarini ve yaptiklarini görünce, Yerusalim'in basina getirdigim yikimdan ve her tür felaketten avuntu bulacaksiniz.
\par 23 Onlarin davranislarini ve yaptiklarini görünce avutulacak, Yerusalim'in basina getirdiklerimin amaçsiz olmadigini anlayacaksiniz. Egemen RAB böyle diyor."

\chapter{15}

\par 1 RAB bana söyle seslendi:
\par 2 "Insanoglu, asma odununun herhangi bir orman agacinin dalindan daha fazla degeri var mi?
\par 3 Asma odunundan yararli bir sey yapilabilir mi? Ya da üzerine esya asmak için ondan aski yaparlar mi?
\par 4 Yakit olarak atese atilir da ates odunun iki ucunu yakip ortasini kömürlestirince, ise yarar mi?
\par 5 Yanmadan önce ise yaramadiysa, yanip kömür haline geldikten sonra bir ise yarar mi?
\par 6 "Bu nedenle Egemen RAB söyle diyor: Orman agaçlari arasinda asma odununu nasil yakit olarak atese verdimse, Yerusalim'de yasayan halka da aynisini yapacagim.
\par 7 Onlara yüz çevirecegim. Simdi atesten kurtulsalar bile, ates onlari yine de yakip yok edecek. Onlara yüz çevirince, benim RAB oldugumu anlayacaksiniz.
\par 8 Ülkeyi viraneye çevirecegim. Çünkü bana sadakatsizlik ettiler. Egemen RAB böyle diyor."

\chapter{16}

\par 1 RAB bana söyle seslendi:
\par 2 "Insanoglu, Yerusalim'e yaptigi igrenç uygulamalari bildir.
\par 3 De ki, 'Egemen RAB Yerusalim'e söyle diyor: Kökenin ve dogumun açisindan Kenan ülkesindensin; baban Amorlu, annense Hititli'ydi*.
\par 4 Dogdugun gün göbek bagin kesilmedi, temizlemek için seni yikamadilar, tuzla ovalamadilar, kundaga sarmadilar.
\par 5 Kimse bunlardan birini yapacak kadar sana acimadi, sevecenlik göstermedi. Senden tiksindikleri için dogdugun gün seni kira attilar.
\par 6 "'Yanindan geçtim, senin kendi kaninin içinde kimildadigini gördüm. Kendi kaninin içindeyken yasa! dedim. Evet, Kendi kaninin içindeyken yasa! dedim.
\par 7 Kirda yetisen bir bitki gibi seni gelistirdim. Gelistin, büyüdün, kusursuz bir güzellige eristin. Gögüslerin olustu, saçlarin uzadi. Ama çirilçiplaktin.
\par 8 "'Yine yanindan geçtim, sana baktim, sevgi çagindi. Giysimin etegini üzerine serdim, çiplakligini örttüm. Sana ant içtim, seninle antlasma yaptim. Egemen RAB böyle diyor. Ve benim oldun.
\par 9 "'Seni yikadim, üzerindeki kani temizledim, derine zeytinyagi sürdüm.
\par 10 Sana islemeli giysiler giydirdim, deriden çarik verdim. Beline ince keten kusak bagladim, seni pahali giysilerle örttüm,
\par 11 takilarla süsledim. Bileklerine bilezikler, boynuna gerdanlik taktim.
\par 12 Burnuna halka, kulaklarina küpeler, basina görkemli bir taç taktim.
\par 13 Altinla gümüsle süslendin; giysilerin ince ketenden, pahali, islemeli kumastandi. Ince unla, balla, zeytinyagiyla beslendin. Gitgide güzellestin, kralliga yarastin.
\par 14 Güzelliginden ötürü ünün uluslar arasinda yayildi. Çünkü seni görkemimle donattigim için güzelligin kusursuzdu. Egemen RAB böyle diyor.
\par 15 "'Ama sen güzelligine güvendin, ününü kullanarak fahiselik ettin. Her geçene gönlünü kaptirdin, kendini teslim ettin.
\par 16 Giysilerinden alip kendine süslü tapinma yerleri yaptin, oralarda fahiselik ettin. Böylesi ne olmustur, ne de olacaktir.
\par 17 Sana verdigim altin, gümüs süslerden erkek suretleri yaptin, onlarla fahiselik ettin.
\par 18 Islemeli giysilerini alip onlarin üzerine örttün. Onlara zeytinyagimi, buhurumu sundun.
\par 19 Yemen için sagladigim yiyecegi -ince unu, zeytinyagini, bali güzel kokulu bir sunu olarak onlara sundun. Böyle yaptin diyor Egemen RAB.
\par 20 "'Bana dogurdugun ogullari, kizlari alip yiyecek olarak putlara kurban ettin. Fahiselik etmen yetmiyormus gibi,
\par 21 çocuklarimi kesip sunu olarak ateste putlara kurban ettin.
\par 22 Bütün igrenç uygulamalarini, fahiseliklerini yaparken gençlik günlerini, çirilçiplak oldugun, kaninin içinde kimildandigin zamani animsamadin.
\par 23 "'Egemen RAB, vay, vay basina diyor! Yaptigin kötülüklere ek olarak,
\par 24 kendine fuhus yuvalari kurdun, bütün meydanlarda yüksek tapinma yerleri yaptin.
\par 25 Her yolun basina kendin için yüksek tapinma yerleri kurdun, güzelligini kirlettin, her geçene kendini teslim ettin, fahiseliklerini artirdin.
\par 26 Sehvet düskünü komsularin Misirlilar'la fahiselik ettin. Fahiseliklerini artirmakla beni öfkelendirdin.
\par 27 Iste bu yüzden elimi sana karsi uzattim, yiyecek payini azalttim. Ahlaksiz davranisindan utanç duyan düsmanlarin Filist kizlari dilediklerini yapsinlar diye seni onlara teslim ettim.
\par 28 Asurlular'la da fahiselik ettin, çünkü doymamistin. Evet, onlarla fahiselik ettin, yine doymadin.
\par 29 Fahiseligini ticaret diyari olan Kildan* ülkesine dek artirdin, yine de doymadin.
\par 30 "'Bütün bunlari yaparken yüregin ne kadar yipranmis diyor Egemen RAB, 'Yüzsüz bir fahise gibi davrandin!
\par 31 Her yolun basina fuhus yuvalari kurarken, bütün meydanlarda yüksek tapinma yerleri yaparken, fahise gibi bile degildin, ücretini küçümsedin.
\par 32 "'Kocasinin yerine yabancilari yegleyen, zina eden bir kadindin!
\par 33 Fahiselere ücret ödenir. Oysa sen bütün oynaslarina armaganlar dagittin. Fahiselik etmek için her yandan sana gelsinler diye rüsvet verdin.
\par 34 Fahiseliginde öbür kadinlara benzemiyorsun. Çünkü fahiselik edesin diye kimse senin pesine düsmüyor. Ücret ödeyen sensin, kimse sana ücret ödemiyor. Bu yüzden öbürlerine benzemiyorsun.
\par 35 "'Bu nedenle, ey fahise, RAB'bin sözünü dinle!
\par 36 Egemen RAB söyle diyor: Yüzsüzlügün ortaya döküldügü, oynaslarinla fahiselik ederken çiplakligin meydana çiktigi için, bütün igrenç putlarin yüzünden, onlara çocuklarinin kanini verdigin için,
\par 37 düsüp kalktigin bütün oynaslarini sevdiklerini de nefret ettiklerini de- toplayacagim. Sana karsi onlari her yandan toplayacak, çiplakligini onlarin önüne serecegim; bütün çiplakligini görecekler.
\par 38 Sana zina eden, kan döken kadinlara verilen cezayi verecegim. Kanini akitarak seni öfkemin ve kiskançligimin öcüne terk edecegim.
\par 39 Seni oynaslarinin eline teslim edecegim. Fuhus yuvalarini yikacak, yüksek tapinma yerlerini bozacaklar. Üzerindeki giysileri soyacak, güzel mücevherlerini alip seni çirilçiplak birakacaklar.
\par 40 Halki sana karsi kiskirtacaklar. Seni taslayacak, kiliçlariyla delik desik edecekler.
\par 41 Evlerini atese verecek, seni birçok kadinin gözü önünde yargilayacaklar. Fahiseliklerine son verecegim, artik oynaslarina ücret ödemeyeceksin.
\par 42 Böylece sana karsi öfkem yatisacak, kiskançligim dinecek. Susacak, bir daha öfkelenmeyecegim.
\par 43 "'Madem gençlik günlerini animsamadin, yaptiklarinla beni öfkelendirdin, ben de yaptiklarini senin basina getirecegim. Böyle diyor Egemen RAB. Bu igrenç uygulamalarina ek olarak ahlaksizlik da ettin.
\par 44 "'Herkes senin için su deyisi söyleyecek: Annesi nasilsa kizi da öyle.
\par 45 Sen kocasindan ve çocuklarindan tiksinen annenin kizisin; kocalarindan ve çocuklarindan tiksinen kizkardeslerinin kizkardesisin. Annen Hititli, baban Amorlu'ydu.
\par 46 Kizlariyla senin kuzeyinde yasayan Samiriye ablan, kizlariyla senin güneyinde yasayan Sodom kizkardesindir.
\par 47 Sen yalniz onlarin yolunda yürümekle, onlarin igrenç uygulamalarina uymakla kalmadin, bütün yaptiklarinla kisa sürede onlardan daha büyük kötülük ettin.
\par 48 Varligim hakki için diyor Egemen RAB, kizkardesin Sodom'la kizlari, kizlarinla senin yaptiklarini asla yapmadilar.
\par 49 "'Kizkardesin Sodom'un günahi suydu: Kendisi de kizlari da gururluydu, ekmege doymuslardi, umursamazlardi. Düsküne, yoksula yardim elini uzatmadilar.
\par 50 Kendilerini begenmislerdi. Önümde igrenç seyler yaptilar. Bu nedenle, gördügün gibi onlari önümden süpürüp attim.
\par 51 Samiriye isledigin günahin yarisini bile islemedi. Sen onlardan çok daha igrenç seyler yaptin. Yaptigin igrençliklerle kizkardeslerini suçsuz çikardin.
\par 52 Düsecegin utanca katlanacaksin. Çünkü kizkardeslerini hakli gibi gösterdin. Isledigin günahlar onlarinkinden daha igrenç oldugundan senin yaninda suçsuz kaliyorlar. Bunun için utan ve düsecegin utanca katlan. Çünkü kizkardeslerini suçsuz çikardin!
\par 53 "'Sodom'la kizlarini, Samiriye'yle kizlarini, onlarla birlikte de seni eski gönencine kavusturacagim.
\par 54 Utanca bogulacaksin. Bütün yaptiklarindan ötürü kizkardeslerine avuntu olacak ve utanacaksin.
\par 55 Kizkardeslerin Sodom ve Samiriye ile kizlari eski durumlarina dönecekler; kizlarinla sen de öyle.
\par 56 Kötülügün açiga çikmadan önce, gururlu oldugun günlerde kizkardesin Sodom'un adini bile anmiyordun. Simdi sen de Edom kizlariyla komsulari ve Filist kizlarinca -çevrende seninle alay edenlerce- küçümseniyorsun.
\par 58 Ahlaksizliginin ve yaptigin igrençliklerin sonuçlarina katlanacaksin. RAB böyle diyor.
\par 59 "'Egemen RAB söyle diyor: Seninle yaptigim antlasmayi bozarak içtigin andi küçümsedin. Ben de hak ettigin biçimde seni cezalandiracagim.
\par 60 Gençlik günlerinde seninle yaptigim antlasmayi animsayacagim. Seninle sonsuza dek kalici bir antlasma yapacagim.
\par 61 Büyük, küçük kizkardeslerini yanina aldiginda yaptiklarini animsayacak ve utanacaksin. Seninle yaptigim antlasmada olmadigi halde onlari kizlarin olsunlar diye sana verecegim.
\par 62 Seninle yeniden antlasma yapacagim, benim RAB oldugumu anlayacaksin.
\par 63 Bütün yaptiklarini bagisladigimda, animsayacak ve utanacaksin. Utancindan bir daha agzini açmayacaksin. Egemen RAB böyle diyor."

\chapter{17}

\par 1 RAB bana söyle seslendi:
\par 2 "Ey insanoglu, Israil halkina bir bilmece sor, simgesel bir öykü anlat.
\par 3 De ki, 'Egemen RAB söyle diyor: Kanatlari uzun ve güçlü, renk renk tüylerle dolu iri bir kartal Lübnan'a geldi, bir sedir agacinin tepesine konup onu ele geçirdi.
\par 4 Agacin tepesindeki filizleri koparip ticaret ülkesine götürdü, tüccarlar kentine yerlestirdi.
\par 5 "'Ülkenin tohumundan alip verimli topraga ekti; onu sögüt agaci gibi akarsularin kiyisina dikti.
\par 6 Tohum filizlenip yerde yayilan bodur bir asma oldu. Dallari kartala dogru yayildi, kökleriyse asagiya, derine indi. Böylece dal salan, filiz veren bir asma oldu.
\par 7 "'Gelgelelim, kanatlari güçlü, bol tüylü baska bir iri kartal da vardi. Asma bu kez dikildigi yerden köklerini bu kartala dogru çevirdi; sulasin diye dallarini ona dogru saldi.
\par 8 Dallansin, ürün versin, görkemli bir asma olsun diye akarsularin kiyisindaki verimli topraga dikilmisti.
\par 9 "Onlara de ki, 'Egemen RAB söyle diyor: Asma serpilecek mi? Kurusun diye ilk kartal kökünü söküp meyvesini koparmayacak mi? Asmanin yeni filizlenen bütün dallari kuruyacak. Kökünden söküp atmak için güçlü ele ya da büyük orduya gerek duyulmayacak.
\par 10 Evet, asma dikilmis, ama serpilip gelisecek mi? Dogu rüzgari ona çarpinca büsbütün kurumayacak mi? Evet, filizlendigi yerde solup kuruyacak."
\par 11 RAB bana söyle seslendi:
\par 12 "O asi halka de ki, 'Bunlarin ne anlama geldigini bilmiyor musunuz? Onlara de ki, 'Babil Krali Yerusalim'e gitti; kralini, önderlerini tutsak alip kendisiyle birlikte Babil'e götürdü.
\par 13 Sonra kralin soyundan gelen birini alip ant içirerek onunla bir antlasma yapti. Ülkenin önderlerini de tutsak aldi.
\par 14 Öyle ki, ülke gerilesin, bir daha yükselmesin, ancak yaptigi antlasmayi yerine getirerek yasayabilsin.
\par 15 Ne var ki, Yahuda Krali, kendisine at ve çok sayida asker vermesi için Misir'a elçiler göndererek Babil Krali'na baskaldirdi. Yahuda Krali basaracak mi? Böyle seyler yapan kurtulur mu? Yaptigi antlasmayi bozan kurtulur mu?
\par 16 "'Egemen RAB, varligim hakki için diyor, onu tahta oturtan kralin ülkesinde, Babil'de ölecek. Çünkü içtigi andi küçümsedi, yaptigi antlasmayi bozdu.
\par 17 Babilliler birçok kisiyi yok etmek için toprak rampalar, kusatma duvarlari yaptiginda, firavun güçlü ordusu ve büyük kalabaliklarla savasta ona yardimci olmayacak.
\par 18 Yaptigi antlasmayi bozarak içtigi andi küçümsedi. Söz verdigi halde, bütün bunlari yapti. Bu yüzden kurtulmayacak.
\par 19 "'Bu nedenle Egemen RAB söyle diyor: Varligim hakki için, bana içtigi andi küçümsedigi, antlasmami bozdugu için onu cezalandiracagim.
\par 20 Agimi gerecegim, tuzagima düsecek. Onu Babil'e getirecek, bana sadakatsizliginden ötürü orada yargilayacagim.
\par 21 En seçkin askerleri kiliçtan geçirilecek, sag kalanlar dünyanin dört bucagina dagilacak. O zaman konusanin ben RAB oldugumu anlayacaksiniz.
\par 22 "'Egemen RAB söyle diyor: Sedir agacinin tepesinden Bir filiz alip dikecegim. En yüksek dallarindan körpe bir çubuk koparip Yüksek, ulu bir dagin üzerine dikecegim.
\par 23 Onu Israil'in en yüksek daginin üzerine dikecegim. Dal budak salip ürün verecek, Görkemli bir sedir agaci olacak. Her çesit kus dallarina tüneyecek, Gölgesinde barinacak.
\par 24 Bütün orman agaçlari Her yüksek agaci bodurlastiranin, Her bodur agaci yükseltenin, Her yesil agaci kurutanin Ve kuru agaci yesertenin Ben RAB oldugumu anlayacaklar. Bunu ben RAB söylüyorum ve dedigimi yapacagim."

\chapter{18}

\par 1 RAB bana söyle seslendi:
\par 2 "Israil için, 'Babalar koruk yedi, Çocuklarin disleri kamasti diyorsunuz. Bu deyisle ne demek istiyorsunuz?
\par 3 "Varligim hakki için diyor Egemen RAB, Israil'de artik bu deyisi agziniza almayacaksiniz.
\par 4 Her yasayan can benimdir. Babanin cani da, çocugun cani da benimdir. Ölecek olan, günah isleyen candir.
\par 5 "Diyelim ki, adil ve dogru olani yapan dogru bir adam var.
\par 6 Daglarda putlara sunulan kurbandan yemez, Israil halkinin putlarina bel baglamaz. Komsusunun karisini kirletmez, Âdet gören kadina yaklasmaz.
\par 7 Kimseye haksizlik etmez, Rehin olarak aldigini geri verir, Soygunculuk etmez, Aç olana ekmegini verir, Çiplagi giydirir.
\par 8 Faizle para vermez, Asiri kâr gütmez. Elini kötülükten çeker, Iki kisi arasinda dogrulukla yargilar.
\par 9 Kurallarimi izler, Ilkelerimi özenle uygular. Iste böyle biri dogru kisidir. O yasayacaktir. Egemen RAB böyle diyor.
\par 10 "Diyelim ki, bu adamin zorba, kan döken, Kardesine bunlardan birini yapan bir oglu var.
\par 11 Babasi bunlardan hiçbirini yapmazken, Ogul daglarda putlara sunulan kurbandan yer, Komsusunun karisini kirletir.
\par 12 Düsküne, yoksula haksizlik eder, Soygunculuk eder, Rehini geri vermez. Putlara bel baglar, Igrenç seyler yapar.
\par 13 Faizle para verir, asiri kâr güder. Böyle biri yasayacak mi? Hayir, yasamayacak! Bütün bu igrençlikleri yapmistir, öldürülecektir. Onun kanindan kendisi sorumlu olacaktir.
\par 14 "Diyelim ki, bu ogulun da bir oglu olur ve babasinin isledigi bütün günahlari görür, Ama hiçbirini yapmaz;
\par 15 Daglarda putlara sunulan kurbandan yemez, Israil halkinin putlarina bel baglamaz, Komsusunun karisini kirletmez;
\par 16 Kimseye haksizlik etmez, Rehin almaz, Soygunculuk etmez, Aç olana ekmegini verir, Çiplagi giydirir.
\par 17 Böyle biri elini kötülükten*fi* çeker, Faiz almaz, asiri kâr gütmez, Kurallarimi izler, Ilkelerimi uygularsa, Babasinin günahi yüzünden ölmeyecek, Kesinlikle yasayacaktir.
\par 18 Ama babasi kendi günahi yüzünden ölecektir. Çünkü zorbalik etti, kardesini soydu, Halki arasinda iyi olmayani yapti.
\par 19 "Ama siz, 'Ogul neden babasinin isledigi suçlardan sorumlu tutulmasin? dersiniz. Bu ogul adil ve dogru olani yapmis, bütün kurallarimi dikkatle izlemistir. Böyle biri kesinlikle yasayacaktir.
\par 20 Ölecek olan günah isleyen kisidir. Ogul babasinin suçundan sorumlu tutulamaz, baba da oglunun suçundan sorumlu tutulamaz. Dogru kisi dogrulugunun, kötü kisi kötülügünün karsiligini alacaktir.
\par 21 "Kötü kisi isledigi bütün günahlardan döner, buyruklarima uyar, adil ve dogru olani yaparsa, kesinlikle yasayacak, ölmeyecektir.
\par 22 Isledigi günahlardan hiçbiri ona karsi anilmayacaktir. Dogrulugu sayesinde yasayacaktir.
\par 23 Ben kötü kisinin ölümünden sevinç duymam, ancak kötü kisinin kötü yollarindan dönüp yasamasindan sevinç duyarim. Egemen RAB böyle diyor.
\par 24 "Dogru kisi dogrulugundan döner, günah isler, kötü kisinin yaptigi bütün igrenç seyleri yaparsa, yasayacak mi? Onun yaptigi dogru islerin hiçbiri anilmayacaktir. Sadakatsizligi yüzünden suçludur, günahlari yüzünden ölecektir.
\par 25 "Siz yine de, 'Rab'bin yolu dogru degil diyorsunuz. Ey Israil halki, dinle: Benim yolum mu dogru degil? Dogru olmayan sizin yollariniz degil mi?
\par 26 Dogru kisi dogrulugundan döner de kötülük yaparsa, bu yüzden ölecek. Evet, isledigi günah yüzünden ölecektir.
\par 27 Ama kötü kisi, yaptigi kötülükten döner, adil ve dogru olani yaparsa, canini kurtaracaktir.
\par 28 Çünkü isyanlarinin farkina variyor ve onlardan dönüyor. Böyle biri kesinlikle yasayacak, ölmeyecektir.
\par 29 Öyleyken, Israil halki, 'Rab'bin yolu dogru degil diyor. Ey Israil halki, benim yollarim mi dogru degil? Dogru olmayan sizin yollariniz degil mi?
\par 30 "Bu yüzden, ey Israil halki, sizleri, her birinizi yolunuza göre yargilayacagim. Egemen RAB böyle diyor. Dönün! Isyanlarinizdan dönün! Günahin sizi yikima sürüklemesine izin vermeyin.
\par 31 Isyanlarinizi kendinizden uzaklastirin. Yeni bir yürek, yeni bir ruh edinin. Neden öleceksin, ey Israil halki?
\par 32 Çünkü ben kimsenin ölümünden sevinç duymam. Egemen RAB böyle diyor. Öyleyse günahinizdan dönün de yasayin!"

\chapter{19}

\par 1 "Sen Israil önderleri için su agiti yak
\par 2 ve de ki, "'Annen neydi? Aslanlar arasinda disi bir aslan! Genç aslanlar arasinda yatar, Yavrularini beslerdi.
\par 3 Büyüttügü yavrulardan biri Genç bir aslan oldu. Avini kapip parçalamayi ögrendi, Insan yiyen bir aslan oldu.
\par 4 Haberi uluslar arasinda duyuldu. Kurduklari tuzaga düstü, Onu çengellerle Misir'a sürüklediler.
\par 5 Disi aslan bekledi, umudunun bosa çiktigini görünce, Yavrularindan baska birini alip Genç bir aslan olarak yetistirdi.
\par 6 Yavru aslanlar arasinda dolasmaya basladi, Genç bir aslan oldu. Avini kapip parçalamayi ögrendi, Insan yiyen bir aslan oldu.
\par 7 Onlarin kalelerini yikti, Kentlerini viraneye çevirdi. Ülkede yasayan herkes Onun kükreyisinden dehsete düstü.
\par 8 Çevredeki uluslar üzerine geldiler, Aglarini gerdiler, Onu tuzaga düsürdüler.
\par 9 Çengel takip onu kafese koydular Ve Babil Krali'na götürdüler. Israil daglarinda kükreyisi bir daha duyulmasin diye Onu gözetim altinda tuttular.
\par 10 "'Annen su kiyisindaki baginda Dikilmis bir asma gibiydi. Bol su sayesinde dal budak saldi, Ürün verdi.
\par 11 Dallari kral asasi olacak kadar güçlendi. Asma boy atti, Bulutlara dek yükseldi. Yüksekligi ve dallarinin çoklugu Herkesçe görüldü.
\par 12 Ama onu öfkeyle kökünden söküp yere attilar. Dogu rüzgari ürününü kuruttu. Güçlü dallari koparilip kurudu, Ates onlari yakip yok etti.
\par 13 Simdi çöle, Kurak, susuz bir yere dikildi.
\par 14 Gövdesi ates aldi, Filizini, ürününü yakip yok etti. Kral asasi olacak kadar güçlü dali kalmadi. Bu bir agittir ve agit olarak kalacaktir."

\chapter{20}

\par 1 Sürgünlügümüzün yedinci yili, besinci ayin* onuncu günü, Israil ileri gelenlerinden bazi kisiler RAB'be danismak için gelip önüme oturdular.
\par 2 RAB o sirada bana seslendi:
\par 3 "Insanoglu, Israil ileri gelenlerine de ki, 'Egemen RAB söyle diyor: Bana danismaya mi geldiniz? Varligim hakki için diyor Egemen RAB, bana danismaniza izin vermeyecegim.
\par 4 "Onlari yargilayacak misin? Ey insanoglu, onlari yargilayacak misin? Öyleyse onlara atalarinin igrenç uygulamalarini animsat.
\par 5 Onlara de ki, 'Egemen RAB söyle diyor: Israil'i seçtigim gün Yakup soyuna ant içtim ve kendimi Misir'da onlara açikladim. Ant içerek, Tanriniz RAB benim dedim.
\par 6 O gün, onlari Misir'dan çikaracagima, kendileri için seçtigim en güzel ülkeye, süt ve bal akan ülkeye götürecegime söz verdim.
\par 7 Onlara, herkes bel bagladigi igrenç putlari atsin, Misir putlariyla kendinizi kirletmeyin, Tanriniz RAB benim dedim.
\par 8 "'Ne var ki, bana karsi geldiler, beni dinlemek istemediler. Bel bagladiklari igrenç putlari hiçbiri atmadi, Misir putlarini da birakmadilar. Bu yüzden Misir'da öfkemi onlarin üzerine yagdiracagimi, kizginligimi dökecegimi söyledim.
\par 9 Ama aralarinda yasadiklari uluslarin gözünde adima leke gelmesin diye bunu yapmadim. Bu uluslarin gözü önünde Israilliler'i Misir'dan çikararak kendimi onlara açiklamistim.
\par 10 Bu yüzden Israilliler'i Misir'dan çikarip çöle götürdüm.
\par 11 Uygulayan kisiye yasam veren kurallarimi onlara verdim, ilkelerimi tanittim.
\par 12 Kendilerini kutsal kilanin ben RAB oldugumu anlasinlar diye aramizda bir belirti olarak Sabat* günlerimi de onlara verdim.
\par 13 "'Böyleyken Israil halki çölde bana baskaldirdi. Uygulayan kisiye yasam veren kurallarimi izlemediler, ilkelerimi reddettiler. Sabat günlerimi de hiçe saydilar. Bu yüzden çölde öfkemi üzerlerine yagdirip onlari yok edecegimi söyledim.
\par 14 Ama Israilliler'i Misir'dan çikardigimi gören uluslarin gözünde adima leke gelmesin diye bunu yapmadim.
\par 15 Ben de kendilerine verdigim en güzel ülkeye, süt ve bal akan ülkeye onlari götürmeyecegime çölde ant içtim.
\par 16 Çünkü ilkelerimi reddettiler, kurallarimi izlemediler, Sabat günlerimi hiçe saydilar. Yürekleri putlarina bagliydi.
\par 17 Yine de onlara acidim, onlari yok etmedim, çölde islerine son vermedim.
\par 18 Çölde çocuklarina atalarinizin kurallarini izlemeyin, ilkelerine göre yasamayin, putlariyla kendinizi kirletmeyin dedim.
\par 19 Ben Tanriniz RAB'bim, benim kurallarimi izleyin, benim ilkelerim uyarinca yasayin.
\par 20 20 Aramizda bir belirti olsun diye Sabat günlerimi kutsal sayin. O zaman benim Tanriniz RAB oldugumu anlayacaksiniz dedim.
\par 21 "'Ne var ki, çocuklar bana karsi geldiler. Kurallarimi izlemediler. Uygulayan kisiye yasam veren ilkelerim uyarinca dikkatle yasamadilar. Sabat günlerimi hiçe saydilar. Bu yüzden çölde öfkemi üzerlerine yagdiracagimi, kizginligimi dökecegimi söyledim.
\par 22 Ama elimi geri çektim, Israilliler'i Misir'dan çikardigimi gören uluslarin gözünde adima leke gelmesin diye bunu yapmadim.
\par 23 Onlari uluslarin arasina dagitacagima, baska ülkelere gönderecegime çölde ant içtim.
\par 24 Çünkü ilkelerimi izlemediler, kurallarimi reddettiler. Sabat günlerimi hiçe saydilar, gözlerini atalarinin putlarina diktiler.
\par 25 Ben de onlara iyi olmayan kurallar, yasam vermeyen ilkeler verdim.
\par 26 Her ilk dogan çocugu ateste kurban ederek sunduklari sunularla kendilerini kirletmelerine izin verdim. Öyle ki, onlari dehsete düsüreyim de benim RAB oldugumu anlasinlar.
\par 27 "Bu nedenle, ey insanoglu, Israil halkina de ki, 'Egemen RAB söyle diyor: Atalariniz yine ihanet etmekle bana küfretmis oldular.
\par 28 Kendilerine vermeye ant içtigim ülkeye onlari getirdigimde, gördükleri her yüksek tepede, sik yaprakli her agacin altinda kurbanlarini kestiler. Beni öfkelendiren sunularini, güzel kokulu sunulariyla dökmelik sunularini orada sundular.
\par 29 Onlara gittikleri bu puta tapilan yerin ne oldugunu sordum." Orasi bugün de Bama adiyla aniliyor.
\par 30 "Bu nedenle Israil halkina de ki, 'Egemen RAB söyle diyor: Atalariniz gibi siz de kendinizi kirletecek misiniz? Onlarin putlarina gönül verecek misiniz?
\par 31 Simdiye dek ogullarinizi ateste kurban edip sunularinizi sunmakla, putlarinizla kendinizi kirlettiniz. Öyleyken gelip bana danismaniza izin verir miyim, ey Israil halki? Varligim hakki için diyor Egemen RAB, bana danismaniza izin vermeyecegim.
\par 32 "'Siz agaca, tasa tapan öteki uluslar gibi, dünyadaki öbür halklar gibi olmak istiyoruz diyorsunuz. Ama bu düsündükleriniz hiçbir zaman gerçeklesmeyecek.
\par 33 Varligim hakki için diyor Egemen RAB, sizi güçlü ve kudretli elle, siddetli öfkeyle yönetecegim.
\par 34 Güçlü ve kudretli elle, siddetli öfkeyle sizi uluslar arasindan çikaracak, dagilmis oldugunuz ülkelerden toplayacagim.
\par 35 Sizi uluslarin çölüne getirecek, orada yüz yüze yargilayacagim.
\par 36 Atalarinizi Misir Çölü'nde nasil yargiladiysam, sizi de öyle yargilayacagim. Egemen RAB böyle diyor.
\par 37 Sizi yoklayip antlasmama bagli kalmanizi saglayacagim.
\par 38 Aranizda bana karsi gelenlerle baskaldiranlari ayiracagim. Onlari yasadiklari ülkelerden çikaracagim. Ama Israil ülkesine girmeyecekler. O zaman benim RAB oldugumu anlayacaksiniz.
\par 39 "'Ey Israil halki, Egemen RAB söyle diyor: Her biriniz gidip putlariniza tapinin! Ama sonra beni dinleyeceksiniz ve armaganlarinizla, putlarinizla bir daha kutsal adimi kirletmeyeceksiniz.
\par 40 Çünkü kutsal dagimda, Israil'in yüksek daginda, diyor Egemen RAB, bütün Israil halki orada, ülkede bana kulluk edecek. Orada onlari kabul edecegim. Orada sunularinizi, seçme armaganlarinizi, bütün kutsal adaklarinizi isteyecegim.
\par 41 Sizi uluslarin arasindan çikarip dagilmis oldugunuz ülkelerden topladigimda, beni hosnut eden bir koku gibi kabul edecegim. Uluslarin gözü önünde aranizda kutsalligimi gösterecegim.
\par 42 Sizleri atalariniza vermeye ant içtigim ülkeye, Israil ülkesine getirdigimde, benim RAB oldugumu anlayacaksiniz.
\par 43 Bütün yaptiklarinizi, kendinizi kirlettiginiz bütün uygulamalari orada animsayacak, yaptiginiz kötülüklerden ötürü kendinizden tiksineceksiniz.
\par 44 Ey Israil halki, kötü yollariniza, yozlasmis uygulamalariniza göre degil, adim ugruna sizinle ilgilendigimde, benim RAB oldugumu anlayacaksiniz. Egemen RAB böyle diyor."
\par 45 RAB bana söyle seslendi:
\par 46 "Insanoglu, yüzünü güneye çevir, güneye seslen, Negev Ormani'na karsi peygamberlik et.
\par 47 Negev Ormani'na de ki, 'RAB'bin sözüne kulak ver. Egemen RAB söyle diyor: Senin içinde ates tutusturacagim. Ates bütün agaçlarini -yesil agaci da kuru agaci da- yiyip bitirecek. Tutusan alev söndürülemeyecek. Güneyden kuzeye, her yüz atesin sicagindan kavrulacak.
\par 48 Atesi tutusturanin ben RAB oldugumu herkes görecek, ates söndürülmeyecek."
\par 49 Bunun üzerine, "Ah, ey Egemen RAB!" dedim, "Onlar benim için, 'Simgesel öyküler anlatan adam degil mi bu? diyorlar."

\chapter{21}

\par 1 RAB bana söyle seslendi:
\par 2 "Ey insanoglu, yüzünü Yerusalim'e çevir, kutsal yerlerine karsi konus, Israil ülkesine karsi peygamberlik et.
\par 3 Ona de ki, 'RAB söyle diyor: Ben sana karsiyim! Kilicimi kinindan çikaracak, içindeki dogru kisiyi de kötü kisiyi de kesip yok edecegim.
\par 4 Dogru kisiyi de kötü kisiyi de kesip yok etmek için kilicim kinindan çikacak ve güneyden kuzeye herkese karsi olacak.
\par 5 Böylece herkes kilicini kinindan çikaranin ben RAB oldugumu anlayacak. Onu bir daha yerine koymayacagim.
\par 6 "Sen, ey insanoglu, inle! Onlarin gözü önünde ezik bir yürekle aci aci inle!
\par 7 Sana, 'Neden böyle inliyorsun? diye sorduklarinda, 'Yakinda duyulacak haberden ötürü diye yanitlayacaksin. 'Her yürek eriyecek, her el gevseyecek, her ruh baygin düsecek, her dizin bagi çözülecek. Evet, haber duyulacak! Bu kesinlikle yerine gelecek. Egemen RAB böyle diyor."
\par 8 RAB bana söyle seslendi:
\par 9 "Insanoglu, peygamberlik et ve de ki, 'Rab söyle diyor: "'Kiliç, kiliç, Bilendi, cilalandi.
\par 10 Öldürmek için bilendi, Simsek gibi çaksin diye cilalandi. Nasil sevinebiliriz? Kiliç oglumun asasini siradan bir sopa gibi küçümsedi.
\par 11 Kiliç kullanilmak için Cilalanmaya verildi; Öldürenin eline verilsin diye bilenip cilalandi.
\par 12 Insanoglu, bagir, haykir! Çünkü bu kiliç halkima karsi; Bütün Israil önderlerine karsi. Onlar halkimla birlikte kilica teslim edildiler. Bunun için bagrini döv.
\par 13 "'Deneme kuskusuz gelecek. Kilicin küçümsedigi asa varligini sürdüremezse ne olur? Böyle diyor Egemen RAB.
\par 14 "Sen, ey insanoglu, peygamberlik et, el çirp. Birak kiliç iki, üç kez vursun. Bu öldüren bir kiliçtir, Çok sayida insan kiran, Insani her yandan saran kiliçtir.
\par 15 Yürekleri erisin, Tökezleyip düsenler çok olsun diye Bütün kapilarinda öldürmek için Görevlendirdim kilici. Ah, kiliç simsek gibi parladi, Öldürmek için bilendi.
\par 16 Ey kiliç, saga, sonra sola savrul, Agzin nereye dönerse, oraya savrul!
\par 17 Ben de elimi çirpacagim Ve öfkem dinecek. Bunu ben RAB söylüyorum."
\par 18 RAB bana söyle seslendi:
\par 19 "Insanoglu, Babil Krali'nin kilici gelsin diye iki yol belirle; ikisi de ayni ülkeden baslamali. Kent yolunun basladigi yere bir isaret koy.
\par 20 Ammonlular'in Rabba Kenti'ne ya da Yahuda'ya ve surlarla çevrili Yerusalim'e ilerlesin diye kiliç için yol belirle.
\par 21 Çünkü Babil Krali iki yolun ayrildigi, yollarin çatallastigi yerde fala bakmak için duracak. Oklari silkeleyecek, aile putlarina danisacak, kurban edilen bir hayvanin cigerine bakacak.
\par 22 Kütük yerlestirmek, öldür buyrugunu vermek, savas naralari atmak, kapilara kütük yerlestirmek, toprak rampalar olusturmak, kusatma duvarlari yapmak için sag elinde Yerusalim'i gösteren ok olacak.
\par 23 Onunla ant içerek antlasma yapanlar fala yanlis bakildigini sanacak. Ama kral suçlarini animsatip onlari tutsak alacak.
\par 24 "Bundan ötürü Egemen RAB söyle diyor: 'Madem suçlarinizi, isyanlarinizi animsattirdiniz, bütün uygulamalarinizda günahlarinizi açiga çikardiniz, madem bütün bunlari yaptiniz, siz de tutsak alinip götürüleceksiniz.
\par 25 "'Sen, ey saygisiz, kötü Israil önderi, günün yaklasti, sonunda yargi günün geldi.
\par 26 Egemen RAB söyle diyor: Sarigi çikar, taci kaldir. Artik eskisi gibi olmayacak. Alçakgönüllü yükseltilecek, gururlu alçaltilacak.
\par 27 Yikim! Yikim! Kenti yerle bir edecegim! Hak sahibi gelinceye dek onarilmayacak. Kenti ona verecegim.
\par 28 "Sen, ey insanoglu, peygamberlik et ve de ki, 'Asagilayici sözler söyleyen Ammonlular için Egemen RAB söyle diyor: "'Kiliç, kiliç, Öldürmek için kinindan çekilmis, Yok etmek için, Simsek gibi parlamak için cilalanmis!
\par 29 Size iliskin görümler aldaticidir, Açilan fal yalandir. Öldürülecek kötülerin enseleri üzerine Yerlestirileceksin, ey kiliç! Onlarin günü yaklasti, Sonunda yargi günleri geldi.
\par 30 Kiliç kinina koyulsun! Yaratildiginiz yerde, Atalarinizin ülkesinde Yargilayacagim sizi.
\par 31 Öfkemi üzerinize dökecegim, Kizginligimi üzerinize üfleyecegim; Acimasiz adamlarin, Yakip yok etmekte usta kisilerin eline Teslim edecegim sizi.
\par 32 Atese yakit olacaksiniz, Kaniniz ülkenizin ortasinda dökülecek, Bir daha anilmayacaksiniz. Çünkü bunu ben RAB söylüyorum."

\chapter{22}

\par 1 RAB bana söyle seslendi:
\par 2 "Insanoglu, Yerusalim'i yargilayacak misin? Kan döken bu kenti yargilayacak misin? Öyleyse bütün igrenç uygulamalarini ona bildir.
\par 3 Söyle diyeceksin: 'Egemen RAB diyor ki: Ey kendi içinde kan dökerek yikimini hazirlayan, putlar yaparak kendini kirleten kent!
\par 4 Döktügün kan yüzünden suçlu bulundun, yaptigin putlarla kirlendin. Böylece günlerin yaklasti, yillarinin sonuna ulastin. Bu yüzden seni uluslara alay konusu edecegim, bütün ülkelerin gözünde seni gülünç duruma düsürecegim.
\par 5 Ey adi kötüye çikmis, kargasa dolu kent, yakindakiler de uzaktakiler de seninle alay edecekler.
\par 6 "'Iste içindeki her Israil önderi yetkisini kullanarak kan döküyor.
\par 7 Senin içinde anneye, babaya kötü davrandilar, yabanciya baski yaptilar, öksüze, dul kadina haksizlik ettiler.
\par 8 Benim kutsal esyalarima saygisizlik ettin, Sabat* günlerimi hiçe saydin.
\par 9 Kan dökmek için iftira edenler, daglarda putlara kurban edilen hayvanlari yiyenler, kendilerini sehvete kaptiranlar senin içinde yasiyor.
\par 10 Babalarinin karilariyla yatanlar, âdet gören dinsel açidan kirli kadinlarla cinsel iliski kuranlar senin içinde yasiyor.
\par 11 Senin içinde kimi komsusunun karisiyla igrenç seyler yapti; kimi utanmadan gelinini kirletti; kimi öz kizkardesiyle iliski kurdu.
\par 12 Senin içinde kan dökmek için rüsvet aldilar. Faiz aldin, tefecilik yaptin, zorbalikla komsularindan haksiz kazanç sagladin. Beni unuttun. Egemen RAB böyle diyor.
\par 13 "'Edindiginiz haksiz kazançtan, içinizde döktügünüz kandan ötürü ellerimi birbirine vuracagim.
\par 14 Sizinle ugrasacagim gün cesaretiniz kalacak mi? Elleriniz güçlü olabilecek mi? Bunu ben RAB söylüyorum ve dedigimi yapacagim.
\par 15 Sizi uluslar arasina dagitip ülkelere sürecegim. Sizdeki ruhsal kirlilige son verecegim.
\par 16 Uluslarin gözünde asagilanacak ve benim RAB oldugumu anlayacaksiniz."
\par 17 RAB bana söyle seslendi:
\par 18 "Insanoglu, Israil halki benim için cüruf gibi oldu. Hepsi potada tunç*, kalay, demir, kursundur; gümüsün cürufudur.
\par 19 Bundan ötürü Egemen RAB söyle diyor: 'Hepiniz cüruf gibi oldugunuz için sizi Yerusalim'in ortasina toplayacagim.
\par 20 Eritmek için atesi üfleyerek gümüsü, tuncu, demiri, kursunu, kalayi nasil potaya atiyorlarsa, ben de öfkemle, kizginligimla sizi toplayacak, kentin ortasina koyup eritecegim.
\par 21 Sizi toplayacak, öfkemin atesini üzerinize üfleyecegim; siz de kentin içinde eriyip yok olacaksiniz.
\par 22 Gümüs potada nasil erirse, siz de kentin içinde öyle eriyeceksiniz. O zaman üzerinize kizginligini dökenin ben, RAB oldugumu anlayacaksiniz."
\par 23 RAB bana söyle seslendi:
\par 24 "Insanoglu, ülkeye de ki, 'Sen öfke günü temizlenmemis, üzerine yagmur yagmamis bir ülkesin.
\par 25 Önderleri*fl* kükreyen, avini parçalayan aslan gibi orada düzen kurdular. Canlara kiydilar, hazineler, degerli nesneler aldilar, birçok kadini dul biraktilar.
\par 26 Kâhinleri yasami hiçe saydilar, kutsal esyalarimi kirlettiler, kutsalla bayagi arasindaki ayrimi yapmadilar, kirliyle temiz arasindaki farki ögretmediler, Sabat günlerimden gözlerini çevirdiler. Kutsalligimi önemsemediler.
\par 27 Yöneticileri avini parçalayan kurt gibidir. Haksiz kazanç elde etmek için kan döküyor, canlara kiyiyorlar.
\par 28 Peygamberleri uydurma görümlerle, yalan fal açarak bu suçlari gizlediler; ben RAB konusmadigim halde, 'Egemen RAB söyle diyor diyorlar.
\par 29 Ülke halki baski uyguladi, soygunculuk etti. Düsküne, yoksula baski yapti, yabanciya haksiz yere kötü davrandi.
\par 30 "Içlerinde duvari örecek, gedikte durup önümde ülkeyi savunacak, onu yerle bir etmemi engelleyecek bir adam aradim, ama hiç kimseyi bulmadim.
\par 31 Bunun için öfkemi üzerlerine bosaltacak, kizginligimla onlari yakip yok edecegim. Yaptiklarini kendi baslarina getirecegim." Egemen RAB böyle diyor. Iki Günahli Kizkardesin Simgesel Öyküsü

\chapter{23}

\par 1 RAB bana söyle seslendi:
\par 2 "Insanoglu, bir anneden dogma iki kadin vardi.
\par 3 Gençliklerinde Misir'da fahiselik ettiler. Memeleri orada oksandi, erdenliklerini orada yitirdiler.
\par 4 Büyügünün adi Ohola, küçügünün Oholiva'ydi. Benim oldular; ogullar, kizlar dogurdular. Ohola Samiriye'dir, Oholiva da Yerusalim.
\par 5 "Ohola benimken fahiselik etti. Oynaslari olan Asurlular'a gönül verdi. Hepsi de genç, yakisikli, lacivertler kusanmis savasçilar, valiler, komutanlar, atli askerlerdi.
\par 7 Asurlular'in en seçkin adamlarina fahise olarak kendini verdi. Gönül verdigi bu kisilerin putlarina baglanarak kendini kirletti.
\par 8 Misir'da basladigi fahiseligi birakmadi. Gençken onunla yattilar, erdenligini bozdular, sehvetlerini onun üzerine bosalttilar.
\par 9 "Bu nedenle onu oynaslarinin, gönül verdigi Asurlular'in eline teslim ettim.
\par 10 Çiplakligini açtilar, ogullarini, kizlarini aldilar, onu kiliçla öldürdüler. Kendisine verilen cezadan ötürü kadinlar arasinda adi kötüye çikti.
\par 11 "Kizkardesi Oholiva bunu gördü, ama sehveti ve fahiselikleri kizkardesininkinden daha utanç vericiydi.
\par 12 O da hepsi de genç, yakisikli Asurlular'a -valilere, komutanlara, iyi donanmis savasçilara, atlilara- gönül verdi.
\par 13 Kendisini ne kadar kirlettigini gördüm. Ikisi de ayni yolu izlediler.
\par 14 "Oholiva fahiseliklerini giderek artirdi. Duvara oyulmus insan resimlerini -bellerine kusak, baslarina genis sarik baglamis kirmizi renkli Kildani* resimlerini- gördü. Hepsi kökeni Kildan ülkesine dayanan Babil subaylarina benziyordu.
\par 16 Oholiva görür görmez onlara gönül verdi, Kildan ülkesine ulaklar gönderdi.
\par 17 Bunun üzerine Babilliler onunla yatakta sevismek üzere geldiler, zina ederek onu kirlettiler. Onu öyle kirlettiler ki, sonunda hepsinden tiksinip yüzünü çevirdi.
\par 18 Fahiseliklerini sergileyip çiplakligini açinca kizkardesinden tiksinerek yüzümü çevirdigim gibi, ondan da tiksinerek yüzümü çevirdim.
\par 19 Gençliginde Misir'da yaptigi fahiselikleri animsayarak, fahiseligini daha da artirdi.
\par 20 Erkeklik organlari eseginkine, menileri aygirinkine benzeyen oynaslarina gönül verdi.
\par 21 Öyle ki, Misir'da gençligindeki sehvet düskünlügünü özledin. Memelerin orada oksanmis, erdenligini orada yitirmistin.
\par 22 "Bundan ötürü, ey Oholiva, Egemen RAB söyle diyor: Tiksindigin oynaslarini sana karsi kiskirtacagim. Onlari her yandan sana karsi ayaklandiracagim.
\par 23 Babilliler'i, bütün Kildaniler'i, Pekotlular'i, Soalilar'i, Koalilar'i, onlarla birlikte bütün Asurlular'i, yakisikli gençleri -valileri, komutanlari, subaylari, ünlü adamlari, atlilari- sana karsi ayaklandiracagim.
\par 24 Silahlarla, savas ve yük arabalariyla, çok uluslu bir orduyla sana saldiracaklar. Seni her yandan büyük, küçük kalkanlarla, migferlerle saracaklar. Cezalandirmalari için seni onlarin eline teslim edecegim. Seni kendi kurallarina göre yargilayacaklar.
\par 25 Öfkemi sana yöneltecegim, onlarin sana kizginlikla davranmalarini saglayacagim. Burnunu, kulaklarini kesecekler. Sag kalanlari kiliçla öldürecekler. Ogullarini, kizlarini alacaklar, sag kalanlari ates yakip yok edecek.
\par 26 Üzerindeki giysiyi soyacak, güzel mücevherlerini alacaklar.
\par 27 Misir'da yaptigin ahlaksizliklara, fahiseliklere son verecegim. Böyle seylere özlem duymayacak, bir daha Misir'i animsamayacaksin.
\par 28 "Egemen RAB söyle diyor: Seni nefret ettigin, tiksindigin adamlarin eline teslim edecegim.
\par 29 Sana düsman gibi davranacak, emeginin bütün ürününü alacaklar. Seni çirilçiplak birakacaklar. Böylece utanç verici fahiseliklerin açiga çikacak. Bütün bunlar sehvet düskünlügünden, fahiseligin yüzünden basina geldi. Çünkü uluslarla fahiselik ettin, onlarin putlariyla kendini kirlettin.
\par 31 Kizkardesinin yolunu izledin. Bu nedenle, sana onun kâsesinden* içirecegim.
\par 32 "Egemen RAB söyle diyor: Kizkardesinin kâsesinden içeceksin, O derin ve genistir; Sana gülecek, seninle alay edecekler, Dopdolu bir kâse.
\par 33 Sarhos olacak, umutsuzluga bogulacaksin, Kizkardesin Samiriye'nin kâsesi Yikim, perisanlik kâsesidir.
\par 34 Ondan içecek, tüketeceksin; Parçalarini kemirecek Ve gögsünü paralayacaksin. Bunu ben söylüyorum diyor Egemen RAB.
\par 35 "Bundan ötürü Egemen RAB söyle diyor: Madem beni unuttun, bana sirt çevirdin, sen de ahlaksizliginin, fahiseliginin cezasini yükleneceksin."
\par 36 RAB bana seslendi: "Insanoglu, Ohola'yla Oholiva'yi yargilayacak misin? Öyleyse onlara igrenç uygulamalarini bildir.
\par 37 Çünkü fahiselik ettiler, kan döktüler. Putlariyla fahiselik ettiler; bana dogurduklari çocuklari yiyecek olarak putlarina sundular.
\par 38 Bununla kalmayarak, sunlari da yaptilar: Çocuklarini putlara sunduklari gün tapinagimi kirlettiler, Sabat* günlerimi hiçe saydilar. Ayni gün tapinagima girip onu kirlettiler. Iste tapinagimda bunlari yaptilar.
\par 40 "Siz iki kizkardes uzaklarda yasayan adamlarin gelmesi için ulaklar gönderdiniz. Adamlar gelince, onlar için yikanip gözlerinize sürme çektiniz, mücevherlerinizi taktiniz.
\par 41 Sik bir divanin üzerine oturdunuz, önüne bir sofra kurup üzerine buhurumu, zeytinyagimi koydunuz.
\par 42 "Kaygisiz kalabaligin sesi yankilandi çevresinde. Düzeysiz bir yigin kalabalikla birlikte çölden Sabalilar getirildi. Iki kizkardesin koluna bilezikler taktilar, baslarina güzel bir taç koydular.
\par 43 Fahiselikten yipranmis kadin için, 'Birakin, fahise olarak kullansinlar onu. Çünkü öyledir dedim.
\par 44 Onunla yattilar. Fahiseye gider gibi, bu iki ahlaksiz kadinin -Ohola'yla Oholiva'nin- yanina gittiler.
\par 45 Ama dogru adamlar zina eden, kan döken kadinlara verilen cezayla onlari cezalandiracaklar. Çünkü bu iki kadin fahiselik ettiler, elleri kanlidir.
\par 46 "Egemen RAB söyle diyor: Onlari dehsete düsürecek, mallarini yagmalayacak bir kalabalik salacagim üzerlerine.
\par 47 Onlari tasa tutacak, kiliçlariyla parçalayacaklar; ogullarini, kizlarini öldürecek, evlerini atese verecekler.
\par 48 "Ülkede ahlaksizliga son verecegim. Öyle ki, bütün kadinlar için bir uyari olsun bu, sizin yaptiginiz ahlaksizligi yapmasinlar.
\par 49 Yaptiginiz fahiseliklerin karsiligini ödeyecek, putlara tapinarak islediginiz günahlarin cezasini çekeceksiniz. Böylece benim Egemen RAB oldugumu anlayacaksiniz."

\chapter{24}

\par 1 Sürgünlügümüzün dokuzuncu yili, onuncu ayin* onuncu günü RAB bana söyle seslendi:
\par 2 "Ey insanoglu, bu günü, bu günün tarihini tam olarak yaz. Çünkü Babil Krali tam bu gün Yerusalim'i kusatmaya basladi.
\par 3 Bu asi halka simgesel bir öykü anlat. Onlara de ki, 'Egemen RAB söyle diyor: "'Kazani atese koyun, atese koyun, Içine su doldurun.
\par 4 Etin parçalarini da koyun, Etin en iyi parçalarini, Budu ve dösü. Seçme kemikleri de doldurun.
\par 5 Sürünün en iyilerini seçin, Kazanin altina odun yigin, Birakin su kaynasin, Kemikler pissin.
\par 6 Egemen RAB diyor ki, Kan döken o kentin vay basina! Pas tutmus, Pasindan temizlenmemis o kazanin vay basina! Kazandan eti kura çekmeden Parça parça çikarin.
\par 7 Çünkü döktügü kan ortalikta duruyor; Çiplak bir kayanin üzerine döktü kani, Toprakla örtülebilecek bir yere dökmedi.
\par 8 Öfkeyi alevlendirmek, Öç almak için, Onun kanini çiplak bir kayanin üzerine döktüm ki, örtülemesin.
\par 9 Egemen RAB söyle diyor: Kan döken kentin vay basina! Ben kendim ates için odun yigacagim.
\par 10 Odunlari yig! Atesi tutustur! Eti iyice pisir! Baharati kat! Kemikler kavrulsun!
\par 11 Sonra bos kazani Ates közlerinin üzerine koy. Kizsin, bakiri yansin, Içindeki pislik erisin, Pasi yok olsun.
\par 12 Bütün emekler bosa çikti, Kazanin kalin pasi çikmiyor. Ates bile pasi temizlemiyor.
\par 13 Yaptigin ahlaksizlik seni kirletti. Seni temizlemek istedim, Ama sen pisliginden temizlenmek istemedin. Sana karsi öfkem yatisincaya dek Pisliginden temizlenmeyeceksin.
\par 14 Bunu ben RAB söylüyorum. Harekete geçmenin zamani geldi, Esirgemeyecegim, Acimayacak, pisman olmayacagim. Yollarina ve yaptiklarina göre yargilanacaksin. Böyle diyor Egemen RAB."
\par 15 RAB bana söyle seslendi:
\par 16 "Insanoglu, en çok sevdigin kisiyi bir vurusta senin elinden alacagim. Yas tutmayacak, aglamayacak, gözyasi dökmeyeceksin.
\par 17 Için için inle; ölüler için yas tutmayacaksin. Sarigin basinda, çarigin ayaklarinda kalsin; yüzünün alt kismini örtme, yas tutanlarin yiyecegini yeme."
\par 18 Sabah halka seslendim, aksam karim öldü. Ertesi sabah bana söyleneni yaptim.
\par 19 Halk bana, "Bu yaptiklarinin bizimle ilgisi ne? Bize açiklamayacak misin?" diye sordu.
\par 20 Bunun üzerine, "RAB bana söyle seslendi" dedim,
\par 21 "Israil halkina de ki, 'Egemen RAB söyle diyor: Övündügünüz güç kaynaginiz, gözünüzde degerli olan, yüreginizin üzerine titredigi tapinagimin kirletilmesine izin verecegim. Geride biraktiginiz ogullarinizla kizlariniz kiliçtan geçirilecek.
\par 22 Ben ne yaptiysam, siz de aynisini yapacaksiniz. Yüzünüzün alt kismini örtmeyeceksiniz, yas tutanlarin*fn* yiyecegini yemeyeceksiniz.
\par 23 Sariklariniz baslarinizda, çariklariniz ayaklarinizda olacak. Yas tutmayacak, aglamayacaksiniz. Ancak günahlarinizin içinde eriyip yok olacaksiniz, kendi aranizda inleyip duracaksiniz.
\par 24 Hezekiel sizin için bir belirti olacak; o ne yaptiysa, siz de aynisini yapacaksiniz. Bunlar olunca, benim Egemen RAB oldugumu anlayacaksiniz.
\par 25 "Övündükleri güç kaynagini, sevinçlerini, yüceliklerini, gözlerinde degerli olani, yüreklerinin diledigini, ogullariyla kizlarini onlardan aldigim gün, yikimdan kaçip kurtulan biri gelip sana haberleri bildirecek, ey insanoglu.
\par 27 O gün dilin çözülecek, kaçip kurtulanla konusacak, bir daha suskun olmayacaksin. O gün onlar için bir belirti olacaksin. O zaman benim RAB oldugumu anlayacaklar."

\chapter{25}

\par 1 RAB bana söyle seslendi:
\par 2 "Insanoglu, yüzünü Ammonlular'a çevir, onlara karsi peygamberlik et.
\par 3 Onlara de ki, 'Egemen RAB'bin sözünü dinleyin! Egemen RAB söyle diyor: Madem tapinagim kirletildigi, Israil ülkesi viraneye çevrildigi, Yahuda halki sürgüne gittigi zaman, Hah, hah! Diyerek alay ettiniz,
\par 4 ben de sizi miras olarak doguda yasayan halka teslim edecegim. Obalarini, çadirlarini ülkenizde kuracaklar; ürününüzü yiyecek, sütünüzü içecekler.
\par 5 Rabba Kenti'ni develer için otlak, Ammon ülkesini sürüler için agil yapacagim. O zaman benim RAB oldugumu anlayacaksiniz.
\par 6 Egemen RAB söyle diyor: Madem Israil'le alay ederek ellerinizi çirptiniz, ayaklarinizi yere vurdunuz, bütün yüreginizle sevindiniz,
\par 7 ben de size karsi elimi uzatacak, çapul mali olarak sizi uluslara teslim edecegim. Sizi halklar arasindan süpürüp atacak, ülkeler arasindan söküp çikaracak, yok edecegim. O zaman benim RAB oldugumu anlayacaksiniz."
\par 8 "Egemen RAB söyle diyor: 'Madem Moav ve Seir halki, Bakin, Yahuda halkinin öteki uluslardan farki yok, dedi,
\par 9 ben de Moav'in sinirini, ülkenin süsü olan sinir kentlerini, Beytyesimot, Baal-Meon ve Kiryatayim'i savunmasiz birakacagim.
\par 10 Ammonlular uluslar arasinda bir daha anilmasin diye Moav'i Ammonlular'la birlikte mülk olarak doguda yasayan halka verecegim.
\par 11 Böylece Moav'i cezalandiracagim. O zaman benim RAB oldugumu anlayacaklar."
\par 12 "Egemen RAB söyle diyor: 'Madem Edom Yahuda halkindan öç alarak büyük suç isledi,
\par 13 Egemen RAB söyle diyor: Ben de Edom'a karsi elimi uzatacak, insanlari da hayvanlari da yok edecek, ülkeyi viraneye çevirecegim. Teman'dan Dedan'a kadar Edomlular kiliçla vurulup yok olacaklar.
\par 14 Halkim Israil araciligiyla Edom'dan öç alacagim. Israilliler onlara öfkem, kizginligim uyarinca davranacak. Böylece Edomlular öcümü anlayacaklar. Egemen RAB böyle diyor."
\par 15 "Egemen RAB söyle diyor: 'Madem Filistliler Yahuda'ya acimasizca davrandilar, eskiden var olan düsmanliklariyla onu yerle bir ederek öç aldilar,
\par 16 Egemen RAB söyle diyor: Elimi Filistliler'e karsi uzatacagim, Keretliler'i söküp atacagim, kiyida yasayanlardan sag kalanlarini yok edecegim.
\par 17 Onlardan agir bir öç alacak, onlari öfkeyle paylayacagim. Kendilerinden öç alinca, benim RAB oldugumu anlayacaklar."

\chapter{26}

\par 1 Sürgünlügümüzün on birinci yili, ayin birinci günü RAB bana söyle seslendi:
\par 2 "Insanoglu, madem Sur Kenti, Yerusalim için, 'Oh, oh! Uluslarin kapisi olan kent yikildi, kapilari bana açildi. O viraneye döndü, ben zenginlesecegim dedi,
\par 3 Egemen RAB söyle diyor: Ey Sur, sana karsiyim! Deniz dalgalarini nasil kabartirsa, ben de uluslari senin üzerine öyle saldirtacagim.
\par 4 Sur'un duvarlarini yikacak, kulelerini yerle bir edecekler. Topragini kazip süpürecek, seni çiplak bir kayalik haline getirecegim.
\par 5 Sur denizin ortasinda, balikçilarin ag gerdikleri bir yer olacak. Egemen RAB böyle diyor. Uluslar Sur'u yagmalayacak,
\par 6 Sur'a bagli kiyi kentlerinde yasayanlari kiliçtan geçirecek. O zaman Surlular benim RAB oldugumu anlayacaklar.
\par 7 "Egemen RAB söyle diyor: Krallar krali Babil Krali Nebukadnessar'i atlarla, savas arabalariyla, atlilarla, büyük bir orduyla kuzeyden Sur'a getiriyorum.
\par 8 Sur'a bagli kiyi kentlerinde yasayanlari kiliçtan geçirecek, size karsi kusatma duvarlari, toprak rampalar yapacak, kalkanini size karsi kaldiracak.
\par 9 Duvarlarinizda gedik açmak için kütükler yerlestirecek, silahlariyla kulelerinizi yikacak.
\par 10 Sayisiz atinin çikardigi toz sizi örtecek. Duvarlarinda gedik açilmis bir kente girer gibi kent kapilarinizdan girdiginde, atlilarin, tekerleklerin, savas arabalarinin gürültüsünden duvarlariniz sarsilacak.
\par 11 Atlarinin tirnaklari bütün sokaklarinizi çigneyecek. Halkiniz kiliçtan geçirilecek, güçlü sütunlariniz devrilecek.
\par 12 Servetinizi alacak, mallarinizi yagmalayacaklar. Duvarlarinizi yikacak, güzel evlerinizi yerle bir edecekler. Taslarinizi, kerestenizi, topraginizi denize atacaklar.
\par 13 Okudugunuz gürültülü sarkilara son verecegim. Lirlerinizin sesi bir daha duyulmayacak.
\par 14 Sizi çiplak bir kayalik haline getirecegim, balikçilarin ag gerdikleri bir yer olacaksiniz. Bir daha kurulmayacaksiniz. Çünkü ben RAB söylüyorum. Egemen RAB böyle diyor.
\par 15 "Egemen RAB Sur'a söyle diyor: Yikiminin sesinden, yaralilarin iniltisinden, senin içinde yapilan kiyim yüzünden kiyi halklari titreyecek.
\par 16 Kiyida yasayan bütün önderler tahtlarindan inecek; kaftanlarini, islemeli giysilerini çikaracaklar. Dehset içinde yere oturup her an titreyerek baslarina gelenlere sasacaklar.
\par 17 Sonra senin için söyle bir agit yakacaklar: "'Nasil oldu da yikildin, Ey denizcilerin oturdugu ünlü kent! Sen ve sende oturanlar, Denizde güçlüydünüz. Dehset salmistiniz Orada yasayan herkese.
\par 18 Yikimin oldugu gün Kiyi halklari titreyecek, Orada yasayanlar Çöküsüne sasacaklar.
\par 19 "Egemen RAB söyle diyor: Issiz kalmis kentler gibi seni viran bir kent yaptigim, engin denizleri üzerine bosalttigim, derin sular seni örttügü zaman,
\par 20 ölüm çukuruna inenlerle birlikte seni eski zaman insanlarinin yanina indirecegim. Ölüm çukuruna inenlerle birlikte eski kalintilar arasina, yeryüzünün derinliklerine yerlestirecegim. Öyle ki, bir daha dönüp yasayanlar diyarinda yerini almayasin.
\par 21 Seni yilginliga düsürecegim, bu senin sonun olacak. Seni arayacaklar ama bulamayacaklar. Egemen RAB böyle diyor."

\chapter{27}

\par 1 RAB bana söyle seslendi:
\par 2 "Insanoglu, Sur Kenti için bir agit yak.
\par 3 Denizin kiyisinda kurulmus, kiyi halklariyla ticaret yapan Sur Kenti'ne de ki, 'Egemen RAB söyle diyor: "'Ey Sur, güzellikte kusursuzum dedin.
\par 4 Sinirlarin denizin bagrindaydi, Kurucularin güzelligini doruga ulastirdilar.
\par 5 Bütün kerestelerini Senir'in çam agaçlarindan yaptilar, Sana direk yapmak için Lübnan'dan sedir agaçlari aldilar.
\par 6 Küreklerini Basan meselerinden, Güverteni Kittim kiyilarindan getirilen Selvi agaçlarindan yaptilar, Fildisiyle süslediler.
\par 7 Misir'in islemeli ince keteninden yelkenin, Bayragin oldu senin. Güvertenin gölgeligi Elisa kiyilarinin Lacivert, mor kumasindandi.
\par 8 Kürekçilerin Saydali ve Arvatli'ydi, Gemicilerin, içindeki becerikli kisilerdi, ey Sur.
\par 9 Gemilerindeki gedikleri onaranlar Geval'in deneyimli, usta adamlariydi. Denizdeki bütün gemiler ve denizciler Mallarini degis tokus etmek için sana geldiler.
\par 10 Persli, Ludlu, Pûtlu askerler Ordunda hizmet etti. Kalkanlarini, migferlerini Duvarlarina astilar, Sana görkem kazandirdilar.
\par 11 Arvat'tan, Helek'ten gelen adamlar Çepeçevre duvarlarini korudular. Gammat'tan gelen adamlar Kulelerinde beklediler. Kalkanlarini duvarlarina astilar. Güzelligini doruga ulastirdilar.
\par 12 "'Tarsis seninle ticaret yapti, Sende her çesit mal vardi. Mallarina karsilik Sana gümüs, demir, kalay, kursun verdiler.
\par 13 Yâvan, Tuval, Mesek seninle ticaret yapti, Mallarina karsilik Sana köle ve tunç* kaplar verdiler.
\par 14 Beyttogarma halki Mallarina karsilik Sana at, savas ati, katir verdi.
\par 15 Rodos halki seninle ticaret yapti. Birçok kiyi halki senin müsterindi. Senden aldiklari mala karsilik Fildisi ve abanoz verdiler.
\par 16 Sende çok çesit ürün oldugundan, Edom seninle ticaret yapti. Mallarina karsilik Sana firuze, mor kumas, islemeli giysiler, Ince keten, mercan, yakut verdiler.
\par 17 Yahuda ve Israil seninle ticaret yapti. Mallarina karsilik Sana Minnit bugdayi, dari, bal, zeytinyagi, pelesenk verdiler.
\par 18 Ürünlerinin çesitliligi, malinin bollugundan ötürü Sam seninle ticaret yapti. Mallarina karsilik Sana Helbon sarabiyla Sahar yünü, Uzal'dan getirilmis sarap tekneleri verdi. Sana getirilen mallar arasinda Islenmis demir, tarçin, güzel kokulu kamis vardi.
\par 20 Dedan halki mallarina karsilik Sana eyerlik kumas verdi.
\par 21 Arabistan ve Kedar önderleri müsterindi, Mallarina karsilik Sana kuzu, koç, teke verdiler.
\par 22 Saba ve Raama tüccarlari seninle ticaret yapti, Mallarina karsilik Sana her çesit baharatin en iyisini, degerli taslar, altin verdiler.
\par 23 Harran, Kanne, Eden, Saba, Asur, Kilmat tüccarlari Seninle ticaret yapti.
\par 24 Pazarlarindaki mallara karsilik Güzel giysiler, lacivert kumas, islemeler, Sik dokunmus, iplerle sarilmis renkli halilar verdiler.
\par 25 Ticaret gemileri senin mallarini tasidi, Denizin bagrinda büyük yükle doldun.
\par 26 Kürekçilerin seni açik denizlere götürdü, Ama dogu rüzgari Denizin bagrinda parçaladi seni.
\par 27 Gemin kazaya ugrayacagi gün, Zenginligin, mallarin, ticari esyalarin, Gemicilerin, kilavuzlarin, kalafatçilarin, Seninle ticaret yapanlar, Askerlerin ve gemide olan herkes Denizin derinliklerine batacak.
\par 28 Gemicilerinin bagirisindan Kiyilar titreyecek.
\par 29 Kürekçiler gemilerini birakacak, Gemicilerle kilavuzlar kiyida duracak.
\par 30 Yüksek sesle haykirip Senin için aci aci aglayacaklar; Baslarina toprak serpecek, Külde yuvarlanacaklar.
\par 31 Senin yüzünden baslarini tiras edecek, Çul kusanacaklar. Senin için aci aci aglayacak, Yas tutacaklar.
\par 32 Aglayip yas tutarken, Senin için bir agit yakacaklar: Her yani denizle çevrili Sur Kenti gibi Susturulmus bir kent var mi?
\par 33 Mallarin denizasiri ülkelere vardiginda Birçok ulusu doyurdun, Büyük zenginligin, çesit çesit malinla Dünya krallarini zenginlestirdin.
\par 34 Simdiyse denizde, sularin derinliklerinde Darmadagin oldun, Mallarin ve çalisanlarinin tümü Seninle birlikte batti.
\par 35 Kiyi halklari Basina gelenlere sastilar; Krallarinin tüyleri korkudan diken diken oldu, Yüzleri sarardi.
\par 36 Uluslarin arasindaki tüccarlar, Basina gelenlere sasacaklar; Sonun korkunç oldu. Bir daha var olmayacaksin."

\chapter{28}

\par 1 RAB bana söyle seslendi:
\par 2 "Insanoglu, Sur önderine de ki, 'Egemen RAB söyle diyor: "'Gurura kapilip Ben tanriyim, Denizlerin bagrinda, Tanri'nin tahtinda oturuyorum dedin. Kendini Tanri sandin, Oysa sen Tanri degil, insansin.
\par 3 Iste, Daniel'den daha bilgesin, Kimse senden bir giz saklayamaz.
\par 4 Bilgeligin, anlayisin sayesinde, Kendine servet biriktirdin, Hazinelerine altin, gümüs yigdin.
\par 5 Ticaretteki üstün becerilerin sayesinde Servetini çogalttin, Zenginligin seni gurura sürükledi.
\par 6 Bu yüzden Egemen RAB söyle diyor: Madem kendini Tanri gibi bilge sandin,
\par 7 Ben de yabancilari, en acimasiz uluslari Üzerine gönderecegim. Bilgeliginin güzelligine kiliç çekecek, Görkemini kirletecekler.
\par 8 Seni ölüm çukuruna indirecekler, Denizlerin bagrinda korkunç bir ölümle öleceksin.
\par 9 O zaman seni öldürenlerin önünde Ben Tanri'yim diyecek misin? Seni öldürenlerin elinde Sen Tanri degil, insansin.
\par 10 Yabancilarin elinde, Sünnetsizin* ölümüyle öleceksin. Egemen RAB böyle diyor."
\par 11 RAB bana söyle seslendi:
\par 12 "Insanoglu, Sur Krali için bir agit yak. Ona diyeceksin ki, 'Egemen RAB söyle diyor: "'Kusursuzlukta örnek biriydin, Bilgeligin ve güzelligin eksiksizdi.
\par 13 Sen Tanri'nin bahçesi Aden'deydin. Yakut, topaz, aytasi, Sari yakut, oniks, yesim, Laciverttasi, firuze, zümrütle, çesit çesit degerli tasla bezenmistin. Kakma ve oyma islerin hep altindandi. Bunlar yaratildigin gün hazirlanmislardi.
\par 14 Meshedilmis*, koruyucu bir Keruv* olarak Seni oraya yerlestirdim. Tanri'nin kutsal dagindaydin, Yanan taslar arasinda dolastin.
\par 15 Yaratildigin günden Sende kötülük bulunana dek Yollarinda kusursuzdun.
\par 16 Ticaretinin bollugundan Zorbalikla doldun Ve günah isledin. Bu yüzden kirli bir sey gibi Seni Tanri'nin dagindan attim, Yanan taslarin arasindan kovdum, Ey koruyucu Keruv.
\par 17 Güzelliginden ötürü Gurura kapildin, Görkeminden ötürü Bilgeligini bozdun. Böylece seni yere attim, Krallarin önünde seni yüzkarasi yaptim.
\par 18 Isledigin pek çok günah Ve ticaretteki hileciligin yüzünden Kutsal yerlerini kirlettin. Seni yakip yok edecek Bir ates çikardim içinden, Bütün seyredenlerin gözü önünde Seni yeryüzünde küle çevirdim.
\par 19 Seni taniyan bütün uluslar sana sasti, Sonun korkunç oldu. Bir daha var olmayacaksin."
\par 20 RAB bana söyle seslendi:
\par 21 "Insanoglu, yüzünü Sayda'ya çevir, ona karsi peygamberlik et.
\par 22 Diyeceksin ki, 'Egemen RAB söyle diyor: "'Iste sana karsiyim, ey Sayda, Senin içinde yücelecegim. Onlari cezalandirinca, Kutsalligimi onlara gösterince, Benim RAB oldugumu anlayacaklar.
\par 23 Üzerine salgin hastalik gönderecek, Sokaklarinda kan akitacagim. Kentin içinde, her yaninda Kiliçla yaralananlar düsüp ölecekler. O zaman benim RAB oldugumu anlayacaklar."
\par 24 "'Israil halkini küçümseyen Çevre uluslardan hiçbiri Bir daha Israil için batan bir çali, Acitan bir diken olmayacak. O zaman benim RAB oldugumu anlayacaklar.
\par 25 "'Egemen RAB söyle diyor: Israil halkini aralarina dagilmis olduklari uluslardan topladigim, uluslarin gözü önünde kutsalligimi gösterdigim zaman kulum Yakup'a verdigim kendi ülkelerine yerlesecekler.
\par 26 Orada güvenlik içinde yasayacak, evler yapacak, baglar dikecekler. Onlari küçümseyen bütün çevre uluslari cezalandirdigimda güvenlik içinde yasayacaklar. O zaman benim Tanrilari RAB oldugumu anlayacaklar."

\chapter{29}

\par 1 Sürgünlügümüzün onuncu yili, onuncu ayin* on ikinci günü RAB bana söyle seslendi:
\par 2 "Insanoglu, yüzünü firavuna çevir, ona ve Misir'a karsi peygamberlik et.
\par 3 Onlara de ki, 'Egemen RAB söyle diyor: "'Kendi kanallarinin içinde yatan Büyük canavar firavun, Iste, sana karsiyim. Sen ki, Nil benimdir, Onu kendim için yaptim dersin.
\par 4 Çenelerine çengeller takacak, Kanallarindaki baliklari Senin pullarina yapistiracagim. Pullarina yapismis baliklarla birlikte Seni kanallarindan çikaracagim.
\par 5 Seni de kanallarindaki bütün baliklari da Çöle atacagim. Kirlara düseceksin, Toplanmayacak, gömülmeyeceksin. Seni yem olarak yabanil hayvanlara Ve yirtici kuslara verecegim.
\par 6 O zaman Misir'da yasayan herkes Benim RAB oldugumu anlayacak. "'Çünkü sen Israil halkina kamis bir degnek oldun.
\par 7 Seni elleriyle tuttuklarinda parçalanip onlarin omuzlarini yardin. Sana dayandiklarinda parçalanip bellerini burktun.
\par 8 "'Bu yüzden Egemen RAB söyle diyor: Üzerine halkini ve hayvanlarini öldürecek bir kiliç gönderiyorum.
\par 9 Misir kimsesiz birakilacak, viraneye çevrilecek. O zaman benim RAB oldugumu anlayacaklar. "'Madem Nil benimdir, onu ben yaptim dedin,
\par 10 ben de sana ve kanallarina karsiyim. Misir'i Migdol'dan Asvan'a, Kûs* sinirina dek kimsesiz birakacak, viraneye çevirecegim.
\par 11 Içinden insan ayagi da, hayvan ayagi da geçmeyecek. Kirk yil orada kimse yasamayacak.
\par 12 Misir'i issiz kalmis ülkeler gibi issiz birakacagim. Kentleri, viran olmus kentler arasinda kirk yil kimsesiz kalacak. Misirlilar'i uluslar arasina gönderecek, ülkelere dagitacagim.
\par 13 "'Egemen RAB söyle diyor: Kirk yil sonra onlari dagilmis olduklari uluslardan toplayacagim.
\par 14 Sürgündekileri geri getirip Patros'a, yurtlarina döndürecegim. Orada güçsüz bir krallik olusturacaklar.
\par 15 Kralliklarin en güçsüzü olacak, bir daha uluslarin üzerinde egemenlik sürmeyecek. Uluslari yönetmesinler diye onlari küçük düsürecegim.
\par 16 Misir bir daha Israil halkinin güvenecegi bir yer olmayacak. Ancak Misirlilar onlara Misir'a dönmekle isledikleri günahi animsatacaklar. O zaman Israilliler benim Egemen RAB oldugumu anlayacaklar."
\par 17 Sürgünlügümüzün yirmi yedinci yili, birinci ayin birinci günü RAB bana söyle seslendi:
\par 18 "Insanoglu, Babil Krali Nebukadnessar ordusunu Sur Kenti'ne karsi büyük bir saldiriya geçirdi; herkesin saçi döküldü, agir yük yüzünden omuz derileri yüzüldü. Ama Sur'a karsi ordusunu saldiriya geçirmesine karsin, bundan ne kendisi ne de ordusu yararlandi.
\par 19 Bu yüzden Egemen RAB söyle diyor: Misir'i Babil Krali Nebukadnessar'a verecegim, onun servetini alip götürecek. Ordusuna ücret olarak ülkeden yagmaladigi çapul malini dagitacak.
\par 20 Hizmetine karsilik Misir'i ona verdim; çünkü o da ordusu da bana hizmet ettiler. Egemen RAB böyle diyor.
\par 21 "O gün Israil halkini güçle donatacagim. Onlarin arasinda senin dilini çözecegim. O zaman benim RAB oldugumu anlayacaklar."

\chapter{30}

\par 1 RAB bana söyle seslendi:
\par 2 "Insanoglu, peygamberlik et ve de ki, 'Egemen RAB söyle diyor: "'Ah o gün diye haykir.
\par 3 Çünkü o gün yakin. RAB'bin günü yakin, Bulutlarin günü, Uluslarin yikim zamani.
\par 4 Bir kiliç Misir'a karsi çikacak. Kûs'u* acilar saracak. Misir'da vurulanlar yere serilince Ülkenin serveti alinip götürülecek, Temelleri yok edilecek.
\par 5 Misir'la birlikte Kûs, Pût, Lud, Arabistan, Kuv ve antlasma yaptigim halkim Kiliçtan geçirilecek.
\par 6 "'RAB söyle diyor: Misir'i destekleyenler öldürülecek, Misir'in övündügü ordu çökecek, Migdol'dan Asvan'a dek kiliçtan geçirilecekler. Böyle diyor Egemen RAB.
\par 7 Kimsesiz kalmis ülkeler arasinda Kimsesiz kalacaklar. Kentleri viran olmus kentler gibi olacak.
\par 8 Misir'i atese verdigimde, Onu destekleyenler ezildiginde, Benim RAB oldugumu anlayacaklar.
\par 9 "'O gün kaygisiz Kûslular'i korkutmak için gemilerle ulaklar gönderecegim. Misir'in yikim günü geldiginde korkuya kapilacaklar. Iste o gün geliyor.
\par 10 "'Egemen RAB söyle diyor: Babil Krali Nebukadnessar araciligiyla Misir'in zenginligine son verecegim.
\par 11 O ve ordusu, uluslarin en acimasizi, Ülkeyi yerle bir etmek için gelecekler. Misir'a karsi kiliçlarini çekecek, Ülkeyi öldürülenlerle dolduracaklar.
\par 12 Nil'in kanallarini kurutup Ülkeyi kötü kisilere teslim edecegim, Ülkeyi de içindeki her seyi de Yabancilar eliyle viran edecegim. Bunu ben RAB söylüyorum.
\par 13 Egemen RAB söyle diyor: Putlari yok edecek, Nof'taki degersiz putlara son verecegim. Misir'da artik önder olmayacak, Ülkeye korku salacagim.
\par 14 Patros'u viraneye çevirecek, Soan'i atese verecek, No Kenti'ni cezalandiracagim.
\par 15 Öfkemi Misir'in kalesi Sin üzerine bosaltacak, Kalabalik No halkina son verecegim.
\par 16 Misir'i atese verecegim, Sin acidan kivranacak, No*fy* Kenti'nin surlari yarilacak, Nof*fü* sürekli tedirgin olacak.
\par 17 On Kenti ve Pi-Beset gençleri Kiliçtan geçirilecek, Oradaki halk sürgüne gönderilecek.
\par 18 Tahpanhes'te Misir'in boyundurugunu kirdigim zaman, Orada gündüz geceye dönecek, Övündügü orduya son verilecek, Kent bulutlarla kaplanacak, Köylerindeki halk sürgüne gönderilecek.
\par 19 Misir'i böyle cezalandirdigimda Benim RAB oldugumu anlayacaklar."
\par 20 Sürgünlügümüzün on birinci yili, birinci ayin* yedinci günü RAB bana söyle seslendi:
\par 21 "Insanoglu, firavunun kolunu kirdim. Iyilesmesin, kiliç tutacak kadar güçlenmesin diye kimse onu baglamadi, sargi beziyle sarmadi.
\par 22 Bu yüzden Egemen RAB söyle diyor: Firavuna karsiyim. Her iki kolunu, saglam olani da kirik olani da kiracagim. Kilici elinden düsürecegim.
\par 23 Misirlilar'i uluslar arasina gönderecek, ülkelere dagitacagim.
\par 24 Babil Krali'nin kollarini güçlendirip kilicimi onun eline verecegim. Firavunun ise kollarini kiracagim. Babil Krali'nin önünde agir yarali biri gibi inleyecek.
\par 25 Babil Krali'nin gücüne güç katacak, firavunun gücünü zayiflatacagim. Kilicimi Babil Krali'nin eline verdigimde ve o kilici Misir'a dogru uzattiginda, benim RAB oldugumu anlayacaklar.
\par 26 Misirlilar'i uluslar arasina gönderecek, ülkelere dagitacagim. O zaman benim RAB oldugumu anlayacaklar."

\chapter{31}

\par 1 Sürgünlügümüzün on birinci yili, üçüncü ayin* birinci günü RAB bana söyle seslendi:
\par 2 "Insanoglu, firavuna ve halkina de ki, "'Görkemde kim seninle boy ölçüsebilir?
\par 3 Asur'a bak! Lübnan'da bir sedir agaciydi, Ormana gölge salan güzel dallari vardi. Çok yüksekti, tepesi bulutlara erisiyordu.
\par 4 Sular agaci besledi, Derin su kaynaklari büyüttü. Akarsular dikili oldugu yerin çevresine akiyor, Kanallari kirdaki bütün agaçlara erisiyordu.
\par 5 Kirdaki bütün agaçlardan daha çok büyüdü. Bol su verildigi için Dal budak saldi, dallari uzadi.
\par 6 Kuslar dallarina yuva yapti, Yabanil hayvanlar dallari altinda yavruladi, Büyük uluslar gölgesinde yasadi.
\par 7 Güzellikte essizdi. Dallari giderek uzadi, Çünkü kökleri bol su aliyordu.
\par 8 Tanri'nin bahçesindeki sedir agaçlarindan hiçbiri Onunla boy ölçüsemezdi, Çam agaçlari dallari kadar bile degildi. Çinarlar onun dallariyla boy ölçüsemezdi. Tanri'nin bahçesindeki agaçlarin hiçbiri Onun kadar güzel degildi.
\par 9 Sik dallarla o sedir agacini güzellestirdim. Tanri'nin bahçesi Aden'deki bütün agaçlar onu kiskandi.
\par 10 "'Bu yüzden Egemen RAB söyle diyor: Agaç büyüyüp boy attigi, tepesi bulutlara eristigi, büyüklügünden ötürü gurura kapildigi için
\par 11 ben de onu kovdum, uluslarin önderinin eline teslim ettim. Ona kötülügü uyarinca davranacak.
\par 12 Yabanci uluslarin en acimasizi onu kesip yalniz birakti. Dallari daglara, derelere düstü; ülkenin vadilerinde kesilmis duruyor. Yeryüzündeki bütün uluslar gölgesinden çekilip onu biraktilar.
\par 13 Bütün kuslar devrik agaca kondu, yabanil hayvanlar dallari arasina yerlesti.
\par 14 Öyle ki, sularin yakininda yetisen hiçbir agaç böylesi büyüyüp boy atmasin, tepesini bulutlara eristirmesin; bol suyla sulanan hiçbir agaç bu denli yükselmesin. Çünkü hepsi ölüm çukuruna inen insanlarla birlikte ölüme, yerin derinliklerine gidecek.
\par 15 "'Egemen RAB söyle diyor: Sedir agaci ölüler diyarina indigi gün, ona yas tutsunlar diye derin su kaynaklarini kapattim. Irmaklarini durdurdum, gür sularinin önünü kestim. O agaç yüzünden Lübnan'i karanliga bogdum, bütün orman agaçlarini kuruttum.
\par 16 Ölüm çukuruna inenlerle birlikte onu ölüler diyarina indirdigimde, yikilisinin gürültüsünden uluslari titrettim. O zaman Aden Bahçesi'ndeki bütün agaçlar, Lübnan'in en seçkin, en iyi, bol sulanan agaçlari yerin derinliklerinde avunç buldu.
\par 17 Gölgesinde yasayanlar, uluslar arasinda onu destekleyenler de onunla birlikte ölüler diyarina, kiliçla öldürülmüslerin yanina indiler.
\par 18 "'Aden agaçlarindan hangisi görkem ve yücelikte seninle boy ölçüsebilir? Ama sen de Aden agaçlariyla birlikte yerin derinliklerine indirilecek, sünnetsizlere*, kiliçla öldürülmüslere katilacaksin. "'Iste firavunla halkinin sonu böyle olacaktir. Egemen RAB böyle diyor."

\chapter{32}

\par 1 Sürgünlügümüzün on ikinci yili, on ikinci ayin* birinci günü RAB bana söyle seslendi:
\par 2 "Insanoglu, firavun için bir agit yak. Ona de ki, "'Uluslar arasinda genç bir aslan gibi kendini öne sürdün, Ama sen denizlerdeki bir canavar gibisin. Irmaklarini karistirir, Ayaklarinla sulari çalkalar, Irmaklari bulandirirsin."
\par 3 Egemen RAB söyle diyor: "Büyük bir kalabalikla Agimi senin üzerine atacagim; Onlar seni agimla çekecekler.
\par 4 Seni karaya atacak, Kirlara firlatacagim. Gökte uçan kuslarin senin üzerine konmalarini saglayacagim, Yeryüzündeki yabanil hayvanlara Seni yem olarak verecegim.
\par 5 Bedenini daglarin üzerine serecek, Vadileri çürüyen bedeninle dolduracagim.
\par 6 Ülkeyi daglara dek akan kaninla islatacagim, Vadiler seninle dolacak.
\par 7 Seni ortadan kaldirdigim zaman Gökleri örtecek, Yildizlari karartacak, Günesi bulutla kapatacagim. Ay isigini vermeyecek.
\par 8 Senin yüzünden gökte isik veren bütün cisimleri karartacak, Ülkeni karanliga gömecegim." Böyle diyor Egemen RAB.
\par 9 "Seni tanimadigin ülkelere, Uluslarin arasina sürgüne gönderdigimde, Pek çok halkin yüregi üzüntüyle sarsilacak.
\par 10 Basina gelenlerden ötürü Pek çok halki saskina çevirecegim. Kilicimi önlerinde salladigim zaman, Senin yüzünden krallar dehsetle ürperecek. Yikima ugradigin gün Hepsi kendi cani için Her an korkuyla titreyecek.
\par 11 Egemen RAB söyle diyor: Babil Krali'nin kilici üzerine gelecek.
\par 12 Yigitlerin, uluslarin en acimasizinin, Senin halkini kiliçtan geçirmesine izin verecegim. Misir'in gururunu kiracak, Bütün ordusunu yok edecekler.
\par 13 Bol sularin yaninda bütün sigirlarini yok edecegim. Bundan böyle insan ayagi da hayvan ayagi da Sulari karistirip bulandirmayacak.
\par 14 O zaman sularini dupduru kilacak, Irmaklarini yag gibi akitacagim. Egemen RAB böyle diyor.
\par 15 Misir'i viraneye çevirdigimde, Ülkeyi her seyden yoksun biraktigimda, Orada yasayan herkesi yok ettigimde, Benim RAB oldugumu anlayacaklar.
\par 16 "Ona yakacaklari agit budur. Uluslarin kizlari bu agiti yakacaklar. Misir için, halki için bu agiti yakacaklar." Egemen RAB böyle diyor.
\par 17 Sürgünlügümüzün on ikinci yili, ayin on besinci günü RAB bana söyle seslendi:
\par 18 "Ey insanoglu, Misir halki için yas tut. Onlari ve güçlü uluslarin kizlarini ölüm çukuruna inenlerle birlikte yerin derinliklerine indir.
\par 19 Onlara de ki, 'Sen baskalarindan daha mi güzelsin? Asagi in ve oradaki sünnetsizlere* katil.
\par 20 Misir halki kiliçla öldürülenlerin arasina düsecek. Kiliç hazir, birakin Misir bütün halkiyla birlikte sürüklensin.
\par 21 Güçlü önderler, ölüler diyarindan, Misir ve onu destekleyenler için, 'Asagi indiler, kiliçla öldürülen sünnetsizlerle birlikte burada yatiyorlar diyecekler.
\par 22 "Asur bütün ordusuyla orada. Kiliçtan geçirilmis, ölmüs askerlerinin mezarlari çevresini sarmis.
\par 23 Mezarlari ölüm çukurunun en dibinde, ordusu mezarinin çevresinde duruyor. Yasayanlar diyarinda korku salanlarin hepsi kiliçtan geçirilmis, ölmüs.
\par 24 "Elam bütün halkiyla kendi mezarinin çevresinde duruyor. Hepsi kiliçtan geçirilmis, ölmüs, sünnetsiz olarak yerin derinliklerine inmis. Yasayanlar diyarinda korku salmislardi, simdiyse utanç içinde ölüm çukuruna inenlere katildilar.
\par 25 Elam için öldürülenler arasinda bir yatak yapildi. Bütün halki mezarinin çevresinde. Hepsi sünnetsiz, kiliçtan geçirilerek ölmüs. Yasayanlar diyarinda korku salmislardi, simdiyse utanç içinde ölüm çukuruna inenlere katildilar, öldürülenlerin arasina yerlestirildiler.
\par 26 "Mesek ve Tuval bütün halkiyla kendi mezarlari çevresinde duruyor. Hepsi sünnetsiz, kiliçtan geçirilerek öldürülmüs. Yasayanlar diyarinda korku salmislardi.
\par 27 Ölüler diyarina savas silahlariyla inen, kiliçlari baslarinin altina konan, kalkanlari kemikleri üzerine yerlestirilen öbür öldürülmüs sünnetsiz yigitlerle birlikte mezara konmayacak mi onlar? Oysa bu yigitler yasayanlar diyarinda korku salmislardi.
\par 28 "Sen de, ey firavun, düsecek ve kiliçla öldürülenlerle birlikte sünnetsizlerin arasina konacaksin.
\par 29 "Edom, krallari ve önderleriyle orada. Güçlü olmalarina karsin kiliçla öldürülenlerin yanina kondular. Ölüm çukuruna inenlerin, sünnetsizlerin yaninda yatiyorlar.
\par 30 "Bütün kuzey önderleri, bütün Saydalilar orada. Güçleriyle korku saldiklari halde öldürülenlerle birlikte utanç içinde indiler. Sünnetsiz olarak kiliçla öldürülenlerle birlikte utanç içinde ölüm çukuruna inenlerin yanina kondular.
\par 31 "Firavunla ordusu kiliçla öldürülmüs bu büyük kalabaligi görünce avunç bulacak." Böyle diyor Egemen RAB.
\par 32 "Yasayanlar diyarinda korku salmasini sagladigim halde, firavunla halki, kiliçla öldürülenlerle birlikte sünnetsizlerin yanina konacak." Böyle diyor Egemen RAB.

\chapter{33}

\par 1 RAB bana söyle seslendi:
\par 2 "Insanoglu, kendi halkina söyle diyeceksin: 'Bir ülkenin üzerine kiliç gönderdigim, ülke halki aralarindan birini seçip bekçi atadigi,
\par 3 bekçi kilicin ülkenin üzerine yaklastigini görüp halki uyarmak için boru çaldigi zaman;
\par 4 kim boru sesini isitip de uyariyi dikkate almazsa, kiliç da gelip onu öldürürse, kanindan kendisi sorumludur.
\par 5 Boru sesini duymus, ama uyariyi dikkate almamistir; kanindan kendisi sorumludur. Uyariyi dikkate alsaydi, canini kurtaracakti.
\par 6 Ne var ki, bekçi kilicin ülkenin üzerine yaklastigini görüp halki uyarmak için boru çalmazsa, kiliç da gelip halktan birini öldürürse, o kisi kendi günahi içinde öldürülmüstür; kanindan bekçiyi sorumlu tutacagim.
\par 7 "Insanoglu, seni Israil halkina bekçi atadim. Benden bir söz duyar duymaz onlari benim yerime uyaracaksin.
\par 8 Kötü kisiye, 'Ey kötü kisi, kesinlikle öleceksin dedigim zaman, onu uyarmaz, kötü yolundan döndürmek için konusmazsan, o kisi günahi içinde ölecek; ama onun kanindan seni sorumlu tutacagim.
\par 9 Ancak kötü kisiyi uyardigin halde yolundan dönmezse, o günahi içinde ölecek. Ama sen canini kurtarmis olacaksin.
\par 10 "Insanoglu, Israil halkina de ki, 'Siz söyle diyorsunuz: Isyanlarimizla günahlarimiz bizi çökertiyor, onlardan ötürü eriyip yok oluyoruz. Durum böyleyken nasil yasayabiliriz?
\par 11 Onlara de ki, 'Varligim hakki için diyor Egemen RAB, ben kötü kisinin ölümünden sevinç duymam, ancak kötü kisinin kötü yollarindan dönüp yasamasindan sevinç duyarim. Dönün! Kötü yollarinizdan dönün! Niçin ölesiniz, ey Israil halki!
\par 12 "Sen, ey insanoglu, halkina de ki, 'Dogru kisi Tanri'ya baskaldirirsa, dogrulugu onu kurtarmaz. Kötü kisi kötülügünden döndügü zaman kötülügü yikimina neden olmaz. Dogru kisi Tanri'ya baskaldirirsa, dogrulugu yasamasini saglamaz.
\par 13 Dogru kisi için, 'Kesinlikle yasayacak desem, ama o dogruluguna güvenip de kötülük yapsa, yaptigi dogru islerin hiçbiri animsanmayacak. Yaptigi kötülükten ötürü ölecek.
\par 14 Kötü kisiye, 'Kesinlikle öleceksin desem, ama o günahindan dönüp adil ve dogru olani yapsa,
\par 15 aldigi rehini geri verse, çaldigini ödese, yasam veren kurallar uyarinca davranip günah islemese kesinlikle yasayacak, ölmeyecektir.
\par 16 Isledigi günahlardan hiçbiri ona karsi animsanmayacaktir, adil ve dogru olani yapmistir; kesinlikle yasayacaktir.
\par 17 "Senin halkin, 'Rab'bin yolu dogru degil diyor. Oysa dogru olmayan onlarin yolu.
\par 18 Dogru kisi dogrulugundan döner de kötülük yaparsa, yaptigi kötülügün içinde ölecektir.
\par 19 Kötü kisi yaptigi kötülükten döner de adil ve dogru olani yaparsa, yaptigi bu islerle yasayacaktir.
\par 20 Ey Israil halki, 'Rab'bin yolu dogru degil diyorsun. Her birinizi kendi yoluna göre yargilayacagim."
\par 21 Sürgünlügümüzün on ikinci yili, onuncu ayin* besinci günü Yerusalim'den kaçip kurtulan biri yanima gelip, "Kent düstü!" dedi.
\par 22 Aksam, Yerusalim'den kaçip kurtulan adam gelmeden önce, RAB'bin eli üzerimdeydi, konusamiyordum. Sabah o yanima gelmeden RAB dilimi çözdü. Dilim açildi, artik konusabilirdim.
\par 23 RAB bana söyle seslendi:
\par 24 "Insanoglu, Israil'in viran olmus kentlerinde yasayanlar, 'Ibrahim tek kisiyken ülkeyi miras almisti. Oysa biz kalabaligiz, ülke miras olarak bize verilmistir diyorlar.
\par 25 Bu nedenle onlara de ki, 'Egemen RAB söyle diyor: Eti kaniyla yiyor, putlariniza bel bagliyor, kan döküyorsunuz. Yine de ülkeyi miras almayi mi umuyorsunuz?
\par 26 Kiliciniza güveniyor, igrenç seyler yapiyor, komsunuzun karisini kirletiyorsunuz. Yine de ülkeyi miras almayi mi umuyorsunuz?
\par 27 "Onlara de ki, 'Egemen RAB söyle diyor: Varligim hakki için, viran olmus kentlerde yasayanlar kiliçtan geçirilecek, kirda yasayanlari yem olarak yabanil hayvanlara verecegim, kalelerde, magaralarda yasayanlar salgin hastalikla yok olacak.
\par 28 Ülkeyi issiz, kimsesiz birakacagim, övündükleri güç son bulacak. Israil daglari issiz kalacak, oradan kimse geçmeyecek.
\par 29 Yaptiklari igrenç seylerden ötürü ülkeyi issiz, kimsesiz biraktigim zaman benim RAB oldugumu anlayacaklar.
\par 30 "Sen, ey insanoglu, halkin duvar diplerinde, evlerin kapilari önünde senin hakkinda konusuyor. Birbirlerine, 'Haydi, gidip RAB'den gelen sözün ne oldugunu duyalim diyorlar.
\par 31 Halk her zamanki gibi sana geliyor. Benim halkim olarak önünde oturuyor, sözlerini dinliyor, ama dediklerini yapmiyorlar. Agizlariyla istekli olduklarini açikliyorlar, ama yürekleri haksiz kazanç pesinde.
\par 32 Sen onlar için güzel sesle sevgi ezgileri okuyan, iyi çalgi çalan biri gibisin. Sözlerini dinliyor, ama dediklerini yapmiyorlar.
\par 33 "Bütün bunlar gerçeklesince -ki gerçeklesecek- aralarinda bir peygamber bulundugunu anlayacaklar."

\chapter{34}

\par 1 RAB bana söyle seslendi:
\par 2 "Insanoglu, Israil'in çobanlarina karsi peygamberlik et ve onlara, bu çobanlara söyle de: 'Egemen RAB diyor ki: Vay kendi kendini güden Israil çobanlarina! Çobanlarin sürüyü gütmesi gerekmez mi?
\par 3 Yagi yiyor, yünü giyiyor, besili koyunlari kesiyorsunuz, ama sürüyü kayirmiyorsunuz.
\par 4 Zayiflari güçlendirmediniz, hastalari iyilestirmediniz, yaralilarin yarasini sarmadiniz. Yolunu sasiranlari geri getirmediniz, yitikleri aramadiniz. Ancak sertlik ve siddetle onlara egemen oldunuz.
\par 5 Çobanlari olmadigi için dagildilar, yabanil hayvanlara yem oldular.
\par 6 Koyunlarim bütün daglarda, yüksek tepelerde basibos dolandilar. Koyunlarim yeryüzüne dagildi. Onlari ne arayan var, ne soran.
\par 7 "'Bu yüzden, ey çobanlar, RAB'bin sözünü dinleyin:
\par 8 Varligim hakki için diyor Egemen RAB, çoban olmadigindan koyunlarim yagma edildi, yabanil hayvanlara yem oldu. Çobanlarim koyunlarimi aramadilar, onlari güdeceklerine kendi kendilerini güttüler.
\par 9 Onun için, ey çobanlar, RAB'bin sözünü dinleyin.
\par 10 Egemen RAB söyle diyor: Ben çobanlara karsiyim! Koyunlarimdan onlari sorumlu tutacagim, koyunlarimi gütmelerine son verecegim. Öyle ki, artik kendi kendilerini güdemeyecekler. Koyunlarimi onlarin agzindan kurtaracagim, artik onlara yem olmayacaklar.
\par 11 "'Egemen RAB söyle diyor: Ben kendim koyunlarimi arayip soracagim.
\par 12 Dagilmis koyunlarinin arasindaki bir çoban sürüsüyle nasil ilgilenirse, ben de koyunlarimla öyle ilgilenecegim. Bulutlu, karanlik bir gün dagilmis olduklari her yerden onlari kurtaracagim.
\par 13 Onlari uluslarin arasindan çikaracak, ülkelerden toplayacak, kendi yurtlarina geri getirecegim. Onlari Israil daglarinda, vadilerde, ülkenin bütün oturulabilir yerlerinde güdecegim.
\par 14 Onlari iyi bir otlakta güdecegim; yaylalari Israil'in yüksek daglari üzerinde olacak. Orada iyi bir otlakta yatacak, Israil'in yüksek daglarindaki verimli otlaklarda otlayacaklar.
\par 15 Ben kendim koyunlarimi güdecegim, onlari kendim yatiracagim. Egemen RAB böyle diyor.
\par 16 Yiteni arayacak, yolunu sasirani geri getirecegim. Yaralinin yarasini saracak, zayifi güçlendirecegim. Ama semizlerle güçlüleri yok edecegim. Koyunlarimi adaletle güdecegim.
\par 17 "'Siz, ey benim sürüm, Egemen RAB söyle diyor: Koyunla koyun arasinda yargiyi ben verecegim. Koçlarla tekelere gelince,
\par 18 iyi otlakta otlamaniz yetmiyor mu ki, otlaklarinizin geri kalanini ayaklarinizla çigniyorsunuz? Duru su içmeniz yetmiyor mu ki, geri kalan suyu ayaklarinizla bulandiriyorsunuz?
\par 19 Koyunlarim ayaklarinizin çignedigini otlamak, ayaklarinizin bulandirdigini içmek zorunda kaliyor.
\par 20 "'Bu nedenle Egemen RAB onlara söyle diyor: Semiz koyunla ciliz koyun arasinda ben kendim yargiçlik yapacagim.
\par 21 Madem bütün ciliz koyunlari kovup dagitincaya dek bögrünüzle vuruyor, omuzunuzla itiyor, boynuzlarinizla kakiyorsunuz,
\par 22 ben de koyunlarimi kurtaracagim, artik çapul mali olmayacaklar. Koyunla koyun arasinda ben yargiçlik yapacagim.
\par 23 Baslarina, onlari güdecek tek çoban olarak kulum Davut'u koyacagim. Onlari o güdecek, çobanlari o olacak.
\par 24 Ben RAB onlarin Tanrisi olacagim, kulum Davut da onlarin arasinda önder olacak. Ben RAB, böyle diyorum.
\par 25 "'Onlarla bir baris antlasmasi yapacagim, ülkedeki yirtici hayvanlari yok edecegim. Çölde güvenlik içinde yasayacak, ormanlarda uyuyacaklar.
\par 26 Onlari da dagimin çevresini de bereketli kilacagim. Yagmuru zamaninda yagdiracagim. Bereketli yagmurlar olacak.
\par 27 Kirdaki agaçlar meyve verecek, toprak ürün verecek. Halk ülkesinde güvenlik içinde olacak. Boyunduruklarinin baglarini koparip onlari köle edenlerin elinden kurtardigim zaman benim RAB oldugumu anlayacaklar.
\par 28 Artik uluslarin çapul mali, yabanil hayvanlarin yemi olmayacaklar. Güvenlik içinde yasayacaklar, kimse onlari korkutmayacak.
\par 29 Onlar için ünlü bir fidanlik yetistirecegim. Artik ülke kitliktan yok olmayacak, uluslarin asagilamasina ugramayacaklar.
\par 30 O zaman ben Tanrilari RAB'bin onlarla birlikte oldugumu ve Israil soyunun da benim halkim oldugunu anlayacaklar. Böyle diyor Egemen RAB.
\par 31 'Benim koyunlarim, otlagimin koyunlari siz insanlarsiniz. Ben sizin Tanriniz'im. Böyle diyor Egemen RAB."

\chapter{35}

\par 1 RAB bana söyle seslendi:
\par 2 "Insanoglu, yüzünü Seir Dagi'na çevir, ona karsi peygamberlik et.
\par 3 Ona de ki, 'Egemen RAB söyle diyor: Ey Seir Dagi, sana karsiyim! Elimi sana karsi uzatacak, seni viran edip kimsesiz birakacagim.
\par 4 Kentlerini yerle bir edecegim, kimsesiz kalacaksin. O zaman benim RAB oldugumu anlayacaksin.
\par 5 "'Madem Israilliler'e hep kin besledin, yikima ugradiklarinda, cezalandirilmalarinin zamani doruga ulastiginda, onlari kilica teslim ettin,
\par 6 varligim hakki için diyor Egemen RAB, senin kanini akitacagim, kan pesini birakmayacak. Madem kan dökmekten nefret etmedin, kan pesini birakmayacak.
\par 7 Seir Dagi'ni viran edip kimsesiz birakacagim, oraya gidip geleni kesip atacagim.
\par 8 Daglarini ölülerle dolduracagim; kiliçtan geçirilenler senin tepelerinde, vadilerinde, derelerinde düsüp ölecekler.
\par 9 Seni sonsuza dek viran edecegim, kentlerinde kimse oturmayacak. O zaman benim RAB oldugumu anlayacaksin.
\par 10 "'Siz, bu iki ulus, bu iki ülke bizim olacak, onlari miras alacagiz demistiniz. Oysa RAB oralardadir.
\par 11 Bundan ötürü varligim hakki için diyor Egemen RAB, beslediginiz kin yüzünden halkima nasil öfkeyle, kiskançlikla davrandiysaniz, ben de size öyle davranacagim. Sizi yargiladigim zaman onlara kendimi tanitacagim.
\par 12 O zaman Israil daglarina sövgülerinizi duydugumu anlayacaksiniz. Söyle demistiniz: "Yerle bir oldular, yutalim diye bize verildiler."
\par 13 Bana karsi böbürlendiginizi, saygisizca konustugunuzu da duydum.
\par 14 Egemen RAB söyle diyor: Bütün yeryüzü sevinirken, seni yerle bir edecegim.
\par 15 Israil halkinin mirasi yerle bir oldugunda nasil sevindinse, ben de sana öyle davranacagim. Ey Seir Dagi, viran olacaksin; bütün Edom da viran olacak. O zaman benim RAB oldugumu anlayacaklar."

\chapter{36}

\par 1 "Insanoglu, Israil daglarina peygamberlik et ve de ki, 'Ey Israil daglari, RAB'bin sözünü dinleyin!
\par 2 Egemen RAB söyle diyor: Düsman sizin hakkinizda, Hah, hah! Bu eski tepeler mülkümüz oldu! dedigi için
\par 3 peygamberlik et ve de ki, Egemen RAB söyle diyor: Daglarinizi viran ettiler, sizi her yandan sikistirip çignediler; böylece uluslarin mülkü oldunuz, dile düstünüz, alay konusu oldunuz,
\par 4 ey Israil daglari, Egemen RAB'bin sözünü dinleyin! Egemen RAB daglarla tepelere, vadilerle derelere, yikintilara, çevrenizdeki uluslarin yagmasina, alayina ugramis, terk edilmis kentlere söyle diyor:
\par 5 Egemen RAB söyle diyor: Yürekleri sevinç dolu, asagilayarak otlaklarinizi yagmalamak için ülkeme sahip çikan öteki uluslara, özellikle Edom'a karsi büyük bir kiskançlikla konustum.
\par 6 Bu nedenle Israil ülkesi için peygamberlik et ve daglara, tepelere, vadilere, derelere de ki, 'Egemen RAB söyle diyor: Uluslarin asagilamasina hedef oldugunuz için öfkeyle, kiskançlikla konustum.
\par 7 Bu nedenle Egemen RAB söyle diyor: Ant içiyorum ki, çevrenizdeki uluslar da asagilanacaktir.
\par 8 "'Ama siz, ey Israil daglari, dal budak salacak ve halkim Israil için ürün vereceksiniz. Çünkü halkim Israil yakinda yurduna dönecek.
\par 9 Sizi kayiracak, size yönelecegim. Islenecek, ekileceksiniz.
\par 10 Ülkenizde yasayanlarin sayisini, evet, bütün Israil halkinin sayisini çogaltacagim. Kentlerde insanlar yasayacak, yikintilar onarilacak.
\par 11 Ülkenizdeki insan ve hayvan sayisini çogaltacagim. Verimli olacak, çogalacaklar. Geçmiste oldugu gibi ülkeniz insanlarla dolup tasacak. Sizi eskisinden daha verimli kilacagim. O zaman benim RAB oldugumu anlayacaksiniz.
\par 12 Ülkenize insanlarin, halkim Israil'in girmesini saglayacagim. Sizi sahiplenecekler. Siz de onlarin mirasi olacaksiniz. Onlari bir daha çocuklarindan yoksun birakmayacaksiniz.
\par 13 "'Egemen RAB söyle diyor: Ey ülke, insanlar sana insan yiyen, ulusunu çocuksuz birakan ülke diyorlar.
\par 14 Bundan böyle artik sen insan yemeyecek, ulusunu çocuksuz birakmayacaksin. Egemen RAB böyle diyor.
\par 15 Artik uluslarin asagilamalarini size isittirmeyecegim. Uluslarin asagilamasina ugramayacaksiniz. Halkinizin bir daha tökezlemesine izin vermeyeceksiniz. Egemen RAB böyle diyor."
\par 16 RAB bana söyle seslendi:
\par 17 "Insanoglu, Israil halki kendi ülkesinde yasarken tutumu ve davranislariyla ülkeyi kirletti. Onlarin davranisi benim gözümde âdet gören bir kadinin kirliligi gibiydi.
\par 18 Bu yüzden öfkemi üzerlerine bosalttim. Çünkü ülkede kan döktüler, putlariyla onu kirlettiler.
\par 19 Onlari uluslara dagittim, ülkelere yayildilar. Onlari tutumlarina ve davranislarina göre yargiladim.
\par 20 Uluslarin arasinda her gittikleri yerde kutsal adimi kirlettiler. Çünkü onlar için, 'Bu RAB'bin halki, öyleyken ülkesinden çikmak zorunda kaldi dendi.
\par 21 Israil halkinin gittigi uluslar arasinda kirlettigi kutsal adimin onuru için kaygilandim.
\par 22 "Bu nedenle Israil halkina de ki, 'Egemen RAB söyle diyor: Ey Israil halki, sizin hatiriniz için degil, gittiginiz uluslar arasinda kirlettiginiz kutsal adimin hatiri için bunlari yapacagim.
\par 23 Uluslar arasinda kirlenen, onlar arasinda kirlettiginiz büyük adimin kutsalligini gösterecegim. Onlarin gözü önünde kutsalligimi sizin araciliginizla kanitladigimda, uluslar benim RAB oldugumu anlayacaklar. Egemen RAB böyle diyor.
\par 24 "'Sizi uluslar arasindan alacak, bütün ülkelerden toplayip ülkenize geri getirecegim.
\par 25 Üzerinize temiz su dökecegim, arinacaksiniz. Sizi bütün kirliliklerinizden ve putlarinizdan arindiracagim.
\par 26 Size yeni bir yürek verecek, içinize yeni bir ruh koyacagim. Içinizdeki tastan yüregi çikaracak, size etten bir yürek verecegim.
\par 27 Ruhumu içinize koyacagim; kurallarimi izlemenizi, buyruklarima uyup onlari uygulamanizi saglayacagim.
\par 28 Atalariniza verdigim ülkede yasayacak, benim halkim olacaksiniz, ben de sizin Tanriniz olacagim.
\par 29 Sizi bütün kirliliklerinizden kurtaracagim. Bugdaya seslenecek ve onu çogaltacagim. Artik size kitlik göndermeyecegim.
\par 30 Uluslarin arasinda bir daha kitlik utanci çekmemeniz için agaçlarin meyvesini, tarlalarin ürününü çogaltacagim.
\par 31 O zaman kötü yollarinizi, kötü islerinizi animsayacaksiniz. Günahlariniz, igrenç uygulamalariniz yüzünden kendinizden tiksineceksiniz.
\par 32 Bunu sizin hatiriniz için yapmadigimi iyi bilin. Egemen RAB böyle diyor. Davranislarinizdan utanin, yüzünüz kizarsin, ey Israil halki!
\par 33 "'Egemen RAB söyle diyor: Sizi bütün günahlarinizdan arittigim gün, kentlerinizde yasamanizi saglayacagim; yikintilar onarilacak.
\par 34 Gelip geçenlerin gözünde viran olan ülkenin topraklari islenecek.
\par 35 Söyle diyecekler: Viran olan bu ülke Aden bahçesi gibi oldu; yikilip yerle bir olmus, kimsesiz kalmis kentler yeniden güçlendiriliyor, içinde oturuluyor.
\par 36 O zaman çevrenizde kalan uluslar yikilani yeniden yapanin, çiplak yerleri yeniden dikenin ben RAB oldugumu anlayacaklar. Bunu ben RAB söylüyorum ve dedigimi yapacagim.
\par 37 "Egemen RAB söyle diyor: Israil halkinin benden yine yardim dilemesini saglayacak ve onlar için sunu yapacagim: Onlari bir koyun sürüsü gibi çogaltacagim.
\par 38 Bayramlarda Yerusalim nasil kurbanlik hayvanlarla doluyorsa, viran olmus kentler de insan topluluklariyla öyle dolup tasacak. O zaman benim RAB oldugumu anlayacaklar."

\chapter{37}

\par 1 RAB'bin eli üzerimdeydi, Ruhu'yla beni disari çikardi, kemiklerle dolu bir ovanin ortasina koydu.
\par 2 Beni onlarin arasinda her yöne dolastirdi. Ovada her yere yayilmis, tamamen kurumus pek çok kemik vardi.
\par 3 RAB, "Insanoglu, bu kemikler canlanabilir mi?" diye sordu. Ben, "Sen bilirsin, ey Egemen RAB" diye yanitladim.
\par 4 Bunun üzerine, "Bu kemikler üzerine peygamberlik et" dedi, "Onlara de ki, 'Kuru kemikler, RAB'bin sözünü dinleyin!
\par 5 Egemen RAB bu kemiklere söyle diyor: Içinize ruh koyacagim, canlanacaksiniz.
\par 6 Size kaslar verecek, üzerinizde et olusturacagim, sizi deriyle kaplayacagim. Içinize ruh koyacagim, canlanacaksiniz. O zaman benim RAB oldugumu anlayacaksiniz."
\par 7 Böylece bana verilen buyruk uyarinca peygamberlik ettim. Ben peygamberlik ederken bir gürültü oldu, bir takirti duyuldu. Kemikler birbirleriyle birlesiyordu.
\par 8 Baktim, iste üzerlerinde kaslar, etler olusuyor, üstlerini deri kapliyordu. Ama onlarda ruh yoktu.
\par 9 Sonra bana söyle dedi: "Rüzgara*fç* peygamberlik et, insanoglu, peygamberlik et ve de ki, 'Egemen RAB söyle diyor: Ey rüzgar, gel dört yandan es. Bu öldürülmüslerin üzerine üfle ki canlansinlar!"
\par 10 Böylece bana verilen buyruk uyarinca peygamberlik ettim. Onlarin içine soluk*fç* girince canlanip ayaga kalktilar. Çok, çok büyük bir kalabalik olusturuyorlardi.
\par 11 Sonra bana, "Insanoglu, bu kemikler bütün Israil halkini simgeliyor" dedi, "Onlar, 'Kemiklerimiz kurudu, umudumuz yok oldu, bittik diyorlar.
\par 12 Bu yüzden peygamberlik et ve onlara de ki, 'Egemen RAB söyle diyor: Ey halkim, mezarlarinizi açip sizi oradan çikaracak, Israil ülkesine geri getirecegim.
\par 13 Mezarlarinizi açip sizi çikardigim zaman benim RAB oldugumu anlayacaksiniz, ey halkim.
\par 14 Ruhumu içinize koyacagim, canlanacaksiniz. Sizi kendi ülkenize yerlestirecegim. O zaman, bunu söyleyenin ve yapanin ben RAB oldugumu anlayacaksiniz." Böyle diyor RAB.
\par 15 RAB bana söyle seslendi:
\par 16 "Insanoglu, bir degnek al, üzerine 'Yahuda ve dostlari Israilliler için diye yaz. Sonra baska bir degnek al, üzerine 'Yusuf'la dostlari Israilliler için Efrayim'in degnegi diye yaz.
\par 17 Iki degnegi yan yana getirerek birlestir. Öyle ki, elinde bir degnek gibi olsun.
\par 18 "Halkindan biri, 'Bu yaptiginin anlami ne? Bize açiklamaz misin? diye sorarsa,
\par 19 söyle yanitlayacaksin: 'Egemen RAB söyle diyor: Efrayim'in elindeki degnegi -Yusuf'la dostlari Israil oymaklarinin degnegini- alip Yahuda degnegiyle birlestirecegim. Ikisinden bir degnek yapip elimde tutacagim.
\par 20 Üzerine yazdigin degnekleri görebilecekleri sekilde elinde tut.
\par 21 Onlara de ki, 'Egemen RAB söyle diyor: Israilliler'i gittikleri uluslarin içinden alacagim. Onlari her yerden toplayip ülkelerine geri getirecegim.
\par 22 Onlari ülkede, Israil daglari üzerinde tek bir ulus yapacagim. Hepsinin tek krali olacak. Artik iki ayri ulus olmayacaklar, iki kralliga bölünmeyecekler.
\par 23 Artik putlariyla, igrenç uygulamalariyla, isyanlariyla kendilerini kirletmeyecekler. Onlari yerlestikleri, içinde günah isledikleri yerlerden kurtarip arindiracagim. Onlar halkim olacak, ben de onlarin Tanrisi olacagim.
\par 24 "'Kulum Davut onlarin krali olacak, hepsinin tek çobani olacak. Buyruklarimi izleyecek, kurallarima uyacak, onlari uygulayacaklar.
\par 25 Kulum Yakup'a verdigim, atalarinizin yasadigi ülkeye yerlesecekler. Kendileri, çocuklari, çocuklarinin çocuklari sonsuza dek orada yasayacaklar. Kulum Davut da sonsuza dek onlarin önderi olacak.
\par 26 Onlarla esenlik antlasmasi yapacagim. Bu onlarla sonsuza dek geçerli bir antlasma olacak. Onlari yeniden oraya yerlestirip sayica çogaltacagim. Tapinagimi sonsuza dek onlarin ortasina kuracagim.
\par 27 Konutum aralarinda olacak; onlarin Tanrisi olacagim, onlar da benim halkim olacak.
\par 28 Tapinagim sonsuza dek onlarin arasinda oldukça uluslar Israil'i kutsal kilanin ben RAB oldugumu anlayacaklar."

\chapter{38}

\par 1 RAB bana söyle seslendi:
\par 2 "Insanoglu, yüzünü Magog ülkesinden Ros'un, Mesek'in, Tuval'in önderi*fd* Gog'a çevir, ona karsi peygamberlik et.
\par 3 De ki, 'Egemen RAB söyle diyor: Ey Ros'un, Mesek'in, Tuval'in önderi Gog, sana karsiyim.
\par 4 Seni geldigin yoldan geri çevirecek, çenelerine çengel takacagim. Seni ve bütün ordunu, atlari, tam donanmis atlilari, küçük büyük kalkanli, hepsi kiliç kullanan büyük kalabaligi disariya sürükleyecegim.
\par 5 Onlarla birlikte hepsi kalkanli, migferli Persliler'i, Kûslular'i*, Pûtlular'i,
\par 6 Gomer'in bütün ordusunu, uzak kuzeydeki Beyttogarma'nin bütün ordusunu ve yanindaki birçok ulusu da sürükleyecegim.
\par 7 "'Hazir ol! Çevrende toplanmis büyük kalabalikla birlikte hazirlan. Onlari sen gözeteceksin.
\par 8 Uzun zaman sonra savasa çagrilacaksin. Gelecek yillarda, halki birçok ulustan uzun zamandir issiz kalmis Israil daglarinda toplanmis, savastan rahata kavusmus bir ülkeye saldiracaksin. Uluslar arasindan çikarilmis olan bu halk, simdi güvenlik içinde yasiyor.
\par 9 Sen, bütün askerlerin ve seninle olan birçok ulus çikip kasirga gibi geleceksiniz; ülkeyi kaplayan bulut gibi olacaksiniz.
\par 10 "'Egemen RAB söyle diyor: O gün aklina bazi düsünceler gelecek, kötü düzenler tasarlayacaksin.
\par 11 Diyeceksin ki: Sursuz köyleri olan bir ülkeye saldiracak, esenlik ve güvenlik içinde yasayan insanlarin üzerine yürüyecegim. Bu köylerin tümü sursuz; kapilari da kapi sürgüleri de yok.
\par 12 Viran olmus kentlerde yasayan halki soyup malini yagma edecegim. Sürüsü, mali olan, dünyanin ortasinda yasayan bu uluslarin arasindan toplanmis halka karsi elimi uzatacagim.
\par 13 Saba, Dedan, Tarsis tüccarlari ve köyleri sana, Yagmalamak için mi geldin? Çapul mali toplamak, altin, gümüs tasimak, hayvan, mal götürmek, bol ganimet elde etmek için mi bu kalabaligi topladin? diyecek.
\par 14 "Bu yüzden, ey insanoglu, peygamberlik et ve Gog'a de ki, 'Egemen RAB söyle diyor: O gün halkim Israil güvenlik içinde yasarken bunu farketmeyecek misin?
\par 15 Sen ve seninle birlikte birçok ulustan olusan tümü ata binmis büyük bir kalabalik, güçlü bir ordu uzak kuzeyden geleceksiniz.
\par 16 Ülkeyi kaplayan bir bulut gibi halkim Israil'in üzerine yürüyeceksiniz. Son günlerde, ey Gog, seni ülkeme saldirtacagim. Öyle ki, uluslarin gözü önünde kutsalligimi senin araciliginla gösterdigim zaman beni taniyabilsinler.
\par 17 "'Egemen RAB söyle diyor: Eski günlerde kullarim Israil peygamberleri araciligiyla hakkinda konustugum kisi degil misin sen? O dönemde seni onlara saldirtacagima iliskin yillarca peygamberlik ettiler.
\par 18 "'Gog Israil ülkesine saldirdigi gün öfkem alevlenecek. Egemen RAB böyle diyor.
\par 19 Kiskançligimla ve öfkemin siddetiyle diyorum ki, o gün Israil ülkesinde büyük bir yer sarsintisi olacak.
\par 20 Denizdeki baliklar, gökteki kuslar, kirdaki hayvanlar, yerde sürünen bütün yaratiklar ve dünyadaki bütün insanlar önümde titreyecekler. Daglar yerle bir edilecek, kayaliklar ufalanacak, her duvar çökecek.
\par 21 Bütün daglarimda Gog'a karsi kilici çagiracagim. Egemen RAB böyle diyor. Herkes birbirine kiliç çekecek.
\par 22 Onu salgin hastalikla, kanla cezalandiracagim; onun, ordusunun, ondan yana olan birçok ulusun üzerine saganak yagmur, dolu, atesli kükürt yagdiracagim.
\par 23 Böylece büyüklügümü, kutsalligimi gösterecek, birçok ulusun gözünde kendimi tanitacagim. O zaman benim RAB oldugumu anlayacaklar."

\chapter{39}

\par 1 "Insanoglu, Gog'a karsi peygamberlik et ve ona de ki, 'Egemen RAB söyle diyor: Ey Ros'un, Mesek'in, Tuval'in önderi Gog, sana karsiyim.
\par 2 Seni geri çevirip sürükleyecegim. Seni uzak kuzeyden çikarip Israil'in daglarina getirecegim.
\par 3 Sol elindeki yayini vuracak, sag elindeki oklarini düsürecegim.
\par 4 Sen de askerlerinle senden yana olan uluslar da Israil daglarina serileceksiniz. Sizi yem olarak her çesit yirtici kusa, yabanil hayvana verecegim.
\par 5 Açik kirlarda düsüp öleceksiniz. Çünkü bunu ben söyledim. Egemen RAB böyle diyor.
\par 6 Magog'un ve kiyida güvenlik içinde yasayanlarin üzerine ates yagdiracagim. O zaman benim RAB oldugumu anlayacaklar.
\par 7 "'Halkim Israil arasinda kutsal adimi tanitacagim. Bundan böyle kutsal adimin asagilanmasina izin vermeyecegim. Uluslar benim Israil'de kutsal olan RAB oldugumu anlayacaklar.
\par 8 O gün yaklasti! Söylediklerim olacak. Egemen RAB böyle diyor. Budur sözünü ettigim gün!
\par 9 "'O zaman Israil kentlerinde yasayanlar disari çikip topladiklari silahlari yakacaklar. Küçük büyük kalkanlari, yaylari, oklari, sopalari, mizraklari atese atacaklar. Bunlarla yedi yil ates yakacaklar.
\par 10 Kirdan odun toplamayacak, ormandan odun kesmeyecekler. Yakmak için silahlari kullanacaklar. Mallarini yagmalayanlari yagmalayacak, kendilerini soyanlari soyacaklar. Egemen RAB böyle diyor.
\par 11 "'O gün Lut Gölü'nün dogusunda, Gezginler Deresi'nde Gog'a Israil'de bilinen bir mezar yeri verecegim. Gog'la bütün ordusu orada gömülecek. Oraya Hamon-Gog Vadisi adi verilecek. Oradan geçecek gezginlerin önü kesilecek.
\par 12 Israil halki ülkeyi arindirmak için onlari gömecek. Bu yedi ay sürecek.
\par 13 Onlari bütün ülke halki gömecek. Görkemimi açikladigim gün onlar için onur olacak. Egemen RAB böyle diyor.
\par 14 "'Ülkeyi arindirmak için adamlar görevlendirilecek. Bazilari ülkeyi sürekli dolasacak, öbürleriyse yerde kalan cesetleri gömecekler. Yedi aylik süre bitince, arastirma isine baslayacaklar.
\par 15 Bu adamlar ülkenin her yanini dolasacak. Bir insan kemigi görünce, mezarcilar onu Hamon-Gog Vadisi'ne gömünceye dek, yanina bir isaret koyacak.
\par 16 Orada Hamona adinda bir kent olacak. Böylelikle ülke arindirilacak.
\par 17 "Insanoglu, Egemen RAB söyle diyor: Her çesit kusa ve yabanil hayvana seslen: 'Sizin için hazirlayacagim kurbana, Israil daglari üzerindeki büyük kurbana gelin, her yandan toplanin! Orada et yiyecek, kan içeceksiniz.
\par 18 Basan'in besili hayvanlarinin -koçlarin, kuzularin, tekelerin, bogalarin- etini yiyip kanini içer gibi yigitlerin etini yiyecek, dünya önderlerinin kanini içeceksiniz.
\par 19 Sizin için hazirlayacagim kurbandan doyana dek yag yiyeceksiniz, sarhos oluncaya dek kan içeceksiniz.
\par 20 Soframda atlardan, atlilardan, yigitlerden ve her çesit askerden bol bol yiyip doyacaksiniz. Egemen RAB böyle diyor.
\par 21 "Görkemimi uluslar arasinda açiklayacagim. Bütün uluslar kendilerine verdigim cezayi, üzerlerine koydugum elimi görecekler.
\par 22 Israil halki o günden baslayarak benim Tanrilari RAB oldugumu anlayacak.
\par 23 Uluslar Israil halkinin isledigi suç yüzünden, bana ihanet ettigi için sürgüne gittigini anlayacaklar. Yüzümü onlardan gizledim, onlari düsmanlarinin eline teslim ettim, hepsi kiliçtan geçirildi.
\par 24 Onlari kirliliklerine, isyanlarina göre cezalandirdim, yüzümü onlardan gizledim.
\par 25 "Bundan ötürü Egemen RAB söyle diyor: Yakup'un sürgündeki soyunu geri getirecek, Israil halkina aciyacagim. Kutsal adimi kiskançlikla koruyacagim.
\par 26 Ülkelerinde güvenlik içinde yasayinca, onlari korkutan kimse olmayinca, utançlarini, bana ettikleri bütün ihanetleri unutacaklar.
\par 27 Onlari uluslar arasindan geri getirip düsman ülkelerinden topladigim zaman, onlar araciligiyla birçok ulusa kutsalligimi gösterecegim.
\par 28 O zaman benim Tanrilari RAB oldugumu anlayacaklar. Onlari uluslar arasina sürgüne göndermeme karsin, hiçbirini birakmadan ülkelerine geri getirecegim.
\par 29 Onlardan bir daha yüzümü gizlemeyecegim, çünkü Israil halki üzerine Ruhum'u dökecegim." Egemen RAB böyle diyor.

\chapter{40}

\par 1 Sürgünlügümüzün yirmi besinci yili, yilin basinda, ayin onuncu günü, Yerusalim Kenti'nin düsüsünün on dördüncü yili, tam o gün RAB'bin eli beni yakalayip oraya götürdü.
\par 2 Görümde Tanri beni Israil ülkesine götürüp çok yüksek bir dagin üzerine koydu. Dagin güneyinde kente benzer yapilar vardi.
\par 3 Tanri beni oraya götürdü, tunca* benzer bir adam gördüm. Elinde keten ip ve bir ölçü degnegi tutarak kapinin girisinde duruyordu.
\par 4 Bana, "Insanoglu, gözlerinle gör, kulaklarinla isit, sana gösterecegim her seye dikkat et" dedi, "Sen bunun için buraya getirildin. Görecegin her seyi Israil halkina anlat."
\par 5 Tapinagi çepeçevre kusatan bir duvar gördüm. Adamin elindeki ölçü degneginin uzunlugu alti arsindi. Her arsina bir elin eni kadar uzunluk eklenmisti. Adam duvari ölçtü; kalinligi ve yüksekligi bir ölçü degnegi kadardi.
\par 6 Sonra doguya bakan kapiya gitti, basamaklari çikip kapi esigini ölçtü. Eni bir ölçü degnegi kadardi.
\par 7 Bekçi odalarinin her birinin uzunlugu ve genisligi bir ölçü degnegi kadardi. Odalarin arasindaki duvarin kalinligi bes arsindi. Tapinaga bakan eyvanin kapi esigi bir ölçü degnegi uzunluktaydi.
\par 8 Eyvani ölçtü;
\par 9 genisligi sekiz arsin, kapi sövelerinin kalinligi ikiser arsindi. Eyvan tapinaga bakiyordu.
\par 10 Dogu Kapisi'nin her yaninda üçer bekçi odasi vardi. Hepsi ayni ölçüdeydi. Odalar arasindaki duvarlarin ölçüsü de ayniydi.
\par 11 Adam kapinin genisligini ölçtü. Genisligi on, iç girisin genisligi on üç arsindi.
\par 12 Her bekçi odasinin önünde bir arsin yüksekliginde bir duvar vardi. Odalar kare seklindeydi, kenarlari altisar arsindi.
\par 13 Sonra girisleri karsi karsiya olan odalarin arka duvarlarinin arasini ölçtü; yirmi bes arsindi.
\par 14 Sütunlari ölçtü, altmis arsindi. Kapinin çevresindeki avlu sütunlara kadar uzaniyordu.
\par 15 Kapi girisinden eyvanin sonuna kadarki uzaklik elli arsindi.
\par 16 Her iki yandaki bekçi odalarinda, odalar arasindaki duvarlarda ve eyvanin çepeçevre duvarlarinda içe bakan kafesli pencereler vardi. Bölme duvarlari hurma agaci motifleriyle kapliydi.
\par 17 Adam bundan sonra beni dis avluya götürdü. Orada odalar ve dis avluyu çevreleyen tas yol vardi. Tas yol boyunca otuz oda vardi.
\par 18 Girisin iki yanindaki tas yolun genisligi kapilarin uzunlugu kadardi. Bu asagi tas yoldu.
\par 19 Avlunun genisligini asagi giristen iç avlunun girisine dek ölçtü. Dogu ve kuzeydeki uzaklik yüz arsindi.
\par 20 Adam dis avlunun kuzeye bakan kapisinin uzunlugunu ve genisligini ölçtü.
\par 21 Iki yandaki üçer bekçi odasinin, aralarindaki duvarlarin ve eyvanin ölçüsü, birinci kapinin ölçüsünün aynisiydi. Uzunlugu elli arsin, genisligi yirmi bes arsindi.
\par 22 Pencerelerin, eyvanin, hurma agaci motiflerinin ölçüsü, doguya bakan kapinin ölçüsünün aynisiydi. Oraya yedi basamakla çikiliyordu, eyvan bunlarin karsisindaydi.
\par 23 Dogu Kapisi'na oldugu gibi, Kuzey Kapisi'na da bakan bir iç avlu kapisi vardi. Adam bu iki kapi arasindaki uzakligi ölçtü, yüz arsindi.
\par 24 Adam beni güneye dogru götürdü. Orada güneye bakan bir kapi gördüm. Adam kapinin sövelerini ve eyvani ölçtü. Ölçüleri öbürlerinin aynisiydi.
\par 25 Öbürlerinde oldugu gibi, bu kapinin ve eyvanin her yaninda da pencereler vardi. Uzunlugu elli arsin, genisligi yirmi bes arsindi.
\par 26 Oraya yedi basamakla çikiliyordu, eyvan bunlarin karsisindaydi. Iki kapi sövesi de hurma agaci motifleriyle kapliydi.
\par 27 Iç avlunun güneye bakan bir kapisi vardi. Adam bu kapidan güneydeki dis kapiya kadar olan uzakligi ölçtü, yüz arsindi.
\par 28 Adam beni Güney Kapisi'ndan iç avluya götürdü. Güney Kapisi'ni ölçtü. Ölçüleri öbürlerinin aynisiydi.
\par 29 Bekçi odalarinin, odalar arasindaki duvarlarin, eyvanin ölçüleri öbürlerinin aynisiydi. Dis duvarlarda ve eyvanin her yaninda pencereler vardi. Girisin uzunlugu elli arsin, genisligi yirmi bes arsindi.
\par 30 Eyvan dis avluya bakiyordu. Kapi söveleri hurma agaci motifleriyle kapliydi. Oraya sekiz basamakla çikiliyordu.
\par 32 Adam beni dogudaki iç avluya götürdü. Oradaki kapiyi ölçtü. Ölçüleri öbürlerinin aynisiydi.
\par 33 Bekçi odalarinin, odalar arasindaki duvarlarin, eyvanin ölçüleri öbürlerinin aynisiydi. Dis duvarlarda ve eyvanin her yaninda pencereler vardi. Girisin uzunlugu elli arsin, genisligi yirmi bes arsindi.
\par 34 Eyvan dis avluya bakiyordu. Kapi söveleri hurma agaci motifleriyle kapliydi. Oraya sekiz basamakla çikiliyordu.
\par 35 Sonra adam beni Kuzey Kapisi'na götürdü. Kapiyi ölçtü. Ölçüleri öbürlerinin aynisiydi.
\par 36 Bunun da bekçi odalari, aralarindaki duvarlar, eyvani ayniydi. Kapinin her yaninda pencereler vardi. Girisin uzunlugu elli arsin, genisligi yirmi bes arsindi.
\par 37 Eyvan dis avluya bakiyordu. Kapi söveleri her yanda hurma agaci motifleriyle kapliydi. Oraya sekiz basamakla çikiliyordu.
\par 38 Iç avlu girislerindeki eyvanlarin yaninda kapisi eyvana açilan bir oda vardi. Yakmalik sunular* burada yikaniyordu.
\par 39 Eyvanin her iki yaninda ikiser masa vardi. Yakmalik sunu, günah sunusu* ve suç sunusu* için hayvanlar bu masalarin üzerinde kesiliyordu.
\par 40 Eyvanin dis duvarinin yaninda, Kuzey Kapisi'nin basamaklarinin her iki yaninda ikiser olmak üzere dört masa daha vardi.
\par 41 Böylece kurbanlik hayvanlarin kesimi için kapinin her iki yaninda dörder olmak üzere sekiz masa vardi.
\par 42 Yakmalik sunular için yontma tastan dört masa vardi. Her masanin uzunlugu ve genisligi birer buçuk arsin, yüksekligi bir arsindi. Yakmalik sunularla öbür kurbanlarin kesiminde kullanilan aletleri bunlarin üzerine koyuyorlardi.
\par 43 Odanin duvarlarina çifte çengeller asilmisti; her biri bir el genisligindeydi. Masalar sunulacak kurban eti için kullaniliyordu.
\par 44 Iç kapinin dis bölümünde, iç avluda iki oda vardi. Bunlardan biri Kuzey Kapisi'nin yanindaydi ve güneye bakiyordu, öbürü Güney Kapisi'nin yanindaydi ve kuzeye bakiyordu.
\par 45 Adam bana, "Güneye bakan oda tapinakta hizmet görecek kâhinler için" dedi,
\par 46 "Kuzeye bakan oda da sunakta hizmet görecek kâhinler için. Bunlar Levi soyundan, RAB'be hizmet etmek için O'na yaklasan Sadokogullari'dir."
\par 47 Adam avluyu ölçtü. Kareydi, uzunlugu yüz arsin, genisligi yüz arsindi. Sunak tapinagin önündeydi.
\par 48 Adam sonra beni tapinagin eyvanina götürüp eyvanin kapi sövelerini ölçtü. Her iki yandaki sövelerin genisligi beser arsindi. Girisin genisligi on dört arsin, iki yandaki duvarlarin genisligi de üçer arsindi.
\par 49 Eyvanin uzunlugu yirmi arsin, genisligi on iki arsindi. Oraya basamaklarla çikiliyordu. Kapi sövelerinin her bir yaninda sütunlar vardi.

\chapter{41}

\par 1 Bundan sonra adam beni tapinagin ana bölümüne götürüp kapi sövelerini ölçtü. Sövelerin genisligi her yandan alti arsindi.
\par 2 Girisinin genisligi on arsin, her yandan buna bagli duvarlarin genisligi beser arsindi. Ana bölümü de ölçtü. Uzunlugu kirk arsin, genisligi yirmi arsindi.
\par 3 Sonra iç odaya gidip girisin sövelerini ölçtü. Her biri iki arsin genisligindeydi. Girisin genisligi alti arsin, her yandan buna bagli duvarlarin genisligi yedi arsindi.
\par 4 Ana bölümün ötesindeki iç odayi ölçtü. Uzunlugu ve genisligi yirmiser arsindi. Adam, "Bu En Kutsal Yer'dir*" dedi.
\par 5 Tapinagin duvarini ölçtü, kalinligi alti arsindi. Tapinagin çevresindeki her yan odanin genisligi dört arsindi.
\par 6 Bu yan odalar üç katti, her katta otuz oda vardi. Tapinagin duvarlari boyunca yan odalara destek görevi yapan çikintilar vardi. Öyle ki, destekler tapinak duvarlarina girmesin.
\par 7 Tapinagin çevresindeki yan odalar yukari kata dogru çiktikça genisliyordu. Tapinagin çevresindeki yapinin yukariya çikan bir merdiveni vardi. Yukariya dogru çikildikça yan odalar genisliyordu. Merdivenle alt kattan orta kata, oradan da üst kata çikiliyordu.
\par 8 Tapinagin çevresinde yan odalarin temelini olusturan yüksek bir kaldirim gördüm. Uzunlugu bir degnek kadar, yani alti arsindi.
\par 9 Yan odalarin dis duvarinin kalinligi bes arsindi. Tapinagin yan odalari ile kâhin odalari arasindaki açik alanin genisligi tapinak çevresi boyunca yirmi arsindi.
\par 11 Yan odalarin girisi açik alana bakiyordu; biri kuzeyde, öbürü güneydeydi. Açik alana bitisik temelin genisligi her yandan bes arsindi.
\par 12 Tapinagin batisinda açik alana bakan bir yapi vardi. Genisligi yetmis arsindi; duvarinin kalinligi her yandan bes arsin, uzunlugu doksan arsindi.
\par 13 Bundan sonra adam tapinagi ölçtü. Uzunlugu yüz arsindi. Tapinagin açik alani, yapi ve duvarlari yüz arsin uzunluktaydi.
\par 14 Doguda tapinagin açik alaninin tapinagin önüyle birlikte genisligi yüz arsindi.
\par 15 Adam tapinagin arkasindaki açik alana bakan yapinin iki yanindaki koridorlarin uzunlugunu ölçtü; yüz arsindi. Ana bölüm, iç oda, avluya bakan eyvan,
\par 16 kapi esikleri, kafesli pencereler, esigin karsisindaki üç kati çevreleyen koridorlar tabandan pencerelere dek agaç kapliydi. Pencereler açilip kapanabiliyordu.
\par 17 Girisin üstü, iç oda, disarisi ve bütün iç ve dis duvarlar düzenli araliklarla
\par 18 Keruv* ve hurma agaci motifleriyle kapliydi. Iki Keruv arasinda bir hurma agaci vardi. Her Keruv'un iki yüzü vardi:
\par 19 Bir yanda hurma agacina bakan insan yüzü, öbür yanda hurma agacina bakan genç aslan yüzü. Tapinak çepeçevre Keruv ve hurma agaci oymalariyla bezenmisti.
\par 20 Tabandan girisin üstündeki bölüme dek ana bölümün duvarlari Keruv ve hurma agaci oymalariyla kapliydi.
\par 21 Ana bölümün kapi söveleri kare seklindeydi, En Kutsal Yer'in önündeki kapi söveleri bunlara benziyordu.
\par 22 Üç arsin yüksekliginde, iki arsin uzunlugunda agaçtan yapilmis bir sunak vardi. Köseleri, ayaklari, yanlari agaçtandi. Adam bana, "RAB'bin önündeki masa budur" dedi.
\par 23 Ana bölümün ve En Kutsal Yer'in çift kanatli birer kapisi vardi.
\par 24 Her kapinin iki menteseli kanadi vardi.
\par 25 Duvarlara oldugu gibi, ana bölümün kapilarina da Keruv ve hurma agaci oymalari yapilmisti. Disarda, eyvanin önünde agaçtan bir asma tavan vardi.
\par 26 Eyvanin yan duvarlarindaki kafesli pencerelerin iki yani hurma agaci oymalariyla kapliydi. Tapinagin yan odalariyla asma tavanlari böyleydi.

\chapter{42}

\par 1 Adam beni kuzeye giden yoldan dis avluya çikardi. Tapinagin açik alanina ve dis avlunun kuzeyindeki yapilara bakan odalara götürdü.
\par 2 Kapisi kuzeye bakan bu yapinin uzunlugu yüz arsin, genisligi elli arsindi.
\par 3 Iç avlunun yirmi arsinlik bölümüyle dis avlunun tas yoluna bakan üç katin koridorlari karsi karsiyaydi.
\par 4 Odalarin önünde genisligi on arsin, uzunlugu yüz arsin olan bir iç koridor vardi. Kapilari kuzeye bakiyordu.
\par 5 Yapinin üst kattaki odalari alt ve orta kattaki odalardan daha dardi. Çünkü üst kattaki koridorlar daha çok yer kapliyordu.
\par 6 Avlularda sütunlar olmasina karsin, üçüncü kattaki odalarin sütunlari yoktu. Bu yüzden bu odalar alt ve orta kattaki odalardan daha dardi.
\par 7 Odalarin önünde, odalara ve dis avluya paralel bir dis duvar vardi, elli arsin uzunluktaydi.
\par 8 Dis avlu yanindaki sira odalarin uzunlugu elli arsinken, ana bölüme daha yakin sira odalarin uzunlugu yüz arsindi.
\par 9 Alt kattaki odalarin dis avludan girilecek gibi dogu yönünde bir girisleri vardi.
\par 10 Iç avlunun güneyi boyunca, açik alana ve dis avludaki yapilara bakan baska odalar vardi.
\par 11 Kuzeydeki odalarda oldugu gibi, bu odalarin önünde de bir geçit vardi. Odalarin uzunluklari, genislikleri ayniydi, çikislari ve boyutlari kuzeydeki odalara benziyordu. Güneydeki odalarin girisleri kuzeydekiler gibiydi. Geçidin baslangicinda bir giris vardi. Arka duvarlar boyunca doguya uzanan bu geçit odalara açiliyordu.
\par 13 Bundan sonra adam, "Tapinagin açik alanina bakan kuzey ve güneydeki odalar kutsaldir" dedi, "RAB'bin önünde hizmet eden kâhinler orada en kutsal sunulardan yiyecekler. En kutsal sunulari -tahil, günah ve suç sunularini*- oraya koyacaklar. Çünkü orasi kutsaldir.
\par 14 Kâhinler kutsal alana girdikten sonra, hizmet ederken giydikleri giysileri orada birakmadan dis avluya çikmayacaklar. Çünkü bu giysiler kutsaldir. Halkin bulundugu yerlere gitmeden önce baska giysiler giymeliler."
\par 15 Adam iç tapinagi ölçmeyi bitirince, beni Dogu Kapisi'ndan disariya götürdü, o alani her yandan ölçtü.
\par 16 Dogu yanini ölçü degnegiyle ölçtü, bes yüz arsin kadardi.
\par 17 Kuzey yanini ölçtü, bes yüz arsin*fs* kadardi.
\par 18 Güney yanini ölçtü, bes yüz arsin*fs* kadardi.
\par 19 Sonra batiya dönüp ölçtü, bes yüz arsin*fs* kadardi.
\par 20 Böylece alanin dört yanini ölçtü. Kutsal olani kutsal olmayandan ayirmak için alanin çevresinde bir duvar vardi; uzunlugu ve genisligi beser yüz arsindi.

\chapter{43}

\par 1 Adam beni doguya bakan kapiya götürdü.
\par 2 Israil Tanrisi'nin görkeminin dogudan geldigini gördüm. Sesi gürül gürül akan sularin sesi gibiydi. Görkeminden yeryüzü aydinlikla doldu.
\par 3 Gördügüm görüm, Tanri kenti yok etmeye geldiginde ve Kevar Irmagi kiyisinda gördügüm görümlere benziyordu. Yüzüstü yere düstüm.
\par 4 RAB'bin görkemi doguya bakan kapidan tapinaga girdi.
\par 5 Ruh beni ayaga kaldirip iç avluya götürdü. RAB'bin görkemi tapinagi doldurdu.
\par 6 Adam orada yanimda dururken, tapinaktan birinin bana seslendigini duydum.
\par 7 Bana söyle dedi: "Insanoglu, tahtimin yeri, ayaklarimin basacagi, Israil halkiyla sonsuza dek yasayacagim yer burasidir. Bundan böyle Israil halki da krallari da fahiselikleriyle ve krallarinin cesetleriyle bir daha kutsal adimi kirletmeyecek.
\par 8 Onlar kapi esiklerini kapi esigimin, sövelerini sövelerimin bitisigine yerlestirdiler. Benimle aralarinda yalnizca bir duvar vardi. Igrenç uygulamalariyla kutsal adimi kirlettiler. Bu yüzden öfkemle onlari yok ettim.
\par 9 Simdi fahiseliklerini, krallarinin cesetlerini benden uzaklastirsinlar; ben de sonsuza dek aralarinda yasayayim.
\par 10 "Insanoglu, günahlarindan utanmalari için bu tapinagi Israil halkina tanit. Tapinagin tasarisini incelesinler.
\par 11 Eger bütün yaptiklarindan utaniyorlarsa, tapinagin tasarini -düzenlemesini, girislerini, çikislarini- kurallarini, yasalarini onlara bildir. Tasari onlarin gözü önünde yaz ki, bütün düzenine, kurallarina baglilikla uyabilsinler.
\par 12 Tapinakla ilgili yasa sudur: Dagin tepesinde tapinagi çevreleyen bütün alan çok kutsal olacak. Iste tapinakla ilgili yasa böyle.
\par 13 "Arsin ölçüsüyle sunagin ölçüleri sunlardir: -Bu arsin, bir arsina ek olarak bir elin eni kadardir.- Sunagi çevreleyen hendegin derinligi bir arsin, genisligi bir arsin, çevresindeki kenarlik bir karis. Sunagin yüksekligiyse söyle:
\par 14 Sunagin yerdeki hendekten alt çikintiya kadarki bölümünün yüksekligi iki arsin, genisligi bir arsin, küçük çikintidan büyük çikintiya kadarki bölümün yüksekligi dört arsin, genisligi bir arsin.
\par 15 Sunagin kurban yakilan üst bölümünün yüksekligi dört arsin; üst bölümden yukari dogru dört boynuz uzanacak.
\par 16 Sunagin üst bölümü kare seklinde olacak. Uzunlugu on iki arsin, genisligi on iki arsin.
\par 17 Üst çikintinin dört yandan uzunlugu ve genisligi de on dörder arsin. Çevresindeki kenarlik yarim arsin, hendegin çevresi bir arsin. Sunagin basamaklari doguya bakacak."
\par 18 Adam konusmasini söyle sürdürdü: "Insanoglu, Egemen RAB söyle diyor: 'Sunak yapilacagi gün, üzerinde yakmalik sunular* sunmak ve kan dökmek için kurallar sunlardir:
\par 19 Bana hizmet etmek üzere önüme gelen Sadok soyundan Levili kâhinlere günah sunusu* olarak bir boga vereceksin. Egemen RAB böyle diyor.
\par 20 Boganin kanindan biraz alip sunagin dört boynuzuna, çikintinin dört kösesine ve çevresindeki kenarligin üzerine süreceksin. Böylece sunagi pak kilip arindiracaksin.
\par 21 Bogayi günah sunusu olarak alacak, tapinagin disinda, tapinak alaninda belirlenen yerde yakacaksin.
\par 22 "'Ikinci gün günah sunusu olarak kusursuz bir teke sunacaksin. Sunagi boganin kaniyla arindirdigin gibi tekenin kaniyla da arindir.
\par 23 Arindirma islemini bitirince, sürüden kusursuz bir bogayla bir koç sunacaksin.
\par 24 Bunlari RAB'bin önüne getireceksin. Kâhinler üzerlerine tuz serpip yakmalik sunu olarak RAB'be sunacaklar.
\par 25 "'Yedi gün boyunca günah sunusu olarak her gün bir teke saglayacaksin; kusursuz bir bogayla sürüden bir koç da saglayacaksin.
\par 26 Yedi gün sunagi arindirip pak kilacaklar. Böylece sunak adanmis olacak.
\par 27 Yedi gün bitince, kâhinler sekizinci gün ve daha sonra yakmalik ve esenlik sunularinizi* sunagin üzerinde sunacak. O zaman sizi kabul edecegim. Egemen RAB böyle diyor."

\chapter{44}

\par 1 Bundan sonra adam beni tapinagin doguya bakan dis kapisina geri getirdi. Kapi kapaliydi.
\par 2 RAB bana, "Bu kapi kapali kalacak, açilmayacak, buradan kimse girmeyecek!" dedi, "Israil'in Tanrisi RAB bu kapidan girdi, bu yüzden kapali kalacak.
\par 3 Yalniz önder -önder oldugu için- RAB'bin önünde oturup ekmek yemek üzere eyvandan girebilir, ayni yoldan da çikabilir."
\par 4 Adam Kuzey Kapisi yolundan tapinagin önüne getirdi beni. Baktim, RAB'bin görkeminin tapinagi doldurdugunu gördüm. Yüzüstü yere düstüm.
\par 5 RAB bana söyle seslendi: "Insanoglu, RAB'bin Tapinagi'nin bütün kurallari ve yasalariyla ilgili söyleyeceklerimi iyi dinle, her seye iyi bak, kulak ver. Tapinaga kimin girip çikacagina dikkat et.
\par 6 Asi Israil halkina de ki, 'Egemen RAB söyle diyor: Ey Israil halki, yaptiginiz igrençliklere bir son verin artik!
\par 7 Yüregi ve bedeni sünnet edilmemis yabancilari tapinagima aldiniz, bana yiyecek olarak yag, kan sunmakla tapinagimi kirlettiniz. Böylece igrenç uygulamalarinizla antlasmami bozdunuz.
\par 8 Kutsal esyalarima iliskin sorumlulugunuzu yerine getirmediniz. Tapinagimda bu esyalara bakmalari için baskalarini görevlendirdiniz.
\par 9 Egemen RAB söyle diyor: Yüregi ve bedeni sünnet edilmemislerden, Israil halki arasinda yasayan yabancilardan hiçbiri tapinagima girmeyecek.
\par 10 "'Israil kötü yola saptigi zaman beni birakan, yoldan sapip putlarina baglanan Levililer'se günahlarinin cezasini çekecekler.
\par 11 Ama tapinagimda onlar hizmet edecek: Tapinagin kapilarindan sorumlu olacaklar; tapinagin hizmetini yapacak, yakmalik sunu* ve kurbanlik hayvanlari halk için kesecek, halkin önünde duracak, halka hizmet edecekler.
\par 12 Putlarinin önünde Israil halkina hizmet ederek halki günaha soktular. Bu nedenle ben RAB onlari günahlari yüzünden cezalandiracagima ant içtim. Egemen RAB böyle diyor.
\par 13 Kâhin olarak hizmet etmek üzere bana yaklasmayacaklar. Kutsal esyalarima, en kutsal sunularima dokunmayacaklar. Igrenç uygulamalarinin utancini yüklenecekler.
\par 14 Yine de tapinagin hizmeti ve orada yapilacak bütün isler için onlari görevlendirecegim.
\par 15 "'Ancak Israil beni birakip kötü yola saptiginda tapinagimin hizmetini sadakatle yapan Sadok soyundan Levili kâhinler önümde hizmet etmek üzere bana yaklasacak. Yag ve kan sunularini sunmak için önümde onlar duracak. Böyle diyor Egemen RAB.
\par 16 Yalniz onlar girecek tapinagima; önümde hizmet etmek için yalniz onlar soframa yaklasacak, görev yapacaklar.
\par 17 "'Kâhinler iç avlunun kapilarindan girecekleri zaman keten giysi giyecek; iç avlunun kapilarinda ya da tapinakta hizmet ederken yünlü giysi giymeyecekler.
\par 18 Baslarina keten sarik saracak, keten don giyecekler. Kendilerini terletecek bir sey giymeyecekler.
\par 19 Dis avluya halkin yanina çikmadan önce, hizmet ederken giydikleri giysileri çikarip kutsal odalara koyacak, baska giysiler giyecekler. Öyle ki, o giysilerin kutsalligini halka geçirmesinler.
\par 20 "'Kâhinler baslarini tiras etmeyecek, saçlarini uzatmayacaklar. Ancak saçlarini kesip düzeltecekler.
\par 21 Iç avluya girecegi zaman hiçbir kâhin içki içmeyecek.
\par 22 Kâhinler dul ya da bosanmis kadinla evlenmeyecek. Israil soyundan erden bir kizla ya da baska bir kâhinden dul kalmis bir kadinla evlenebilirler.
\par 23 Kutsalla bayagi arasindaki ayrimi halkima onlar ögretecek, kirliyle temizi ayirt etmeyi onlar gösterecekler.
\par 24 "'Davalarda yargiç olarak kâhinler görev yapacak, ilkelerim uyarinca karar verecekler. Bayramlarimla ilgili yasalarima, kurallarima uyacak, Sabat* günlerimi kutsal tutacaklar.
\par 25 "'Kâhin bir ölünün yanina giderek kendini kirletmeyecek; ölü annesi, babasi, oglu, kizi, kardesi ya da evlenmemis kizkardesiyse kendini kirletebilir.
\par 26 Arindiktan sonra yedi gün bekleyecek.
\par 27 Tapinakta hizmet etmek üzere iç avluya girecegi gün, kendisi için bir günah sunusu* sunacak. Egemen RAB böyle diyor.
\par 28 "'Kâhinlerin payi vardir, onlarin mirasi benim. Israil'de onlara mülk vermeyeceksiniz. Onlarin mirasi benim.
\par 29 Kâhinler tahil, günah ve suç sunularini* yiyecekler. Israil'de RAB'be adanan her sey onlarin olacak.
\par 30 Ilk ürünlerin en iyileri ve bütün özel armaganlariniz kâhinlerin olacak. Evinize bereket yagsin diye tahilinizin ilkini onlara vereceksiniz.
\par 31 Kâhinler ölü bulunmus ya da yabanil hayvan tarafindan parçalanmis hiçbir kus ya da hayvan yemeyecek."

\chapter{45}

\par 1 "'Ülkeyi mülk olarak paylastirdiginiz zaman, RAB'be ülkeden pay olarak 25 000 arsin uzunlukta, 20 000 arsin genislikte kutsal bir bölge ayiracaksiniz. Bütün bu bölge kutsal olacak.
\par 2 Uzunlugu ve genisligi 500 arsinlik bir bölüm kutsal yer için, 50 arsinlik bir yer de çevresindeki alan için ayrilacak.
\par 3 Bu bölgeden uzunlugu 25 000 arsinlik, genisligi 10 000 arsinlik bir bölüm ölçeceksiniz. Tapinak, En Kutsal Yer* orada olacak.
\par 4 Burasi tapinakta hizmet etmek üzere RAB'be yaklasan kâhinlere ayrilacak ve ülkenin kutsal payi olacak. Kâhinlerin evleri de tapinak da o kutsal bölgede olacak.
\par 5 Tapinakta hizmet eden Levililer'e miras olarak 25 000 arsin uzunlukta, 10 000 arsin genislikte bir bölge verilecek. Orada, yasamalari için kendilerine ait kentler olacak.
\par 6 "'Kutsal bölgeye düsen payla birlikte kent için uzunlugu 25 000, genisligi 5 000 arsinlik bir pay ayiracaksiniz; bu bütün Israil halki için olacak.
\par 7 "'Kutsal bölgeye düsen pay ile kente düsen payin iki yanindaki topraklar öndere verilecek. Batidan batiya, dogudan doguya dogru uzanacak. Bati sinirindan dogu sinirina dek uzunlugu bir Israil oymagina düsen pay kadardir.
\par 8 Bu toprak Israil'de önderin payi olacak. Bundan böyle önderlerim halkima bir daha baski yapmayacak, ama oymaklarina göre Israil halkina ülkeyi miras olarak verecekler.
\par 9 "'Egemen RAB söyle diyor: Yeter artik, ey Israil önderleri! Zorbaligi, baskiyi birakin. Adil ve dogru olani yapin. Halkimi kendi topraklarindan kovmayin. Egemen RAB böyle diyor.
\par 10 Dogru ölçüler kullanin, kullandiginiz efa ve bat dogru olsun.
\par 11 Efa ile bat ayni ölçüde olsun. Bat homerin onda birine, efa da homerin onda birine esit olmali. Ikisinin de ölçüsü homere göre olacak.
\par 12 Bir sekel yirmi geraya esit olmali. Altmis sekel de bir minaya esit olmali."
\par 13 "'Sunacaginiz sunular sunlardir: Her homer bugdaydan efanin altida biri, her homer arpadan efanin altida biri kadarini vereceksiniz.
\par 14 Bat ölçüsüne göre istenen zeytinyagi miktari, her kordan batin onda biri kadardir. Bir kor on bat ya da bir homere esittir.
\par 15 Israil'in sulak otlaklarindaki sürüden iki yüz koyundan bir koyun alinacak. Halkin günahlarini bagislatmak için bu koyunlar yakmalik sunular*, tahil ve esenlik sunulari* için kullanilacak. Egemen RAB böyle diyor.
\par 16 Ülke halki bu armaganlari Israil'deki öndere verecek.
\par 17 Israil'de kutlanan bütün bayramlarda -senliklerde, Yeni Ay törenlerinde, Sabat* günlerinde- tahil sunularini*, yakmalik ve dökmelik sunulari önder saglayacak. Israil halkinin günahlarini bagislatmak için yakmalik sunulari, günah, tahil, esenlik sunularini saglayacak.
\par 18 "'Egemen RAB söyle diyor: Birinci ayin* birinci günü kusursuz bir boga alacak, tapinagi arindiracaksin.
\par 19 Kâhin günah sunusunun* kanindan alip tapinagin kapi sövelerine, sunagin üst çikintisinin dört kösesine, iç avlunun kapi sövelerine sürecek.
\par 20 Yanlislikla ya da bilgisizlikten günah isleyen biri için ayin yedinci günü aynisini yapacaksin. Böylece tapinagi arindiracaksin.
\par 21 "'Birinci ayin on dördüncü günü Fisih Bayrami'ni* yedi gün kutlayacak, mayasiz ekmek yiyeceksiniz.
\par 22 O gün önder kendisi ve ülke halki için günah sunusu olarak bir boga saglayacak.
\par 23 Yedi gün bayram boyunca her gün RAB'be yakmalik sunu* olarak kusursuz yedi bogayla yedi koç, günah sunusu olarak da bir teke saglayacak.
\par 24 Tahil sunusu olarak her boga ve koç için birer efa tahil, her efa için bir hin zeytinyagi saglayacak.
\par 25 "'Yedinci ayin on besinci günü baslayan ve yedi gün süren bayramda önder yakmalik sunulari, günah ve tahil sunularini, zeytinyagini her gün ayni miktarda saglayacak.

\chapter{46}

\par 1 "'Egemen RAB söyle diyor: Iç avlunun doguya bakan kapisi alti çalisma günü kapali, Sabat Günü* ve Yeni Ay Günü ise açik kalacak.
\par 2 Önder disaridan eyvana girip kapi sövesinin yaninda duracak. Kâhinler onun yakmalik ve esenlik sunularini sunacaklar. Önder kapi esiginde tapindiktan sonra çikip gidecek. Kapi aksama dek açik kalacak.
\par 3 Sabat günleri ve Yeni Ay törenlerinde ülke halki bu kapinin girisinde RAB'bin önünde tapinacak.
\par 4 Önder Sabat Günü RAB'be sunacagi yakmalik sunu olarak kusursuz alti kuzu, bir koç sunacak.
\par 5 Koç için verilecek tahil sunusu bir efa tahil olacak, kuzular için verebilecegi kadar tahil sunusu sunabilir. Her efa tahil için bir hin zeytinyagi verilecek.
\par 6 Yeni Ay Günü kusursuz bir boga, alti kuzu ve bir koç sunacak.
\par 7 Boga ve koç için tahil sunusu olarak birer efa*fj* tahil saglayacak; kuzular için istedigi kadar tahil saglayabilir. Her efa tahil için bir hin*fk* zeytinyagi saglayacak.
\par 8 Önder içeri girecegi zaman eyvandan girecek ve ayni yoldan disari çikacak.
\par 9 "'Ülke halki bayramlarda RAB'bin önüne geldiginde, tapinmak için Kuzey Kapisi'ndan giren Güney Kapisi'ndan çikacak, Güney Kapisi'ndan giren Kuzey Kapisi'ndan çikacak. Hiç kimse girdigi kapidan çikmayacak. Herkes girdigi kapinin karsisindaki kapidan çikacak.
\par 10 Önder halkin arasinda olacak. Halkla birlikte girecek, halkla birlikte çikacak.
\par 11 "'Bayramlarda ve kutsal günlerde boga ve koç için tahil sunusu olarak birer efa*fj* tahil verecek; kuzular için verebilecegi kadar tahil saglayabilir. Her efa tahil için bir hin*fk* zeytinyagi verecek.
\par 12 Önder RAB'be gönülden verilen yakmalik sunular ya da esenlik sunulari sunacagi zaman doguya bakan kapi kendisine açilacak. Yakmalik sunulari ya da esenlik sunularini Sabat Günü sundugu gibi sunacak. Sonra disari çikacak; o çiktiktan sonra kapi kapanacak.
\par 13 "'Her gün, her sabah yakmalik sunu olarak RAB'be bir yasinda kusursuz bir kuzu saglayacaksin.
\par 14 Bununla birlikte her sabah tahil sunusu olarak efanin altida biri tahil ve ince unu islatmak için bir hinin üçte biri kadar zeytinyagi saglayacaksin. Bu tahil sunusunun RAB'be sunulmasi sürekli bir kural olacak.
\par 15 Böylece günlük yakmalik sunu olarak her sabah kuzu, tahil sunusu ve zeytinyagi sunulacak."
\par 16 "'Egemen RAB söyle diyor: Eger önder ogullarindan birine kendi mülkünden armagan ederse, bu mülk torunlarina da geçecek. Miras yoluyla bu onlarin mülkü olacak.
\par 17 Önder görevlilerinden birine kendi mülkünden armagan ederse, görevli toprak parçasini özgürlük yilina dek elinde tutacak. Sonra öndere geri verecek. Önderin mirasi ancak ogullarina geçebilir, onlarin olacak.
\par 18 Önder halki mülkünden kovarak miraslarindan etmemeli. Ogullarina ancak kendi mülkünden miras verebilir. Öyle ki, halkimdan hiç kimse mülkünden ayrilip dagilmasin."
\par 19 Bundan sonra adam beni kapi yanindaki giristen kuzeye bakan, kâhinlere ait kutsal odalara getirdi. Bana batida bir yer gösterdi.
\par 20 "Kâhinlerin suç sunusuyla* günah sunusunun* etini haslayacaklari, tahil sunusunu* pisirecekleri yer burasi" dedi, "Öyle ki, bunlari dis avluya çikarip kutsalliklarini halka geçirmesinler."
\par 21 Daha sonra adam beni dis avluya çikarip sirayla avlunun dört kösesine götürdü. Avlunun her kösesinde küçük birer avlu oldugunu gördüm.
\par 22 Dis avlunun dört kösesinde kirk arsin uzunlugunda, otuz arsin genisliginde birer kapali avlu vardi. Köselerdeki avlularin ölçüsü ayniydi.
\par 23 Dört avlunun çevresinde de tas duvar vardi; duvarin dibinde yemek pisirmek için yerler yapilmisti.
\par 24 Bana, "Bunlar tapinakta hizmet edenlerin halkin sundugu kurban etini pisirecekleri mutfaklar" dedi.

\chapter{47}

\par 1 Adam beni tapinagin girisine geri getirdi. Doguya dogru tapinagin kapi esiginin altindan sular aktigini gördüm. Tapinak doguya bakiyordu. Sular tapinagin güney yaninin altindan, sunagin güneyinden asagiya akiyordu.
\par 2 Beni oradan, Kuzey Kapisi'ndan çikarip dis yoldan doguya bakan dis kapiya götürdü. Sular güney yönünden akiyordu.
\par 3 Adam elinde bir ölçü ipiyle doguya dogru gitti. Bin arsin ölçtükten sonra beni ayak bilegine dek çikan sulara getirdi.
\par 4 Bin arsin daha ölçtü ve beni dize kadar çikan sulara getirdi. Bin arsin daha ölçtü, beni bele kadar çikan sulara getirdi.
\par 5 Bin arsin daha ölçtü, içinden geçemedigim bir irmak olustu. Sular yükselmisti, içinden yürüyerek karsiya geçilemezdi, yüzülecek kadar derin bir irmak olusmustu.
\par 6 Bana, "Insanoglu, bunu gördün mü?" diye sordu. Daha sonra beni irmagin kiyisina geri getirdi.
\par 7 Oraya varinca, irmagin her iki kiyisinda birçok agaç gördüm.
\par 8 Bana söyle dedi: "Bu sular dogu bölgesine dogru akiyor, oradan Arava Vadisi'ne, sonra Lut Gölü'ne dökülüyor. Göle dökülünce oradaki sular tatli suya dönüsecek.
\par 9 Irmagin aktigi yerlerde her çesit canli yaratik kaynasacak. Çok sayida balik olacak. Çünkü bu sular oraya akiyor, oradaki tuzlu suyu tatli suya dönüstürüyor. Irmak aktigi her yere yasam getirecek.
\par 10 Irmak kiyisi boyunca balikçilar duracak; Eyn-Gedi'den Eyn- Eglayim'e dek ag gerecek yerler olacak. Akdeniz'deki gibi çok sayida balik çesidi olacak.
\par 11 Ama Lut Gölü'nün çamurlu, bataklik kesimi tatli suya dönüsmeyecek, tuzla olarak kalacak.
\par 12 Irmagin her iki yaninda her çesit meyve agaci yetisecek. Yapraklari solmayacak, meyveleri tükenmeyecek. Her ay meyve verecekler, çünkü tapinaktan çikan sular oraya akiyor. Meyveleri yiyecek olarak, yapraklari sifa için kullanilacak."
\par 13 Egemen RAB söyle diyor: "Ülkeyi mülk olarak Israil'in on iki oymagina böleceginiz sinirlar söyle olacak: Yusuf'a iki pay düsecek.
\par 14 Ülkeyi on iki oymak arasinda esit olarak paylasacaksiniz. Ülkeyi atalariniza verecegime ant içtim. Bu ülke size mülk olarak verilecek.
\par 15 "Ülkenin siniri söyle olacak: Kuzeyde Akdeniz'den, Hetlon yoluyla Levo-Hamat'a, Sedat'a,
\par 16 Berota'ya ve Sam'la Hama'nin topraklari arasinda bulunan Sivrayim'e, Havran sinirinda Haser-Hattikon'a kadar uzanacak.
\par 17 Sinir denizden Hasar-Enan'a, Sam'in kuzey siniri boyunca uzanacak, Hama siniri kuzeyde olacak. Kuzey siniri bu olacak.
\par 18 "Doguda sinir Havran'la Sam arasinda Gilat'i Israil'den ayiran Seria Irmagi boyunca Lut Gölü'ne ve Tamar'a dek uzanacak. Dogu siniri bu olacak.
\par 19 "Güneyde sinir Tamar'dan Meriva-Kades sularina, Misir Vadisi boyunca Akdeniz'e dek uzanacak. Güney siniri bu olacak.
\par 20 "Batida Levo-Hamat'in karsisindaki noktaya dek Akdeniz sinir olusturacak. Bati siniri bu olacak.
\par 21 "Bu ülkeyi Israil oymaklarina göre aranizda paylasacaksiniz.
\par 22 Ülkeyi içinizde yasayan ve içinizdeyken çocuklari olan yabancilarla kendiniz arasinda mülk olarak bölüseceksiniz. Onlari Israil'de dogan yerliler sayacaksiniz. Onlarin da Israil oymaklari arasinda sizin gibi mülkleri olacak.
\par 23 Yabanci hangi oymaga yerlesmisse, orada ona düsen payi mülk olarak vereceksiniz." Egemen RAB böyle diyor.

\chapter{48}

\par 1 "Oymaklarin adlari sunlardir: Kuzey sinirinda Dan'a bir pay verilecek. Dan siniri Hetlon yolundan Levo-Hamat'a uzanacak; Hasar-Enan ve Hama'ya yakin Sam'in kuzey siniri dogudan batiya uzanan sinirin bir bölümünü olusturacak.
\par 2 "Aser'e bir pay verilecek; siniri Dan'in dogudan batiya uzanan sinirina bitisik olacak.
\par 3 "Naftali'ye bir pay verilecek; siniri Aser'in dogudan batiya uzanan sinirina bitisik olacak.
\par 4 "Manasse'ye bir pay verilecek; siniri Naftali'nin dogudan batiya uzanan sinirina bitisik olacak.
\par 5 "Efrayim'e bir pay verilecek; siniri Manasse'nin dogudan batiya uzanan sinirina bitisik olacak.
\par 6 "Ruben'e bir pay verilecek; siniri Efrayim'in dogudan batiya uzanan sinirina bitisik olacak.
\par 7 "Yahuda'ya bir pay verilecek; siniri Ruben'in dogudan batiya uzanan sinirina bitisik olacak.
\par 8 "Yahuda'nin dogudan batiya uzanan sinirina bitisik topraklar, RAB'be ayiracaginiz özel armagan olacak. Genisligi 25 000 arsin, dogudan batiya uzunlugu bir oymaga düsen pay kadar olacak. Tapinak bunun ortasinda olacak.
\par 9 "RAB'be özel olarak sunacaginiz payin uzunlugu 25 000 arsin, genisligi 10 000 arsin olacak.
\par 10 Bu kâhinler için kutsal pay olacak. Kuzeyde uzunlugu 25 000 arsin, batida genisligi 10 000 arsin, doguda genisligi 10 000 arsin, güneyde uzunlugu 25 000 arsin olacak. RAB'bin Tapinagi bunun ortasinda olacak.
\par 11 Bu bölge Sadok soyundan gelen kutsanmis kâhinler için olacak. Onlar bana baglilikla hizmet ettiler, Israil halki yoldan saptiginda Levililer de yoldan sapti, ama onlar sapmadi.
\par 12 Orasi ülkenin kutsal payindan özel bir armagan olarak onlara verilecek. Levililer'in topraklarina bitisik çok kutsal bir bölge olacak.
\par 13 "Levililer'in kâhinlerin siniri yakininda 25 000 arsin uzunlukta, 10 000 arsin genislikte bir paylari olacak. Bu bölgenin uzunlugu 25 000 arsin, genisligi 10 000 arsin olacak.
\par 14 Levililer orayi satmayacak, degis tokus etmeyecekler. Bu, ülkenin en iyi bölümüdür, baskasinin eline geçmemeli. Çünkü orasi RAB'be adanmistir.
\par 15 "Bölgenin geri kalan 25 000 arsin uzunlukta, 5 000 arsin genislikteki bölümü halkin yerlesmesi içindir. Orasi evlere, otlaklara ayrilacak. Kent bunun ortasinda kurulacak.
\par 16 Ölçüleri söyle olacak: Kuzeyde, güneyde, doguda, batida 4 500'er arsin.
\par 17 Kent için ayrilan otlak da kuzeyde, güneyde, doguda, batida 250'ser arsin olacak.
\par 18 Kutsal bölgenin sinirinda kalan yerin dogusu 10 000 arsin, batisi 10 000 arsin olacak. Kutsal bölgeye bitisik topraklarda yetisen ürün kentte çalisanlarin olacak.
\par 19 Israil'in her oymagindan kentte çalisanlar topragi isleyecekler.
\par 20 Bu bölgenin tamami kare seklindedir. Her yani 25 000 arsindir. Özel bir armagan olarak kentin mülküyle birlikte kutsal bölgeye düsen payi ayiracaksiniz.
\par 21 "Kutsal bölgeye düsen pay ile kente düsen payin iki yanindaki topraklar öndere verilecek. Bu topraklar kutsal bölgeye düsen 25 000 arsinlik payin dogusundan ve batisindan ülkenin dogu ve bati sinirlarina uzanacak. Oymaklara düsen paylar boyunca uzanan bu iki bölge önderin olacak. Kutsal bölgeye düsen pay ile tapinak bunun ortasinda olacak.
\par 22 Böylece Levililer'e düsen pay ile kente düsen pay öndere verilen topraklarin ortasinda kalacak. Öndere verilecek topraklar Yahuda'yla Benyamin siniri arasinda kalacak.
\par 23 "Geri kalan oymaklara düsen pay söyle: Benyamin'e bir pay verilecek; siniri dogudan batiya uzanacak.
\par 24 "Simon'a bir pay verilecek; siniri Benyamin'in dogudan batiya uzanan sinirina bitisik olacak.
\par 25 "Issakar'a bir pay verilecek; siniri Simon'un dogudan batiya uzanan sinirina bitisik olacak.
\par 26 "Zevulun'a bir pay verilecek; siniri Issakar'in dogudan batiya uzanan sinirina bitisik olacak.
\par 27 "Gad'a bir pay verilecek; siniri Zevulun'un dogudan batiya uzanan sinirina bitisik olacak.
\par 28 "Gad'in güney siniri Tamar'dan güneye, oradan Meriva-Kades sularina, oradan da Misir Vadisi boyunca Akdeniz'e dek uzanacak.
\par 29 "Mülk olarak Israil oymaklarina bölüstüreceginiz ülke budur. Onlara düsecek paylar bunlardir." Böyle diyor Egemen RAB.
\par 30 "Kentin çikis kapilari sunlar olacak: 4 500 arsin uzunluktaki kuzey yaninda üç kapi olacak. Kentin kapilarina Israil oymaklarinin adlari verilecek. Kuzeyde Ruben Kapisi, Yahuda Kapisi, Levi Kapisi olacak.
\par 32 4 500 arsin uzunluktaki dogu yaninda üç kapi olacak: Yusuf Kapisi, Benyamin Kapisi, Dan Kapisi.
\par 33 4 500 arsin uzunluktaki güney yaninda üç kapi olacak: Simon Kapisi, Issakar Kapisi, Zevulun Kapisi.
\par 34 4 500 arsin uzunluktaki bati yaninda üç kapi olacak: Gad Kapisi, Aser Kapisi, Naftali Kapisi.
\par 35 "Kentin çevresi 18 000 arsin olacak ve o günden baslayarak kentin adi 'RAB orada olacak."


\end{document}