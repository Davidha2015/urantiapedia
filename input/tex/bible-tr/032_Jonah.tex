\begin{document}

\title{Yunus}


\chapter{1}

\par 1 RAB bir gün Amittay oglu Yunus'a, "Kalk, Ninova'ya, o büyük kente git ve halki uyar" diye seslendi, "Çünkü kötülükleri önüme kadar yükseldi."
\par 3 Ne var ki, Yunus RAB'bin huzurundan Tarsis'e kaçmaya kalkisti. Yafa'ya inip Tarsis'e giden bir gemi buldu. Ücretini ödeyip gemiye bindi, RAB'den uzaklasmak için Tarsis'e dogru yola çikti.
\par 4 Yolda RAB siddetli bir rüzgar gönderdi denize. Öyle bir firtina koptu ki, gemi neredeyse parçalanacakti.
\par 5 Gemiciler korkuya kapildi, her biri kendi ilahina yalvarmaya basladi. Gemiyi hafifletmek için yükleri denize attilar. Yunus ise teknenin ambarina inmis, yatip derin bir uykuya dalmisti.
\par 6 Gemi kaptani Yunus'un yanina gidip, "Hey! Nasil uyursun sen?" dedi, "Kalk, tanrina yalvar, belki halimizi görür de yok olmayiz."
\par 7 Sonra denizciler birbirlerine, "Gelin, kura çekelim" dediler, "Bakalim, bu bela kimin yüzünden basimiza geldi." Kura çektiler, kura Yunus'a düstü.
\par 8 Bunun üzerine Yunus'a, "Söyle bize!" dediler, "Bu bela kimin yüzünden basimiza geldi? Ne is yapiyorsun sen, nereden geliyorsun, nerelisin, hangi halka mensupsun?"
\par 9 Yunus, "Ibrani'yim" diye karsilik verdi, "Denizi ve karayi yaratan Göklerin Tanrisi RAB'be taparim."
\par 10 Denizciler bu yanit karsisinda dehsete düstüler. "Neden yaptin bunu?" diye sordular. Yunus'un RAB'den uzaklasmak için kaçtigini biliyorlardi. Daha önce onlara anlatmisti.
\par 11 Deniz gittikçe kuduruyordu. Yunus'a, "Denizin dinmesi için sana ne yapalim?" diye sordular.
\par 12 Yunus, "Beni kaldirip denize atin" diye yanitladi, "O zaman sular durulur. Çünkü biliyorum, bu siddetli firtinaya benim yüzümden yakalandiniz."
\par 13 Denizciler karaya dönmek için küreklere asildilar, ama basaramadilar. Çünkü deniz gittikçe kuduruyordu.
\par 14 RAB'be seslenerek, "Ya RAB, yalvariyoruz" dediler, "Bu adamin cani yüzünden yok olmayalim. Suçsuz bir adamin ölümünden bizi sorumlu tutma. Çünkü sen kendi istedigini yaptin, ya RAB."
\par 15 Sonra Yunus'u kaldirip denize attilar, kuduran deniz sakinlesti.
\par 16 Bu olaydan ötürü denizciler RAB'den öyle korktular ki, O'na kurbanlar sundular, adaklar adadilar.
\par 17 Bu arada RAB Yunus'u yutacak büyük bir balik sagladi. Yunus üç gün üç gece bu baligin karninda kaldi.

\chapter{2}

\par 1 Yunus baligin karnindan Tanrisi RAB'be söyle dua etti:
\par 2 "Ya RAB, sikinti içinde sana yakardim, Yanitladin beni. Yardim istedim ölüler diyarinin bagrindan, Kulak verdin sesime.
\par 3 Beni engine, denizin ta dibine firlattin. Sular sardi çevremi. Azgin dalgalar geçti üzerimden.
\par 4 'Huzurundan kovuldum' dedim, 'Yine de görecegim kutsal tapinagini.'
\par 5 Sular bogacak kadar kusatti beni, Çevremi enginler sardi, Yosunlar dolasti basima.
\par 6 Daglarin köklerine kadar battim, Dünya sonsuza dek sürgülendi arkamdan; Ama, ya RAB, Tanrim, Canimi sen kurtardin çukurdan.
\par 7 Solugum tükenince seni andim, ya RAB, Duam sana, kutsal tapinagina eristi.
\par 8 Degersiz putlara tapanlar, Vefasizlik etmis olurlar.
\par 9 Ama sükranla kurban sunacagim sana, Adagimi yerine getirecegim. Kurtulus senden gelir, ya RAB!"
\par 10 RAB baliga buyruk verdi ve balik Yunus'u karaya kustu.

\chapter{3}

\par 1 RAB Yunus'a ikinci kez söyle seslendi:
\par 2 "Kalk, Ninova'ya, o büyük kente git ve sana söyleyeceklerimi halka bildir."
\par 3 Yunus RAB'bin sözü uyarinca kalkip Ninova'ya gitti. Ninova öyle büyük bir kentti ki, ancak üç günde dolasilabilirdi.
\par 4 Yunus kente girip dolasmaya basladi. Bir gün geçince, "Kirk gün sonra Ninova yikilacak!" diye ilan etti.
\par 5 Ninova halki Tanri'ya inandi. Oruç* ilan ederek büyügünden küçügüne hepsi çula sarindi.
\par 6 Ninova Krali olanlari duyunca, tahtindan kalkip kaftanini çikardi; çula sarinarak küle oturdu.
\par 7 Ardindan Ninova'da su buyrugu yayimladi: "Kral ve soylularin buyrugudur: Hiçbir insan ya da hayvan -ister sigir, ister davar olsun- agzina bir sey koymayacak, otlamayacak, içmeyecek.
\par 8 Bütün insanlar ve hayvanlar çula sarinsin. Herkes var gücüyle Tanri'ya yakararak kötü yoldan, zorbaliktan vazgeçsin.
\par 9 Belki o zaman Tanri fikrini degistirip bize acir, kizgin öfkesinden döner de yok olmayiz."
\par 10 Tanri Ninovalilar'in yaptiklarini, kötü yoldan döndüklerini görünce, onlara acidi, yapacagini söyledigi kötülükten vazgeçti.

\chapter{4}

\par 1 Yunus buna çok gücenip öfkelendi.
\par 2 RAB'be söyle dua etti: "Ah, ya RAB, ben daha ülkemdeyken böyle olacagini söylemedim mi? Bu yüzden Tarsis'e kaçmaya kalkistim. Biliyordum, sen lütfeden, aciyan, tez öfkelenmeyen, sevgisi engin, cezalandirmaktan vazgeçen bir Tanri'sin.
\par 3 Ya RAB, lütfen simdi canimi al. Çünkü benim için ölmek yasamaktan iyidir."
\par 4 RAB, "Ne hakla öfkeleniyorsun?" diye karsilik verdi.
\par 5 Yunus kentten çikti, kentin dogusundaki bir yerde durdu. Kendisine bir çardak yapti, gölgesinde oturup kentin basina neler gelecegini görmek için beklemeye basladi.
\par 6 RAB Tanri Yunus'un üzerine gölge salacak, sikintisini giderecek bir keneotu sagladi. Yunus buna çok sevindi.
\par 7 Ama ertesi gün safak sökerken, Tanri'nin sagladigi bir bitki kurdu keneotunu kemirip kuruttu.
\par 8 Günes dogunca Tanri yakici bir dogu rüzgari estirdi. Yunus basina vuran günesten bayilmak üzereydi. Ölümü dileyerek, "Benim için ölmek yasamaktan iyidir" dedi.
\par 9 Ama Tanri, "Keneotu yüzünden öfkelenmeye hakkin var mi?" dedi. Yunus, "Elbette hakkim var, ölesiye öfkeliyim" diye karsilik verdi.
\par 10 RAB, "Keneotu bir gecede çikti ve bir gecede yok oldu" dedi, "Sen emek vermedigin, büyütmedigin bir keneotuna aciyorsun da,
\par 11 ben Ninova'ya, o koca kente acimayayim mi? O kentte sagini solundan ayirt edemeyen yüz yirmi bini askin insan, çok sayida hayvan var."


\end{document}