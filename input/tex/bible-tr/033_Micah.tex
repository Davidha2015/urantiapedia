\begin{document}

\title{Mika}


\chapter{1}

\par 1 Yahuda krallari Yotam, Ahaz ve Hizkiya zamaninda RAB Moresetli Mika'ya, Samiriye ve Yerusalim'le ilgili olarak bir görümde sunu bildirdi:
\par 2 Ey halklar, hepiniz duyun; Ey dünya ve bütün içindekiler, dinleyin. Egemen RAB kendi kutsal tapinagindan size karsi taniklik edecek.
\par 3 Iste, RAB yerinden çikip gelecek, Yeryüzüne inip dag doruklarinda yürüyecek.
\par 4 Daglar O'nun önünde ates karsisinda eriyen balmumu gibi eriyecek, Vadiler, bayir asagi akan sular gibi yarilacak.
\par 5 Bütün bunlar Yakupogullari'nin isyani Ve Israil halkinin günahlari yüzünden olacak. Yakupogullari'nin isyanindan kim sorumlu? Samiriye degil mi? Yahuda'daki putperestlikten kim sorumlu? Yerusalim degil mi?
\par 6 Bu yüzden RAB, "Samiriye'yi kirdaki tas yiginina, Bag dikilecek yere çevirecegim" diyor, "Taslarini vadiye döküp temellerini açacagim.
\par 7 Bütün putlari paramparça edilecek, Tapinaklarindaki fahiselere verilen armaganlar yakilacak. Samiriye'nin bütün putlarini yok edecegim. Fahiselerin ücretiyle topladigi armaganlar Yine fahiselere ücret olacak."
\par 8 Ben Mika, bundan ötürü aglayip agit yakacagim, Çirçiplak, yalinayak dolasacagim. Çakal gibi uluyup baykus gibi ötecegim.
\par 9 Çünkü Samiriye'nin yaralari onmaz. Yahuda da ayni sona ugramak üzere. Halkimin yasadigi Yerusalim'in kapilarina dayandi yikim.
\par 10 Bunu Gat'a duyurmayin, Aglamayin sakin! Beytofra'da toz toprak içinde yuvarlanin.
\par 11 Ey Safir halki, Çiplak ve utanç içinde geçip git. Saanan'da yasayanlar kentlerinden çikamayacaklar, Beytesel halki yas tutacak. Kesecek sizden yardimini.
\par 12 Marot'ta yasayanlar kurtulmayi sabirsizlikla bekliyor. Çünkü RAB'bin gönderdigi felaket Yerusalim'in kapilarina dayandi.
\par 13 Ey Lakis'te oturanlar, atlari kosun arabalara. Siyon Kenti'ni günaha ilk düsüren siz oldunuz. Çünkü Israil'in isyanini örnek aldiniz.
\par 14 Bundan ötürü Moreset-Gat'a veda armaganlari vereceksiniz. Israil krallari Akziv Kenti'nden bosuna yardim bekleyecek.
\par 15 Ey Maresa'da yasayanlar, RAB kentinizi ele geçirecek olani üzerinize gönderecek. Israil'in yüce önderleri Adullam'daki magaraya siginacak.
\par 16 Sevgili çocuklariniz için saçlarinizi yolup kaziyin. Akbabalar gibi kafalarinizin keli görünsün. Çünkü çocuklariniz sizden alinip sürgüne götürülecek. Halki Sömürenlerin Cezalandirilmasi

\chapter{2}

\par 1 Yatarken fesat ve kötülük tasarlayanlarin vay haline! Ortalik agarinca tasarladiklarini yaparlar. Çünkü güçleri buna yeter.
\par 2 Göz diktikleri tarlalari zorla alir, evlere el koyarlar. Birini evinden, bir baskasini mirasindan ederler.
\par 3 Bu nedenle RAB bu halka söyle diyor: "Bakin, size öyle bir bela hazirliyorum ki, Bundan yakanizi kurtaramayacaksiniz. Öyle amansiz bir zaman gelecek ki, Basiniz dik yürüyemeyeceksiniz.
\par 4 O gün sizinle alay edecekler. Sizin için su acikli ezgiyi söyleyecekler: 'Büsbütün mahvolduk! RAB halkimizin varini yogunu baskalarina bölüstürüyor, Topraklarimizi hainlere dagitiyor.'"
\par 5 Bu nedenle, ülkeyi kurayla bölüstürme zamani gelince RAB'bin toplulugunda sizden kimse bulunmayacak.
\par 6 Insanlar, "Peygamberlik etmeyin" diyorlar, "Bu konularda peygamberlik etmemeli. Utandirilmayacagiz."
\par 7 Ey Yakupogullari, böyle konusulur mu? "RAB'bin sabri mi tükendi acaba? O böyle seyler yapar mi? Benim sözlerim dogru yolda yürüyenin yararina degil mi?" diyor RAB,
\par 8 "Daha dün halkim düsman gibi ayaklandi. Savastan dönenlerin, kaygisizca önünüzden geçenlerin Sirtindan Güzel giysilerini siyirip alirsiniz.
\par 9 Halkimin kadinlarini rahat evlerinden kovar, Çocuklarini yüce huzurumdan yoksun birakirsiniz.
\par 10 Kalkip gidin, dinlenme yeriniz degil burasi! Murdarliginiz* yüzünden bu yer korkunç biçimde yikilacak.
\par 11 Yalanci, aldatici biri gelip, 'Size sarap ve içkiden söz edeyim' dese, Bu halk onu peygamber kabul edecek."
\par 12 "Ey Yakupogullari, Elbette hepinizi bir araya getirecegim. Israil'in geride kalanlarini elbette toplayacagim. Agildaki davar gibi, Otlaktaki sürü gibi bir araya getirecegim sizleri. Topraklariniz insanlarla dolacak."
\par 13 Tanri yolu açip halkin önünden gidecek. Kent kapilarini kirip disari çikacaklar. Krallari olan RAB önlerinden gidecek. Önderlerin ve Peygamberlerin Suçu

\chapter{3}

\par 1 Dedim ki, "Ey Yakupogullari'nin önderleri, Israil halkinin yöneticileri, Dinleyin! Adil olmaniz gerekmez mi?
\par 2 Siz ki iyiden nefret eder, kötüyü seversiniz. Halkimin derisini yüzer, etini kemiginden siyirirsiniz.
\par 3 Halkimin derisini yüzer, etini yersiniz. Kemiklerini kirar, Tencerede, kazanda haslanacak et gibi dograrsiniz."
\par 4 Gün gelecek RAB'be yakaracaklar. Ama O yanitlamayacak, Yüzünü onlardan gizleyecek. Çünkü kötülük yaptilar.
\par 5 RAB diyor ki, "Ey halkimi saptiran peygamberler, Sizi doyuranlara esenlik diler, Doyurmayanlara savas açarsiniz.
\par 6 Bu nedenle üzerinize görümsüz geceler çökecek. Karanliktan fal bakamayacaksiniz. Ey peygamberler, günesiniz batacak, gününüz kararacak.
\par 7 Biliciler* utandirilacak. Rezil olacak falcilar. Utançtan yüzlerini örtecekler. Çünkü Tanri'dan yanit gelmeyecek."
\par 8 Ama Yakupogullari'na isyanlarini, Israil halkina günahlarini bildirmek için Ben RAB'bin Ruhu'yla, güçle, Adalet ve cesaretle donatildim.
\par 9 Adaletten nefret eden, Dogrulari çarpitan ey Yakupogullari'nin önderleri Ve Israil halkinin yöneticileri, iyi dinleyin:
\par 10 Siyon'u kan dökerek, Yerusalim'i zorbalikla bina ediyorsunuz.
\par 11 Önderleri rüsvetle yönetir, Kâhinleri* ücretle ögretir, Peygamberleri para için falcilik eder. Sonra da, "RAB bizimle birlikte degil mi? Basimiza bir sey gelmez" diyerek RAB'be dayanmaya kalkisirlar.
\par 12 Siyon tarla gibi sürülecek sizin yüzünüzden. Tas yiginina dönecek Yerusalim. Tapinagin kuruldugu dag Çalilarla kaplanacak.

\chapter{4}

\par 1 RAB'bin Tapinagi'nin kuruldugu dag, Son günlerde daglarin en yücesi, Tepelerin en yüksegi olacak. Oraya akin edecek halklar.
\par 2 Birçok ulus gelecek, "Haydi, RAB'bin Dagi'na, Yakup'un Tanrisi'nin Tapinagi'na çikalim" diyecekler, "O bize kendi yolunu ögretsin, Biz de O'nun yolundan gidelim. Çünkü yasa Siyon'dan, RAB'bin sözü Yerusalim'den çikacak."
\par 3 RAB halklar arasinda yargiçlik edecek, Uzaklardaki güçlü uluslarin anlasmazliklarini çözecek. Insanlar kiliçlarini çekiçle dövüp saban demiri, Mizraklarini bagci biçagi yapacaklar. Ulus ulusa kiliç kaldirmayacak, Savas egitimi yapmayacaklar artik.
\par 4 Herkes kendi asmasinin, incir agacinin altinda oturacak. Kimse kimseyi korkutmayacak. Bunu söyleyen, Her Seye Egemen RAB'dir.
\par 5 Bütün halklar ilahlarinin izinden gitse bile, Biz sonsuza dek Tanrimiz RAB'bin izinden gidecegiz.
\par 6 "Gün gelecek, düskünü, sürgüne gönderip ezdigim halki Bir araya getirecegim" diyor RAB,
\par 7 "Düskünü yasatacak, Uzaklara sürülenleri güçlü bir ulus yapacagim. Onlari Siyon Dagi'nda bugünden sonsuza dek ben yönetecegim."
\par 8 Ve sen, sürünün gözcü kulesi olan ey Siyon Kenti'nin dorugu, Eski egemenligine kavusacaksin. Ey Yerusalim, kralligini yeniden elde edeceksin.
\par 9 Neden öyle hiçkira hiçkira agliyorsun simdi? Doguran kadin gibi neden aci çekiyorsun? Kralin olmadigi için mi, Ögütçün öldügü için mi?
\par 10 Doguran kadin gibi agri çek, aciyla kivran, ey Siyon halki. Simdi kentten çikip kirlarda konaklayacaksin. Babil'e gidecek, Orada özgürlüge kavusacaksin. RAB seni orada kurtaracak düsmanlarinin elinden.
\par 11 Ama simdi birçok ulus sana karsi birlesti. "Siyon murdar* olsun, Basina gelenleri gözlerimizle görelim" diyorlar.
\par 12 Ne var ki, RAB'bin ne düsündügünü bilmiyorlar, O'nun tasarilarini anlamiyorlar. RAB onlari harman yerinde dövülen bugday demetleri gibi Cezalandirmak için topladi.
\par 13 RAB söyle diyor: "Ey Siyon halki, kalk ve harmani döv. Çünkü seni demir boynuzlu, Tunç* tirnakli bogalar kadar güçlü kilacagim. Birçok halki ezip geçecek, Zorbalikla elde ettikleri serveti, zenginlikleri bana, Yeryüzünün sahibi olan Rab'be adayacaksin."

\chapter{5}

\par 1 Ey ordular kenti, simdi ordularini topla. Çevremizi sardilar, Israil'i yönetenin yanagina degnekle vuracaklar.
\par 2 Ama sen, ey Beytlehem Efrata, Yahuda boylari arasinda önemsiz oldugun halde, Israil'i benim adima yönetecek olan senden çikacak. Onun kökeni öncesizlige, zamanin baslangicina dayanir.
\par 3 Bu yüzden onu doguracak olan kadin dogurana dek RAB Israilliler'i düsmanlarina teslim edecek. Sonra öbür soydaslari Israilliler'e katilacak.
\par 4 O gelince, halkini RAB'den aldigi güçle Tanrisi RAB'bin görkemli adina yönetecek. Halk güvenlik içinde yasayacak. Çünkü bütün dünya onun büyüklügünü kabul edecek.
\par 5 Halkina esenlik getirecek. Kurtulus ve Yikim Asurlular ülkemize saldirip Kalelerimizi ele geçirince, Onlara karsi çok sayida önder çikaracagiz.
\par 6 Asur topraklarini kiliçla, Nemrut'un topraklarini yalin kiliçla yönetecekler. Ülkemize saldirip sinirlarimizdan içeri girecek olan Asurlular'dan Bizi bu önderler kurtaracak.
\par 7 Yakup'un soyundan geride kalanlar, Birçok halkin arasinda RAB'bin gönderdigi çiy gibi, Kimseye dayanmadan, kimsenin onayini beklemeden Otlari sulayan saganak yagmurlari gibi olacaklar.
\par 8 Orman hayvanlari arasinda aslan ne ise, Davar sürülerini paralayip dagitan, kurtulma firsati vermeyen genç aslan ne ise, Yakup'un soyundan geride kalanlar da uluslar arasinda, Halklarin ortasinda öyle olacaklar.
\par 9 Ey Israilliler, düsmanlarinizi yeneceksiniz. Karsitlarinizin hepsi ortadan kalkacak.
\par 10 RAB diyor ki, "O gün atlarinizi elinizden alacak, Savas arabalarinizi yok edecegim.
\par 11 Ülkenizdeki kentleri yikacak, Bütün kalelerinizi yerle bir edecegim.
\par 12 Büyü yapma gücünüzü kiracagim, Aranizda falci kalmayacak.
\par 13 Putlarinizi, dikili taslarinizi kaldirip atacagim. Ellerinizle yaptiginiz putlara artik tapmayacaksiniz.
\par 14 Asera* putlarinizi söküp atacagim, Yerle bir edecegim kentlerinizi.
\par 15 Söz dinlemeyen uluslari öfke ve gazapla cezalandiracagim."

\chapter{6}

\par 1 RAB'bin söyledigine kulak verin: Kalkin, davanizi daglarin önünde dile getirin. Tepeler duysun sesinizi.
\par 2 Ey daglar ve yeryüzünün sarsilmaz temelleri, RAB'bin suçlamasini dinleyin. Çünkü RAB halkindan davaci, Israil'den sikâyetçi.
\par 3 "Ey halkim, sana ne yaptim?" diyor RAB, "Sana nasil yük oldum, yanitla.
\par 4 Seni Misir'dan ben çikardim, Ben kurtardim seni kölelik diyarindan. Sana öncülük etsinler diye Musa'yi, Harun'u, Miryam'i ben gönderdim.
\par 5 Ey halkim, Moav Krali Balak'in neler ögütledigini, Beor oglu Balam'in onu nasil yanitladigini animsa. Sittim'den Gilgal'a dek olup biteni an. Sizleri nasil kurtardigimi o zaman anlayacaksin."
\par 6 RAB'bin önüne ne ile çikayim, Yüce Tanri'ya nasil tapinayim? O'nun önüne yakmalik sunuyla* mi, Bir yasinda danayla mi çikayim?
\par 7 Binlerce koç sunsam, Zeytinyagindan on binlerce dere akitsam, RAB hosnut kalir mi? Suçuma karsilik ilk oglumu, Isledigim günah için bedenimin ürününü versem olur mu?
\par 8 Ey insanlar, RAB iyi olani size bildirdi; Adil davranmanizdan, sadakati sevmenizden Ve alçakgönüllülükle yolunda yürümenizden baska Tanriniz RAB sizden ne istedi?
\par 9 Dinleyin! RAB kente sesleniyor. O'nun adindan korkmak bilgeliktir. Diyor ki, "Ey halk ve kent meclisi, dinleyin.
\par 10 Kötü adamlarin evleri Haksizca kazanilmis servetlerle dolu, Bilmiyor muyum saniyorsunuz? Eksik ölçek lanetlidir.
\par 11 Hileli terazi kullanan, Torbasinda eksik agirliklar olan adami nasil aklayayim?
\par 12 Kentin zenginleri zorba, Halki da yalancidir. Dillerinden aldatici sözler dökülür.
\par 13 Günahlarinizdan ötürü yikiminizi, Mahvinizi hazirladim bile.
\par 14 Yiyecek, ama doymayacaksiniz. Aç kalacak karniniz, Biriktireceksiniz, ama saklayamayacaksiniz. Koruyabildiginizi kiliçla yok edecegim.
\par 15 Ekecek, ama biçemeyeceksiniz. Zeytin ezecek, ama yagini sürünemeyeceksiniz. Üzümü sikacak, ama sarabini içemeyeceksiniz.
\par 16 Kral Omri'nin buyruklarina, Ahav soyunun kötü adetlerine uydugunuz, Onlarin törelerini izlediginiz için sizi utanca bogacagim, yikima ugratacagim. Halkim olarak asagilanmaya dayanmak zorunda kalacaksiniz."

\chapter{7}

\par 1 Vay halime benim! Yazin meyve toplandiktan Ve bagbozumundan artakalan üzümler alindiktan sonra
\par 2 Tek bir salkim bulamayan adam gibiyim. Canim turfanda inciri nasil da çekiyor! Ülkede Tanri'ya sadik kul kalmadi. Insanlar arasinda dürüst kimse yok. Herkes kan dökmek için pusuda. Kardes kardese tuzak kuruyor.
\par 3 Kötülük yapmakta elleri ne becerikli! Önderler armagan istiyor, yargiçlar rüsvet aliyor. Güçlüler her istediklerini zorla yaptiriyor, Düzen üstüne düzen kuruyorlar.
\par 4 En iyileri çali çirpidan degersiz, En dürüstleri dikenli çitten beterdir. Ama peygamberlerinin uyardigi gibi, Cezalandirilacaklari gün geldi çatti. Saskinlik içindeler simdi.
\par 5 Inanmayin komsunuza, Dostunuza güvenmeyin. Koynunuzda yatan karinizin yaninda bile Siki tutun agzinizi.
\par 6 Çünkü ogul babasina saygisizlik ediyor, Kiz annesine, gelin kaynanasina karsi geliyor. Insanin düsmani kendi ev halkidir.
\par 7 Ama ben umutla RAB'be bakiyor, Kurtaricim olan Tanri'yi bekliyorum. Duyacak beni Tanrim.
\par 8 Halime sevinme, ey düsmanim! Düssem de kalkarim. Karanlikta kalsam bile RAB bana isik olur.
\par 9 RAB'be karsi günah isledigim için, O'nun öfkesine dayanmaliyim. Sonunda davami savunup hakkimi alacak, Beni isiga çikaracak, adaletini görecegim.
\par 10 Düsmanim da görecek ve utanç içinde kalacak. O düsman ki, "Hani Tanrin RAB nerede?" diye soruyordu bana. Onun düsüsünü gözlerimle görecegim. Sokaktaki çamur gibi ayak altinda çignenecek.
\par 11 Ey Yerusalim, Surlarinin onarilacagi, Sinirlarinin genisletilecegi gün gelecek.
\par 12 Halkimizdan olanlar o gün Asur'dan Misir'a, Misir'dan Firat'a kadar uzanan topraklardan, Denizler arasinda, daglar arasinda kalan topraklardan sana gelecekler.
\par 13 Ama ülke, içinde yasayanlarin yaptigi kötülükler yüzünden viraneye dönecek.
\par 14 Ya RAB, mirasin olan Ve Karmel'in ortasindaki ormanda ayri yasayan sürünü, halkini Degneginle güt. Geçmiste oldugu gibi, Basan'da ve Gilat'ta beslensinler.
\par 15 Bizi Misir'dan çikardigin günlerdeki gibi, Harikalar yarat halkin için.
\par 16 Uluslar bunu görünce Yaptiklari bunca zorbaliktan utanacaklar. Elleriyle agizlarini kapayacak, kulaklarini tikayacaklar.
\par 17 Yilanlar gibi, sürüngenler* gibi toprak yalayacak, Titreyerek siginaklarindan çikacaklar. Ey Tanrimiz RAB, dehset içinde sana dönecek Ve senden korkacaklar.
\par 18 Senin gibi suçlari silen, Kendi halkindan geride kalanlarin isyanlarini bagislayan baska tanri var mi? Sonsuza dek öfkeli kalmazsin, Çünkü sadik olmaktan hoslanirsin.
\par 19 Bize yine aciyacaksin, Çigneyeceksin suçlarimizi ayak altinda. Bütün günahlarimizi denizin dibine atacaksin.
\par 20 Geçmiste atalarimiza ant içtigin gibi, Yakup'un ve Ibrahim'in torunlari olan bizlere de verdigin sözü tutacak ve sadik kalacaksin.


\end{document}