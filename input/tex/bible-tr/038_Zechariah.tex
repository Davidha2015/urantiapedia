\begin{document}

\title{Zekeriya}


\chapter{1}

\par 1 Darius'un kralliginin ikinci yilinin sekizinci ayinda* RAB Iddo oglu Berekya oglu Peygamber Zekeriya araciligiyla söyle seslendi:
\par 2 "RAB atalariniza çok öfkelendi.
\par 3 Bu nedenle halka de ki, 'Her Seye Egemen RAB, bana dönün, ben de size dönerim diyor.
\par 4 Atalariniz gibi davranmayin! Önceki peygamberler, Her Seye Egemen RAB kötü yollarinizdan ve kötü uygulamalarinizdan dönün diyor, diyerek onlari uyardilar. Ne var ki, onlar dinlemediler, bana aldiris etmediler. Böyle diyor RAB.
\par 5 Hani atalariniz nerede? Peygamberler de sonsuza kadar mi yasar?
\par 6 Peygamber kullarima buyurdugum sözler ve kurallar atalariniza ulasmadi mi?' "Onlar da dönüp, 'Her Seye Egemen RAB yollarimiza ve uygulamalarimiza bakarak bizim için ne düsündüyse aynen yapti' dediler."
\par 7 Darius'un kralliginin ikinci yilinda, on birinci ay* olan Sevat ayinin yirmi dördüncü günü RAB Iddo oglu Berekya oglu Peygamber Zekeriya'ya görümlerle seslendi.
\par 8 Gece vadideki mersin agaçlarinin arasinda kizil ata binmis bir adam gördüm. Arkasinda kizil, kula ve beyaz atlar vardi.
\par 9 "Efendim, bunlar ne?" diye sordum. Benimle konusan melek, "Bunlarin ne oldugunu sana gösterecegim" diye yanitladi.
\par 10 Mersin agaçlari arasinda duran adam da, "Bunlar dünyayi dolasmak için RAB'bin gönderdikleridir" diye açikladi.
\par 11 Mersin agaçlari arasinda duran RAB'bin melegine, "Dünyayi dolastik" dediler, "Iste bütün dünya esenlik ve güvenlik içinde!"
\par 12 Bunun üzerine RAB'bin melegi, "Ey Her Seye Egemen RAB, yetmis yildir öfkelendigin Yerusalim'den ve Yahuda kentlerinden sevecenligini ne zamana dek esirgeyeceksin?" dedi.
\par 13 RAB benimle konusan melegi tatli, avutucu sözlerle yanitladi.
\par 14 Bunun üzerine benimle konusan melek, "Sunu duyur!" dedi, "Her Seye Egemen RAB, 'Yerusalim ve Siyon için büyük kiskançlik duyuyorum' diyor,
\par 15 'Tasasiz uluslara ise çok öfkeliyim; çünkü ben biraz öfkelenmistim, onlarsa kötülüge kötülük kattilar.'
\par 16 "Onun için RAB, 'Yerusalim'e sevecenlikle dönecegim' diyor, 'Tapinagim orada yeniden kurulacak ve Yerusalim üzerine ölçü ipi çekilecek!' Böyle diyor Her Seye Egemen RAB.
\par 17 "Sunu da duyur: Her Seye Egemen RAB, 'Kentlerim yine bollukla dolup tasacak' diyor, 'Ben RAB, Siyon'u yine avutacagim, Yerusalim'i yine seçecegim.'"
\par 18 Sonra gözlerimi kaldirip baktim, dört boynuz vardi.
\par 19 Benimle konusan melege, "Bunlar ne?" diye sordum. Melek, "Bunlar Yahuda, Israil ve Yerusalim halkini dagitmis olan boynuzlardir" diye karsilik verdi.
\par 20 Sonra RAB bana dört usta gösterdi.
\par 21 "Bunlar ne yapmaya geliyor?" diye sordum. Melek, "Su boynuzlar Yahuda halkini öyle dagitti ki, kimse basini kaldiramadi" dedi, "Bu ustalar da Yahuda halkini dagitmak için boynuz kaldiran uluslari yildirip boynuzlarini yere çalmaya geldiler."

\chapter{2}

\par 1 Sonra gözlerimi kaldirip baktim, elinde ölçü ipi tutan bir adam vardi.
\par 2 "Nereye gidiyorsun?" diye sordum. Adam, "Yerusalim'i ölçmeye, genisliginin, uzunlugunun ne kadar oldugunu ögrenmeye gidiyorum" diye yanitladi.
\par 3 Benimle konusan melek yanimdan ayrilinca baska bir melek onu karsilamaya çikti.
\par 4 Önceki melege söyle dedi: "Kos, o gence de ki, içinde barinacak sayisiz insan ve hayvandan ötürü Yerusalim sursuz bir kent olacak.
\par 5 RAB, 'Ben kendim onun çevresinde atesten sur ve içindeki görkem olacagim' diyor."
\par 6 RAB, "Haydi! Haydi! Kuzey ülkesinden kaçin!" diyor, "Çünkü sizi gögün dört bucagina dagittim." Böyle diyor RAB.
\par 7 "Babil'de oturan Siyon halki, haydi kaçip kurtul!"
\par 8 Çünkü Her Seye Egemen RAB beni onurlandirdi ve sizi yagmalamis uluslara su haberle gönderdi: "Size dokunan gözbebegime dokunmus olur" diyor,
\par 9 "Elimi onlara karsi kaldiracagim, köleleri onlari yagmalayacak." O zaman siz de beni Her Seye Egemen RAB'bin gönderdigini anlayacaksiniz.
\par 10 RAB, "Ey Siyon kizi*, sevinçle bagir! Çünkü aranizda yasamaya geliyorum" diyor.
\par 11 O gün birçok ulus RAB'be baglanacak, O'nun halki olacak. O zaman RAB aranizda yasayacak, siz de beni Her Seye Egemen RAB'bin gönderdigini anlayacaksiniz.
\par 12 RAB kutsal topraklarda Yahuda'yi kendi payi olarak miras edinecek ve Yerusalim'i yine seçecek.
\par 13 Ey insanlar, RAB'bin önünde sessiz durun! RAB kutsal konutundan kalkmis geliyor!

\chapter{3}

\par 1 RAB, meleginin önünde duran Baskâhin Yesu'yu ve onu suçlamak için saginda duran Seytan'i bana gösterdi.
\par 2 RAB'bin melegi Seytan'a, "RAB seni azarlasin, ey Seytan!" dedi, "Yerusalim'i seçen RAB seni azarlasin! Bu adam atesten çikarilan yari yanmis odun parçasi degil mi?"
\par 3 Yesu melegin önünde çok kirli giysiler içinde duruyordu.
\par 4 Melek önündeki meleklere, "Üzerinden kirli giysileri çikarin" dedi. Sonra Yesu'ya, "Bak, suçunu kaldirdim. Sana bayramlik giysiler giydirecegim" dedi.
\par 5 Ben de Yesu'nun basina temiz bir sarik sarmalarini söyledim. Basina temiz bir sarik sarip onu giydirdiler. RAB'bin melegi de onun yaninda duruyordu.
\par 6 Sonra RAB'bin melegi Yesu'yu uyardi:
\par 7 "Her Seye Egemen RAB diyor ki, 'Eger yollarimda yürür, verdigim görevleri yerine getirirsen, tapinagimi sen yönetecek, avlularimi sen koruyacaksin. Sana burada duranlarin arasina katilip huzuruma çikma ayricaligini verecegim.
\par 8 "'Ey Baskâhin Yesu, sen ve önünde oturan kâhin* arkadaslarin, dinleyin! Çünkü onlar gelecek olaylarin önbelirtisidir. Dal adindaki kulumu ortaya çikariyorum.
\par 9 Yesu'nun önüne koydugum tasa bakin! O tek tasin yedi gözü var; onun üzerine bir yazit oyacagim' diyor Her Seye Egemen RAB, 'Bir günde bu ülkenin günahini kaldiracagim.
\par 10 O gün her biriniz komsusunu asmasinin, incir agacinin altinda oturmaya çagiracak.' Böyle diyor Her Seye Egemen RAB."

\chapter{4}

\par 1 Benimle konusan melek yine geldi ve uykudan uyandirir gibi beni uyandirdi.
\par 2 "Ne görüyorsun?" diye sordu. "Som altin bir kandillik görüyorum" diye yanitladim, "Tepesinde zeytinyagi için bir tas, üzerinde yedi kandil, kandillerde yediser oluk var.
\par 3 Ayrica kandilligin yaninda, biri zeytinyagi tasinin saginda, öbürü solunda iki zeytin agaci da var."
\par 4 Benimle konusan melege, "Bunlarin anlami nedir, efendim?" diye sordum.
\par 5 Melek, "Bunlarin anlamini bilmiyor musun?" diye karsilik verdi. "Hayir, efendim" dedim.
\par 6 Bunun üzerine söyle dedi: "RAB Zerubbabil'e, 'Güçle kuvvetle degil, ancak benim Ruhum'la basaracaksin' diyor. Böyle diyor Her Seye Egemen RAB.
\par 7 Sen kim oluyorsun, ey ulu dag? Zerubbabil'in önünde bir düzlük olacaksin! O tapinagin son tasini çikarirken, halk da, 'Ne güzel, ne güzel!' diye bagiracak."
\par 8 RAB bana yine seslendi:
\par 9 "Bu tapinagin temelini Zerubbabil'in elleri atti, tapinagi tamamlayacak olan da onun elleridir. O zaman beni size Her Seye Egemen RAB'bin gönderdigini anlayacaksiniz.
\par 10 "Küçük isleri yapma gününü kim küçümsüyor? Insanlar Zerubbabil'in elinde çekülü görünce sevinecekler. -"Bu yedi kandil RAB'bin bütün yeryüzünde dolasan gözleridir."-
\par 11 Melege, "Kandilligin sagindaki ve solundaki bu iki zeytin agaci nedir?" diye sordum,
\par 12 "Altin gibi yag akitan iki altin olugun yanindaki bu iki zeytin dali nedir?"
\par 13 "Bunlarin anlamini bilmiyor musun?" diye karsilik verdi. "Hayir, efendim" dedim.
\par 14 :14 Melek, "Bunlar bütün dünyanin Rabbi'ne hizmet eden, zeytinyagiyla kutsanmis iki kisidir" diye açikladi. Altinci Görüm: Uçan Tomar

\chapter{5}

\par 1 Gözlerimi yine kaldirip bakinca, uçan bir tomar gördüm.
\par 2 Melek, "Ne görüyorsun?" diye sordu. "Uçan bir tomar görüyorum. Uzunlugu yirmi, genisligi on arsin" diye yanitladim.
\par 3 Melek, "Bütün ülkeye yagacak lanettir bu" dedi, "Tomarin bir yanina yazilanlar uyarinca, hirsizlik eden herkes sökülüp atilacak; öbür yanina yazilanlar uyarinca da yalan yere ant içenler kovulacak.
\par 4 Her Seye Egemen RAB, 'Lanet yagdiracagim' diyor, 'Hirsizin ve benim adimla yalan yere ant içenin evi üzerine lanet yagacak. Ve lanet o evin üzerinde kalacak; kerestesiyle, taslariyla birlikte evin tümünü yok edecek.'"
\par 5 Sonra benimle konusan melek yaklasip, "Gözlerini kaldir" dedi, "Ortaya çikan su nesnenin ne olduguna bak."
\par 6 "Nedir?" diye sordum. "Bir ölçü kabi" dedi, sonra ekledi: "Bu, bütün ülke halkinin suçudur."
\par 7 Derken kursun kapak kaldirildi. Kabin içinde bir kadin oturuyordu.
\par 8 Melek, "Iste bu kötülüktür!" diyerek kadini gerisingeri ölçü kabina itip kursun kapagi yerine koydu.
\par 9 Gözlerimi kaldirip bakinca, rüzgarda uçarak yaklasan iki kadin gördüm. Leylek kanatlarina benzeyen kanatlari vardi. Kabi yerle gök arasina kaldirdilar.
\par 10 Benimle konusan melege, "Kabi nereye götürüyorlar?" diye sordum.
\par 11 "Kadin için bir ev yapmak üzere Sinar topraklarina" diye yanitladi, "Ev hazir olunca kap oraya, yerine konulacak."

\chapter{6}

\par 1 Yine gözlerimi kaldirip baktim, iki tunç* dagin arasindan çikip gelen dört savas arabasi gördüm.
\par 2 Birinci savas arabasinin kizil, ikincisinin siyah,
\par 3 üçüncüsünün beyaz, dördüncüsünün benekli atlari vardi. Atlarin hepsi güçlüydü.
\par 4 Benimle konusan melege, "Bunlar ne, efendim?" diye sordum.
\par 5 Melek söyle karsilik verdi: "Bunlar bütün dünyanin Rab'bine hizmet ettikleri yerden çikan gögün dört ruhudur.
\par 6 Siyah atlarin çektigi savas arabasi bölgenin kuzeyine, beyaz atlarinki batiya, benekli atlarinki de güneye dogru gidiyor."
\par 7 Yola çiktiklarinda güçlü atlar yeryüzünü dolasmak üzere gitmek istiyorlardi. Melek, "Gidin, yeryüzünü dolasin!" deyince, gidip yeryüzünü dolastilar.
\par 8 Sonra melek bana seslendi: "Bak, bölgenin kuzeyine gidenler, orada öfkemi yatistirdilar."
\par 9 RAB bana söyle seslendi:
\par 10 "Armaganlari sürgünden dönenlerden -Babil'den gelen Helday, Toviya ve Yedaya'dan- al ve ayni gün Sefanya oglu Yosiya'nin evine git.
\par 11 Aldigin altinla gümüsten bir taç yaparak Yehosadak oglu Baskâhin Yesu'nun basina tak.
\par 12 Ona Her Seye Egemen RAB söyle diyor de: 'Iste Dal adindaki adam! Bulundugu yerde filizlenecek ve RAB'bin Tapinagi'ni kuracak.
\par 13 Evet, RAB'bin Tapinagi'ni kuracak olan odur. Görkemle kusanacak, tahtinda oturup egemenlik sürecek. Tahtinda oturan kâhin olacak. Ikisi arasinda tam bir uyum olacak.'
\par 14 Helday'in, Toviya'nin, Yedaya'nin, Sefanya oglu Yosiya'nin anisina taç RAB'bin Tapinagi'na konulacak.
\par 15 Uzaktakiler de gelip RAB'bin Tapinagi'nin yapiminda çalisacak. Böylece beni size Her Seye Egemen RAB'bin gönderdigini anlayacaksiniz. Tanriniz RAB'bin sözüne özenle uyarsaniz bütün bunlar gerçeklesecektir."

\chapter{7}

\par 1 Kral Darius'un kralliginin dördüncü yilinin dokuzuncu ayi* olan Kislev ayinin dördüncü günü RAB Zekeriya'ya seslendi.
\par 2 Beytel halki, Her Seye Egemen RAB'bin Tapinagi'ndaki kâhinlerle peygamberlere, "Yillardir yaptigimiz gibi besinci ay oruç* tutup aglayalim mi?" diye sormus ve RAB'be yalvarmalari için Sareser'i, Regem-Melek'i ve adamlarini göndermisti.
\par 4 Her Seye Egemen RAB bana dedi ki,
\par 5 "Bütün ülke halkina ve kâhinlere sor: 'Yetmis yildir besinci ve yedinci aylarda oruç tutup dövündügünüzde gerçekten benim için mi oruç tuttunuz?
\par 6 Yiyip içerken kendiniz için yiyip içmiyor muydunuz?
\par 7 Yerusalim'le çevresindeki kentler gönenç içinde yasarken, Negev ve Sefela insanlarla doluyken, RAB'bin önceki peygamberler araciligiyla açikladigi sözler bunlar degil mi?'"
\par 8 RAB Zekeriya'ya yine seslendi:
\par 9 "Her Seye Egemen RAB diyor ki, 'Gerçek adaletle yargilayin; birbirinize sevgi ve sevecenlik gösterin.
\par 10 Dul kadina, öksüze, yabanciya, yoksula baski yapmayin. Yüreginizde birbirinize karsi kötülük tasarlamayin.'
\par 11 "Ama atalarimiz dinlemek istemediler; inatla sirtlarini çevirdiler, duymamak için kulaklarini tikadilar.
\par 12 Kutsal Yasa'yi ve Her Seye Egemen RAB'bin kendi Ruhu'yla gönderdigi, önceki peygamberler araciligiyla ilettigi sözleri dinlememek için yüreklerini tas gibi sertlestirdiler. Bu yüzden Her Seye Egemen RAB onlara çok öfkelendi.
\par 13 "'Madem ben çagirinca dinlemediler' diyor Her Seye Egemen RAB, 'Onlar çagirinca, ben de onlari dinlemeyecegim.
\par 14 Onlari tanimadiklari uluslarin arasina firtina gibi dagittim. Geride biraktiklari ülke öyle issiz kaldi ki, oraya kimse gidip gelemez oldu. Güzelim ülkeyi viraneye çevirdiler.'"

\chapter{8}

\par 1 Her Seye Egemen RAB bana yine seslendi:
\par 2 Her Seye Egemen RAB, "Siyon için büyük kiskançlik duyuyorum" diyor, "Evet, onu siddetle kiskaniyorum.
\par 3 Siyon'a dönecek ve Yerusalim'de oturacagim. Yerusalim'e Sadik Kent, Her Seye Egemen RAB'bin dagina Kutsal Dag denecek.
\par 4 "Ilerlemis yaslarindan ötürü ellerinde bastonlariyla yasli erkeklerle kadinlar yine Yerusalim meydanlarinda oturacaklar. Kentin meydanlari orada oynayan erkek ve kiz çocuklarla dolacak." Böyle diyor Her Seye Egemen RAB.
\par 6 Her Seye Egemen RAB diyor ki, "O günlerde sürgünden dönen halkin gözünde bu olanaksiz olsa da, benim gözümde de böyle mi olmali?" Böyle diyor Her Seye Egemen RAB.
\par 7 "Halkimi dogudaki, batidaki ülkelerden kurtarip geri getirecegim. Yerusalim'de yasayacak, halkim olacaklar; ben de onlarin sadik ve adil Tanrisi olacagim." Böyle diyor Her Seye Egemen RAB.
\par 9 "Her Seye Egemen RAB'bin Tapinagi'nin kurulmasi için temel atildiginda orada bulunan peygamberlerin bu günlerde söyledigi sözleri duyan sizler yüreklenin!" diyor Her Seye Egemen RAB,
\par 10 "O günlerden önce insan ya da hayvan için ücret yoktu. Düsman yüzünden hiç kimse güvenlik içinde gidip gelemiyordu. Çünkü herkesi birbirine düsürmüstüm.
\par 11 Ama simdi sürgünden dönen bu halka geçmis günlerde davrandigim gibi davranmayacagim." Böyle diyor Her Seye Egemen RAB,
\par 12 "Ekilen tohum verimli olacak; asma üzüm, toprak ürün, gökler çiy verecek. Bunlarin tümünü sürgünden dönen bu halka mülk olarak verecegim.
\par 13 Sizi kurtaracagim, ey Yahuda ve Israil halki. Siz uluslar arasinda nasil lanet konusu olduysaniz, simdi de bereket kaynagi olacaksiniz. Korkmayin, yürekli olun!"
\par 14 Her Seye Egemen RAB söyle diyor: "Atalariniz beni öfkelendirdiginde basiniza felaket getirmeyi tasarladim ve vazgeçmedim" diyor Her Seye Egemen RAB,
\par 15 "Simdi de Yerusalim ve Yahuda halkina yine iyilik yapmayi tasarladim. Korkmayin!
\par 16 Yapmaniz gerekenler sunlardir: Birbirinize gerçegi söyleyin, kent kapilarinizda esenligi saglayan gerçek adaletle yargilayin,
\par 17 yüreginizde birbirinize karsi kötülük tasarlamayin, yalan yere ant içmekten tiksinin. Çünkü ben bütün bunlardan nefret ederim." Böyle diyor RAB.
\par 18 Her Seye Egemen RAB bana yine seslendi:
\par 19 Her Seye Egemen RAB diyor ki, "Dördüncü, besinci, yedinci ve onuncu aylarin* oruçlari* Yahuda halki için sevinç, cosku dolu mutlu bayramlar olacak. Bu nedenle gerçegi ve esenligi sevin."
\par 20 Her Seye Egemen RAB diyor ki, "Daha birçok halk, birçok kentte yasayanlar gelecek.
\par 21 Bir kentte yasayanlar baska kente gidip, 'RAB'be yalvarmak, Her Seye Egemen RAB'be yönelmek için hemen yola çikalim. Ben de gidecegim' diyecekler.
\par 22 Her Seye Egemen RAB'be yönelmek, O'na yalvarmak için çok sayida halkla birçok ulus Yerusalim'e gelecek."
\par 23 Her Seye Egemen RAB diyor ki, "O günlerde her dil ve ulustan on kisi bir Yahudi'nin eteginden tutup, 'Izin verin, sizinle gidelim. Çünkü Tanri'nin sizinle oldugunu duyduk' diyecekler."

\chapter{9}

\par 1 Bildiri: RAB'bin sözü Hadrak ülkesine ve Sam Kenti'ne yöneliktir. Çünkü insanlarin, özellikle bütün Israil oymaklarinin gözü RAB'be çevrilidir.
\par 2 Bu söz Hadrak sinirindaki Hama'ya, Çok becerikli olmasina karsin Sur ve Sayda kentlerine de yöneliktir.
\par 3 Sur kendine bir kale yapti; Toprak kadar gümüs Ve sokaktaki çamur kadar altin biriktirdi.
\par 4 Ama Rab onun mal varligini alip götürecek; Denizdeki gücünü yok edecek Ve ates kenti yiyip bitirecek.
\par 5 Askelon bunu görünce korkacak; Gazze acidan kivranacak, Ekron da öyle, çünkü umudu sönecek. Gazze kralini yitirecek, Askelon issiz kalacak.
\par 6 Asdot'ta melez bir halk oturacak, Filistliler'in gururunu kiracagim.
\par 7 Agizlarindan kani alinmamis eti, Dislerinin arasindan yasak yiyecekleri alacagim. Sag kalanlar Tanrimiz'a baglanacak Ve Yahuda oymaginda bir boy sayilacak. Ekron Yevuslular gibi olacak.
\par 8 Akin eden ordulara karsi Evimin çevresinde ordugah kuracagim. Hiçbir kiyici Bir daha halkimin üzerinden geçmeyecek, Çünkü artik halkimi ben gözetiyorum.
\par 9 Ey Siyon kizi*, sevinçle cos! Sevinç çigliklari at, ey Yerusalim kizi*! Iste kralin! O adil kurtarici ve alçakgönüllüdür. Esege, evet, sipaya, Esek yavrusuna binmis sana geliyor!
\par 10 Savas arabalarini Efrayim'den, Atlari Yerusalim'den uzaklastiracagim. Savas yaylari kirilacak. Kraliniz uluslara barisi duyuracak, Onun egemenligi bir denizden bir denize, Firat'tan yeryüzünün uçlarina dek uzanacak.
\par 11 Size gelince, Sizinle yaptigim kurban kaniyla yürürlüge girmis antlasma uyarinca, Sürgündeki halkinizi Susuz çukurdan çikarip özgür kilacagim.
\par 12 Kalenize dönün, Ey siz, umut sürgünleri! Bugün bildiriyorum ki, Size yitirdiginizin iki katini verecegim.
\par 13 Yahuda'yi yayimi gerer gibi gerecegim Ve Efrayim'i ok gibi ona dolduracagim. Ogullarini Grekler'e karsi uyandiracagim, ey Siyon Ve seni bir savasçinin kilici gibi yapacagim.
\par 14 O zaman RAB halkinin üzerinde görünecek, Oku simsek gibi çakacak. Egemen RAB boru çalacak Ve güney firtinalariyla ilerleyecek.
\par 15 Onlari Her Seye Egemen RAB koruyacak. Düsmanlarini yok edecek Ve sapan taslariyla yenecekler. Sarap içmis gibi içip gürleyecek Ve kurban kani serpmekte kullanilan çanaklar gibi sunagin köselerine dolacaklar.
\par 16 O gün Tanrilari RAB Sürüsü olan halkini kurtaracak. O'nun ülkesinde taç mücevherleri gibi parlayacaklar.
\par 17 Ne yakisikli ve güzel olacaklar! Delikanlilar tahilla, Genç kizlar yeni sarapla güçlenecek. Rab Yahuda ve Efrayim'i Kayiracak

\chapter{10}

\par 1 Ilkbaharda RAB'den yagmur dileyin. O'dur yagmur bulutlarini olusturan. Insanlara yagmur saganaklari Ve herkese tarlada ot verir.
\par 2 Oysa aile putlarindan alinan yanitlar bostur, Falcilar yalan görümler görür Ve gerçek yani olmayan düsler anlatarak Bosuna avuturlar. Bu yüzden halk sürü gibi daginik, Sikinti çekiyor, çünkü çoban yok.
\par 3 RAB söyle diyor: "Öfkem çobanlara karsi alevlendi, Önderleri cezalandiracagim. Her Seye Egemen RAB kendi sürüsünü, Yahuda halkini kayiracak. Görkemli savas ati gibi yapacak onlari.
\par 4 Köse tasi, Çadir kazigi, Savas yayi Ve bütün önderler Yahuda'dan çikacak.
\par 5 Savasta düsmanlarini sokaklardaki çamurda çigneyen yigitler gibi olacaklar. RAB onlarla oldugu için Savasacak ve atlilari utandiracaklar.
\par 6 "Yahuda halkini güçlendirecegim, Yusuf soyunu kurtarip Sürgünden geri getirecegim. Çünkü onlara aciyorum. Sanki onlari reddetmemisim gibi olacaklar. Çünkü ben onlarin Tanrisi RAB'bim ve onlari yanitlayacagim.
\par 7 Efrayimliler yigitler gibi olacaklar, Sarap içmis gibi yürekleri cosacak. Çocuklari bunu görüp neselenecek, Yürekleri RAB'de sevinç bulacak.
\par 8 "Islik çalip onlari toplayacagim, Onlari kesinlikle kurtaracagim. Eskiden oldugu gibi Yine çogalacaklar.
\par 9 Onlari halklar arasina dagittimsa da, Uzak ülkelerde beni animsayacaklar; Çocuklariyla birlikte sag kalacak ve geri dönecekler.
\par 10 Onlari Misir'dan geri getirecek, Asur'dan toplayacagim; Gilat'a, Lübnan'a getirecegim. Onlara yeterince yer bulunmayacak.
\par 11 Sikinti denizinden geçecekler, Denizin dalgalari yatisacak, Nil'in bütün derinlikleri kuruyacak. Asur'un gururu alasagi edilecek, Misir'in krallik asasi elinden alinacak.
\par 12 Halkimi kendi gücümle güçlendirecegim, Adima layik bir yasam sürdürecekler." Böyle diyor RAB.

\chapter{11}

\par 1 Ey Lübnan, kapilarini aç ki, Ates sedir agaçlarini yakip yok etsin!
\par 2 Ey çam agaci, haykir! Sedir agaci yikildi, Ulu agaçlar yok oldu! Haykirin, ey Basan meseleri, Gür ormanin agaçlari devrildi!
\par 3 Çobanlarin haykirisini duy, Çünkü güzelim otlaklari yok oldu! Genç aslanlarin kükremesini dinle, Çünkü Seria Irmagi'nin kiyisindaki agaçlik yok oldu!
\par 4 Tanrim RAB, "Kesime ayrilmis sürüyü sen güt" diyor,
\par 5 "Sürüyü satin alanlar koyunlari kesiyor ama cezalarini çekmiyorlar. Koyunlari satanlar da, 'Tanri'ya övgüler olsun, zengin oldum!' diyorlar. Çobanlar kendi sürülerine acimiyor.
\par 6 Çünkü ülkede yasayan halka artik acimayacagim" diyor RAB, "Herkesi kendi komsusunun ve kralinin eline teslim edecegim. Ülkeyi ezecekler, ben de halki ellerinden kurtarmayacagim."
\par 7 Bunun üzerine kesime ayrilmis sürünün özellikle ezilenlerini güttüm. Elime iki degnek aldim; birine "Lütuf", ötekine "Birlik" adini koydum. Böylece sürüyü gütmeye basladim.
\par 8 Bir ayda üç çobani basimdan savdim. Çünkü ben sürüden bikmistim, sürü de benden tiksinmisti.
\par 9 Sürüye, "Artik sizi gütmeyecegim. Ölen ölsün, kesilen kesilsin, geri kalanlar da birbirinin etini yesin" dedim.
\par 10 Sonra "Lütuf" adindaki degnegimi aldim ve bütün uluslarla yapmis oldugum antlasmayi bozmak için kirdim.
\par 11 Böylece antlasma o gün bozuldu. Beni gözleyen sürünün ezilenleri RAB'bin sözünün yerine geldigini anladilar.
\par 12 Onlara, "Uygun görürseniz ücretimi ödeyin, yoksa bos verin" dedim. Onlar da ücret olarak bana otuz gümüs verdiler.
\par 13 RAB bana, "Çömlekçiye at" dedi. Böylece bana biçtikleri yüksek degerin karsiligi olan otuz gümüsü alip RAB'bin Tapinagi'ndaki çömlekçiye attim.
\par 14 Sonra Yahuda ile Israil arasindaki kardesligi bozmak için "Birlik" adindaki öteki degnegimi kirdim.
\par 15 RAB bana, "Sen yine akilsiz bir çoban gibi donat kendini" dedi,
\par 16 "Ülkeye öyle bir çoban atayacagim ki, yitiklere bakmayacak, dagilmislari aramayacak, yaralilari iyilestirmeyecek, saglamlari beslemeyecek. Ancak semiz koyunlarin etini yiyecek, tirnaklarini koparacak.
\par 17 "Sürüyü terk eden degersiz çobanin vay haline! Kiliç kolunu ve sag gözünü vursun! Kolu tamamen kurusun, Sag gözü kör olsun!"

\chapter{12}

\par 1 Bildiri: Iste RAB'bin Israil'e iliskin sözleri. Gökleri geren, yeryüzünün temelini atan, insanin içindeki ruha biçim veren RAB söyle diyor:
\par 2 "Yerusalim'i çevredeki bütün halklari sersemleten bir kâse* yapacagim. Yerusalim gibi Yahuda da kusatma altina alinacak.
\par 3 O gün Yerusalim'i bütün halklar için agir bir tas yapacagim. Onu kaldirmaya yeltenen herkes agir yaralanacak. Yeryüzünün bütün uluslari Yerusalim'e karsi birlesecek.
\par 4 O gün her ati dehsete düsürecek, her atliyi çilgina döndürecegim. Yahuda halkini gözetecegim, ama öbür halklarin bütün atlarini kör edecegim.
\par 5 O zaman Yahuda önderleri, 'Her Seye Egemen Tanrisi RAB'be güvenen Yerusalim halki güç kaynagimizdir' diye düsünecekler.
\par 6 "O gün Yahuda önderlerini odunlarin ortasinda yanan bir mangal gibi, ekin demetleri arasinda alev alev yanan bir mesale gibi yapacagim. Sagda solda, çevredeki bütün halklari yakip yok edecekler. Yerusalim ise sapasaglam yerinde duracak.
\par 7 "Ben RAB önce çadirlarda oturan Yahuda halkini kurtaracagim. Öyle ki, Davut soyuyla Yerusalim'de oturanlar Yahuda'dan daha çok onura kavusmasin.
\par 8 Ben RAB o gün Yerusalim'de oturanlari koruyacagim. Böylece aralarindaki en güçsüz kisi Davut gibi, Davut soyu da Tanri gibi, kendilerine öncülük eden RAB'bin melegi gibi olacak.
\par 9 O gün Yerusalim'e saldiran bütün uluslari yok etmeye baslayacagim.
\par 10 "Davut soyuyla Yerusalim'de oturanlarin üzerine lütuf ve yakaris ruhunu dökecegim. Bana, yani destiklerine bakacaklar; biricik oglu için yas tutan biri gibi yas tutacak, ilk oglu için aci çeken biri gibi aci çekecekler.
\par 11 O gün Yerusalim'de tutulan yas, Megiddo Ovasi'nda, Hadat-Rimmon'da tutulan yas gibi büyük olacak.
\par 12 Ülkede her boy kendi içinde yas tutacak: Davut, Natan, Levi, Simi boyundan ve geri kalan boylardan aileler. Her boyun erkekleri ayri, kadinlari ayri yas tutacak."

\chapter{13}

\par 1 "O gün Davut soyunu ve Yerusalim'de yasayanlari günahtan ve ruhsal kirlilikten arindirmak için bir pinar açilacak.
\par 2 O gün ülkeden putlarin adlarini kaldiracagim, bir daha anilmayacaklar" diyor Her Seye Egemen RAB, "Sahte peygamberleri de, kirli ruhu da ülkeden uzaklastiracagim.
\par 3 Biri yine peygamberlik edecek olursa, öz annesiyle babasi, 'Öleceksin, çünkü RAB'bin adiyla yalan söylüyorsun' diyecekler. Peygamberlik ettiginde de öz annesi babasi onun bedenini desecekler.
\par 4 "O gün her peygamber peygamberlik ederken gördügü görümden utanacak; insanlari aldatmak için çuldan giysi giymeyecek.
\par 5 'Ben peygamber degilim, çiftçiyim. Gençligimden beri hep tarlada çalistim' diyecek.
\par 6 Biri, 'Bagrindaki bu yaralar ne?' diye sordugunda da, 'Bunlar dostlarimin evinde aldigim yaralar' diye yanitlayacak."
\par 7 "Uyan, ey kiliç! Çobanima, yakinima karsi harekete geç" Diyor Her Seye Egemen RAB. "Çobani vur da Koyunlar darmadagin olsun. Ben de elimi küçüklere karsi kaldiracagim.
\par 8 Bütün ülkede" diyor RAB, "Halkin üçte ikisi vurulup ölecek, Üçte biri sag kalacak.
\par 9 Kalan üçte birini atesten geçirecegim, Onlari gümüs gibi aritacagim, Altin gibi sinayacagim. Bana yakaracaklar, Ben de onlara karsilik verecegim, 'Bunlar benim halkim' diyecegim. Onlar da, 'Tanrimiz RAB'dir' diyecekler." Mesih'in Gelisi ve Kralligi

\chapter{14}

\par 1 Iste RAB'bin günü geliyor! Ey Yerusalim halki, senden yagmalanan mal gözlerinin önünde paylasilacak.
\par 2 Yerusalim'e karsi savasmalari için bütün uluslari bir araya getirecegim. Kent ele geçirilecek, evler yagmalanacak, kadinlarin irzina geçilecek. Kentte yasayanlarin yarisi sürgüne gönderilecek, geri kalanlar kentte kalacak.
\par 3 Sonra RAB, savas zamanlarinda yaptigi gibi, gidip bu uluslara karsi savasacak.
\par 4 O gün O'nun ayaklari Yerusalim'in dogusundaki Zeytin Dagi'nin üzerinde duracak. Zeytin Dagi doguya ve batiya dogru ortadan yarilip çok büyük bir vadi olusturacak. Dagin yarisi kuzeye, öbür yarisi güneye çekilecek.
\par 5 Yarilan dagimin olusturdugu vadiden kaçacaksiniz, çünkü vadi Asal'a dek uzanacak. Yahuda Krali Uzziya döneminde depremden nasil kaçtiysaniz, öyle kaçacaksiniz. O zaman Tanrim RAB bütün kutsallarla birlikte gelecek!
\par 6 O gün isik olmayacak, isik veren cisimler kararacak.
\par 7 Özel bir gün, yalniz RAB'bin bildigi bir gün olacak. Gece de gündüz de olmayacak. Gece aydinlik olacak.
\par 8 O gün Yerusalim'in içinden diri sular akacak. Yaz kis sularin yarisi Lut Gölü'ne, öbür yarisi Akdeniz'e akacak.
\par 9 RAB bütün dünyanin krali olacak. O gün yalniz RAB, yalniz O'nun adi kalacak.
\par 10 Bütün ülke Geva'dan Yerusalim'in güneyindeki Rimmon'a dek Arava Ovasi gibi olacak. Ama Yerusalim yükseltilecek ve Benyamin Kapisi'ndan ilk kapiya, Köse Kapisi'na, Hananel Kulesi'nden kralin üzüm sikma çukurlarina dek yerli yerinde duracak.
\par 11 Insanlar oraya yerlesip güvenlik içinde yasayacak. Yerusalim bir daha yikima ugramayacak.
\par 12 Yerusalim'e karsi savasan bütün halklari RAB su belayla cezalandiracak: Daha sagken bedenleri, gözleri, dilleri çürüyecek.
\par 13 O gün RAB insanlari büyük dehsete düsürecek. Herkes yanindakinin elini yakalayacak, birbirlerine saldiracaklar.
\par 14 Yahudalilar da Yerusalim'de savasacak. Çevredeki bütün uluslarin serveti, çok miktarda altin, gümüs, giysi toplanacak.
\par 15 Düsman ordugahlarindaki bütün hayvanlar da at, katir, deve, esek benzer bir belaya çarptirilacak.
\par 16 Yerusalim'e saldiran uluslardan sag kalanlarin hepsi Her Seye Egemen RAB olan Kral'a tapinmak ve Çardak Bayrami'ni* kutlamak için yildan yila Yerusalim'e gidecekler.
\par 17 Yeryüzü halklarindan hangisi Her Seye Egemen RAB olan Kral'a tapinmak için Yerusalim'e gitmezse, ülkesine yagmur yagmayacak.
\par 18 Misirlilar bunlara katilip Yerusalim'e gitmezlerse, RAB onlari da Çardak Bayrami'ni kutlamak için Yerusalim'e gitmeyen bütün uluslarin basina getirdigi ayni belayla cezalandiracak.
\par 19 Misirlilar'a ve Çardak Bayrami'ni kutlamak için Yerusalim'e gitmeyen bütün uluslara verilecek ceza budur.
\par 20 O gün atlarin çingiraklari üzerine, "RAB'be adanmistir" diye yazilacak. RAB'bin Tapinagi'ndaki kazanlar da sunagin önündeki çanaklar gibi olacak.
\par 21 Yerusalim ve Yahuda'da her kazan Her Seye Egemen RAB'be adanacak. Kurban kesmeye gelenler bu kazanlari kurban etini pisirmek için kullanacaklar. O gün Her Seye Egemen RAB'bin Tapinagi'nda artik tüccar bulunmayacak.


\end{document}