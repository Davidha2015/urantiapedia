\begin{document}

\title{Markos}


\chapter{1}

\par 1 Tanri'nin Oglu Isa Mesih'le* ilgili Müjde'nin baslangici.
\par 2 Peygamber Yesaya'nin Kitabi'nda söyle yazilmistir: "Iste, habercimi senin önünden gönderiyorum; O senin yolunu hazirlayacak."
\par 3 "Çölde haykiran, `Rab'bin yolunu hazirlayin, Geçecegi patikalari düzleyin' diye sesleniyor."
\par 4 Böylece Vaftizci Yahya çölde ortaya çikti. Insanlari, günahlarinin bagislanmasi için tövbe edip vaftiz* olmaya çagiriyordu.
\par 5 Bütün Yahudiye halki ve Yerusalimliler'in hepsi ona geliyor, günahlarini itiraf ediyor, onun tarafindan Seria Irmagi'nda vaftiz ediliyordu.
\par 6 Yahya'nin deve tüyünden giysisi, belinde deri kusagi vardi. Çekirge ve yaban bali yerdi.
\par 7 Su haberi yayiyordu: "Benden sonra benden daha güçlü olan geliyor. Egilip O'nun çariklarinin bagini çözmeye bile layik degilim.
\par 8 Ben sizi suyla vaftiz ettim, ama O sizi Kutsal Ruh'la vaftiz edecektir."
\par 9 O günlerde Celile'nin Nasira Kenti'nden çikip gelen Isa, Yahya tarafindan Seria Irmagi'nda vaftiz edildi.
\par 10 Tam sudan çikarken, göklerin yarildigini ve Ruh'un güvercin gibi üzerine indigini gördü.
\par 11 Göklerden, "Sen benim sevgili Oglum'sun, senden hosnudum" diyen bir ses duyuldu.
\par 12 O an Ruh, Isa'yi çöle gönderdi.
\par 13 Isa çölde kaldigi kirk gün boyunca Seytan tarafindan denendi. Yabanil hayvanlar arasindaydi, melekler O'na hizmet ediyordu.
\par 14 Yahya'nin tutuklanmasindan sonra Isa, Tanri'nin Müjdesi'ni duyura duyura Celile'ye gitti.
\par 15 "Zaman doldu" diyordu, "Tanri'nin Egemenligi yaklasti. Tövbe edin, Müjde'ye inanin!"
\par 16 Isa, Celile Gölü'nün kiyisindan geçerken, göle ag atmakta olan Simun ile kardesi Andreas'i gördü. Bu adamlar balikçiydi.
\par 17 Isa onlara, "Ardimdan gelin" dedi, "Sizleri insan tutan balikçilar yapacagim."
\par 18 Onlar da hemen aglarini birakip O'nun ardindan gittiler.
\par 19 Isa biraz ileri gidince Zebedi'nin ogullari Yakup'la Yuhanna'yi gördü. Teknede aglarini onariyorlardi.
\par 20 Hemen onlari çagirdi. Onlar da babalari Zebedi'yi isçilerle birlikte teknede birakip Isa'nin ardindan gittiler.
\par 21 Kefarnahum'a girdiler. Sabat Günü* Isa havraya gidip ögretmeye basladi.
\par 22 Halk O'nun ögretisine sasip kaldi. Çünkü onlara din bilginleri* gibi degil, yetkili biri gibi ögretiyordu.
\par 23 Tam o sirada havrada bulunan ve kötü ruha tutulmus bir adam, "Ey Nasirali Isa, bizden ne istiyorsun?" diye bagirdi. "Bizi mahvetmeye mi geldin? Senin kim oldugunu biliyorum, Tanri'nin Kutsali'sin sen!"
\par 25 Isa, "Sus, çik adamdan!" diyerek kötü ruhu azarladi.
\par 26 Kötü ruh adami sarsti ve büyük bir çiglik atarak içinden çikti.
\par 27 Herkes sasip kaldi. Birbirlerine, "Bu nasil sey?" diye sormaya basladilar. "Yepyeni bir ögreti! Kötü ruhlara bile yetkiyle buyruk veriyor, onlar da sözünü dinliyor."
\par 28 Böylece Isa'yla ilgili haber, Celile bölgesinin her yerine hizla yayildi.
\par 29 Isa havradan çikar çikmaz, Yakup ve Yuhanna ile birlikte Simun ve Andreas'in evine gitti.
\par 30 Simun'un kaynanasi atesler içinde yatiyordu. Durumu hemen Isa'ya bildirdiler.
\par 31 O da hastaya yaklasti, elinden tutup kaldirdi. Kadinin atesi düstü, onlara hizmet etmeye basladi.
\par 32 Aksam olup günes batinca, bütün hastalari ve cinlileri Isa'ya getirdiler.
\par 33 Bütün kent halki kapiya toplanmisti.
\par 34 Isa, çesitli hastaliklara yakalanmis birçok kisiyi iyilestirdi, birçok cini kovdu. Cinlerin konusmasina izin vermiyordu. Çünkü onlar kendisinin kim oldugunu biliyorlardi.
\par 35 Sabah çok erkenden, ortalik henüz agarmadan Isa kalkti, evden çikip issiz bir yere gitti, orada dua etmeye basladi.
\par 36 Simun ile yanindakiler Isa'yi aramaya çiktilar.
\par 37 O'nu bulunca, "Herkes seni ariyor!" dediler.
\par 38 Isa onlara, "Baska yerlere, yakin kasabalara gidelim" dedi. "Oralarda da Tanri sözünü duyurayim. Bunun için çikip geldim."
\par 39 Böylece havralarinda Tanri sözünü duyurarak ve cinleri kovarak bütün Celile bölgesini dolasti.
\par 40 Isa'ya cüzamli* biri geldi, diz çökerek, "Istersen beni temiz kilabilirsin" diye yalvardi.
\par 41 Isa'nin yüregi sizladi, elini uzatip adama dokundu, "Isterim, temiz ol!" dedi.
\par 42 Adam aninda cüzamdan kurtulup tertemiz oldu.
\par 43 Isa onu sikica uyararak hemen yanindan uzaklastirdi.
\par 44 "Sakin kimseye bir sey söyleme!" dedi. "Git, kâhine* görün ve cüzamdan temizlendigini herkese kanitlamak için Musa'nin buyurdugu sunulari sun."
\par 45 Ne var ki, adam çikip gitti, olayla ilgili haberi her tarafa yayip duyurmaya basladi. Öyle ki, Isa artik hiçbir kente açikça giremez oldu. Ancak disarida, issiz yerlerde kaliyordu. Ve halk her yerden O'na akin ediyordu.

\chapter{2}

\par 1 Birkaç gün sonra Isa tekrar Kefarnahum'a geldiginde, evde oldugu duyuldu.
\par 2 O kadar çok insan toplandi ki, artik kapinin önünde bile duracak yer kalmamisti. Isa onlara Tanri sözünü anlatiyordu.
\par 3 Bu arada O'na dört kisinin tasidigi felçli bir adami getirdiler.
\par 4 Kalabaliktan O'na yaklasamadiklari için, bulundugu yerin üzerindeki dami delip açarak felçliyi üstünde yattigi silteyle birlikte asagi indirdiler.
\par 5 Isa onlarin imanini görünce felçliye, "Oglum, günahlarin bagislandi" dedi.
\par 6 Orada oturan bazi din bilginleri ise içlerinden söyle düsündüler: "Bu adam neden böyle konusuyor? Tanri'ya küfrediyor! Tanri'dan baska kim günahlari bagislayabilir?"
\par 8 Akillarindan geçeni hemen ruhunda sezen Isa onlara, "Aklinizdan neden böyle seyler geçiriyorsunuz?" dedi.
\par 9 "Hangisi daha kolay, felçliye, `Günahlarin bagislandi' demek mi, yoksa, 'Kalk, silteni topla, yürü' demek mi?
\par 10 Ne var ki, Insanoglu'nun* yeryüzünde günahlari bagislama yetkisine sahip oldugunu bilesiniz diye..." Sonra felçliye, "Sana söylüyorum, kalk, silteni topla, evine git!" dedi.
\par 12 Adam kalkti, hemen siltesini topladi, hepsinin gözü önünde çikip gitti. Herkes sasakalmisti. Tanri'yi övüyorlar, "Böylesini hiç görmemistik" diyorlardi.
\par 13 Isa yine çikip göl kiyisina gitti. Bütün halk yanina geldi, O da onlara ögretmeye basladi.
\par 14 Yoldan geçerken, vergi toplama yerinde oturan Alfay oglu Levi'yi gördü. Ona, "Ardimdan gel" dedi. Levi de kalkip Isa'nin ardindan gitti.
\par 15 Sonra Isa, Levi'nin evinde yemek yerken, birçok vergi görevlisiyle* günahkâr O'nunla ve ögrencileriyle birlikte sofraya oturmustu. O'nu izleyen böyle birçok kisi vardi.
\par 16 Ferisiler'den* bazi din bilginleri, O'nu günahkârlar ve vergi görevlileriyle birlikte yemekte görünce ögrencilerine, "Niçin vergi görevlileri ve günahkârlarla birlikte yemek yiyor?" diye sordular.
\par 17 Bunu duyan Isa onlara, "Saglamlarin degil, hastalarin hekime ihtiyaci var" dedi. "Ben dogru kisileri degil, günahkârlari çagirmaya geldim."
\par 18 Yahya'nin ögrencileriyle Ferisiler oruç tutarken, bazi kisiler Isa'ya gelip, "Yahya'nin ve Ferisiler'in ögrencileri oruç tutuyor da senin ögrencilerin neden tutmuyor?" diye sordular.
\par 19 Isa söyle karsilik verdi: "Güvey aralarinda oldugu sürece davetliler oruç tutar mi? Güvey aralarinda oldukça oruç tutmazlar!
\par 20 Ama güveyin aralarindan alinacagi günler gelecek, onlar iste o zaman, o gün oruç tutacaklar.
\par 21 Hiç kimse eski giysiyi yeni kumas parçasiyla yamamaz. Yoksa yeni yama çeker, eski giysiden kopar, yirtik daha beter olur.
\par 22 Hiç kimse yeni sarabi eski tulumlara doldurmaz. Yoksa sarap tulumlari patlatir, sarap da tulumlar da mahvolur. Yeni sarap yeni tulumlara doldurulur."
\par 23 Bir Sabat Günü* Isa ekinler arasindan geçiyordu. Ögrencileri yolda giderken basaklari koparmaya basladilar.
\par 24 Ferisiler Isa'ya, "Bak, Sabat Günü yasak olani neden yapiyorlar?" dediler.
\par 25 Isa onlara, "Davut'la yanindakiler aç ve muhtaç kalinca Davut'un ne yaptigini hiç okumadiniz mi?" diye sordu.
\par 26 "Baskâhin Aviyatar'in zamaninda Davut, Tanri'nin evine girdi, kâhinlerden baskasinin yemesi yasak olan adak ekmeklerini* yedi ve yanindakilere de verdi."
\par 27 Sonra onlara, "Insan Sabat Günü için degil, Sabat Günü insan için yaratildi" dedi.
\par 28 "Bu nedenle Insanoglu* Sabat Günü'nün de Rabbi'dir."

\chapter{3}

\par 1 Isa yine havraya girdi. Orada eli sakat bir adam vardi.
\par 2 Bazilari Isa'yi suçlamak amaciyla, Sabat Günü* hastayi iyilestirecek mi diye O'nu gözlüyorlardi.
\par 3 Isa, eli sakat adama, "Kalk, öne çik!" dedi.
\par 4 Sonra havradakilere, "Kutsal Yasa'ya göre Sabat Günü iyilik yapmak mi dogru, kötülük yapmak mi? Can kurtarmak mi dogru, can almak mi?" diye sordu. Onlardan ses çikmadi.
\par 5 Isa, çevresindekilere öfkeyle bakti. Yüreklerinin duygusuzlugu O'nu kederlendirmisti. Adama, "Elini uzat!" dedi. Adam elini uzatti, eli yine sapasaglam oluverdi.
\par 6 Bunun üzerine Ferisiler disari çiktilar, Isa'yi yok etmek için Hirodes yanlilariyla hemen görüsmeye basladilar.
\par 7 Isa, ögrencileriyle birlikte göl kiyisina çekildi. Celile'den büyük bir kalabalik O'nun ardindan geldi. Ayrica, bütün yaptiklarini duyan büyük kalabaliklar Yahudiye'den, Yerusalim'den, Idumeya'dan, Seria Irmagi'nin karsi yakasindan, Sur ve Sayda bölgelerinden kendisine akin etti.
\par 9 Isa, kalabaligin arasinda sikisip kalmamak için ögrencilerine bir kayik hazir bulundurmalarini söyledi.
\par 10 Birçoklarini iyilestirmis oldugundan, çesitli hastaliklara yakalananlar O'na dokunmak için üzerine üsüsüyordu.
\par 11 Kötü ruhlar O'nu görünce ayaklarina kapaniyor, "Sen Tanri'nin Oglu'sun!" diye bagiriyorlardi.
\par 12 Ama Isa, kim oldugunu açiklamamalari için onlari siki sikiya uyardi.
\par 13 Isa, daga çikarak istedigi kisileri yanina çagirdi. Onlar da yanina gittiler.
\par 14 Isa bunlardan on iki kisiyi yaninda bulundurmak, Tanri sözünü duyurmaya göndermek ve cinleri kovmaya yetkili kilmak üzere seçti. Seçtigi bu on iki kisi sunlardir: Petrus adini verdigi Simun, Beni-Reges, yani Gökgürültüsü Ogullari adini verdigi Zebedi'nin ogullari Yakup ve Yuhanna, Andreas, Filipus, Bartalmay, Matta, Tomas, Alfay oglu Yakup, Taday, Yurtsever* Simun ve Isa'ya ihanet eden Yahuda Iskariot.
\par 20 Isa bundan sonra eve gitti. Yine öyle büyük bir kalabalik toplandi ki, Isa'yla ögrencileri yemek bile yiyemediler.
\par 21 Yakinlari bunu duyunca, "Aklini kaçirmis" diyerek O'nu almaya geldiler.
\par 22 Yerusalim'den gelen din bilginleri ise, "Baalzevul* O'nun içine girmis" ve "Cinleri, cinlerin önderinin gücüyle kovuyor" diyorlardi.
\par 23 Bunun üzerine Isa din bilginlerini yanina çagirip onlara benzetmelerle seslendi. "Seytan, Seytan'i nasil kovabilir?" dedi.
\par 24 "Bir ülke kendi içinde bölünmüsse, ayakta kalamaz.
\par 25 Bir ev kendi içinde bölünmüsse, ayakta kalamaz.
\par 26 Seytan da kendine karsi gelip kendi içinde bölünmüsse, artik ayakta kalamaz; sonu gelmis demektir.
\par 27 Hiç kimse güçlü adamin evine girip malini çalamaz. Ancak onu bagladiktan sonra evini soyabilir.
\par 28 Size dogrusunu söyleyeyim, insanlarin isledigi her günah, ettigi her küfür bagislanacak, ama Kutsal Ruh'a küfreden asla bagislanmayacak. Bunu yapan, asla silinmeyecek bir günah islemis olur."
\par 30 Isa bu sözleri, "O'nda kötü ruh var" dedikleri için söyledi.
\par 31 Daha sonra Isa'nin annesiyle kardesleri geldi. Disarida durdular, haber gönderip O'nu çagirdilar.
\par 32 Isa'nin çevresinde oturan kalabaliktan bazilari, "Bak" dediler, "Annenle kardeslerin disarida, seni istiyorlar."
\par 33 Isa buna karsilik onlara, "Kimdir annem ve kardeslerim?" dedi.
\par 34 Sonra çevresinde oturanlara bakip söyle dedi: "Iste annem, iste kardeslerim!
\par 35 Tanri'nin istegini kim yerine getirirse, kardesim, kizkardesim ve annem odur."

\chapter{4}

\par 1 Isa göl kiyisinda halka yine ögretmeye basladi. Çevresinde çok büyük bir kalabalik toplandi. Bu yüzden Isa göldeki bir tekneye binip oturdu. Bütün kalabalik göl kiyisinda duruyordu.
\par 2 Isa onlara benzetmelerle birçok sey ögretiyordu. Ögretirken, "Sunu dinleyin" dedi. "Ekincinin biri tohum ekmeye çikti.
\par 4 Ektigi tohumlardan kimi yol kenarina düstü. Kuslar gelip bunlari yedi.
\par 5 Kimi, topragi az kayalik yerlere düstü. Toprak derin olmadigindan hemen filizlendi.
\par 6 Ne var ki, günes dogunca kavruldular, kök salamadiklari için kuruyup gittiler.
\par 7 Kimi, dikenler arasina düstü. Dikenler büyüdü, filizleri bogdu ve filizler ürün vermedi.
\par 8 Kimi ise iyi topraga düstü, büyüyüp çogaldi, ürün verdi. Bazisi otuz, bazisi altmis, bazisi da yüz kat ürün verdi."
\par 9 Sonra Isa sunu ekledi: "Isitecek kulagi olan isitsin!"
\par 10 Onikiler'le* öbür izleyicileri Isa'yla yalniz kalinca, kendisinden benzetmelerin anlamini sordular.
\par 11 O da onlara söyle dedi: "Tanri'nin Egemenligi'nin sirri sizlere açiklandi, ama disarida olanlara her sey benzetmelerle anlatilir.
\par 12 Öyle ki, 'Bakip bakip görmesinler, Duyup duyup anlamasinlar da, Dönüp bagislanmasinlar.'"
\par 13 Isa sonra onlara, "Siz bu benzetmeyi anlamiyor musunuz?" dedi. "Öyleyse bütün benzetmeleri nasil anlayacaksiniz?
\par 14 Ekincinin ektigi, Tanri sözüdür.
\par 15 Bazi insanlar sözün ekildigi yerde yol kenarina düsen tohumlara benzer. Bunlar sözü isitir isitmez, Seytan gelir, yüreklerine ekilen sözü alir götürür.
\par 16 Kayalik yerlere ekilenler ise, isittikleri sözü hemen sevinçle kabul eden, ama kök salamadiklari için ancak bir süre dayanan kisilerdir. Böyleleri Tanri sözünden ötürü sikinti ya da zulme ugrayinca hemen sendeleyip düserler.
\par 18 Yine bazilari dikenler arasinda ekilen tohumlara benzerler. Bunlar sözü isitirler, ama dünyasal kaygilar, zenginligin aldaticiligi ve daha baska hevesler araya girip sözü bogar ve ürün vermesini engeller.
\par 20 Iyi topraga ekilenler ise, sözü isiten, onu benimseyen, kimi otuz, kimi altmis, kimi de yüz kat ürün veren kisilerdir."
\par 21 Onlara, "Kandili, tahil ölçeginin ya da yatagin altina koymak için mi getirirler?" dedi. "Kandillige koymak için degil mi?
\par 22 Gizli olan ne varsa, açiga çikarilmak üzere gizlenmistir; sakli olan ne varsa, aydinliga çikmak üzere saklanmistir.
\par 23 Isitecek kulagi olan isitsin!"
\par 24 Isa söyle devam etti: "Isittiklerinize dikkat edin! Hangi ölçekle verirseniz, ayni ölçekle alacaksiniz. Hatta size daha fazlasi verilecek.
\par 25 Çünkü kimde varsa, ona daha çok verilecek. Ama kimde yoksa, elindeki de alinacak."
\par 26 Sonra Isa söyle dedi: "Tanri'nin Egemenligi, topraga tohum saçan adama benzer.
\par 27 Gece olur, uyur; gündüz olur, kalkar. Kendisi nasil oldugunu bilmez ama, tohum filizlenir, gelisir.
\par 28 Toprak kendiliginden ürün verir. Önce filizi, sonra basagi, sonunda da basagi dolduran taneleri verir.
\par 29 Ürün olgunlasinca, adam hemen oragi vurur. Çünkü biçim vakti gelmistir."
\par 30 Isa sonra söyle dedi: "Tanri'nin Egemenligi'ni neye benzetelim, nasil bir benzetmeyle anlatalim?
\par 31 Tanri'nin Egemenligi, hardal tanesine benzer. Hardal, yeryüzünde topraga ekilen tohumlarin en küçügü olmakla birlikte, ekildikten sonra gelisir, bütün bahçe bitkilerinin boyunu asar. Öylesine dal budak salar ki, kuslar gölgesinde barinabilir."
\par 33 Isa, Tanri sözünü, buna benzer birçok benzetmeyle halkin anlayabildigi ölçüde anlatirdi.
\par 34 Benzetme kullanmadan onlara hiçbir sey anlatmazdi. Ama kendi ögrencileriyle yalniz kaldiginda, onlara her seyi açiklardi.
\par 35 O gün aksam olunca ögrencilerine, "Karsi yakaya geçelim" dedi.
\par 36 Ögrenciler kalabaligi geride birakarak Isa'yi, içinde bulundugu tekneyle götürdüler. Yaninda baska tekneler de vardi.
\par 37 Bu sirada büyük bir firtina koptu. Dalgalar tekneye öyle bindirdi ki, tekne neredeyse suyla dolmustu.
\par 38 Isa, teknenin kiç tarafinda bir yastiga yaslanmis uyuyordu. Ögrenciler O'nu uyandirip, "Ögretmenimiz, ölecegiz! Hiç aldirmiyor musun?" dediler.
\par 39 Isa kalkip rüzgari azarladi, göle, "Sus, sakin ol!" dedi. Rüzgar dindi, ortalik sütliman oldu.
\par 40 Isa ögrencilerine, "Neden korkuyorsunuz? Hâlâ imaniniz yok mu?" dedi.
\par 41 Onlar ise büyük korku içinde birbirlerine, "Bu adam kim ki, rüzgar da göl de O'nun sözünü dinliyor?" dediler.

\chapter{5}

\par 1 Gölün karsi yakasina, Gerasalilar'in memleketine vardilar.
\par 2 Isa tekneden iner inmez, kötü ruha tutulmus bir adam mezarlik magaralardan çikip O'nu karsiladi.
\par 3 Mezarlarin içinde yasayan bu adami artik kimse zincirle bile bagli tutamiyordu.
\par 4 Birçok kez zincir ve kösteklerle baglandigi halde, zincirleri koparmis, köstekleri parçalamisti. Hiç kimse onunla basa çikamiyordu.
\par 5 Gece gündüz mezarlarda, daglarda bagirip duruyor, kendini taslarla yaraliyordu.
\par 6 Uzaktan Isa'yi görünce kosup geldi, O'nun önünde yere kapandi.
\par 7 Yüksek sesle haykirarak, "Ey Isa, yüce Tanri'nin Oglu, benden ne istiyorsun? Tanri hakki için sana yalvaririm, bana iskence etme!" dedi.
\par 8 Çünkü Isa, "Ey kötü ruh, adamin içinden çik!" demisti.
\par 9 Sonra Isa adama, "Adin ne?" diye sordu. "Adim Tümen*. Çünkü sayimiz çok" dedi.
\par 10 Ruhlari o bölgeden çikarmamasi için Isa'ya yalvarip yakardi.
\par 11 Orada, dagin yamacinda otlayan büyük bir domuz sürüsü vardi.
\par 12 Kötü ruhlar Isa'ya, "Bizi su domuzlara gönder, onlara girelim" diye yalvardilar.
\par 13 Isa'nin izin vermesi üzerine kötü ruhlar adamdan çikip domuzlarin içine girdiler. Yaklasik iki bin domuzdan olusan sürü, dik yamaçtan asagi kosusarak göle atlayip boguldu.
\par 14 Domuzlari güdenler kaçip kentte ve köylerde olayin haberini yaydilar. Halk olup biteni görmeye çikti.
\par 15 Isa'nin yanina geldiklerinde, önceleri bir tümen cine tutulan adami giyinmis, akli basina gelmis, oturmus görünce korktular.
\par 16 Olayi görenler, cinli adama olanlari ve domuzlarin basina gelenleri halka anlattilar.
\par 17 Bunun üzerine halk, bölgelerinden ayrilmasi için Isa'ya yalvarmaya basladi.
\par 18 Isa tekneye binerken, önceleri cinli olan adam O'na, "Seninle geleyim" diye yalvardi.
\par 19 Ama Isa adama izin vermedi. Ona, "Evine, yakinlarinin yanina dön" dedi. "Rab'bin senin için neler yaptigini, sana nasil merhamet ettigini onlara anlat."
\par 20 Adam da gitti, Isa'nin kendisi için neler yaptigini Dekapolis'te duyurmaya basladi. Anlattiklarina herkes sasip kaliyordu.
\par 21 Isa tekneyle karsi yakaya dönünce, çevresinde büyük bir kalabalik toplandi. Kendisi gölün kiyisinda duruyordu.
\par 22 Bu sirada havra yöneticilerinden Yair adinda biri geldi. Isa'yi görünce ayaklarina kapandi, "Küçük kizim can çekisiyor. Gelip ellerini onun üzerine koy da kurtulsun, yasasin!" diye yalvardi.
\par 24 Isa adamla birlikte gitti. Büyük bir kalabalik da ardindan gidiyor, O'nu sikistiriyordu.
\par 25 Orada, on iki yildir kanamasi olan bir kadin vardi.
\par 26 Birçok hekimin elinden çok çekmis, varini yogunu harcamis, ama iyilesecegine daha da kötülesmisti.
\par 27 Kadin, Isa hakkinda anlatilanlari duymustu. Bu nedenle, kalabalikta O'nun arkasindan gelip giysisine dokundu.
\par 28 Içinden, "Giysilerine bile dokunsam kurtulurum" diyordu.
\par 29 O anda kanamasi kesiliverdi. Kadin, bedeninin derinliginde acidan kurtuldugunu hissetti.
\par 30 Isa ise, kendisinden bir gücün akip gittigini hemen anladi. Kalabaligin ortasinda dönüp, "Giysilerime kim dokundu?" diye sordu.
\par 31 Ögrencileri O'na, "Seni sikistiran kalabaligi görüyorsun! Nasil oluyor da, 'Bana kim dokundu' diye soruyorsun?" dediler.
\par 32 Isa kendisine dokunani görmek için çevresine bakindi.
\par 33 Kadin da kendisindeki degisikligi biliyordu. Korkuyla titreyerek geldi, Isa'nin ayaklarina kapandi ve O'na bütün gerçegi anlatti.
\par 34 Isa ona, "Kizim" dedi, "Imanin seni kurtardi. Esenlikle git. Acilarin son bulsun."
\par 35 Isa daha konusurken, havra yöneticisinin evinden adamlar geldi. Yöneticiye, "Kizin öldü" dediler. "Ögretmeni neden hâlâ rahatsiz ediyorsun?"
\par 36 Isa bu sözlere aldirmadan havra yöneticisine, "Korkma, yalniz iman et!" dedi.
\par 37 Isa, Petrus, Yakup ve Yakup'un kardesi Yuhanna'dan baska hiç kimsenin kendisiyle birlikte gitmesine izin vermedi.
\par 38 Havra yöneticisinin evine vardiklarinda Isa, aci aci aglayip feryat eden gürültülü bir kalabalikla karsilasti.
\par 39 Içeri girerek onlara, "Niye gürültü edip agliyorsunuz?" dedi. "Çocuk ölmedi, uyuyor."
\par 40 Onlar ise kendisiyle alay ettiler. Ama Isa hepsini disari çikardiktan sonra çocugun annesini babasini ve kendisiyle birlikte olanlari alip çocugun bulundugu odaya girdi.
\par 41 Çocugun elini tutarak ona, "Talita kumi!" dedi. Bu söz, "Kizim, sana söylüyorum, kalk" demektir.
\par 42 On iki yasinda olan kiz hemen ayaga kalkti, yürümeye basladi. Oradakileri derin bir saskinlik aldi.
\par 43 Isa, "Bunu kimse bilmesin" diyerek onlari siki sikiya uyardi ve kiza yemek verilmesini buyurdu.

\chapter{6}

\par 1 Isa oradan ayrilarak kendi memleketine gitti. Ögrencileri de ardindan gittiler.
\par 2 Sabat Günü* olunca Isa havrada ögretmeye basladi. Söylediklerini isiten birçok kisi sasip kaldi. "Bu adam bunlari nereden ögrendi?" diye soruyorlardi. "Kendisine verilen bu bilgelik nedir? Nasil böyle mucizeler yapabiliyor?
\par 3 Meryem'in oglu, Yakup, Yose, Yahuda ve Simun'un kardesi olan marangoz degilmi bu? Kizkardesleri burada, aramizda yasamiyor mu?" Ve gücenip O'nu reddettiler.
\par 4 Isa da onlara, "Bir peygamber, kendi memleketinden, akraba çevresinden ve kendi evinden baska yerde hor görülmez" dedi.
\par 5 Orada birkaç hastayi, üzerlerine ellerini koyarak iyilestirmekten baska hiçbir mucize yapamadi.
\par 6 Halkin imansizligina sasiyordu. Isa çevredeki köyleri dolasip ögretiyordu.
\par 7 On iki ögrencisini yanina çagirdi ve onlari ikiser ikiser halk arasina göndermeye basladi. Onlara kötü ruhlar üzerinde yetki verdi.
\par 8 Yolculuk için yanlarina degnekten baska bir sey almamalarini söyledi. Ne ekmek, ne torba, ne de kusaklarinda para götüreceklerdi.
\par 9 Onlara çarik giymelerini söyledi. Ama, "Iki mintan giymeyin" dedi.
\par 10 "Bir yere gittiginiz zaman, oradan ayrilincaya dek hep ayni evde kalin" diye devam etti.
\par 11 "Insanlarin sizi kabul etmedikleri, sizi dinlemedikleri bir yerden ayrilirken, onlara uyari olsun diye ayaginizin altindaki tozu silkin!"
\par 12 Böylece ögrenciler yola çikip insanlari tövbeye çagirmaya basladilar.
\par 13 Birçok cin kovdular; birçok hastayi, üzerlerine yag sürerek iyilestirdiler.
\par 14 Kral Hirodes* de olup bitenleri duydu. Çünkü Isa'nin ünü her tarafa yayilmisti. Bazilari, "Bu adam, ölümden dirilen Vaftizci Yahya'dir. Olaganüstü güçlerin onda etkin olmasinin nedeni budur" diyordu.
\par 15 Baskalari, "O Ilyas'tir" diyor, yine baskalari, "Eski peygamberlerden biri gibi bir peygamberdir" diyordu.
\par 16 Hirodes bunlari duyunca, "Basini kestirdigim Yahya dirildi!" dedi.
\par 17 Hirodes'in kendisi, kardesi Filipus'un karisi Hirodiya'nin yüzünden adam gönderip Yahya'yi tutuklatmis, zindana attirip zincire vurdurmustu. Çünkü Hirodes bu kadinla evlenince Yahya ona, "Kardesinin karisiyla evlenmen Kutsal Yasa'ya aykiridir" demisti.
\par 19 Hirodiya bu yüzden Yahya'ya kin baglamisti; onu öldürtmek istiyor, ama basaramiyordu.
\par 20 Çünkü Yahya'nin dogru ve kutsal bir adam oldugunu bilen Hirodes ondan korkuyor ve onu koruyordu. Yahya'yi dinledigi zaman büyük bir saskinlik içinde kaliyor, yine de onu dinlemekten zevk aliyordu.
\par 21 Ne var ki, Hirodes'in kendi dogum gününde saray büyükleri, komutanlar ve Celile'nin ileri gelenleri için verdigi sölende beklenen firsat dogdu.
\par 22 Hirodiya'nin kizi içeri girip dans etti. Bu, Hirodes'le konuklarinin hosuna gitti. Kral genç kiza, "Dile benden, ne dilersen veririm" dedi.
\par 23 Ant içerek, "Benden ne dilersen, kralligimin yarisi da olsa, veririm" dedi.
\par 24 Kiz disari çikip annesine, "Ne isteyeyim?" diye sordu. "Vaftizci Yahya'nin basini iste" dedi annesi.
\par 25 Kiz hemen kosup kralin yanina girdi, "Vaftizci Yahya'nin basini bir tepsi üzerinde hemen bana vermeni istiyorum" diyerek dilegini açikladi.
\par 26 Kral buna çok üzüldüyse de, konuklarinin önünde içtigi anttan ötürü kizi reddetmek istemedi.
\par 27 Hemen bir cellat gönderip Yahya'nin basini getirmesini buyurdu. Cellat zindana giderek Yahya'nin basini kesti.
\par 28 Kesik basi bir tepsi üzerinde getirip genç kiza verdi, kiz da annesine götürdü.
\par 29 Yahya'nin ögrencileri bunu duyunca gelip cesedi aldilar ve mezara koydular.
\par 30 Elçiler, Isa'nin yanina dönerek yaptiklari ve ögrettikleri her seyi O'na anlattilar.
\par 31 Isa onlara, "Gelin, tek basimiza tenha bir yere gidelim de biraz dinlenin" dedi. Gelen giden öyle çoktu ki, yemek yemeye bile vakit bulamiyorlardi.
\par 32 Tekneye binip tek baslarina tenha bir yere dogru yol aldilar.
\par 33 Gittiklerini gören birçok kisi onlari tanidi. Halk civardaki bütün kentlerden yaya olarak yola dökülüp onlardan önce oraya vardi.
\par 34 Isa tekneden inince büyük bir kalabalikla karsilasti. Çobansiz koyunlara benzeyen bu insanlara acidi ve onlara birçok konuda ögretmeye basladi.
\par 35 Vakit ilerlemisti. Ögrencileri Isa'ya gelip, "Burasi issiz bir yer" dediler, "Vakit de ilerledi. Halki saliver de çevredeki çiftlik ve köylere gidip kendilerine yiyecek alsinlar."
\par 37 Isa ise, "Onlara siz yiyecek verin" diye karsilik verdi. Ögrenciler Isa'ya, "Gidip iki yüz dinarlik ekmek alip onlara yedirelim mi yani?" diye sordular.
\par 38 Isa onlara, "Kaç ekmeginiz var, gidin bakin" dedi. Ögrenip geldiler, "Bes ekmekle iki baligimiz var" dediler.
\par 39 Isa herkesi küme küme yesil çayira oturtmalarini buyurdu.
\par 40 Halk yüzer elliser kisilik bölükler halinde oturdu.
\par 41 Isa bes ekmekle iki baligi aldi, gözlerini göge kaldirarak sükretti; sonra ekmekleri böldü ve halka dagitmalari için ögrencilerine verdi. Iki baligi da hepsinin arasinda paylastirdi.
\par 42 Herkes yiyip doydu. Artakalan ekmek ve baliktan on iki sepet dolusu topladilar.
\par 44 Yemek yiyen erkeklerin sayisi bes bin kadardi.
\par 45 Bundan hemen sonra Isa ögrencilerine, tekneye binip kendisinden önce karsi yakada bulunan Beytsayda'ya geçmelerini buyurdu. Bu arada kendisi halki evlerine gönderecekti.
\par 46 Onlari ugurladiktan sonra, dua etmek için daga çikti.
\par 47 Aksam oldugunda, tekne gölün ortasina varmisti. Yalniz basina karada kalan Isa, ögrencilerinin kürek çekmekte çok zorlandiklarini gördü. Çünkü rüzgar onlara karsi esiyordu. Sabaha karsi Isa, gölün üstünde yürüyerek onlara yaklasti. Yanlarindan geçip gidecekti.
\par 49 Onlar ise, gölün üstünde yürüdügünü görünce O'nu hayalet sanarak bagristilar.
\par 50 Hepsi O'nu görmüs ve dehsete kapilmisti. Isa hemen onlara seslenerek, "Cesur olun, benim, korkmayin!" dedi.
\par 51 Tekneye binip onlara katilinca rüzgar dindi. Onlarsa büyük bir saskinlik içindeydi.
\par 52 Ekmekle ilgili mucizeyi bile anlamamislardi; zihinleri körelmisti.
\par 53 Isa'yla ögrencileri gölü astilar, Ginnesar'da karaya çikip tekneyi bagladilar.
\par 54 Onlar tekneden inince, halk Isa'yi hemen tanidi.
\par 55 Bazilari kosarak bütün yöreyi dolasti. Isa'nin bulundugu yeri ögrenenler, hastalari silteleriyle oraya götürmeye basladilar.
\par 56 Köy olsun, kent ya da çiftlik olsun, Isa'nin gittigi her yerde, hastalari meydanlara yatiriyor, sadece giysisinin etegine dokunmalarina izin vermesi için yalvariyorlardi. Dokunanlarin hepsi de iyilesti.

\chapter{7}

\par 1 Yerusalim'den gelen Ferisiler ve bazi din bilginleri, Isa'nin çevresinde toplandilar.
\par 2 O'nun ögrencilerinden bazilarinin murdar*, yani yikanmamis ellerle yemek yediklerini gördüler.
\par 3 Ferisiler, hatta bütün Yahudiler, atalarinin töresi uyarinca ellerini iyice yikamadan yemek yemezler.
\par 4 Çarsidan dönünce de, yikanmadan yemek yemezler. Ayrica kâse, testi ve bakir kaplarin yikanmasiyla ilgili baska birçok töreye de uyarlar.
\par 5 Ferisiler ve din bilginleri Isa'ya, "Ögrencilerin neden atalarimizin töresine uymuyorlar, niçin murdar ellerle yemek yiyorlar?" diye sordular.
\par 6 Isa onlari söyle yanitladi: "Yesaya'nin siz ikiyüzlülerle ilgili peygamberlik sözü ne kadar yerindedir! Yazmis oldugu gibi, 'Bu halk, dudaklariyla beni sayar, Ama yürekleri benden uzak.
\par 7 Bana bosuna taparlar. Çünkü ögrettikleri, sadece insan buyruklaridir.'
\par 8 Siz Tanri buyrugunu bir yana birakmis, insan töresine uyuyorsunuz."
\par 9 Isa onlara ayrica sunu söyledi: "Kendi törenizi sürdürmek için Tanri buyrugunu bir kenara itmeyi ne de güzel beceriyorsunuz!
\par 10 Musa, 'Annene babana saygi göstereceksin' ve, 'Annesine ya da babasina söven kesinlikle öldürülecektir' diye buyurmustu.
\par 11 Ama siz, 'Eger bir adam annesine ya da babasina, benden alacagin bütün yardim kurbandir, yani Tanri'ya adanmistir derse, artik annesi ya da babasi için bir sey yapmasina izin yok' diyorsunuz.
\par 13 Böylece kusaktan kusaga aktardiginiz törelerle Tanri'nin sözünü geçersiz kiliyorsunuz. Buna benzer daha birçok sey yapiyorsunuz."
\par 14 Isa, halki yine yanina çagirip onlara, "Hepiniz beni dinleyin ve sunu belleyin" dedi.
\par 15 "Insanin disinda olup içine giren hiçbir sey onu kirletemez. Insani kirleten, insanin içinden çikandir."
\par 17 Isa kalabaligi birakip eve girince, ögrencileri O'na bu benzetmenin anlamini sordular.
\par 18 O da onlara, "Demek siz de anlamiyorsunuz, öyle mi?" dedi. "Disaridan insanin içine giren hiçbir seyin onu kirletemeyecegini bilmiyor musunuz?
\par 19 Distan giren, insanin yüregine degil, midesine gider, oradan da helaya atilir." Isa bu sözlerle, bütün yiyeceklerin temiz oldugunu bildirmis oluyordu.
\par 20 Isa söyle devam etti: "Insani kirleten, insanin içinden çikandir.
\par 21 Çünkü kötü düsünceler, fuhus, hirsizlik, cinayet, zina, açgözlülük, kötülük, hile, sefahat, kiskançlik, iftira, kibir ve akilsizlik içten, insanin yüreginden kaynaklanir.
\par 23 Bu kötülüklerin hepsi içten kaynaklanir ve insani kirletir."
\par 24 Isa oradan ayrilarak Sur bölgesine gitti. Burada bir eve girdi. Kimsenin bunu bilmesini istemiyordu, ama gizlenemedi.
\par 25 Küçük kizi kötü ruha tutulmus bir kadin, Isa'yla ilgili haberi duyar duymaz geldi, ayaklarina kapandi.
\par 26 Yahudi olmayan bu kadin Suriye-Fenike irkindandi. Kizindan cini kovmasi için Isa'ya rica etti.
\par 27 Isa ona, "Birak, önce çocuklar doysunlar" dedi. "Çocuklarin ekmegini alip köpeklere atmak dogru degildir."
\par 28 Kadin buna karsilik, "Haklisin, Rab" dedi. "Ama köpekler de sofranin altinda çocuklarin ekmek kirintilarini yer."
\par 29 Isa ona, "Bu sözden ötürü cin kizindan çikti, gidebilirsin" dedi.
\par 30 Kadin evine gittiginde çocugunu cinden kurtulmus, yatakta yatar buldu.
\par 31 Sur bölgesinden ayrilan Isa, Sayda yoluyla Dekapolis bölgesinin ortasindan geçerek tekrar Celile Gölü'ne geldi.
\par 32 Ona sagir ve dili tutuk bir adam getirdiler, elini üzerine koymasi için yalvardilar.
\par 33 Isa adami kalabaliktan ayirip bir yana çekti. Parmaklarini adamin kulaklarina soktu, tükürüp onun diline dokundu.
\par 34 Sonra göge bakarak içini çekti ve adama, "Effata", yani "Açil!" dedi.
\par 35 Adamin kulaklari hemen açildi, dili çözüldü ve düzgün bir sekilde konusmaya basladi.
\par 36 Isa orada bulunanlari, bunu kimseye söylememeleri için uyardi. Ama onlari ne kadar uyardiysa, onlar da haberi o kadar yaydilar.
\par 37 Halk büyük bir hayret içinde kalmisti. "Yaptigi her sey iyi. Sagirlarin kulaklarini açiyor, dilsizleri konusturuyor!" diyorlardi.

\chapter{8}

\par 1 O günlerde yine büyük bir kalabalik toplanmisti. Yiyecek bir seyleri olmadigi için Isa ögrencilerini yanina çagirip, "Halka aciyorum" dedi. "Üç gündür yanimdalar, yiyecek hiçbir seyleri yok.
\par 3 Onlari aç aç evlerine gönderirsem, yolda bayilirlar. Hem bazilari uzak yoldan geliyor."
\par 4 Ögrencileri buna karsilik, "Böyle issiz bir yerde bu kadar kisiyi doyuracak ekmegi insan nereden bulabilir?" dediler.
\par 5 Isa, "Kaç ekmeginiz var?" diye sordu. "Yedi tane" dediler.
\par 6 Bunun üzerine Isa, halka yere oturmalarini buyurdu. Sonra yedi ekmegi aldi, sükredip bunlari böldü, dagitmalari için ögrencilerine verdi. Onlar da halka dagittilar.
\par 7 Birkaç küçük baliklari da vardi. Isa sükredip bunlari da dagitmalarini söyledi.
\par 8 Herkes yiyip doydu. Artakalan parçalardan yedi küfe dolusu topladilar.
\par 9 Orada yaklasik dört bin kisi vardi. Isa onlari evlerine gönderdikten sonra ögrencileriyle birlikte hemen tekneye binip Dalmanuta bölgesine geçti.
\par 11 Ferisiler* gelip Isa'yla tartismaya basladilar. O'nu denemek amaciyla gökten bir belirti göstermesini istediler.
\par 12 Isa içten bir ah çekerek, "Bu kusak neden bir belirti istiyor?" dedi. "Size dogrusunu söyleyeyim, bu kusaga hiçbir belirti gösterilmeyecek."
\par 13 Sonra onlari orada birakip yine tekneye bindi ve karsi yakaya yöneldi.
\par 14 Ögrenciler ekmek almayi unutmuslardi. Teknede, yanlarinda yalniz bir ekmek vardi.
\par 15 Isa onlara su uyarida bulundu: "Dikkatli olun, Ferisiler'in mayasindan ve Hirodes'in* mayasindan sakinin!"
\par 16 Onlar ise kendi aralarinda, "Ekmegimiz olmadigi için böyle diyor" seklinde tartistilar.
\par 17 Bunun farkinda olan Isa, "Ekmeginiz yok diye niçin tartisiyorsunuz?" dedi. "Hâlâ akil erdiremiyor, anlamiyor musunuz? Zihniniz köreldi mi?
\par 18 Gözleriniz oldugu halde görmüyor musunuz? Kulaklariniz oldugu halde isitmiyor musunuz? Hatirlamiyor musunuz, bes ekmegi bes bin kisiye bölüstürdügümde kaç sepet dolusu yemek fazlasi topladiniz?" "On iki" dediler.
\par 20 "Yedi ekmegi dört bin kisiye bölüstürdügümde kaç küfe dolusu yemek fazlasi topladiniz?" "Yedi" dediler.
\par 21 Isa onlara, "Hâlâ anlamiyor musunuz?" dedi.
\par 22 Isa ile ögrencileri Beytsayda'ya geldiler. Orada bazi kisiler Isa'ya kör bir adam getirip ona dokunmasi için yalvardilar.
\par 23 Isa körün elinden tutarak onu köyün disina çikardi. Gözlerine tükürüp ellerini üzerine koydu ve, "Bir sey görüyor musun?" diye sordu.
\par 24 Adam basini kaldirip, "Insanlar görüyorum" dedi, "Agaçlara benziyorlar, ama yürüyorlar."
\par 25 Sonra Isa ellerini yeniden adamin gözleri üzerine koydu. Adam gözlerini açti, bakti; iyilesmis ve her seyi açik seçik görmeye baslamisti.
\par 26 Isa, "Köye bile girme!" diyerek onu evine gönderdi.
\par 27 Isa, ögrencileriyle birlikte Filipus Sezariyesi'ne bagli köylere gitti. Yolda ögrencilerine, "Halk benim kim oldugumu söylüyor?" diye sordu.
\par 28 Ögrencileri O'na su karsiligi verdiler: "Vaftizci Yahya diyorlar. Ama kimi Ilyas, kimi de peygamberlerden biri oldugunu söylüyor."
\par 29 O da onlara, "Siz ne dersiniz, sizce ben kimim?" diye sordu. Petrus, "Sen Mesih'sin*" yanitini verdi.
\par 30 Bunun üzerine Isa bu konuda kimseye bir sey söylememeleri için onlari uyardi.
\par 31 Isa, Insanoglu'nun* çok aci çekmesi, ileri gelenler, baskâhinler ve din bilginlerince* reddedilmesi, öldürülmesi ve üç gün sonra dirilmesi gerektigini onlara anlatmaya basladi.
\par 32 Bunlari açikça söylüyordu. Bunun üzerine Petrus O'nu bir kenara çekip azarlamaya basladi.
\par 33 Isa dönüp öteki ögrencilerine bakti; Petrus'u azarlayarak, "Çekil önümden, Seytan!" dedi. "Düsüncelerin Tanri'ya degil, insana özgüdür."
\par 34 Ögrencileriyle birlikte halki da yanina çagirip söyle konustu: "Ardimdan gelmek isteyen kendini inkâr etsin, çarmihini yüklenip beni izlesin.
\par 35 Canini kurtarmak isteyen onu yitirecek, canini benim ve Müjde'nin ugruna yitiren ise onu kurtaracaktir.
\par 36 Insan bütün dünyayi kazanip da canindan olursa, bunun kendisine ne yarari olur?
\par 37 Insan kendi canina karsilik ne verebilir?
\par 38 Bu vefasiz ve günahkâr kusagin ortasinda, kim benden ve benim sözlerimden utanirsa, Insanoglu da, Babasi'nin görkemi içinde kutsal meleklerle birlikte geldiginde o kisiden utanacaktir."

\chapter{9}

\par 1 Isa, "Size dogrusunu söyleyeyim" diye devam etti, "Burada bulunanlar arasinda, Tanri Egemenligi'nin güçlü biçimde gerçeklestigini görmeden ölümü tatmayacak olanlar var."
\par 2 Alti gün sonra Isa, yanina yalniz Petrus, Yakup ve Yuhanna'yi alarak yüksek bir daga çikti. Onlarin gözü önünde Isa'nin görünümü degisti.
\par 3 Giysileri göz kamastirici bir beyazliga büründü; yeryüzünde hiçbir çamasircinin erisemeyecegi bir beyazlikti bu.
\par 4 O anda Musa'yla Ilyas ögrencilere göründü. Isa'yla konusuyorlardi.
\par 5 Petrus Isa'ya, "Rabbî*, burada bulunmamiz ne iyi oldu! Üç çardak kuralim: Biri sana, biri Musa'ya, biri de Ilyas'a" dedi.
\par 6 Ne söyleyecegini bilmiyordu. Çünkü çok korkmuslardi.
\par 7 Bu sirada bir bulut gelip onlara gölge saldi. Buluttan gelen bir ses, "Sevgili Oglum budur, O'nu dinleyin!" dedi.
\par 8 Ögrenciler birden çevrelerine baktilar, ama bu kez yanlarinda Isa'dan baska kimseyi göremediler.
\par 9 Dagdan inerlerken Isa, Insanoglu* ölümden dirilmeden orada gördüklerini hiç kimseye söylememeleri için onlari uyardi.
\par 10 Bu uyariya uymakla birlikte kendi aralarinda, "Ölümden dirilmek ne demek?"diye tartisip durdular.
\par 11 Isa'ya, "Din bilginleri* neden önce Ilyas'in gelmesi gerektigini söylüyorlar?" diye sordular.
\par 12 O da onlara söyle dedi: "Gerçekten de önce Ilyas gelir ve her seyi yeniden düzene koyar. Ama nasil oluyor da Insanoglu'nun çok aci çekecegi ve hiçe sayilacagi yazilmistir?
\par 13 Size sunu söyleyeyim, Ilyas geldi bile, onun hakkinda yazilmis oldugu gibi, ona yapmadiklarini birakmadilar."
\par 14 Öteki ögrencilerin yanina döndüklerinde, onlarin çevresinde büyük bir kalabaligin toplandigini, birtakim din bilginlerinin onlarla tartistigini gördüler.
\par 15 Kalabalik Isa'yi görünce büyük bir saskinliga kapildi ve kosup O'nu selamladi.
\par 16 Isa ögrencilerine, "Onlarla ne tartisiyorsunuz?" diye sordu.
\par 17 Halktan biri O'na, "Ögretmenim" diye karsilik verdi, "Dilsiz bir ruha tutulan oglumu sana getirdim.
\par 18 Ruh onu nerede yakalarsa yere çarpiyor. Çocuk agzindan köpükler saçiyor, dislerini gicirdatiyor ve kaskati kesiliyor. Ruhu kovmalari için ögrencilerine basvurdum, ama basaramadilar."
\par 19 Isa onlara, "Ey imansiz kusak!" dedi. "Sizinle daha ne kadar kalacagim? Size daha ne kadar katlanacagim? Çocugu bana getirin!"
\par 20 Çocugu kendisine getirdiler. Ruh, Isa'yi görür görmez çocugu siddetle sarsti; çocuk yere düstü, agzindan köpükler saçarak yuvarlanmaya basladi.
\par 21 Isa çocugun babasina, "Bu hal çocugun basina geleli ne kadar oldu?" diye sordu. "Küçüklügünden beri böyle" dedi babasi.
\par 22 "Üstelik ruh onu öldürmek için sik sik atese, suya atti. Elinden bir sey gelirse, bize yardim et, halimize aci!"
\par 23 Isa ona, "Elimden gelirse mi? Iman eden biri için her sey mümkün!" dedi.
\par 24 Çocugun babasi hemen, "Iman ediyorum, imansizligimi yenmeme yardim et!"diye feryat etti.
\par 25 Isa, halkin kosusup geldigini görünce kötü ruhu azarlayarak, "Sana buyuruyorum, dilsiz ve sagir ruh, çocugun içinden çik ve ona bir daha girme!" dedi.
\par 26 Bunun üzerine ruh bir çiglik atti ve çocugu siddetle sarsarak çikti. Çocuk ölü gibi hareketsiz kaldi, öyle ki oradakilerin birçogu, "Öldü!" diyordu.
\par 27 Ama Isa elinden tutup kaldirinca, çocuk ayaga kalkti.
\par 28 Isa eve girdikten sonra ögrencileri özel olarak O'na, "Biz kötü ruhu neden kovamadik?" diye sordular.
\par 29 Isa onlara, "Bu tür ruhlar ancak duayla kovulabilir" yanitini verdi.
\par 30 Oradan ayrilmis, Celile bölgesinden geçiyorlardi. Isa hiç kimsenin bunu bilmesini istemiyordu.
\par 31 Ögrencilerine ögretirken söyle diyordu: "Insanoglu*, insanlarin eline teslim edilecek ve öldürülecek, ama öldürüldükten üç gün sonra dirilecek."
\par 32 Onlar bu sözleri anlamiyor, Isa'ya soru sormaktan da korkuyorlardi.
\par 33 Kefarnahum'a vardilar. Eve girdikten sonra Isa onlara, "Yolda neyi tartisiyordunuz?" diye sordu.
\par 34 Hiç birinden ses çikmadi. Çünkü yolda aralarinda kimin en büyük oldugunu tartismislardi.
\par 35 Isa oturup Onikiler'i* yanina çagirdi. Onlara söyle dedi: "Birinci olmak isteyen en sonuncu olsun, herkesin hizmetkâri olsun."
\par 36 Küçük bir çocugu alip orta yere dikti, sonra onu kucagina alarak onlara söyle dedi: "Böyle bir çocugu benim adim ugruna kabul eden, beni kabul etmis olur. Beni kabul eden de beni degil, beni göndereni kabul etmis olur."
\par 38 Yuhanna O'na, "Ögretmenim" dedi, "Senin adinla cin kovan birini gördük, ama bizi izleyenlerden olmadigi için ona engel olmaya çalistik."
\par 39 "Ona engel olmayin!" dedi Isa. "Çünkü benim adimla mucize yapip da ardindan beni kötüleyecek kimse yoktur.
\par 40 Bize karsi olmayan, bizden yanadir.
\par 41 Size dogrusunu söyleyeyim, Mesih'e ait oldugunuz için sizlere bir bardak su veren ödülsüz kalmayacaktir."
\par 42 "Kim bana iman eden bu küçüklerden birini günaha düsürürse, boynuna kocamanbir degirmen tasi geçirilip denize atilmasi kendisi için daha iyi olur.
\par 43 Eger elin günah islemene neden olursa, onu kes. Tek elle yasama kavusman, iki elle sönmez atese, cehenneme gitmenden iyidir.
\par 45 Eger ayagin günah islemene neden olursa, onu kes. Tek ayakla yasama kavusman, iki ayakla cehenneme atilmandan iyidir.
\par 47 Eger gözün günah islemene neden olursa, onu çikar at. Tanri'nin Egemenligi'ne tek gözle girmen, iki gözle cehenneme atilmandan iyidir.
\par 48 'Oradakileri kemiren kurt ölmez, Yakan ates sönmez.'
\par 49 Çünkü herkes atesle tuzlanacaktir.
\par 50 Tuz yararlidir. Ama tuz tuzlulugunu yitirirse, bir daha ona nasil tat verebilirsiniz? Içinizde tuz olsun ve birbirinizle baris içinde yasayin!"

\chapter{10}

\par 1 Isa oradan ayrilip Yahudiye'nin* Seria Irmagi'nin karsi yakasindaki topraklarina geçti. Çevresinde yine kalabaliklar toplanmisti; her zamanki gibi onlara ögretiyordu.
\par 2 Yanina gelen bazi Ferisiler O'nu denemek amaciyla, "Bir erkegin, karisini bosamasi Kutsal Yasa'ya uygun mudur?" diye sordular.
\par 3 Isa karsilik olarak, "Musa size ne buyurdu?" dedi.
\par 4 Onlar, "Musa, erkegin bir bosanma belgesi yazarak karisini bosamasina izin vermistir" dediler.
\par 5 Isa onlara, "Inatçi oldugunuz için Musa bu buyrugu yazdi" dedi.
\par 6 "Tanri, yaratilisin baslangicindan 'Insanlari erkek ve disi olarak yaratti.'
\par 7 'Bu nedenle adam annesini babasini birakip karisina baglanacak, ikisi tek beden olacak.' Söyle ki, onlar artik iki degil, tek bedendir.
\par 9 O halde Tanri'nin birlestirdigini insan ayirmasin."
\par 10 Ögrencileri evde O'na yine bu konuyla ilgili bazi sorular sordular.
\par 11 Isa onlara, "Karisini bosayip baskasiyla evlenen, karisina karsi zina etmis olur" dedi.
\par 12 "Kocasini bosayip baskasiyla evlenen kadin da zina etmis olur."
\par 13 Bu arada bazilari küçük çocuklari Isa'nin yanina getiriyor, onlara dokunmasini istiyorlardi. Ne var ki, ögrenciler onlari azarladilar.
\par 14 Isa bunu görünce kizdi. Ögrencilerine, "Birakin, çocuklar bana gelsin" dedi. "Onlara engel olmayin! Çünkü Tanri'nin Egemenligi böylelerinindir.
\par 15 Size dogrusunu söyleyeyim, Tanri'nin Egemenligi'ni bir çocuk gibi kabul etmeyen, bu egemenlige asla giremez."
\par 16 Çocuklari kucagina aldi, ellerini üzerlerine koyup onlari kutsadi.
\par 17 Isa yola çikarken, biri kosarak yanina geldi. Önünde diz çöküp O'na, "Iyi ögretmenim, sonsuz yasama kavusmak için ne yapmaliyim?" diye sordu.
\par 18 Isa, "Bana neden iyi diyorsun?" dedi. "Iyi olan yalniz biri var, O da Tanri'dir.
\par 19 O'nun buyruklarini biliyorsun: 'Adam öldürmeyeceksin, zina etmeyeceksin, çalmayacaksin, yalan yere taniklik etmeyeceksin, kimsenin hakkini yemeyeceksin, annene babana saygi göstereceksin.'"
\par 20 Adam, "Ögretmenim, bunlarin hepsini gençligimden beri yerine getiriyorum dedi.
\par 21 Ona sevgiyle bakan Isa, "Bir eksigin var" dedi. "Git neyin varsa sat, parasini yoksullara ver; böylece gökte hazinen olur. Sonra gel, beni izle."
\par 22 Bu sözler üzerine adamin yüzü asildi, üzüntü içinde oradan uzaklasti. Çünkü çok mali vardi.
\par 23 Isa çevresine göz gezdirdikten sonra ögrencilerine, "Varlikli kisilerin Tanri Egemenligi'ne girmesi ne güç olacak!" dedi.
\par 24 Ögrenciler O'nun sözlerine sastilar. Ama Isa onlara yine, "Çocuklar" dedi, "Tanri'nin Egemenligi'ne girmek ne güçtür!
\par 25 Devenin igne deliginden geçmesi, zenginin Tanri Egemenligi'ne girmesinden daha kolaydir."
\par 26 Ögrenciler büsbütün sasirmislardi. Birbirlerine, "Öyleyse kim kurtulabilir?" diyorlardi.
\par 27 Isa onlara bakarak, "Insanlar için bu imkânsiz, ama Tanri için degil. Tanri için her sey mümkündür" dedi.
\par 28 Petrus O'na, "Bak, biz her seyi birakip senin ardindan geldik" demeye basladi.
\par 29 "Size dogrusunu söyleyeyim" dedi Isa, "Benim ve Müjde'nin ugruna evini,kardeslerini, anne ya da babasini, çocuklarini ya da topraklarini birakip da simdi, bu çagda çekecegi zulümlerle birlikte yüz kat daha fazla eve, kardese, anneye, çocuga, topraga ve gelecek çagda sonsuz yasama kavusmayacak hiç kimse yoktur.
\par 31 Ne var ki, birincilerin birçogu sonuncu, sonuncularin birçogu da birinci olacak."
\par 32 Yola çikmis Yerusalim'e gidiyorlardi. Isa önlerinde yürüyordu. Ögrencileri saskinlik içindeydi, ardindan gelenler ise korkuyorlardi. Isa Onikiler'i* yine bir yana çekip kendi basina gelecekleri anlatmaya basladi: "Simdi Yerusalim'e gidiyoruz" dedi. "Insanoglu*, baskâhinlerin ve din bilginlerinin eline teslim edilecek. Onlar da O'nu ölüm cezasina çarptiracak ve öteki uluslara teslim edecekler.
\par 34 O'nunla alay edecek, üzerine tükürecek ve O'nu kamçilayip öldürecekler. Ne var ki O, üç gün sonra dirilecek."
\par 35 Zebedi'nin ogullari Yakup ile Yuhanna Isa'ya yaklasip, "Ögretmenimiz, bir dilegimiz var, bunu yapmani istiyoruz" dediler.
\par 36 Isa onlara, "Sizin için ne yapmami istiyorsunuz?" diye sordu.
\par 37 "Sen yüceligine kavusunca birimize saginda, ötekimize de solunda oturma ayricaligini ver" dediler.
\par 38 "Siz ne dilediginizi bilmiyorsunuz" dedi Isa. "Benim içecegim kâseden siz içebilir misiniz? Benim vaftiz olacagim gibi siz de vaftiz olabilir misiniz?"
\par 39 "Evet, olabiliriz" dediler. Isa onlara, "Benim içecegim kâseden siz de içeceksiniz, benim vaftiz olacagim gibi siz de vaftiz olacaksiniz" dedi. "Ama sagimda ya da solumda oturmaniza izin vermek benim elimde degil. Bu yerler belirli kisiler için hazirlanmistir."
\par 41 Bunu isiten on ögrenci Yakup'la Yuhanna'ya kizmaya basladilar.
\par 42 Isa onlari yanina çagirip söyle dedi: "Bilirsiniz ki, uluslarin önderleri sayilanlar, onlara egemen kesilir, ileri gelenleri de onlara agirliklarini hissettirirler.
\par 43 Sizin aranizda böyle olmayacak. Aranizda büyük olmak isteyen, ötekilerin hizmetkâri olsun.
\par 44 Aranizda birinci olmak isteyen, hepinizin kulu olsun.
\par 45 Çünkü Insanoglu bile hizmet edilmeye degil, hizmet etmeye ve canini birçoklari için fidye olarak vermeye geldi."
\par 46 Sonra Eriha'ya geldiler. Isa, ögrencileri ve büyük bir kalabalikla birlikte Eriha'dan ayrilirken, Timay oglu Bartimay adinda kör bir dilenci yol kenarinda oturuyordu
\par 47 Nasirali Isa'nin orada oldugunu duyunca, "Ey Davut Oglu Isa, halime aci!" diye bagirmaya basladi.
\par 48 Birçok kimse onu azarlayarak susturmak istediyse de o, "Ey Davut Oglu, halime aci!" diyerek daha çok bagirdi.
\par 49 Isa durdu, "Çagirin onu" dedi. Kör adama seslenerek, "Ne mutlu sana! Kalk, seni çagiriyor!" dediler.
\par 50 Adam abasini üstünden atarak ayaga firladi ve Isa'nin yanina geldi.
\par 51 Isa, "Senin için ne yapmami istiyorsun?" diye sordu. Kör adam, "Rabbuni, gözlerim görsün" dedi.
\par 52 Isa, "Gidebilirsin, imanin seni kurtardi" dedi. Adam o anda yeniden görmeyebasladi ve yol boyunca Isa'nin ardindan gitti.

\chapter{11}

\par 1 Yerusalim'e yaklasip Zeytin Dagi'nin yamacindaki Beytfaci ile Beytanya'ya geldiklerinde Isa iki ögrencisini önden gönderdi. Onlara, "Karsinizdaki köye gidin" dedi, "Köye girer girmez, üzerine daha hiç kimsenin binmedigi, bagli duran bir sipa bulacaksiniz. Onu çözüp bana getirin.
\par 3 Biri size, 'Bunu niye yapiyorsunuz?' derse, 'Rab'bin ona ihtiyaci var, hemen geri gönderecek' dersiniz."
\par 4 Gittiler ve yol üzerinde, bir evin sokak kapisinin yaninda bagli bulduklari sipayi çözdüler.
\par 5 Orada duranlardan bazilari, "Sipayi ne diye çözüyorsunuz?" dediler.
\par 6 Ögrenciler Isa'nin kendilerine söylediklerini tekrarlayinca, adamlar onlari rahat birakti.
\par 7 Sipayi Isa'ya getirip üzerine kendi giysilerini yaydilar. Isa sipaya bindi.
\par 8 Birçoklari giysilerini, bazilari da çevredeki agaçlardan kestikleri dallari yola serdiler.
\par 9 Önden gidenler ve arkadan gelenler söyle bagiriyorlardi: "Hozana*! Rab'bin adiyla gelene övgüler olsun!
\par 10 Atamiz Davut'un yaklasan egemenligi kutlu olsun! En yücelerde hozana!"
\par 11 Isa Yerusalim'e varinca tapinaga gitti, her tarafi gözden geçirdi. Sonra vakit ilerlemis oldugundan Onikiler'le* birlikte Beytanya'ya döndü.
\par 12 Ertesi gün Beytanya'dan çiktiklarinda Isa acikmisti.
\par 13 Uzakta, yapraklanmis bir incir agaci görünce belki incir bulurum diye yaklasti. Agacin yanina vardiginda yapraktan baska bir sey bulamadi. Çünkü incir mevsimi degildi.
\par 14 Isa agaca, "Artik sonsuza dek senden kimse meyve yiyemesin!" dedi. Ögrencileri de bunu duydular.
\par 15 Oradan Yerusalim'e geldiler. Isa tapinagin avlusuna girerek oradaki alici ve saticilari disari kovdu. Para bozanlarin* masalarini, güvercin satanlarin sehpalarini devirdi.
\par 16 Yük tasiyan hiç kimsenin tapinagin avlusundan geçmesine izin vermedi.
\par 17 Halka ögretirken sunlari söyledi: "'Evime, bütün uluslarin dua evi denecek' diye yazilmamis mi? Ama siz onu haydut inine çevirdiniz."
\par 18 Baskâhinler ve din bilginleri bunu duyunca Isa'yi yok etmek için bir yol aramaya basladilar. O'ndan korkuyorlardi. Çünkü bütün halk O'nun ögretisine hayrandi.
\par 19 Aksam olunca Isa'yla ögrencileri kentten ayrildi.
\par 20 Sabah erkenden incir agacinin yanindan geçerlerken, agacin kökten kurumus oldugunu gördüler.
\par 21 Olayi hatirlayan Petrus, "Rabbî*, bak! Lanetledigin incir agaci kurumus!" dedi.
\par 22 Isa onlara söyle karsilik verdi: "Tanri'ya iman edin.
\par 23 Size dogrusunu söyleyeyim, kim su daga, 'Kalk, denize atil!' der ve yüreginde kusku duymadan dediginin olacagina inanirsa, dilegi yerine gelecektir.
\par 24 Bunun için size diyorum ki, duayla dilediginiz her seyi daha simdiden almis oldugunuza inanin, dileginiz yerine gelecektir.
\par 25 Kalkip dua ettiginiz zaman, birine karsi bir sikâyetiniz varsa onu bagislayin ki, göklerdeki Babaniz da sizin suçlarinizi bagislasin."
\par 27 Yine Yerusalim'e geldiler. Isa tapinakta gezinirken baskâhinler, din bilginleri* ve ileri gelenler O'nun yanina gelip, "Bunlari hangi yetkiyle yapiyorsun, bunlari yapma yetkisini sana kim verdi?" diye sordular.
\par 29 Isa da onlara, "Size bir soru soracagim" dedi. "Bana yanit verin, ben de size bunlari hangi yetkiyle yaptigimi söylerim.
\par 30 Yahya'nin vaftiz etme yetkisi Tanri'dan miydi, insanlardan mi? Yanit verin bana."
\par 31 Bunu aralarinda söyle tartismaya basladilar: "'Tanri'dan' dersek, 'Öyleyse ona niçin inanmadiniz?' diyecek.
\par 32 Yok eger 'Insanlardan' dersek..." Halkin tepkisinden korkuyorlardi. Çünkü herkes Yahya'yi gerçekten peygamber sayiyordu.
\par 33 Isa'ya, "Bilmiyoruz" diye yanit verdiler. Isa da onlara, "Ben de size bunlari hangi yetkiyle yaptigimi söylemeyecegim" dedi.

\chapter{12}

\par 1 Isa onlara benzetmelerle konusmaya basladi. "Adamin biri bag dikti, çevresini çitle çevirdi, üzüm sikmak için bir çukur kazdi, bir de bekçi kulesi yapti. Sonra bagi bagcilara kiralayip yolculuga çikti.
\par 2 Mevsimi gelince bagin ürününden payina düseni almak üzere bagcilara bir köle yolladi.
\par 3 Bagcilar köleyi yakalayip dövdü ve eli bos gönderdi.
\par 4 Bag sahibi bu kez onlara baska bir köle yolladi. Onu da basindan yaralayip asagiladilar.
\par 5 Birini daha yolladi, onu öldürdüler. Daha birçok köle yolladi. Kimini dövüp kimini öldürdüler.
\par 6 "Bag sahibinin yaninda tek kisi kaldi, o da sevgili ogluydu. 'Oglumu sayarlar' diyerek bagcilara en son onu yolladi.
\par 7 "Ama bagcilar birbirlerine, 'Mirasçi budur, gelin onu öldürelim, miras bizim olur' dediler.
\par 8 Böylece onu yakaladilar, öldürüp bagdan disari attilar.
\par 9 "Bu durumda bagin sahibi ne yapacak? Gelip bagcilari yok edecek, bagi da baskalarina verecek.
\par 10 Su Kutsal Yazi'yi okumadiniz mi?'Yapicilarin reddettigi tas, Iste kösenin bas tasi oldu. Rab'bin isidir bu, Gözümüzde harika bir is!'"
\par 12 Isa'nin bu benzetmede kendilerinden söz ettigini anlayan Yahudi önderler O'nu tutuklamak istediler; ama halkin tepkisinden korktuklari için O'nu birakip gittiler.
\par 13 Daha sonra Isa'yi söyleyecegi sözlerle tuzaga düsürmek amaciyla Ferisiler'den ve Hirodes yanlilarindan bazilarini O'na gönderdiler.
\par 14 Bunlar gelip Isa'ya, "Ögretmenimiz" dediler, "Senin dürüst biri oldugunu, kimseyi kayirmadan, insanlar arasinda ayrim yapmadan Tanri yolunu dürüstçe ögrettigini biliyoruz. Sezar'a* vergi vermek Kutsal Yasa'ya uygun mu, degil mi? Verelim mi, vermeyelim mi?"
\par 15 Onlarin ikiyüzlülügünü bilen Isa söyle dedi: "Beni neden deniyorsunuz? Bana bir dinar getirin bakayim."
\par 16 Parayi getirdiler. Isa, "Bu resim, bu yazi kimin?" diye sordu. "Sezar'in" dediler.
\par 17 Isa da, "Sezar'in hakkini Sezar'a, Tanri'nin hakkini Tanri'ya verin" dedi. Isa'nin sözlerine sasakaldilar.
\par 18 Ölümden sonra dirilis olmadigini söyleyen Sadukiler Isa'ya gelip sunu sordular: "Ögretmenimiz, Musa yazilarinda bize söyle buyurmustur: 'Eger bir adam ölür, geride bir dul birakir, ama çocuk birakmazsa, kardesi onun karisini alip soyunu sürdürsün.'
\par 20 Yedi kardes vardi. Birincisi evlendi ve çocuk birakmadan öldü.
\par 21 Ikincisi ayni kadini aldi, o da çocuk sahibi olmadan öldü. Üçüncüsüne de öyle oldu.
\par 22 Yedisi de çocuksuz öldü. Hepsinden sonra kadin da öldü.
\par 23 Dirilis günü, ölümden dirildiklerinde kadin bunlardan hangisinin karisi olacak? Çünkü yedisi de onunla evlendi."
\par 24 Isa onlara söyle karsilik verdi: "Ne Kutsal Yazilar'i ne de Tanri'nin gücünü biliyorsunuz. Yanilmanizin nedeni de bu degil mi?
\par 25 Insanlar ölümden dirilince ne evlenir ne evlendirilir, göklerdeki melekler gibidirler.
\par 26 Ölülerin dirilmesi konusuna gelince, Musa'nin Kitabi'nda, alevlenen çaliyla ilgili bölümde Tanri'nin Musa'ya söylediklerini okumadiniz mi? 'Ben Ibrahim'in Tanrisi, Ishak'in Tanrisi ve Yakup'un Tanrisi'yim' diyor.
\par 27 Tanri ölülerin degil, dirilerin Tanrisi'dir. Siz büyük bir yanilgi içindesiniz."
\par 28 Onlarin tartismalarini dinleyen ve Isa'nin onlara güzel yanit verdigini gören bir din bilgini* yaklasip O'na, "Buyruklarin en önemlisi hangisidir?" diye sordu.
\par 29 Isa söyle karsilik verdi: "En önemlisi sudur: 'Dinle, ey Israil! Tanrimiz Rab tek Rab'dir.
\par 30 Tanrin Rab'bi bütün yüreginle, bütün caninla, bütün aklinla ve bütün gücünle seveceksin.'
\par 31 Ikincisi de sudur: 'Komsunu kendin gibi seveceksin.' Bunlardan daha büyük buyruk yoktur."
\par 32 Din bilgini Isa'ya, "Iyi söyledin, ögretmenim" dedi. "'Tanri tektir ve O'ndan baskasi yoktur' demekle dogruyu söyledin.
\par 33 Insanin Tanri'yi bütün yüregiyle, bütün anlayisiyla ve bütün gücüyle sevmesi, komsusunu da kendi gibi sevmesi, bütün yakmalik sunulardan* ve kurbanlardan daha önemlidir."
\par 34 Isa onun akillica yanit verdigini görünce, "Sen Tanri'nin Egemenligi'nden uzak degilsin" dedi. Bundan sonra kimse O'na soru sormaya cesaret edemedi.
\par 35 Isa tapinakta ögretirken sunu sordu: "Nasil oluyor da din bilginleri, 'Mesih*, Davut'un Oglu'dur' diyorlar?
\par 36 Davut'un kendisi, Kutsal Ruh'tan esinlenerek söyle demisti: Rab Rabbim'e dedi ki, Ben düsmanlarini Ayaklarinin altina serinceye dek Sagimda otur.'
\par 37 Davut'un kendisi O'ndan Rab diye söz ettigine göre, O nasil Davut'un Oglu olur?" Oradaki büyük kalabalik O'nu zevkle dinliyordu.
\par 38 Isa ögretirken söyle dedi: "Uzun kaftanlar içinde dolasmaktan, meydanlarda selamlanmaktan, havralarda en seçkin yerlere ve sölenlerde basköselere kurulmaktan hoslanan din bilginlerinden sakinin.
\par 40 Dul kadinlarin malini mülkünü sömüren, gösteris için uzun uzun dua eden bu kisilerin cezasi daha agir olacaktir."
\par 41 Isa tapinakta bagis toplanan yerin karsisinda oturmus, kutulara para atan halki seyrediyordu. Birçok zengin kisi kutuya bol para atti.
\par 42 Yoksul bir dul kadin da geldi, birkaç kurus degerinde iki bakir para atti.
\par 43 Isa ögrencilerini yanina çagirarak, "Size dogrusunu söyleyeyim" dedi, "Bu yoksul dul kadin kutuya herkesten daha çok para atti.
\par 44 Çünkü ötekilerin hepsi, zenginliklerinden artani attilar. Bu kadin ise yoksulluguna karsin, varini yogunu, geçinmek için elinde ne varsa, tümünü verdi."

\chapter{13}

\par 1 Isa tapinaktan çikarken ögrencilerinden biri O'na, "Ögretmenim" dedi, "Su güzel taslara, su görkemli yapilara bak!"
\par 2 Isa ona, "Bu büyük yapilari görüyor musun? Burada tas üstünde tas kalmayacak, hepsi yikilacak!" dedi.
\par 3 Isa, Zeytin Dagi'nda, tapinagin karsisinda otururken Petrus, Yakup, Yuhanna ve Andreas özel olarak kendisine sunu sordular: "Söyle bize, bu dediklerin ne zaman olacak, bütün bunlarin gerçeklesmek üzere oldugunu gösteren belirti ne olacak?"
\par 5 Isa onlara anlatmaya basladi: "Sakin kimse sizi saptirmasin" dedi.
\par 6 "Birçoklari, 'Ben O'yum' diyerek benim adimla gelip birçok kisiyi saptiracaklar.
\par 7 Savas gürültüleri, savas haberleri duyunca korkmayin. Bunlarin olmasi gerek,ama bu daha son demek degildir.
\par 8 Ulus ulusa, devlet devlete savas açacak; yer yer depremler, kitliklar olacak. Bunlar, dogum sancilarinin baslangicidir.
\par 9 "Ama siz kendinize dikkat edin! Insanlar sizi mahkemelere verecek, havralarda dövecekler. Benden ötürü valilerin, krallarin önüne çikarilacak, böylece onlara taniklik edeceksiniz.
\par 10 Ne var ki, önce Müjde'nin bütün uluslara duyurulmasi gerekir.
\par 11 Sizi tutuklayip mahkemeye verdiklerinde, 'Ne söyleyecegiz?' diye önceden kaygilanmayin. O anda size ne esinlenirse onu söyleyin. Çünkü konusan siz degil, Kutsal Ruh olacak.
\par 12 Kardes kardesi, baba çocugunu ölüme teslim edecek. Çocuklar anne babalarina baskaldirip onlari öldürtecek.
\par 13 Benim adimdan ötürü herkes sizden nefret edecek. Ama sonuna kadar dayanan kurtulacaktir.
\par 14 "Yikici igrenç seyin*, bulunmamasi gereken yerde dikildigini gördügünüz zaman -okuyan anlasin- Yahudiye'de bulunanlar daglara kaçsin.
\par 15 Damda olan, evinden bir sey almak için asagi inmesin, içeri girmesin.
\par 16 Tarlada olan, abasini almak için geri dönmesin.
\par 17 O günlerde gebe olan, çocuk emziren kadinlarin vay haline!
\par 18 Dua edin ki, kaçisiniz kisa rastlamasin.
\par 19 Çünkü o günlerde öyle bir sikinti olacak ki, Tanri'nin var ettigi yaratilisin baslangicindan bu yana böylesi olmamis, bundan sonra da olmayacaktir.
\par 20 Rab o günleri kisaltmamis olsaydi, hiç kimse kurtulamazdi. Ama Rab, seçilmis olanlar, kendi seçtigi kisiler ugruna o günleri kisaltmistir.
\par 21 Eger o zaman biri size, 'Iste Mesih* burada', ya da, 'Iste surada' derse, inanmayin.
\par 22 Çünkü sahte mesihler, sahte peygamberler türeyecek; bunlar, belirtiler ve harikalar yapacaklar. Öyle ki, ellerinden gelse seçilmis olanlari saptiracaklar.
\par 23 Ama siz dikkatli olun. Iste size her seyi önceden söylüyorum."
\par 24 "Ama o günlerde, o sikintidan sonra, 'Günes kararacak, Ay isik vermez olacak, Yildizlar gökten düsecek, Göksel güçler sarsilacak.'
\par 26 "O zaman Insanoglu'nun* bulutlar içinde büyük güç ve görkemle geldigini görecekler.
\par 27 Insanoglu o zaman meleklerini gönderecek, seçtiklerini yeryüzünün bir ucundan gögün öbür ucuna dek, dünyanin dört bucagindan toplayacak.
\par 28 "Incir agacindan ders alin. Dallari filizlenip yapraklari sürünce, yaz mevsiminin yakin oldugunu anlarsiniz.
\par 29 Ayni sekilde, bu olaylarin gerçeklestigini gördügünüzde bilin ki Insanoglu yakindir, kapidadir.
\par 30 Size dogrusunu söyleyeyim, bütün bunlar olmadan bu kusak ortadan kalkmayacak.
\par 31 Yer ve gök ortadan kalkacak, ama benim sözlerim asla ortadan kalkmayacaktir."
\par 32 "O günü ve o saati, ne gökteki melekler, ne de Ogul bilir; Baba'dan baska kimse bilmez.
\par 33 Dikkat edin, uyanik kalin, dua edin. Çünkü o anin ne zaman gelecegini bilemezsiniz.
\par 34 Bu, yolculuga çikan bir adamin durumuna benzer. Evinden ayrilirken kölelerine yetki ve görev verir, kapidaki nöbetçiye de uyanik kalmasini buyurur.
\par 35 Siz de uyanik kalin. Çünkü ev sahibi ne zaman gelecek, aksam mi, gece yarisi mi, horoz öttügünde mi, sabaha dogru mu, bilemezsiniz.
\par 36 Ansizin gelip sizi uykuda bulmasin!
\par 37 Size söylediklerimi herkese söylüyorum; uyanik kalin!"

\chapter{14}

\par 1 Fisih ve Mayasiz Ekmek Bayrami'na* iki gün kalmisti. Baskâhinlerle din bilginleri Isa'yi hileyle tutuklayip öldürmenin bir yolunu ariyorlardi.
\par 2 "Bayramda olmasin, yoksa halk arasinda kargasalik çikar" diyorlardi.
\par 3 Isa Beytanya'da cüzamli* Simun'un evinde sofrada otururken yanina bir kadin geldi. Kadin kaymaktasindan bir kap içinde çok degerli, saf hintsümbülü yagi getirmisti. Kabi kirarak yagi O'nun basina döktü.
\par 4 Bazilari buna kizdilar; birbirlerine, "Bu yag niçin böyle bos yere harcandi? Üç yüz dinardan fazlaya satilabilir, parasi yoksullara verilebilirdi" diyerek kadini azarlamaya basladilar.
\par 6 "Kadini rahat birakin" dedi Isa. "Neden üzüyorsunuz onu? Benim için güzel bir sey yapti.
\par 7 Yoksullar her zaman aranizdadir, dilediginiz anda onlara yardim edebilirsiniz; ama ben her zaman aranizda olmayacagim.
\par 8 Kadin elinden geleni yapti, beni gömülmeye hazirlamak üzere daha simdiden bedenimi yagladi.
\par 9 Size dogrusunu söyleyeyim, Müjde dünyanin neresinde duyurulursa, bu kadinin yaptigi da onun anilmasi için anlatilacak."
\par 10 Bu arada Onikiler'den* biri olan Yahuda Iskariot, Isa'yi ele vermek amaciyla baskâhinlerin yanina gitti.
\par 11 Onlar bunu isitince sevindiler, Yahuda'ya para vermeyi vaat ettiler. O da Isa'yi ele vermek için firsat kollamaya basladi.
\par 12 Fisih* kurbaninin kesildigi Mayasiz Ekmek Bayrami'nin* ilk günü ögrencileri Isa'ya, "Fisih yemegini yemen için nereye gidip hazirlik yapmamizi istersin?" diye sordular.
\par 13 O da ögrencilerinden ikisini su sözlerle önden gönderdi: "Kente gidin, orada su testisi tasiyan bir adam çikacak karsiniza. Onu izleyin.
\par 14 Adamin gidecegi evin sahibine söyle deyin: 'Ögretmen, ögrencilerimle birlikte Fisih yemegini yiyecegim konuk odasi nerede? diye soruyor.'
\par 15 Ev sahibi size üst katta dösenmis, hazir büyük bir oda gösterecek. Orada bizim için hazirlik yapin."
\par 16 Ögrenciler yola çikip kente gittiler. Her seyi, Isa'nin kendilerine söyledigi gibi buldular ve Fisih yemegi için hazirlik yaptilar.
\par 17 Aksam olunca Isa Onikiler'le* birlikte geldi.
\par 18 Sofraya oturmus yemek yerlerken Isa, "Size dogrusunu söyleyeyim" dedi, "Sizden biri, benimle yemek yiyen biri bana ihanet edecek."
\par 19 Onlar da kederlenerek birer birer kendisine, "Beni demek istemedin ya?" diye sormaya basladilar.
\par 20 Isa onlara, "Onikiler'den biridir, ekmegini benimle birlikte sahana batirandir" dedi.
\par 21 "Evet, Insanoglu* kendisi için yazilmis oldugu gibi gidiyor, ama Insanoglu'na ihanet edenin vay haline! O adam hiç dogmamis olsaydi, kendisi için daha iyi olurdu."
\par 22 Isa yemek sirasinda eline ekmek aldi, sükredip ekmegi böldü ve, "Alin, bu benim bedenimdir" diyerek ögrencilerine verdi.
\par 23 Sonra bir kâse alip sükretti ve bunu ögrencilerine verdi. Hepsi bundan içti.
\par 24 "Bu benim kanim" dedi Isa, "Birçoklari ugruna akitilan antlasma kanidir.
\par 25 Size dogrusunu söyleyeyim, Tanri'nin Egemenligi'nde tazesini içecegim o güne dek, asmanin ürününden bir daha içmeyecegim."
\par 26 Ilahi söyledikten sonra disari çikip Zeytin Dagi'na dogru gittiler.
\par 27 Bu arada Isa ögrencilerine, "Hepiniz sendeleyip düseceksiniz" dedi. "Çünkü söyle yazilmistir: 'Çobani vuracagim, Koyunlar darmadagin olacak.'
\par 28 Ama ben dirildikten sonra sizden önce Celile'ye gidecegim."
\par 29 Petrus O'na, "Herkes sendeleyip düsse bile ben düsmem" dedi.
\par 30 "Sana dogrusunu söyleyeyim" dedi Isa, "Bugün, bu gece, horoz iki kez ötmeden sen beni üç kez inkâr edeceksin."
\par 31 Ama Petrus üsteleyerek, "Seninle birlikte ölmem gerekse bile seni asla inkâr etmem" dedi. Ögrencilerin hepsi de ayni seyi söyledi.
\par 32 Sonra Getsemani denilen yere geldiler. Isa ögrencilerine, "Ben dua ederken siz burada oturun" dedi.
\par 33 Petrus'u, Yakup'u ve Yuhanna'yi yanina aldi. Hüzünlenmeye ve agir bir sikinti duymaya baslamisti.
\par 34 Onlara, "Ölesiye kederliyim" dedi. "Burada kalin, uyanik durun."
\par 35 Biraz ilerledi, yüzüstü yere kapanip dua etmeye basladi. "Mümkünse o saati yasamayayim" dedi.
\par 36 "Abba, Baba, senin için her sey mümkün, bu kâseyi* benden uzaklastir. Ama benim degil, senin istedigin olsun."
\par 37 Ögrencilerinin yanina döndügünde onlari uyumus buldu. Petrus'a, "Simun" dedi, "Uyuyor musun? Bir saat uyanik kalamadin mi?
\par 38 Uyanik durup dua edin ki, ayartilmayasiniz. Ruh isteklidir, ama beden güçsüzdür."
\par 39 Yine uzaklasti, ayni sözleri tekrarlayarak dua etti.
\par 40 Geri geldiginde ögrencilerini yine uyumus buldu. Onlarin göz kapaklarina agirlik çökmüstü. Isa'ya ne diyeceklerini bilemiyorlardi.
\par 41 Isa üçüncü kez yanlarina döndü, "Hâlâ uyuyor, dinleniyor musunuz?" dedi. "Yeter! Saat geldi. Iste Insanoglu* günahkârlarin eline veriliyor.
\par 42 Kalkin, gidelim. Iste bana ihanet eden geldi!"
\par 43 Tam o anda, Isa daha konusurken, Onikiler'den* biri olan Yahuda çikageldi. Yaninda baskâhinler, din bilginleri ve ileri gelenler tarafindan gönderilmis kiliçli sopali bir kalabalik vardi.
\par 44 Isa'ya ihanet eden Yahuda, "Kimi öpersem, Isa O'dur. O'nu tutuklayin, güvenlik altina alip götürün" diye onlarla sözlesmisti.
\par 45 Gelir gelmez Isa'ya yaklasti, "Rabbî*" diyerek O'nu öptü.
\par 46 Onlar da Isa'yi yakalayip tutukladilar.
\par 47 Isa'nin yaninda bulunanlardan biri kilicini çekti, baskâhinin kölesine vurup kulagini uçurdu.
\par 48 Isa onlara, "Niçin bir haydutmusum gibi beni kiliç ve sopalarla yakalamaya geldiniz?" dedi.
\par 49 "Her gün tapinakta, yanibasinizda ögretiyordum, beni tutuklamadiniz. Ama bu, Kutsal Yazilar yerine gelsin diye oldu."
\par 50 O zaman ögrencilerinin hepsi O'nu birakip kaçti.
\par 51 Isa'nin ardindan sadece keten beze sarinmis bir genç gidiyordu. Bu genç de yakalandi.
\par 52 Ama keten bezden siyrilip çiplak olarak kaçti.
\par 53 Isa'yi görevli baskâhine götürdüler. Bütün baskâhinler, ileri gelenler ve din bilginleri* de orada toplandi.
\par 54 Petrus, Isa'yi baskâhinin avlusunun içine kadar uzaktan izledi. Avluda nöbetçilerle birlikte atesin basinda oturup isinmaya basladi.
\par 55 Baskâhinler ve Yüksek Kurul'un* öteki üyeleri, Isa'yi ölüm cezasina çarptirmak için kendisine karsi tanik ariyor, ama bulamiyorlardi.
\par 56 Birçok kisi O'na karsi yalan yere taniklik ettiyse de, tanikliklari birbirini tutmadi.
\par 57 Bazilari kalkip O'na karsi yalan yere söyle taniklik ettiler: "Biz O'nun, 'Elle yapilmis bu tapinagi yikacagim ve üç günde, elle yapilmamis baska bir tapinak kuracagim' dedigini isittik."
\par 59 Ama bu noktada bile tanikliklari birbirini tutmadi.
\par 60 Sonra baskâhin toplulugun ortasinda ayaga kalkarak Isa'ya, "Hiç yanit vermeyecek misin? Nedir bunlarin sana karsi ettigi bu tanikliklar?" diye sordu.
\par 61 Ne var ki, Isa susmaya devam etti, hiç yanit vermedi. Baskâhin O'na yeniden, "Yüce Olan'in Oglu Mesih* sen misin?" diye sordu.
\par 62 Isa, "Benim" dedi. "Ve sizler, Insanoglu'nun* Kudretli Olan'in saginda oturdugunu ve gögün bulutlariyla geldigini göreceksiniz."
\par 63 Baskâhin giysilerini yirtarak, "Artik taniklara ne ihtiyacimiz var?" dedi. "Küfürü isittiniz. Buna ne diyorsunuz?" Hepsi Isa'nin ölüm cezasini hak ettigine karar verdiler.
\par 65 Bazilari O'nun üzerine tükürmeye, gözlerini baglayarak O'nu yumruklamaya basladilar. "Haydi, peygamberligini göster!" diyorlardi. Nöbetçiler de O'nu aralarina alip tokatladilar.
\par 66 Petrus asagida, avludayken, baskâhinin hizmetçi kizlarindan biri geldi. Isinmakta olan Petrus'u görünce onu dikkatle süzüp, "Sen de Nasirali Isa'yla birlikteydin" dedi.
\par 68 Petrus ise bunu inkâr ederek, "Senin neden söz ettigini bilmiyorum, anlamiyorum" dedi ve disariya, dis kapinin önüne çikti. Bu arada horoz öttü.
\par 69 Hizmetçi kiz Petrus'u görünce çevrede duranlara yine, "Bu adam onlardan biri" demeye basladi.
\par 70 Petrus tekrar inkâr etti. Çevrede duranlar az sonra Petrus'a yine, "Gerçekten onlardansin; sen de Celileli'sin" dediler.
\par 71 Petrus kendine lanet okuyup ant içerek, "Sözünü ettiginiz o adami tanimiyorum" dedi.
\par 72 Tam o anda horoz ikinci kez öttü. Petrus, Isa'nin kendisine, "Horoz iki kez ötmeden beni üç kez inkâr edeceksin" dedigini hatirladi ve hüngür hüngür aglamaya basladi.

\chapter{15}

\par 1 Sabah olunca baskâhinler, ileri gelenler, din bilginleri ve Yüksek Kurul'un* öteki üyeleri bir danisma toplantisi yaptiktan sonra Isa'yi bagladilar, götürüp Pilatus'a teslim ettiler.
\par 2 Pilatus O'na, "Sen Yahudiler'in Krali misin?" diye sordu. Isa, "Söyledigin gibidir" yanitini verdi.
\par 3 Baskâhinler O'na karsi birçok suçlamada bulundular.
\par 4 Pilatus O'na yeniden, "Hiç yanit vermeyecek misin?" diye sordu. "Bak, seni ne çok seyle suçluyorlar!"
\par 5 Ama Isa artik yanit vermiyordu. Pilatus buna sasti.
\par 6 Pilatus, her Fisih Bayrami'nda* halkin istedigi bir tutukluyu saliverirdi.
\par 7 Ayaklanma sirasinda adam öldüren isyancilarla birlikte Barabba adinda bir tutuklu da vardi.
\par 8 Halk, Pilatus'a gelip her zamanki gibi kendileri için birini salivermesini istedi.
\par 9 Pilatus onlara, "Sizin için Yahudiler'in Krali'ni salivermemi ister misiniz?" dedi.
\par 10 Baskâhinlerin Isa'yi kiskançliktan ötürü kendisine teslim ettiklerini biliyordu.
\par 11 Ne var ki baskâhinler, Isa'nin degil, Barabba'nin saliverilmesini istemeleri için halki kiskirttilar.
\par 12 Pilatus onlara tekrar seslenerek, "Öyleyse Yahudiler'in Krali dediginiz adami ne yapayim?" diye sordu.
\par 13 "O'nu çarmiha ger!" diye bagirdilar yine.
\par 14 Pilatus onlara, "O ne kötülük yapti ki?" dedi. Onlar ise daha yüksek sesle, "O'nu çarmiha ger!" diye bagristilar.
\par 15 Halki memnun etmek isteyen Pilatus, onlar için Barabba'yi saliverdi. Isa'yi ise kamçilattiktan sonra çarmiha gerilmek üzere askerlere teslim etti.
\par 16 Askerler Isa'yi, Pretorium denilen vali konagina götürüp bütün taburu topladilar.
\par 17 O'na mor bir giysi giydirdiler, dikenlerden bir taç örüp basina geçirdiler.
\par 18 "Selam, ey Yahudiler'in Krali!" diyerek O'nu selamlamaya basladilar.
\par 19 Basina bir kamisla vuruyor, üzerine tükürüyor, diz çöküp önünde yere kapaniyorlardi.
\par 20 O'nunla böyle alay ettikten sonra mor giysiyi üzerinden çikarip kendi giysilerini giydirdiler ve çarmiha germek üzere O'nu disari götürdüler.
\par 21 Kirdan gelmekte olan Simun adinda Kireneli bir adam oradan geçiyordu. Iskender ve Rufus'un babasi olan bu adama Isa'nin çarmihini zorla tasittilar.
\par 22 Isa'yi Golgota, yani Kafatasi denilen yere götürdüler.
\par 23 O'na mürle* karisik sarap vermek istediler, ama içmedi.
\par 24 Sonra O'nu çarmiha gerdiler. Kim ne alacak diye kura çekerek giysilerini aralarinda paylastilar.
\par 25 Isa'yi çarmiha gerdiklerinde saat dokuzdu.
\par 26 Üzerindeki suç yaftasinda, YAHUDILER'IN KRALI diye yaziliydi.
\par 27 Isa'yla birlikte, biri saginda öbürü solunda olmak üzere iki haydudu da çarmiha gerdiler.
\par 29 Oradan geçenler baslarini sallayip Isa'ya sövüyor, "Hani sen tapinagi yikip üç günde yeniden kuracaktin? Çarmihtan in de kurtar kendini!" diyorlardi.
\par 31 Ayni sekilde baskâhinler ve din bilginleri de O'nunla alay ederek aralarinda, "Baskalarini kurtardi, kendini kurtaramiyor" diye konusuyorlardi.
\par 32 "Israil'in Krali Mesih simdi çarmihtan insin de görüp iman edelim." Isa'yla birlikte çarmiha gerilenler de O'na hakaret ettiler.
\par 33 Ögleyin on ikiden üçe kadar bütün ülkenin üzerine karanlik çöktü.
\par 34 Saat üçte Isa yüksek sesle, "Elohi, Elohi, lema sevaktani" yani, "Tanrim, Tanrim, beni neden terk ettin?" diye bagirdi.
\par 35 Orada duranlardan bazilari bunu isitince, "Bakin, Ilyas'i çagiriyor" dediler.
\par 36 Aralarindan biri kosup bir süngeri eksi saraba batirdi, bir kamisin ucuna takarak Isa'ya içirdi. "Dur bakalim, Ilyas gelip O'nu indirecek mi?" dedi.
\par 37 Ama Isa yüksek sesle bagirarak son nefesini verdi.
\par 38 O anda tapinaktaki perde* yukaridan asagiya yirtilarak ikiye bölündü.
\par 39 Isa'nin karsisinda duran yüzbasi, O'nun bu sekilde son nefesini verdigini görünce, "Bu adam gerçekten Tanri'nin Oglu'ydu" dedi.
\par 40 Olup bitenleri uzaktan izleyen bazi kadinlar da vardi. Aralarinda Mecdelli Meryem, küçük Yakup ile Yose'nin annesi Meryem ve Salome bulunuyordu.
\par 41 Isa daha Celile'deyken bu kadinlar O'nun ardindan gitmis, O'na hizmet etmislerdi. O'nunla birlikte Yerusalim'e gelmis olan daha birçok kadin da olup bitenleri izliyordu.
\par 42 O gün Hazirlik Günü, yani Sabat Günü'nden* önceki gündü. Artik aksam oluyordu. Bu nedenle, Yüksek Kurul'un* saygin bir üyesi olup Tanri'nin Egemenligi'ni umutla bekleyen Aramatyali Yusuf geldi, cesaretini toplayarak Pilatus'un huzuruna çikti, Isa'nin cesedini istedi.
\par 44 Pilatus, Isa'nin bu kadar çabuk ölmüs olmasina sasti. Yüzbasiyi çagirip, "Öleli çok oldu mu?" diye sordu.
\par 45 Yüzbasidan durumu ögrenince Yusuf'a, cesedi almasi için izin verdi.
\par 46 Yusuf keten bez satin aldi, cesedi çarmihtan indirip beze sardi, kayaya oyulmus bir mezara yatirarak mezarin girisine bir tas yuvarladi.
\par 47 Mecdelli Meryem ile Yose'nin annesi Meryem, Isa'nin nereye konuldugunu gördüler.

\chapter{16}

\par 1 Sabat Günü* geçince, Mecdelli Meryem, Yakup'un annesi Meryem ve Salome gidip Isa'nin cesedine sürmek üzere baharat satin aldilar.
\par 2 Haftanin ilk günü* sabah çok erkenden, günesin dogusuyla birlikte mezara gittiler.
\par 3 Aralarinda, "Mezarin girisindeki tasi bizim için kim yana yuvarlayacak?" diye konusuyorlardi.
\par 4 Baslarini kaldirip bakinca, o kocaman tasin yana yuvarlanmis oldugunu gördüler.
\par 5 Mezara girip sag tarafta, beyaz kaftan giyinmis genç bir adamin oturdugunu görünce çok sasirdilar.
\par 6 Adam onlara, "Sasirmayin!" dedi. "Çarmiha gerilen Nasirali Isa'yi ariyorsunuz. O dirildi, burada yok. Iste O'nu yatirdiklari yer.
\par 7 Simdi ögrencilerine ve Petrus'a gidip söyle deyin: 'Isa sizden önce Celile'ye gidiyor. Size bildirdigi gibi, kendisini orada göreceksiniz.'"
\par 8 Kadinlar mezardan çikip kaçtilar. Onlari bir titreme, bir saskinlik almisti. Korkularindan kimseye bir sey söylemediler.
\par 9 Isa, haftanin ilk günü* sabah erkenden dirildigi zaman önce Mecdelli Meryem'e göründü. Ondan yedi cin kovmustu.
\par 10 Meryem gitti, Isa'yla bulunmus olan, simdiyse yas tutup gözyasi döken ögrencilerine haberi verdi.
\par 11 Ne var ki onlar, Isa'nin yasadigini, Meryem'e göründügünü duyunca inanmadilar.
\par 12 Bundan sonra Isa kirlara dogru yürümekte olan ögrencilerinden ikisine degisik bir biçimde göründü.
\par 13 Bunlar geri dönüp öbürlerine haber verdiler, ama öbürleri bunlara da inanmadilar.
\par 14 Isa daha sonra, sofrada otururlarken Onbirler'e* göründü. Onlari imansizliklarindan ve yüreklerinin duygusuzlugundan ötürü azarladi. Çünkü kendisini diri görenlere inanmamislardi.
\par 15 Isa onlara söyle buyurdu: "Dünyanin her yanina gidin, Müjde'yi bütün yaratilisa duyurun.
\par 16 Iman edip vaftiz olan kurtulacak, iman etmeyen ise hüküm giyecek.
\par 17 Iman edenlerle birlikte görülecek belirtiler sunlardir: Benim adimla cinleri kovacaklar, yeni dillerle konusacaklar, yilanlari elleriyle tutacaklar. Öldürücü bir zehir içseler bile, zarar görmeyecekler. Ellerini hastalarin üzerine koyacaklar ve hastalar iyilesecek."
\par 19 Rab Isa, onlara bu sözleri söyledikten sonra göge alindi ve Tanri'nin saginda oturdu.
\par 20 Ögrencileri de gidip Tanri sözünü her yere yaydilar. Rab onlarla birlikte çalisiyor, görülen belirtilerle sözünü dogruluyordu.


\end{document}