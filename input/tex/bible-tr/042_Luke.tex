\begin{document}

\title{Luka}


\chapter{1}

\par 1 Sayin Teofilos, Birçok kisi aramizda olup bitenlerin tarihçesini yazmaya giristi. Nitekim baslangiçtan beri bu olaylarin görgü tanigi ve Tanri sözünün hizmetkâri olanlar bunlari bize ilettiler. Ben de bütün bu olaylari ta basindan özenle arastirmis biri olarak bunlari sana sirasiyla yazmayi uygun gördüm.
\par 4 Öyle ki, sana verilen bilgilerin dogrulugunu bilesin.
\par 5 Yahudiye Krali Hirodes* zamaninda, Aviya bölügünden Zekeriya adinda bir kâhin* vardi. Harun soyundan gelen karisinin adi ise Elizabet'ti.
\par 6 Her ikisi de Tanri'nin gözünde dogru kisilerdi, Rab'bin bütün buyruk ve kurallarina eksiksizce uyarlardi.
\par 7 Elizabet kisir oldugu için çocuklari olmuyordu. Ikisinin de yasi ilerlemisti.
\par 8 Zekeriya, hizmet sirasinin kendi bölügünde oldugu bir gün, Tanri'nin önünde kâhinlik görevini yerine getiriyordu.
\par 9 Kâhinlik gelenegi uyarinca Rab'bin Tapinagi'na girip buhur yakma görevi kurayla ona verilmisti.
\par 10 Buhur yakma saatinde bütün halk toplulugu disarida dua ediyordu.
\par 11 Bu sirada, Rab'bin bir melegi buhur sunaginin saginda durup Zekeriya'ya göründü.
\par 12 Zekeriya onu görünce sasirdi, korkuya kapildi.
\par 13 Melek, "Korkma, Zekeriya" dedi, "Duan kabul edildi. Karin Elizabet sana bir ogul doguracak, adini Yahya koyacaksin.
\par 14 Sevinip cosacaksin. Birçoklari da onun dogumuna sevinecek.
\par 15 O, Rab'bin gözünde büyük olacak. Hiç sarap ve içki içmeyecek; daha annesinin rahmindeyken Kutsal Ruh'la dolacak.
\par 16 Israilogullari'ndan birçogunu, Tanrilari Rab'be döndürecek.
\par 17 Babalarin yüreklerini çocuklarina döndürmek, söz dinlemeyenleri dogru kisilerin anlayisina yöneltmek ve Rab için hazirlanmis bir halk yetistirmek üzere, Ilyas'in ruhu ve gücüyle Rab'bin önünden gidecektir."
\par 18 Zekeriya melege, "Bundan nasil emin olabilirim?" dedi. "Çünkü ben yaslandim, karimin da yasi ilerledi."
\par 19 Melek ona söyle karsilik verdi: "Ben Tanri'nin huzurunda duran Cebrail'im. Seninle konusmak ve bu müjdeyi sana bildirmek için gönderildim.
\par 20 Iste, belirlenen zamanda yerine gelecek olan sözlerime inanmadigin için dilin tutulacak, bunlarin gerçeklesecegi güne dek konusamayacaksin."
\par 21 Zekeriya'yi bekleyen halk, onun tapinakta bu kadar uzun süre kalmasina sasti.
\par 22 Zekeriya ise disari çiktiginda onlarla konusamadi. O zaman tapinakta bir görüm gördügünü anladilar. Kendisi onlara isaretler yapiyor, ama konusamiyordu.
\par 23 Görev süresi bitince Zekeriya evine döndü.
\par 24 Bir süre sonra karisi Elizabet gebe kaldi ve bes ay evine kapandi.
\par 25 "Bunu benim için yapan Rab'dir" dedi. "Bu günlerde benimle ilgilenerek insanlar arasinda utancimi giderdi."
\par 26 Elizabet'in hamileliginin altinci ayinda Tanri, Melek Cebrail'i Celile'de bulunan Nasira adli kente, Davut'un soyundan Yusuf adindaki adamla nisanli kiza gönderdi. Kizin adi Meryem'di.
\par 28 Onun yanina giren melek, "Selam, ey Tanri'nin lütfuna erisen kiz! Rab seninledir" dedi.
\par 29 Söylenenlere çok sasiran Meryem, bu selamin ne anlama gelebilecegini düsünmeye basladi.
\par 30 Ama melek ona, "Korkma Meryem" dedi, "Sen Tanri'nin lütfuna eristin.
\par 31 Bak, gebe kalip bir ogul doguracak, adini Isa koyacaksin.
\par 32 O büyük olacak, kendisine 'Yüceler Yücesi'nin Oglu' denecek. Rab Tanri O'na, atasi Davut'un tahtini verecek.
\par 33 O da sonsuza dek Yakup'un soyu üzerinde egemenlik sürecek, egemenliginin sonu gelmeyecektir."
\par 34 Meryem melege, "Bu nasil olur? Ben erkege varmadim ki" dedi.
\par 35 Melek ona söyle yanit verdi: "Kutsal Ruh senin üzerine gelecek, Yüceler Yücesi'nin gücü sana gölge salacak. Bunun için dogacak olana kutsal, Tanri Oglu denecek.
\par 36 Bak, senin akrabalarindan Elizabet de yasliliginda bir ogula gebe kaldi. Kisir bilinen bu kadin simdi altinci ayindadir.
\par 37 Tanri'nin yapamayacagi hiçbir sey yoktur."
\par 38 "Ben Rab'bin kuluyum" dedi Meryem, "Bana dedigin gibi olsun." Bundan sonar melek onun yanindan ayrildi.
\par 39 O günlerde Meryem kalkip aceleyle Yahuda'nin daglik bölgesindeki bir kente gitti.
\par 40 Zekeriya'nin evine girip Elizabet'i selamladi.
\par 41 Elizabet Meryem'in selamini duyunca rahmindeki çocuk hopladi. Kutsal Ruh'la dolan Elizabet yüksek sesle söyle dedi: "Kadinlar arasinda kutsanmis bulunuyorsun, rahminin ürünü de kutsanmistir!
\par 43 Nasil oldu da Rabbim'in annesi yanima geldi?
\par 44 Bak, selamin kulaklarima eristigi an, çocuk rahmimde sevinçle hopladi.
\par 45 Iman eden kadina ne mutlu! Çünkü Rab'bin ona söyledigi sözler gerçeklesecektir."
\par 46 Meryem de söyle dedi: "Canim Rab'bi yüceltir; Ruhum, Kurtaricim Tanri sayesinde sevinçle cosar.
\par 48 Çünkü O, siradan biri olan kuluyla ilgilendi. Iste, bundan böyle bütün kusaklar beni mutlu sayacak.
\par 49 Çünkü Güçlü Olan, benim için büyük isler yapti. O'nun adi kutsaldir.
\par 50 Kusaklar boyunca kendisinden korkanlara merhamet eder.
\par 51 Bilegiyle büyük isler yapti; Gururlulari yüreklerindeki kuruntularla darmadagin etti.
\par 52 Hükümdarlari tahtlarindan indirdi, Siradan insanlari yükseltti.
\par 53 Aç olanlari iyiliklerle doyurdu, Zenginleri ise elleri bos çevirdi.
\par 54 Atalarimiza söz verdigi gibi, Ibrahim'e ve onun soyuna sonsuza dek Merhamet etmeyi unutmayarak Kulu Israil'in yardimina yetisti."
\par 56 Meryem, üç ay kadar Elizabet'in yaninda kaldi, sonra kendi evine döndü.
\par 57 Elizabet'in dogurma vakti geldi ve bir ogul dogurdu.
\par 58 Komsulariyla akrabalari, Rab'bin ona ne büyük merhamet gösterdigini duyunca, onun sevincine katildilar.
\par 59 Sekizinci gün çocugun sünnetine geldiler. Ona babasi Zekeriya'nin adini vereceklerdi.
\par 60 Ama annesi, "Hayir, adi Yahya olacak" dedi.
\par 61 Ona, "Akrabalarin arasinda bu adi tasiyan kimse yok ki" dediler.
\par 62 Bunun üzerine babasina isaretle çocugun adini ne koymak istedigini sordular.
\par 63 Zekeriya bir yazi levhasi istedi ve, "Adi Yahya'dir" diye yazdi. Herkes sasakaldi.
\par 64 O anda Zekeriya'nin agzi açildi, dili çözüldü. Tanri'yi överek konusmaya basladi.
\par 65 Çevrede oturanlarin hepsi korkuya kapildi. Bütün bu olaylar, Yahudiye'nin daglik bölgesinin her yaninda konusulur oldu.
\par 66 Duyan herkes derin derin düsünüyor, "Acaba bu çocuk ne olacak?" diyordu. Çünkü Rab onunla birlikteydi.
\par 67 Çocugun babasi Zekeriya, Kutsal Ruh'la dolarak su peygamberlikte bulundu:
\par 68 "Israil'in Tanrisi Rab'be övgüler olsun! Çünkü halkinin yardimina gelip onlari fidyeyle kurtardi.
\par 69 Eski çaglardan beri Kutsal peygamberlerinin agzindan bildirdigi gibi, Kulu Davut'un soyundan Bizim için güçlü bir kurtarici çikardi; Düsmanlarimizdan, Bizden nefret edenlerin hepsinin elinden Kurtulusumuzu sagladi.
\par 72 Böylece atalarimiza merhamet ederek Kutsal antlasmasini anmis oldu.
\par 73 Nitekim bizi düsmanlarimizin elinden kurtaracagina Ve ömrümüz boyunca Kendi önünde kutsallik ve dogruluk içinde, Korkusuzca kendisine tapinmamizi saglayacagina dair Atamiz Ibrahim'e ant içerek söz vermisti.
\par 76 Sen de, ey çocuk, Yüceler Yücesi'nin peygamberi diye anilacaksin. Rab'bin yollarini hazirlamak üzere önünden gidecek Ve O'nun halkina, Günahlarinin bagislanmasiyla kurtulacaklarini bildireceksin.
\par 78 Çünkü Tanrimiz'in yüregi merhamet doludur. O'nun merhameti sayesinde, Yücelerden dogan Günes, Karanlikta ve ölümün gölgesinde yasayanlara isik saçmak Ve ayaklarimizi esenlik yoluna yöneltmek üzere Yardimimiza gelecektir."
\par 80 Çocuk büyüyor, ruhsal yönden güçleniyordu. Israil halkina görünecegi güne dek issiz yerlerde yasadi.

\chapter{2}

\par 1 O günlerde Sezar* Avgustus bütün Roma dünyasinda bir nüfus sayiminin yapilmasi için buyruk çikardi.
\par 2 Bu ilk sayim, Kirinius'un Suriye valiligi zamaninda yapildi.
\par 3 Herkes yazilmak için kendi kentine gitti.
\par 4 Böylece Yusuf da, Davut'un soyundan ve torunlarindan oldugu için Celile'nin Nasira Kenti'nden Yahudiye bölgesine, Davut'un kenti Beytlehem'e gitti.
\par 5 Orada, hamile olan nisanlisi Meryem'le birlikte yazilacakti.
\par 6 Onlar oradayken, Meryem'in dogurma vakti geldi ve ilk oglunu dogurdu. Onu kundaga sarip bir yemlige yatirdi. Çünkü handa yer yoktu.
\par 8 Ayni yörede, sürülerinin yaninda nöbet tutarak geceyi kirlarda geçiren çobanlar vardi.
\par 9 Rab'bin bir melegi onlara göründü ve Rab'bin görkemi çevrelerini aydinlatti. Büyük bir korkuya kapildilar.
\par 10 Melek onlara, "Korkmayin!" dedi. "Size, bütün halki çok sevindirecek bir haber müjdeliyorum: Bugün size, Davut'un kentinde bir Kurtarici dogdu. Bu, Rab olan Mesih'tir*.
\par 12 Iste size bir isaret: Kundaga sarilmis ve yemlikte yatan bir bebek bulacaksiniz."
\par 13 Birdenbire melegin yaninda, göksel ordulardan olusan büyük bir topluluk belirdi. Tanri'yi överek, "En yücelerde Tanri'ya yücelik olsun, Yeryüzünde O'nun hosnut kaldigi insanlara Esenlik olsun!" dediler.
\par 15 Melekler yanlarindan ayrilip göge çekildikten sonra çobanlar birbirlerine, "Haydi, Beytlehem'e gidelim, Rab'bin bize bildirdigi bu olayi görelim" dediler.
\par 16 Aceleyle gidip Meryem'le Yusuf'u ve yemlikte yatan bebegi buldular.
\par 17 Onlari görünce, çocukla ilgili kendilerine anlatilanlari bildirdiler.
\par 18 Bunu duyanlarin hepsi, çobanlarin söylediklerine sasip kaldilar.
\par 19 Meryem ise bütün bu sözleri derin derin düsünerek yüreginde sakliyordu.
\par 20 Çobanlar, isitip gördüklerinin tümü için Tanri'yi yüceltip överek geri döndüler. Her seyi, kendilerine anlatildigi gibi bulmuslardi.
\par 21 Sekizinci gün, çocugu sünnet etme zamani gelince, O'na Isa adi verildi. Bu, O'nun anne rahmine düsmesinden önce melegin kendisine verdigi isimdi.
\par 22 Musa'nin Yasasi'na göre arinma günlerinin bitiminde Yusuf'la Meryem çocugu Rab'be adamak için Yerusalim'e* götürdüler.
\par 23 Nitekim Rab'bin Yasasi'nda, "Ilk dogan her erkek çocuk Rab'be adanmis sayilacak" diye yazilmistir.
\par 24 Ayrica Rab'bin Yasasi'nda buyruldugu gibi, kurban olarak "bir çift kumru ya da iki güvercin yavrusu" sunacaklardi.
\par 25 O sirada Yerusalim'de Simon adinda bir adam vardi. Dogru ve dindar biriydi. Israil'in avutulmasini özlemle bekliyordu. Kutsal Ruh onun üzerindeydi.
\par 26 Rab'bin Mesihi'ni görmeden ölmeyecegi Kutsal Ruh araciligiyla kendisine bildirilmisti.
\par 27 Böylece Simon, Ruh'un yönlendirmesiyle tapinaga geldi. Küçük Isa'nin annesi babasi, Kutsal Yasa'nin ilgili kuralini yerine getirmek üzere O'nu içeri getirdiklerinde, Simon O'nu kucagina aldi, Tanri'yi överek söyle dedi:
\par 29 "Ey Rabbim, verdigin sözü tuttun; Artik ben, kulun huzur içinde ölebilirim.
\par 30 Çünkü senin sagladigin, Bütün halklarin gözü önünde hazirladigin kurtulusu, Uluslari aydinlatip Halkin Israil'e yücelik kazandiracak isigi Gözlerimle gördüm."
\par 33 Isa'nin annesiyle babasi, O'nun hakkinda söylenenlere sastilar.
\par 34 Simon onlari kutsayip çocugun annesi Meryem'e söyle dedi: "Bu çocuk, Israil'de birçok kisinin düsmesine ya da yükselmesine yol açmak ve aleyhinde konusulacak bir belirti olmak üzere belirlenmistir.
\par 35 Senin kalbine de adeta bir kiliç saplanacak. Bütün bunlar, birçoklarinin yüregindeki düsüncelerin açiga çikmasi için olacak."
\par 36 Anna adinda çok yasli bir kadin peygamber vardi. Aser oymagindan Fanuel'in kiziydi. Genç kiz olarak evlenip kocasiyla yedi yil yasadiktan sonra dul kalmisti. Simdi seksen dört yasindaydi. Tapinaktan ayrilmaz, oruç tutup dua ederek gece gündüz Tanri'ya tapinirdi.
\par 38 Tam o sirada ortaya çikan Anna, Tanri'ya sükrederek Yerusalim'in kurtulusunu bekleyen herkese Isa'dan söz etmeye basladi.
\par 39 Yusuf'la Meryem, Rab'bin Yasasi'nda öngörülen her seyi yerine getirdikten sonra Celile'ye, kendi kentleri Nasira'ya döndüler.
\par 40 Çocuk büyüyor, güçleniyor ve bilgelikte yetkinlesiyordu. Tanri'nin lütfu O'nun üzerindeydi.
\par 41 Isa'nin annesi babasi her yil Fisih Bayrami'nda* Yerusalim'e giderlerdi.
\par 42 Isa on iki yasina gelince, bayram gelenegine uyarak yine gittiler.
\par 43 Bayramdan sonra eve dönerlerken küçük Isa Yerusalim'de kaldi. Bunu farketmeyen annesiyle babasi, çocugun yol arkadaslariyla birlikte oldugunu sanarak bir günlük yol gittiler. Sonra O'nu akrabalar ve dostlar arasinda aramaya basladilar.
\par 45 Bulamayinca O'nu araya araya Yerusalim'e döndüler.
\par 46 Üç gün sonra O'nu tapinakta buldular. Din ögretmenleri arasinda oturmus, onlari dinliyor, sorular soruyordu.
\par 47 O'nu dinleyen herkes, zekâsina ve verdigi yanitlara hayran kaliyordu.
\par 48 Annesiyle babasi O'nu görünce sasirdilar. Annesi, "Çocugum, bize bunu niçin yaptin? Bak, babanla ben büyük kaygi içinde seni arayip durduk" dedi.
\par 49 O da onlara, "Beni niçin arayip durdunuz?" dedi. "Babam'in evinde bulunmam gerektigini bilmiyor muydunuz?"
\par 50 Ne var ki onlar ne demek istedigini anlamadilar.
\par 51 Isa onlarla birlikte yola çikip Nasira'ya döndü. Onlarin sözünü dinlerdi. Annesi bütün bu olup bitenleri yüreginde sakladi.
\par 52 Isa bilgelikte ve boyda gelisiyor, Tanri'nin ve insanlarin begenisini kazaniyordu.

\chapter{3}

\par 1 Sezar* Tiberius'un egemenliginin on besinci yiliydi. Yahudiye'de Pontius Pilatus valilik yapiyordu. Celile'yi Hirodes*, Itureya ve Trahonitis bölgesini Hirodes'in kardesi Filipus, Avilini'yi Lisanias yönetiyordu.
\par 2 Hanan ile Kayafa baskâhinlik ediyorlardi. Bu sirada Tanri çölde bulunan Zekeriya oglu Yahya'ya seslendi.
\par 3 O da Seria Irmagi'nin çevresindeki bütün bölgeyi dolasarak insanlari, günahlarinin bagislanmasi için tövbe edip vaftiz* olmaya çagirdi.
\par 4 Nitekim Peygamber Yesaya'nin sözlerini içeren kitapta söyle yazilmistir: "Çölde haykiran, 'Rab'bin yolunu hazirlayin, Geçecegi patikalari düzleyin' diye sesleniyor.
\par 5 'Her vadi doldurulacak, Her dag ve her tepe alçaltilacak. Dolambaçli yollar dogrultulacak, Engebeli yollar düzlestirilecek.
\par 6 Ve bütün insanlar Tanri'nin sagladigi kurtulusu görecektir.'"
\par 7 Yahya, vaftiz olmak için kendisine gelen kalabaliklara söyle seslendi: "Ey engerekler soyu! Gelecek gazaptan kaçmak için sizi kim uyardi?
\par 8 Bundan böyle tövbeye yarasir meyveler verin! Kendi kendinize, 'Biz Ibrahim'in soyundaniz' demeye kalkmayin. Ben size sunu söyleyeyim: Tanri, Ibrahim'e su taslardan da çocuk yaratabilir.
\par 9 Balta agaçlarin köküne dayanmis bile. Iyi meyve vermeyen her agaç kesilip atese atilir."
\par 10 Halk ona, "Öyleyse biz ne yapalim?" diye sordu.
\par 11 Yahya onlara, "Iki mintani olan birini mintani olmayana versin; yiyecegi olan yiyecegi olmayanla paylassin" yanitini verdi.
\par 12 Bazi vergi görevlileri* de vaftiz olmaya gelerek, "Ögretmenimiz, biz ne yapalim?" dediler.
\par 13 Yahya, "Size buyrulandan çok vergi almayin" dedi.
\par 14 Bazi askerler de, "Ya biz ne yapalim?" diye sordular. O da, "Kaba kuvvetle ya da yalan suçlamalarla kimseden para koparmayin" dedi, "Ücretinizle yetinin."
\par 15 Halk umut içinde bekliyordu. Yahya'yla ilgili olarak herkesin aklinda, "Acaba Mesih* bu mu?" sorusu vardi.
\par 16 Yahya ise hepsine söyle yanit verdi: "Ben sizi suyla vaftiz ediyorum, ama benden daha güçlü Olan geliyor. Ben O'nun çariklarinin bagini çözmeye bile layik degilim. O sizi Kutsal Ruh'la ve atesle vaftiz edecek.
\par 17 Harman yerini temizlemek ve bugdayi toplayip ambarina yigmak için yabasi elinde hazir duruyor. Samani ise sönmeyen ateste yakacak."
\par 18 Yahya baska birçok konuda halka çagrida bulunuyor, Müjde'yi duyuruyordu.
\par 19 Ne var ki bölgenin krali* Hirodes, kardesinin karisi Hirodiya'yla ilgili olayi ve kendi yapmis oldugu bütün kötülükleri yüzüne vuran Yahya'yi hapse attirarak kötülüklerine bir yenisini ekledi.
\par 21 Bütün halk vaftiz olduktan sonra Isa da vaftiz oldu. Dua ederken gök açildi ve Kutsal Ruh, bedensel görünümde, güvercin gibi O'nun üzerine indi. Gökten, "Sen benim sevgili Oglum'sun, senden hosnudum" diyen bir ses duyuldu.
\par 23 Isa görevine basladigi zaman otuz yaslarindaydi. Yusuf'un oglu oldugu saniliyordu. Yusuf da Eli oglu,
\par 24 Mattat oglu, Levi oglu, Malki oglu, Yannay oglu, Yusuf oglu,
\par 25 Mattitya oglu, Amos oglu, Nahum oglu, Hesli oglu, Nagay oglu,
\par 26 Mahat oglu, Mattitya oglu, Simi oglu, Yosek oglu, Yoda oglu,
\par 27 Yohanan oglu, Resa oglu, Zerubbabil oglu, Sealtiel oglu, Neri oglu,
\par 28 Malki oglu, Addi oglu, Kosam oglu, Elmadam oglu, Er oglu,
\par 29 Yesu oglu, Eliezer oglu, Yorim oglu, Mattat oglu, Levi oglu,
\par 30 Simon oglu, Yahuda oglu, Yusuf oglu, Yonam oglu, Elyakim oglu,
\par 31 Mala oglu, Menna oglu, Mattata oglu, Natan oglu, Davut oglu,
\par 32 Isay oglu, Ovet oglu, Boaz oglu, Salmon oglu, Nahson oglu,
\par 33 Amminadav oglu, Ram oglu, Hesron oglu, Peres oglu, Yahuda oglu,
\par 34 Yakup oglu, Ishak oglu, Ibrahim oglu, Terah oglu, Nahor oglu,
\par 35 Seruk oglu, Reu oglu, Pelek oglu, Ever oglu, Selah oglu,
\par 36 Kenan oglu, Arpaksat oglu, Sam oglu, Nuh oglu, Lemek oglu,
\par 37 Metuselah oglu, Hanok oglu, Yeret oglu, Mahalalel oglu, Kenan oglu,
\par 38 Enos oglu, Sit oglu, Adem oglu, Tanri Oglu'ydu.

\chapter{4}

\par 1 Kutsal Ruh'la dolu olarak Seria Irmagi'ndan dönen Isa, Ruh'un yönlendirmesiyle çölde dolastirilarak kirk gün Iblis tarafindan denendi. O günlerde hiçbir sey yemedi. Dolayisiyla bu süre sonunda acikti.
\par 3 Bunun üzerine Iblis O'na, "Tanri'nin Oglu'ysan, su tasa söyle ekmek olsun" dedi.
\par 4 Isa, "'Insan yalniz ekmekle yasamaz' diye yazilmistir" karsiligini verdi.
\par 5 Sonra Iblis Isa'yi yükseklere çikararak bir anda O'na dünyanin bütün ülkelerini gösterdi.
\par 6 O'na, "Bütün bunlarin yönetimini ve zenginligini sana verecegim" dedi. "Bunlar bana teslim edildi, ben de diledigim kisiye veririm.
\par 7 Bana taparsan, hepsi senin olacak."
\par 8 Isa ona su karsiligi verdi: "'Tanrin Rab'be tapacak, yalniz O'na kulluk edeceksin' diye yazilmistir."
\par 9 Iblis O'nu Yerusalim'e götürüp tapinagin tepesine çikardi. "Tanri'nin Oglu'ysan, kendini buradan asagi at" dedi.
\par 10 "Çünkü söyle yazilmistir: 'Tanri, seni korumalari için Meleklerine buyruk verecek.'
\par 11 'Ayagin bir tasa çarpmasin diye Seni elleri üzerinde tasiyacaklar.'"
\par 12 Isa ona söyle karsilik verdi: "'Tanrin Rab'bi denemeyeceksin!' diye buyrulmustur."
\par 13 Iblis, Isa'yi her bakimdan denedikten sonra bir süre için O'nun yanindan ayrildi.
\par 14 Isa, Ruh'un gücüyle donanmis olarak Celile'ye döndü. Haber bütün bölgeye yayildi.
\par 15 Oranin havralarinda ögretiyor, herkes tarafindan övülüyordu.
\par 16 Isa, büyüdügü Nasira Kenti'ne geldiginde her zamanki gibi Sabat Günü* havraya gitti. Kutsal Yazilar'i okumak üzere ayaga kalkinca O'na Peygamber Yesaya'nin Kitabi verildi. Kitabi açarak su sözlerin yazili oldugu yeri buldu:
\par 18 "Rab'bin Ruhu üzerimdedir. Çünkü O beni yoksullara Müjde'yi iletmek için meshetti*. Tutsaklara serbest birakilacaklarini, Körlere gözlerinin açilacagini duyurmak için, Ezilenleri özgürlüge kavusturmak Ve Rab'bin lütuf yilini ilan etmek için Beni gönderdi."
\par 20 Sonra kitabi kapatti, görevliye geri verip oturdu. Havradakilerin hepsi dikkatle O'na bakiyordu.
\par 21 Isa, "Dinlediginiz bu Yazi bugün yerine gelmistir" diye konusmaya basladi.
\par 22 Herkes Isa'yi övüyor, agzindan çikan lütufkâr sözlere hayran kaliyordu. "Yusuf'un oglu degil mi bu?" diyorlardi.
\par 23 Isa onlara söyle dedi: "Kuskusuz bana su deyimi hatirlatacaksiniz: 'Ey hekim, önce kendini iyilestir! Kefarnahum'da yaptiklarini duyduk. Aynisini burada, kendi memleketinde de yap.'"
\par 24 "Size dogrusunu söyleyeyim" diye devam etti Isa, "Hiçbir peygamber kendi memleketinde kabul görmez.
\par 25 Yine size gerçegi söyleyeyim, gökyüzünün üç yil alti ay kapali kaldigi, bütün ülkede korkunç bir kitligin bas gösterdigi Ilyas zamaninda Israil'de çok sayida dul kadin vardi.
\par 26 Ilyas bunlardan hiçbirine gönderilmedi; yalniz Sayda bölgesinin Sarefat Kenti'nde bulunan dul bir kadina gönderildi.
\par 27 Peygamber Elisa'nin zamaninda Israil'de çok sayida cüzamli* vardi. Bunlardan hiçbiri iyilestirilmedi; yalniz Suriyeli Naaman iyilestirildi."
\par 28 Havradakiler bu sözleri duyunca öfkeden kudurdular.
\par 29 Ayaga kalkip Isa'yi kentin disina kovdular. O'nu uçurumdan asagi atmak için kentin kuruldugu tepenin yamacina götürdüler.
\par 30 Ama Isa onlarin arasindan geçerek oradan uzaklasti.
\par 31 Sonra Isa Celile'nin Kefarnahum Kenti'ne gitti. Sabat Günü* halka ögretiyordu.
\par 32 Yetkiyle konustugu için O'nun ögretisine sasip kaldilar.
\par 33 Havrada cinli, içinde kötü ruh olan bir adam vardi. Adam yüksek sesle, "Ey Nasirali Isa, birak bizi! Bizden ne istiyorsun?" diye bagirdi. "Bizi mahvetmeye mi geldin? Senin kim oldugunu biliyorum, Tanri'nin Kutsali'sin sen!"
\par 35 Isa, "Sus, çik adamdan!" diyerek cini azarladi. Cin adami herkesin önünde yere vurduktan sonra, ona hiç zarar vermeden içinden çikti.
\par 36 Herkes saskina dönmüstü. Birbirlerine, "Bu nasil söz? Güç ve yetkiyle kötü ruhlara çikmalarini buyuruyor, onlar da çikiyor!" diyorlardi.
\par 37 Isa'yla ilgili haber o bölgenin her yaninda yankilandi.
\par 38 Isa havradan ayrilarak Simun'un evine gitti. Simun'un kaynanasi hastaydi, atesler içindeydi. Onun için Isa'dan yardim istediler.
\par 39 Isa kadinin basucunda durup atesi azarladi, kadinin atesi düstü. Kadin hemen ayaga kalkip onlara hizmet etmeye basladi.
\par 40 Günes batarken herkes çesitli hastaliklara yakalanmis akrabalarini Isa'ya getirdi. Isa her birinin üzerine ellerini koyarak onlari iyilestirdi.
\par 41 Birçogunun içinden cinler de, "Sen Tanri'nin Oglu'sun!" diye bagirarak çikiyordu. Ne var ki, Isa onlari azarladi, konusmalarina izin vermedi. Çünkü kendisinin Mesih* oldugunu biliyorlardi.
\par 42 Sabah olunca Isa disari çikip issiz bir yere gitti. Halk ise O'nu ariyordu. Bulundugu yere geldiklerinde O'nu yanlarinda alikoymaya çalistilar.
\par 43 Ama Isa, "Öbür kentlerde de Tanri'nin Egemenligi'yle ilgili Müjde'yi yaymam gerek" dedi. "Çünkü bunun için gönderildim."
\par 44 Böylece Yahudiye'deki havralarda Tanri sözünü duyurmaya devam etti.

\chapter{5}

\par 1 Halk, Ginnesar Gölü'nün kiyisinda duran Isa'nin çevresini sarmis, Tanri'nin sözünü dinliyordu.
\par 2 Isa, gölün kiyisinda iki tekne gördü. Balikçilar teknelerinden inmis aglarini yikiyorlardi.
\par 3 Iki tekneden Simun'a ait olanina binen Isa, ona kiyidan biraz açilmasini rica etti. Sonra oturdu, teknenin içinden halka ögretmeye devam etti.
\par 4 Konusmasini bitirince Simun'a, "Derin sulara açilin, balik tutmak için aglarinizi atin" dedi.
\par 5 Simun su karsiligi verdi: "Efendimiz, bütün gece çabaladik, hiçbir sey tutamadik. Yine de senin sözün üzerine aglari atacagim."
\par 6 Bunu yapinca öyle çok balik yakaladilar ki, aglari yirtilmaya basladi.
\par 7 Öbür teknedeki ortaklarina isaret ederek gelip yardim etmelerini istediler. Onlar da geldiler ve her iki tekneyi balikla doldurdular; tekneler neredeyse batiyordu.
\par 8 Simun Petrus bunu görünce, "Ya Rab, benden uzak dur, ben günahli bir adamim" diyerek Isa'nin dizlerine kapandi.
\par 9 Kendisi ve yanindakiler, tutmus olduklari baliklarin çokluguna sasip kalmislardi.
\par 10 Simun'un ortaklari olan Zebedi ogullari Yakup'la Yuhanna'yi da ayni saskinlik almisti. Isa Simun'a, "Korkma" dedi, "Bundan böyle balik yerine insan tutacaksin."
\par 11 Sonra onlar tekneleri karaya çektiler ve her seyi birakip Isa'nin ardindan gittiler.
\par 12 Isa kentlerden birindeyken, her yanini cüzam* kaplamis bir adamla karsilasti. Adam Isa'yi görünce yüzüstü yere kapanip yalvardi: "Ya Rab, istersen beni temiz kilabilirsin" dedi.
\par 13 Isa elini uzatip adama dokundu, "Isterim, temiz ol!" dedi. Adam aninda cüzamdan kurtuldu.
\par 14 Isa ona, bundan kimseye söz etmemesini buyurdu. "Git, kâhine* görün ve cüzamdan temizlendigini herkese kanitlamak için Musa'nin buyurdugu sunulari sun" dedi.
\par 15 Ne var ki, Isa'yla ilgili haber daha da çok yayildi. Kalabalik halk topluluklari Isa'yi dinlemek ve hastaliklarindan kurtulmak amaciyla akin akin geliyordu.
\par 16 Kendisi ise issiz yerlere çekilip dua ediyordu.
\par 17 Bir gün Isa ögretiyordu. Celile'nin ve Yahudiye'nin bütün köylerinden ve Yerusalim'den gelen Ferisiler'le* Kutsal Yasa ögretmenleri O'nun çevresinde oturuyorlardi. Isa, Rab'bin gücü sayesinde hastalari iyilestiriyordu.
\par 18 O sirada birkaç kisi, yatak üzerinde tasidiklari felçli bir adami evden içeri sokup Isa'nin önüne koymaya çalisiyordu.
\par 19 Kalabaliktan ötürü onu içeri sokacak yol bulamayinca dama çiktilar, kiremitleri kaldirip adami yatakla birlikte orta yere, Isa'nin önüne indirdiler.
\par 20 Isa onlarin imanini görünce, "Dostum, günahlarin bagislandi" dedi.
\par 21 Din bilginleriyle* Ferisiler, "Tanri'ya küfreden bu adam kim? Tanri'dan baska kim günahlari bagislayabilir?" diye düsünmeye basladilar.
\par 22 Akillarindan geçenleri bilen Isa onlara söyle seslendi: "Aklinizdan neden böyle seyler geçiriyorsunuz?
\par 23 Hangisi daha kolay, 'Günahlarin bagislandi' demek mi, yoksa 'Kalk, yürü' demek mi?
\par 24 Ne var ki, Insanoglu'nun* yeryüzünde günahlari bagislama yetkisine sahip oldugunu bilesiniz diye..." Sonra felçli adama, "Sana söylüyorum, kalk, yatagini toplayip evine git!" dedi.
\par 25 Adam onlarin gözü önünde hemen ayaga kalkti, üzerinde yattigi yatagi topladi ve Tanri'yi yücelterek evine gitti.
\par 26 Herkesi bir saskinlik almisti. Tanri'yi yüceltiyor, büyük korku içinde, "Bugün sasilacak isler gördük!" diyorlardi.
\par 27 Bu olaydan sonra Isa disari çikti, vergi toplama yerinde oturan Levi adinda bir vergi görevlisini* gördü. Adama, "Ardimdan gel" dedi.
\par 28 O da kalkti, her seyi birakip Isa'nin ardindan gitti.
\par 29 Sonra Levi, evinde Isa'nin onuruna büyük bir sölen verdi. Vergi görevlileriyle baska kisilerden olusan büyük bir kalabalik onlarla birlikte yemege oturmustu.
\par 30 Ferisiler'le onlarin din bilginleri söylenmeye basladilar. Isa'nin ögrencilerine, "Siz neden vergi görevlileri ve günahkârlarla birlikte yiyip içiyorsunuz?" dediler.
\par 31 Isa onlara su karsiligi verdi: "Saglikli olanlarin degil, hastalarin hekime ihtiyaci var.
\par 32 Ben dogru kisileri degil, günahkârlari tövbeye çagirmaya geldim."
\par 33 Onlar Isa'ya, "Yahya'nin ögrencileri sik sik oruç tutup dua ediyorlar, Ferisiler'in ögrencileri de öyle. Seninkiler ise yiyip içiyor" dediler.
\par 34 Isa söyle karsilik verdi: "Güvey aralarinda oldugu sürece davetlilere oruç tutturabilir misiniz?
\par 35 Ama güveyin aralarindan alinacagi günler gelecek, onlar iste o zaman, o günler oruç tutacaklar."
\par 36 Isa onlara su benzetmeyi de anlatti: "Hiç kimse yeni giysiden bir parça yirtip eski giysiyi yamamaz. Yoksa hem yeni giysi yirtilir, hem de o giysiden koparilan yama eskisine uymaz.
\par 37 Hiç kimse yeni sarabi eski tulumlara doldurmaz. Yoksa yeni sarap tulumlari patlatir; hem sarap dökülür, hem de tulumlar mahvolur.
\par 38 Yeni sarabi yeni tulumlara doldurmak gerek.
\par 39 Üstelik hiç kimse eski sarabi içtikten sonra yenisini istemez. 'Eskisi güzel' der."

\chapter{6}

\par 1 Bir Sabat Günü* Isa ekinler arasindan geçiyordu. Ögrencileri basaklari kopariyor, avuçlarinda ufalayip yiyorlardi.
\par 2 Ferisiler'den bazilari, "Sabat Günü yasak olani neden yapiyorsunuz?" dediler.
\par 3 Isa onlara söyle karsilik verdi: "Davut'la yanindakiler acikinca Davut'un ne yaptigini okumadiniz mi?
\par 4 Tanri'nin evine girdi, kâhinlerden baskasinin yemesi yasak olan adak ekmeklerini* alip yedi ve yanindakilere de verdi."
\par 5 Sonra Isa onlara, "Insanoglu* Sabat Günü'nün de Rabbi'dir" dedi.
\par 6 Bir baska Sabat Günü Isa havraya girmis ögretiyordu. Orada sag eli sakat bir adam vardi.
\par 7 Isa'yi suçlamak için firsat kollayan din bilginleriyle Ferisiler, Sabat Günü hastalari iyilestirecek mi diye O'nu gözlüyorlardi.
\par 8 Isa, onlarin ne düsündüklerini biliyordu. Eli sakat olan adama, "Ayaga kalk, öne çik" dedi. O da kalkti, orta yerde durdu.
\par 9 Isa onlara, "Size sorayim" dedi, "Kutsal Yasa'ya göre Sabat Günü iyilik yapmak mi dogru, kötülük yapmak mi? Can kurtarmak mi dogru, öldürmek mi?"
\par 10 Gözlerini hepsinin üzerinde gezdirdikten sonra adama, "Elini uzat" dedi. Adam elini uzatti, eli yine sapasaglam oluverdi.
\par 11 Onlar ise öfkeden deliye döndüler ve aralarinda Isa'ya ne yapabileceklerini tartismaya basladilar.
\par 12 O günlerde Isa, dua etmek için daga çikti ve bütün geceyi Tanri'ya dua ederek geçirdi.
\par 13 Gün dogunca ögrencilerini yanina çagirdi ve onlarin arasindan, elçi diye adlandirdigi su on iki kisiyi seçti: Petrus adini verdigi Simun, onun kardesi Andreas, Yakup, Yuhanna, Filipus, Bartalmay, Matta, Tomas, Alfay oglu Yakup, Yurtsever* diye taninan Simun, Yakup oglu Yahuda ve Isa'ya ihanet eden Yahuda Iskariot.
\par 17 Isa bunlarla birlikte asagi inip düzlük bir yerde durdu. Ögrencilerinden büyük bir kalabalik ve bütün Yahudiye'den, Yerusalim'den, Sur'la Sayda yakinlarindaki kiyi bölgesinden gelen büyük bir halk toplulugu da oradaydi.
\par 18 Isa'yi dinlemek ve hastaliklarina sifa bulmak için gelmislerdi. Kötü ruhlar yüzünden sikinti çekenler de iyilestiriliyordu.
\par 19 Kalabalikta herkes Isa'ya dokunmak için çabaliyordu. Çünkü O'nun içinden akan bir güç herkese sifa veriyordu.
\par 20 Isa, gözlerini ögrencilerine çevirerek söyle dedi: "Ne mutlu size, ey yoksullar! Çünkü Tanri'nin Egemenligi sizindir.
\par 21 Ne mutlu size, simdi açlik çekenler! Çünkü doyurulacaksiniz. Ne mutlu size, simdi aglayanlar! Çünkü güleceksiniz.
\par 22 Insanoglu'na* bagliliginiz yüzünden Insanlar sizden nefret ettikleri, Sizi toplum disi edip asagiladiklari Ve adinizi kötüleyip sizi reddettikleri zaman Ne mutlu size!
\par 23 O gün sevinin, coskuyla ziplayin! Çünkü gökteki ödülünüz büyüktür. Nitekim onlarin atalari da Peygamberlere böyle davrandilar.
\par 24 Ama vay halinize, ey zenginler, Çünkü tesellinizi almis bulunuyorsunuz!
\par 25 Vay halinize, simdi karni tok olan sizler, Çünkü açlik çekeceksiniz! Vay halinize, ey simdi gülenler, Çünkü yas tutup aglayacaksiniz!
\par 26 Bütün insanlar sizin için iyi sözler söyledikleri zaman, Vay halinize! Çünkü onlarin atalari da Sahte peygamberlere böyle davrandilar."
\par 27 "Ama beni dinleyen sizlere sunu söylüyorum: Düsmanlarinizi sevin, sizden nefret edenlere iyilik yapin, size lanet edenler için iyilik dileyin, size hakaret edenler için dua edin.
\par 29 Bir yanaginiza vurana öbür yanaginizi da çevirin. Abanizi alandan mintaninizi da esirgemeyin.
\par 30 Sizden bir sey dileyen herkese verin, malinizi alandan onu geri istemeyin.
\par 31 Insanlarin size nasil davranmasini istiyorsaniz, siz de onlara öyle davranin.
\par 32 "Eger yalniz sizi sevenleri severseniz, bu size ne övgü kazandirir? Günahkârlar bile kendilerini sevenleri sever.
\par 33 Size iyilik yapanlara iyilik yaparsaniz, bu size ne övgü kazandirir? Günahkârlar bile böyle yapar.
\par 34 Geri alacaginizi umdugunuz kisilere ödünç verirseniz, bu size ne övgü kazandirir? Günahkârlar bile verdiklerini geri almak kosuluyla günahkârlara ödünç verirler.
\par 35 Ama siz düsmanlarinizi sevin, iyilik yapin, hiçbir karsilik beklemeden ödünç verin. Alacaginiz ödül büyük olacak, Yüceler Yücesi'nin ogullari olacaksiniz. Çünkü O, nankör ve kötü kisilere karsi iyi yüreklidir.
\par 36 Babaniz merhametli oldugu gibi, siz de merhametli olun."
\par 37 "Baskasini yargilamayin, siz de yargilanmazsiniz. Suçlu çikarmayin, siz de suçlu çikarilmazsiniz. Baskasini bagislayin, siz de bagislanirsiniz.
\par 38 Verin, size verilecektir. Iyice bastirilmis, silkelenmis ve tasmis, dolu bir ölçekle kucaginiza bosaltilacak. Hangi ölçekle verirseniz, ayni ölçekle alacaksiniz."
\par 39 Isa onlara su benzetmeyi de anlatti: "Kör köre kilavuzluk edebilir mi? Ikisi de çukura düsmez mi?
\par 40 Ögrenci ögretmeninden üstün degildir, ama egitimini tamamlayan her ögrenci ögretmeni gibi olacaktir.
\par 41 "Sen neden kardesinin gözündeki çöpü görürsün de kendi gözündeki mertegi farketmezsin?
\par 42 Kendi gözündeki mertegi görmezken, kardesine nasil, 'Kardes, izin ver, gözündeki çöpü çikarayim' dersin? Seni ikiyüzlü! Önce kendi gözündeki mertegi çikar, o zaman kardesinin gözündeki çöpü çikarmak için daha iyi görürsün."
\par 43 "Iyi agaç kötü meyve, kötü agaç da iyi meyve vermez.
\par 44 Her agaç meyvesinden taninir. Dikenli bitkilerden incir toplanmaz, çalilardan üzüm devsirilmez.
\par 45 Iyi insan yüregindeki iyilik hazinesinden iyilik, kötü insan içindeki kötülük hazinesinden kötülük çikarir. Insanin agzi, yüreginden tasani söyler.
\par 46 "Niçin beni 'Ya Rab, ya Rab' diye çagiriyorsunuz da söylediklerimi yapmiyorsunuz?
\par 47 Bana gelen ve sözlerimi duyup uygulayan kisinin kime benzedigini size anlatayim.
\par 48 Böyle bir kisi, evini yaparken topragi kazan, derinlere inip temeli kaya üzerine atan adama benzer. Sel sulariyla kabaran irmak o eve saldirsa da, onu sarsamaz. Çünkü ev saglam yapilmistir.
\par 49 Ama sözlerimi duyup da uygulamayan kisi, evini temel koymaksizin topragin üzerine kuran adama benzer. Kabaran irmak saldirinca ev hemen çöker. Evin yikilisi da korkunç olur."

\chapter{7}

\par 1 Isa, kendisini dinleyen halka bütün bu sözleri söyledikten sonra Kefarnahum'a gitti.
\par 2 Orada bir yüzbasinin çok deger verdigi kölesi ölüm döseginde hasta yatiyordu.
\par 3 Isa'yla ilgili haberleri duyan yüzbasi, gelip kölesini iyilestirmesini rica etmek üzere O'na Yahudiler'in bazi ileri gelenlerini gönderdi.
\par 4 Bunlar Isa'nin yanina gelince içten bir yalvarisla O'na söyle dediler: "Bu adam senin yardimina layiktir.
\par 5 Çünkü ulusumuzu seviyor. Havramizi yaptiran da kendisidir."
\par 6 Isa onlarla birlikte yola çikti. Eve yaklastigi sirada, yüzbasi bazi dostlarini yollayip O'na su haberi gönderdi: "Ya Rab, zahmet etme; evime girmene layik degilim.
\par 7 Bu yüzden yanina gelmeye de kendimi layik görmedim. Sen yeter ki bir söz söyle, usagim iyilesir.
\par 8 Ben de buyruk altinda bir görevliyim, benim de buyrugumda askerlerim var. Birine, 'Git' derim, gider; ötekine, 'Gel' derim, gelir; köleme, 'Sunu yap' derim, yapar."
\par 9 Bu sözleri duyan Isa yüzbasiya hayran kaldi. Ardindan gelen kalabaliga dönerek, "Size sunu söyleyeyim" dedi, "Israil'de bile böyle iman görmedim."
\par 10 Gönderilenler eve döndüklerinde köleyi iyilesmis buldular.
\par 11 Bundan kisa bir süre sonra Isa, Nain denilen bir kente gitti. Ögrencileriyle büyük bir kalabalik O'na eslik ediyordu.
\par 12 Isa kentin kapisina tam yaklastigi sirada, dul annesinin tek oglu olan bir adamin cenazesi kaldiriliyordu. Kent halkindan büyük bir kalabalik da kadinla birlikteydi.
\par 13 Rab kadini görünce ona acidi. Kadina, "Aglama" dedi.
\par 14 Yaklasip cenaze sedyesine dokununca sedyeyi tasiyanlar durdu. Isa, "Delikanli" dedi, "Sana söylüyorum, kalk!"
\par 15 Ölü dogrulup oturdu ve konusmaya basladi. Isa onu annesine geri verdi.
\par 16 Herkesi bir korku almisti. "Aramizda büyük bir peygamber ortaya çikti!" ve "Tanri, halkinin yardimina geldi!" diyerek Tanri'yi yüceltmeye basladilar.
\par 17 Isa'yla ilgili bu haber bütün Yahudiye'ye ve çevre bölgelere yayildi.
\par 18 Yahya'nin ögrencileri bütün bu olup bitenleri kendisine bildirdiler. Ögrencilerinden ikisini yanina çagiran Yahya, "Gelecek Olan sen misin, yoksa baskasini mi bekleyelim?" diye sormalari için onlari Rab'be gönderdi.
\par 20 Adamlar Isa'nin yanina gelince söyle dediler: "Bizi sana Vaftizci Yahya gönderdi. 'Gelecek Olan sen misin, yoksa baskasini mi bekleyelim?' diye soruyor."
\par 21 Tam o sirada Isa, çesitli hastaliklara, illetlere ve kötü ruhlara tutulmus birçok kisiyi iyilestirdi, birçok körün gözünü açti.
\par 22 Sonra Yahya'nin ögrencilerine söyle karsilik verdi: "Gidin, görüp isittiklerinizi Yahya'ya bildirin. Körlerin gözleri açiliyor, kötürümler yürüyor, cüzamlilar temiz kiliniyor, sagirlar isitiyor, ölüler diriliyor ve Müjde yoksullara duyuruluyor.
\par 23 Benden ötürü sendeleyip düsmeyene ne mutlu!"
\par 24 Yahya'nin gönderdigi haberciler gittikten sonra Isa, halka Yahya'dan söz etmeye basladi. "Çöle ne görmeye gittiniz?" dedi. "Rüzgarda sallanan bir kamis mi?
\par 25 Söyleyin, ne görmeye gittiniz? Pahali giysiler giymis bir adam mi? Oysa sahane giysiler giyip bolluk içinde yasayanlar kral saraylarinda bulunur.
\par 26 Öyleyse ne görmeye gittiniz? Bir peygamber mi? Evet! Size sunu söyleyeyim, gördügünüz kisi peygamberden de üstündür.
\par 27 'Iste, habercimi senin önünden gönderiyorum; O önden gidip senin yolunu hazirlayacak' diye yazilmis olan sözler onunla ilgilidir.
\par 28 Size sunu söyleyeyim, kadindan doganlar arasinda Yahya'dan daha üstün olani yoktur. Bununla birlikte, Tanri'nin Egemenligi'nde en küçük olan ondan üstündür."
\par 29 Yahya tarafindan vaftiz edilen halk, hatta vergi görevlileri* bile bunu duyunca Tanri'nin adil oldugunu dogruladilar.
\par 30 Oysa Yahya tarafindan vaftiz edilmeye yanasmayan Ferisiler'le Kutsal Yasa uzmanlari, Tanri'nin kendileriyle ilgili tasarisini reddettiler.
\par 31 Isa, "Bu kusagin insanlarini neye benzeteyim? Bunlar neye benziyorlar?" dedi.
\par 32 "Çarsi meydaninda oturup birbirlerine, 'Size kaval çaldik, oynamadiniz; Agit yaktik, aglamadiniz' \m diye seslenen çocuklara benziyorlar.
\par 33 Vaftizci Yahya geldigi zaman oruç tutup saraptan kaçindi, ona 'cinli' diyorsunuz.
\par 34 Insanoglu* geldigi zaman yiyip içti. Bu kez de diyorsunuz ki, 'Su obur ve ayyas adama bakin! Vergi görevlileri* ve günahkârlarla dost oldu!'
\par 35 Ne var ki bilgelik, onu benimseyen herkes tarafindan dogrulanir."
\par 36 Ferisiler'den biri Isa'yi yemege çagirdi. O da Ferisi'nin evine gidip sofraya oturdu.
\par 37 O sirada, kentte günahkâr olarak taninan bir kadin, Isa'nin, Ferisi'nin evinde yemek yedigini ögrenince kaymaktasindan bir kap içinde güzel kokulu yag getirdi. Isa'nin arkasinda, ayaklarinin dibinde durup aglayarak, gözyaslariyla O'nun ayaklarini islatmaya basladi. Saçlariyla ayaklarini sildi, öptü ve yagi üzerlerine sürdü.
\par 39 Isa'yi evine çagirmis olan Ferisi bunu görünce kendi kendine, "Bu adam peygamber olsaydi, kendisine dokunan bu kadinin kim ve ne tür bir kadin oldugunu, günahkâr biri oldugunu anlardi" dedi.
\par 40 Bunun üzerine Isa Ferisi'ye, "Simun" dedi, "Sana bir söyleyecegim var." O da, "Buyur, ögretmenim" dedi.
\par 41 "Tefeciye borçlu iki kisi vardi. Biri bes yüz, öbürü de elli dinar borçluydu.
\par 42 Borçlarini ödeyecek güçte olmadiklarindan, tefeci her ikisinin de borcunu bagisladi. Buna göre, hangisi onu çok sever?"
\par 43 Simun, "Sanirim, kendisine daha çok bagislanan" diye yanitladi. Isa ona, "Dogru söyledin" dedi.
\par 44 Sonra kadina bakarak Simun'a sunlari söyledi: "Bu kadini görüyor musun? Ben senin evine geldim, ayaklarim için bana su vermedin. Bu kadin ise ayaklarimi gözyaslariyla islatip saçlariyla sildi.
\par 45 Sen beni öpmedin, ama bu kadin eve girdigimden beri ayaklarimi öpüp duruyor.
\par 46 Sen basima zeytinyagi sürmedin, ama bu kadin ayaklarima güzel kokulu yag sürdü.
\par 47 Bu nedenle sana sunu söyleyeyim, kendisinin çok olan günahlari bagislanmistir. Çok sevgi göstermesinin nedeni budur. Oysa kendisine az bagislanan, az sever."
\par 48 Sonra kadina, "Günahlarin bagislandi" dedi.
\par 49 Isa'yla birlikte sofrada oturanlar kendi aralarinda, "Kim bu adam? Günahlari bile bagisliyor!" seklinde konusmaya basladilar.
\par 50 Isa ise kadina, "Imanin seni kurtardi, esenlikle git" dedi.

\chapter{8}

\par 1 Bundan kisa bir süre sonra Isa on iki ögrencisiyle birlikte köy kent dolasmaya basladi. Tanri'nin Egemenligi'ni duyurup müjdeliyordu.
\par 2 Kötü ruhlardan ve hastaliklardan kurtulan bazi kadinlar, içinden yedi cin çikmis olan Mecdelli denilen Meryem, Hirodes'in* kâhyasi Kuza'nin karisi Yohanna, Suzanna ve daha birçoklari Isa'yla birlikte dolasiyordu. Bunlar, kendi olanaklariyla Isa'ya ve ögrencilerine yardim ediyorlardi.
\par 4 Büyük bir kalabaligin toplandigi, insanlarin her kentten kendisine akin akin geldigi bir sirada Isa su benzetmeyi anlatti: "Ekincinin biri tohum ekmeye çikti. Ektigi tohumlardan kimi yol kenarina düstü, ayak altinda çignenip gökteki kuslara yem oldu.
\par 6 Kimi kayalik yere düstü, filizlenince susuzluktan kuruyup gitti.
\par 7 Kimi, dikenler arasina düstü. Filizlerle birlikte büyüyen dikenler filizleri bogdu.
\par 8 Kimi ise iyi topraga düstü, büyüyünce yüz kat ürün verdi." Bunlari söyledikten sonra, "Isitecek kulagi olan isitsin!" diye seslendi.
\par 9 Isa, bu benzetmenin anlamini kendisinden soran ögrencilerine, "Tanri Egemenligi'nin sirlarini bilme ayricaligi size verildi" dedi. "Ama baskalarina benzetmelerle sesleniyorum. Öyle ki, 'Gördükleri halde görmesinler, Duyduklari halde anlamasinlar.'
\par 11 "Benzetmenin anlami sudur: Tohum Tanri'nin sözüdür.
\par 12 Yol kenarindakiler sözü isiten kisilerdir. Ama sonra Iblis gelir, inanip kurtulmasinlar diye sözü yüreklerinden alir götürür.
\par 13 Kayalik yere düsenler, isittikleri sözü sevinçle kabul eden, ama kök salamadiklari için ancak bir süre inanan kisilerdir. Böyleleri denendikleri zaman imandan dönerler.
\par 14 Dikenler arasina düsenler, sözü isiten ama zamanla yasamin kaygilari, zenginlikleri ve zevkleri içinde bogulan, dolayisiyla olgun ürün vermeyenlerdir.
\par 15 Iyi topraga düsenler ise, sözü isitince onu iyi ve saglam bir yürekte saklayanlardir. Bunlar sabirla dayanarak ürün verirler."
\par 16 "Hiç kimse kandil yakip bunu bir kapla örtmez, ya da yatagin altina koymaz. Tersine, içeri girenler isigi görsünler diye onu kandillige koyar.
\par 17 Çünkü açiga çikarilmayacak gizli hiçbir sey yok; bilinmeyecek, aydinliga çikmayacak sakli hiçbir sey yoktur.
\par 18 Bunun için, nasil dinlediginize dikkat edin. Kimde varsa, ona daha çok verilecek. Ama kimde yoksa, kendisinde var sandigi bile elinden alinacak."
\par 19 Isa'nin annesiyle kardesleri O'na geldiler, ama kalabaliktan ötürü kendisine yaklasamadilar.
\par 20 Isa'ya, "Annenle kardeslerin disarida duruyor, seni görmek istiyorlar" diye haber verildi.
\par 21 Isa haberi getirenlere söyle karsilik verdi: "Annemle kardeslerim, Tanri'nin sözünü duyup yerine getirenlerdir."
\par 22 Bir gün Isa ögrencileriyle birlikte bir tekneye binerek onlara, "Gölün karsi yakasina geçelim" dedi. Böylece kiyidan açildilar.
\par 23 Teknede giderlerken Isa uykuya daldi. O sirada gölde firtina koptu. Tekne su almaya baslayinca tehlikeli bir duruma düstüler.
\par 24 Gidip Isa'yi uyandirarak, "Efendimiz, Efendimiz, ölecegiz!" dediler. Isa kalkip rüzgari ve kabaran dalgalari azarladi. Firtina dindi ve ortalik sütliman oldu.
\par 25 Isa ögrencilerine, "Nerede imaniniz?" dedi. Onlar korku ve saskinlik içindeydiler. Birbirlerine, "Bu adam kim ki, rüzgara, suya bile buyruk veriyor, onlar da sözünü dinliyor!" dediler.
\par 26 Celile'nin karsisinda bulunan Gerasalilar'in memleketine vardilar.
\par 27 Isa karaya çikinca kentten bir adam O'nu karsiladi. Cinli ve uzun zamandan beri giysi giymeyen bu adam evde degil, mezarlik magaralarda yasiyordu.
\par 28 Adam Isa'yi görünce çiglik atip önünde yere kapandi. Yüksek sesle, "Ey Isa, yüce Tanri'nin Oglu, benden ne istiyorsun?" dedi. "Sana yalvaririm, bana iskence etme!"
\par 29 Çünkü Isa, kötü ruha adamin içinden çikmasini buyurmustu. Kötü ruh adami sik sik etkisi altina aliyordu. Adam zincir ve kösteklerle baglanip basina nöbetçi konuldugu halde baglarini paraliyor ve cin tarafindan issiz yerlere sürülüyordu.
\par 30 Isa ona, "Adin ne?" diye sordu. O da, "Tümen*" diye yanitladi. Çünkü onun içine bir sürü cin girmisti.
\par 31 Cinler, dipsiz derinliklere gitmelerini buyurmasin diye Isa'ya yalvarip durdular.
\par 32 Orada, dagin yamacinda otlayan büyük bir domuz sürüsü vardi. Cinler, domuzlarin içine girmelerine izin vermesi için Isa'ya yalvardilar. O da onlara izin verdi.
\par 33 Adamdan çikan cinler domuzlarin içine girdiler. Sürü dik yamaçtan asagi kosusarak göle atlayip boguldu.
\par 34 Domuzlari güdenler olup biteni görünce kaçtilar, kentte ve köylerde olayin haberini yaydilar.
\par 35 Bunun üzerine halk olup biteni görmeye çikti. Isa'nin yanina geldikleri zaman, cinlerden kurtulan adami giyinmis ve akli basina gelmis olarak Isa'nin ayaklari dibinde oturmus buldular ve korktular.
\par 36 Olayi görenler, cinli adamin nasil kurtuldugunu halka anlattilar.
\par 37 O zaman Gerasa yöresinden gelen bütün kalabalik büyük bir korkuya kapilarak Isa'nin yanlarindan ayrilmasini rica ettiler. O da geri dönmek üzere tekneye bindi.
\par 38 Cinlerden kurtulan adam Isa'nin yaninda kalmak için O'na yalvardi. Ama Isa, "Evine dön, Tanri'nin senin için neler yaptigini anlat" diyerek onu saliverdi. Adam da gitti, Isa'nin kendisi için neler yaptigini bütün kentte duyurdu.
\par 40 Karsi yakaya dönen Isa'yi halk karsiladi. Çünkü herkes O'nu bekliyordu.
\par 41 O sirada, havra yöneticisi olan Yair adinda bir adam gelip Isa'nin ayaklarina kapandi, evine gelmesi için yalvardi.
\par 42 Çünkü on iki yaslarindaki biricik kizi ölmek üzereydi. Isa oraya giderken kalabalik O'nu her yandan sikistiriyordu.
\par 43 On iki yildir kanamasi olan bir kadin da oradaydi. Varini yogunu hekimlere harcamisti; ama hiçbiri onu iyilestirememisti.
\par 44 Isa'nin arkasindan yetisip giysisinin etegine dokundu ve o anda kanamasi kesildi.
\par 45 Isa, "Bana kim dokundu?" dedi. Herkes inkâr ederken Petrus, "Efendimiz, kalabalik seni çepeçevre sarmis sikistiriyor" dedi.
\par 46 Ama Isa, "Birisi bana dokundu" dedi. "Içimden bir gücün akip gittigini hissettim."
\par 47 Yaptigini gizleyemeyecegini anlayan kadin titreyerek geldi, Isa'nin ayaklarina kapandi. Bütün halkin önünde, O'na neden dokundugunu ve o anda nasil iyilestigini anlatti.
\par 48 Isa ona, "Kizim" dedi, "Imanin seni kurtardi. Esenlikle git."
\par 49 Isa daha konusurken havra yöneticisinin evinden biri geldi. Yöneticiye, "Kizin öldü" dedi, "Artik ögretmeni rahatsiz etme."
\par 50 Isa bunu duyunca havra yöneticisine söyle dedi: "Korkma, yalniz iman et, kizin kurtulacak."
\par 51 Isa adamin evine gelince Petrus, Yuhanna, Yakup ve kizin annesi babasi disinda hiç kimsenin kendisiyle birlikte içeri girmesine izin vermedi.
\par 52 Herkes kiz için agliyor, dövünüyordu. Isa, "Aglamayin" dedi, "Kiz ölmedi, uyuyor."
\par 53 Kizin öldügünü bildikleri için Isa'yla alay ettiler.
\par 54 O ise kizin elini tutarak, "Kizim, kalk!" diye seslendi.
\par 55 Ruhu yeniden bedenine dönen kiz hemen ayaga kalkti. Isa, kiza yemek verilmesini buyurdu.
\par 56 Kizin annesiyle babasi saskinlik içindeydi. Isa, olanlari hiç kimseye anlatmamalari için onlari uyardi.

\chapter{9}

\par 1 Isa, Onikiler'i* yanina çagirip onlara bütün cinler üzerinde ve hastaliklari iyilestirmek için güç ve yetki verdi.
\par 2 Sonra onlari Tanri'nin Egemenligi'ni duyurmaya ve hastalara sifa vermeye gönderdi.
\par 3 Onlara söyle dedi: "Yolculuk için yaniniza hiçbir sey almayin: Ne degnek, ne torba, ne ekmek, ne para, ne de yedek mintan.
\par 4 Hangi eve girerseniz, kentten ayrilincaya dek orada kalin.
\par 5 Sizi kabul etmeyenlere gelince, kentten ayrilirken onlara uyari olsun diye ayaklarinizin tozunu silkin."
\par 6 Onlar da yola çiktilar, her yerde Müjde'yi yayarak ve hastalari iyilestirerek köy köy dolastilar.
\par 7 Bölgenin krali* Hirodes bütün bu olanlari duyunca saskina döndü. Çünkü bazilari Yahya'nin ölümden dirildigini, bazilari Ilyas'in göründügünü, baskalari ise eski peygamberlerden birinin dirildigini söylüyordu.
\par 9 Hirodes, "Yahya'nin basini ben kestirdim. Simdi hakkinda böyle haberler duydugum bu adam kim?" diyor ve Isa'yi görmenin bir yolunu ariyordu.
\par 10 Elçiler geri dönünce, yaptiklari her seyi Isa'ya anlattilar. Sonra Isa yalnizca onlari yanina alip Beytsayda denilen bir kente çekildi.
\par 11 Bunu ögrenen halk O'nun ardindan gitti. Isa onlari ilgiyle karsiladi, kendilerine Tanri'nin Egemenligi'nden söz etti ve sifaya ihtiyaci olanlari iyilestirdi.
\par 12 Günbatimina dogru Onikiler gelip O'na, "Halki saliver de çevredeki köylere ve çiftliklere gidip kendilerine barinak ve yiyecek bulsunlar. Çünkü issiz bir yerdeyiz" dediler.
\par 13 Isa, "Onlara siz yiyecek verin" dedi. "Bes ekmekle iki baliktan baska bir seyimiz yok" dediler. "Yoksa bunca halk için yiyecek almaya biz mi gidelim?"
\par 14 Orada yaklasik bes bin erkek vardi. Isa ögrencilerine, "Halki yaklasik elliser kisilik kümeler halinde yere oturtun" dedi.
\par 15 Ögrenciler öyle yapip herkesi yere oturttular.
\par 16 Isa, bes ekmekle iki baligi aldi, gözlerini göge kaldirarak sükretti; sonra bunlari böldü ve halka dagitmalari için ögrencilerine verdi.
\par 17 Herkes yiyip doydu. Artakalan parçalardan on iki sepet dolusu toplandi.
\par 18 Bir gün Isa tek basina dua ediyordu, ögrencileri de yanindaydi. Isa onlara, "Halk benim kim oldugumu söylüyor?" diye sordu.
\par 19 Söyle yanitladilar: "Vaftizci Yahya diyorlar. Ama kimi Ilyas, kimi de eski peygamberlerden biri dirilmis, diyor."
\par 20 Isa onlara, "Siz ne dersiniz" dedi, "Sizce ben kimim?" Petrus, "Sen Tanri'nin Mesihi'sin*" yanitini verdi.
\par 21 Isa, onlari uyararak bunu hiç kimseye söylememelerini buyurdu.
\par 22 Insanoglu'nun* çok aci çekmesi, ileri gelenler, baskâhinler ve din bilginlerince reddedilmesi, öldürülmesi ve üçüncü gün dirilmesi gerektigini söyledi.
\par 23 Sonra hepsine, "Ardimdan gelmek isteyen kendini inkâr etsin, her gün çarmihini yüklenip beni izlesin" dedi,
\par 24 "Canini kurtarmak isteyen onu yitirecek, canini benim ugruma yitiren ise onu kurtaracaktir.
\par 25 Insan bütün dünyayi kazanip da canini yitirirse, canindan olursa, bunun kendisine ne yarari olur?
\par 26 Kim benden ve benim sözlerimden utanirsa, Insanoglu da kendisinin, Babasi'nin ve kutsal meleklerin görkemi içinde geldiginde o kisiden utanacaktir.
\par 27 Size gerçegi söyleyeyim, burada bulunanlar arasinda, Tanri'nin Egemenligi'ni görmeden ölümü tatmayacak olanlar var."
\par 28 Bu sözleri söyledikten yaklasik sekiz gün sonra Isa, yanina Petrus, Yuhanna ve Yakup'u alarak dua etmek üzere daga çikti.
\par 29 Isa dua ederken yüzünün görünümü degisti, giysileri simsek gibi parildayan bir beyazliga büründü.
\par 30 O anda görkem içinde beliren iki kisi Isa'yla konusmaya basladilar. Bunlar Musa ile Ilyas'ti. Isa'nin yakinda Yerusalim'de gerçeklesecek olan ayrilisini konusuyorlardi.
\par 32 Petrus ile yanindakilerin üzerine uyku çökmüstü. Ama uykulari iyice dagilinca Isa'nin görkemini ve yaninda duran iki kisiyi gördüler.
\par 33 Bunlar Isa'nin yanindan ayrilirken Petrus Isa'ya, "Efendimiz" dedi, "Burada bulunmamiz ne iyi oldu! Üç çardak kuralim: Biri sana, biri Musa'ya, biri de Ilyas'a." Aslinda ne söylediginin farkinda degildi.
\par 34 Petrus daha bunlari söylerken bir bulut gelip onlara gölge saldi. Bulut onlari sarinca korktular.
\par 35 Buluttan gelen bir ses, "Bu benim Oglum'dur, seçilmis Olan'dir. O'nu dinleyin!" dedi.
\par 36 Ses kesilince Isa'nin tek basina oldugu görüldü. Ögrenciler bunu gizli tuttular ve o günlerde hiç kimseye gördüklerinden söz etmediler.
\par 37 Ertesi gün dagdan indikleri zaman, Isa'yi büyük bir kalabalik karsiladi.
\par 38 Kalabaligin içinden bir adam, "Ögretmenim" diye seslendi, "Yalvaririm, oglumu bir gör, o tek çocugumdur.
\par 39 Bir ruh onu yakaliyor, o da birdenbire çiglik atiyor. Ruh onu, agzindan köpükler gelene dek siddetle sarsiyor. Bedenini yara bere içinde birakarak güçbela ayriliyor.
\par 40 Ruhu kovmalari için ögrencilerine yalvardim, ama basaramadilar."
\par 41 Isa söyle karsilik verdi: "Ey imansiz ve sapmis kusak! Sizinle daha ne kadar kalip size katlanacagim? Oglunu buraya getir."
\par 42 Çocuk daha Isa'ya yaklasirken cin onu yere vurup siddetle sarsti. Ama Isa kötü ruhu azarladi, çocugu iyilestirerek babasina geri verdi.
\par 43 Herkes Tanri'nin büyük gücüne sasip kaldi. Herkes Isa'nin bütün yaptiklari karsisinda hayret içindeyken, Isa ögrencilerine, "Su sözlerime iyice kulak verin" dedi. "Insanoglu*, insanlarin eline teslim edilecek."
\par 45 Onlar bu sözü anlamadilar. Sözü kavramasinlar diye anlami kendilerinden gizlenmisti. Üstelik Isa'ya bu sözle ilgili soru sormaktan korkuyorlardi.
\par 46 Ögrenciler, aralarinda kimin en büyük oldugunu tartismaya basladilar.
\par 47 Akillarindan geçeni bilen Isa, küçük bir çocugu tutup yanina çekti ve onlara söyle dedi: "Bu çocugu benim adim ugruna kabul eden, beni kabul etmis olur. Beni kabul eden de beni göndereni kabul etmis olur. Aranizda en küçük kim ise, iste en büyük odur."
\par 49 Yuhanna buna karsilik, "Efendimiz" dedi, "Senin adinla cin kovan birini gördük, ama bizimle birlikte seni izlemedigi için ona engel olmaya çalistik."
\par 50 Isa, "Ona engel olmayin!" dedi. "Size karsi olmayan, sizden yanadir."
\par 51 Göge alinacagi gün yaklasinca Isa, kararli adimlarla Yerusalim'e dogru yola çikti.
\par 52 Kendi önünden haberciler gönderdi. Bunlar, kendisi için hazirlik yapmak üzere gidip Samiriyeliler'e* ait bir köye girdiler.
\par 53 Ama Samiriyeliler Isa'yi kabul etmediler. Çünkü Yerusalim'e dogru gidiyordu.
\par 54 Ögrencilerden Yakup'la Yuhanna bunu görünce, "Rab, bunlari yok etmek için bir buyrukla gökten ates yagdirmamizi ister misin?" dediler.
\par 55 Ama Isa dönüp onlari azarladi.
\par 56 Sonra baska bir köye gittiler.
\par 57 Yolda giderlerken bir adam Isa'ya, "Nereye gidersen, senin ardindan gelecegim" dedi.
\par 58 Isa ona, "Tilkilerin ini, kuslarin yuvasi var, ama Insanoglu'nun basini yaslayacak bir yeri yok" dedi.
\par 59 Bir baskasina, "Ardimdan gel" dedi. Adam ise, "Izin ver, önce gidip babami gömeyim" dedi.
\par 60 Isa ona söyle dedi: "Birak ölüleri, kendi ölülerini kendileri gömsün. Sen gidip Tanri'nin Egemenligi'ni duyur."
\par 61 Bir baskasi, "Ya Rab" dedi, "Senin ardindan gelecegim ama, izin ver, önce evimdekilerle vedalasayim."
\par 62 Isa ona, "Sabani tutup da geriye bakan, Tanri'nin Egemenligi'ne layik degildir" dedi.

\chapter{10}

\par 1 Bu olaylardan sonra Rab yetmis kisi daha görevlendirdi. Bunlari ikiser ikiser, kendisinin gidecegi her kente, her yere kendi önünden gönderdi.
\par 2 Onlara, "Ürün bol, ama isçi az" dedi, "Bu nedenle ürünün sahibi Rab'be yalvarin, ürününü kaldiracak isçiler göndersin.
\par 3 Haydi gidin! Iste, sizi kuzular gibi kurtlarin arasina gönderiyorum.
\par 4 Yaniniza ne kese, ne torba, ne de çarik alin. Yolda hiç kimseyle selamlasmayin.
\par 5 Hangi eve girerseniz, önce, 'Bu eve esenlik olsun!' deyin.
\par 6 Orada esenliksever biri varsa, dilediginiz esenlik onun üzerinde kalacak; yoksa, size dönecektir.
\par 7 Girdiginiz evde kalin, size ne verirlerse onu yiyip için. Çünkü isçi ücretini hak eder. Evden eve tasinmayin.
\par 8 "Bir kente girdiginizde sizi kabul ederlerse, önünüze konulani yiyin.
\par 9 Orada bulunan hastalari iyilestirin ve kendilerine, 'Tanri'nin Egemenligi size yaklasti' deyin.
\par 10 Ama bir kente girdiginizde sizi kabul etmezlerse, o kentin caddelerine çikip söyle deyin: 'Kentinizden ayaklarimizda kalan tozu bile size karsi silkiyoruz. Yine de sunu bilin ki, Tanri'nin Egemenligi yaklasti.'
\par 12 Size sunu söyleyeyim, yargi günü o kentin hali Sodom Kenti'nin halinden beter olacaktir.
\par 13 "Vay haline, ey Horazin! Vay haline, ey Beytsayda! Sizlerde yapilan mucizeler Sur ve Sayda'da yapilmis olsaydi, çoktan çul* kusanip kül içinde oturarak tövbe etmis olurlardi.
\par 14 Ama yargi günü sizin haliniz Sur ve Sayda'nin halinden beter olacaktir.
\par 15 Ya sen, ey Kefarnahum, göge mi çikarilacaksin? Hayir, ölüler diyarina indirileceksin!
\par 16 "Sizi dinleyen beni dinlemis olur, sizi reddeden beni reddetmis olur. Beni reddeden de beni göndereni reddetmis olur."
\par 17 Yetmisler sevinç içinde döndüler. "Ya Rab" dediler, "Senin adini andigimizda cinler bile bize boyun egiyor."
\par 18 Isa onlara söyle dedi: "Seytan'in gökten yildirim gibi düstügünü gördüm.
\par 19 Ben size, yilanlari ve akrepleri ayak altinda ezmek ve düsmanin bütün gücünü alt etmek için yetki verdim. Hiçbir sey size zarar vermeyecektir.
\par 20 Bununla birlikte, ruhlarin size boyun egmesine sevinmeyin, adlarinizin gökte yazilmis olmasina sevinin."
\par 21 O anda Isa Kutsal Ruh'un etkisiyle cosarak söyle dedi: "Baba, yerin ve gögün Rabbi! Bu gerçekleri bilge ve akilli kisilerden gizleyip küçük çocuklara açtigin için sana sükrederim. Evet Baba, senin istegin buydu.
\par 22 "Babam her seyi bana teslim etti. Ogul'un kim oldugunu Baba'dan baska kimse bilmez. Baba'nin kim oldugunu da Ogul'dan ve Ogul'un O'nu tanitmak istedigi kisilerden baskasi bilmez."
\par 23 Sonra ögrencilerine dönüp özel olarak söyle dedi: "Sizin gördüklerinizi gören gözlere ne mutlu!
\par 24 Size sunu söyleyeyim, nice peygamberler, nice krallar sizin ördüklerinizi görmek istediler, ama göremediler. Sizin isittiklerinizi isitmek istediler, ama isitemediler."
\par 25 Bir Kutsal Yasa uzmani Isa'yi denemek amaciyla gelip söyle dedi: "Ögretmenim, sonsuz yasami miras almak için ne yapmaliyim?"
\par 26 Isa ona, "Kutsal Yasa'da ne yazilmistir?" diye sordu. "Orada ne okuyorsun?"
\par 27 Adam söyle karsilik verdi: "Tanrin Rab'bi bütün yüreginle, bütün caninla, bütün gücünle ve bütün aklinla seveceksin. Komsunu da kendin gibi seveceksin."
\par 28 Isa ona, "Dogru yanit verdin" dedi. "Bunu yap ve yasayacaksin."
\par 29 Oysa adam kendini hakli çikarmak isteyerek Isa'ya, "Peki, komsum kim?"dedi.
\par 30 Isa söyle yanit verdi: "Adamin biri Yerusalim'den Eriha'ya inerken haydutlarin eline düstü. Onu soyup dövdüler, yari ölü birakip gittiler.
\par 31 Bir rastlanti olarak o yoldan bir kâhin geçiyordu. Adami görünce yolun öbür yanindan geçip gitti.
\par 32 Bir Levili* de oraya varip adami görünce ayni sekilde geçip gitti.
\par 33 O yoldan geçen bir Samiriyeli* ise adamin bulundugu yere gelip onu görünce, yüregi sizladi.
\par 34 Adamin yanina gitti, yaralarinin üzerine yagla sarap dökerek sardi. Sonra adami kendi hayvanina bindirip hana götürdü, onunla ilgilendi.
\par 35 Ertesi gün iki dinar çikararak hanciya verdi. 'Ona iyi bak' dedi, 'Bundan fazla ne harcarsan, dönüsümde sana öderim.'
\par 36 "Sence bu üç kisiden hangisi haydutlar arasina düsen adama komsu gibi davrandi?"
\par 37 Yasa uzmani, "Ona aciyip yardim eden" dedi. Isa, "Git, sen de öyle yap" dedi.
\par 38 Isa, ögrencileriyle birlikte yola devam edip bir köye girdi. Marta adinda bir kadin Isa'yi evinde konuk etti.
\par 39 Marta'nin Meryem adindaki kizkardesi, Rab'bin ayaklari dibine oturmus O'nun konusmasini dinliyordu.
\par 40 Marta ise islerinin çoklugundan ötürü telas içindeydi. Isa'nin yanina gelerek, "Ya Rab" dedi, "Kardesimin beni hizmet islerinde yalniz birakmasina aldirmiyor musun? Ona söyle de bana yardim etsin."
\par 41 Rab ona su karsiligi verdi: "Marta, Marta, sen çok sey için kaygilanip telaslaniyorsun.
\par 42 Oysa gerekli olan tek bir sey vardir. Meryem iyi olani seçti ve bu kendisinden alinmayacak."

\chapter{11}

\par 1 Isa bir yerde dua ediyordu. Duasini bitirince ögrencilerinden biri, "Ya Rab"dedi, "Yahya'nin kendi ögrencilerine ögrettigi gibi sen de bize dua etmesini ögret."
\par 2 Isa onlara, "Dua ederken söyle söyleyin" dedi: "Baba, adin kutsal kilinsin. Egemenligin gelsin.
\par 3 Her gün bize gündelik ekmegimizi ver.
\par 4 Günahlarimizi bagisla. Çünkü biz de bize karsi suç isleyen herkesi bagisliyoruz. Ayartilmamiza izin verme."
\par 5 Sonra söyle dedi: "Sizlerden birinin bir arkadasi olur da gece yarisi ona gidip, 'Arkadas, bana üç ekmek ödünç ver. Bir arkadasim yoldan geldi, önüne koyacak bir seyim yok' derse, öbürü içerden, 'Beni rahatsiz etme! Kapi kilitli, çocuklarim da yanimda yatiyor. Kalkip sana bir sey veremem' der mi hiç?
\par 8 Size sunu söyleyeyim, arkadaslik geregi kalkip ona istedigini vermese bile, adamin yüzsüzlügünden ötürü kalkar, ihtiyaci neyse ona verir.
\par 9 "Ben size sunu söyleyeyim: Dileyin, size verilecek; arayin, bulacaksiniz; kapiyi çalin, size açilacaktir.
\par 10 Çünkü her dileyen alir, arayan bulur, kapi çalana açilir.
\par 11 "Aranizda hangi baba, ekmek isteyen ogluna tas verir? Ya da balik isterse balik yerine yilan verir?
\par 12 Ya da yumurta isterse ona akrep verir?
\par 13 Sizler kötü yürekli oldugunuz halde çocuklariniza güzel armaganlar vermeyi biliyorsaniz, gökteki Baba'nin, kendisinden dileyenlere Kutsal Ruh'u verecegi çok daha kesin degil mi?"
\par 14 Isa adamin birinden dilsiz bir cini kovuyordu. Cin çikinca adamin dili çözüldü. Halk hayret içinde kaldi.
\par 15 Ama içlerinden bazilari, "Cinleri, cinlerin önderi Baalzevul'un* gücüyle kovuyor" dediler.
\par 16 Bazilari ise O'nu denemek amaciyla gökten bir belirti göstermesini istediler.
\par 17 Onlarin ne düsündügünü bilen Isa söyle dedi: "Kendi içinde bölünen ülke yikilir, kendi içinde bölünen ev çöker.
\par 18 Seytan da kendi içinde bölünmüsse, onun egemenligi nasil ayakta kalabilir? Siz, benim Baalzevul'un gücüyle cinleri kovdugumu söylüyorsunuz.
\par 19 Eger ben cinleri Baalzevul'un gücüyle kovuyorsam, sizin adamlariniz kimin gücüyle kovuyor? Sizi bu durumda kendi adamlariniz yargilayacak.
\par 20 Ama ben cinleri Tanri'nin eliyle kovuyorsam, Tanri'nin Egemenligi üzerinize gelmis demektir.
\par 21 "Tepeden tirnaga silahlanmis güçlü bir adam kendi evini korudugu sürece, mallari güvenlik içinde olur.
\par 22 Ne var ki, ondan daha güçlü biri saldirip onu alt ettiginde güvendigi bütün silahlari elinden alir ve mallarini yagmalayarak bölüstürür.
\par 23 Benden yana olmayan bana karsidir, benimle birlikte toplamayan dagitiyor demektir.
\par 24 "Kötü ruh insandan çikinca kurak yerlerde dolanip huzur arar. Bulamayinca da, 'Çiktigim eve, kendi evime döneyim' der.
\par 25 Eve gelince orayi süpürülmüs, düzeltilmis bulur.
\par 26 Bunun üzerine gider, kendisinden kötü yedi ruh daha alir ve eve girip yerlesirler. Böylece o kisinin son durumu ilkinden beter olur."
\par 27 Isa bu sözleri söylerken kalabaligin içinden bir kadin O'na, "Ne mutlu seni tasimis olan rahme, emzirmis olan memelere!" diye seslendi.
\par 28 Isa, "Daha dogrusu, ne mutlu Tanri'nin sözünü dinleyip uygulayanlara!" dedi.
\par 29 Çevredeki kalabalik büyürken Isa konusmaya basladi. "Simdiki kusak kötü bir kusaktir" dedi. "Dogaüstü bir belirti istiyor, ama ona Yunus'un belirtisinden baska bir belirti gösterilmeyecek.
\par 30 Yunus nasil Ninova halkina bir belirti olduysa, Insanoglu* da bu kusak için öyle olacaktir.
\par 31 Güney Kraliçesi, yargi günü bu kusagin adamlariyla birlikte kalkip onlari yargilayacak. Çünkü kraliçe, Süleyman'in bilgece sözlerini dinlemek için dünyanin ta öbür ucundan gelmisti. Bakin, Süleyman'dan daha üstün olan buradadir.
\par 32 Ninova halki, yargi günü bu kusakla birlikte kalkip bu kusagi yargilayacak. Çünkü Ninovalilar, Yunus'un çagrisi üzerine tövbe ettiler. Bakin, Yunus'tan daha üstün olan buradadir."
\par 33 "Hiç kimse kandil yakip onu gizli yere ya da tahil ölçeginin altina koymaz. Tersine, içeri girenler isigi görsünler diye onu kandillige koyar.
\par 34 Bedenin isigi gözdür. Gözün saglamsa, bütün bedenin de aydinlik olur. Gözün bozuksa, bedenin de karanlik olur.
\par 35 Öyleyse dikkat et, sendeki 'isik' karanlik olmasin.
\par 36 Eger bütün bedenin aydinlik olur ve hiçbir yani karanlik kalmazsa, kandilin seni isinlariyla aydinlattigi zamanki gibi, bedenin tümden aydinlik olur."
\par 37 Isa konusmasini bitirince bir Ferisi* O'nu evine yemege çagirdi. O da içeri girerek sofraya oturdu.
\par 38 Isa'nin yemekten önce yikanmadigini gören Ferisi sasti.
\par 39 Rab ona söyle dedi: "Siz Ferisiler, bardagin ve tabagin disini temizlersiniz, ama içiniz açgözlülük ve kötülükle doludur.
\par 40 Ey akilsizlar! Disi yapanla içi yapan ayni degil mi?
\par 41 Siz kaplarinizin içindekini sadaka olarak verin, o zaman sizin için her sey temiz olur.
\par 42 "Ama vay halinize, ey Ferisiler! Siz nanenin, sedefotunun ve her tür sebzenin ondaligini verirsiniz de, adaleti ve Tanri sevgisini ihmal edersiniz. Ondalik vermeyi ihmal etmeden esas bunlari yerine getirmeniz gerekirdi.
\par 43 Vay halinize, ey Ferisiler! Havralarda en seçkin yerlere kurulmaya, meydanlarda selamlanmaya bayilirsiniz.
\par 44 Vay halinize! Insanlarin, farkinda olmadan üzerlerinde gezindigi belirsiz mezarlara benziyorsunuz."
\par 45 Kutsal Yasa uzmanlarindan biri söz alip Isa'ya, "Ögretmenim, bunlari söylemekle bize de hakaret etmis oluyorsun" dedi.
\par 46 Isa, "Sizin de vay halinize, ey Yasa uzmanlari!" dedi. "Insanlara tasinmasi güç yükler yüklersiniz, kendiniz ise bu yükleri kaldirmak için parmaginizi bile kipirdatmazsiniz.
\par 47 Vay halinize! Peygamberlerin anitlarini yaparsiniz, oysa onlari sizin atalariniz öldürmüstür.
\par 48 Böylelikle atalarinizin yaptiklarina taniklik ederek bunlari onaylamis oluyorsunuz. Çünkü onlar peygamberleri öldürdüler, siz de anitlarini yapiyorsunuz.
\par 49 Iste bunun için Tanri'nin Bilgeligi söyle demistir: 'Ben onlara peygamberler ve elçiler gönderecegim, bunlardan kimini öldürecek, kimine zulmedecekler.'
\par 50 Böylece bu kusak, Habil'in kanindan tutun da, sunakla tapinak arasinda öldürülen Zekeriya'nin kanina degin, dünyanin kurulusundan beri akitilan bütün peygamberlerin kanindan sorumlu tutulacaktir. Evet, size söylüyorum, bu kusak sorumlu tutulacaktir.
\par 52 Vay halinize, ey Yasa uzmanlari! Bilgi kapisinin anahtarini alip götürdünüz. Kendiniz bu kapidan girmediniz, girmek isteyenlere de engel oldunuz."
\par 53 Isa oradan ayrilinca, din bilginleriyle Ferisiler O'nu siddetle sikistirarak birçok konuda agzini aramaya basladilar.
\par 54 Agzindan çikacak bir sözle O'nu tuzaga düsürmek için firsat kolluyorlardi.

\chapter{12}

\par 1 O sirada halktan binlerce kisi birbirlerini ezercesine toplanmisti. Isa önce kendi ögrencilerine sunlari söylemeye basladi: "Ferisiler'in mayasindan -yani, ikiyüzlülükten- kaçinin.
\par 2 Örtülü olup da açiga çikarilmayacak, gizli olup da bilinmeyecek hiçbir sey yoktur.
\par 3 Bunun için karanlikta söylediginiz her söz gün isiginda duyulacak, kapali kapilar ardinda kulaga fisildadiklariniz damlardan duyurulacaktir.
\par 4 "Siz dostlarima söylüyorum, bedeni öldüren, ama ondan sonra baska bir sey yapamayanlardan korkmayin.
\par 5 Kimden korkmaniz gerektigini size açiklayayim: Kisiyi öldürdükten sonra cehenneme atma yetkisine sahip olan Tanri'dan korkun. Evet, size söylüyorum, O'ndan korkun.
\par 6 Bes serçe iki metelige satilmiyor mu? Ama bunlardan bir teki bile Tanri katinda unutulmus degildir.
\par 7 Nitekim basinizdaki bütün saçlar bile sayilidir. Korkmayin, siz birçok serçeden daha degerlisiniz.
\par 8 "Size sunu söyleyeyim, insanlarin önünde beni açikça kabul eden herkesi, Insanoglu* da Tanri'nin melekleri önünde açikça kabul edecek.
\par 9 Ama kim beni insanlar önünde inkâr ederse, kendisi de Tanri'nin melekleri önünde inkâr edilecek.
\par 10 Insanoglu'na karsi bir söz söyleyen herkes bagislanacak. Oysa Kutsal Ruh'a küfreden bagislanmayacaktir.
\par 11 "Sizi havra topluluklarinin, yöneticilerin ve yetkililerin önüne çikardiklarinda, 'Kendimizi neyle, nasil savunacagiz?' ya da, 'Ne söyleyecegiz?' diye kaygilanmayin.
\par 12 Kutsal Ruh o anda size ne söylemeniz gerektigini ögretecektir."
\par 13 Kalabaligin içinden biri Isa'ya, "Ögretmenim, kardesime söyle de mirasi benimle paylassin" dedi.
\par 14 Isa ona söyle dedi: "Ey adam! Kim beni üzerinizde yargiç ya da hakem yapti?"
\par 15 Sonra onlara, "Dikkatli olun!" dedi. "Her türlü açgözlülükten sakinin. Çünkü insanin yasami, malinin çokluguna bagli degildir."
\par 16 Isa onlara su benzetmeyi anlatti: "Zengin bir adamin topraklari bol ürün verdi.
\par 17 Adam kendi kendine, 'Ne yapacagim? Ürünlerimi koyacak yerim yok' diye düsündü.
\par 18 Sonra, 'Söyle yapacagim' dedi. 'Ambarlarimi yikip daha büyüklerini yapacagim, bütün tahillarimi ve mallarimi oraya yigacagim.
\par 19 Kendime, ey canim, yillarca yetecek kadar bol malin var. Rahatina bak, ye, iç, yasamin tadini çikar diyecegim.'
\par 20 "Ama Tanri ona, 'Ey akilsiz!' dedi. 'Bu gece canin senden istenecek. Biriktirdigin bu seyler kime kalacak?'
\par 21 "Kendisi için servet biriktiren, ama Tanri katinda zengin olmayan kisinin sonu böyle olur."
\par 22 Isa ögrencilerine söyle dedi: "Bu nedenle size sunu söylüyorum: 'Ne yiyecegiz?' diye caniniz için, 'Ne giyecegiz?' diye bedeniniz için kaygilanmayin.
\par 23 Can yiyecekten, beden de giyecekten daha önemlidir.
\par 24 Kargalara bakin! Ne eker, ne biçerler; ne kilerleri, ne ambarlari vardir. Tanri yine de onlari doyurur. Siz kuslardan çok daha degerlisiniz!
\par 25 Hangi biriniz kaygilanmakla ömrünü bir anlik uzatabilir?
\par 26 Bu küçücük ise bile gücünüz yetmedigine göre, öbür konularda neden kaygilaniyorsunuz?
\par 27 "Zambaklarin nasil büyüdügüne bakin! Ne çalisirlar, ne de iplik egirirler. Ama size sunu söyleyeyim, bütün görkemine karsin Süleyman bile bunlardan biri gibi giyinmis degildi.
\par 28 Ey kit imanlilar, bugün var olup yarin ocaga atilacak olan kir otunu böyle giydiren Tanri'nin sizi de giydirecegi çok daha kesindir.
\par 29 'Ne yiyecegiz, ne içecegiz?' diye düsünüp tasalanmayin.
\par 30 Dünya uluslari hep bu seylerin pesinden giderler. Oysa Babaniz, bunlara gereksinmeniz oldugunu bilir.
\par 31 Siz O'nun egemenliginin ardindan gidin, o zaman size bunlar da verilecektir.
\par 32 "Korkma, ey küçük sürü! Çünkü Babaniz, egemenligi size vermeyi uygun gördü.
\par 33 Mallarinizi satin, sadaka olarak verin. Kendinize eskimeyen keseler, göklerde tükenmeyen bir hazine edinin. Orada ne hirsiz ona yaklasir, ne de güve onu yer.
\par 34 Hazineniz neredeyse, yüreginiz de orada olacaktir."
\par 35 "Kusaklariniz belinizde bagli ve kandilleriniz yanar durumda hazir olun.
\par 36 Dügün senliginden dönecek olan efendilerinin gelip kapiyi çaldigi an kapiyi açmak için hazir bekleyen köleler gibi olun.
\par 37 Efendileri geldiginde uyanik bulunan kölelere ne mutlu! Size dogrusunu söyleyeyim, efendileri beline kusagini baglayacak, kölelerini sofraya oturtacak ve gelip onlara hizmet edecek.
\par 38 Efendi gecenin ister ikinci, ister üçüncü nöbetinde gelsin, uyanik bulacagi kölelere ne mutlu!
\par 39 Ama sunu bilin ki, ev sahibi, hirsizin hangi saatte gelecegini bilse, evinin soyulmasina firsat vermez.
\par 40 Siz de hazir olun. Çünkü Insanoglu* beklemediginiz saatte gelecektir."
\par 41 Petrus, "Ya Rab" dedi, "Bu benzetmeyi bizim için mi anlatiyorsun, yoksa herkes için mi?"
\par 42 Rab de söyle dedi: "Efendinin, usaklarina vaktinde azik vermek için baslarina atadigi güvenilir ve akilli kâhya kimdir?
\par 43 Efendisi eve döndügünde isinin basinda bulacagi o köleye ne mutlu!
\par 44 Size gerçegi söyleyeyim, efendisi onu bütün malinin üzerinde yetkili kilacak.
\par 45 Ama o köle içinden, 'Efendim gecikiyor' der, kadin ve erkek hizmetkârlari dövmeye, yiyip içip sarhos olmaya baslarsa, efendisi, onun beklemedigi günde, ummadigi saatte gelecek, onu siddetle cezalandirip imansizlarla bir tutacaktir.
\par 47 "Efendisinin istegini bilip de hazirlik yapmayan, onun istegini yerine getirmeyen köle çok dayak yiyecek.
\par 48 Oysa bilmeden dayagi hak eden davranislarda bulunan, az dayak yiyecek. Kime çok verilmisse, ondan çok istenecek. Kime çok sey emanet edilmisse, kendisinden daha fazlasi istenecektir.
\par 49 "Ben dünyaya ates yagdirmaya geldim. Keske bu ates daha simdiden alevlenmis olsaydi!
\par 50 Katlanmam gereken bir vaftiz var. Bu vaftiz gerçeklesinceye dek nasil da sikinti çekiyorum!
\par 51 Yeryüzüne baris getirmeye mi geldigimi saniyorsunuz? Size hayir diyorum, ayrilik getirmeye geldim.
\par 52 Bundan böyle bir evde bes kisi, ikiye karsi üç, üçe karsi iki bölünmüs olacak.
\par 53 Baba ogluna karsi, ogul babasina karsi, anne kizina karsi, kiz annesine karsi, kaynana gelinine karsi, gelin kaynanasina karsi olacaktir."
\par 54 Isa halka sunlari da söyledi: "Batida bir bulutun yükseldigini görünce siz hemen, 'Saganak geliyor' diyorsunuz, ve öyle oluyor.
\par 55 Rüzgarin güneyden estigini görünce, 'Çok sicak olacak' diyorsunuz, ve öyle oluyor.
\par 56 Sizi ikiyüzlüler! Yeryüzünün ve gökyüzünün görünümünden bir anlam çikarabiliyorsunuz da, simdiki zamanin anlamini nasil oluyor da çikaramiyorsunuz?
\par 57 "Dogru olana neden kendiniz karar vermiyorsunuz?
\par 58 Sizden davaci olanla birlikte yargica giderken, yolda onunla anlasmak için elinizden geleni yapin. Yoksa o sizi yargicin önüne sürükler, yargiç gardiyanin eline verir, gardiyan da sizi hapse atar.
\par 59 Size sunu söyleyeyim, borcunuzun son kurusunu ödemedikçe oradan asla çikamazsiniz."

\chapter{13}

\par 1 O sirada bazi kisiler gelip Isa'ya bir haber getirdiler. Pilatus'un nasil bazi Celileliler'i öldürüp kanlarini kendi kestikleri kurbanlarin kanina kattigini anlattilar.
\par 2 Isa onlara söyle karsilik verdi: "Böyle aci çeken bu Celileliler'in, bütün öbür Celileliler'den daha günahli oldugunu mu saniyorsunuz?
\par 3 Size hayir diyorum. Ama tövbe etmezseniz, hepiniz böyle mahvolacaksiniz.
\par 4 Ya da, Siloah'taki kule üzerlerine yikilinca ölen o on sekiz kisinin, Yerusalim'de yasayan öbür insanlarin hepsinden daha suçlu oldugunu mu saniyorsunuz?
\par 5 Size hayir diyorum. Ama tövbe etmezseniz, hepiniz böyle mahvolacaksiniz."
\par 6 Isa su benzetmeyi anlatti: "Adamin birinin baginda dikili bir incir agaci vardi. Adam gelip agaçta meyve aradi, ama bulamadi.
\par 7 Bagciya, 'Bak' dedi, 'Ben üç yildir gelip bu incir agacinda meyve ariyorum, bulamiyorum. Onu kes. Topragin besinini neden bos yere tüketsin?'
\par 8 "Bagci, 'Efendim' diye karsilik verdi, 'Agaci bir yil daha birak, bu arada ben çevresini kazip gübreleyeyim.
\par 9 Gelecek yil meyve verirse, ne iyi; vermezse, onu kesersin.'" Beli Bükük Bir Kadinin Iyilestirilmesi
\par 10 Bir Sabat Günü* Isa, havralardan birinde ögretiyordu.
\par 11 On sekiz yildir içinde hastalik ruhu bulunan bir kadin da oradaydi. Iki büklüm olmus, belini hiç dogrultamiyordu.
\par 12 Isa onu görünce yanina çagirdi. "Kadin" dedi, "Hastaligindan kurtuldun."
\par 13 Ellerini kadinin üzerine koydu. Kadin hemen dogruldu ve Tanri'yi yüceltmeye basladi.
\par 14 Isa'nin hastayi Sabat Günü iyilestirmesine kizan havra yöneticisi kalabaliga seslenerek, "Çalismak için alti gün vardir" dedi. "O günler gelip iyilesin, Sabat Günü degil."
\par 15 Rab ona su karsiligi verdi: "Sizi ikiyüzlüler! Her biriniz Sabat Günü kendi öküzünü ya da esegini yemlikten çözüp suya götürmez mi?
\par 16 Buna göre, Seytan'in on sekiz yildir bagli tuttugu, Ibrahim'in bir kizi olan bu kadinin da Sabat Günü bu bagdan çözülmesi gerekmez miydi?"
\par 17 Isa'nin bu sözleri, kendisine karsi gelenlerin hepsini utandirdi. Bütün kalabalik ise O'nun yaptigi görkemli islerin tümünü sevinçle karsiladi.
\par 18 Sonra Isa sunlari söyledi: "Tanri'nin Egemenligi neye benzer, onu neye benzeteyim?
\par 19 Tanri'nin Egemenligi, bir adamin bahçesine ektigi hardal tanesine benzer. Tane gelisip agaç olur, kuslar dallarinda barinir."
\par 20 Isa yine, "Tanri'nin Egemenligi'ni neye benzeteyim?" dedi.
\par 21 "O, bir kadinin üç ölçek una karistirdigi mayaya benzer. Sonunda bütün hamur kabarir."
\par 22 Isa köy kent dolasarak ögretiyor, Yerusalim'e dogru ilerliyordu.
\par 23 Biri O'na, "Ya Rab" dedi, "Kurtulanlarin sayisi az mi olacak?" Isa oradakilere söyle dedi: "Dar kapidan girmeye gayret edin. Size sunu söyleyeyim, çok kisi içeri girmek isteyecek, ama giremeyecek.
\par 25 Ev sahibi kalkip kapiyi kapattiktan sonra disarida durup, 'Ya Rab, kapiyi aç bize!' diyerek kapiyi vurmaya baslayacaksiniz. "O da size, 'Kim oldugunuzu, nereden geldiginizi bilmiyorum' diye karsilik verecek.
\par 26 "O zaman, 'Biz senin önünde yiyip içtik, sen de bizim sokaklarimizda ögrettin' demeye baslayacaksiniz.
\par 27 "O da size söyle diyecek: 'Kim oldugunuzu, nereden geldiginizi bilmiyorum. Çekilin önümden, ey kötülük yapanlar!'
\par 28 "Ibrahim'i, Ishak'i, Yakup'u ve bütün peygamberleri Tanri'nin Egemenligi'nde, kendinizi ise disari atilmis gördügünüz zaman, aranizda aglayis ve dis gicirtisi olacaktir.
\par 29 Insanlar dogudan batidan, kuzeyden güneyden gelecek ve Tanri'nin Egemenigi'nde sofraya oturacaklar.
\par 30 Ve iste, sonuncu olan bazilari birinci, birinci olan bazilari da sonuncu olacak."
\par 31 Tam o sirada bazi Ferisiler gelip Isa'ya, "Buradan ayrilip baska yere git. Hirodes* seni öldürmek istiyor" dediler.
\par 32 Isa onlara söyle dedi: "Gidin, o tilkiye söyleyin, 'Bugün ve yarin cinleri kovup hastalari iyilestirecegim ve üçüncü gün hedefime ulasacagim.'
\par 33 Yine de bugün, yarin ve öbür gün yoluma devam etmeliyim. Çünkü bir peygamberin Yerusalim'in disinda ölmesi düsünülemez!
\par 34 "Ey Yerusalim! Peygamberleri öldüren, kendisine gönderilenleri taslayan Yerusalim! Tavugun civcivlerini kanatlari altina topladigi gibi ben de kaç kez senin çocuklarini toplamak istedim, ama siz istemediniz.
\par 35 Bakin, eviniz issiz birakilacak! Size sunu söyleyeyim: 'Rab'bin adiyla gelene övgüler olsun!' diyeceginiz zamana dek beni bir daha görmeyeceksiniz."

\chapter{14}

\par 1 Bir Sabat Günü* Isa Ferisiler'in ileri gelenlerinden birinin evine yemek yemeye gitti. Herkes O'nu dikkatle gözlüyordu.
\par 2 Önünde, vücudu su toplamis bir adam vardi.
\par 3 Isa, Kutsal Yasa uzmanlarina ve Ferisiler'e, "Sabat Günü bir hastayi iyilestirmek Kutsal Yasa'ya uygun mudur, degil midir?" diye sordu.
\par 4 Onlar ses çikarmadilar. Isa adami tutup iyilestirdi, sonra eve gönderdi.
\par 5 Isa onlara söyle dedi: "Hanginiz oglu ya da öküzü Sabat Günü kuyuya düser de hemen çikarmaz?"
\par 6 Onlar buna hiçbir karsilik veremediler.
\par 7 Yemege çagrilanlarin basköseleri seçtigini farkeden Isa, onlara su benzetmeyi anlatti: "Biri seni dügüne çagirdigi zaman basköseye kurulma. Belki senden daha saygin birini de çagirmistir. Ikinizi de çagiran gelip, 'Yerini bu adama ver' diyebilir. O zaman utançla kalkip en arkaya geçersin.
\par 10 Bir yere çagrildigin zaman git, en arkada otur. Öyle ki, seni çagiran gelince, 'Arkadasim, daha öne buyurmaz misin?' desin. O zaman seninle birlikte sofrada oturan herkesin önünde onurlandirilmis olursun.
\par 11 Kendini yücelten herkes alçaltilacak, kendini alçaltan yüceltilecektir."
\par 12 Isa kendisini yemege çagirmis olana da söyle dedi: "Bir öglen ya da aksam yemegi verdigin zaman dostlarini, kardeslerini, akrabalarini ve zengin komsularini çagirma. Yoksa onlar da seni çagirarak karsilik verirler.
\par 13 Ama ziyafet verdigin zaman yoksullari, kötürümleri, sakatlari, körleri çagir.
\par 14 Böylece mutlu olursun. Çünkü bunlar sana karsilik verecek durumda degildirler. Karsiligi sana, dogru kisiler dirildigi zaman verilecektir."
\par 15 Sofrada oturanlardan biri bunu duyunca Isa'ya, "Tanri'nin Egemenligi'nde yemek yiyecek olana ne mutlu!" dedi.
\par 16 Isa ona söyle dedi: "Adamin biri büyük bir sölen hazirlayip birçok konuk çagirdi.
\par 17 Sölen saati gelince davetlilere, 'Buyurun, her sey hazir' diye haber vermek üzere kölesini gönderdi.
\par 18 "Ne var ki, hepsi anlasmisçasina özür dilemeye basladilar. Birincisi, 'Bir tarla satin aldim, gidip görmek zorundayim. Rica ederim, beni hos gör' dedi.
\par 19 "Bir baskasi, 'Bes çift öküz aldim, onlari denemeye gidiyorum. Rica ederim, beni hos gör' dedi.
\par 20 "Yine bir baskasi, 'Yeni evlendim, bu nedenle gelemiyorum' dedi.
\par 21 "Köle geri dönüp durumu efendisine bildirdi. Bunun üzerine ev sahibi öfkelenerek kölesine, 'Kos' dedi, 'Kentin caddelerine, sokaklarina çik; yoksullari, kötürümleri, körleri, sakatlari buraya getir.'
\par 22 "Köle, 'Efendim, buyrugun yerine getirilmistir, ama daha yer var' dedi.
\par 23 "Efendisi köleye, 'Çikip yollari ve çit boylarini dolas, bulduklarini gelmeye zorla da evim dolsun' dedi.
\par 24 'Size sunu söyleyeyim, ilk çagrilan o adamlardan hiçbiri benim yemegimden tatmayacaktir.'"
\par 25 Kalabalik halk topluluklari Isa'yla birlikte yol aliyordu. Isa dönüp onlara söyle dedi: "Biri bana gelip de babasini, annesini, karisini, çocuklarini, kardeslerini, hatta kendi canini bile gözden çikarmazsa, ögrencim olamaz.
\par 27 Çarmihini yüklenip ardimdan gelmeyen, ögrencim olamaz.
\par 28 "Aranizdan biri bir kule yapmak isterse, bunu tamamlayacak kadar parasi var mi yok mu diye önce oturup yapacagi masrafi hesap etmez mi?
\par 29 Çünkü temel atip da isi bitiremezse, durumu gören herkes, 'Bu adam insaata basladi, ama bitiremedi' diyerek onunla eglenmeye baslar.
\par 31 "Ya da hangi kral baska bir kralla savasa gittiginde, üzerine yirmi bin askerle yürüyen düsmana on bin askerle karsi koyabilir miyim diye önce oturup bir degerlendirme yapmaz?
\par 32 Eger karsi koyamayacaksa, öbürü henüz uzaktayken elçiler gönderip baris kosullarini ister.
\par 33 Ayni sekilde sizden kim varini yogunu gözden çikarmazsa, ögrencim olamaz.
\par 34 "Tuz yararlidir. Ama tuz tadini yitirirse, bir daha nasil o tadi kazanabilir?
\par 35 Ne topraga, ne de gübreye yarar; onu çöpe atarlar. Isitecek kulagi olan isitsin."

\chapter{15}

\par 1 Bütün vergi görevlileriyle* günahkârlar Isa'yi dinlemek için O'na akin ediyordu.
\par 2 Ferisiler'le din bilginleri ise, "Bu adam günahkârlari kabul ediyor, onlarla birlikte yemek yiyor" diye söyleniyorlardi.
\par 3 Bunun üzerine Isa onlara su benzetmeyi anlatti: "Sizlerden birinin yüz koyunu olsa ve bunlardan bir tanesini kaybetse, doksan dokuzu bozkirda birakarak kaybolani bulana dek onun ardina düsmez mi?
\par 5 Onu bulunca da sevinç içinde omuzlarina alir, evine döner; arkadaslarini, komsularini çagirip onlara, 'Benimle birlikte sevinin, kaybolan koyunumu buldum!' der.
\par 7 Size sunu söyleyeyim, ayni sekilde gökte, tövbe eden tek bir günahkâr için, tövbeyi gereksinmeyen doksan dokuz dogru kisi için duyulandan daha büyük sevinç duyulacaktir."
\par 8 "Ya da on gümüs parasi olan bir kadin bunlardan bir tanesini kaybetse, kandil yakip evi süpürerek parayi bulana dek her tarafi dikkatle aramaz mi?
\par 9 Parayi bulunca da arkadaslarini, komsularini çagirip, 'Benimle birlikte sevinin, kaybettigim parayi buldum!' der.
\par 10 Size sunu söyleyeyim, ayni sekilde Tanri'nin melekleri de tövbe eden bir tek günahkâr için sevinç duyacaklar."
\par 11 Isa, "Bir adamin iki oglu vardi" dedi.
\par 12 "Bunlardan küçügü babasina, 'Baba' dedi, 'Malindan payima düseni ver bana.' Baba da servetini iki oglu arasinda paylastirdi.
\par 13 "Bundan birkaç gün sonra küçük ogul her seyini toplayip uzak bir ülkeye gitti. Orada sefahat içinde bir yasam sürerek varini yogunu çarçur etti.
\par 14 Delikanli her seyini harcadiktan sonra, o ülkede siddetli bir kitlik bas gösterdi, o da yokluk çekmeye basladi.
\par 15 Bunun üzerine gidip o ülkenin vatandaslarindan birinin hizmetine girdi. Adam onu, domuz gütmek üzere otlaklarina yolladi.
\par 16 Delikanli, domuzlarin yedigi keçiboynuzlariyla karnini doyurmaya can atiyordu. Ama hiç kimse ona bir sey vermedi.
\par 17 "Akli basina gelince söyle dedi: 'Babamin nice isçisinin fazlasiyla yiyecegi var, bense burada açliktan ölüyorum.
\par 18 Kalkip babamin yanina dönecegim, ona, Baba diyecegim, Tanri'ya ve sana karsi günah isledim.
\par 19 Ben artik senin oglun olarak anilmaya layik degilim. Beni isçilerinden biri gibi kabul et.'
\par 20 "Böylece kalkip babasinin yanina döndü. Kendisi daha uzaktayken babasi onu gördü, ona acidi, kosup boynuna sarildi ve onu öptü.
\par 21 Oglu ona, 'Baba' dedi, 'Tanri'ya ve sana karsi günah isledim. Ben artik senin oglun olarak anilmaya layik degilim.'
\par 22 "Babasi ise kölelerine, 'Çabuk, en iyi kaftani getirip ona giydirin!' dedi. 'Parmagina yüzük takin, ayaklarina çarik giydirin!
\par 23 Besili danayi getirip kesin, yiyelim, eglenelim.
\par 24 Çünkü benim bu oglum ölmüstü, yasama döndü; kaybolmustu, bulundu.' Böylece eglenmeye basladilar.
\par 25 "Babanin büyük oglu ise tarladaydi. Gelip eve yaklastiginda çalgi ve oyun seslerini duydu.
\par 26 Usaklardan birini yanina çagirip, 'Ne oluyor?' diye sordu.
\par 27 "O da, 'Kardesin geldi, baban da ona sag salim kavustugu için besili danayi kesti' dedi.
\par 28 "Büyük ogul öfkelendi, içeri girmek istemedi. Babasi disari çikip ona yalvardi. Ama o, babasina söyle yanit verdi: 'Bak, bunca yil senin için köle gibi çalistim, hiçbir zaman buyrugundan çikmadim. Ne var ki sen bana, arkadaslarimla eglenmem için hiçbir zaman bir oglak bile vermedin.
\par 30 Oysa senin malini fahiselerle yiyen su oglun eve dönünce, onun için besili danayi kestin.'
\par 31 "Babasi ona, 'Oglum, sen her zaman yanimdasin, neyim varsa senindir' dedi.
\par 32 'Ama sevinip eglenmek gerekiyordu. Çünkü bu kardesin ölmüstü, yasama döndü; kaybolmustu, bulundu!'"

\chapter{16}

\par 1 Isa ögrencilerine sunlari da anlatti: "Zengin bir adamin bir kâhyasi vardi. Kâhya, efendisinin mallarini çarçur ediyor diye efendisine ihbar edildi.
\par 2 Efendisi kâhyayi çagirip ona, 'Nedir bu senin hakkinda duyduklarim? Kâhyaliginin hesabini ver. Çünkü sen artik kâhyalik edemezsin' dedi.
\par 3 "Kâhya kendi kendine, 'Ne yapacagim ben?' dedi. 'Efendim kâhyaligi elimden aliyor. Toprak kazmaya gücüm yetmez, dilenmekten utanirim.
\par 4 Kâhyaliktan kovuldugum zaman baskalari beni evlerine kabul etsinler diye ne yapacagimi biliyorum.'
\par 5 "Böylelikle efendisine borcu olanlarin hepsini tek tek yanina çagirdi. Birincisine, 'Efendime ne kadar borcun var?' dedi.
\par 6 "Adam, 'Yüz ölçek zeytinyagi' karsiligini verdi. "Kâhya ona, 'Borç senedini al ve hemen otur, elli ölçek diye yaz' dedi.
\par 7 "Sonra bir baskasina, 'Senin borcun ne kadar?' dedi. "'Yüz ölçek bugday' dedi öteki. "Ona da, 'Borç senedini al, seksen ölçek diye yaz' dedi.
\par 8 "Efendisi, dürüst olmayan kâhyayi, akillica davrandigi için övdü. Gerçekten bu çagin insanlari, kendilerine benzer kisilerle iliskilerinde, isikta yürüyenlerden daha akilli oluyorlar.
\par 9 Size sunu söyleyeyim, dünyanin aldatici servetini kendinize dost edinmek için kullanin ki, bu servet yok olunca sizi sonsuza dek kalacak konutlara Kabul etsinler."
\par 10 "En küçük iste güvenilir olan kisi, büyük iste de güvenilir olur. En küçük iste dürüst olmayan kisi, büyük iste de dürüst olmaz.
\par 11 Dünyanin aldatici serveti konusunda güvenilir degilseniz, gerçek serveti size kim emanet eder?
\par 12 Baskasinin mali konusunda güvenilir degilseniz, kendi maliniz olmak üzere size kim bir sey verir?
\par 13 "Hiçbir usak iki efendiye kulluk edemez. Ya birinden nefret edip öbürünü sever, ya da birine baglanip öbürünü hor görür. Siz hem Tanri'ya, hem paraya kulluk edemezsiniz."
\par 14 Parayi seven Ferisiler bütün bu sözleri duyunca Isa'yla alay etmeye basladilar.
\par 15 O da onlara söyle dedi: "Siz insanlar önünde kendinizi temize çikariyorsunuz, ama Tanri yüreginizi biliyor. Insanlarin gururlandiklari ne varsa, Tanri'ya igrenç gelir.
\par 16 "Kutsal Yasa ve peygamberlerin devri Yahya'nin zamanina dek sürdü. O zamandan bu yana Tanri'nin Egemenligi müjdeleniyor ve herkes oraya zorla girmeye çalisiyor.
\par 17 Yerin ve gögün ortadan kalkmasi, Kutsal Yasa'nin ufacik bir noktasinin yok olmasindan daha kolaydir.
\par 18 "Karisini bosayip baskasiyla evlenen zina etmis olur. Kocasindan bosanmis bir kadinla evlenen de zina etmis olur."
\par 19 "Zengin bir adam vardi. Mor, ince keten giysiler giyer, bolluk içinde her gün eglenirdi.
\par 20 Her tarafi yara içinde olan Lazar adinda yoksul bir adam bu zenginin kapisinin önüne birakilirdi; zenginin sofrasindan düsen kirintilarla karnini doyurmaya can atardi. Bir yandan da köpekler gelip onun yaralarini yalardi.
\par 22 "Bir gün yoksul adam öldü, melekler onu alip Ibrahim'in yanina götürdüler. Sonra zengin adam da öldü ve gömüldü.
\par 23 Ölüler diyarinda istirap çeken zengin adam basini kaldirip uzakta Ibrahim'i ve onun yaninda Lazar'i gördü.
\par 24 'Ey babamiz Ibrahim, aci bana!' diye seslendi. 'Lazar'i gönder de parmaginin ucunu suya batirip dilimi serinletsin. Bu alevlerin içinde azap çekiyorum.'
\par 25 "Ibrahim, 'Oglum' dedi, 'Yasamin boyunca senin iyilik payini, Lazar'in da kötülük payini aldigini unutma. Simdiyse o burada teselli ediliyor, sen de azap çekiyorsun.
\par 26 Üstelik, aramiza öyle bir uçurum kondu ki, ne buradan size gelmek isteyenler gelebilir, ne de oradan kimse bize gelebilir.'
\par 27 "Zengin adam söyle dedi: 'Öyleyse baba, sana rica ederim, Lazar'i babamin evine gönder.
\par 28 Çünkü bes kardesim var. Lazar onlari uyarsin ki, onlar da bu istirap yerine düsmesinler.'
\par 29 "Ibrahim, 'Onlarda Musa'nin ve peygamberlerin sözleri var, onlari dinlesinler' dedi.
\par 30 "Zengin adam, 'Hayir, Ibrahim baba, dinlemezler!' dedi. 'Ancak ölüler arasindan biri onlara giderse, tövbe ederler.'
\par 31 "Ibrahim ona, 'Eger Musa ile peygamberleri dinlemezlerse, ölüler arasindan biri dirilse bile ikna olmazlar' dedi."

\chapter{17}

\par 1 Isa ögrencilerine söyle dedi: "Insani günaha düsüren tuzaklarin olmasi kaçinilmazdir. Ama bu tuzaklara aracilik eden kisinin vay haline!
\par 2 Böyle bir kisi bu küçüklerden birini günaha düsürecegine, boynuna bir degirmen tasi geçirilip denize atilsa, kendisi için daha iyi olur.
\par 3 Yasantiniza dikkat edin! Kardesiniz günah islerse, onu azarlayin; tövbe ederse, bagislayin.
\par 4 Günde yedi kez size karsi günah isler ve yedi kez size gelip, 'Tövbe ediyorum' derse, onu bagislayin."
\par 5 Elçiler Rab'be, "Imanimizi artir!" dediler.
\par 6 Rab söyle dedi: "Bir hardal tanesi kadar imaniniz olsa, su dut agacina, 'Kökünden sökül ve denizin içine dikil' dersiniz, o da sözünüzü dinler.
\par 7 "Hanginizin çift süren ya da çobanlik eden bir kölesi olur da, tarladan dönüsünde ona, 'Çabuk gel, sofraya otur' der?
\par 8 Tersine ona, 'Yemegimi hazirla, kusagini bagla, ben yiyip içerken bana hizmet et. Sonra sen yiyip içersin' demez mi?
\par 9 Verdigi buyruklari yerine getirdi diye köleye tesekkür eder mi?
\par 10 Siz de böylece, size verilen buyruklarin hepsini yerine getirdikten sonra, 'Biz degersiz kullariz; sadece yapmamiz gerekeni yaptik' deyin."
\par 11 Yerusalim'e dogru yoluna devam eden Isa, Samiriye ile Celile arasindaki sinir bölgesinden geçiyordu.
\par 12 Köyün birine girerken O'nu cüzamli* on adam karsiladi. Bunlar uzakta durarak, "Isa, Efendimiz, halimize aci!" diye seslendiler.
\par 14 Isa onlari görünce, "Gidin, kâhinlere* görünün" dedi. Adamlar yolda giderken cüzamdan temizlendiler.
\par 15 Onlardan biri, iyilestigini görünce yüksek sesle Tanri'yi yücelterek geri döndü, yüzüstü Isa'nin ayaklarina kapanip O'na tesekkür etti. Bu adam Samiriyeli'ydi*.
\par 17 Isa, "Iyilesenler on kisi degil miydi?" diye sordu. "Öbür dokuzu nerede?
\par 18 Tanri'yi yüceltmek için bu yabancidan baska geri dönen olmadi mi?"
\par 19 Sonra adama, "Ayaga kalk, git" dedi. "Imanin seni kurtardi."
\par 20 Ferisiler Isa'ya, "Tanri'nin Egemenligi ne zaman gelecek?" diye sordular. Isa onlara söyle yanit verdi: "Tanri'nin Egemenligi göze görünür bir sekilde gelmez.
\par 21 Insanlar da, 'Iste burada' ya da, 'Iste surada' demeyecekler. Çünkü Tanri'nin Egemenligi içinizdedir."
\par 22 Isa ögrencilerine söyle dedi: "Öyle günler gelecek ki, Insanoglu'nun* günlerinden birini görmeyi özleyeceksiniz, ama görmeyeceksiniz.
\par 23 Insanlar size, 'Iste orada', 'Iste burada' diyecekler. Gitmeyin, onlarin arkasindan kosmayin.
\par 24 Simsek çakip gögü bir ucundan öbür ucuna dek nasil aydinlatirsa, Insanoglu kendi gününde öyle olacaktir.
\par 25 Ama önce O'nun çok aci çekmesi ve bu kusak tarafindan reddedilmesi gerekir.
\par 26 "Nuh'un günlerinde nasil olduysa, Insanoglu'nun günlerinde de öyle olacak.
\par 27 Nuh'un gemiye bindigi güne dek insanlar yiyip içiyor, evlenip evlendiriliyorlardi. Sonra tufan gelip hepsini yok etti.
\par 28 Lut'un günlerinde de durum ayniydi. Insanlar yiyip içiyor, alip satiyor, tohum ekiyor, ev yapiyorlardi.
\par 29 Ama Lut'un Sodom'dan ayrildigi gün gökten atesle kükürt yagdi ve hepsini yok etti.
\par 30 "Insanoglu'nun ortaya çikacagi gün durum ayni olacaktir.
\par 31 O gün damda olan, evdeki esyalarini almak için asagi inmesin. Tarlada olan da geri dönmesin.
\par 32 Lut'un karisini hatirlayin!
\par 33 Canini esirgemek isteyen onu yitirecek. Canini yitiren ise onu yasatacaktir.
\par 34 Size sunu söyleyeyim, o gece ayni yatakta olan iki kisiden biri alinacak, öbürü birakilacak.
\par 35 Birlikte bugday ögüten iki kadindan biri alinacak, öbürü birakilacak."
\par 37 Onlar Isa'ya, "Bu olaylar nerede olacak, Rab?" diye sordular. O da onlara, "Les neredeyse, akbabalar da oraya üsüsecek" dedi.

\chapter{18}

\par 1 Isa ögrencilerine, hiç usanmadan, her zaman dua etmeleri gerektigini belirten su benzetmeyi anlatti: "Kentin birinde Tanri'dan korkmayan, insana saygi duymayan bir yargiç vardi.
\par 3 Yine o kentte bir dul kadin vardi. Yargica sürekli gidip, 'Davaci oldugum kisiden hakkimi al' diyordu.
\par 4 "Yargiç bir süre ilgisiz kaldi. Ama sonunda kendi kendine, 'Ben her ne kadar Tanri'dan korkmaz, insana saygi duymazsam da, bu dul kadin beni rahatsiz ettigi için hakkini alacagim. Yoksa sürekli gelip beni canimdan bezdirecek' dedi."
\par 6 Rab söyle devam etti: "Adaletsiz yargicin ne söyledigini duydunuz.
\par 7 Tanri da, gece gündüz kendisine yakaran seçilmislerinin hakkini almayacak mi? Onlari çok bekletecek mi?
\par 8 Size sunu söyleyeyim, onlarin hakkini tez alacaktir. Ama Insanoglu* geldigi zaman acaba yeryüzünde iman bulacak mi?"
\par 9 Kendi dogruluklarina güvenip baskalarina tepeden bakan bazi kisilere Isa su benzetmeyi anlatti: "Biri Ferisi*, öbürü vergi görevlisi* iki kisi dua etmek üzere tapinaga çikti.
\par 11 Ferisi ayakta kendi kendine söyle dua etti: 'Tanrim, öbür insanlara -soygunculara, hak yiyenlere, zina edenlere- ya da su vergi görevlisine benzemedigim için sana sükrederim.
\par 12 Haftada iki gün oruç tutuyor, bütün kazancimin ondaligini veriyorum.'
\par 13 "Vergi görevlisi ise uzakta durdu, gözlerini göge kaldirmak bile istemiyordu, ancak gögsünü döverek, 'Tanrim, ben günahkâra merhamet et' diyordu.
\par 14 "Size sunu söyleyeyim, Ferisi degil, bu adam aklanmis olarak evine döndü. Çünkü kendini yücelten herkes alçaltilacak, kendini alçaltan ise yüceltilecektir."
\par 15 Bazilari bebekleri bile Isa'ya getiriyor, onlara dokunmasini istiyorlardi. Bunu gören ögrenciler onlari azarladilar.
\par 16 Ama Isa çocuklari yanina çagirarak, "Birakin, çocuklar bana gelsin, onlara engel olmayin!" dedi. "Çünkü Tanri'nin Egemenligi böylelerinindir.
\par 17 Size dogrusunu söyleyeyim, Tanri'nin Egemenligi'ni bir çocuk gibi kabul etmeyen, bu egemenlige asla giremez."
\par 18 Ileri gelenlerden biri Isa'ya, "Iyi ögretmenim, sonsuz yasama kavusmak için ne yapmaliyim?" diye sordu.
\par 19 Isa, "Bana neden iyi diyorsun?" dedi. "Iyi olan yalniz biri var, O da Tanri'dir.
\par 20 O'nun buyruklarini biliyorsun: 'Zina etmeyeceksin, adam öldürmeyeceksin, çalmayacaksin, yalan yere taniklik etmeyeceksin, annene babana saygi göstereceksin.'"
\par 21 "Bunlarin hepsini gençligimden beri yerine getiriyorum" dedi adam.
\par 22 Isa bunu duyunca ona, "Hâlâ bir eksigin var" dedi. "Neyin varsa hepsini sat, parasini yoksullara dagit; böylece göklerde hazinen olur. Sonra gel, beni izle."
\par 23 Adam bu sözleri duyunca çok üzüldü. Çünkü son derece zengindi.
\par 24 Onun üzüntüsünü gören Isa, "Varlikli kisilerin Tanri Egemenligi'ne girmesi ne kadar güç!" dedi.
\par 25 "Nitekim devenin igne deliginden geçmesi, zenginin Tanri Egemenligi'ne girmesinden daha kolaydir."
\par 26 Bunu isitenler, "Öyleyse kim kurtulabilir?" dediler.
\par 27 Isa, "Insanlar için imkânsiz olan, Tanri için mümkündür" dedi.
\par 28 Petrus, "Bak, biz her seyimizi birakip senin ardindan geldik" dedi.
\par 29 Isa onlara söyle dedi: "Size dogrusunu söyleyeyim, Tanri'nin Egemenligi ugruna evini, karisini, kardeslerini, annesiyle babasini ya da çocuklarini birakip da bu çagda bunlarin kat kat fazlasina ve gelecek çagda sonsuz yasama kavusmayacak hiç kimse yoktur."
\par 31 Isa, Onikiler'i* bir yana çekip onlara söyle dedi: "Simdi Yerusalim'e gidiyoruz. Peygamberlerin Insanoglu'yla* ilgili yazdiklarinin tümü yerine gelecektir.
\par 32 O, öteki uluslara teslim edilecek. O'nunla alay edecek, O'na hakaret edecekler; üzerine tükürecek ve O'nu kamçilayip öldürecekler. Ne var ki O, üçüncü gün dirilecek."
\par 34 Ögrenciler bu sözlerden hiçbir sey anlamadilar. Bu sözlerin anlami onlardan gizlenmisti, anlatilanlari kavrayamiyorlardi.
\par 35 Isa Eriha'ya yaklasirken kör bir adam yol kenarinda oturmus dileniyordu.
\par 36 Adam oradan geçen kalabaligi duyunca, "Ne oluyor?" diye sordu.
\par 37 Ona, "Nasirali Isa geçiyor" dediler.
\par 38 O da, "Ey Davut Oglu Isa, halime aci!" diye bagirdi.
\par 39 Önden gidenler onu azarlayarak susturmak istedilerse de o, "Ey Davut Oglu, halime aci!" diyerek daha çok bagirdi.
\par 40 Isa durup adamin kendisine getirilmesini buyurdu. Adam yaklasinca Isa, "Senin için ne yapmami istiyorsun?" diye sordu. O da, "Ya Rab, gözlerim görsün" dedi.
\par 42 Isa, "Gözlerin görsün" dedi. "Imanin seni kurtardi."
\par 43 Adam o anda yeniden görmeye basladi ve Tanri'yi yücelterek Isa'nin ardindan gitti. Bunu gören bütün halk Tanri'ya övgüler sundu.

\chapter{19}

\par 1 Isa Eriha'ya girdi. Kentin içinden geçiyordu.
\par 2 Orada vergi görevlilerinin* basi olan, Zakkay adinda zengin bir adam vardi.
\par 3 Isa'nin kim oldugunu görmek istiyor, ama boyu kisa oldugu için kalabaliktan ötürü göremiyordu.
\par 4 Isa'yi görebilmek için önden kosup bir yabanil incir agacina tirmandi. Çünkü Isa oradan geçecekti.
\par 5 Isa oraya varinca yukari bakip, "Zakkay, çabuk asagi in!" dedi. "Bugün senin evinde kalmam gerekiyor."
\par 6 Zakkay hizla asagi indi ve sevinç içinde Isa'yi evine buyur etti.
\par 7 Bunu görenlerin hepsi söylenmeye basladi: "Gidip günahkâr birine konuk oldu!" dediler.
\par 8 Zakkay ayaga kalkip Rab'be söyle dedi: "Ya Rab, iste malimin yarisini yoksullara veriyorum. Bir kimseden haksizlikla bir sey aldimsa, dört katini geri verecegim."
\par 9 Isa dedi ki, "Bu ev bugün kurtulusa kavustu. Çünkü bu adam da Ibrahim'in ogludur.
\par 10 Nitekim Insanoglu*, kaybolani arayip kurtarmak için geldi."
\par 11 Oradakiler bu sözleri dinlerken Isa konusmasini bir benzetmeyle sürdürdü. Çünkü Yerusalim'e yaklasmisti ve onlar, Tanri'nin Egemenligi'nin hemen ortaya çikacagini saniyorlardi.
\par 12 Bu nedenle Isa söyle dedi: "Soylu bir adam, kral atanip dönmek üzere uzak bir ülkeye gitti.
\par 13 Gitmeden önce kölelerinden onunu çagirip onlara birer mina verdi. 'Ben dönünceye dek bu paralari isletin' dedi.
\par 14 "Ne var ki, ülkesinin halki adamdan nefret ediyordu. Arkasindan temsilciler göndererek, 'Bu adamin üzerimize kral olmasini istemiyoruz' diye haber ilettiler.
\par 15 "Adam kral atanmis olarak geri döndügünde, parayi vermis oldugu köleleri çagirtip ne kazandiklarini ögrenmek istedi.
\par 16 Birincisi geldi, 'Efendimiz' dedi, 'Senin bir minan on mina daha kazandi.'
\par 17 "Efendisi ona, 'Aferin, iyi köle!' dedi. 'En küçük iste güvenilir oldugunu gösterdigin için on kent üzerinde yetkili olacaksin.'
\par 18 "Ikincisi gelip, 'Efendimiz, senin bir minan bes mina daha kazandi' dedi.
\par 19 "Efendisi ona da, 'Sen bes kent üzerinde yetkili olacaksin' dedi.
\par 20 "Baska biri geldi, 'Efendimiz' dedi, 'Iste senin minan! Onu bir mendile sarip sakladim.
\par 21 Çünkü senden korktum, sert adamsin; kendinden koymadigini alir, ekmedigini biçersin.'
\par 22 "Efendisi ona, 'Ey kötü köle, seni kendi agzindan çikan sözle yargilayacagim' dedi. 'Kendinden koymadigini alan, ekmedigini biçen sert bir adam oldugumu bildigine göre,
\par 23 neden parami faize vermedin? Ben de geldigimde onu faiziyle geri alirdim.'
\par 24 "Sonra çevrede duranlara, 'Elindeki minayi alin, on minasi olana verin' dedi.
\par 25 "Ona, 'Efendimiz' dediler, 'Onun zaten on minasi var!'
\par 26 "O da, 'Size sunu söyleyeyim, kimde varsa ona daha çok verilecek. Ama kimde yoksa, kendisinde olan da elinden alinacak' dedi.
\par 27 'Beni kral olarak istemeyen o düsmanlarima gelince, onlari buraya getirin ve gözümün önünde kiliçtan geçirin!'"
\par 28 Isa, bu sözleri söyledikten sonra önden yürüyerek Yerusalim'e dogru ilerledi.
\par 29 Zeytin Dagi'nin yamacindaki Beytfaci ile Beytanya'ya yaklastiginda iki ögrencisini önden gönderdi. Onlara, "Karsidaki köye gidin" dedi, "Köye girince, üzerine daha hiç kimsenin binmedigi, bagli duran bir sipa bulacaksiniz. Onu çözüp bana getirin.
\par 31 Biri size, 'Onu niçin çözüyorsunuz?' diye sorarsa, 'Rab'bin ona ihtiyaci var' dersiniz."
\par 32 Gönderilen ögrenciler gittiler, her seyi Isa'nin kendilerine anlattigi gibi buldular.
\par 33 Sipayi çözerlerken hayvanin sahipleri onlara, "Sipayi niye çözüyorsunuz?" dediler.
\par 34 Onlar da, "Rab'bin ona ihtiyaci var" karsiligini verdiler.
\par 35 Sipayi Isa'ya getirdiler, üzerine kendi giysilerini atarak Isa'yi üstüne bindirdiler.
\par 36 Isa ilerlerken halk, giysilerini yola seriyordu.
\par 37 Isa Zeytin Dagi'ndan asagi inen yola yaklastigi sirada, ögrencilerinden olusan kalabaligin tümü, görmüs olduklari bütün mucizelerden ötürü, sevinç içinde yüksek sesle Tanri'yi övmeye basladilar.
\par 38 "Rab'bin adiyla gelen Kral'a övgüler olsun! Gökte esenlik, en yücelerde yücelik olsun!" diyorlardi.
\par 39 Kalabaligin içinden bazi Ferisiler O'na, "Ögretmen, ögrencilerini sustur!" dediler.
\par 40 Isa, "Size sunu söyleyeyim, bunlar susacak olsa, taslar bagiracaktir!" diye karsilik verdi.
\par 41 Isa Yerusalim'e yaklasip kenti görünce agladi.
\par 42 "Keske bugün sen de esenlige giden yolu bilseydin" dedi. "Ama simdilik bu senin gözlerinden gizlendi.
\par 43 Senin için öyle günler gelecek ki, düsmanlarin seni setlerle çevirecek, kusatip her yandan sikistiracaklar.
\par 44 Seni de, bagrindaki çocuklari da yere çalacaklar. Sende tas üstünde tas birakmayacaklar. Çünkü Tanri'nin senin yardimina geldigi zamani farketmedin."
\par 45 Sonra Isa tapinagin avlusuna girerek saticilari disari kovmaya basladi.
\par 46 Onlara, "'Evim dua evi olacak' diye yazilmistir. Ama siz onu haydut inine çevirdiniz" dedi.
\par 47 Isa her gün tapinakta ögretiyordu. Baskâhinler, din bilginleri ve halkin ileri gelenleri ise O'nu yok etmek istiyor, ama bunu nasil yapacaklarini bilemiyorlardi. Çünkü bütün halk O'nu can kulagiyla dinliyordu.

\chapter{20}

\par 1 O günlerden birinde, Isa tapinakta halka ögretip Müjde'yi duyururken, baskâhinler ve din bilginleri*, ileri gelenlerle birlikte çikageldiler.
\par 2 O'na, "Söyle bize, bunlari hangi yetkiyle yapiyorsun? Bu yetkiyi sana kim verdi?" diye sordular.
\par 3 Isa onlara su karsiligi verdi: "Ben de size bir soru soracagim. Söyleyin bana, Yahya'nin vaftiz* etme yetkisi Tanri'dan miydi, insanlardan mi?"
\par 5 Bunu aralarinda söyle tartistilar: "'Tanri'dan' dersek, 'Ona niçin inanmadiniz?' diyecek.
\par 6 Yok eger 'Insanlardan' dersek, bütün halk bizi tasa tutacak. Çünkü Yahya'nin peygamber olduguna inanmislardir."
\par 7 Sonunda, "Nereden oldugunu bilmiyoruz" yanitini verdiler.
\par 8 Isa da onlara, "Ben de size bunlari hangi yetkiyle yaptigimi söylemeyecegim" dedi.
\par 9 Isa sözüne devam ederek halka su benzetmeyi anlatti: "Adamin biri bag dikti, bunu bagcilara kiralayip uzun süre yolculuga çikti.
\par 10 Mevsimi gelince, bagin ürününden payina düseni vermeleri için bagcilara bir köle yolladi. Ama bagcilar köleyi dövüp eli bos gönderdiler.
\par 11 Bag sahibi baska bir köle daha yolladi. Bagcilar onu da dövdüler, asagilayip eli bos gönderdiler.
\par 12 Adam bir üçüncüsünü yolladi, bagcilar onu da yaralayip kovdular.
\par 13 "Bagin sahibi, 'Ne yapacagim?' dedi. 'Sevgili oglumu göndereyim. Belki onu sayarlar.'
\par 14 "Ama bagcilar onu görünce aralarinda söyle konustular: 'Mirasçi budur; onu öldürelim de miras bize kalsin.'
\par 15 Böylece, onu bagdan disari atip öldürdüler. "Bu durumda bagin sahibi onlara ne yapacak?
\par 16 Gelip o bagcilari yok edecek, bagi da baskalarina verecek." Halk bunu duyunca, "Tanri korusun!" dedi.
\par 17 Isa gözlerinin içine bakarak söyle dedi: "Öyleyse Kutsal Yazilar'daki su sözün anlami nedir? 'Yapicilarin reddettigi tas, Iste kösenin bas tasi oldu.'
\par 18 O tasin üzerine düsen herkes paramparça olacak, tas da kimin üzerine düserse onu ezip toz edecek."
\par 19 Isa'nin bu benzetmeyi kendilerine karsi anlattigini farkeden din bilginleriyle baskâhinler O'nu o anda yakalamak istediler, ama halkin tepkisinden korktular.
\par 20 Isa'yi dikkatle gözlüyorlardi. O'na, kendilerine dürüst süsü veren muhbirler gönderdiler. O'nu, söyleyecegi bir sözle tuzaga düsürmek ve böylelikle valinin yetki ve yargisina teslim etmek istiyorlardi.
\par 21 Muhbirler O'na, "Ögretmenimiz, senin dogru olani söyleyip ögrettigini, insanlar arasinda ayrim yapmaksizin Tanri yolunu dürüstçe ögrettigini biliyoruz. Sezar'a* vergi vermemiz Kutsal Yasa'ya uygun mu, degil mi?" diye sordular.
\par 23 Onlarin hilesini anlayan Isa, "Bana bir dinar gösterin" dedi. "Üzerindeki resim ve yazi kimin?" "Sezar'in" dediler.
\par 25 O da, "Öyleyse Sezar'in hakkini Sezar'a, Tanri'nin hakkini Tanri'ya verin" dedi.
\par 26 Isa'yi, halkin önünde söyledigi sözlerle tuzaga düsüremediler. Verdigi yanita sasarak susup kaldilar.
\par 27 Ölümden sonra dirilisi yadsiyan Sadukiler'den bazilari Isa'ya gelip sunu sordular: "Ögretmenimiz, Musa yazilarinda bize söyle buyurmustur: 'Eger bir adamin evli kardesi çocuksuz ölürse, adam ölenin karisini alip soyunu sürdürsün.'
\par 29 Yedi kardes vardi. Birincisi kendine bir es aldi, ama çocuksuz öldü.
\par 30 Ikincisi de, üçüncüsü de kadini aldi; böylece kardeslerin yedisi de çocuk birakmadan öldü.
\par 32 Son olarak kadin da öldü.
\par 33 Buna göre, dirilis günü kadin bunlardan hangisinin karisi olacak? Çünkü yedisi de onunla evlendi."
\par 34 Isa onlara söyle dedi: "Bu çagin insanlari evlenip evlendirilirler.
\par 35 Ama gelecek çaga ve ölülerin dirilisine erismeye layik görülenler ne evlenir, ne evlendirilir.
\par 36 Bir daha ölmeleri de söz konusu degildir. Çünkü meleklere benzerler ve dirilisin çocuklari olarak Tanri'nin çocuklaridirlar.
\par 37 Musa bile alevlenen çaliyla ilgili bölümde Rab için, 'Ibrahim'in Tanrisi, Ishak'in Tanrisi ve Yakup'un Tanrisi' deyimini kullanarak ölülerin dirilecegine isaret etmisti.
\par 38 Tanri ölülerin degil, dirilerin Tanrisi'dir. Çünkü O'na göre bütün insanlar yasamaktadir."
\par 39 Artik O'na baska soru sormaya cesaret edemeyen din bilginlerinden bazilari, "Ögretmenimiz, güzel konustun" dediler.
\par 41 Isa onlara söyle dedi: "Nasil oluyor da, 'Mesih* Davut'un Oglu'dur' diyorlar?
\par 42 Çünkü Davut'un kendisi Mezmurlar Kitabi'nda söyle diyor: Rab Rabbim'e dedi ki, Ben düsmanlarini Ayaklarinin altina serinceye dek Sagimda otur.'
\par 44 Davut O'ndan 'Rab' diye söz ettigine göre, O nasil Davut'un Oglu olur?"
\par 45 Bütün halk dinlerken Isa ögrencilerine söyle dedi: "Uzun kaftanlar içinde dolasmaktan hoslanan, meydanlarda selamlanmaya, havralarda en seçkin yerlere, sölenlerde basköselere kurulmaya bayilan din bilginlerinden sakinin.
\par 47 Dul kadinlarin malini mülkünü sömüren, gösteris için uzun uzun dua eden bu kisilerin cezasi daha agir olacaktir."

\chapter{21}

\par 1 Isa basini kaldirdi ve bagis toplanan yerde bagislarini birakan zenginleri gördü.
\par 2 Yoksul bir dul kadinin oraya iki bakir para attigini görünce, "Size gerçegi söyleyeyim" dedi, "Bu yoksul dul kadin herkesten daha çok verdi.
\par 4 Çünkü bunlarin hepsi kutuya, zenginliklerinden artani attilar. Bu kadin ise yoksulluguna karsin, geçinmek için elinde ne varsa hepsini verdi."
\par 5 Bazi kisiler tapinagin nasil güzel taslar ve adaklarla süslenmis oldugundan söz edince Isa, "Burada gördüklerinize gelince, öyle günler gelecek ki, tas üstünde tas kalmayacak, hepsi yikilacak!" dedi.
\par 7 Onlar da, "Peki, ögretmenimiz, bu dediklerin ne zaman olacak? Bunlarin gerçeklesmek üzere oldugunu gösteren belirti ne olacak?" diye sordular.
\par 8 Isa, "Sakin sizi saptirmasinlar" dedi. "Birçoklari, 'Ben O'yum' ve 'Zaman yaklasti' diyerek benim adimla gelecekler. Onlarin ardindan gitmeyin.
\par 9 Savas ve isyan haberleri duyunca telaslanmayin. Önce bunlarin olmasi gerek, ama son hemen gelmeyecek."
\par 10 Sonra onlara söyle dedi: "Ulus ulusa, devlet devlete savas açacak.
\par 11 Siddetli depremler, yer yer kitliklar ve salgin hastaliklar, korkunç olaylar ve gökte olaganüstü belirtiler olacak.
\par 12 "Ama bütün bu olaylardan önce sizi yakalayip zulmedecekler. Sizi havralara teslim edecek, zindanlara atacaklar. Benim adimdan ötürü krallarin, valilerin önüne çikarilacaksiniz.
\par 13 Bu size taniklik etme firsati olacak.
\par 14 Buna göre kendinizi nasil savunacaginizi önceden düsünmemekte kararli olun.
\par 15 Çünkü ben size öyle bir konusma yetenegi, öyle bir bilgelik verecegim ki, size karsi çikanlarin hiçbiri buna karsi direnemeyecek, bir sey diyemeyecek.
\par 16 Anne babaniz, kardesleriniz, akraba ve dostlariniz bile sizi ele verecek ve bazilarinizi öldürtecekler.
\par 17 Benim adimdan ötürü herkes sizden nefret edecek.
\par 18 Ne var ki, basinizdaki saçlardan bir tel bile yok olmayacaktir.
\par 19 Dayanmakla canlarinizi kazanacaksiniz.
\par 20 "Yerusalim'in ordular tarafindan kusatildigini görünce bilin ki, kentin yikilacagi zaman yaklasmistir.
\par 21 O zaman Yahudiye'de bulunanlar daglara kaçsin, kentte olanlar disari çiksin, kirdakiler kente dönmesin.
\par 22 Çünkü o günler, yazilmis olanlarin tümünün gerçeklesecegi ceza günleridir.
\par 23 O günlerde gebe olan, çocuk emziren kadinlarin vay haline! Çünkü ülke büyük sikintiya düsecek ve bu halk gazaba ugrayacaktir.
\par 24 Kiliçtan geçirilecek, tutsak olarak bütün uluslar arasina sürülecekler. Yerusalim, öteki uluslarin dönemleri tamamlanincaya dek onlarin ayaklari altinda çignenecektir.
\par 25 "Güneste, ayda ve yildizlarda belirtiler görülecek. Yeryüzünde uluslar denizin ve dalgalarin ugultusundan saskina dönecek, dehsete düsecekler.
\par 26 Dünyanin üzerine gelecek felaketleri bekleyen insanlar korkudan bayilacak. Çünkü göksel güçler sarsilacak.
\par 27 O zaman Insanoglu'nun* bulut içinde büyük güç ve görkemle geldigini görecekler.
\par 28 Bu olaylar gerçeklesmeye baslayinca dogrulun ve baslarinizi kaldirin. Çünkü kurtulusunuz yakin demektir."
\par 29 Isa onlara su benzetmeyi anlatti: "Incir agacina ya da herhangi bir agaca bakin.
\par 30 Bunlarin yapraklandigini gördügünüz zaman yaz mevsiminin yakin oldugunu kendiliginizden anlarsiniz.
\par 31 Ayni sekilde, bu olaylarin gerçeklestigini gördügünüzde bilin ki, Tanri'nin Egemenligi yakindir.
\par 32 Size dogrusunu söyleyeyim, bütün bunlar olmadan, bu kusak ortadan kalkmayacak.
\par 33 Yer ve gök ortadan kalkacak, ama benim sözlerim asla ortadan kalkmayacaktir.
\par 34 "Kendinize dikkat edin! Yürekleriniz sefahat, sarhosluk ve bu yasamin kaygilariyla agirlasmasin. O gün, üzerinize bir tuzak gibi aniden inmesin. Çünkü o gün bütün yeryüzünde yasayan herkesin üzerine gelecektir.
\par 36 Her an uyanik kalin, gerçeklesmek üzere olan bütün bu olaylardan kurtulabilmek ve Insanoglu'nun önünde durabilmek için dua edin."
\par 37 Isa gündüz tapinakta ögretiyor, geceleri ise kentten disari çikip Zeytin Dagi'nda sabahliyordu.
\par 38 Sabah erkenden bütün halk O'nu tapinakta dinlemek için O'na akin ediyordu.

\chapter{22}

\par 1 Fisih* denilen Mayasiz Ekmek Bayrami* yaklasmisti.
\par 2 Baskâhinlerle din bilginleri Isa'yi ortadan kaldirmak için bir yol ariyor, ama halktan korkuyorlardi.
\par 3 Seytan, Onikiler'den* biri olup Iskariot diye adlandirilan Yahuda'nin yüregine girdi.
\par 4 Yahuda gitti, baskâhinler ve tapinak koruyucularinin komutanlariyla Isa'yi nasil ele verebilecegini görüstü.
\par 5 Onlar buna sevindiler ve kendisine para vermeye razi oldular.
\par 6 Bunu kabul eden Yahuda, kalabaligin olmadigi bir zamanda Isa'yi ele vermek için firsat kollamaya basladi.
\par 7 Fisih* kurbaninin kesilmesi gereken Mayasiz Ekmek Günü geldi.
\par 8 Isa, Petrus'la Yuhanna'yi, "Gidin, Fisih yemegini yiyebilmemiz için hazirlik yapin" diyerek önden gönderdi.
\par 9 O'na, "Nerede hazirlik yapmamizi istersin?" diye sordular.
\par 10 Isa onlara, "Bakin" dedi, "Kente girdiginizde karsiniza su testisi tasiyan bir adam çikacak. Adami, gidecegi eve kadar izleyin ve evin sahibine söyle deyin: 'Ögretmen, ögrencilerimle birlikte Fisih yemegini yiyecegim konuk odasi nerede? diye soruyor.'
\par 12 Ev sahibi size üst katta, dösenmis büyük bir oda gösterecek. Orada hazirlik yapin."
\par 13 Onlar da gittiler, her seyi Isa'nin kendilerine söyledigi gibi buldular ve Fisih yemegi için hazirlik yaptilar.
\par 14 Yemek saati gelince Isa, elçileriyle birlikte sofraya oturdu ve onlara söyle dedi: "Ben aci çekmeden önce bu Fisih yemegini sizinle birlikte yemeyi çok arzulamistim.
\par 16 Size sunu söyleyeyim, Fisih yemegini, Tanri'nin Egemenligi'nde yetkinlige erisecegi zamana dek, bir daha yemeyecegim."
\par 17 Sonra kâseyi alarak sükretti ve, "Bunu alin, aranizda paylasin" dedi.
\par 18 "Size sunu söyleyeyim, Tanri'nin Egemenligi gelene dek, asmanin ürününden bir daha içmeyecegim."
\par 19 Sonra eline ekmek aldi, sükredip ekmegi böldü ve onlara verdi. "Bu sizin ugrunuza feda edilen bedenimdir. Beni anmak için böyle yapin" dedi.
\par 20 Ayni sekilde, yemekten sonra kâseyi alip söyle dedi: "Bu kâse, sizin ugrunuza akitilan kanimla gerçeklesen yeni antlasmadir.
\par 21 Ama bana ihanet edecek kisinin eli su anda benimkiyle birlikte sofradadir.
\par 22 Insanoglu*, belirlenmis olan yoldan gidiyor. Ama O'na ihanet eden adamin vay haline!"
\par 23 Elçiler, aralarinda bunu kimin yapabilecegini tartismaya basladilar.
\par 24 Ayrica aralarinda hangisinin en üstün sayilacagi konusunda bir çekisme oldu.
\par 25 Isa onlara, "Uluslarin krallari, kendi uluslarina egemen kesilirler. Ileri gelenleri de kendilerine iyiliksever unvanini yakistirirlar" dedi.
\par 26 "Ama siz böyle olmayacaksiniz. Aranizda en büyük olan, en küçük gibi olsun; yöneten, hizmet eden gibi olsun.
\par 27 Hangisi daha büyük, sofrada oturan mi, hizmet eden mi? Sofrada oturan degil mi? Oysa ben aranizda hizmet eden biri gibi oldum.
\par 28 Denendigim zamanlar benimle birlikte dayanmis olanlar sizlersiniz.
\par 29 Babam bana nasil bir egemenlik verdiyse, ben de size bir egemenlik veriyorum.
\par 30 Öyle ki, egemenligimde benim soframda yiyip içesiniz ve tahtta oturarak Israil'in on iki oymagini yargilayasiniz.
\par 31 "Simun, Simun, Seytan sizleri bugday gibi kalburdan geçirmek için izin almistir.
\par 32 Ama ben, imanini yitirmeyesin diye senin için dua ettim. Geri döndügün zaman kardeslerini güçlendir."
\par 33 Simun Isa'ya, "Ya Rab, ben seninle birlikte zindana da, ölüme de gitmeye hazirim" dedi.
\par 34 Isa, "Sana sunu söyleyeyim, Petrus, bu gece horoz ötmeden beni tanidigini üç kez inkâr edeceksin" dedi.
\par 35 Sonra Isa onlara, "Ben sizi kesesiz, torbasiz ve çariksiz gönderdigim zaman, herhangi bir eksiginiz oldu mu?" diye sordu. "Hiçbir eksigimiz olmadi" dediler.
\par 36 O da onlara, "Simdi ise kesesi olan da, torbasi olan da yanina alsin" dedi. "Kilici olmayan, abasini satip bir kiliç alsin.
\par 37 Size sunu söyleyeyim, yazilmis olan su sözün yasamimda yerine gelmesi gerekiyor: 'O, suçlularla bir sayildi.' Gerçekten de benimle ilgili yazilmis olanlar yerine gelmektedir."
\par 38 "Ya Rab, iste burada iki kiliç var" dediler. O da onlara, "Yeter!" dedi.
\par 39 Isa disari çikti, her zamanki gibi Zeytin Dagi'na gitti. Ögrenciler de O'nun ardindan gittiler.
\par 40 Oraya varinca Isa onlara, "Dua edin ki ayartilmayasiniz" dedi.
\par 41 Onlardan bir tas atimi kadar uzaklasti ve diz çökerek söyle dua etti: "Baba, senin istegine uygunsa, bu kâseyi* benden uzaklastir. Yine de benim degil, senin istedigin olsun."
\par 43 Gökten bir melek Isa'ya görünerek O'nu güçlendirdi.
\par 44 Derin bir aci içinde olan Isa daha hararetle dua etti. Teri, topraga düsen kan damlalarini andiriyordu.
\par 45 Isa duadan kalkip ögrencilerin yanina dönünce onlari üzüntüden uyumus buldu.
\par 46 Onlara, "Niçin uyuyorsunuz?" dedi. "Kalkip dua edin ki ayartilmayasiniz."
\par 47 Isa daha konusurken bir kalabalik çikageldi. Onikiler'den* biri, Yahuda adindaki kisi, kalabaliga öncülük ediyordu. Isa'yi öpmek üzere yaklasinca Isa, "Yahuda" dedi, "Insanoglu'na* bir öpücükle mi ihanet ediyorsun?"
\par 49 Isa'nin çevresindekiler olacaklari anlayinca, "Ya Rab, kiliçla vuralim mi?" dediler.
\par 50 Içlerinden biri baskâhinin kölesine vurarak sag kulagini uçurdu.
\par 51 Ama Isa, "Birakin, yeter!" dedi, sonra kölenin kulagina dokunarak onu iyilestirdi.
\par 52 Isa, üzerine yürüyen baskâhinlere, tapinak koruyucularinin komutanlarina ve ileri gelenlere söyle dedi: "Niçin bir haydutmusum gibi kiliç ve sopalarla geldiniz?
\par 53 Her gün tapinakta sizinle birlikteydim, bana el sürmediniz. Ama bu saat sizindir, karanligin egemen oldugu saattir."
\par 54 Isa'yi tutukladilar, alip baskâhinin evine götürdüler. Petrus onlari uzaktan izliyordu.
\par 55 Avlunun ortasinda ates yakip çevresinde oturduklarinda Petrus da gelip onlarla birlikte oturdu.
\par 56 Bir hizmetçi kiz atesin isiginda oturan Petrus'u gördü. Onu dikkatle süzerek, "Bu da O'nunla birlikteydi" dedi.
\par 57 Ama Petrus, "Ben O'nu tanimiyorum, kadin!" diye inkâr etti.
\par 58 Biraz sonra onu gören baska biri, "Sen de onlardansin" dedi. Petrus, "Degilim, arkadas!" dedi.
\par 59 Yaklasik bir saat sonra yine bir baskasi israrla, "Gerçekten bu da O'nunla birlikteydi" dedi. "Çünkü Celileli'dir."
\par 60 Petrus, "Sen ne diyorsun be adam, anlamiyorum!" dedi. Tam o anda, Petrus daha konusurken horoz öttü.
\par 61 Rab arkasina dönüp Petrus'a bakti. O zaman Petrus, Rab'bin kendisine, "Bu gece horoz ötmeden beni üç kez inkâr edeceksin" dedigini hatirladi ve disari çikip aci aci agladi.
\par 63 Isa'yi göz altinda tutan adamlar O'nunla alay ediyor, O'nu dövüyorlardi.
\par 64 Gözlerini baglayip, "Peygamberligini göster bakalim, sana vuran kim?" diye soruyorlardi.
\par 65 Kendisine daha bir sürü küfür yagdirdilar.
\par 66 Gün dogunca halkin ileri gelenleri, baskâhinler ve din bilginleri toplandilar. Isa, bunlardan olusan Yüksek Kurul'un* önüne çikarildi.
\par 67 O'na, "Sen Mesih* isen, söyle bize" dediler. Isa onlara söyle dedi: "Size söylesem, inanmazsiniz.
\par 68 Size soru sorsam, yanit vermezsiniz.
\par 69 Ne var ki, bundan böyle Insanoglu*, kudretli Tanri'nin saginda oturacaktir."
\par 70 Onlarin hepsi, "Yani, sen Tanri'nin Oglu musun?" diye sordular. O da onlara, "Söylediginiz gibi, ben O'yum" dedi.
\par 71 "Artik tanikliga ne ihtiyacimiz var?" dediler. "Iste kendi agzindan duyduk!"

\chapter{23}

\par 1 Sonra bütün kurul üyeleri kalkip Isa'yi Pilatus'a götürdüler.
\par 2 O'nu söyle suçlamaya basladilar: "Bu adamin ulusumuzu yoldan saptirdigini gördük. Sezar'a* vergi ödenmesine engel oluyor, kendisinin de Mesih*, yani bir kral oldugunu söylüyor."
\par 3 Pilatus Isa'ya, "Sen Yahudiler'in Krali misin?" diye sordu. Isa, "Söyledigin gibidir" yanitini verdi.
\par 4 Pilatus, baskâhinlerle halka, "Bu adamda hiçbir suç görmüyorum" dedi.
\par 5 Ama onlar üstelediler: "Yahudiye'nin her tarafinda ögretisini yayarak halki kiskirtiyor; Celile'den baslayip ta buraya kadar geldi" dediler.
\par 6 Pilatus bunu duyunca, "Bu adam Celileli mi?" diye sordu.
\par 7 Isa'nin, Hirodes'in* yönetimindeki bölgeden geldigini ögrenince, kendisini o sirada Yerusalim'de bulunan Hirodes'e gönderdi.
\par 8 Hirodes Isa'yi görünce çok sevindi. O'na iliskin haberleri duydugu için çoktandir O'nu görmek istiyor, gerçeklestirecegi bir belirtiye tanik olmayi umuyordu.
\par 9 O'na birçok soru sordu, ama O hiç karsilik vermedi.
\par 10 Orada duran baskâhinlerle din bilginleri, Isa'yi agir bir dille suçladilar.
\par 11 Hirodes de askerleriyle birlikte O'nu asagilayip alay etti. O'na gösterisli bir kaftan giydirip Pilatus'a geri gönderdi.
\par 12 Bu olaydan önce birbirine düsman olan Hirodes'le Pilatus, o gün dost oldular.
\par 13 Pilatus, baskâhinleri, yöneticileri ve halki toplayarak onlara, "Siz bu adami bana, halki saptiriyor diye getirdiniz" dedi. "Oysa ben bu adami sizin önünüzde sorguya çektim ve kendisinde öne sürdügünüz suçlardan hiçbirini bulmadim.
\par 15 Hirodes de bulmamis olmali ki, O'nu bize geri gönderdi. Görüyorsunuz, ölüm cezasini gerektiren hiçbir sey yapmadi.
\par 16 Bu nedenle ben O'nu dövdürüp saliverecegim."
\par 18 Ama onlar hep bir agizdan, "Yok et bu adami, bize Barabba'yi saliver!" diye bagirdilar.
\par 19 Barabba, kentte çikan bir ayaklanmaya katilmaktan ve adam öldürmekten hapse atilmisti.
\par 20 Isa'yi salivermek isteyen Pilatus onlara yeniden seslendi.
\par 21 Onlar ise, "O'nu çarmiha ger, çarmiha ger!" diye bagrisip durdular.
\par 22 Pilatus üçüncü kez, "Bu adam ne kötülük yapti ki?" dedi. "Ölüm cezasini gerektirecek hiçbir suç bulmadim O'nda. Bu nedenle O'nu dövdürüp saliverecegim."
\par 23 Ne var ki onlar, yüksek sesle bagrisarak Isa'nin çarmiha gerilmesi için direttiler. Sonunda bagirislari baskin çikti ve Pilatus, onlarin isteginin yerine getirilmesine karar verdi.
\par 25 Istedikleri kisiyi, ayaklanmaya katilmak ve adam öldürmekten hapse atilan kisiyi saliverdi. Isa'yi ise onlarin istegine birakti.
\par 26 Askerler Isa'yi götürürken, kirdan gelmekte olan Simun adinda Kireneli bir adami yakaladilar, çarmihi sirtina yükleyip Isa'nin arkasindan yürüttüler.
\par 27 Büyük bir halk toplulugu da Isa'nin ardindan gidiyordu. Aralarinda Isa için dövünüp agit yakan kadinlar vardi.
\par 28 Isa bu kadinlara dönerek, "Ey Yerusalim kizlari, benim için aglamayin" dedi. "Kendiniz ve çocuklariniz için aglayin.
\par 29 Çünkü öyle günler gelecek ki, 'Kisir kadinlara, hiç dogurmamis rahimlere, emzirmemis memelere ne mutlu!' diyecekler.
\par 30 O zaman daglara, 'Üzerimize düsün!' ve tepelere, 'Bizi örtün!' diyecekler.
\par 31 Çünkü yas agaca böyle yaparlarsa, kuruya neler olacaktir?"
\par 32 Isa'yla birlikte idam edilmek üzere ayrica iki suçlu da götürülüyordu.
\par 33 Kafatasi denilen yere vardiklarinda Isa'yi, biri saginda öbürü solunda olmak üzere, iki suçluyla birlikte çarmiha gerdiler.
\par 34 Isa, "Baba, onlari bagisla" dedi. "Çünkü ne yaptiklarini bilmiyorlar." O'nun giysilerini aralarinda paylasmak için kura çektiler.
\par 35 Halk orada durmus, olanlari seyrediyordu. Yöneticiler Isa'yla alay ederek, "Baskalarini kurtardi; eger Tanri'nin Mesihi*, Tanri'nin seçtigi O ise, kendini de kurtarsin" diyorlardi.
\par 36 Askerler de yaklasip Isa'yla eglendiler. O'na eksi sarap sunarak, "Sen Yahudiler'in Krali'ysan, kurtar kendini!" dediler.
\par 38 Basinin üzerinde su yafta vardi: YAHUDILER'IN KRALI BUDUR
\par 39 Çarmiha asilan suçlulardan biri, "Sen Mesih degil misin? Haydi, kendini de bizi de kurtar!" diye küfür etti.
\par 40 Ne var ki, öbür suçlu onu azarladi. "Sende Tanri korkusu da mi yok?" diye karsilik verdi. "Sen de ayni cezayi çekiyorsun.
\par 41 Nitekim biz hakli olarak cezalandiriliyor, yaptiklarimizin karsiligini aliyoruz. Oysa bu adam hiçbir kötülük yapmadi."
\par 42 Sonra, "Ey Isa, kendi egemenligine girdiginde beni an" dedi.
\par 43 Isa ona, "Sana dogrusunu söyleyeyim, sen bugün benimle birlikte cennette olacaksin" dedi.
\par 44 Ögleyin on iki sularinda günes karardi, üçe kadar bütün ülkenin üzerine karanlik çöktü. Tapinaktaki perde* ortasindan yirtildi.
\par 46 Isa yüksek sesle, "Baba, ruhumu ellerine birakiyorum!" diye seslendi. Bunu söyledikten sonra son nefesini verdi.
\par 47 Olanlari gören yüzbasi, "Bu adam gerçekten dogru biriydi" diyerek Tanri'yi yüceltmeye basladi.
\par 48 Olayi seyretmek için biriken halkin tümü olup bitenleri görünce gögüslerini döve döve geri döndüler.
\par 49 Ama Isa'nin bütün tanidiklari ve Celile'den O'nun ardindan gelen kadinlar uzakta durmus, olanlari seyrediyorlardi.
\par 50 Yüksek Kurul* üyelerinden Yusuf adinda iyi ve dogru bir adam vardi.
\par 51 Bir Yahudi kenti olan Aramatya'dan olup Tanri'nin Egemenligi'ni umutla bekleyen Yusuf, Kurul'un kararini ve eylemini onaylamamisti.
\par 52 Pilatus'a gidip Isa'nin cesedini istedi.
\par 53 Cesedi çarmihtan indirip keten beze sardi, hiç kimsenin konulmadigi, kayaya oyulmus bir mezara yatirdi.
\par 54 Hazirlik Günü'ydü* ve Sabat Günü* baslamak üzereydi.
\par 55 Isa'yla birlikte Celile'den gelen kadinlar da Yusuf'un ardindan giderek mezari ve Isa'nin cesedinin oraya nasil konuldugunu gördüler.
\par 56 Evlerine dönerek baharat ve güzel kokulu yaglar hazirladilar. Ama Sabat Günü, Tanri'nin buyrugu uyarinca dinlendiler.

\chapter{24}

\par 1 Kadinlar haftanin ilk günü*, sabah çok erkenden, hazirlamis olduklari baharati alip mezara gittiler.
\par 2 Tasi mezarin girisinden yuvarlanmis buldular.
\par 3 Ama içeri girince Rab Isa'nin cesedini bulamadilar.
\par 4 Onlar bu durum karsisinda sasirip kalmisken, simsek gibi parildayan giysilere bürünmüs iki kisi yanlarinda belirdi.
\par 5 Korkuya kapilan kadinlar baslarini yere egdiler. Adamlar ise onlara, "Diri olani neden ölüler arasinda ariyorsunuz?" dediler.
\par 6 "O burada yok, dirildi. Daha Celile'deyken size söyledigini animsayin.
\par 7 Insanoglu'nun* günahli insanlarin eline verilmesi, çarmiha gerilmesi ve üçüncü gün dirilmesi gerektigini bildirmisti."
\par 8 O zaman Isa'nin sözlerini animsadilar.
\par 9 Mezardan dönen kadinlar bütün bunlari Onbirler'e* ve ötekilerin hepsine bildirdiler.
\par 10 Bunlari elçilere anlatanlar, Mecdelli Meryem, Yohanna, Yakup'un annesi Meryem ve bunlarla birlikte bulunan öbür kadinlardi.
\par 11 Ne var ki, bu sözler elçilere saçma geldi ve kadinlara inanmadilar.
\par 12 Yine de, Petrus kalkip mezara kostu. Egilip içeri baktiginda keten bezlerden baska bir sey görmedi. Olay karsisinda saskina dönmüs bir halde oradan uzaklasti.
\par 13 Ayni gün ögrencilerden ikisi, Yerusalim'den altmis ok atimi uzaklikta bulunan ve Emmaus denilen bir köye gitmekteydiler.
\par 14 Bütün bu olup bitenleri kendi aralarinda konusuyorlardi.
\par 15 Bunlari konusup tartisirlarken Isa yanlarina geldi ve onlarla birlikte yürümeye basladi.
\par 16 Ama onlarin gözleri O'nu tanima gücünden yoksun birakilmisti.
\par 17 Isa, "Yolda birbirinizle ne tartisip duruyorsunuz?" dedi. Üzgün bir halde, olduklari yerde durdular.
\par 18 Bunlardan adi Kleopas olan O'na, "Yerusalim'de bulunup da bu günlerde orada olup bitenleri bilmeyen tek yabanci sen misin?" diye karsilik verdi.
\par 19 Isa onlara, "Hangi olup bitenleri?" dedi. O'na, "Nasirali Isa'yla ilgili olaylari" dediler. "O adam, Tanri'nin ve bütün halkin önünde gerek söz, gerek eylemde güçlü bir peygamberdi.
\par 20 Baskâhinlerle yöneticilerimiz O'nu, ölüm cezasina çarptirmak için valiye teslim ederek çarmiha gerdirdiler; oysa biz O'nun, Israil'i kurtaracak kisi oldugunu ummustuk. Dahasi var, bu olaylar olali üç gün oldu ve aramizdan bazi kadinlar bizi saskina çevirdiler. Bu sabah erkenden mezara gittiklerinde, O'nun cesedini bulamamislar. Sonra geldiler, bir görümde, Isa'nin yasamakta oldugunu bildiren melekler gördüklerini söylediler.
\par 24 Bizimle birlikte olanlardan bazilari mezara gitmis ve durumu, tam kadinlarin anlatmis oldugu gibi bulmuslar. Ama O'nu görmemisler."
\par 25 Isa onlara, "Sizi akilsizlar! Peygamberlerin bütün söylediklerine inanmakta agir davranan kisiler!
\par 26 Mesih'in* bu acilari çekmesi ve yüceligine kavusmasi gerekli degil miydi?" dedi.
\par 27 Sonra Musa'nin ve bütün peygamberlerin yazilarindan baslayarak, Kutsal Yazilar'in hepsinde kendisiyle ilgili olanlari onlara açikladi.
\par 28 Gitmekte olduklari köye yaklastiklari sirada Isa, yoluna devam edecekmis gibi davrandi. Ama onlar, "Bizimle kal. Neredeyse aksam olacak, gün batmak üzere" diyerek O'nu zorladilar. Böylece Isa onlarla birlikte kalmak üzere içeri girdi.
\par 30 Onlarla sofrada otururken Isa ekmek aldi, sükretti ve ekmegi bölüp onlara verdi.
\par 31 O zaman onlarin gözleri açildi ve kendisini tanidilar. Isa ise gözlerinin önünden kayboldu.
\par 32 Onlar birbirine, "Yolda kendisi bizimle konusurken ve Kutsal Yazilar'i bize açiklarken yüreklerimiz nasil da sevinçle çarpiyordu, degil mi?" dediler.
\par 33 Kalkip hemen Yerusalim'e döndüler. Onbirler'i* ve onlarla birlikte olanlari toplanmis buldular.
\par 34 Bunlar, "Rab gerçekten dirildi, Simun'a görünmüs!" diyorlardi.
\par 35 Kendileri de yolda olup bitenleri ve ekmegi böldügü zaman Isa'yi nasil tanidiklarini anlattilar.
\par 36 Bunlari anlatirlarken Isa gelip aralarinda durdu. Onlara, "Size esenlik olsun!" dedi.
\par 37 Ürktüler, bir hayalet gördüklerini sanarak korkuya kapildilar.
\par 38 Isa onlara, "Neden telaslaniyorsunuz? Neden kuskular doguyor içinizde?" dedi.
\par 39 "Ellerime, ayaklarima bakin; iste benim! Dokunun da görün. Hayaletin eti kemigi olmaz, ama görüyorsunuz, benim var."
\par 40 Bunu söyledikten sonra onlara ellerini ve ayaklarini gösterdi.
\par 41 Sevinçten hâlâ inanamayan, saskinlik içindeki ögrencilerine, "Sizde yiyecek bir sey var mi?" diye sordu.
\par 42 Kendisine bir parça kizarmis balik verdiler.
\par 43 Isa onu alip gözlerinin önünde yedi.
\par 44 Sonra onlara söyle dedi: "Daha sizlerle birlikteyken, 'Musa'nin Yasasi'nda, peygamberlerin yazilarinda ve Mezmurlar'da benimle ilgili yazilmis olanlarin tümünün gerçeklesmesi gerektir' demistim."
\par 45 Bundan sonra Kutsal Yazilar'i anlayabilmeleri için zihinlerini açti.
\par 46 Onlara dedi ki, "Söyle yazilmistir: Mesih aci çekecek ve üçüncü gün ölümden dirilecek; günahlarin bagislanmasi için tövbe çagrisi da Yerusalim'den baslayarak bütün uluslara O'nun adiyla duyurulacak.
\par 48 Sizler bu olaylarin taniklarisiniz.
\par 49 Ben de Babam'in vaat ettigini size gönderecegim. Ama siz, yücelerden gelecek güçle kusanincaya dek kentte kalin."
\par 50 Isa onlari kentin disina, Beytanya'nin yakinlarina kadar götürdü. Ellerini kaldirarak onlari kutsadi.
\par 51 Ve onlari kutsarken yanlarindan ayrildi, göge alindi.
\par 52 Ögrencileri O'na tapindilar ve büyük sevinç içinde Yerusalim'e döndüler.
\par 53 Sürekli tapinakta bulunuyor, Tanri'yi övüyorlardi.


\end{document}