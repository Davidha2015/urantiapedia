\begin{document}

\title{Yuhanna}


\chapter{1}

\par 1 Baslangiçta Söz vardi. Söz Tanri'yla birlikteydi ve Söz Tanri'ydi.
\par 2 Baslangiçta O, Tanri'yla birlikteydi.
\par 3 Her sey O'nun araciligiyla var oldu, var olan hiçbir sey O'nsuz olmadi.
\par 4 Yasam O'ndaydi ve yasam insanlarin isigiydi.
\par 5 Isik karanlikta parlar. Karanlik onu alt edemedi.
\par 6 Tanri'nin gönderdigi Yahya adli bir adam ortaya çikti.
\par 7 Taniklik amaciyla, isiga taniklik etsin ve herkes onun araciligiyla iman etsin diye geldi.
\par 8 Kendisi isik degildi, ama isiga taniklik etmeye geldi.
\par 9 Dünyaya gelen, her insani aydinlatan gerçek isik vardi.
\par 10 O, dünyadaydi, dünya O'nun araciligiyla var oldu, ama dünya O'nu tanimadi.
\par 11 Kendi yurduna geldi, ama kendi halki O'nu kabul etmedi.
\par 12 Kendisini kabul edip adina iman edenlerin hepsine Tanri'nin çocuklari olma hakkini verdi.
\par 13 Onlar ne kandan, ne beden ne de insan isteginden dogdular; tersine, Tanri'dan dogdular.
\par 14 Söz, insan olup aramizda yasadi. O'nun yüceligini Baba'dan gelen, lütuf ve gerçekle dolu biricik Ogul'un yüceligini gördük.
\par 15 Yahya O'na taniklik etti. Yüksek sesle söyle dedi: "'Benden sonra gelen benden üstündür. Çünkü O benden önce vardi' diye sözünü ettigim kisi budur."
\par 16 Nitekim hepimiz O'nun dolulugundan lütuf üzerine lütuf aldik.
\par 17 Kutsal Yasa Musa araciligiyla verildi, ama lütuf ve gerçek Isa Mesih* araciligiyla geldi.
\par 18 Tanri'yi hiçbir zaman hiç kimse görmedi. Baba'nin bagrinda bulunan ve Tanri olan biricik Ogul O'nu tanitti.
\par 19 Yahudi yetkililer Yahya'ya, "Sen kimsin?" diye sormak üzere Yerusalim'den* kâhinlerle* Levililer'i* gönderdikleri zaman Yahya'nin tanikligi söyle oldu açikça konustu, inkâr etmedi "Ben Mesih* degilim" diye açikça konustu.
\par 21 Onlar da kendisine, "Öyleyse sen kimsin? Ilyas misin?" diye sordular. O da, "Degilim" dedi. "Sen bekledigimiz peygamber misin?" sorusuna, "Hayir" yanitini verdi.
\par 22 Bu kez, "Kim oldugunu söyle de bizi gönderenlere bir yanit verelim" dediler. "Kendin için ne diyorsun?"
\par 23 Yahya, "Peygamber Yesaya'nin dedigi gibi, 'Rab'bin yolunu düzleyin' diye çölde haykiranin sesiyim ben" dedi.
\par 24 Yahya'ya gönderilen bazi Ferisiler* ona, "Sen Mesih, Ilyas ya da bekledigimiz peygamber degilsen, niye vaftiz* ediyorsun?" diye sordular.
\par 26 Yahya onlara söyle yanit verdi: "Ben suyla vaftiz ediyorum, ama aranizda tanimadiginiz biri duruyor.
\par 27 Benden sonra gelen O'dur. Ben O'nun çariginin bagini çözmeye bile layik degilim."
\par 28 Bütün bunlar Seria Irmagi'nin ötesinde bulunan Beytanya'da, Yahya'nin vaftiz ettigi yerde oldu.
\par 29 Yahya ertesi gün Isa'nin kendisine dogru geldigini görünce söyle dedi: "Iste, dünyanin günahini ortadan kaldiran Tanri Kuzusu!
\par 30 Kendisi için, 'Benden sonra biri geliyor, O benden üstündür. Çünkü O benden önce vardi' dedigim kisi iste budur.
\par 31 Ben O'nu tanimiyordum, ama Israil'in O'nu tanimasi için ben suyla vaftiz ederek geldim."
\par 32 Yahya tanikligini söyle sürdürdü: "Ruh'un güvercin gibi gökten indigini, O'nun üzerinde durdugunu gördüm.
\par 33 Ben O'nu tanimiyordum. Ama suyla vaftiz etmek için beni gönderen, 'Ruh'un kimin üzerine inip durdugunu görürsen, Kutsal Ruh'la vaftiz eden O'dur' dedi.
\par 34 Ben de gördüm ve 'Tanri'nin Oglu budur' diye taniklik ettim."
\par 35 Ertesi gün Yahya yine ögrencilerinden ikisiyle birlikteydi.
\par 36 Oradan geçen Isa'ya bakarak, "Iste Tanri Kuzusu!" dedi.
\par 37 Onun söylediklerini duyan iki ögrenci Isa'nin ardindan gitti.
\par 38 Isa arkasina dönüp ardindan geldiklerini görünce, "Ne ariyorsunuz?" diye sordu. Onlar da, "Rabbî, nerede oturuyorsun?" dediler. Rabbî, ögretmenim anlamina gelir.
\par 39 Isa, "Gelin, görün" dedi. Gidip O'nun nerede oturdugunu gördüler ve o gün O'nunla kaldilar. Saat* dört sulariydi.
\par 40 Yahya'yi isitip Isa'nin ardindan giden iki kisiden biri Simun Petrus'un kardesi Andreas'ti.
\par 41 Andreas önce kendi kardesi Simun'u bularak ona, "Biz Mesih'i bulduk" dedi. Mesih, meshedilmis* anlamina gelir.
\par 42 Andreas kardesini Isa'ya götürdü. Isa ona bakti, "Sen Yuhanna'nin oglu Simun'sun. Kefas diye çagrilacaksin" dedi. Kefas, kaya anlamina gelir.
\par 43 Ertesi gün Isa, Celile'ye gitmeye karar verdi. Filipus'u bulup ona, "Ardimdan gel" dedi.
\par 44 Filipus da Andreas ile Petrus'un kenti olan Beytsayda'dandi.
\par 45 Filipus, Natanel'i bularak ona, "Musa'nin Kutsal Yasa'da hakkinda yazdigi, peygamberlerin de sözünü ettigi kisiyi, Yusuf oglu Nasirali Isa'yi bulduk" dedi.
\par 46 Natanel Filipus'a, "Nasira'dan iyi bir sey çikabilir mi?" diye sordu. Filipus, "Gel de gör" dedi.
\par 47 Isa, Natanel'in kendisine dogru geldigini görünce onun için, "Iste, içinde hile olmayan gerçek bir Israilli!" dedi.
\par 48 Natanel, "Beni nereden taniyorsun?" diye sordu. Isa, "Filipus çagirmadan önce seni incir agacinin altinda gördüm" yanitini verdi.
\par 49 Natanel, "Rabbî, sen Tanri'nin Oglu'sun, sen Israil'in Krali'sin!" dedi.
\par 50 Isa ona dedi ki, "Seni incir agacinin altinda gördügümü söyledigim için mi inaniyorsun? Bunlardan daha büyük seyler göreceksin."
\par 51 Sonra da, "Size dogrusunu söyleyeyim, gögün açildigini, Tanri meleklerinin Insanoglu* üzerinde yükselip indiklerini göreceksiniz" dedi.

\chapter{2}

\par 1 Üçüncü gün Celile'nin Kana Köyü'nde bir dügün vardi. Isa'nin annesi de oradaydi.
\par 2 Isa'yla ögrencileri de dügüne çagrilmislardi.
\par 3 Sarap tükenince annesi Isa'ya, "Saraplari kalmadi" dedi.
\par 4 Isa, "Anne, benden ne istiyorsun? Benim saatim daha gelmedi" dedi.
\par 5 Annesi hizmet edenlere, "Size ne derse onu yapin" dedi.
\par 6 Yahudiler'in geleneksel temizligi için oraya konmus, her biri seksenle yüz yirmi litre alan alti tas küp vardi.
\par 7 Isa hizmet edenlere, "Küpleri suyla doldurun" dedi. Küpleri agizlarina kadar doldurdular.
\par 8 Sonra hizmet edenlere, "Simdi biraz alip sölen baskanina götürün" dedi. Onlar da götürdüler.
\par 9 Sölen baskani, saraba dönüsmüs suyu tatti. Bunun nereden geldigini bilmiyordu, oysa suyu küpten alan hizmetkârlar biliyorlardi. Sölen baskani güveyi çagirip, "Herkes önce iyi sarabi, çok içildikten sonra da kötüsünü sunar" dedi, "Ama sen iyi sarabi simdiye dek saklamissin."
\par 11 Isa bu ilk dogaüstü belirtisini Celile'nin Kana Köyü'nde gerçeklestirdi ve yüceligini gösterdi. Ögrencileri de O'na iman ettiler.
\par 12 Bundan sonra Isa, annesi, kardesleri ve ögrencileri Kefarnahum'a gidip orada birkaç gün kaldilar.
\par 13 Yahudiler'in Fisih Bayrami* yakindi. Isa da Yerusalim'e gitti.
\par 14 Tapinagin avlusunda sigir, koyun ve güvercin satanlari, orada oturmus para bozanlari* gördü.
\par 15 Ipten bir kamçi yaparak hepsini koyunlar ve sigirlarla birlikte tapinaktan kovdu, para bozanlarin paralarini döküp masalarini devirdi.
\par 16 Güvercin satanlara, "Bunlari buradan kaldirin, Babam'in evini pazar yerine çevirmeyin!" dedi.
\par 17 Ögrencileri, "Evin için gösterdigim gayret beni yiyip bitirecek" diye yazilmis olan sözü hatirladilar.
\par 18 Yahudi yetkililer Isa'ya, "Bunlari yaptigina göre, bize nasil bir belirti göstereceksin?" diye sordular.
\par 19 Isa su yaniti verdi: "Bu tapinagi yikin, üç günde onu yeniden kuracagim."
\par 20 Yahudi yetkililer, "Bu tapinak kirk alti yilda yapildi, sen onu üç günde mi kuracaksin?" dediler.
\par 21 Ama Isa'nin sözünü ettigi tapinak kendi bedeniydi.
\par 22 Isa ölümden dirilince ögrencileri bu sözü söyledigini hatirladilar, Kutsal Yazi'ya ve Isa'nin söyledigi bu söze iman ettiler.
\par 23 Fisih Bayrami'nda Isa'nin Yerusalim'de bulundugu sirada gerçeklestirdigi belirtileri gören birçoklari O'nun adina iman ettiler.
\par 24 Ama Isa bütün insanlarin yüregini bildigi için onlara güvenmiyordu.
\par 25 Insan hakkinda kimsenin O'na bir sey söylemesine gerek yoktu. Çünkü kendisi insanin içinden geçenleri biliyordu.

\chapter{3}

\par 1 Yahudiler'in Nikodim adli bir önderi vardi. Ferisiler'den* olan bu adam bir gece Isa'ya gelerek, "Rabbî*, senin Tanri'dan gelmis bir ögretmen oldugunu biliyoruz. Çünkü Tanri kendisiyle olmadikça kimse senin yaptigin bu mucizeleri yapamaz" dedi.
\par 3 Isa ona su karsiligi verdi: "Sana dogrusunu söyleyeyim, bir kimse yeniden dogmadikça Tanri'nin Egemenligi'ni göremez."
\par 4 Nikodim, "Yaslanmis bir adam nasil dogabilir? Annesinin rahmine ikinci kez girip dogabilir mi?" diye sordu.
\par 5 Isa söyle yanit verdi: "Sana dogrusunu söyleyeyim, bir kimse sudan ve Ruh'tan dogmadikça Tanri'nin Egemenligi'ne giremez.
\par 6 Bedenden dogan bedendir, Ruh'tan dogan ruhtur.
\par 7 Sana, 'Yeniden dogmalisiniz' dedigime sasma.
\par 8 Yel diledigi yerde eser; sesini isitirsin, ama nereden gelip nereye gittigini bilemezsin. Ruh'tan dogan herkes böyledir."
\par 9 Nikodim Isa'ya, "Bunlar nasil olabilir?" diye sordu.
\par 10 Isa ona söyle yanit verdi: "Sen Israil'in ögretmeni oldugun halde bunlari anlamiyor musun?
\par 11 Sana dogrusunu söyleyeyim, biz bildigimizi söylüyoruz, gördügümüze taniklik ediyoruz. Sizler ise bizim tanikligimizi kabul etmiyorsunuz.
\par 12 Sizlere yeryüzüyle ilgili seyleri söyledigim zaman inanmazsaniz, gökle ilgili seyleri söyledigimde nasil inanacaksiniz?
\par 13 Gökten inmis olan Insanoglu'ndan* baska hiç kimse göge çikmamistir.
\par 14 Musa çölde yilani nasil yukari kaldirdiysa, Insanoglu'nun da öylece yukari kaldirilmasi gerekir.
\par 15 Öyle ki, O'na iman eden herkes sonsuz yasama kavussun.
\par 16 "Çünkü Tanri dünyayi o kadar çok sevdi ki, biricik Oglu'nu verdi. Öyle ki, O'na iman edenlerin hiçbiri mahvolmasin, hepsi sonsuz yasama kavussun.
\par 17 Tanri, Oglu'nu dünyayi yargilamak için göndermedi, dünya O'nun araciligiyla kurtulsun diye gönderdi.
\par 18 O'na iman eden yargilanmaz, iman etmeyen ise zaten yargilanmistir. Çünkü Tanri'nin biricik Oglu'nun adina iman etmemistir.
\par 19 Yargi da sudur: Dünyaya isik geldi, ama insanlar isik yerine karanligi sevdiler. Çünkü yaptiklari isler kötüydü.
\par 20 Kötülük yapan herkes isiktan nefret eder ve yaptiklari açiga çikmasin diye isiga yaklasmaz.
\par 21 Ama gerçegi uygulayan kisi yaptiklarini, Tanri'ya dayanarak yaptigini göstermek için isiga gelir."
\par 22 Bundan sonra Isa'yla ögrencileri Yahudiye diyarina gittiler. Isa onlarla birlikte orada bir süre kalarak vaftiz* etti.
\par 23 Yahya da Salim yakinindaki Aynon'da vaftiz ediyordu. Çünkü orada bol su vardi. Insanlar gelip vaftiz oluyorlardi.
\par 24 Yahya henüz hapse atilmamisti.
\par 25 O siralarda Yahya'nin ögrencileriyle bir Yahudi arasinda temizlenme konusunda bir tartisma çikti.
\par 26 Ögrencileri Yahya'ya gelerek, "Rabbî" dediler, "Seria Irmagi'nin karsi yakasinda birlikte oldugun ve kendisi için taniklik ettigin adam var ya, iste O vaftiz ediyor, herkes de O'na gidiyor."
\par 27 Yahya söyle yanit verdi: "Insan, kendisine gökten verilmedikçe hiçbir sey alamaz.
\par 28 'Ben Mesih* degilim, ama O'nun öncüsü olarak gönderildim' dedigime siz kendiniz taniksiniz.
\par 29 Gelin kiminse, güvey odur. Ama güveyin yaninda duran ve onu dinleyen dostu onun sesini isitince çok sevinir. Iste benim sevincim böylece tamamlandi.
\par 30 O büyümeli, bense küçülmeliyim."
\par 31 Yukaridan gelen, herkesten üstündür. Dünyadan olan dünyaya aittir ve dünyadan söz eder. Gökten gelen ise, herkesten üstündür.
\par 32 Ne görmüs ne isitmisse ona taniklik eder, ama tanikligini kimse kabul etmez.
\par 33 O'nun tanikligini kabul eden, Tanri'nin gerçek olduguna mührünü basmistir.
\par 34 Tanri'nin gönderdigi kisi Tanri'nin sözlerini söyler. Çünkü Tanri, Ruh'u ölçüyle vermez.
\par 35 Baba Ogul'u sever; her seyi O'na teslim etmistir.
\par 36 Ogul'a iman edenin sonsuz yasami vardir. Ama Ogul'un sözünü dinlemeyen yasami görmeyecektir. Tanri'nin gazabi böylesinin üzerinde kalir.

\chapter{4}

\par 1 Ferisiler*, Isa'nin Yahya'dan daha çok ögrenci edinip vaftiz* ettigini duydular aslinda Isa'nin kendisi degil, ögrencileri vaftiz ediyorlardi Isa bunu ögrenince Yahudiye'den ayrilip yine Celile'ye gitti.
\par 4 Giderken Samiriye'den geçmesi gerekiyordu.
\par 5 Böylece Samiriye'nin Sihar denilen kentine geldi. Burasi Yakup'un kendi oglu Yusuf'a vermis oldugu topragin yakinindaydi.
\par 6 Yakup'un kuyusu da oradaydi. Isa, yolculuktan yorulmus oldugu için kuyunun yanina oturmustu. Saat* on iki sulariydi.
\par 7 Samiriyeli* bir kadin su çekmeye geldi. Isa ona, "Bana su ver, içeyim" dedi.
\par 8 Isa'nin ögrencileri yiyecek satin almak için kente gitmislerdi.
\par 9 Samiriyeli kadin, "Sen Yahudi'sin, bense Samiriyeli bir kadinim" dedi, "Nasil olur da benden su istersin?" Çünkü Yahudiler'in Samiriyeliler'le iliskileri yoktur.
\par 10 Isa kadina su yaniti verdi: "Eger sen Tanri'nin armaganini ve sana, 'Bana su ver, içeyim' diyenin kim oldugunu bilseydin, sen O'ndan dilerdin, O da sana yasam suyunu verirdi."
\par 11 Kadin, "Efendim" dedi, "Su çekecek bir seyin yok, kuyu da derin, yasam suyunu nereden bulacaksin?
\par 12 Sen, bu kuyuyu bize vermis, kendisi, ogullari ve davarlari ondan içmis olan atamiz Yakup'tan daha mi büyüksün?"
\par 13 Isa söyle yanit verdi: "Bu sudan her içen yine susayacak.
\par 14 Oysa benim verecegim sudan içen sonsuza dek susamaz. Benim verecegim su, içende sonsuz yasam için fiskiran bir pinar olacak."
\par 15 Kadin, "Efendim" dedi, "Bu suyu bana ver. Böylece ne susayayim, ne de su çekmek için buraya kadar geleyim."
\par 16 Isa, "Git, kocani çagir ve buraya gel" dedi.
\par 17 Kadin, "Kocam yok" diye yanitladi. Isa, "Kocam yok demekle dogruyu söyledin" dedi.
\par 18 "Bes kocaya vardin. Simdi birlikte yasadigin adam kocan degil. Dogruyu söyledin."
\par 19 Kadin, "Efendim, anliyorum, sen bir peygambersin" dedi.
\par 20 "Atalarimiz bu dagda tapindilar, ama sizler tapilmasi gereken yerin Yerusalim'de oldugunu söylüyorsunuz."
\par 21 Isa ona söyle dedi: "Kadin, bana inan, öyle bir saat geliyor ki, Baba'ya ne bu dagda, ne de Yerusalim'de tapinacaksiniz!
\par 22 Siz bilmediginize tapiyorsunuz, biz bildigimize tapiyoruz. Çünkü kurtulus Yahudiler'dendir.
\par 23 Ama içtenlikle tapinanlarin Baba'ya ruhta ve gerçekte tapinacaklari saat geliyor. Iste, o saat simdidir. Baba da kendisine böyle tapinanlari ariyor.
\par 24 Tanri ruhtur, O'na tapinanlar da ruhta ve gerçekte tapinmalidirlar."
\par 25 Kadin Isa'ya, "Mesih denilen meshedilmis* Olan'in gelecegini biliyorum" dedi, "O gelince bize her seyi bildirecek."
\par 26 Isa, "Seninle konusan ben, O'yum" dedi.
\par 27 Bu sirada Isa'nin ögrencileri geldiler. O'nun bir kadinla konusmasina sastilar. Bununla birlikte hiçbiri, "Ne istiyorsun?" ya da, "O kadinla neden konusuyorsun?" demedi.
\par 28 Sonra kadin su testisini birakarak kente gitti ve halka söyle dedi: "Gelin, yaptigim her seyi bana söyleyen adami görün. Acaba Mesih bu mudur?"
\par 30 Halk da kentten çikip Isa'ya dogru gelmeye basladi.
\par 31 Bu arada ögrencileri O'na, "Rabbî*, yemek ye!" diye rica ediyorlardi.
\par 32 Ama Isa, "Benim, sizin bilmediginiz bir yiyecegim var" dedi.
\par 33 Ögrenciler birbirlerine, "Acaba biri O'na yiyecek mi getirdi?" diye sordular.
\par 34 Isa, "Benim yemegim, beni gönderenin istegini yerine getirmek ve O'nun isini tamamlamaktir" dedi.
\par 35 "Sizler, 'Ekinleri biçmeye daha dört ay var' demiyor musunuz? Iste, size söylüyorum, basinizi kaldirip tarlalara bakin. Ekinler sararmis, biçilmeye hazir!
\par 36 Eken ve biçen birlikte sevinsinler diye, biçen kisi simdiden ücretini alir ve sonsuz yasam için ürün toplar.
\par 37 'Biri eker, baskasi biçer' sözü bu durumda dogrudur.
\par 38 Ben sizi, emek vermediginiz bir ürünü biçmeye gönderdim. Baskalari emek verdiler, siz ise onlarin emeginden yararlandiniz."
\par 39 O kentten birçok Samiriyeli, "Yaptigim her seyi bana söyledi" diye taniklik eden kadinin sözü üzerine Isa'ya iman etti.
\par 40 Samiriyeliler O'na gelip yanlarinda kalmasi için rica ettiler. O da orada iki gün kaldi.
\par 41 O'nun sözü üzerine daha birçoklari iman etti.
\par 42 Bunlar kadina, "Bizim iman etmemizin nedeni artik senin sözlerin degil" diyorlardi. "Kendimiz isittik, O'nun gerçekten dünyanin Kurtaricisi oldugunu biliyoruz."
\par 43 Bu iki günden sonra Isa oradan ayrilip Celile'ye gitti.
\par 44 Isa'nin kendisi, bir peygamberin kendi memleketinde saygi görmedigine taniklik etmisti.
\par 45 Celile'ye geldigi zaman Celileliler O'nu iyi karsiladilar. Çünkü onlar da bayram için gitmisler ve bayramda O'nun Yerusalim'de yaptigi her seyi görmüslerdi.
\par 46 Isa yine, suyu saraba çevirdigi Celile'nin Kana Köyü'ne geldi. Orada saraya bagli bir memur vardi. Oglu Kefarnahum'da hastaydi.
\par 47 Adam, Isa'nin Yahudiye'den Celile'ye geldigini isitince yanina gitti, evine gelip ölmek üzere olan oglunu iyilestirmesi için O'na yalvardi.
\par 48 Isa adama, "Sizler belirtiler ve harikalar görmedikçe iman etmeyeceksiniz" dedi.
\par 49 Saray memuru Isa'ya, "Efendim, çocugum ölmeden yetis!" dedi.
\par 50 Isa, "Git, oglun yasayacak" dedi. Adam, Isa'nin söyledigi söze iman ederek gitti.
\par 51 Daha yoldayken köleleri onu karsilayip oglunun yasadigini bildirdiler.
\par 52 Adam onlara, oglunun iyilesmeye basladigi saati sordu. "Dün ögle üstü saat* birde atesi düstü" dediler.
\par 53 Baba bunun, Isa'nin, "Oglun yasayacak" dedigi saat oldugunu anladi. Kendisi ve bütün ev halki iman etti.
\par 54 Isa, bu ikinci belirtiyi de Yahudiye'den Celile'ye döndükten sonra gerçeklestirdi.

\chapter{5}

\par 1 Isa bundan sonra Yahudiler'in bir bayrami nedeniyle Yerusalim'e gitti.
\par 2 Yerusalim'de Koyun Kapisi yaninda, Ibranice'de* Beytesta denilen bes eyvanli bir havuz vardir.
\par 3 Bu eyvanlarin altinda kör, kötürüm, felçli hastalardan bir kalabalik yatardi.
\par 5 Orada otuz sekiz yildir hasta olan bir adam vardi.
\par 6 Isa hasta yatan bu adami görünce ve uzun zamandir bu durumda oldugunu anlayinca, "Iyi olmak ister misin?" diye sordu.
\par 7 Hasta söyle yanit verdi: "Efendim, su çalkandigi zaman beni havuza indirecek kimsem yok, tam girecegim an benden önce baskasi giriyor."
\par 8 Isa ona, "Kalk, silteni topla ve yürü" dedi.
\par 9 Adam o anda iyilesti. Siltesini toplayip yürümeye basladi. O gün Sabat Günü'ydü*.
\par 10 Bu yüzden Yahudi yetkililer iyilesen adama, "Bugün Sabat Günü" dediler, "Silteni toplaman yasaktir."
\par 11 Ama adam onlara söyle yanit verdi: "Beni iyilestiren kisi bana, 'Silteni topla ve yürü' dedi."
\par 12 "Sana, 'Silteni topla ve yürü' diyen adam kim?" diye sordular.
\par 13 Iyilesen adam ise O'nun kim oldugunu bilmiyordu. Orasi kalabalikti, Isa da çekilip gitmisti.
\par 14 Isa daha sonra adami tapinakta buldu. "Bak, iyi oldun. Artik günah isleme de basina daha kötü bir sey gelmesin" dedi.
\par 15 Adam gidip Yahudi yetkililere kendisini iyilestirenin Isa oldugunu bildirdi.
\par 16 Sabat Günü böyle seyler yaptigi için Isa'ya zulmetmeye basladilar.
\par 17 Ama Isa onlara su karsiligi verdi: "Babam hâlâ çalismaktadir, ben de çalisiyorum."
\par 18 Iste bu nedenle Yahudi yetkililer O'nu öldürmek için daha çok gayret ettiler. Çünkü yalniz Sabat Günü düzenini bozmakla kalmamis, Tanri'nin kendi Babasi oldugunu söyleyerek kendisini Tanri'ya esit kilmisti.
\par 19 Isa Yahudi yetkililere söyle karsilik verdi: "Size dogrusunu söyleyeyim, Ogul, Baba'nin yaptiklarini görmedikçe kendiliginden bir sey yapamaz. Baba ne yaparsa Ogul da ayni seyi yapar.
\par 20 Çünkü Baba Ogul'u sever ve yaptiklarinin hepsini O'na gösterir. Sasasiniz diye O'na bunlardan daha büyük isler de gösterecektir.
\par 21 Baba nasil ölüleri diriltip onlara yasam veriyorsa, Ogul da diledigi kimselere yasam verir.
\par 22 Baba kimseyi yargilamaz, bütün yargilama isini Ogul'a vermistir.
\par 23 Öyle ki, herkes Baba'yi onurlandirdigi gibi Ogul'u onurlandirsin. Ogul'u onurlandirmayan, O'nu gönderen Baba'yi da onurlandirmaz.
\par 24 "Size dogrusunu söyleyeyim, sözümü isitip beni gönderene iman edenin sonsuz yasami vardir. Böyle biri yargilanmaz, ölümden yasama geçmistir.
\par 25 Size dogrusunu söyleyeyim, ölülerin Tanri Oglu'nun sesini isitecekleri ve isitenlerin yasayacaklari saat geliyor, geldi bile.
\par 26 Çünkü Baba, kendisinde yasam oldugu gibi, Ogul'a da kendisinde yasam olma özelligini verdi.
\par 27 O'na yargilama yetkisini de verdi. Çünkü O Insanoglu'dur*.
\par 28 Buna sasmayin. Mezarda olanlarin hepsinin O'nun sesini isitecekleri saat geliyor.
\par 29 Ve onlar mezarlarindan çikacaklar. Iyilik yapmis olanlar yasamak, kötülük yapmis olanlar yargilanmak üzere dirilecekler."
\par 30 "Ben kendiligimden hiçbir sey yapamam. Isittigim gibi yargilarim ve benim yargim adildir. Çünkü amacim kendi istedigimi degil, beni gönderenin istedigini yapmaktir.
\par 31 Eger kendim için ben taniklik edersem, tanikligim geçerli olmaz.
\par 32 Ama benim için taniklik eden baska biri vardir. O'nun benim için ettigi tanikligin geçerli oldugunu bilirim.
\par 33 Siz Yahya'ya adamlar gönderdiniz, o da gerçege taniklik etti.
\par 34 Insanin tanikligini kabul ettigim için degil, kurtulmaniz için bunlari söylüyorum.
\par 35 Yahya, yanan ve isik saçan bir çiraydi. Sizler onun isiginda bir süre için cosmak istediniz.
\par 36 Ama benim, Yahya'ninkinden daha büyük bir tanikligim var. Tamamlamam için Baba'nin bana verdigi isler, su yaptigim isler, beni Baba'nin gönderdigine taniklik ediyor.
\par 37 Beni gönderen Baba da benim için taniklik etmistir. Siz hiçbir zaman ne O'nun sesini isittiniz, ne de seklini gördünüz.
\par 38 O'nun sözü sizde yasamiyor. Çünkü O'nun gönderdigi kisiye iman etmiyorsunuz.
\par 39 Kutsal Yazilar'i arastiriyorsunuz. Çünkü bunlar araciligiyla sonsuz yasama sahip oldugunuzu saniyorsunuz. Bana taniklik eden de bu yazilardir!
\par 40 Öyleyken siz, yasama kavusmak için bana gelmek istemiyorsunuz.
\par 41 "Insanlardan övgü kabul etmiyorum.
\par 42 Ama ben sizi bilirim, içinizde Tanri sevgisi yoktur.
\par 43 Ben Babam'in adina geldim, ama beni kabul etmiyorsunuz. Oysa baska birisi kendi adina gelirse, onu kabul edeceksiniz.
\par 44 Birbirinizden övgüler kabul ediyor, ama tek olan Tanri'nin övgüsünü kazanmaya çalismiyorsunuz. Bu durumda nasil iman edebilirsiniz?
\par 45 Baba'nin önünde sizi suçlayacagimi sanmayin. Sizi suçlayan, umut bagladiginiz Musa'dir.
\par 46 Musa'ya iman etmis olsaydiniz, bana da iman ederdiniz. Çünkü o benim hakkimda yazmistir.
\par 47 Ama onun yazilarina iman etmezseniz, benim sözlerime nasil iman edeceksiniz?"

\chapter{6}

\par 1 Bundan sonra Isa, Celile Taberiye Gölü'nün karsi yakasina geçti.
\par 2 Ardindan büyük bir kalabalik gidiyordu. Çünkü hastalar üzerinde yaptigi mucizeleri görmüslerdi.
\par 3 Isa daga çikip orada ögrencileriyle birlikte oturdu.
\par 4 Yahudiler'in Fisih Bayrami* yakindi.
\par 5 Isa basini kaldirip büyük bir kalabaligin kendisine dogru geldigini görünce Filipus'a, "Bunlari doyurmak için nereden ekmek alalim?" diye sordu.
\par 6 Bu sözü onu denemek için söyledi, aslinda kendisi ne yapacagini biliyordu.
\par 7 Filipus O'na su yaniti verdi: "Her birinin bir lokma yiyebilmesi için iki yüz dinarlik ekmek bile yetmez."
\par 8 Ögrencilerinden biri, Simun Petrus'un kardesi Andreas, Isa'ya dedi ki, "Burada bes arpa ekmegiyle iki baligi olan bir çocuk var. Ama bu kadar adam için bunlar nedir ki?"
\par 10 Isa, "Halki yere oturtun" dedi. Orasi çayirlikti. Böylece halk yere oturdu. Yaklasik bes bin erkek vardi.
\par 11 Isa ekmekleri aldi, sükrettikten sonra oturanlara dagitti. Baliklardan da istedikleri kadar verdi.
\par 12 Herkes doyunca Isa ögrencilerine, "Artakalan parçalari toplayin, hiçbir sey ziyan olmasin" dedi.
\par 13 Onlar da topladilar. Yedikleri bes arpa ekmeginden artakalan parçalarla on iki sepet doldurdular.
\par 14 Halk, Isa'nin yaptigi mucizeyi görünce, "Gerçekten dünyaya gelecek olan peygamber budur" dedi.
\par 15 Isa onlarin gelip kendisini kral yapmak üzere zorla götüreceklerini bildiginden tek basina yine daga çekildi.
\par 16 Aksam olunca ögrencileri göle indiler.
\par 17 Bir tekneye binerek gölün karsi yakasindaki Kefarnahum'a dogru yol aldilar. Karanlik basmis, Isa henüz yanlarina gelmemisti.
\par 18 Güçlü bir rüzgar estiginden göl kabarmaya basladi.
\par 19 Ögrenciler üç mil kadar kürek çektikten sonra, Isa'nin gölün üstünde yürüyerek tekneye yaklastigini görünce korktular.
\par 20 Ama Isa, "Korkmayin, benim!" dedi.
\par 21 Bunun üzerine O'nu tekneye almak istediler. O anda tekne gidecekleri kiyiya ulasti.
\par 22 Ertesi gün, gölün karsi yakasinda kalan halk, önceden orada sadece bir tek tekne bulundugunu, Isa'nin kendi ögrencileriyle birlikte bu tekneye binmedigini, ögrencilerinin yalniz gittiklerini anladi.
\par 23 Rab'bin sükretmesinden sonra halkin ekmek yedigi yerin yakinina Taberiye'den baska tekneler geldi.
\par 24 Halk, Isa'nin ve ögrencilerinin orada olmadigini görünce teknelere binerek Kefarnahum'a, Isa'yi aramaya gitti.
\par 25 O'nu gölün karsi yakasinda bulduklari zaman, "Rabbî*, buraya ne zaman geldin?" diye sordular.
\par 26 Isa söyle yanit verdi: "Size dogrusunu söyleyeyim, dogaüstü belirtiler gördügünüz için degil, ekmeklerden yiyip doydugunuz için beni ariyorsunuz.
\par 27 Geçici yiyecek için degil, sonsuz yasam boyunca kalici yiyecek için çalisin. Bunu size Insanoglu* verecek. Çünkü Baba Tanri O'na bu onayi vermistir."
\par 28 Onlar da sunu sordular: "Tanri'nin istedigi isleri yapmak için ne yapmaliyiz?"
\par 29 Isa, "Tanri'nin isi O'nun gönderdigi kisiye iman etmenizdir" diye yanit verdi.
\par 30 Bunun üzerine, "Görüp sana iman etmemiz için nasil bir belirti gerçeklestireceksin? Ne yapacaksin?" dediler.
\par 31 "Atalarimiz çölde man* yediler. Yazilmis oldugu gibi, 'Yemeleri için onlara gökten ekmek verdi.'"
\par 32 Isa onlara dedi ki, "Size dogrusunu söyleyeyim, gökten ekmegi size Musa vermedi, gökten size gerçek ekmegi Babam verir.
\par 33 Çünkü Tanri'nin ekmegi, gökten inen ve dünyaya yasam verendir."
\par 34 Onlar da, "Efendimiz, bizlere her zaman bu ekmegi ver!" dediler.
\par 35 Isa, "Yasam ekmegi Ben'im. Bana gelen asla acikmaz, bana iman eden hiçbir zaman susamaz" dedi.
\par 36 "Ama ben size dedim ki, 'Beni gördünüz, yine de iman etmiyorsunuz.'
\par 37 Baba'nin bana verdiklerinin hepsi bana gelecek ve bana geleni asla kovmam.
\par 38 Çünkü kendi istegimi degil, beni gönderenin istegini yerine getirmek için gökten indim.
\par 39 Beni gönderenin istegi, bana verdiklerinden hiçbirini yitirmemem, son gün hepsini diriltmemdir.
\par 40 Çünkü Babam'in istegi, Ogul'u gören ve O'na iman eden herkesin sonsuz yasama kavusmasidir. Ben de böylelerini son günde diriltecegim."
\par 41 "Gökten inmis olan ekmek Ben'im" dedigi için Yahudiler O'na karsi söylenmeye basladilar.
\par 42 "Yusuf oglu Isa degil mi bu?" diyorlardi. "Annesini de, babasini da taniyoruz. Simdi nasil oluyor da, 'Gökten indim' diyor?"
\par 43 Isa, "Aranizda söylenmeyin" dedi.
\par 44 "Beni gönderen Baba bir kimseyi bana çekmedikçe, o kimse bana gelemez. Bana geleni de son günde diriltecegim.
\par 45 Peygamberlerin yazdigi gibi, 'Tanri onlarin hepsine kendi yollarini ögretecektir.' Baba'yi isiten ve O'ndan ögrenen herkes bana gelir.
\par 46 Bu, bir kimsenin Baba'yi gördügü anlamina gelmez. Baba'yi sadece Tanri'dan gelen görmüstür.
\par 47 Size dogrusunu söyleyeyim, iman edenin sonsuz yasami vardir.
\par 48 Yasam ekmegi Ben'im.
\par 49 Atalariniz çölde man yediler, yine de öldüler.
\par 50 Gökten inen öyle bir ekmek var ki, ondan yiyen ölmeyecek.
\par 51 Gökten inmis olan diri ekmek Ben'im. Bu ekmekten yiyen sonsuza dek yasayacak. Dünyanin yasami ugruna verecegim ekmek de benim bedenimdir."
\par 52 Bunun üzerine Yahudiler, "Bu adam yememiz için bedenini bize nasil verebilir?" diyerek birbirleriyle çekismeye basladilar.
\par 53 Isa onlara söyle dedi: "Size dogrusunu söyleyeyim, Insanoglu'nun bedenini yiyip kanini içmedikçe, sizde yasam olmaz.
\par 54 Bedenimi yiyenin, kanimi içenin sonsuz yasami vardir ve ben onu son günde diriltecegim.
\par 55 Çünkü bedenim gerçek yiyecek, kanim gerçek içecektir.
\par 56 Bedenimi yiyip kanimi içen bende yasar, ben de onda.
\par 57 Yasayan Baba beni gönderdigi ve ben Baba'nin araciligiyla yasadigim gibi, bedenimi yiyen de benim araciligimla yasayacak.
\par 58 Iste gökten inmis olan ekmek budur. Atalarinizin yedikleri man gibi degildir. Atalariniz öldüler. Oysa bu ekmegi yiyen sonsuza dek yasar."
\par 59 Isa bu sözleri Kefarnahum'da havrada ögretirken söyledi.
\par 60 Ögrencilerinin birçogu bunu isitince, "Bu söz çok çetin, kim kabul edebilir?" dediler.
\par 61 Ögrencilerinin buna karsi söylendigini anlayan Isa, "Bu sizi sasirtiyor mu?" dedi.
\par 62 "Ya Insanoglu'nun* önceden bulundugu yere yükseldigini görürseniz...?
\par 63 Yasam veren Ruh'tur. Beden bir yarar saglamaz. Sizlere söyledigim sözler ruhtur, yasamdir.
\par 64 Yine de aranizda iman etmeyenler var." Isa iman etmeyenlerin ve kendisine ihanet edecek kisinin kim oldugunu bastan beri biliyordu.
\par 65 "Sizlere, 'Baba'nin bana yöneltmedigi hiç kimse bana gelemez' dememin nedeni budur" dedi.
\par 66 Bunun üzerine ögrencilerinin birçogu geri döndüler, artik O'nunla dolasmaz oldular.
\par 67 Isa o zaman Onikiler'e*, "Siz de mi ayrilmak istiyorsunuz?" diye sordu.
\par 68 Simun Petrus su yaniti verdi: "Rab, biz kime gidelim? Sonsuz yasamin sözleri sendedir.
\par 69 Iman ediyor ve biliyoruz ki, sen Tanri'nin Kutsali'sin."
\par 70 Isa onlara su karsiligi verdi: "Siz Onikiler'i seçen ben degil miyim? Buna karsin içinizden biri iblistir."
\par 71 Simun Iskariot'un oglu Yahuda'dan söz ediyordu. Çünkü Yahuda Onikiler'den biri oldugu halde Isa'ya ihanet edecekti.

\chapter{7}

\par 1 Bundan sonra Isa Celile'de dolasmaya basladi. Yahudi yetkililer O'nu öldürmeyi amaçladiklari için Yahudiye'de dolasmak istemiyordu.
\par 2 Yahudiler'in Çardak Bayrami* yaklasmisti.
\par 3 Bu nedenle Isa'nin kardesleri O'na, "Buradan ayril, Yahudiye'ye git" dediler, "Ögrencilerin de yaptigin isleri görsünler.
\par 4 Çünkü kendini açikça tanitmak isteyen bir kimse yaptiklarini gizlemez. Mademki bu seyleri yapiyorsun, kendini dünyaya göster!"
\par 5 Kardesleri bile O'na iman etmiyorlardi.
\par 6 Isa onlara, "Benim zamanim daha gelmedi" dedi, "Oysa sizin için zaman hep uygundur.
\par 7 Dünya sizden nefret edemez, ama benden nefret ediyor. Çünkü yaptiklarinin kötü olduguna taniklik ediyorum.
\par 8 Siz bu bayrami kutlamaya gidin. Ben simdilik gitmeyecegim. Çünkü benim zamanim daha dolmadi."
\par 9 Isa bu sözleri söyleyip Celile'de kaldi.
\par 10 Ne var ki, kardesleri bayrami kutlamaya gidince, kendisi de gitti. Ancak açiktan açiga degil, gizlice gitti.
\par 11 Yahudi yetkililer O'nu bayram sirasinda ariyor, "O nerede?" diye soruyorlardi.
\par 12 Kalabalik arasinda O'nunla ilgili bir sürü laf fisildaniyordu. Bazilari, "Iyi adamdir", bazilari da, "Hayir, tam tersine, halki saptiriyor" diyorlardi.
\par 13 Bununla birlikte yetkililerden korktuklari için, hiç kimse O'ndan açikça söz etmiyordu.
\par 14 Bayramin yarisi geçmisti. Isa tapinaga gidip ögretmeye basladi.
\par 15 Yahudiler sasirdilar. "Bu adam hiç ögrenim görmedigi halde, nasil bu kadar bilgili olabilir?" dediler.
\par 16 Isa onlara, "Benim ögretim benim degil, beni gönderenindir" diye karsilik verdi.
\par 17 "Eger bir kimse Tanri'nin istegini yerine getirmek istiyorsa, bu ögretinin Tanri'dan mi oldugunu, yoksa kendiligimden mi konustugumu bilecektir.
\par 18 Kendiliginden konusan kendini yüceltmek ister, ama kendisini göndereni yüceltmek isteyen dogrudur ve O'nda haksizlik yoktur.
\par 19 Musa size Kutsal Yasa'yi vermedi mi? Yine de hiçbiriniz Yasa'yi yerine getirmiyor. Neden beni öldürmek istiyorsunuz?"
\par 20 Kalabalik, "Cin çarpmis seni!" dedi. "Seni öldürmek isteyen kim?"
\par 21 Isa, "Ben bir mucize yaptim, hepiniz saskina döndünüz" diye yanit verdi.
\par 22 "Musa size sünneti buyurdugu için aslinda bu, Musa'dan degil, atalarinizdan kalmadir Sabat Günü* birini sünnet edersiniz.
\par 23 Musa'nin Yasasi bozulmasin diye Sabat Günü biri sünnet ediliyor da, Sabat Günü bir adami tamamen iyilestirdim diye bana neden kiziyorsunuz?
\par 24 Dis görünüse göre yargilamayin, yarginiz adil olsun."
\par 25 Yerusalimliler'in bazilari, "Öldürmek istedikleri adam bu degil mi?" diyorlardi.
\par 26 "Bakin, açikça konusuyor, O'na bir sey demiyorlar. Yoksa önderler O'nun Mesih* oldugunu gerçekten kabul ettiler mi?
\par 27 Ama biz bu adamin nereden geldigini biliyoruz. Oysa Mesih geldigi zaman O'nun nereden geldigini kimse bilmeyecek."
\par 28 O sirada tapinakta ögreten Isa yüksek sesle söyle dedi: "Hem beni taniyorsunuz, hem de nereden oldugumu biliyorsunuz! Ben kendiligimden gelmedim. Beni gönderen gerçektir. O'nu siz tanimiyorsunuz.
\par 29 Ben O'nu tanirim. Çünkü ben O'ndanim, beni O gönderdi."
\par 30 Bunun üzerine O'nu yakalamak istediler, ama kimse O'na el sürmedi. Çünkü O'nun saati henüz gelmemisti.
\par 31 Halktan birçok kisi ise O'na iman etti. "Mesih gelince, bunun yaptiklarindan daha mi çok mucize yapacak?" diyorlardi.
\par 32 Ferisiler* halkin Isa hakkinda böyle fisildastigini duydular. Baskâhinler ve Ferisiler O'nu yakalamak için görevliler gönderdiler.
\par 33 Isa, "Kisa bir süre daha sizinleyim" dedi, "Sonra beni gönderene gidecegim.
\par 34 Beni arayacaksiniz ama bulamayacaksiniz. Ve benim bulundugum yere siz gelemezsiniz."
\par 35 Bunun üzerine Yahudiler birbirlerine, "Bu adam nereye gidecek de biz O'nu bulamayacagiz?" dediler. "Yoksa Grekler* arasinda dagilmis olanlara gidip Grekler'e mi ögretecek?
\par 36 'Beni arayacaksiniz ama bulamayacaksiniz. Ve benim bulundugum yere siz gelemezsiniz' diyor. Ne demek istiyor?"
\par 37 Bayramin son ve en önemli günü Isa ayaga kalkti, yüksek sesle söyle dedi: "Bir kimse susamissa bana gelsin, içsin.
\par 38 Kutsal Yazi'da dendigi gibi, bana iman edenin 'içinden diri su irmaklari akacaktir.'"
\par 39 Bunu, kendisine iman edenlerin alacagi Ruh'la ilgili olarak söylüyordu. Ruh henüz verilmemisti. Çünkü Isa henüz yüceltilmemisti.
\par 40 Halktan bazilari bu sözleri isitince, "Gerçekten bekledigimiz peygamber budur" dediler.
\par 41 Bazilari da, "Bu Mesih'tir" diyorlardi. Baskalari ise, "Olamaz! Mesih Celile'den mi gelecek?" dediler.
\par 42 "Kutsal Yazi'da, 'Mesih, Davut'un soyundan, Davut'un yasadigi Beytlehem Kenti'nden gelecek' denmemis midir?"
\par 43 Böylece Isa'dan dolayi halk arasinda ayrilik dogdu.
\par 44 Bazilari O'nu yakalamak istedilerse de, kimse O'na el sürmedi.
\par 45 Görevliler geri dönünce, baskâhinlerle Ferisiler, "Niçin O'nu getirmediniz?" diye sordular.
\par 46 Görevliler, "Hiç kimse hiçbir zaman bu adamin konustugu gibi konusmamistir" karsiligini verdiler.
\par 47 Ferisiler, "Yoksa siz de mi aldandiniz?" dediler.
\par 48 "Önderlerden ya da Ferisiler'den O'na iman eden oldu mu hiç?
\par 49 Kutsal Yasa'yi bilmeyen bu halk lanetlidir."
\par 50 Içlerinden biri, daha önce Isa'ya gelen Nikodim, onlara söyle dedi: "Yasamiza göre, bir adami dinlemeden, ne yaptigini ögrenmeden onu yargilamak dogru mu?"
\par 52 Ona, "Yoksa sen de mi Celile'densin?" diye karsilik verdiler. "Arastir, bak, Celile'den peygamber çikmaz."
\par 53 Bundan sonra herkes evine gitti.

\chapter{8}

\par 1 Isa ise Zeytin Dagi'na gitti.
\par 2 Ertesi sabah erkenden yine tapinaga döndü. Bütün halk O'nun yanina geliyordu. O da oturup onlara ögretmeye basladi.
\par 3 Din bilginleri* ve Ferisiler*, zina ederken yakalanmis bir kadin getirdiler. Kadini orta yere çikararak Isa'ya, "Ögretmen, bu kadin tam zina ederken yakalandi" dediler.
\par 5 "Musa, Yasa'da bize böyle kadinlarin taslanmasini buyurdu, sen ne dersin?"
\par 6 Bunlari Isa'yi denemek amaciyla söylüyorlardi; O'nu suçlayabilmek için bir neden ariyorlardi. Isa egilmis, parmagiyla topraga yazi yaziyordu.
\par 7 Durmadan ayni soruyu sormalari üzerine dogruldu ve, "Içinizde kim günahsizsa, ilk tasi o atsin!" dedi.
\par 8 Sonra yine egildi, topraga yazmaya basladi.
\par 9 Bunu isittikleri zaman, basta yaslilar olmak üzere, birer birer disari çikip Isa'yi yalniz biraktilar. Kadin ise orta yerde duruyordu.
\par 10 Isa dogrulup ona, "Kadin, nerede onlar? Hiçbiri seni yargilamadi mi?" diye sordu.
\par 11 Kadin, "Hiçbiri, Efendim" dedi. Isa, "Ben de seni yargilamiyorum" dedi. "Git, artik bundan sonra günah isleme!"
\par 12 Isa yine halka seslenip söyle dedi: "Ben dünyanin isigiyim. Benim ardimdan gelen, asla karanlikta yürümez, yasam isigina sahip olur."
\par 13 Ferisiler, "Sen kendin için taniklik ediyorsun, tanikligin geçerli degil" dediler.
\par 14 Isa onlara su karsiligi verdi: "Kendim için taniklik etsem bile tanikligim geçerlidir. Çünkü nereden geldigimi ve nereye gidecegimi biliyorum. Oysa siz nereden geldigimi, nereye gidecegimi bilmiyorsunuz.
\par 15 Siz insan gözüyle yargiliyorsunuz. Ben kimseyi yargilamam.
\par 16 Yargilasam bile benim yargim dogrudur. Çünkü ben yalniz degilim, ben ve beni gönderen Baba, birlikte yargilariz.
\par 17 Yasanizda da, 'Iki kisinin tanikligi geçerlidir' diye yazilmistir.
\par 18 Kendim için taniklik eden bir ben varim, bir de beni gönderen Baba benim için taniklik ediyor."
\par 19 O zaman O'na, "Baban nerede?" diye sordular. Isa su karsiligi verdi: "Siz ne beni tanirsiniz, ne de Babam'i. Beni tanisaydiniz, Babam'i da tanirdiniz."
\par 20 Isa bu sözleri tapinakta ögretirken, bagis toplanan yerde söyledi. Kimse O'nu yakalamadi. Çünkü saati henüz gelmemisti.
\par 21 Isa yine onlara, "Ben gidiyorum. Beni arayacaksiniz ve günahinizin içinde öleceksiniz. Benim gidecegim yere siz gelemezsiniz" dedi.
\par 22 Yahudi yetkililer, "Yoksa kendini mi öldürecek?" dediler. "Çünkü, 'Benim gidecegim yere siz gelemezsiniz' diyor."
\par 23 Isa onlara, "Siz asagidansiniz, ben yukaridanim" dedi. "Siz bu dünyadansiniz, ben bu dünyadan degilim.
\par 24 Iste bu nedenle size, 'Günahlarinizin içinde öleceksiniz' dedim. Benim O olduguma iman etmezseniz, günahlarinizin içinde öleceksiniz."
\par 25 O'na, "Sen kimsin?" diye sordular. Isa, "Baslangiçtan beri size ne söyledimse, O'yum" dedi.
\par 26 "Sizinle ilgili söyleyecek ve sizleri yargilayacak çok seyim var. Beni gönderen gerçektir. Ben O'ndan isittiklerimi dünyaya bildiriyorum."
\par 27 Isa'nin kendilerine Baba'dan söz ettigini anlamadilar.
\par 28 Bu nedenle Isa söyle dedi: "Insanoglu'nu* yukari kaldirdiginiz zaman benim O oldugumu, kendiligimden hiçbir sey yapmadigimi, ama tipki Baba'nin bana ögrettigi gibi konustugumu anlayacaksiniz.
\par 29 Beni gönderen benimledir, O beni yalniz birakmadi. Çünkü ben her zaman O'nu hosnut edeni yaparim."
\par 30 Bu sözler üzerine birçoklari O'na iman etti.
\par 31 Isa kendisine iman etmis olan Yahudiler'e, "Eger benim sözüme bagli kalirsaniz, gerçekten ögrencilerim olursunuz. Gerçegi bileceksiniz ve gerçek sizi özgür kilacak" dedi.
\par 33 "Biz Ibrahim'in soyundaniz" diye karsilik verdiler, "Hiçbir zaman kimseye kölelik etmedik. Nasil oluyor da sen, 'Özgür olacaksiniz' diyorsun?"
\par 34 Isa, "Size dogrusunu söyleyeyim, günah isleyen herkes günahin kölesidir" dedi.
\par 35 "Köle ev halkinin sürekli bir üyesi degildir, ama ogul sürekli üyesidir.
\par 36 Bunun için, Ogul sizi özgür kilarsa, gerçekten özgür olursunuz.
\par 37 Ibrahim'in soyundan oldugunuzu biliyorum. Yine de beni öldürmek istiyorsunuz. Çünkü yüreginizde sözüme yer vermiyorsunuz.
\par 38 Ben Babam'in yaninda gördüklerimi söylüyorum, siz de babanizdan isittiklerinizi yapiyorsunuz."
\par 39 "Bizim babamiz Ibrahim'dir" diye karsilik verdiler. Isa, "Ibrahim'in çocuklari olsaydiniz, Ibrahim'in yaptiklarini yapardiniz" dedi.
\par 40 "Ama simdi beni Tanri'dan isittigi gerçegi sizlere bildireni öldürmek istiyorsunuz. Ibrahim bunu yapmadi.
\par 41 Siz babanizin yaptiklarini yapiyorsunuz." "Biz zinadan dogmadik. Bir tek Babamiz var, o da Tanri'dir" dediler.
\par 42 Isa, "Tanri Babaniz olsaydi, beni severdiniz" dedi. "Çünkü ben Tanri'dan çikip geldim. Kendiligimden gelmedim, beni O gönderdi.
\par 43 Söylediklerimi neden anlamiyorsunuz? Benim sözümü dinlemeye dayanamiyorsunuz da ondan.
\par 44 Siz babaniz Iblis'tensiniz ve babanizin arzularini yerine getirmek istiyorsunuz. O baslangiçtan beri katildi. Gerçege bagli kalmadi. Çünkü onda gerçek yoktur. Yalan söylemesi dogaldir. Çünkü o yalancidir ve yalanin babasidir.
\par 45 Ama ben gerçegi söylüyorum. Iste bunun için bana iman etmiyorsunuz.
\par 46 Hanginiz bana günahli oldugumu kanitlayabilir? Gerçegi söylüyorsam, niçin bana iman etmiyorsunuz?
\par 47 Tanri'dan olan, Tanri'nin sözlerini dinler. Iste siz Tanri'dan olmadiginiz için dinlemiyorsunuz."
\par 48 Yahudiler O'na su karsiligi verdiler: "'Sen, cin çarpmis bir Samiriyeli'sin*' demekte hakli degil miyiz?"
\par 49 Isa, "Beni cin çarpmadi" dedi. "Ben Babam'i onurlandiriyorum, ama siz beni asagiliyorsunuz.
\par 50 Ben kendimi yüceltmek istemiyorum, ama bunu isteyen ve yargilayan biri vardir.
\par 51 Size dogrusunu söyleyeyim, bir kimse sözüme uyarsa, ölümü asla görmeyecektir."
\par 52 Yahudiler, "Seni cin çarptigini simdi anliyoruz" dediler. "Ibrahim öldü, peygamberler de öldü. Oysa sen, 'Bir kimse sözüme uyarsa, ölümü asla tatmayacaktir' diyorsun.
\par 53 Yoksa sen babamiz Ibrahim'den üstün müsün? O öldü, peygamberler de öldü. Sen kendini kim saniyorsun?"
\par 54 Isa su karsiligi verdi: "Eger ben kendimi yüceltirsem, yüceligim hiçtir. Beni yücelten, 'Tanrimiz' diye çagirdiginiz Babam'dir.
\par 55 Siz O'nu tanimiyorsunuz, ama ben taniyorum. O'nu tanimadigimi söylersem, sizin gibi yalanci olurum. Ama ben O'nu taniyor ve sözüne uyuyorum.
\par 56 Babaniz Ibrahim günümü görecegi için sevinçle cosmustu. Gördü ve sevindi."
\par 57 Yahudiler, "Sen daha elli yasinda bile degilsin. Ibrahim'i de mi gördün?" dediler.
\par 58 Isa, "Size dogrusunu söyleyeyim, Ibrahim dogmadan önce ben varim" dedi.
\par 59 O zaman Isa'yi taslamak için yerden tas aldilar, ama O gizlenip tapinaktan çikti.

\chapter{9}

\par 1 Isa yolda giderken dogustan kör bir adam gördü.
\par 2 Ögrencileri Isa'ya, "Rabbî*, kim günah isledi de bu adam kör dogdu? Kendisi mi, yoksa annesi babasi mi?" diye sordular.
\par 3 Isa su yaniti verdi: "Ne kendisi, ne de annesi babasi günah isledi. Tanri'nin isleri onun yasaminda görülsün diye kör dogdu.
\par 4 Beni gönderenin islerini vakit daha gündüzken yapmaliyiz. Gece geliyor, o zaman kimse çalisamaz.
\par 5 Dünyada oldugum sürece dünyanin isigi Ben'im."
\par 6 Bu sözleri söyledikten sonra yere tükürdü, tükürükle çamur yapti ve çamuru adamin gözlerine sürdü.
\par 7 Adama, "Git, Siloah Havuzu'nda yikan" dedi. Siloah, gönderilmis anlamina gelir. Adam gidip yikandi, gözleri açilmis olarak döndü.
\par 8 Komsulari ve onu daha önce dilenirken görenler, "Oturup dilenen adam degil mi bu?" dediler.
\par 9 Kimi, "Evet, odur" dedi, kimi de "Hayir, ama ona benziyor" dedi. Kendisi ise, "Ben oyum" dedi.
\par 10 "Öyleyse, gözlerin nasil açildi?" diye sordular.
\par 11 O da söyle yanit verdi: "Isa adindaki adam çamur yapip gözlerime sürdü ve bana, 'Siloah'a git, yikan' dedi. Ben de gidip yikandim ve gözlerim açildi."
\par 12 Ona, "Nerede O?" diye sordular. "Bilmiyorum" dedi.
\par 13 Eskiden kör olan adami Ferisiler'in* yanina götürdüler.
\par 14 Isa'nin çamur yapip adamin gözlerini açtigi gün Sabat Günü'ydü*.
\par 15 Bu nedenle Ferisiler de adama gözlerinin nasil açildigini sordular. O da, "Isa gözlerime çamur sürdü, yikandim ve simdi görüyorum" dedi.
\par 16 Bunun üzerine Ferisiler'in bazilari, "Bu adam Tanri'dan degildir" dediler. "Çünkü Sabat Günü'nü tutmuyor." Ama baskalari, "Günahkâr bir adam nasil bu tür belirtiler gerçeklestirebilir?" dediler. Böylece aralarinda ayrilik dogdu.
\par 17 Eskiden kör olan adama yine sordular: "Senin gözlerini açtigina göre, O'nun hakkinda sen ne diyorsun?" Adam, "O bir peygamberdir" dedi.
\par 18 Yahudi yetkililer, gözleri açilan adamin annesiyle babasini çagirmadan onun daha önce kör olduguna ve gözlerinin açildigina inanmadilar.
\par 19 Onlara, "Kör dogdu dediginiz oglunuz bu mu? Peki, simdi nasil görüyor?" diye sordular.
\par 20 Adamin annesiyle babasi su karsiligi verdiler: "Bunun bizim oglumuz oldugunu ve kör dogdugunu biliyoruz.
\par 21 Ama simdi nasil gördügünü, gözlerini kimin açtigini bilmiyoruz, ona sorun. Ergin yastadir, kendisi için kendisi konussun."
\par 22 Yahudi yetkililerden korktuklari için böyle konustular. Çünkü yetkililer, Isa'nin Mesih* oldugunu açikça söyleyeni havra disi etmek için aralarinda sözbirligi etmislerdi.
\par 23 Bundan dolayi adamin annesiyle babasi, "Ergin yastadir, ona sorun" dediler.
\par 24 Eskiden kör olan adami ikinci kez çagirip, "Tanri hakki için dogruyu söyle" dediler, "Biz bu adamin günahkâr oldugunu biliyoruz."
\par 25 O da söyle yanit verdi: "O'nun günahkâr olup olmadigini bilmiyorum. Bildigim bir sey var, kördüm, simdi görüyorum."
\par 26 O zaman ona, "Sana ne yapti? Gözlerini nasil açti?" dediler.
\par 27 Onlara, "Size demin söyledim, ama dinlemediniz" dedi. "Niçin yeniden isitmek istiyorsunuz? Yoksa siz de mi O'nun ögrencileri olmak niyetindesiniz?"
\par 28 Adama söverek, "O'nun ögrencisi sensin!" dediler. "Biz Musa'nin ögrencileriyiz.
\par 29 Tanri'nin Musa'yla konustugunu biliyoruz. Ama bu adamin nereden geldigini bilmiyoruz."
\par 30 Adam onlara su karsiligi verdi: "Sasilacak sey! O'nun nereden geldigini bilmiyorsunuz, ama gözlerimi O açti.
\par 31 Tanri'nin, günahkârlari dinlemedigini biliriz. Ama Tanri, kendisine tapan ve istegini yerine getiren kisiyi dinler.
\par 32 Dünya var olali, bir kimsenin dogustan kör olan birinin gözlerini açtigi duyulmamistir.
\par 33 Bu adam Tanri'dan olmasaydi, hiçbir sey yapamazdi."
\par 34 Onlar buna karsilik, "Tamamen günah içinde dogdun, sen mi bize ders vereceksin?" diyerek onu disari attilar.
\par 35 Isa adami kovduklarini duydu. Onu bularak, "Sen Insanoglu'na* iman ediyor musun?" diye sordu.
\par 36 Adam su yaniti verdi: "Efendim, O kimdir? Söyle de kendisine iman edeyim."
\par 37 Isa, "O'nu gördün. Simdi seninle konusan O'dur" dedi.
\par 38 Adam, "Rab, iman ediyorum!" diyerek Isa'ya tapindi.
\par 39 Isa, "Görmeyenler görsün, görenler kör olsun diye yargiçlik etmek üzere bu dünyaya geldim" dedi.
\par 40 O'nun yaninda bulunan bazi Ferisiler bu sözleri isitince, "Yoksa biz de mi körüz?" diye sordular.
\par 41 Isa, "Kör olsaydiniz günahiniz olmazdi" dedi, "Ama simdi, 'Görüyoruz' dediginiz için günahiniz duruyor."

\chapter{10}

\par 1 "Size dogrusunu söyleyeyim, koyun agilina kapidan girmeyip baska yoldan giren kisi hirsiz ve hayduttur.
\par 2 Kapidan giren ise koyunlarin çobanidir.
\par 3 Kapiyi bekleyen ona kapiyi açar. Koyunlar çobanin sesini isitirler, o da kendi koyunlarini adlariyla çagirir ve onlari disari götürür.
\par 4 Kendi koyunlarinin hepsini disari çikarinca önlerinden gider, koyunlar da onu izler. Çünkü onun sesini tanirlar.
\par 5 Bir yabancinin pesinden gitmezler, ondan kaçarlar. Çünkü yabancilarin sesini tanimazlar."
\par 6 Isa onlara bu örnegi anlattiysa da, ne demek istedigini anlamadilar.
\par 7 Bunun için Isa yine, "Size dogrusunu söyleyeyim" dedi, "Ben koyunlarin kapisiyim.
\par 8 Benden önce gelenlerin hepsi hirsiz ve hayduttu, ama koyunlar onlari dinlemedi.
\par 9 Kapi Ben'im. Bir kimse benim araciligimla içeri girerse kurtulur. Girer, çikar ve otlak bulur.
\par 10 Hirsiz ancak çalip öldürmek ve yok etmek için gelir. Bense insanlar yasama, bol yasama sahip olsunlar diye geldim.
\par 11 Ben iyi çobanim. Iyi çoban koyunlari ugruna canini verir.
\par 12 Koyunlarin çobani ve sahibi olmayan ücretli adam, kurdun geldigini görünce koyunlari birakip kaçar. Kurt da onlari kapar ve dagitir.
\par 13 Adam kaçar. Çünkü ücretlidir ve koyunlar için kaygi duymaz.
\par 14 Ben iyi çobanim. Benimkileri tanirim. Baba beni tanidigi, ben de Baba'yi tanidigim gibi, benimkiler de beni tanir. Ben koyunlarimin ugruna canimi veririm.
\par 16 Bu agildan olmayan baska koyunlarim var. Onlari da getirmeliyim. Benim sesimi isitecekler ve tek sürü, tek çoban olacak.
\par 17 Canimi, tekrar geri almak üzere veririm. Bunun için Baba beni sever.
\par 18 Canimi kimse benden alamaz; ben onu kendiligimden veririm. Onu vermeye de tekrar geri almaya da yetkim var. Bu buyrugu Babam'dan aldim."
\par 19 Bu sözlerden dolayi Yahudiler arasinda yine ayrilik dogdu.
\par 20 Birçogu, "O'nu cin çarpmis, delidir. Niçin O'nu dinliyorsunuz?" diyordu.
\par 21 Baskalari ise, "Bunlar, cin çarpmis bir adamin sözleri degil" dediler. "Cin, körlerin gözlerini açabilir mi?"
\par 22 O sirada Yerusalim'de Tapinagin Açilisini Anma Bayrami* kutlaniyordu. Mevsim kisti.
\par 23 Isa tapinagin avlusunda, Süleyman'in Eyvani'nda yürüyordu.
\par 24 Yahudi yetkililer O'nun çevresini sararak, "Bizi daha ne kadar zaman kuskuda birakacaksin?" dediler. "Eger Mesih* isen, bize açikça söyle."
\par 25 Isa onlara su karsiligi verdi: "Size söyledim, ama iman etmiyorsunuz. Babam'in adiyla yaptigim isler bana taniklik ediyor.
\par 26 Ama siz iman etmiyorsunuz. Çünkü benim koyunlarimdan degilsiniz.
\par 27 Koyunlarim sesimi isitir. Ben onlari tanirim, onlar da beni izler.
\par 28 Onlara sonsuz yasam veririm; asla mahvolmayacaklar. Onlari hiç kimse elimden kapamaz.
\par 29 Onlari bana veren Babam her seyden üstündür. Onlari Baba'nin elinden kapmaya kimsenin gücü yetmez.
\par 30 Ben ve Baba biriz."
\par 31 Yahudi yetkililer O'nu taslamak için yerden yine tas aldilar.
\par 32 Isa onlara, "Size Baba'dan kaynaklanan birçok iyi isler gösterdim" dedi. "Bu islerden hangisi için beni tasliyorsunuz?"
\par 33 Söyle yanit verdiler: "Seni iyi islerden ötürü degil, küfür ettigin için tasliyoruz. Insan oldugun halde Tanri oldugunu ileri sürüyorsun."
\par 34 Isa su karsiligi verdi: "Yasanizda, 'Siz ilahlarsiniz, dedim' diye yazili degil mi?
\par 35 Tanri, kendilerine sözünü gönderdigi kimseleri ilahlar diye adlandirir. Kutsal Yazi da geçerliligini yitirmez.
\par 36 Baba beni kendine ayirip dünyaya gönderdi. Öyleyse 'Tanri'nin Oglu'yum' dedigim için bana nasil 'Küfür ediyorsun' dersiniz?
\par 37 Eger Babam'in islerini yapmiyorsam, bana iman etmeyin.
\par 38 Ama yapiyorsam, bana iman etmeseniz bile, yaptigim islere iman edin. Öyle ki, Baba'nin bende, benim de Baba'da oldugumu bilesiniz ve anlayasiniz."
\par 39 O'nu yine yakalamaya çalistilarsa da, ellerinden siyrilip kurtuldu.
\par 40 Tekrar Seria Irmagi'nin karsi yakasina, Yahya'nin baslangiçta vaftiz ettigi yere gitti ve orada kaldi.
\par 41 Birçoklari, "Yahya hiç mucize yapmadi, ama bu adam için söylediklerinin hepsi dogru çikti" diyerek Isa'ya geldiler.
\par 42 Ve orada birçoklari O'na iman etti.

\chapter{11}

\par 1 Meryem ile kizkardesi Marta'nin köyü olan Beytanya'dan Lazar adinda bir adam hastalanmisti.
\par 2 Meryem, Rab'be güzel kokulu yag sürüp saçlariyla O'nun ayaklarini silen kadindi. Hasta Lazar ise Meryem'in kardesiydi.
\par 3 Iki kizkardes Isa'ya, "Rab, sevdigin kisi hasta" diye haber gönderdiler.
\par 4 Isa bunu isitince, "Bu hastalik ölümle sonuçlanmayacak; Tanri'nin yüceligine, Tanri Oglu'nun yüceltilmesine hizmet edecek" dedi.
\par 5 Isa Marta'yi, kizkardesini ve Lazar'i severdi.
\par 6 Bu nedenle, Lazar'in hasta oldugunu duyunca bulundugu yerde iki gün daha kaldiktan sonra ögrencilere, "Yahudiye'ye dönelim" dedi.
\par 8 Ögrenciler, "Rabbî*" dediler, "Yahudi yetkililer demin seni taslamaya kalkistilar. Yine oraya mi gidiyorsun?"
\par 9 Isa su karsiligi verdi: "Günün on iki saati yok mu? Gündüz yürüyen sendelemez. Çünkü bu dünyanin isigini görür.
\par 10 Oysa gece yürüyen sendeler. Çünkü kendisinde isik yoktur."
\par 11 Bu sözleri söyledikten sonra, "Dostumuz Lazar uyudu" diye ekledi, "Onu uyandirmaya gidiyorum."
\par 12 Ögrenciler, "Ya Rab" dediler, "Uyuduysa iyilesecektir."
\par 13 Isa Lazar'in ölümünden söz ediyordu, ama onlar olagan uykudan söz ettigini sanmislardi.
\par 14 Bunun üzerine Isa açikça, "Lazar öldü" dedi.
\par 15 "Iman edesiniz diye, orada bulunmadigima sizin için seviniyorum. Simdi onun yanina gidelim."
\par 16 "Ikiz" diye anilan Tomas öbür ögrencilere, "Biz de gidelim, O'nunla birlikte ölelim!" dedi.
\par 17 Isa Beytanya'ya yaklasinca Lazar'in dört gündür mezarda oldugunu ögrendi.
\par 18 Beytanya, Yerusalim'e on bes ok atimi kadar uzakliktaydi.
\par 19 Birçok Yahudi, kardeslerini yitiren Marta'yla Meryem'i avutmaya gelmisti.
\par 20 Marta Isa'nin geldigini duyunca O'nu karsilamaya çikti, Meryem ise evde kaldi.
\par 21 Marta Isa'ya, "Ya Rab" dedi, "Burada olsaydin, kardesim ölmezdi.
\par 22 Simdi bile, Tanri'dan ne dilersen Tanri'nin onu sana verecegini biliyorum."
\par 23 Isa, "Kardesin dirilecektir" dedi.
\par 24 Marta, "Son gün, dirilis günü onun dirilecegini biliyorum" dedi.
\par 25 Isa ona, "Dirilis ve yasam Ben'im" dedi. "Bana iman eden kisi ölse de yasayacaktir.
\par 26 Yasayan ve bana iman eden asla ölmeyecek. Buna iman ediyor musun?"
\par 27 Marta, "Evet, ya Rab" dedi. "Senin, dünyaya gelecek olan Tanri'nin Oglu Mesih* olduguna iman ettim."
\par 28 Bunu söyledikten sonra gidip kizkardesi Meryem'i gizlice çagirdi. "Ögretmen burada, seni çagiriyor" dedi.
\par 29 Meryem bunu isitince hemen kalkip Isa'nin yanina gitti.
\par 30 Isa henüz köye varmamisti, hâlâ Marta'nin kendisini karsiladigi yerdeydi.
\par 31 Meryem'le birlikte evde bulunan ve kendisini teselli eden Yahudiler, onun hizla kalkip disari çiktigini gördüler. Aglamak için mezara gittigini sanarak onu izlediler.
\par 32 Meryem Isa'nin bulundugu yere vardi. O'nu görünce ayaklarina kapanarak, "Ya Rab" dedi, "Burada olsaydin, kardesim ölmezdi."
\par 33 Meryem'in ve onunla gelen Yahudiler'in agladigini gören Isa'nin ruhunu hüzün kapladi, yüregi sizladi.
\par 34 "Onu nereye koydunuz?" diye sordu. O'na, "Ya Rab, gel gör" dediler.
\par 35 Isa agladi.
\par 36 Yahudiler, "Bakin, onu ne kadar seviyormus!" dediler.
\par 37 Ama içlerinden bazilari, "Körün gözlerini açan bu kisi, Lazar'in ölümünü de önleyemez miydi?" dediler.
\par 38 Isa yine derinden hüzünlenerek mezara vardi. Mezar bir magaraydi, girisinde de bir tas duruyordu.
\par 39 Isa, "Tasi çekin!" dedi. Ölenin kizkardesi Marta, "Rab, o artik kokmustur, öleli dört gün oldu" dedi.
\par 40 Isa ona, "Ben sana, 'Iman edersen Tanri'nin yüceligini göreceksin' demedim mi?" dedi.
\par 41 Bunun üzerine tasi çektiler. Isa gözlerini gökyüzüne kaldirarak söyle dedi: "Baba, beni isittigin için sana sükrediyorum.
\par 42 Beni her zaman isittigini biliyordum. Ama bunu, çevrede duran halk için, beni senin gönderdigine iman etsinler diye söyledim."
\par 43 Bunlari söyledikten sonra yüksek sesle, "Lazar, disari çik!" diye bagirdi.
\par 44 Ölü, elleri ayaklari sargilarla bagli, yüzü peskirle sarilmis olarak disari çikti. Isa oradakilere, "Onu çözün, birakin gitsin" dedi.
\par 45 O zaman, Meryem'e gelen ve Isa'nin yaptiklarini gören Yahudiler'in birçogu Isa'ya iman etti.
\par 46 Ama içlerinden bazilari Ferisiler'e* giderek Isa'nin yaptiklarini onlara bildirdiler.
\par 47 Bunun üzerine baskâhinler ve Ferisiler, Yüksek Kurul'u* toplayip dediler ki, "Ne yapacagiz? Bu adam birçok dogaüstü belirti gerçeklestiriyor.
\par 48 Böyle devam etmesine izin verirsek, herkes O'na iman edecek. Romalilar da gelip kutsal yerimizi ve ulusumuzu ortadan kaldiracaklar."
\par 49 Içlerinden biri, o yil baskâhin olan Kayafa, "Hiçbir sey bilmiyorsunuz" dedi.
\par 50 "Bütün ulus yok olacagina, halk ugruna bir tek adamin ölmesi sizin için daha uygun. Bunu anlamiyor musunuz?"
\par 51 Bunu kendiliginden söylemiyordu. O yilin baskâhini olarak Isa'nin, ulusun ugruna, ve yalniz ulusun ugruna degil, Tanri'nin dagilmis çocuklarini toplayip birlestirmek için de ölecegine iliskin peygamberlikte bulunuyordu.
\par 53 Böylece o günden itibaren Isa'yi öldürmek için düzen kurmaya basladilar.
\par 54 Bu yüzden Isa artik Yahudiler arasinda açikça dolasmaz oldu. Oradan ayrilarak çöle yakin bir yere, Efrayim denilen kente gitti. Ögrencileriyle birlikte orada kaldi.
\par 55 Yahudiler'in Fisih Bayrami* yakindi. Tasradakilerin birçogu bayramdan önce arinmak için Yerusalim'e gitti.
\par 56 Orada Isa'yi arayip durdular. Tapinaktayken birbirlerine, "Ne dersiniz, bayrama hiç gelmeyecek mi?" diye soruyorlardi.
\par 57 Baskâhinlerle Ferisiler O'nu yakalayabilmek için, yerini bilenlerin haber vermesini buyurmuslardi.

\chapter{12}

\par 1 Isa, Fisih Bayrami'ndan* alti gün önce, ölümden dirilttigi Lazar'in bulundugu Beytanya'ya geldi.
\par 2 Orada kendisi için bir ziyafet düzenlediler. Marta hizmet ediyordu. Isa'yla birlikte sofrada oturanlardan biri de Lazar'di.
\par 3 Meryem, çok degerli saf hintsümbülü yagindan yarim litre kadar getirerek Isa'nin ayaklarina sürdü ve saçlariyla ayaklarini sildi. Ev yagin güzel kokusuyla doldu.
\par 4 Ama ögrencilerinden biri, Isa'ya sonradan ihanet eden Yahuda Iskariot, "Bu yag neden üç yüz dinara satilip parasi yoksullara verilmedi?" dedi.
\par 6 Bunu, yoksullarla ilgilendigi için degil, hirsiz oldugu için söylüyordu. Ortak para kutusu ondaydi ve kutuya konulandan asiriyordu.
\par 7 Isa, "Kadini rahat birak" dedi. "Bunu benim gömülecegim gün için saklasin.
\par 8 Yoksullar her zaman aranizdadir, ama ben her zaman aranizda olmayacagim."
\par 9 Yahudiler'den büyük bir kalabalik Isa'nin Beytanya'da bulundugunu ögrendi ve yalniz Isa için degil, O'nun ölümden dirilttigi Lazar'i da görmek için oraya geldi.
\par 10 Baskâhinler ise Lazar'i da öldürmeyi tasarladilar.
\par 11 Çünkü onun yüzünden birçok Yahudi gidip Isa'ya iman ediyordu.
\par 12 Ertesi gün, bayrami kutlamaya gelen büyük kalabalik Isa'nin Yerusalim'e gelmekte oldugunu duydu.
\par 13 Hurma dallari alarak O'nu karsilamaya çiktilar. "Hozana*! Rab'bin adiyla gelene, Israil'in Krali'na övgüler olsun!" diye bagiriyorlardi.
\par 14 Isa bir sipa bulup üzerine bindi. Yazilmis oldugu gibi, "Korkma, ey Siyon* kizi! Iste, Kralin sipaya binmis geliyor."
\par 16 Ögrencileri ilkin bunlari anlamadilar. Ama Isa yüceltildikten sonra bu sözlerin O'nun hakkinda yazildigini, halkin bunlari O'nun için yaptigini hatirladilar.
\par 17 Lazar'i mezardan çagirip ölümden dirilttigi sirada Isa'yla birlikte bulunan kalabalik buna taniklik etti.
\par 18 Isa'nin bu dogaüstü belirtiyi gerçeklestirdigini duyan halk O'nu karsilamaya çikti.
\par 19 Ferisiler* ise birbirlerine, "Görüyorsunuz, elinizden hiçbir sey gelmiyor. Bütün dünya O'nun pesine takildi" dediler.
\par 20 Bayramda tapinmak üzere Yerusalim'e gidenler arasinda bazi Grekler* vardi.
\par 21 Bunlar, Celile'nin Beytsayda Kenti'nden olan Filipus'a gelerek, "Efendimiz, Isa'yi görmek istiyoruz" diye rica ettiler.
\par 22 Filipus gitti, bunu Andreas'a bildirdi. Andreas ve Filipus da gidip Isa'ya haber verdiler.
\par 23 Isa, "Insanoglu'nun* yüceltilecegi saat geldi" diye karsilik verdi.
\par 24 "Size dogrusunu söyleyeyim, bugday tanesi topraga düsüp ölmedikçe yalniz kalir. Ama ölürse çok ürün verir.
\par 25 Canini seven onu yitirir. Ama bu dünyada canini gözden çikaran onu sonsuz yasam için koruyacaktir.
\par 26 Bana hizmet etmek isteyen, ardimdan gelsin. Ben neredeysem bana hizmet eden de orada olacak. Baba, bana hizmet edeni onurlandiracaktir.
\par 27 Simdi yüregim sikiliyor, ne diyeyim? 'Baba, beni bu saatten kurtar' mi diyeyim? Ama ben bu amaç için bu saate geldim.
\par 28 Baba, adini yücelt!" Bunun üzerine gökten bir ses geldi: "Adimi yücelttim ve yine yüceltecegim."
\par 29 Orada duran ve bunu isiten kalabalik, "Gök gürledi" dedi. Baskalari, "Bir melek O'nunla konustu" dedi.
\par 30 Isa, "Bu ses benim için degil, sizin içindi" dedi.
\par 31 "Bu dünya simdi yargilaniyor. Bu dünyanin egemeni simdi disari atilacak.
\par 32 Ben yerden yukari kaldirildigim zaman bütün insanlari kendime çekecegim."
\par 33 Isa bunu, nasil ölecegini belirtmek için söylüyordu.
\par 34 Kalabalik O'na söyle karsilik verdi: "Kutsal Yasa'dan ögrendigimize göre Mesih* sonsuza dek kalacaktir. Nasil oluyor da sen, 'Insanoglu yukari kaldirilmalidir' diyorsun? Kimdir bu Insanoglu?"
\par 35 Isa, "Isik kisa bir süre daha aranizdadir" dedi. "Karanlikta kalmamak için isiginiz varken yürüyün. Karanlikta yürüyen nereye gittigini bilmez.
\par 36 Sizde isik varken isiga iman edin ki, isik ogullari olasiniz." Isa bu sözleri söyledikten sonra uzaklasip onlardan gizlendi.
\par 37 Gözleri önünde bunca dogaüstü belirti gerçeklestirdigi halde O'na iman etmediler.
\par 38 Bütün bunlar Peygamber Yesaya'nin söyledigi su söz yerine gelsin diye oldu: "Rab, verdigimiz habere kim inandi? Rab'bin gücü kime açiklandi?"
\par 39 Iste bu yüzden iman edemiyorlardi. Nitekim Yesaya baska bir yerde de söyle demisti: "Tanri onlarin gözlerini kör etti Ve yüreklerini nasirlastirdi. Öyle ki, gözleri görmesin, Yürekleri anlamasin Ve bana dönmesinler. Dönselerdi, onlari iyilestirirdim."
\par 41 Bunlari söyleyen Yesaya, Isa'nin yüceligini görmüs ve O'nun hakkinda konusmustu.
\par 42 Bununla birlikte, önderlerin bile birçogu Isa'ya iman etti. Ama Ferisiler* yüzünden, havra disi edilmemek için iman ettiklerini açikça söylemediler.
\par 43 Çünkü insandan gelen övgüyü, Tanri'dan gelen övgüden daha çok seviyorlardi.
\par 44 Isa yüksek sesle, "Bana iman eden bana degil, beni gönderene iman etmis olur" dedi.
\par 45 "Beni gören beni göndereni de görür.
\par 46 Bana iman eden hiç kimse karanlikta kalmasin diye, dünyaya isik olarak geldim.
\par 47 Sözlerimi isitip de onlara uymayani ben yargilamam. Çünkü ben dünyayi yargilamaya degil, dünyayi kurtarmaya geldim.
\par 48 Beni reddeden ve sözlerimi kabul etmeyen kisiyi yargilayacak biri var. O kisiyi son günde yargilayacak olan, söyledigim sözdür.
\par 49 Çünkü ben kendiligimden konusmadim. Beni gönderen Baba'nin kendisi ne söylemem ve ne konusmam gerektigini bana buyurdu.
\par 50 O'nun buyrugunun sonsuz yasam oldugunu biliyorum. Bunun için ne söylüyorsam, Baba'nin bana söyledigi gibi söylüyorum."

\chapter{13}

\par 1 Fisih Bayrami'ndan* önceydi. Isa, bu dünyadan ayrilip Baba'ya gidecegi saatin geldigini biliyordu. Dünyada kendisine ait olanlari hep sevmisti; sonuna kadar da sevdi.
\par 2 Aksam yemegi sirasinda Iblis, Simun Iskariot'un oglu Yahuda'nin yüregine Isa'ya ihanet etme istegini koymustu bile.
\par 3 Isa, Baba'nin her seyi kendisine teslim ettigini, kendisinin Tanri'dan çikip geldigini ve Tanri'ya dönecegini biliyordu.
\par 4 Yemekten kalkti, üstlügünü bir yana koydu, bir havlu alip beline doladi.
\par 5 Sonra bir legene su doldurup ögrencilerin ayaklarini yikamaya ve beline doladigi havluyla kurulamaya basladi.
\par 6 Isa, Simun Petrus'a geldi. Simun, "Ya Rab, ayaklarimi sen mi yikayacaksin?" dedi.
\par 7 Isa ona su yaniti verdi: "Ne yaptigimi simdi anlayamazsin, ama sonra anlayacaksin."
\par 8 Petrus, "Benim ayaklarimi asla yikamayacaksin!" dedi. Isa, "Yikamazsam yanimda yerin olmaz" diye yanitladi.
\par 9 Simun Petrus, "Ya Rab, o halde yalniz ayaklarimi degil, ellerimi ve basimi da yika!" dedi.
\par 10 Isa ona dedi ki, "Yikanmis olan tamamen temizdir; ayaklarinin yikanmasindan baska seye ihtiyaci yoktur. Sizler temizsiniz, ama hepiniz degil."
\par 11 Isa, kendisine kimin ihanet edecegini biliyordu. Bu nedenle, "Hepiniz temiz degilsiniz" demisti.
\par 12 Onlarin ayaklarini yikadiktan sonra giyinip yine sofraya oturdu. "Size ne yaptigimi anliyor musunuz?" dedi.
\par 13 "Siz beni Ögretmen ve Rab diye çagiriyorsunuz. Dogru söylüyorsunuz, öyleyim.
\par 14 Ben Rab ve Ögretmen oldugum halde ayaklarinizi yikadim; öyleyse, sizler de birbirinizin ayaklarini yikamalisiniz.
\par 15 Size yaptigimin aynisini yapmaniz için bir örnek gösterdim.
\par 16 Size dogrusunu söyleyeyim, köle efendisinden, elçi de kendisini gönderenden üstün degildir.
\par 17 Bildiginiz bu seyleri yaparsaniz, ne mutlu size!"
\par 18 "Hepiniz için söylemiyorum, ben seçtiklerimi bilirim. Ama, 'Ekmegimi yiyen bana ihanet etti' diyen Kutsal Yazi'nin yerine gelmesi için böyle olacak.
\par 19 Size simdiden, bunlar olmadan önce söylüyorum ki, bunlar olunca, benim O olduguma inanasiniz.
\par 20 Size dogrusunu söyleyeyim, benim gönderdigim herhangi bir kimseyi kabul eden beni kabul etmis olur. Beni kabul eden de beni göndereni kabul etmis olur."
\par 21 Isa bunlari söyledikten sonra ruhunda derin bir sikinti duydu. Açikça konusarak, "Size dogrusunu söyleyeyim, sizden biri bana ihanet edecek" dedi.
\par 22 Ögrenciler, kimden söz ettigini merak ederek birbirlerine baktilar.
\par 23 Ögrencilerinden biri Isa'nin gögsüne yaslanmisti. Isa onu severdi.
\par 24 Simun Petrus bu ögrenciye, kimden söz ettigini Isa'ya sormasi için isaret etti.
\par 25 O da Isa'nin gögsüne yaslanmis durumda, "Ya Rab, kimdir o?" diye sordu.
\par 26 Isa, "Lokmayi sahana batirip kime verirsem odur" diye yanitladi. Sonra lokmayi batirip Simun Iskariot'un oglu Yahuda'ya verdi.
\par 27 Yahuda lokmayi alir almaz Seytan onun içine girdi. Isa da ona, "Yapacagini tez yap!" dedi.
\par 28 Sofrada oturanlarin hiçbiri, Isa'nin ona bu sözleri neden söyledigini anlamadi.
\par 29 Para kutusu Yahuda'da oldugundan, bazilari Isa'nin ona, "Bayram için bize gerekli seyleri al" ya da, "Yoksullara bir sey ver" demek istedigini sandilar.
\par 30 Yahuda lokmayi aldiktan hemen sonra disari çikti. Gece olmustu. Birbirinizi Sevin
\par 31 Yahuda disari çikinca Isa, "Insanoglu* simdi yüceltildi" dedi. "Tanri da O'nda yüceltildi.
\par 32 Tanri O'nda yüceltildigine göre, Tanri da O'nu kendinde yüceltecek. Hem de hemen yüceltecektir.
\par 33 Çocuklar! Kisa bir süre daha sizinleyim. Beni arayacaksiniz, ama Yahudiler'e söyledigim gibi, simdi size de söylüyorum, benim gidecegim yere siz gelemezsiniz.
\par 34 Size yeni bir buyruk veriyorum: Birbirinizi sevin. Sizi sevdigim gibi siz de birbirinizi sevin.
\par 35 Birbirinize sevginiz olursa, herkes bununla benim ögrencilerim oldugunuzu anlayacaktir."
\par 36 Simun Petrus O'na, "Ya Rab, nereye gidiyorsun?" diye sordu. Isa, "Gidecegim yere simdi ardimdan gelemezsin, ama sonra geleceksin" diye yanitladi.
\par 37 Petrus, "Ya Rab, neden simdi senin ardindan gelemeyeyim? Senin için canimi veririm!" dedi.
\par 38 Isa söyle yanitladi: "Benim için canini mi vereceksin? Sana dogrusunu söyleyeyim, horoz ötmeden beni üç kez inkâr edeceksin."

\chapter{14}

\par 1 "Yüreginiz sikilmasin. Tanri'ya iman edin, bana da iman edin.
\par 2 Babam'in evinde kalacak çok yer var. Öyle olmasa size söylerdim. Çünkü size yer hazirlamaya gidiyorum.
\par 3 Gider ve size yer hazirlarsam, siz de benim bulundugum yerde olasiniz diye yine gelip sizi yanima alacagim.
\par 4 Benim gidecegim yerin yolunu biliyorsunuz."
\par 5 Tomas, "Ya Rab, senin nereye gidecegini bilmiyoruz, yolu nasil bilebiliriz?" dedi.
\par 6 Isa, "Yol, gerçek ve yasam Ben'im" dedi. "Benim araciligim olmadan Baba'ya kimse gelemez.
\par 7 Beni tanisaydiniz, Babam'i da tanirdiniz. Artik O'nu taniyorsunuz, O'nu gördünüz."
\par 8 Filipus, "Ya Rab, bize Baba'yi göster, bu bize yeter" dedi.
\par 9 Isa, "Filipus" dedi, "Bunca zamandir sizinle birlikteyim. Beni daha tanimadin mi? Beni görmüs olan, Baba'yi görmüstür. Sen nasil, 'Bize Baba'yi göster' diyorsun?
\par 10 Benim Baba'da, Baba'nin da bende olduguna inanmiyor musun? Size söyledigim sözleri kendiligimden söylemiyorum, ama bende yasayan Baba kendi islerini yapiyor.
\par 11 Bana iman edin; ben Baba'dayim, Baba da bendedir. Hiç degilse bu islerden dolayi iman edin.
\par 12 Size dogrusunu söyleyeyim, benim yaptigim isleri, bana iman eden de yapacak; hatta daha büyüklerini yapacaktir. Çünkü ben Baba'ya gidiyorum.
\par 13 Baba Ogul'da yüceltilsin diye, benim adimla dilediginiz her seyi yapacagim.
\par 14 Benim adimla benden ne dilerseniz yapacagim."
\par 15 "Beni seviyorsaniz, buyruklarimi yerine getirirsiniz.
\par 16 Ben de Baba'dan dileyecegim. O sonsuza dek sizinle birlikte olsun diye size baska bir Yardimci*, Gerçegin Ruhu'nu verecek. Dünya O'nu kabul edemez. Çünkü O'nu ne görür, ne de tanir. Siz O'nu taniyorsunuz. Çünkü O aranizda yasiyor ve içinizde olacaktir.
\par 18 Sizi öksüz birakmayacagim, size geri dönecegim.
\par 19 Az sonra dünya artik beni görmeyecek, ama siz beni göreceksiniz. Ben yasadigim için siz de yasayacaksiniz.
\par 20 O gün anlayacaksiniz ki, ben Babam'dayim, siz bendesiniz, ben de sizdeyim.
\par 21 Kim buyruklarimi bilir ve yerine getirirse, iste beni seven odur. Beni seveni Babam da sevecektir. Ben de onu sevecegim ve kendimi ona gösterecegim."
\par 22 Yahuda Iskariot degil O'na, "Ya Rab, nasil olur da kendini dünyaya göstermeyip bize göstereceksin?" diye sordu.
\par 23 Isa ona su karsiligi verdi: "Beni seven sözüme uyar, Babam da onu sever. Biz de ona gelir, onunla birlikte yasariz.
\par 24 Beni sevmeyen, sözlerime uymaz. Isittiginiz söz benim degil, beni gönderen Baba'nindir.
\par 25 "Ben daha aranizdayken size bunlari söyledim.
\par 26 Ama Baba'nin benim adimla gönderecegi Yardimci, Kutsal Ruh, size her seyi ögretecek, bütün söylediklerimi size hatirlatacak.
\par 27 Size esenlik birakiyorum, size kendi esenligimi veriyorum. Ben size dünyanin verdigi gibi vermiyorum. Yüreginiz sikilmasin ve korkmasin.
\par 28 Size, 'Gidiyorum, ama yaniniza dönecegim' dedigimi isittiniz. Beni sevseydiniz, Baba'ya gidecegim için sevinirdiniz. Çünkü Baba benden üstündür.
\par 29 Bunlari size simdiden, her sey olup bitmeden önce söyledim. Öyle ki, bunlar olunca inanasiniz.
\par 30 Artik sizinle uzun uzun konusmayacagim. Çünkü bu dünyanin egemeni geliyor. Onun benim üzerimde hiçbir yetkisi yoktur.
\par 31 Ama dünyanin, Baba'yi sevdigimi ve Baba'nin bana buyurdugu her seyi yerine getirdigimi anlamasini istiyorum. Haydi kalkin, buradan gidelim."

\chapter{15}

\par 1 "Ben gerçek asmayim ve Babam bagcidir.
\par 2 Bende meyve vermeyen her çubugu kesip atar, meyve veren her çubugu ise daha çok meyve versin diye budayip temizler.
\par 3 Size söyledigim sözle siz simdiden temizsiniz.
\par 4 Bende kalin, ben de sizde kalayim. Çubuk asmada kalmazsa kendiliginden meyve veremez. Bunun gibi, siz de bende kalmazsaniz meyve veremezsiniz.
\par 5 Ben asmayim, siz çubuklarsiniz. Bende kalan ve benim kendisinde kaldigim kisi çok meyve verir. Bensiz hiçbir sey yapamazsiniz.
\par 6 Bir kimse bende kalmazsa, çubuk gibi disari atilir ve kurur. Böylelerini toplar, atese atip yakarlar.
\par 7 Eger bende kalirsaniz ve sözlerim sizde kalirsa, ne isterseniz dileyin, size verilecektir.
\par 8 Babam çok meyve vermenizle yüceltilir. Böylelikle ögrencilerim olursunuz.
\par 9 "Baba'nin beni sevdigi gibi, ben de sizi sevdim. Benim sevgimde kalin.
\par 10 Eger buyruklarimi yerine getirirseniz sevgimde kalirsiniz, tipki benim de Babam'in buyruklarini yerine getirdigim ve sevgisinde kaldigim gibi...
\par 11 Bunlari size, sevincim sizde olsun ve sevinciniz tamamlansin diye söyledim.
\par 12 Benim buyrugum sudur: Sizi sevdigim gibi birbirinizi sevin.
\par 13 Hiç kimsede, insanin, dostlari ugruna canini vermesinden daha büyük bir sevgi yoktur.
\par 14 Size buyurduklarimi yaparsaniz, benim dostlarim olursunuz.
\par 15 Artik size kul demiyorum. Çünkü kul efendisinin ne yaptigini bilmez. Size dost dedim. Çünkü Babam'dan bütün isittiklerimi size bildirdim.
\par 16 Siz beni seçmediniz, ben sizi seçtim. Gidip meyve veresiniz, meyveniz de kalici olsun diye sizi ben atadim. Öyle ki, benim adimla Baba'dan ne dilerseniz size versin.
\par 17 Size su buyrugu veriyorum: Birbirinizi sevin!"
\par 18 "Dünya sizden nefret ederse, sizden önce benden nefret etmis oldugunu bilin.
\par 19 Dünyadan olsaydiniz, dünya kendisine ait olani severdi. Ne var ki, dünyanin degilsiniz; ben sizi dünyadan seçtim. Bunun için dünya sizden nefret ediyor.
\par 20 Size söyledigim sözü hatirlayin: 'Köle efendisinden üstün degildir.' Bana zulmettilerse, size de zulmedecekler. Benim sözüme uydularsa, sizinkine de uyacaklar.
\par 21 Bütün bunlari size benim adimdan ötürü yapacaklar. Çünkü beni göndereni tanimiyorlar.
\par 22 Eger gelmemis ve onlara söylememis olsaydim, günahlari olmazdi; ama simdi günahlari için özürleri yoktur.
\par 23 Benden nefret eden, Babam'dan da nefret eder.
\par 24 Baska hiç kimsenin yapmadigi isleri onlarin arasinda yapmamis olsaydim, günahlari olmazdi. Simdiyse yaptiklarimi gördükleri halde hem benden hem de Babam'dan nefret ettiler.
\par 25 Bu, yasalarinda yazili, 'Yok yere benden nefret ettiler' sözü yerine gelsin diye oldu.
\par 26 "Baba'dan size gönderecegim Yardimci*, yani Baba'dan çikan Gerçegin Ruhu geldigi zaman, bana taniklik edecek.
\par 27 Siz de taniklik edeceksiniz. Çünkü baslangiçtan beri benimle birliktesiniz.

\chapter{16}

\par 1 "Bunlari size, sendeleyip düsmeyesiniz diye söyledim.
\par 2 Sizi havra disi edecekler. Evet, öyle bir saat geliyor ki, sizi öldüren herkes Tanri'ya hizmet ettigini sanacak.
\par 3 Bunlari, Baba'yi ve beni tanimadiklari için yapacaklar.
\par 4 Bunlari size simdiden bildiriyorum. Öyle ki, saati gelince bunlari size söyledigimi hatirlayasiniz. Baslangiçta bunlari size söylemedim. Çünkü sizinle birlikteydim."
\par 5 "Simdiyse beni gönderenin yanina gidiyorum. Ne var ki, içinizden hiçbiri bana, 'Nereye gidiyorsun?' diye sormuyor.
\par 6 Ama bunlari söyledigim için yüreginiz kederle doldu.
\par 7 Size gerçegi söylüyorum, benim gidisim sizin yararinizadir. Gitmezsem, Yardimci* size gelmez. Ama gidersem, O'nu size gönderirim.
\par 8 O gelince günah, dogruluk ve gelecek yargi konusunda dünyayi suçlu olduguna ikna edecektir:
\par 9 Günah konusunda, çünkü bana iman etmezler;
\par 10 dogruluk konusunda, çünkü Baba'ya gidiyorum, artik beni görmeyeceksiniz;
\par 11 yargi konusunda, çünkü bu dünyanin egemeni yargilanmis bulunuyor.
\par 12 "Size daha çok söyleyeceklerim var, ama simdi bunlara dayanamazsiniz.
\par 13 Ne var ki O, yani Gerçegin Ruhu gelince, sizi tüm gerçege yöneltecek. Çünkü kendiliginden konusmayacak, yalniz duyduklarini söyleyecek ve gelecekte olacaklari size bildirecek.
\par 14 O beni yüceltecek. Çünkü benim olandan alip size bildirecek.
\par 15 Baba'nin nesi varsa benimdir. 'Benim olandan alip size bildirecek' dememin nedeni budur.
\par 16 "Kisa süre sonra beni artik görmeyeceksiniz; yine kisa süre sonra beni göreceksiniz."
\par 17 Ögrencilerinden bazilari birbirlerine, "Ne demek istiyor?" diye sordular. "'Kisa süre sonra beni görmeyeceksiniz; yine kisa süre sonra beni göreceksiniz' diyor. Ayrica, 'Çünkü Baba'ya gidiyorum' diyor."
\par 18 Onun için, "Bu 'kisa süre' dedigi nedir? Söylediklerini anlamiyoruz" deyip durdular.
\par 19 Isa kendisine soru sormak istediklerini anladi. Onlara dedi ki, "'Kisa süre sonra beni görmeyeceksiniz; yine kisa süre sonra beni göreceksiniz' dememi mi tartisiyorsunuz?
\par 20 Size dogrusunu söyleyeyim, siz aglayip yas tutacaksiniz, dünya ise sevinecektir. Kederleneceksiniz, ama kederiniz sevince dönüsecek.
\par 21 Kadin dogum yapacagi zaman agri çeker. Çünkü saati gelmistir. Ama dogurunca, dünyaya bir çocuk getirmenin sevinciyle çektigi aciyi unutur.
\par 22 Bunun gibi, siz de simdi kederleniyorsunuz, ama sizi yine görecegim. O zaman yürekten sevineceksiniz. Sevincinizi kimse sizden alamaz.
\par 23 O gün bana hiçbir sey sormayacaksiniz. Size dogrusunu söyleyeyim, benim adimla Baba'dan ne dilerseniz, size verecektir.
\par 24 Simdiye dek benim adimla bir sey dilemediniz. Dileyin, alacaksiniz. Öyle ki, sevinciniz tam olsun.
\par 25 "Size bunlari örneklerle anlattim. Öyle bir saat geliyor ki, artik örneklerle konusmayacagim; Baba'yi size açikça tanitacagim.
\par 26 O gün dileyeceginizi benim adimla dileyeceksiniz. Sizin için Baba'dan istekte bulunacagimi söylemiyorum.
\par 27 Çünkü beni sevdiginiz ve Baba'dan çikip geldigime iman ettiginiz için Baba'nin kendisi sizi seviyor.
\par 28 Ben Baba'dan çikip dünyaya geldim. Simdi dünyayi birakip Baba'ya dönüyorum."
\par 29 Ögrencileri, "Iste, simdi açikça konusuyorsun, hiç örnek kullanmiyorsun" dediler.
\par 30 "Simdi senin her seyi bildigini anliyoruz. Kimsenin sana soru sormasina gerek yok. Tanri'dan geldigine bunun için iman ediyoruz."
\par 31 Isa onlara, "Simdi iman ediyor musunuz?" diye karsilik verdi.
\par 32 "Iste, hepinizin evlerinize gitmek üzere dagilacaginiz ve beni yalniz birakacaginiz saat geliyor, geldi bile. Ama ben yalniz degilim, Baba benimle birliktedir.
\par 33 Bunlari size, bende esenliginiz olsun diye söyledim. Dünyada sikintiniz olacak. Ama cesur olun, ben dünyayi yendim!"

\chapter{17}

\par 1 Isa bunlari söyledikten sonra, gözlerini gökyüzüne kaldirip söyle dedi: "Baba, saat geldi. Oglun'u yücelt ki, Ogul da seni yüceltsin.
\par 2 Çünkü sen O'na bütün insanlik üzerinde yetki verdin. Öyle ki, O'na verdiklerinin hepsine sonsuz yasam versin.
\par 3 Sonsuz yasam, tek gerçek Tanri olan seni ve gönderdigin Isa Mesih'i tanimalaridir.
\par 4 Yapmam için bana verdigin isi tamamlamakla seni yeryüzünde yücelttim.
\par 5 Baba, dünya var olmadan önce ben senin yanindayken sahip oldugum yücelikle simdi beni yaninda yücelt.
\par 6 "Dünyadan bana verdigin insanlara senin adini açikladim. Onlar senindiler, bana verdin ve senin sözüne uydular.
\par 7 Bana verdigin her seyin senden oldugunu simdi biliyorlar.
\par 8 Çünkü bana ilettigin sözleri onlara ilettim, onlar da kabul ettiler. Senden çikip geldigimi gerçekten anladilar, beni senin gönderdigine iman ettiler.
\par 9 Onlar için istekte bulunuyorum. Dünya için degil, bana verdigin kimseler için istekte bulunuyorum. Çünkü onlar senindir.
\par 10 Benim olan her sey senindir, seninkiler de benimdir. Ben onlarda yüceltildim.
\par 11 Ben artik dünyada degilim, ama onlar dünyadalar. Ben sana geliyorum. Kutsal Baba, onlari bana verdigin kendi adinla koru ki, bizim gibi bir olsunlar.
\par 12 Kendileriyle birlikte oldugum sürece, bana verdigin kendi adinla onlari esirgeyip korudum. Kutsal Yazi yerine gelsin diye, mahva giden adamdan baska içlerinden hiçbiri mahvolmadi.
\par 13 "Iste simdi sana geliyorum. Sevincimin onlarda tamamlanmasi için bunlari ben dünyadayken söylüyorum.
\par 14 Ben onlara senin sözünü ilettim, dünya ise onlardan nefret etti. Çünkü ben dünyadan olmadigim gibi, onlar da dünyadan degiller.
\par 15 Onlari dünyadan uzaklastirmani degil, kötü olandan* korumani istiyorum.
\par 16 Ben dünyadan olmadigim gibi, onlar da dünyadan degiller.
\par 17 Onlari gerçekle kutsal kil. Senin sözün gerçektir.
\par 18 Sen beni dünyaya gönderdigin gibi, ben de onlari dünyaya gönderdim.
\par 19 Onlar da gerçekle kutsal kilinsinlar diye kendimi onlarin ugruna adiyorum.
\par 20 "Yalniz onlar için degil, onlarin sözüyle bana iman edenler için de istekte bulunuyorum, hepsi bir olsunlar. Baba, senin bende oldugun ve benim sende oldugum gibi, onlar da bizde olsunlar. Dünya da beni senin gönderdigine iman etsin.
\par 22 Bana verdigin yüceligi onlara verdim. Öyle ki, bizim bir oldugumuz gibi bir olsunlar.
\par 23 Ben onlarda, sen bende olmak üzere tam bir birlik içinde bulunsunlar ki, dünya beni senin gönderdigini, beni sevdigin gibi onlari da sevdigini anlasin.
\par 24 Baba, bana verdiklerinin de bulundugum yerde benimle birlikte olmalarini ve benim yüceligimi, bana verdigin yüceligi görmelerini istiyorum. Çünkü dünyanin kurulusundan önce sen beni sevdin.
\par 25 Adil Baba, dünya seni tanimiyor, ama ben seni taniyorum. Bunlar da beni senin gönderdigini biliyorlar.
\par 26 Bana besledigin sevgi onlarda olsun, ben de onlarda olayim diye senin adini onlara bildirdim ve bildirmeye devam edecegim."

\chapter{18}

\par 1 Isa bu sözleri söyledikten sonra ögrencileriyle birlikte disari çikip Kidron Vadisi'nin ötesine geçti. Orada bir bahçe vardi. Isa'yla ögrencileri bu bahçeye girdiler.
\par 2 O'na ihanet eden Yahuda da burayi biliyordu. Çünkü Isa, ögrencileriyle orada sik sik bulusurdu.
\par 3 Böylece Yahuda yanina bir bölük askerle baskâhinlerin ve Ferisiler'in* gönderdigi görevlileri alarak oraya geldi. Onlarin ellerinde fenerler, mesaleler ve silahlar vardi.
\par 4 Isa, basina geleceklerin hepsini biliyordu. Öne çikip onlara, "Kimi ariyorsunuz?" diye sordu.
\par 5 "Nasirali Isa'yi" diye karsilik verdiler. Isa onlara, "Benim" dedi. O'na ihanet eden Yahuda da onlarla birlikte duruyordu.
\par 6 Isa, "Benim" deyince gerileyip yere düstüler.
\par 7 Bunun üzerine Isa onlara yine, "Kimi ariyorsunuz?" diye sordu. "Nasirali Isa'yi" dediler.
\par 8 Isa, "Size söyledim, benim" dedi. "Eger beni ariyorsaniz, bunlari birakin gitsinler."
\par 9 Kendisinin daha önce söyledigi, "Senin bana verdiklerinden hiçbirini yitirmedim" seklindeki sözü yerine gelsin diye böyle konustu.
\par 10 Simun Petrus yaninda tasidigi kilici çekti, baskâhinin Malkus adindaki kölesine vurup sag kulagini kopardi.
\par 11 Isa Petrus'a, "Kilicini kinina koy! Baba'nin bana verdigi kâseden* içmeyeyim mi?" dedi.
\par 12 Bunun üzerine komutanla buyrugundaki asker bölügü ve Yahudi görevliler Isa'yi tutup bagladilar.
\par 13 O'nu önce, o yil baskâhin olan Kayafa'nin kayinbabasi Hanan'a götürdüler.
\par 14 Halkin ugruna bir tek adamin ölmesinin daha uygun olacagini Yahudi yetkililere telkin eden Kayafa idi.
\par 15 Simun Petrus'la baska bir ögrenci Isa'nin ardindan gidiyorlardi. O ögrenci baskâhinin tanidigi oldugu için Isa'yla birlikte baskâhinin avlusuna girdi.
\par 16 Petrus ise disarida, kapinin yaninda duruyordu. Baskâhinin tanidigi ögrenci disari çikip kapici kizla konustu ve Petrus'u içeri getirdi.
\par 17 Kapici kiz Petrus'a, "Sen de bu adamin ögrencilerinden degil misin?" diye sordu. Petrus, "Hayir, degilim" dedi.
\par 18 Hava soguk oldugu için köleler ve nöbetçiler yaktiklari kömür atesinin çevresinde durmus isiniyorlardi. Petrus da onlarla birlikte ayakta isiniyordu.
\par 19 Baskâhin Isa'ya, ögrencileri ve ögretisiyle ilgili sorular sordu.
\par 20 Isa onu söyle yanitladi: "Ben söylediklerimi dünyaya açikça söyledim. Her zaman bütün Yahudiler'in toplandiklari havralarda ve tapinakta ögrettim. Gizli hiçbir sey söylemedim.
\par 21 Beni neden sorguya çekiyorsun? Konustuklarimi isitenlerden sor. Onlar ne söyledigimi biliyorlar."
\par 22 Isa bunlari söyleyince, yaninda duran görevlilerden biri, "Baskâhine nasil böyle karsilik verirsin?" diyerek O'na bir tokat atti.
\par 23 Isa ona, "Eger yanlis bir sey söyledimse, yanlisimi göster!" diye yanitladi. "Ama söylediklerim dogruysa, niçin bana vuruyorsun?"
\par 24 Bunun üzerine Hanan, O'nu bagli olarak baskâhin Kayafa'ya gönderdi.
\par 25 Simun Petrus hâlâ atesin yaninda durmus isiniyordu. O'na, "Sen de O'nun ögrencilerinden degil misin?" dediler. "Hayir, degilim" diyerek inkâr etti.
\par 26 Baskâhinin kölelerinden biri, Petrus'un, kulagini kestigi adamin akrabasiydi. Bu köle Petrus'a, "Bahçede, seni O'nunla birlikte görmedim mi?" diye sordu.
\par 27 Petrus yine inkâr etti ve tam o anda horoz öttü.
\par 28 Sabah erkenden Yahudi yetkililer Isa'yi Kayafa'nin yanindan alarak vali konagina götürdüler. Dinsel kurallari bozmamak ve Fisih* yemegini yiyebilmek için kendileri vali konagina girmediler.
\par 29 Bunun üzerine Pilatus disari çikip yanlarina geldi. "Bu adami neyle suçluyorsunuz?" diye sordu.
\par 30 Ona su karsiligi verdiler: "Bu adam kötülük eden biri olmasaydi, O'nu sana getirmezdik."
\par 31 Pilatus, "O'nu siz alin, kendi yasaniza göre yargilayin" dedi. Yahudi yetkililer, "Bizim hiç kimseyi ölüm cezasina çarptirmaya yetkimiz yok" dediler.
\par 32 Bu, Isa'nin nasil ölecegini belirtmek için söyledigi sözler yerine gelsin diye oldu.
\par 33 Pilatus yine vali konagina girdi. Isa'yi çagirip O'na, "Sen Yahudiler'in Krali misin?" diye sordu.
\par 34 Isa söyle karsilik verdi: "Bunu kendiliginden mi söylüyorsun, yoksa baskalari mi sana söyledi?"
\par 35 Pilatus, "Ben Yahudi miyim?" dedi. "Seni bana kendi ulusun ve baskâhinlerin teslim ettiler. Ne yaptin?"
\par 36 Isa, "Benim kralligim bu dünyadan degildir" diye karsilik verdi. "Kralligim bu dünyadan olsaydi, yandaslarim, Yahudi yetkililere teslim edilmemem için savasirlardi. Oysa benim kralligim buradan degildir."
\par 37 Pilatus, "Demek sen bir kralsin, öyle mi?" dedi. Isa, "Kral oldugumu sen söylüyorsun" karsiligini verdi. "Ben gerçege taniklik etmek için dogdum, bunun için dünyaya geldim. Gerçekten yana olan herkes benim sesimi isitir."
\par 38 Pilatus O'na, "Gerçek nedir?" diye sordu. Bunu söyledikten sonra Pilatus yine disariya, Yahudiler'in yanina çikti. Onlara, "Ben O'nda hiçbir suç görmüyorum" dedi.
\par 39 "Ama sizin bir geleneginiz var, her Fisih Bayrami'nda sizin için birini saliveriyorum. Yahudiler'in Krali'ni sizin için salivermemi ister misiniz?"
\par 40 Onlar yine, "Bu adami degil, Barabba'yi isteriz!" diye bagristilar. Oysa Barabba bir hayduttu.

\chapter{19}

\par 1 O zaman Pilatus Isa'yi tutup kamçilatti.
\par 2 Askerler de dikenlerden bir taç örüp O'nun basina geçirdiler. Sonra O'na mor bir kaftan giydirdiler.
\par 3 Önüne geliyor, "Selam, ey Yahudiler'in Krali!" diyor, yüzüne tokat atiyorlardi.
\par 4 Pilatus yine disari çikti. Yahudiler'e, "Iste, O'nu disariya, size getiriyorum. O'nda hiçbir suç bulmadigimi bilesiniz" dedi.
\par 5 Böylece Isa, basindaki dikenli taç ve üzerindeki mor kaftanla disari çikti. Pilatus onlara, "Iste o adam!" dedi.
\par 6 Baskâhinler ve görevliler Isa'yi görünce, "Çarmiha ger, çarmiha ger!" diye bagristilar. Pilatus, "O'nu siz alip çarmiha gerin!" dedi. "Ben O'nda bir suç bulamiyorum!"
\par 7 Yahudiler su karsiligi verdiler: "Bizim bir yasamiz var, bu yasaya göre O'nun ölmesi gerekir. Çünkü kendisinin Tanri Oglu oldugunu ileri sürüyor."
\par 8 Pilatus bu sözü isitince daha çok korktu.
\par 9 Yine vali konagina girip Isa'ya, "Sen nereden geliyorsun?" diye sordu. Isa ona yanit vermedi.
\par 10 Pilatus, "Benimle konusmayacak misin?" dedi. "Seni salivermeye de, çarmiha germeye de yetkim oldugunu bilmiyor musun?"
\par 11 Isa, "Sana gökten verilmeseydi, benim üzerimde hiçbir yetkin olmazdi" diye karsilik verdi. "Bu nedenle beni sana teslim edenin günahi daha büyüktür."
\par 12 Bunun üzerine Pilatus Isa'yi salivermek istedi. Ama Yahudiler, "Bu adami saliverirsen, Sezar'in* dostu degilsin!" diye bagristilar. "Kral oldugunu ileri süren herkes Sezar'a karsi gelmis olur."
\par 13 Pilatus bu sözleri isitince Isa'yi disari çikardi. Tas Döseme Ibranice'de* Gabbata denilen yerde yargi kürsüsüne oturdu.
\par 14 Fisih Bayrami'na* Hazirlik Günü'ydü*. Saat* on iki sulariydi. Pilatus Yahudiler'e, "Iste, sizin Kraliniz!" dedi.
\par 15 Onlar, "Yok et O'nu! Yok et, çarmiha ger!" diye bagristilar. Pilatus, "Kralinizi mi çarmiha gereyim?" diye sordu. Baskâhinler, "Sezar'dan baska kralimiz yok!" karsiligini verdiler.
\par 16 Bunun üzerine Pilatus Isa'yi, çarmiha gerilmek üzere onlara teslim etti.
\par 17 Askerler Isa'yi alip götürdüler. Isa çarmihini kendisi tasiyip Kafatasi Ibranice'de* Golgota denilen yere çikti.
\par 18 Orada O'nu ve iki kisiyi daha çarmiha gerdiler. Biri bir yanda, öbürü öteki yanda, Isa ise ortadaydi.
\par 19 Pilatus bir de yafta yazip çarmihin üzerine astirdi. Yaftada söyle yaziliydi: NASIRALI ISA YAHUDILER'IN KRALI
\par 20 Isa'nin çarmiha gerildigi yer kente yakindi. Böylece Ibranice, Latince ve Grekçe yazilan bu yaftayi Yahudiler'in birçogu okudu.
\par 21 Bu yüzden Yahudi baskâhinler Pilatus'a, "'Yahudiler'in Krali' diye yazma" dediler. "Kendisi, 'Ben Yahudiler'in Krali'yim dedi' diye yaz."
\par 22 Pilatus, "Ne yazdimsa yazdim" karsiligini verdi.
\par 23 Askerler Isa'yi çarmiha gerdikten sonra giysilerini alip her birine birer pay düsecek biçimde dört parçaya böldüler. Mintanini da aldilar. Mintan boydan boya tek parça dikissiz bir dokumaydi.
\par 24 Birbirlerine, "Bunu yirtmayalim" dediler, "Kime düsecek diye kura çekelim." Bu olay, su Kutsal Yazi yerine gelsin diye oldu: "Giysilerimi aralarinda paylastilar, Elbisem üzerine kura çektiler." Bunlari askerler yapti.
\par 25 Isa'nin çarmihinin yaninda ise annesi, teyzesi, Klopas'in karisi Meryem ve Mecdelli Meryem duruyordu.
\par 26 Isa, annesiyle sevdigi ögrencinin yakininda durdugunu görünce annesine, "Anne, iste oglun!" dedi.
\par 27 Sonra ögrenciye, "Iste, annen!" dedi. O andan itibaren bu ögrenci Isa'nin annesini kendi evine aldi.
\par 28 Daha sonra Isa, her seyin artik tamamlandigini bilerek Kutsal Yazi yerine gelsin diye, "Susadim!" dedi.
\par 29 Orada eksi sarap dolu bir kap vardi. Saraba batirilmis bir süngeri mercankösk dalina takarak O'nun agzina uzattilar.
\par 30 Isa sarabi tadinca, "Tamamlandi!" dedi ve basini egerek ruhunu teslim etti.
\par 31 Yahudi yetkililer Pilatus'tan çarmiha gerilenlerin bacaklarinin kirilmasini ve cesetlerin kaldirilmasini istediler. Hazirlik Günü* oldugundan, cesetlerin Sabat Günü* çarmihta kalmasini istemiyorlardi. Çünkü o Sabat Günü büyük bayramdi.
\par 32 Bunun üzerine askerler gidip birinci adamin, sonra da Isa'yla birlikte çarmiha gerilen öteki adamin bacaklarini kirdilar.
\par 33 Isa'ya gelince O'nun ölmüs oldugunu gördüler. Bu yüzden bacaklarini kirmadilar.
\par 34 Ama askerlerden biri O'nun bögrünü mizrakla deldi. Bögründen hemen kan ve su akti.
\par 35 Bunu gören adam taniklik etmistir ve tanikligi dogrudur. Dogruyu söyledigini bilir. Siz de iman edesiniz diye taniklik etmistir.
\par 36 Bunlar, "O'nun bir tek kemigi kirilmayacak" diyen Kutsal Yazi'nin yerine gelmesi için oldu.
\par 37 Yine baska bir Yazi'da, "Bedenini destiklerine bakacaklar" deniyor.
\par 38 Bundan sonra Aramatyali Yusuf, Isa'nin cesedini kaldirmak için Pilatus'a basvurdu. Yusuf, Isa'nin ögrencisiydi, ama Yahudi yetkililerden korktugundan bunu gizli tutuyordu. Pilatus izin verince, Yusuf gelip Isa'nin cesedini kaldirdi.
\par 39 Daha önce geceleyin Isa'nin yanina gelen Nikodim de otuz litre kadar karisik mür* ve sarisabir özü alarak geldi.
\par 40 Ikisi, Isa'nin cesedini alip Yahudiler'in gömme gelenegine uygun olarak onu baharatla keten bezlere sardilar.
\par 41 Isa'nin çarmiha gerildigi yerde bir bahçe, bu bahçenin içinde de henüz hiç kimsenin konulmadigi yeni bir mezar* vardi.
\par 42 O gün Yahudiler'in Hazirlik Günü'ydü*. Mezar da yakin oldugundan Isa'yi oraya koydular.

\chapter{20}

\par 1 Haftanin ilk günü* erkenden, ortalik daha karanlikken Mecdelli Meryem mezara gitti. Tasin mezarin girisinden kaldirilmis oldugunu gördü.
\par 2 Kosarak Simun Petrus'a ve Isa'nin sevdigi öbür ögrenciye geldi. "Rab'bi mezardan almislar, nereye koyduklarini da bilmiyoruz" dedi.
\par 3 Bunun üzerine Petrus'la öteki ögrenci disari çikip mezara yöneldiler.
\par 4 Ikisi birlikte kosuyordu. Ama öteki ögrenci Petrus'tan daha hizli kosarak mezara önce vardi.
\par 5 Egilip içeri bakti, keten bezleri orada serili gördü, ama içeri girmedi.
\par 6 Ardindan Simun Petrus geldi ve mezara girdi. Orada serili duran bezleri ve Isa'nin basina sarilmis olan peskiri gördü. Peskir keten bezlerle birlikte degildi, ayri bir yerde dürülmüs duruyordu.
\par 8 O zaman mezara ilk varan öteki ögrenci de içeri girdi. Olanlari gördü ve iman etti.
\par 9 Isa'nin ölümden dirilmesi gerektigini belirten Kutsal Yazi'yi henüz anlamamislardi.
\par 10 Bundan sonra ögrenciler yine evlerine döndüler.
\par 11 Meryem ise mezarin disinda durmus agliyordu. Aglarken egilip mezarin içine bakti.
\par 12 Beyazlara bürünmüs iki melek gördü; biri Isa'nin cesedinin yattigi yerin basucunda, öteki ayakucunda oturuyordu.
\par 13 Meryem'e, "Kadin, niçin agliyorsun?" diye sordular. Meryem, "Rabbim'i almislar" dedi. "O'nu nereye koyduklarini bilmiyorum."
\par 14 Bunlari söyledikten sonra arkasina döndü, Isa'nin orada, ayakta durdugunu gördü. Ama O'nun Isa oldugunu anlamadi.
\par 15 Isa, "Kadin, niçin agliyorsun?" dedi. "Kimi ariyorsun?" Meryem O'nu bahçivan sanarak, "Efendim" dedi, "Eger O'nu sen götürdünse, nereye koydugunu söyle de gidip O'nu alayim."
\par 16 Isa ona, "Meryem!" dedi. O da döndü, Isa'ya Ibranice*, "Rabbuni!" dedi. Rabbuni, ögretmenim demektir.
\par 17 Isa, "Bana dokunma!" dedi. "Çünkü daha Baba'nin yanina çikmadim. Kardeslerime git ve onlara söyle, benim Babam'in ve sizin Babaniz'in, benim Tanrim'in ve sizin Tanriniz'in yanina çikiyorum."
\par 18 Mecdelli Meryem ögrencilerin yanina gitti. Onlara, "Rab'bi gördüm!" dedi. Sonra Rab'bin kendisine söylediklerini onlara anlatti.
\par 19 Haftanin o ilk günü aksam olunca, ögrencilerin Yahudi yetkililerden korkusu nedeniyle bulunduklari yerin kapilari kapaliyken Isa geldi, ortalarinda durup, "Size esenlik olsun!" dedi.
\par 20 Bunu söyledikten sonra onlara ellerini ve bögrünü gösterdi. Ögrenciler Rab'bi görünce sevindiler.
\par 21 Isa yine onlara, "Size esenlik olsun!" dedi. "Baba beni gönderdigi gibi, ben de sizi gönderiyorum."
\par 22 Bunu söyledikten sonra onlarin üzerine üfleyerek, "Kutsal Ruh'u alin!" dedi.
\par 23 "Kimin günahlarini bagislarsaniz, bagislanmis olur; kimin günahlarini bagislamazsaniz, bagislanmamis kalir."
\par 24 Onikiler'den* biri, "Ikiz" diye anilan Tomas, Isa geldiginde onlarla birlikte degildi.
\par 25 Öbür ögrenciler ona, "Biz Rab'bi gördük!" dediler. Tomas ise, "O'nun ellerinde çivilerin izini görmedikçe, çivilerin izine parmagimla dokunmadikça ve elimi bögrüne sokmadikça inanmam" dedi.
\par 26 Sekiz gün sonra Isa'nin ögrencileri yine evdeydiler. Tomas da onlarla birlikteydi. Kapilar kapaliyken Isa gelip ortalarinda durdu, "Size esenlik olsun!" dedi.
\par 27 Sonra Tomas'a, "Parmagini uzat" dedi, "Ellerime bak, elini uzat, bögrüme koy. Imansiz olma, imanli ol!"
\par 28 Tomas O'na, "Rabbim ve Tanrim!" diye yanitladi.
\par 29 Isa, "Beni gördügün için mi iman ettin?" dedi. "Görmeden iman edenlere ne mutlu!"
\par 30 Isa, ögrencilerinin önünde, bu kitapta yazili olmayan baska birçok dogaüstü belirti gerçeklestirdi.
\par 31 Ne var ki yazilanlar, Isa'nin, Tanri'nin Oglu Mesih* olduguna iman edesiniz ve iman ederek O'nun adiyla yasama kavusasiniz diye yazilmistir.

\chapter{21}

\par 1 Bundan sonra Isa Taberiye Gölü'nün kenarinda ögrencilerine yine göründü. Bu da söyle oldu: Simun Petrus, "Ikiz" diye anilan Tomas, Celile'nin Kana Köyü'nden Natanel, Zebedi'nin ogullari ve Isa'nin ögrencilerinden iki kisi daha birlikte bulunuyorlardi.
\par 3 Simun Petrus ötekilere, "Ben balik tutmaya gidiyorum" dedi. Onlar, "Biz de seninle geliyoruz" dediler. Disari çikip tekneye bindiler. Ama o gece bir sey tutamadilar.
\par 4 Sabah olurken Isa kiyida duruyordu. Ne var ki ögrenciler, O'nun Isa oldugunu anlamadilar.
\par 5 Isa, "Çocuklar, baliginiz yok mu?" diye sordu. "Yok" dediler.
\par 6 Isa, "Agi teknenin sag yanina atin, tutarsiniz" dedi. Bunun üzerine agi attilar. O kadar çok balik tuttular ki, artik agi çekemez olmuslardi.
\par 7 Isa'nin sevdigi ögrenci, Petrus'a, "Bu Rab'dir!" dedi. Simun Petrus O'nun Rab oldugunu isitince üzerinden çikarmis oldugu üstlügü giyip göle atladi.
\par 8 Öbür ögrenciler balik dolu agi çekerek tekneyle geldiler. Çünkü karadan ancak iki yüz arsin kadar uzaktaydilar.
\par 9 Karaya çikinca orada yanan bir kömür atesi, atesin üzerinde balik ve ekmek gördüler.
\par 10 Isa onlara, "Simdi tuttugunuz baliklardan getirin" dedi.
\par 11 Simun Petrus tekneye atladi ve tam yüz elli üç iri balikla yüklü agi karaya çekti. Bu kadar çok balik oldugu halde ag yirtilmamisti.
\par 12 Isa onlara, "Gelin, yemek yiyin" dedi. Ögrencilerden hiçbiri O'na, "Sen kimsin?" diye sormaya cesaret edemedi. Çünkü O'nun Rab oldugunu biliyorlardi.
\par 13 Isa gidip ekmegi aldi, onlara verdi. Ayni sekilde baliklari da verdi.
\par 14 Iste bu, Isa'nin ölümden dirildikten sonra ögrencilere üçüncü görünüsüydü.
\par 15 Yemekten sonra Isa, Simun Petrus'a, "Yuhanna oglu Simun, beni bunlardan daha çok seviyor musun?" diye sordu. Petrus, "Evet, ya Rab" dedi, "Seni sevdigimi bilirsin." Isa ona, "Kuzularimi otlat" dedi.
\par 16 Ikinci kez yine ona, "Yuhanna oglu Simun, beni seviyor musun?" diye sordu. O da, "Evet, ya Rab, seni sevdigimi bilirsin" dedi. Isa ona, "Koyunlarimi güt" dedi.
\par 17 Üçüncü kez ona, "Yuhanna oglu Simun, beni seviyor musun?" diye sordu. Petrus kendisine üçüncü kez, "Beni seviyor musun?" diye sormasina üzüldü. "Ya Rab, sen her seyi bilirsin, seni sevdigimi de bilirsin" dedi. Isa ona, "Koyunlarimi otlat" dedi.
\par 18 "Sana dogrusunu söyleyeyim, gençliginde kendi kusagini kendin baglar, istedigin yere giderdin. Ama yaslaninca ellerini uzatacaksin, baskasi seni baglayacak ve istemedigin yere götürecek."
\par 19 Bunu, Tanri'yi ne tür bir ölümle yüceltecegini belirtmek için söyledi. Sonra ona, "Ardimdan gel" dedi.
\par 20 Petrus arkasina döndü, Isa'nin sevdigi ögrencinin kendilerini izledigini gördü. Bu ögrenci, aksam yemeginde Isa'nin gögsüne yaslanan ve, "Ya Rab, sana kim ihanet edecek?" diye soran ögrencidir.
\par 21 Petrus onu görünce Isa'ya, "Ya Rab, ya bu ne olacak?" diye sordu.
\par 22 Isa, "Ben gelinceye dek onun yasamasini istiyorsam, bundan sana ne?" dedi. "Sen ardimdan gel!"
\par 23 Bu yüzden kardesler arasinda o ögrencinin ölmeyecegine dair bir söylenti çikti. Ama Isa Petrus'a, "O ölmeyecek" dememisti. Sadece, "Ben gelinceye dek onun yasamasini istiyorsam, bundan sana ne?" demisti.
\par 24 Bütün bunlara taniklik eden ve bunlari yazan ögrenci budur. Onun tanikliginin dogru oldugunu biliyoruz.
\par 25 Isa'nin yaptigi daha baska çok sey vardir. Bunlar tek tek yazilsaydi, sanirim yazilan kitaplar dünyaya sigmazdi.


\end{document}