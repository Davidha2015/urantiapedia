\begin{document}

\title{Elçilerin işleri}


\chapter{1}

\par 1 Ey Teofilos, Ilk kitabimda Isa'nin yapip ögretmeye basladigi her seyi, seçmis oldugu elçilere Kutsal Ruh araciligiyla buyruklar verip yukari alindigi güne dek olanlari yazmistim.
\par 3 Isa, ölüm acisini çektikten sonra birçok inandirici kanitlarla elçilere dirilmis oldugunu gösterdi. Kirk gün süreyle onlara görünerek Tanri'nin Egemenligi* hakkinda konustu.
\par 4 Kendileriyle birlikteyken onlara su buyrugu vermisti: "Yerusalim'den* ayrilmayin, Baba'nin verdigi ve benden duydugunuz sözün gerçeklesmesini bekleyin.
\par 5 Söyle ki, Yahya suyla vaftiz* etti, ama sizler birkaç güne kadar Kutsal Ruh'la vaftiz edileceksiniz."
\par 6 Elçiler bir araya geldiklerinde Isa'ya sunu sordular: "Ya Rab, Israil'e egemenligi simdi mi geri vereceksin?"
\par 7 Isa onlara, "Baba'nin kendi yetkisiyle belirlemis oldugu zamanlari ve tarihleri bilmenize gerek yok" karsiligini verdi.
\par 8 "Ama Kutsal Ruh üzerinize inince güç alacaksiniz. Yerusalim'de, bütün Yahudiye ve Samiriye'de ve dünyanin dört bucaginda benim taniklarim olacaksiniz."
\par 9 Isa bunlari söyledikten sonra, onlarin gözleri önünde yukari alindi. Bir bulut O'nu alip gözlerinin önünden uzaklastirdi.
\par 10 Isa giderken onlar gözlerini göge dikmis bakiyorlardi. Tam o sirada, beyaz giysiler içinde iki adam yanlarinda belirdi.
\par 11 "Ey Celileliler, neden göge bakip duruyorsunuz?" diye sordular. "Aranizdan göge alinan Isa, göge çiktigini nasil gördünüzse, ayni sekilde geri gelecektir."
\par 12 Bundan sonra elçiler, Yerusalim'den yaklasik bir kilometre uzakliktaki Zeytin Dagi'ndan Yerusalim'e döndüler.
\par 13 Kente girince kaldiklari evin üst katindaki odaya çiktilar. Petrus, Yuhanna, Yakup, Andreas, Filipus, Tomas, Bartalmay, Matta, Alfay oglu Yakup, Yurtsever* Simun ve Yakup oglu Yahuda oradaydi.
\par 14 Bunlar Isa'nin annesi Meryem, öbür kadinlar ve Isa'nin kardesleriyle tam bir birlik içinde sürekli dua ediyordu.
\par 15 O günlerde Petrus, yaklasik yüz yirmi kardesten olusan bir toplulugun ortasinda ayaga kalkip söyle konustu: "Kardesler, Kutsal Ruh'un, Isa'yi tutuklayanlara kilavuzluk eden Yahuda ile ilgili olarak Davut'un agziyla önceden bildirdigi Kutsal Yazi'nin yerine gelmesi gerekiyordu.
\par 17 Yahuda bizden biri sayilmis ve bu hizmette yerini almisti."
\par 18 Bu adam, yaptigi kötülügün karsiliginda aldigi ücretle bir tarla satin aldi. Sonra bas asagi düstü, bedeni yarildi ve bütün bagirsaklari disari döküldü.
\par 19 Yerusalim'de yasayan herkes olayi duydu. Tarlaya kendi dillerinde Kan Tarlasi anlamina gelen Hakeldema adini verdiler.
\par 20 "Nitekim Mezmurlar Kitabi'nda söyle yazilmistir" dedi Petrus. "'Onun konutu issiz kalsin, Içinde oturan olmasin.' Ve, 'Onun görevini bir baskasi üstlensin.'
\par 21 "Buna göre, Yahya'nin vaftiz* döneminden baslayarak Rab Isa'nin aramizdan yukari alindigi güne degin bizimle birlikte geçirdigi bütün süre boyunca yanimizda bulunan adamlardan birinin, Isa'nin dirilisine taniklik etmek üzere bize katilmasi gerekir."
\par 23 Böylece iki kisiyi, Barsabba denilen ve Yustus diye de bilinen Yusuf ile Mattiya'yi önerdiler.
\par 24 Sonra söyle dua ettiler: "Ya Rab, sen herkesin yüregini bilirsin. Yahuda'nin, ait oldugu yere gitmek için biraktigi bu hizmeti ve elçilik görevini üstlenmek üzere bu iki kisiden hangisini seçtigini göster bize."
\par 26 Ardindan bu iki kisiye kura çektirdiler; kura Mattiya'ya düstü. Böylelikle Mattiya on bir elçiye katildi.

\chapter{2}

\par 1 Pentikost Günü geldiginde bütün imanlilar bir arada bulunuyordu.
\par 2 Ansizin gökten, güçlü bir rüzgarin esisini andiran bir ses geldi ve bulunduklari evi tümüyle doldurdu.
\par 3 Atesten dillere benzer bir seylerin dagilip her birinin üzerine indigini gördüler.
\par 4 Imanlilarin hepsi Kutsal Ruh'la doldular, Ruh'un onlari konusturdugu baska dillerle konusmaya basladilar.
\par 5 O sirada Yerusalim'de, dünyanin her ülkesinden gelmis dindar Yahudiler bulunuyordu.
\par 6 Sesin duyulmasi üzerine büyük bir kalabalik toplandi. Herkes kendi dilinin konusuldugunu duyunca sasakaldi.
\par 7 Hayret ve saskinlik içinde, "Bakin, bu konusanlarin hepsi Celileli degil mi?" diye sordular.
\par 8 "Nasil oluyor da her birimiz kendi ana dilini isitiyor?
\par 9 Aramizda Partlar, Medler, Elamlilar var. Mezopotamya'da, Yahudiye ve Kapadokya'da, Pontus ve Asya Ili'nde*, Frikya ve Pamfilya'da, Misir ve Libya'nin Kirene'ye yakin bölgelerinde yasayanlar var. Hem Yahudi hem de Yahudilige dönen Romali konuklar, Giritliler ve Araplar var aramizda. Ama her birimiz Tanri'nin büyük islerinin kendi dilimizde konusuldugunu isitiyoruz."
\par 12 Hepsi hayret ve saskinlik içinde birbirlerine, "Bunun anlami ne?" diye sordular.
\par 13 Baskalariysa, "Bunlar taze sarabi fazla kaçirmis" diye alay ettiler.
\par 14 Bunun üzerine Onbirler'le birlikte öne çikan Petrus yüksek sesle kalabaliga söyle seslendi: "Ey Yahudiler ve Yerusalim'de bulunan herkes, bu durumu size açiklayayim. Sözlerime kulak verin. Bu adamlar, sandiginiz gibi sarhos degiller. Saat* daha sabahin dokuzu!
\par 16 Bu gördügünüz, Peygamber Yoel araciligiyla önceden bildirilen olaydir: 'Son günlerde, diyor Tanri, Bütün insanlarin üzerine Ruhum'u dökecegim. Ogullariniz, kizlariniz peygamberlikte bulunacaklar. Gençleriniz görümler, Yaslilariniz düsler görecek.
\par 18 O günler kadin erkek Kullarimin üzerine Ruhum'u dökecegim, Onlar da peygamberlik edecekler.
\par 19 Yukarida, gökyüzünde harikalar yaratacagim. Asagida, yeryüzünde belirtiler, Kan, ates ve duman bulutlari görülecek.
\par 20 Rab'bin büyük ve görkemli günü gelmeden önce Günes kararacak, Ay kan rengine dönecek.
\par 21 O zaman Rab'be yakaran herkes kurtulacak.'
\par 22 "Ey Israilliler, su sözleri dinleyin: Bildiginiz gibi Nasirali Isa, Tanri'nin, kendisi araciligiyla aranizda yaptigi mucizeler, harikalar ve belirtilerle kimligi kanitlanmis bir kisidir.
\par 23 Tanri'nin belirlenmis amaci ve öngörüsü uyarinca elinize teslim edilen bu adami, yasa tanimaz kisilerin eliyle çarmiha çivileyip öldürdünüz.
\par 24 Tanri ise, ölüm acilarina son vererek O'nu diriltti. Çünkü O'nun ölüme tutsak kalmasi olanaksizdi.
\par 25 O'nunla ilgili olarak Davut söyle der: 'Rab'bi her zaman önümde gördüm, Sagimda durdugu için sarsilmam.
\par 26 Bu nedenle yüregim mutlu, dilim sevinçlidir. Dahasi, bedenim de umut içinde yasayacak.
\par 27 Çünkü sen canimi ölüler diyarina terk etmeyeceksin, Kutsalinin çürümesine izin vermeyeceksin.
\par 28 Yasam yollarini bana bildirdin; Varliginla beni sevinçle dolduracaksin.'
\par 29 "Kardesler, size açikça söyleyebilirim ki, büyük atamiz Davut öldü, gömüldü, mezari da bugüne dek yanibasimizda duruyor.
\par 30 Davut bir peygamberdi ve soyundan birini tahtina oturtacagina dair Tanri'nin kendisine ant içerek söz verdigini biliyordu.
\par 31 Gelecegi görerek Mesih'in* ölümden dirilisine iliskin sunlari söyledi: 'O, ölüler diyarina terk edilmedi, bedeni çürümedi.'
\par 32 Tanri, Isa'yi ölümden diriltti ve biz hepimiz bunun taniklariyiz.
\par 33 O, Tanri'nin sagina yüceltilmis, vaat edilen Kutsal Ruh'u Baba'dan almis ve simdi gördügünüz ve isittiginiz gibi, bu Ruh'u üzerimize dökmüstür.
\par 34 Davut, kendisi göklere çikmadigi halde söyle der: 'Rab Rabbim'e dedi ki, Ben düsmanlarini Ayaklarinin altina serinceye dek, Sagimda otur.'
\par 36 "Böylelikle bütün Israil halki sunu kesinlikle bilsin: Tanri, sizin çarmiha gerdiginiz Isa'yi hem Rab hem Mesih yapmistir."
\par 37 Bu sözleri duyanlar, yüreklerine hançer saplanmis gibi oldular. Petrus ve öbür elçilere, "Kardesler, ne yapmaliyiz?" diye sordular.
\par 38 Petrus onlara su karsiligi verdi: "Tövbe edin, her biriniz Isa Mesih'in adiyla vaftiz* olsun. Böylece günahlariniz bagislanacak ve Kutsal Ruh armaganini alacaksiniz.
\par 39 Bu vaat sizler, çocuklariniz, uzaktakilerin hepsi için, Tanrimiz Rab'bin çagiracagi herkes için geçerlidir."
\par 40 Petrus daha birçok sözlerle onlari uyardi. "Kendinizi bu sapik kusaktan kurtarin!" diye yalvardi.
\par 41 Onun sözünü benimseyenler vaftiz oldu. O gün yaklasik üç bin kisi topluluga katildi.
\par 42 Bunlar kendilerini elçilerin ögretisine, paydasliga, ekmek bölmeye ve duaya adadilar.
\par 43 Herkesi bir korku sarmisti. Elçilerin araciligiyla birçok belirtiler ve harikalar yapiliyordu.
\par 44 Imanlilarin tümü bir arada bulunuyor, her seyi ortaklasa kullaniyorlardi.
\par 45 Mallarini mülklerini satiyor ve bunun parasini herkese ihtiyacina göre dagitiyorlardi.
\par 46 Her gün tapinakta toplanmaya devam eden imanlilar, kendi evlerinde de ekmek bölüp içten bir sevinç ve sadelikle yemek yiyor ve Tanri'yi övüyorlardi. Bütün halkin begenisini kazanmislardi. Rab de her gün yeni kurtulanlari topluluga katiyordu.

\chapter{3}

\par 1 Bir gün Petrus'la Yuhanna, saat* üçte, dua vaktinde tapinaga çikiyorlardi.
\par 2 O sirada, dogustan kötürüm olan bir adam, tapinagin Güzel Kapi diye adlandirilan kapisina getiriliyordu. Tapinaga girenlerden para dilenmesi için onu her gün getirip oraya birakirlardi.
\par 3 Tapinaga girmek üzere olan Petrus'la Yuhanna'yi gören adam, kendilerinden sadaka istedi.
\par 4 Petrus'la Yuhanna ona dikkatle baktilar. Sonra Petrus, "Bize bak" dedi.
\par 5 Adam, onlardan bir sey alacagini umarak gözlerini onlarin üzerine dikti.
\par 6 Petrus, "Bende altin ve gümüs yok, ama bende olani sana veriyorum" dedi. "Nasirali Isa Mesih'in adiyla, yürü!"
\par 7 Sonra onu sag elinden kavrayip kaldirdi. Adamin ayaklari ve bilekleri o anda sapasaglam oldu.
\par 8 Siçrayip ayaga kalkti, yürümeye basladi. Yürüyüp siçrayarak, Tanri'yi överek onlarla birlikte tapinaga girdi.
\par 9 Bütün halk, onun yürüyüp Tanri'yi övdügünü gördü.
\par 10 Onun, tapinagin Güzel Kapisi'nda oturup para dilenen kisi oldugunu anlayinca ondaki degisiklik karsisinda büyük bir hayret ve saskinliga düstüler.
\par 11 Adam, Petrus'la Yuhanna'ya tutunuyordu. Bütün halk hayret içinde Süleyman'in Eyvani denilen yerde onlara dogru kosustu.
\par 12 Bunu gören Petrus halka söyle seslendi: "Ey Israilliler, buna neden sastiniz? Neden gözlerinizi dikmis bize bakiyorsunuz? Kendi gücümüz ya da dindarligimizla bu adamin yürümesini saglamisiz gibi...!
\par 13 Ibrahim'in, Ishak'in ve Yakup'un Tanrisi, atalarimizin Tanrisi, Kulu Isa'yi yüceltti. Siz O'nu ele verdiniz. Pilatus O'nu serbest birakmaya karar verdigi halde, siz O'nu Pilatus'un önünde reddettiniz.
\par 14 Kutsal ve adil Olan'i reddedip bir katilin saliverilmesini istediniz.
\par 15 Siz Yasam Önderi'ni öldürdünüz, ama Tanri O'nu ölümden diriltti. Biz bunun taniklariyiz.
\par 16 Gördügünüz ve tanidiginiz bu adam, Isa'nin adi sayesinde, O'nun adina olan imanla sapasaglam oldu. Hepinizin gözü önünde onu tam sagliga kavusturan, Isa'nin araciligiyla etkin olan imandir.
\par 17 "Simdi ey kardesler, yöneticileriniz gibi sizin de bilgisizlikten ötürü böyle davrandiginizi biliyorum.
\par 18 Ama bütün peygamberlerin agzindan Mesihi'nin* aci çekecegini önceden bildiren Tanri, sözünü bu sekilde yerine getirmistir.
\par 19 Öyleyse, günahlarinizin silinmesi için tövbe edin ve Tanri'ya dönün. Öyle ki, Rab size yenilenme firsatlari versin ve sizin için önceden belirlenen Mesih'i, yani Isa'yi göndersin.
\par 21 Tanri'nin eski çaglardan beri kutsal peygamberlerinin agzindan bildirdigi gibi, her seyin yeniden düzenlenecegi zamana dek Isa'nin gökte kalmasi gerekiyor.
\par 22 Musa söyle demisti: 'Tanriniz Rab size, kendi kardeslerinizin arasindan benim gibi bir peygamber çikaracak. O'nun size söyleyecegi her sözü dinleyin.
\par 23 O peygamberi dinlemeyen herkes Tanri'nin halkindan koparilip yok edilecektir.'
\par 24 "Samuel ve ondan sonra konusan peygamberlerin hepsi bu günleri duyurdu.
\par 25 Sizler peygamberlerin mirasçilari, Tanri'nin atalarinizla yaptigi antlasmanin mirasçilarisiniz. Nitekim Tanri Ibrahim'e söyle demisti: 'Senin soyunun araciligiyla yeryüzündeki bütün halklar kutsanacak.'
\par 26 Tanri, sizleri kötü yollarinizdan döndürüp kutsamak için Kulu'nu ortaya çikarip önce size gönderdi."

\chapter{4}

\par 1 Kâhinler, tapinak koruyucularinin komutani ve Sadukiler*, halka seslenmekte olan Petrus'la Yuhanna'nin üzerine yürüdüler.
\par 2 Çünkü onlarin halka ögretmelerine ve Isa'yi örnek göstererek ölülerin dirilecegini söylemelerine çok kizmislardi.
\par 3 Onlari yakaladilar, aksam oldugu için ertesi güne dek hapiste tuttular.
\par 4 Ne var ki, konusmayi dinlemis olanlarin birçogu iman etti. Böylece imanli erkeklerin sayisi asagi yukari bes bine ulasti.
\par 5 Ertesi gün Yahudiler'in yöneticileri, ileri gelenleri ve din bilginleri* Yerusalim'de toplandilar.
\par 6 Baskâhin Hanan'in yanisira, Kayafa, Yuhanna, Iskender ve baskâhin soyundan gelen herkes oradaydi.
\par 7 Petrus'la Yuhanna'yi huzurlarina getirtip onlara, "Siz bunu hangi güçle ya da kimin adina dayanarak yaptiniz?" diye sordular.
\par 8 O zaman Kutsal Ruh'la dolan Petrus onlara söyle dedi: "Halkin yöneticileri ve ileri gelenler!
\par 9 Eger bugün bir hastaya yapilan iyilik nedeniyle bizden hesap soruluyor ve bu adamin nasil iyilestigi sorusturuluyorsa, hepiniz ve bütün Israil halki sunu bilin: Bu adam, sizin çarmiha gerdiginiz, ama Tanri'nin ölümden dirilttigi Nasirali Isa Mesih'in adi sayesinde önünüzde sapasaglam duruyor.
\par 11 Isa,'Siz yapicilar tarafindan hiçe sayilan, Ama kösenin bas tasi durumuna gelen tas'tir.
\par 12 Baska hiç kimsede kurtulus yoktur. Bu gögün altinda insanlara bagislanmis, bizi kurtarabilecek baska hiçbir ad yoktur."
\par 13 Kurul üyeleri, Petrus'la Yuhanna'nin yürekliligini görüp de bunlarin egitim görmemis, siradan kisiler olduklarini anlayinca sastilar ve onlarin Isa'yla birlikte bulunduklarini farkettiler.
\par 14 Iyilestirilen adam, Petrus ve Yuhanna'yla birlikte gözleri önünde duruyordu; bunun için hiçbir karsilik veremediler.
\par 15 Kurul üyeleri onlara disari çikmalarini buyurduktan sonra durumu kendi aralarinda tartismaya basladilar.
\par 16 "Bu adamlari ne yapacagiz?" dediler. "Yerusalim'de yasayan herkes, bunlarin eliyle olaganüstü bir belirti gerçeklestirildigini biliyor. Biz bunu inkâr edemeyiz.
\par 17 Ama bu haberin halk arasinda daha çok yayilmasini önlemek için onlari tehdit edelim ki, bundan böyle Isa'nin adindan kimseye söz etmesinler."
\par 18 Böylece onlari çagirdilar, Isa'nin adini hiç anmamalarini, o adi kullanarak hiçbir sey ögretmemelerini buyurdular.
\par 19 Ama Petrus'la Yuhanna söyle karsilik verdiler: "Tanri'nin önünde, Tanri'nin sözünü degil de sizin sözünüzü dinlemek dogru mudur, kendiniz karar verin.
\par 20 Biz gördüklerimizi ve isittiklerimizi anlatmadan edemeyiz."
\par 21 Kurul üyeleri onlari bir daha tehdit ettikten sonra serbest biraktilar; onlari cezalandirmak için hiçbir gerekçe bulamamislardi. Çünkü bütün halk, olup bitenler için Tanri'yi yüceltiyordu.
\par 22 Nitekim mucize sonucu iyilesen adamin yasi kirki geçmisti.
\par 23 Serbest birakilan Petrus'la Yuhanna, arkadaslarinin yanina dönerek baskâhinlerle ileri gelenlerin kendilerine söyledigi her seyi bildirdiler.
\par 24 Arkadaslari bunu duyunca hep birlikte Tanri'ya söyle seslendiler: "Ey Efendimiz! Yeri gögü, denizi ve onlarin içindekilerin tümünü yaratan sensin.
\par 25 Kutsal Ruh araciligiyla kulun atamiz Davut'un agzindan söyle dedin: 'Uluslar neden hiddetlendi, Halklar neden bos düzenler kurdu?
\par 26 Dünyanin krallari saf bagladi, Hükümdarlar birlesti Rab'be ve Mesihi'ne karsi.'
\par 27 "Gerçekten de Hirodes* ile Pontius Pilatus, bu kentte Israil halki ve öteki uluslarla birlikte senin meshettigin* kutsal Kulun Isa'ya karsi bir araya geldiler. Senin kendi gücün ve isteginle önceden kararlastirdigin her seyi gerçeklestirdiler.
\par 29 Ve simdi ya Rab, onlarin savurdugu tehditlere bak! Senin sözünü tam bir yüreklilikle duyurmak için biz kullarina güç ver.
\par 30 Kutsal Kulun Isa'nin adiyla hastalari iyilestirmek için, belirtiler ve harikalar yapmak için elini uzat."
\par 31 Dualari bitince toplandiklari yer sarsildi. Hepsi Kutsal Ruh'la doldular ve Tanri'nin sözünü cesaretle duyurmaya devam ettiler.
\par 32 Inananlar toplulugunun yüregi ve düsüncesi birdi. Hiç kimse sahip oldugu herhangi bir sey için "Bu benimdir" demiyor, her seylerini ortak kabul ediyorlardi.
\par 33 Elçiler, Rab Isa'nin ölümden dirildigine çok etkili bir biçimde taniklik ediyorlardi. Tanri'nin büyük lütfu hepsinin üzerindeydi.
\par 34 Aralarinda yoksul olan yoktu. Çünkü toprak ya da ev sahibi olanlar bunlari satar, sattiklarinin bedelini getirip elçilerin buyruguna verirlerdi; bu da herkese ihtiyacina göre dagitilirdi.
\par 36 Örnegin, Kibris dogumlu bir Levili olan ve elçilerin Barnaba, yani Cesaret Verici diye adlandirdiklari Yusuf, sahip oldugu bir tarlayi satti, parasini getirip elçilerin buyruguna verdi.

\chapter{5}

\par 1 Hananya adinda bir adam, karisi Safira'nin onayiyla bir mülk satti, paranin bir kismini kendine saklayarak gerisini getirip elçilerin buyruguna verdi. Karisinin da olup bitenlerden haberi vardi.
\par 3 Petrus ona, "Hananya, nasil oldu da Seytan'a uydun, Kutsal Ruh'a yalan söyleyip tarlanin parasinin bir kismini kendine sakladin?" dedi.
\par 4 "Tarla satilmadan önce sana ait degil miydi? Sen onu sattiktan sonra da parayi diledigin gibi kullanamaz miydin? Neden yüreginde böyle bir düzen kurdun? Sen insanlara degil, Tanri'ya yalan söylemis oldun."
\par 5 Hananya bu sözleri isitince yere yikilip can verdi. Olanlari duyan herkesi büyük bir korku sardi.
\par 6 Gençler kalkip Hananya'nin ölüsünü kefenlediler ve disari tasiyip gömdüler.
\par 7 Bundan yaklasik üç saat sonra Hananya'nin karisi, olanlardan habersiz içeri girdi.
\par 8 Petrus, "Söyle bana, tarlayi bu fiyata mi sattiniz?" diye sordu. "Evet, bu fiyata" dedi Safira.
\par 9 Petrus ona söyle dedi: "Rab'bin Ruhu'nu sinamak için nasil oldu da sözbirligi ettiniz? Iste, kocani gömenlerin ayak sesleri kapida, seni de disari tasiyacaklar."
\par 10 Kadin o anda Petrus'un ayaklari dibine yikilip can verdi. Içeri giren gençler onu ölmüs buldular, onu da disari tasiyarak kocasinin yanina gömdüler.
\par 11 Inanlilar toplulugunun* tümünü ve olayi duyanlarin hepsini büyük bir korku sardi.
\par 12 Elçilerin araciligiyla halk arasinda birçok belirtiler ve harikalar yapiliyordu. Imanlilarin hepsi Süleyman'in Eyvani'nda toplaniyordu.
\par 13 Halk onlara büyük saygi duydugu halde, disaridan hiç kimse onlara katilmayi göze alamiyordu.
\par 14 Buna karsin, Rab'be inanip topluluga katilan erkek ve kadinlarin sayisi giderek artti.
\par 15 Bütün bunlarin sonucu, yoldan geçen Petrus'un hiç degilse gölgesi bazilarinin üzerine düssün diye halk, hasta olanlari caddelere çikartip silteler ve dösekler üzerine yatirir oldu.
\par 16 Yerusalim'in çevresindeki kasabalardan da kalabaliklar geliyor, hastalari ve kötü ruhlardan aci çekenleri getiriyorlardi. Bunlarin hepsi iyilestirildi.
\par 17 Bunun üzerine, kiskançlikla dolan baskâhin ve yanindakilerin hepsi, yani Saduki* mezhebinden olanlar, elçileri yakalatip devlet tutukevine attirdilar.
\par 19 Ama geceleyin Rab'bin bir melegi zindanin kapilarini açip onlari disari çikartti. "Gidin! Tapinaga girip bu yeni yasamla ilgili sözlerin hepsini halka duyurun" dedi.
\par 21 Elçiler bu buyruga uyarak gün dogarken tapinaga girip ögretmeye basladilar Baskâhin ve yanindakiler gelince Yüksek Kurul'u*, Israil halkinin bütün ileri gelenlerini toplantiya çagirdilar. Sonra elçileri getirtmek için tutukevine adam yolladilar.
\par 22 Ne var ki, görevliler zindana vardiklarinda elçileri bulamadilar. Geri dönerek su haberi ilettiler: "Tutukevini kilitli ve tam bir güvenlik altinda, nöbetçileri de kapilarda durur bulduk. Ama kapilari açtigimizda içerde kimseyi bulamadik!"
\par 24 Bu sözleri isiten tapinak koruyucularinin komutaniyla baskâhinler saskina döndüler, bu isin sonunun nereye varacagini merak etmeye basladilar.
\par 25 O sirada yanlarina gelen biri, "Bakin, hapse attiginiz adamlar tapinakta dikilmis, halka ögretiyor" diye haber getirdi.
\par 26 Bunun üzerine komutanla görevliler gidip elçileri getirdiler. Halkin kendilerini taslamasindan korktuklari için zor kullanmadilar.
\par 27 Elçileri getirip Yüksek Kurul'un önüne çikardilar. Baskâhin onlari sorguya çekti: "Bu adi kullanarak ögretmeyin diye size kesin buyruk vermistik" dedi. "Ama siz ögretinizi Yerusalim Kenti'nin her tarafina yaydiniz. Ille de bizi bu adamin kanini dökmekten sorumlu göstermek istiyorsunuz."
\par 29 Petrus ve öbür elçiler söyle karsilik verdiler: "Insanlardan çok, Tanri'nin sözünü dinlemek gerek.
\par 30 Atalarimizin Tanrisi, sizin çarmiha gererek öldürdügünüz Isa'yi diriltti.
\par 31 Israil'e, günahlarindan tövbe etme ve bagislanma firsatini vermek için Tanri O'nu Önder ve Kurtarici olarak kendi sagina yükseltti.
\par 32 Biz, Tanri'nin kendi sözünü dinleyenlere verdigi Kutsal Ruh'la birlikte bu olaylarin taniklariyiz."
\par 33 Kurul üyeleri bu sözleri isitince çok öfkelendiler ve elçileri yok etmek istediler.
\par 34 Ama bütün halkin saygisini kazanmis bir Kutsal Yasa ögretmeni olan Gamaliel adli bir Ferisi*, Yüksek Kurul'da ayaga kalkti, elçilerin kisa bir süre için disari çikartilmasini buyurarak kurul üyelerine sunlari söyledi: "Ey Israilliler, bu adamlara yapacaginizi iyi düsünün!
\par 36 Bir süre önce Tevdas da kendi kendisiyle ilgili büyük iddialarda bulunarak baskaldirdi. Dört yüz kadar kisi de ona katildi. Ama adam öldürüldü, izleyicilerinin hepsi dagitildi, hareket yok oldu.
\par 37 Ondan sonra, sayim yapildigi günlerde ortaya çikan Celileli Yahuda, pek çok insani ayartip pesine takti. Ama o da öldürüldü ve izleyicilerinin hepsi darmadagin oldu.
\par 38 Simdi size sunu söyleyeyim: Bu adamlarla ugrasmayin, onlari rahat birakin! Çünkü bu girisim, bu hareket insan isiyse, yok olup gidecektir.
\par 39 Yok eger Tanri'nin isiyse, bu adamlari yok edemezsiniz. Hatta kendinizi Tanri'ya karsi savasir durumda bulabilirsiniz." Kurul üyeleri Gamaliel'in bu ögüdünü kabul ettiler.
\par 40 Elçileri içeri çagirtip kamçilattilar ve Isa'nin adindan söz etmemelerini buyurduktan sonra saliverdiler.
\par 41 Elçiler Isa'nin adi ugruna hakarete layik görüldükleri için Yüksek Kurul'un huzurundan sevinç içinde ayrildilar.
\par 42 Her gün tapinakta ve evlerde ögretmekten ve Mesih Isa'yla ilgili Müjde'yi yaymaktan geri kalmadilar.

\chapter{6}

\par 1 Isa'nin ögrencilerinin sayica çogaldigi o günlerde, Grekçe konusan Yahudiler, günlük yardim dagitiminda kendi dullarina gereken ilginin gösterilmedigini ileri sürerek Ibranice* konusan Yahudiler'den yakinmaya basladilar.
\par 2 Bunun üzerine Onikiler*, bütün ögrencileri bir araya toplayip söyle dediler: "Tanri'nin sözünü yayma isini birakip maddi islerle ugrasmamiz dogru olmaz.
\par 3 Bu nedenle, kardesler, aranizdan Ruh'la ve bilgelikle dolu, yedi saygin kisi seçin. Onlari bu is için görevlendirelim.
\par 4 Biz ise kendimizi duaya ve Tanri sözünü yaymaya adayalim."
\par 5 Bu öneri bütün toplulugu hosnut etti. Böylece, iman ve Kutsal Ruh'la dolu biri olan Istefanos'un yanisira Filipus, Prohoros, Nikanor, Timon, Parmenas ve Yahudilige dönen Antakyali Nikolas'i seçip elçilerin önüne çikardilar. Elçiler de dua edip ellerini onlarin üzerine koydular.
\par 7 Böylece Tanri'nin sözü yayiliyor, Yerusalim'deki ögrencilerin sayisi arttikça artiyor, kâhinlerden birçogu da iman çagrisina uyuyordu.
\par 8 Tanri'nin lütfuyla ve kudretle dolu olan Istefanos, halk arasinda büyük belirtiler ve harikalar yapiyordu.
\par 9 Ne var ki, Azatlilar Havrasi diye bilinen havranin bazi üyeleri ve Kirene'den, Iskenderiye'den, Kilikya'dan ve Asya Ili'nden* bazi kisiler Istefanos'la çekismeye basladilar.
\par 10 Ama Istefanos'un konusmasindaki bilgelige ve Ruh'a karsi koyamadilar.
\par 11 Bunun üzerine birkaç kisiyi el altindan ayartarak onlara, "Bu adamin Musa'ya ve Tanri'ya karsi küfür dolu sözler söyledigini duyduk" dedirttiler.
\par 12 Böylelikle halki, ileri gelenleri ve din bilginlerini kiskirttilar. Gidip Istefanos'u yakaladilar ve Yüksek Kurul'un* önüne çikardilar.
\par 13 Getirdikleri yalanci taniklar, "Bu adam durmadan bu kutsal yere ve Yasa'ya karsi konusuyor" dediler.
\par 14 "'Nasirali Isa burayi yikacak, Musa'nin bize emanet ettigi töreleri de degistirecek' dedigini duyduk."
\par 15 Kurul'da oturanlarin hepsi, Istefanos'a baktiklarinda yüzünün bir melek yüzüne benzedigini gördüler.

\chapter{7}

\par 1 Baskâhin, "Bu iddialar dogru mu?" diye sordu.
\par 2 Istefanos söyle karsilik verdi: "Kardesler ve babalar, beni dinleyin. Atamiz Ibrahim daha Mezopotamya'dayken, Harran'a yerlesmeden önce, yüce Tanri ona görünüp söyle dedi: 'Ülkeni, akrabalarini birak, sana gösterecegim ülkeye git.'
\par 4 "Bunun üzerine Ibrahim Kildaniler'in* ülkesini birakip Harran'a yerlesti. Babasinin ölümünden sonra da Tanri onu oradan alip simdi sizin yasadiginiz bu ülkeye getirdi.
\par 5 Burada ona herhangi bir miras, bir karis toprak bile vermemisti. Ama Ibrahim'in o sirada hiç çocugu olmadigi halde, Tanri bu ülkeyi mülk olarak ona ve ondan sonra gelecek torunlarina verecegini vaat etti.
\par 6 Tanri söyle dedi: 'Senin soyun yabanci bir ülkede, gurbette yasayacak. Dört yüz yil köle olarak çalistirilacak, baski görecek.
\par 7 Ama ben kölelik edecekleri ulusu cezalandiracagim. Bundan sonra oradan çikacak ve bana bu yerde tapinacaklar.'
\par 8 Sonra Tanri onunla, sünnete dayali antlasmayi yapti. Böylelikle Ibrahim, Ishak'in babasi oldu ve onu sekiz günlükken sünnet etti. Ve Ishak Yakup'un, Yakup da on iki büyük atamizin babasi oldu.
\par 9 "Yusuf'u kiskanan atalarimiz, onu köle olarak Misir'a sattilar. Ama Tanri onunlaydi ve onu bütün sikintilarindan kurtardi. Ona bilgelik vererek Misir Firavunu'nun gözüne girmesini sagladi. Firavun da onu Misir ve bütün saray halki üzerine yönetici atadi.
\par 11 "Sonra bütün Misir ve Kenan ülkesini kitlik vurdu, büyük sikintilar basladi. Atalarimiz yiyecek bulamadilar.
\par 12 Misir'da tahil bulundugunu duyan Yakup, atalarimizi oraya ilk yolculuklarina gönderdi.
\par 13 Misir'a ikinci gelislerinde Yusuf kardeslerine kimligini açikladi. Firavun böylece Yusuf'un ailesini tanimis oldu.
\par 14 Yusuf haber yollayip babasi Yakup'u ve bütün akrabalarini, toplam yetmis bes kisiyi çagirtti.
\par 15 Böylece Yakup Misir'a gitti. Kendisi de atalarimiz da orada öldüler.
\par 16 Kemikleri sonra Sekem'e getirilerek Ibrahim'in Sekem'de Hamor ogullarindan bir miktar gümüs karsiliginda satin almis oldugu mezara konuldu.
\par 17 "Tanri'nin Ibrahim'e verdigi sözün gerçeklesecegi zaman yaklastiginda, Misir'daki halkimizin nüfusu bir hayli çogalmisti.
\par 18 Sonunda Yusuf hakkinda bilgisi olmayan yeni bir kral Misir'da tahta çikti.
\par 19 Bu adam, halkimiza karsi haince davrandi, atalarimiza kötülük etti. Onlari, yeni dogan çocuklarini açikta birakip ölüme terk etmeye zorladi.
\par 20 "O sirada, son derece güzel bir çocuk olan Musa dogdu. Musa, üç ay babasinin evinde beslendikten sonra açikta birakildi. Firavunun kizi onu bulup evlat edindi ve kendi oglu olarak yetistirdi.
\par 22 Musa, Misirlilar'in bütün bilim dallarinda egitildi. Gerek sözde, gerek eylemde güçlü biri oldu.
\par 23 "Kirk yasini doldurunca Musa'nin yüreginde öz kardesleri Israilogullari'nin durumunu yakindan görme arzusu dogdu.
\par 24 Onlardan birine haksizlik edildigini gören Musa, onu savundu. Haksizligi yapan Misirli'yi öldürerek ezilenin öcünü aldi.
\par 25 'Kardeslerim Tanri'nin benim araciligimla kendilerini kurtaracagini anlarlar' diye düsünüyordu. Ama onlar bunu anlamadilar.
\par 26 Ertesi gün Musa, kavga eden iki Ibrani'yle karsilasinca onlari baristirmak istedi. 'Efendiler' dedi, 'Siz kardessiniz. Niye birbirinize haksizlik ediyorsunuz?'
\par 27 "Ne var ki, soydasina haksizlik eden kisi Musa'yi yana iterek, 'Kim seni basimiza yönetici ve yargiç atadi?' dedi.
\par 28 'Yoksa dün Misirli'yi öldürdügün gibi beni de mi öldürmek istiyorsun?'
\par 29 Bu söz üzerine Musa Midyan ülkesine kaçti. Orada gurbette yasadi ve iki ogul babasi oldu.
\par 30 "Kirk yil geçtikten sonra Musa'ya, Sina Dagi'nin yakinlarindaki çölde, yanan bir çalinin alevleri içinde bir melek göründü.
\par 31 Musa gördüklerine sasti. Daha yakindan bakmak için yaklastiginda, Rab ona söyle seslendi: 'Senin atalarinin Tanrisi, Ibrahim'in, Ishak'in ve Yakup'un Tanrisi benim.' Korkuyla titreyen Musa bakmaya cesaret edemedi.
\par 33 "Sonra Rab, 'Çariklarini çikar! Çünkü bastigin yer kutsal topraktir' dedi.
\par 34 'Misir'da halkima yapilan baskiyi yakindan gördüm, iniltilerini duydum ve onlari kurtarmaya geldim. Simdi gel, seni Misir'a gönderecegim.'
\par 35 "Bu Musa, 'Kim seni yönetici ve yargiç atadi?' diye reddettikleri Musa'ydi. Tanri onu, çalida kendisine görünen melegin araciligiyla yönetici ve kurtarici olarak gönderdi.
\par 36 Halki Misir'dan çikaran, orada, Kizildeniz'de* ve kirk yil boyunca çölde belirtiler ve harikalar yapan oydu.
\par 37 Israilogullari'na, 'Tanri size kendi kardeslerinizin arasindan benim gibi bir peygamber çikaracak' diyen Musa odur.
\par 38 Çöldeki toplulugun arasinda yasamis, Sina Dagi'nda kendisiyle konusan melekle ve atalarimizla birlikte bulunmus olan odur. Bize iletmek üzere yasam dolu sözler aldi.
\par 39 "Ne var ki, atalarimiz onun sözünü dinlemek istemediler. Onu reddettiler, Misir'a dönmeyi özler oldular.
\par 40 Harun'a, 'Bize öncülük edecek ilahlar yap' dediler. 'Çünkü bizi Misir'dan çikaran o Musa'ya ne oldu bilmiyoruz!'
\par 41 Ve o günlerde buzagi biçiminde bir put yapip ona kurban sundular. Kendi elleriyle yaptiklari bu put için bir senlik düzenlediler.
\par 42 Bu yüzden Tanri onlardan yüz çevirip onlari göksel cisimlere kulluk etmeye terk etti. Peygamberlerin kitabinda yazilmis oldugu gibi: 'Ey Israil halki, Çölde kirk yil boyunca Bana mi sunular, kurbanlar sundunuz?
\par 43 Siz Molek'in* çadirini Ve ilahiniz Refan'in yildizini tasidiniz. Tapinmak için yaptiginiz putlardi bunlar. Bu yüzden sizi Babil'in ötesine sürecegim.'
\par 44 "Çölde atalarimizin Taniklik Çadiri vardi. Musa bunu, kendisiyle konusan Tanri'nin buyurdugu gibi, gördügü örnege göre yapmisti.
\par 45 Taniklik Çadiri'ni önceki kusaktan teslim alan atalarimiz, Yesu'nun önderliginde öteki uluslarin topraklarini ele geçirdikleri zaman, çadiri yanlarinda getirdiler. Uluslari atalarimizin önünden kovan, Tanri'nin kendisiydi. Çadir Davut'un zamanina dek kaldi.
\par 46 Tanri'nin begenisini kazanmis olan Davut, Yakup'un Tanrisi için bir konut yapmaya izin istedi.
\par 47 Oysa Tanri için bir ev yapan Süleyman oldu.
\par 48 "Ne var ki, en yüce Olan, elle yapilmis konutlarda oturmaz. Peygamberin belirttigi gibi, 'Gök tahtim, Yeryüzü ayaklarimin taburesidir. Benim için nasil bir ev yapacaksiniz? Ya da, neresi dinlenecegim yer? Bütün bunlari yapan elim degil mi? diyor Rab.'
\par 51 "Ey dik kafalilar, yürekleri ve kulaklari sünnet edilmemis olanlar! Siz tipki atalariniza benziyorsunuz, her zaman Kutsal Ruh'a karsi direniyorsunuz.
\par 52 Atalariniz peygamberlerin hangisine zulmetmediler ki? Adil Olan'in gelecegini önceden bildirenleri de öldürdüler. Melekler araciligiyla buyrulan Yasa'yi alip da buna uymayan sizler, simdi de Adil Olan'a ihanet edip O'nu katlettiniz!"
\par 54 Kurul üyeleri bu sözleri duyunca öfkeden kudurdular, Istefanos'a karsi dislerini gicirdattilar.
\par 55 Kutsal Ruh'la dolu olan Istefanos ise, gözlerini göge dikip Tanri'nin görkemini ve Tanri'nin saginda duran Isa'yi gördü.
\par 56 "Bakin" dedi, "Göklerin açildigini ve Insanoglu'nun* Tanri'nin saginda durmakta oldugunu görüyorum."
\par 57 Bunun üzerine kulaklarini tikayip çigliklar atarak hep birlikte Istefanos'a saldirdilar.
\par 58 Onu kentten disari atip tasa tuttular. Istefanos'a karsi taniklik etmis olanlar, kaftanlarini Saul adli bir gencin ayaklarinin dibine biraktilar.
\par 59 Istefanos tas yagmuru altinda, "Rab Isa, ruhumu al!" diye yakariyordu.
\par 60 Sonra diz çökerek yüksek sesle söyle dedi: "Ya Rab, bu günahi onlara yükleme!" Bunu söyledikten sonra gözlerini yasama kapadi.

\chapter{8}

\par 1 Istefanos'un öldürülmesini Saul da onaylamisti. O gün Yerusalim'deki kiliseye* karsi korkunç bir baski dönemi basladi. Elçiler hariç bütün imanlilar Yahudiye ve Samiriye'nin her yanina dagildilar.
\par 2 Bazi dindar kisiler, Istefanos'u gömdükten sonra onun için büyük yas tuttular.
\par 3 Saul ise inanlilar toplulugunu* kirip geçiriyordu. Ev ev dolasarak, kadin erkek demeden imanlilari disari sürüklüyor, hapse atiyordu.
\par 4 Bunun sonucu dagilan imanlilar, gittikleri her yerde Tanri sözünü müjdeliyorlardi.
\par 5 Filipus, Samiriye* Kenti'ne gidip oradakilere Mesih'i* tanitmaya basladi.
\par 6 Filipus'u dinleyen ve gerçeklestirdigi belirtileri gören kalabaliklar, hep birlikte onun söylediklerine kulak verdiler.
\par 7 Birçoklarinin içinden kötü ruhlar yüksek sesle haykirarak çikti; birçok felçli ve kötürüm iyilestirildi.
\par 8 Ve o kentte büyük sevinç oldu.
\par 9 Ne var ki, kentte bir süreden beri büyücülük yapan ve Samiriye halkini saskina çeviren Simun adli biri vardi. Simun, büyük adam oldugunu iddia ediyordu.
\par 10 Küçük büyük, herkes onu dikkatle dinler, "Büyük Güç dedikleri Tanri gücü iste budur" derlerdi.
\par 11 Uzun zamandan beri onlari büyücülügüyle saskina çevirdigi için onu dikkatle dinlerlerdi.
\par 12 Ama Tanri'nin Egemenligi* ve Isa Mesih adiyla ilgili Müjde'yi duyuran Filipus'un söylediklerine inandiklari zaman, erkekler de kadinlar da vaftiz* oldular.
\par 13 Simun'un kendisi de inanip vaftiz oldu. Ondan sonra sürekli olarak Filipus'un yaninda kaldi. Dogaüstü belirtileri ve yapilan büyük mucizeleri görünce saskina döndü.
\par 14 Yerusalim'deki elçiler, Samiriye halkinin, Tanri'nin sözünü benimsedigini duyunca Petrus'la Yuhanna'yi onlara gönderdiler.
\par 15 Petrus'la Yuhanna oraya varinca, Samiriyeli imanlilarin Kutsal Ruh'u almalari için dua ettiler.
\par 16 Çünkü Ruh daha hiçbirinin üzerine inmemisti. Rab Isa'nin adiyla vaftiz olmuslardi, o kadar.
\par 17 Petrus'la Yuhanna onlarin üzerine ellerini koyunca, onlar da Kutsal Ruh'u aldilar.
\par 18 Elçilerin bu el koyma hareketiyle Kutsal Ruh'un verildigini gören Simun onlara para teklif ederek, "Bana da bu yetkiyi verin, kimin üzerine ellerimi koysam Kutsal Ruh'u alsin" dedi.
\par 20 Petrus, "Paran da yok olsun, sen de!" dedi, "Çünkü Tanri'nin armaganini parayla elde edebilecegini sandin.
\par 21 Senin bu iste bir payin, bir hakkin yok. Yüregin, Tanri'nin gözünde dogru degildir.
\par 22 Bu kötülügünden tövbe et ve Rab'be yalvar, yüregindeki bu düsünce belki bagislanir.
\par 23 Senin kin dolu, kötülüge tutsak biri oldugunu görüyorum."
\par 24 Simun, "Benim için Rab'be yalvarin da söylediklerinizden hiçbiri basima gelmesin" diye karsilik verdi.
\par 25 Petrus'la Yuhanna taniklik edip Rab'bin sözünü bildirdikten sonra, Samiriye'nin birçok köyünde de Müjde'yi duyura duyura Yerusalim'e döndüler.
\par 26 Bu arada Rab'bin bir melegi Filipus'a söyle seslendi: "Kalk, güneye dogru, Yerusalim'den Gazze'ye inen yola, çöl yoluna git."
\par 27 Filipus da kalkip gitti. Giderken Etiyopyali bir hadim gördü. Bu adam Etiyopya Kraliçesi Kandaki'nin vezirlerinden biriydi. Kraliçenin bütün hazinelerinden sorumluydu. Yerusalim'e, tapinmaya gelmisti.
\par 28 Geri dönerken arabasinda oturmus, Peygamber Yesaya'nin Kitabi'ni okuyordu.
\par 29 Ruh Filipus'a, "Git" dedi, "Su arabaya yetis."
\par 30 Filipus kosup arabanin yanina geldi ve hadimin Peygamber Yesaya'yi okumakta oldugunu isitti. "Acaba okuduklarini anliyor musun?" diye sordu.
\par 31 Hadim, "Biri bana yol göstermedikçe nasil anlayabilirim ki?" diyerek Filipus'un arabaya binip yanina oturmasini rica etti.
\par 32 Kutsal Yazilar'dan okudugu bölüm suydu: "Koyun gibi kesime götürüldü; Kirkicinin önünde kuzu nasil ses çikarmazsa, O da öylece agzini açmadi.
\par 33 Asagilandiginda adalet O'ndan esirgendi. O'nun soyunu kim anacak? Çünkü yeryüzündeki yasamina son verildi."
\par 34 Hadim Filipus'a, "Lütfen açiklar misin, peygamber kimden söz ediyor, kendisinden mi, bir baskasindan mi?" diye sordu.
\par 35 Bunun üzerine Filipus anlatmaya koyuldu. Kutsal Yazilar'in bu bölümünden baslayarak ona Isa'yla ilgili Müjde'yi bildirdi.
\par 36 Yolda giderlerken su bulunan bir yere geldiler. Hadim, "Bak, burada su var" dedi. "Vaftiz* olmama ne engel var?"
\par 38 Sonra arabanin durmasini buyurdu. Filipus'la hadim birlikte suya girdiler ve Filipus hadimi vaftiz etti.
\par 39 Sudan çiktiklari zaman Rab'bin Ruhu Filipus'u hemen oradan uzaklastirdi. Filipus'u bir daha görmeyen hadim sevinç içinde yoluna devam etti.
\par 40 Filipus ise kendini Asdot Kenti'nde buldu. Sezariye'ye varincaya dek bütün kentleri dolasarak Müjde'yi duyurdu.

\chapter{9}

\par 1 Saul ise Rab'bin ögrencilerine karsi hâlâ tehdit ve ölüm soluyordu. Baskâhine gitti, Sam'daki havralara verilmek üzere mektuplar yazmasini istedi. Orada Isa'nin yolunda yürüyen kadin erkek, kimi bulsa tutuklayip Yerusalim'e getirmek niyetindeydi.
\par 3 Yol alip Sam'a yaklastigi sirada, birdenbire gökten gelen bir isik çevresini aydinlatti.
\par 4 Yere yikilan Saul, bir sesin kendisine, "Saul, Saul, neden bana zulmediyorsun?" dedigini isitti.
\par 5 Saul, "Ey Efendim, sen kimsin?" dedi. "Ben senin zulmettigin Isa'yim" diye yanit geldi.
\par 6 "Haydi kalk ve kente gir, ne yapman gerektigi sana bildirilecek."
\par 7 Saul'la birlikte yolculuk eden adamlarin dilleri tutuldu, olduklari yerde kalakaldilar. Sesi duydularsa da, kimseyi göremediler.
\par 8 Saul yerden kalkti, ama gözlerini açtiginda hiçbir sey göremiyordu. Sonra kendisini elinden tutup Sam'a götürdüler.
\par 9 Üç gün boyunca gözleri görmeyen Saul hiçbir sey yiyip içmedi.
\par 10 Sam'da Hananya adinda bir Isa ögrencisi vardi. Bir görümde Rab ona, "Hananya!" diye seslendi. "Buradayim, ya Rab" dedi Hananya.
\par 11 Rab ona, "Kalk" dedi, "Dogru Sokak denilen sokaga git ve Yahuda'nin evinde Saul adinda Tarsuslu birini sor. Su anda orada dua ediyor.
\par 12 Görümünde yanina Hananya adli birinin geldigini ve gözlerini açmak için ellerini kendisinin üzerine koydugunu görmüstür."
\par 13 Hananya söyle karsilik verdi: "Ya Rab, birçoklarinin bu adam hakkinda neler anlattiklarini duydum. Yerusalim'de senin kutsallarina nice kötülük yapmis!
\par 14 Burada da senin adini anan herkesi tutuklamak için baskâhinlerden yetki almistir."
\par 15 Rab ona, "Git!" dedi. "Bu adam, benim adimi öteki uluslara*, krallara ve Israilogullari'na duyurmak üzere seçilmis bir aracimdir.
\par 16 Benim adim ugruna ne kadar sikinti çekmesi gerekecegini ona gösterecegim."
\par 17 Bunun üzerine Hananya gitti, eve girdi ve ellerini Saul'un üzerine koydu. "Saul kardes" dedi, "Sen buraya gelirken yolda sana görünen Rab, yani Isa, gözlerin açilsin ve Kutsal Ruh'la dolasin diye beni yolladi."
\par 18 O anda Saul'un gözlerinden balik pulunu andiran seyler düstü. Saul yeniden görmeye basladi. Kalkip vaftiz* oldu.
\par 19 Sonra yemek yiyip kuvvet buldu. Saul, Sam'da ve Yerusalim'de Saul birkaç gün Sam'daki ögrencilerin yaninda kaldi.
\par 20 Havralarda Isa'nin Tanri'nin Oglu oldugunu hemen duyurmaya basladi.
\par 21 Onu duyanlarin hepsi saskina döndü. "Yerusalim'de bu adi ananlari kirip geçiren adam bu degil mi? Buraya da, öylelerini tutuklayip baskâhinlere götürmek amaciyla gelmedi mi?" diyorlardi.
\par 22 Saul ise günden güne güçleniyordu. Isa'nin Mesih* olduguna dair kanitlar göstererek Sam'da yasayan Yahudiler'i saskina çeviriyordu.
\par 23 Aradan günler geçti. Yahudiler Saul'u öldürmek için bir düzen kurdular.
\par 24 Ne var ki, kurduklari düzenle ilgili haber Saul'a ulasti. Yahudiler onu öldürmek için gece gündüz kentin kapilarini gözlüyorlardi.
\par 25 Ama Saul'un ögrencileri geceleyin kendisini aldilar, kentin surlarindan sarkittiklari bir küfe içinde asagi indirdiler.
\par 26 Saul Yerusalim'e varinca oradaki ögrencilere katilmaya çalisti. Ama hepsi ondan korkuyor, Isa'nin ögrencisi olduguna inanamiyorlardi.
\par 27 O zaman Barnaba onu alip elçilere götürdü. Onlara, Saul'un Sam yolunda Rab'bi nasil gördügünü, Rab'bin de onunla konustugunu, Sam'da ise onun Isa adini nasil korkusuzca duyurdugunu anlatti.
\par 28 Böylelikle Saul, Yerusalim'de girip çiktiklari her yerde ögrencilerle birlikte bulunarak Rab'bin adini korkusuzca duyurmaya basladi.
\par 29 Dili Grekçe olan Yahudiler'le konusup tartisiyordu. Ama onlar onu öldürmeyi tasarliyorlardi.
\par 30 Kardesler bunu ögrenince onu Sezariye'ye götürüp oradan Tarsus'a yolladilar.
\par 31 Bütün Yahudiye, Celile ve Samiriye'deki* inanlilar toplulugu* esenlige kavustu. Gelisen ve Rab korkusu içinde yasayan topluluk Kutsal Ruh'un yardimiyla sayica büyüyordu.
\par 32 Bu arada her tarafi dolasan Petrus, Lidda'da yasayan kutsallara da ugradi.
\par 33 Orada Eneas adinda birine rastladi. Eneas felçliydi. Sekiz yildan beri yatalakti.
\par 34 Petrus ona, "Eneas, Isa Mesih seni iyilestiriyor" dedi. "Kalk, yatagini topla." Eneas hemen ayaga kalkti.
\par 35 Lidda ve Saron'da yasayan herkes onu gördü ve Rab'be döndü.
\par 36 Yafa'da, Isa ögrencisi olan Tabita adinda bir kadin vardi. Tabita, ceylan anlamina gelir. Bu kadin her zaman iyilik yapip yoksullara yardim ederdi.
\par 37 O günlerde hastalanip öldü. Ölüsünü yikayip üst kattaki odaya koydular.
\par 38 Lidda Yafa'ya yakin oldugundan, Petrus'un Lidda'da bulundugunu duyan ögrenciler ona iki kisi yollayip, "Vakit kaybetmeden yanimiza gel" diye yalvardilar.
\par 39 Petrus kalkip onlarla birlikte gitti. Eve varinca onu üst kattaki odaya çikardilar. Bütün dul kadinlar aglayarak Petrus'un çevresinde toplandilar. Ona, Ceylan'in kendileriyle birlikteyken diktigi entarilerle üstlükleri gösterdiler.
\par 40 Petrus, herkesi disari çikartti, diz çöküp dua etti. Sonra ölüye dogru dönerek, "Tabita, kalk" dedi. Kadin gözlerini açti, Petrus'u görünce dogrulup oturdu.
\par 41 Petrus elini uzatarak onu ayaga kaldirdi. Sonra kutsallarla dul kadinlari çagirdi, Ceylan'i diri olarak onlara teslim etti.
\par 42 Bu olayin haberi bütün Yafa'ya yayildi ve birçoklari Rab'be inandi.
\par 43 Petrus uzunca bir süre Yafa'da, Simun adinda bir dericinin evinde kaldi.

\chapter{10}

\par 1 Sezariye'de Kornelius adinda bir adam vardi. "Italyan" taburunda yüzbasiydi.
\par 2 Dindar bir adamdi. Hem kendisi hem de bütün ev halki Tanri'dan korkardi. Halka çok yardimda bulunur, Tanri'ya sürekli dua ederdi.
\par 3 Bir gün saat* üç sularinda, bir görümde Tanri'nin bir meleginin kendisine geldigini açikça gördü. Melek ona, "Kornelius" diye seslendi.
\par 4 Kornelius korku içinde gözlerini ona dikti, "Ne var, efendim?" dedi. Melek ona söyle dedi: "Dualarin ve sadakalarin anilmak üzere Tanri katina ulasti.
\par 5 Simdi Yafa'ya adam yolla, Petrus olarak da taninan Simun'u çagirt.
\par 6 Petrus, evi deniz kiyisinda bulunan Simun adli bir dericinin yaninda kaliyor."
\par 7 Kendisiyle konusan melek uzaklastiktan sonra Kornelius, iki usagiyla özel yardimcilarindan dindar bir askeri çagirdi.
\par 8 Kendilerine her seyi anlattiktan sonra onlari Yafa'ya gönderdi.
\par 9 Ertesi gün onlar yol alip kente yaklasirlarken, saat* on iki sularinda Petrus dua etmek için dama çikti.
\par 10 Acikinca da yemek istedi. Yemek hazirlanirken Petrus kendinden geçti.
\par 11 Gögün açildigini ve büyük bir çarsafi andiran bir nesnenin dört kösesinden sarkitilarak yeryüzüne indirildigini gördü.
\par 12 Çarsafin içinde, yeryüzünde yasayan her türden dört ayakli hayvanlar, sürüngenler ve kuslar vardi.
\par 13 Bir ses ona, "Kalk Petrus, kes ve ye!" dedi.
\par 14 "Asla olmaz, ya Rab!" dedi Petrus. "Hiçbir zaman bayagi ya da murdar* herhangi bir sey yemedim."
\par 15 Ses tekrar, ikinci kez duyuldu; Petrus'a, "Tanri'nin temiz kildiklarina sen bayagi deme" dedi.
\par 16 Bu, üç kez tekrarlandi. Sonra çarsafi andiran nesne hemen göge alindi.
\par 17 Petrus saskinlik içindeydi. Gördügü görümün ne anlama gelebilecegini düsünürken, Kornelius'un gönderdigi adamlar sora sora Simun'un evinin kapisina kadar geldiler.
\par 18 Evdekilere seslenerek, "Petrus diye taninan Simun burada mi kaliyor?" diye sordular.
\par 19 Petrus hâlâ görümün anlamini düsünürken Ruh ona, "Bak, üç kisi seni ariyor" dedi.
\par 20 "Haydi kalk, asagi in. Hiç çekinmeden onlarla git. Çünkü onlari ben gönderdim."
\par 21 Petrus asagi inip adamlara, "Aradiginiz kisi benim" dedi. "Gelisinizin sebebi ne acaba?"
\par 22 "Dogru ve Tanri'dan korkan, bütün Yahudi ulusunca iyiligiyle taninan, Kornelius adinda bir yüzbasi var" dediler. "Kutsal bir melek ona, seni evine çagirtip senin söyleyeceklerini dinlemesini buyurdu."
\par 23 Bunun üzerine Petrus onlari içeri alip konuk etti. Ertesi gün Petrus kalkti, onlarla birlikte yola çikti. Yafa'daki kardeslerden bazilari da ona katildi.
\par 24 Ikinci gün Sezariye'ye vardilar. Bu arada Kornelius, akraba ve yakin dostlarini toplamis onlari bekliyordu.
\par 25 Eve giren Petrus'u karsiladi, tapinircasina ayaklarina kapandi.
\par 26 Petrus ise onu ayaga kaldirarak, "Kalk, ben de insanim" dedi.
\par 27 Petrus Kornelius'la konusa konusa içeri girdiginde birçok insanin toplanmis oldugunu gördü.
\par 28 Onlara söyle dedi: "Bir Yahudi'nin baska ulustan biriyle iliski kurmasinin, onu ziyaret etmesinin töremize aykiri oldugunu bilirsiniz. Oysa Tanri bana, hiç kimseye bayagi ya da murdar* dememem gerektigini gösterdi.
\par 29 Bu nedenle, çagrildigim zaman hiç itiraz etmeden geldim. Simdi, beni ne amaçla çagirttiginizi sorabilir miyim?"
\par 30 Kornelius, "Üç gün önce bu siralarda, saat* üçte evimde dua ediyordum" dedi. "Birdenbire, parlak giysili bir adam önüme çikiverdi.
\par 31 'Kornelius' dedi, 'Tanri senin duani isitti, verdigin sadakalari andi.
\par 32 Yafa'ya adam yolla, Petrus diye taninan Simun'u çagirt. O, deniz kiyisinda oturan derici Simun'un evinde kaliyor.'
\par 33 Bunun üzerine sana hemen adam yolladim. Sen de lütfedip geldin. Iste simdi biz hepimiz, Rab'bin sana buyurdugu her seyi dinlemek üzere Tanri'nin önünde toplanmis bulunuyoruz."
\par 34 O zaman Petrus söz alip söyle dedi: "Tanri'nin insanlar arasinda ayrim yapmadigini, ama kendisinden korkan ve dogru olani yapan kisiyi, ulusuna bakmaksizin kabul ettigini gerçekten anliyorum.
\par 36 Tanri'nin, herkesin Rabbi olan Isa Mesih araciligiyla esenligi müjdeleyerek Israilogullari'na ilettigi bildiriden haberiniz vardir.
\par 37 Yahya'nin vaftiz* çagrisindan sonra Celile'den baslayarak bütün Yahudiye'de meydana gelen olaylari, Tanri'nin, Nasirali Isa'yi nasil Kutsal Ruh'la ve kudretle meshettigini* biliyorsunuz. Isa her yani dolasarak iyilik yapiyor, Iblis'in baskisi altinda olanlarin hepsini iyilestiriyordu. Çünkü Tanri O'nunla birlikteydi.
\par 39 "Biz Isa'nin, Yahudiler'in ülkesinde ve Yerusalim'de yaptiklarinin hepsine tanik olduk. O'nu çarmiha gerip öldürdüler.
\par 40 Ama Tanri O'nu üçüncü gün diriltti ve açikça görünmesini sagladi.
\par 41 Isa halkin tümüne degil de, Tanri'nin önceden seçtigi taniklara -ölümden dirilmesinden sonra kendisiyle birlikte yiyip içen bizlere- göründü.
\par 42 Tanri tarafindan ölülerle dirilerin Yargici olarak atanan kisinin kendisi oldugunu halka duyurmamizi, buna taniklik etmemizi buyurdu.
\par 43 Peygamberlerin hepsi O'nunla ilgili taniklikta bulunuyorlar. Söyle ki, O'na inanan herkesin günahlari O'nun adiyla bagislanir."
\par 44 Petrus daha bu sözleri söylerken Kutsal Ruh, konusmayi dinleyen herkesin üzerine indi.
\par 45 Petrus'la birlikte gelen Yahudi imanlilar, Kutsal Ruh armaganinin öteki uluslardan olanlarin da üzerine dökülmesini saskinlikla karsiladilar.
\par 46 Çünkü onlarin, bilmedikleri dillerle konusup Tanri'yi yücelttiklerini duyuyorlardi. O zaman Petrus, "Bunlar, tipki bizim gibi Kutsal Ruh'u almislar. Suyla vaftiz* olmalarina kim engel olabilir?" dedi.
\par 48 Böylelikle onlarin Isa Mesih adiyla vaftiz olmalarini buyurdu. Sonra onlar Petrus'a, birkaç gün yanlarinda kalmasi için ricada bulundular.

\chapter{11}

\par 1 Elçilerle bütün Yahudiye'deki kardesler, öteki uluslarin da Tanri'nin sözünü kabul ettiklerini duydular.
\par 2 Ama Petrus Yerusalim'e gittigi zaman sünnet yanlilari onu elestirdiler.
\par 3 "Sünnetsiz* kisilerin evine gidip yemek yemissin!" dediler.
\par 4 Petrus bastan baslayarak olanlari tek tek onlara anlatti.
\par 5 "Ben Yafa Kenti'nde dua ediyordum" dedi. "Kendimden geçerek bir görüm gördüm. Büyük bir çarsafi andiran bir nesnenin dört kösesinden sarkitildigini, bunun gökten inip benim bulundugum yere kadar geldigini gördüm.
\par 6 Gözlerimi çarsafa dikip dikkatle baktim. Çarsafin içinde, yeryüzünde yasayan dört ayaklilar, yabanil hayvanlar, sürüngenler ve kuslar gördüm.
\par 7 Sonra bir sesin bana, 'Kalk, Petrus, kes ve ye!' dedigini isittim.
\par 8 "'Asla olmaz, ya Rab!' dedim. 'Agzima hiçbir zaman bayagi ya da murdar* bir sey girmedi.'
\par 9 "Ses ikinci kez gökten geldi: 'Tanri'nin temiz kildiklarina sen bayagi deme' dedi.
\par 10 Bu, üç kez tekrarlandi; sonra her sey yeniden göge alindi.
\par 11 "Tam o sirada Sezariye'den bana gönderilen üç kisi, bulundugumuz evin önünde durdular.
\par 12 Ruh bana, ayrim gözetmeden onlarla birlikte gitmemi söyledi. Bu alti kardes de benimle geldiler, varip adamin evine girdik.
\par 13 Adam bize, evinde beliren melegi nasil gördügünü anlatti. Melek ona söyle demis: 'Yafa'ya adam yolla, Petrus diye taninan Simun'u çagirt.
\par 14 O sana, senin ve bütün ev halkinin kurtulus bulacagi sözler söyleyecek.'
\par 15 "Ben konusmaya baslayinca Kutsal Ruh, baslangiçta bizim üzerimize indigi gibi, onlarin da üzerine indi.
\par 16 O zaman Rab'bin söyledigi su sözü animsadim: 'Yahya suyla vaftiz* etti, sizler ise Kutsal Ruh'la vaftiz edileceksiniz.'
\par 17 Böylelikle Tanri, Rab Isa Mesih'e inanmis olan bizlere verdigi armaganin aynisini onlara verdiyse, ben kimim ki Tanri'ya karsi koyayim?"
\par 18 Bunlari dinledikten sonra yatistilar. Tanri'yi yücelterek söyle dediler: "Demek ki Tanri, tövbe etme ve yasama kavusma firsatini öteki uluslara da vermistir."
\par 19 Istefanos'un öldürülmesiyle baslayan baski sonucu dagilan imanlilar, Fenike, Kibris ve Antakya'ya kadar gittiler. Tanri sözünü sadece Yahudiler'e duyuruyorlardi.
\par 20 Ama içlerinden Kibrisli ve Kireneli olan bazi adamlar Antakya'ya gidip Grekler'le* de konusmaya basladilar. Onlara Rab Isa'yla ilgili Müjde'yi bildirdiler.
\par 21 Onlarin arasinda etkin olan Rab'bin gücü sayesinde çok sayida kisi inanip Rab'be döndü.
\par 22 Olup bitenlerin haberi, Yerusalim'deki kiliseye* ulasti. Bunun üzerine imanlilar Barnaba'yi Antakya'ya gönderdiler.
\par 23 Kutsal Ruh'la ve imanla dolu, iyi bir adam olan Barnaba, Antakya'ya varip Tanri lütfunun meyvelerini görünce sevindi. Herkesi, candan ve yürekten Rab'be bagli kalmaya özendirdi. Sonuç olarak Rab'be daha birçok kisi kazanildi.
\par 25 Sonra Barnaba, Saul'u aramak için Tarsus'a gitti. Onu bulunca da Antakya'ya getirdi. Böylece Barnaba'yla Saul bir yil boyunca oradaki inanlilar topluluguyla* bir araya gelerek büyük bir kitleyi egittiler. Ögrencilere ilk kez Antakya'da Mesihçiler adi verildi.
\par 27 O günlerde Yerusalim'den Antakya'ya bazi peygamberler geldi.
\par 28 Bunlardan Hagavos adli biri ortaya çikip bütün dünyada siddetli bir kitlik olacagini Ruh araciligiyla bildirdi. Bu kitlik, Klavdius'un imparatorlugu sirasinda oldu.
\par 29 Ögrenciler, her biri kendi gücü oraninda, Yahudiye'de yasayan kardeslere gönderilmek üzere yardim toplamayi kararlastirdilar.
\par 30 Bu karari yerine getirip bagislarini Barnaba ve Saul'un eliyle kilisenin ihtiyarlarina* gönderdiler.

\chapter{12}

\par 1 O sirada kral Hirodes*, kiliseden* bazi kisilere eziyet etmeye basladi.
\par 2 Yuhanna'nin kardesi Yakup'u kiliçla öldürttü.
\par 3 Yahudiler'in bundan memnun kaldigini görünce ardindan Petrus'u da yakalatti. Bunu, Mayasiz Ekmek Bayrami* sirasinda yapti.
\par 4 Petrus'u tutuklatip hapse attirdi ve dörder kisilik dört takim askerin gözetimine teslim etti. Fisih Bayrami'ndan* sonra onu halkin önünde yargilamak niyetindeydi.
\par 5 Bu nedenle Petrus hapiste tutuldu. Ama inanlilar toplulugu* onun için Tanri'ya hararetle dua ediyordu.
\par 6 Petrus, Hirodes'in kendisini yargilayacagi günden önceki gece, çift zincirle bagli olarak iki askerin arasinda uyuyordu. Kapida duran nöbetçiler de zindanin güvenligini sagliyordu.
\par 7 Birdenbire Rab'bin bir melegi göründü ve hücrede bir isik parladi. Melek, Petrus'un bögrüne dokunup onu uyandirdi. "Çabuk, kalk!" dedi. O anda zincirler Petrus'un bileklerinden düstü.
\par 8 Melek ona, "Kusagini bagla, çariklarini giy" dedi. Petrus da söyleneni yapti. "Abani giy, beni izle" dedi melek.
\par 9 Petrus onu izleyerek disari çikti. Ama melegin yaptiginin gerçek oldugunu anlamiyor, bir görüm gördügünü saniyordu.
\par 10 Birinci ve ikinci nöbetçiyi geçerek kente açilan demir kapiya geldiler. Kapi, önlerinde kendiliginden açildi. Disari çikip bir sokak boyunca yürüdüler, sonra melek ansizin Petrus'un yanindan ayrildi.
\par 11 O zaman kendine gelen Petrus, "Rab'bin bana melegini gönderdigini simdi gerçekten anliyorum" dedi. "O beni Hirodes'in elinden ve Yahudi halkinin ugrayacagimi umdugu bütün belalardan kurtardi."
\par 12 Petrus olanlarin farkina varinca Markos diye taninan Yuhanna'nin annesi Meryem'in evine gitti. Orada birçok kisi toplanmis dua ediyordu.
\par 13 Petrus'un dis kapiyi çalmasi üzerine Roda adli bir hizmetçi kiz kapiya bakmaya gitti.
\par 14 Petrus'un sesini taniyan kiz, sevincinden kapiyi açmadan tekrar içeri kosarak, "Petrus kapida duruyor!" diye haber verdi.
\par 15 "Çildirmissin sen!" dediler ona. Ama kiz üsteleyince, "Onun melegi olmali" dediler.
\par 16 Petrus ise kapiyi çalmaya devam etti. Kapiyi açip onu görünce sasip kaldilar.
\par 17 Petrus, eliyle susmalarini isaret ederek Rab'bin onu zindandan nasil çikardigini anlatti. Sonra, "Bu haberleri Yakup'la öbür kardeslere iletin" diyerek oradan ayrilip baska bir yere gitti.
\par 18 Askerler sabahleyin büyük bir telasa kapildilar. Birbirlerine, "Petrus'a ne oldu?" diye sordular.
\par 19 Hirodes onu aratti, bulamayinca da nöbetçileri sorguya çekti ve idam edilmeleri için buyruk verdi. Bundan sonra Hirodes, Yahudiye'den Sezariye'ye gidip bir süre orada kaldi.
\par 20 Bu arada Sur ve Sayda halklarina ates püskürüyordu. Bunlar birlesip kendisiyle görüsmeye geldiler. Önce kralin basdanismani Vlastus'u kendi taraflarina çekerek baris isteginde bulundular. Çünkü kendi ülkelerinin gereksindigi yiyecekler kralin ülkesinden saglaniyordu.
\par 21 Belirlenen günde krallik giysilerini giyen Hirodes tahtina oturarak halka bir konusma yapti.
\par 22 Halk, "Bu bir insanin sesi degil, bir ilahin sesidir!" diye bagiriyordu.
\par 23 O anda Rab'bin bir melegi Hirodes'i vurdu. Çünkü Tanri'ya ait olan yüceligi kendine mal etmisti. Içi kurtlarca kemirilerek can verdi.
\par 24 Tanri'nin sözü ise yayiliyor, etkisini artiriyordu.
\par 25 Görevlerini tamamlayan Barnaba'yla Saul, Markos diye taninan Yuhanna'yi yanlarina alarak Yerusalim'den döndüler.

\chapter{13}

\par 1 Antakya'daki kilisede* peygamberler ve ögretmenler vardi: Barnaba, Niger denilen Simon, Kireneli Lukius, bölge krali* Hirodes'le birlikte büyümüs olan Menahem ve Saul.
\par 2 Bunlar Rab'be tapinip oruç* tutarlarken Kutsal Ruh kendilerine söyle dedi: "Barnaba'yla Saul'u, kendilerini çagirmis oldugum görev için bana ayirin."
\par 3 Böylece oruç tutup dua ettikten sonra, Barnaba'yla Saul'un üzerine ellerini koyup onlari yolcu ettiler.
\par 4 Kutsal Ruh'un buyruguyla yola çikan Barnaba'yla Saul, Selefkiye'ye gittiler, oradan da gemiyle Kibris'a geçtiler.
\par 5 Salamis'e varinca Yahudiler'in havralarinda Tanri'nin sözünü duyurmaya basladilar. Yuhanna'yi da yardimci olarak yanlarina almislardi.
\par 6 Adayi bastan basa geçerek Baf'a geldiler. Orada büyücü ve sahte peygamber Baryesu adinda bir Yahudi'yle karsilastilar.
\par 7 Baryesu, Vali Sergius Pavlus'a yakin biriydi. Akilli bir kisi olan vali, Barnaba'yla Saul'u çagirtip Tanri'nin sözünü dinlemek istedi. Ne var ki Baryesu büyücü anlamina gelen öbür adiyla Elimas- onlara karsi koyarak valiyi iman etmekten caydirmaya çalisti.
\par 9 Ama Kutsal Ruh'la dolan Saul, yani Pavlus, gözlerini Elimas'a dikerek, "Ey Iblis'in oglu!" dedi. "Yüregin her türlü hile ve sahtekârlikla dolu; dogru olan her seyin düsmanisin. Rab'bin düz yollarini çarpitmaktan vazgeçmeyecek misin?
\par 11 Iste simdi Rab'bin eli sana karsi kalkti. Kör olacaksin, bir süre gün isigini göremeyeceksin." O anda adamin üzerine bir sis, bir karanlik çöktü. Dört dönerek, elinden tutup kendisine yol gösterecek birilerini aramaya basladi.
\par 12 Olanlari gören vali, Rab'le ilgili ögretiyi hayranlikla karsiladi ve iman etti.
\par 13 Pavlus'la beraberindekiler Baf'tan denize açilip Pamfilya bölgesinin Perge Kenti'ne gittiler. Yuhanna ise onlari birakip Yerusalim'e döndü.
\par 14 Onlar Perge'den yollarina devam ederek Pisidya sinirindaki Antakya'ya geçtiler. Sabat Günü* havraya girip oturdular.
\par 15 Kutsal Yasa ve peygamberlerin yazilari okunduktan sonra, havranin yöneticileri onlara, "Kardesler, halka verecek bir ögüdünüz varsa buyurun, konusun" diye haber yolladilar.
\par 16 Pavlus ayaga kalkti, eliyle bir isaret yaparak, "Ey Israilliler ve Tanri'dan korkan* yabancilar, dinleyin" dedi.
\par 17 "Bu halkin, yani Israil'in Tanrisi, bizim atalarimizi seçti ve Misir'da gurbette yasadiklari süre içinde onlari büyük bir ulus yapti. Sonra güçlü eliyle onlari oradan çikardi, çölde yaklasik kirk yil onlara katlandi.
\par 19 Kenan ülkesinde yenilgiye ugrattigi yedi ulusun topraklarini Israil halkina miras olarak verdi. Bütün bunlar asagi yukari dört yüz elli yil sürdü. "Sonra Tanri, Peygamber Samuel'in zamanina kadar onlar için hakimler yetistirdi.
\par 21 Halk bir kral isteyince, Tanri onlar için Benyamin oymagindan Kis oglu Saul'u yetistirdi. Saul kirk yil krallik yapti.
\par 22 Tanri, onu tahttan indirdikten sonra onlara kral olarak Davut'u basa geçirdi. Onunla ilgili su taniklikta bulundu: 'Isay oglu Davut'u gönlüme uygun bir adam olarak gördüm, o her istedigimi yapar.'
\par 23 Tanri, verdigi sözü tutarak bu adamin soyundan Israil'e bir Kurtarici, Isa'yi gönderdi.
\par 24 Isa'nin gelisinden önce Yahya, bütün Israil halkini, tövbe edip vaftiz* olmaya çagirdi.
\par 25 Yahya görevini tamamlarken söyle diyordu: 'Beni kim saniyorsunuz? Ben Mesih* degilim. Ama O benden sonra geliyor. Ben O'nun ayagindaki çarigin bagini çözmeye bile layik degilim.'
\par 26 "Kardesler, Ibrahim'in soyundan gelenler ve Tanri'dan korkan yabancilar, bu kurtulus bildirisi bize gönderildi.
\par 27 Çünkü Yerusalim'de yasayanlar ve onlarin yöneticileri Isa'yi reddettiler. O'nu mahkûm etmekle her Sabat Günü okunan peygamberlerin sözlerini yerine getirmis oldular.
\par 28 O'nda ölüm cezasini gerektiren herhangi bir suç bulamadiklari halde, Pilatus'tan O'nun idamini istediler.
\par 29 O'nunla ilgili yazilanlarin hepsini yerine getirdikten sonra O'nu çarmihtan indirip mezara koydular.
\par 30 Ama Tanri O'nu ölümden diriltti.
\par 31 Isa, daha önce kendisiyle birlikte Celile'den Yerusalim'e gelenlere günlerce göründü. Bu kisiler simdi halka O'nun tanikligini yapiyor.
\par 32 "Biz de size Müjde'yi duyuruyoruz: Tanri Isa'yi diriltmekle, atalarimiza verdigi sözü, onlarin çocuklari olan bizler için yerine getirmistir. Ikinci Mezmur'da da yazildigi gibi: 'Sen benim Oglum'sun, Bugün ben sana Baba oldum.'
\par 34 "Tanri, O'nu asla çürümemek üzere ölümden dirilttigini su sözlerle belirtmistir: 'Size, Davut'a söz verdigim Kutsal ve güvenilir nimetleri verecegim.'
\par 35 "Bunun için baska bir yerde de söyle der: 'Kutsalinin çürümesine izin vermeyeceksin.'
\par 36 "Davut, kendi kusaginda Tanri'nin amaci uyarinca hizmet ettikten sonra gözlerini yasama kapadi, atalari gibi gömüldü ve bedeni çürüyüp gitti.
\par 37 Oysa Tanri'nin dirilttigi Kisi'nin bedeni çürümedi.
\par 38 Dolayisiyla kardesler, sunu bilin ki, günahlarin bu Kisi araciligiyla bagislanacagi size duyurulmus bulunuyor. Söyle ki, iman eden herkes, Musa'nin Yasasi'yla aklanamadiginiz her suçtan O'nun araciligiyla aklanir.
\par 40 Dikkat edin, peygamberlerin sözünü ettigi su durum sizin basiniza gelmesin: 'Bakin, siz alay edenler, Saskina dönüp yok olun! Sizin gününüzde bir is yapiyorum, Öyle bir is ki, biri size anlatsa inanmazsiniz.'"
\par 42 Pavlus'la Barnaba havradan çikarken halk onlari, bir sonraki Sabat Günü ayni konular üzerinde konusmaya çagirdi.
\par 43 Havradaki topluluk dagilinca, Yahudiler ve Yahudilige dönüp Tanri'ya tapan yabancilardan birçogu onlarin ardindan gitti. Pavlus'la Barnaba onlarla konusarak onlari devamli Tanri'nin lütfunda yasamaya özendirdiler.
\par 44 Ertesi Sabat Günü kent halkinin hemen hemen tümü Rab'bin sözünü dinlemek için toplanmisti.
\par 45 Kalabaligi gören Yahudiler büyük bir kiskançlik içinde, küfürlerle Pavlus'un söylediklerine karsi çiktilar.
\par 46 Pavlus'la Barnaba ise cesaretle karsilik verdiler: "Tanri'nin sözünü ilk önce size bildirmemiz gerekiyordu. Siz onu reddettiginize ve kendinizi sonsuz yasama layik görmediginize göre, biz simdi öteki uluslara gidiyoruz.
\par 47 Çünkü Rab bize söyle buyurmustur: 'Yeryüzünün dört bucagina kurtulus götürmen için Seni uluslara isik yaptim.'"
\par 48 Öteki uluslardan olanlar bunu isitince sevindiler ve Rab'bin sözünü yücelttiler. Sonsuz yasam için belirlenmis olanlarin hepsi iman etti.
\par 49 Böylece Rab'bin sözü bütün yörede yayildi.
\par 50 Ne var ki Yahudiler, Tanri'ya tapan saygin kadinlarla kentin ileri gelen erkeklerini kiskirttilar, Pavlus'la Barnaba'ya karsi bir baski hareketi baslatip onlari bölge sinirlarinin disina attilar.
\par 51 Bunun üzerine Pavlus'la Barnaba, onlara bir uyari olsun diye ayaklarinin tozunu silkerek Konya'ya gittiler.
\par 52 Ögrenciler ise sevinç ve Kutsal Ruh'la doluydu.

\chapter{14}

\par 1 Ayni sekilde Konya'da da Yahudiler'in havrasina giren Pavlus'la Barnaba öyle etkili konustular ki, hem Yahudiler'den hem de Grekler'den* çok kisi iman etti.
\par 2 Ama inanmayan Yahudiler, öteki uluslardan olanlari kardeslere karsi kiskirtarak zihinlerini bulandirdilar.
\par 3 Orada uzunca bir süre kalan Pavlus'la Barnaba, Rab hakkinda cesaretle konusuyorlardi. Rab de onlara belirtiler ve harikalar yapma gücü vererek kendi lütfunu açiklayan bildiriyi dogruladi.
\par 4 Kent halki ikiye bölündü. Bazilari Yahudiler'in, bazilari da elçilerin tarafini tuttu.
\par 5 Yahudiler'le öteki uluslardan olanlar ve bunlarin yöneticileri, elçileri hirpalayip tasa tutmak için düzen kurdular.
\par 6 Bunu ögrenen Pavlus'la Barnaba, Likaonya'nin Listra ve Derbe kentlerine ve çevre bölgeye kaçarak oralarda da Müjde'yi yaydilar.
\par 8 Listra'da, ayaklari tutmayan bir adam vardi. Dogustan kötürümdü, hiç yürüyemiyordu.
\par 9 Pavlus'un söylediklerini dinledi. Onu dikkatle süzen Pavlus, iyilestirilebilecegine imani oldugunu görerek yüksek sesle ona, "Kalk, ayaklarinin üzerinde dur!" dedi. Adam yerinden firlayip yürümeye basladi.
\par 11 Pavlus'un ne yaptigini gören halk Likaonya dilinde, "Tanrilar insan kiligina girip yanimiza inmis!" diye haykirdi.
\par 12 Barnaba'ya Zeus, Pavlus'a da konusmada öncülük ettigi için Hermes adini taktilar.
\par 13 Kentin hemen disinda bulunan Zeus Tapinagi'nin kâhini kent kapilarina bogalar ve çelenkler getirdi, halkla birlikte elçilere kurban sunmak istedi.
\par 14 Ne var ki elçiler, Barnaba'yla Pavlus, bunu duyunca giysilerini yirtarak kalabaligin içine daldilar.
\par 15 "Efendiler, neden böyle seyler yapiyorsunuz?" diye bagirdilar. "Biz de sizin gibi insaniz, ayni yaradilisa sahibiz. Size müjde getiriyoruz. Sizi bu bos seylerden vazgeçmeye, yeri, gögü, denizi ve bunlarin içindekilerin hepsini yaratan, yasayan Tanri'ya dönmeye çagiriyoruz.
\par 16 Geçmis çaglarda Tanri, bütün uluslarin kendi yollarindan gitmelerine izin verdi.
\par 17 Yine de kendini taniksiz birakmadi. Size iyilik ediyor. Gökten yagmur yagdiriyor, çesitli ürünleriyle mevsimleri düzenliyor, sizi yiyecekle doyurup yüreklerinizi sevinçle dolduruyor."
\par 18 Bu sözlerle bile halkin kendilerine kurban sunmasini güçlükle engelleyebildiler.
\par 19 Ne var ki, Antakya ve Konya'dan gelen bazi Yahudiler, halki kendi taraflarina çekerek Pavlus'u tasladilar; onu ölmüs sanarak kentin disina sürüklediler.
\par 20 Ama ögrenciler çevresinde toplaninca Pavlus ayaga kalkip kente döndü. Ertesi gün Barnaba'yla birlikte Derbe'ye gitti.
\par 21 O kentte de Müjde'yi duyurup birçok ögrenci edindiler. Antakya'ya Dönüs Pavlus'la Barnaba daha sonra Listra, Konya ve Antakya'ya dönerek ögrencileri ruhça pekistirdiler, imana bagli kalmalari için onlara cesaret verdiler. "Tanri'nin Egemenligi'ne*, birçok sikintidan geçerek girmemiz gerekir" diyorlardi.
\par 23 Imanlilar için her kilisede* ihtiyarlar* seçtiler. Dua ve oruçla* onlari, inandiklari Rab'be emanet ettiler.
\par 24 Pisidya bölgesinden geçerek Pamfilya'ya geldiler.
\par 25 Perge'de Tanri sözünü bildirdikten sonra Antalya'ya gittiler.
\par 26 Oradan gemiyle, artik tamamlamis bulunduklari görev için Tanri'nin lütfuna emanet edildikleri yer olan Antakya'ya döndüler.
\par 27 Oraya vardiklarinda inanlilar toplulugunu* bir araya getirip Tanri'nin kendileri araciligiyla neler yaptigini, öteki uluslara* iman kapisini nasil açtigini anlattilar.
\par 28 Oradaki ögrencilerin yaninda uzun bir süre kaldilar.

\chapter{15}

\par 1 Yahudiye'den gelen bazi kisiler Antakya'daki kardeslere, "Siz Musa'nin töresi uyarinca sünnet olmadikça kurtulamazsiniz" diye ögretiyorlardi.
\par 2 Pavlus'la Barnaba bu adamlarla bir hayli çekisip tartistilar. Sonunda Pavlus'la Barnaba'nin, baska birkaç kardesle birlikte Yerusalim'e gidip bu sorunu elçiler ve ihtiyarlarla* görüsmesi kararlastirildi.
\par 3 Böylece kilise* tarafindan gönderilenler, öteki uluslardan* olanlarin Tanri'ya nasil döndügünü anlata anlata Fenike ve Samiriye bölgelerinden geçerek bütün kardeslere büyük sevinç verdiler.
\par 4 Yerusalim'e geldiklerinde inanlilar toplulugu*, elçiler ve ihtiyarlarca iyi karsilandilar. Tanri'nin kendileri araciligiyla yapmis oldugu her seyi anlattilar.
\par 5 Ne var ki, Ferisi* mezhebinden bazi imanlilar kalkip söyle dediler: "Öteki uluslardan olanlari sünnet etmek ve onlara Musa'nin Yasasi'na uymalarini buyurmak gerekir."
\par 6 Elçilerle ihtiyarlar bu konuyu görüsmek için toplandilar.
\par 7 Uzunca bir tartismadan sonra Petrus ayaga kalkip onlara, "Kardesler" dedi, "Öteki uluslar Müjde'nin bildirisini benim agzimdan duyup inansinlar diye Tanri'nin uzun zaman önce aranizdan beni seçtigini biliyorsunuz.
\par 8 Insanin yüregini bilen Tanri, Kutsal Ruh'u tipki bize verdigi gibi onlara da vermekle, onlari kabul ettigini gösterdi.
\par 9 Onlarla bizim aramizda hiçbir ayrim yapmadi, iman etmeleri üzerine yüreklerini arindirdi.
\par 10 Öyleyse, ne bizim ne de atalarimizin tasiyamadigi bir boyundurugu ögrencilerin boynuna geçirerek simdi neden Tanri'yi deniyorsunuz?
\par 11 Bizler, Rab Isa'nin lütfuyla kurtuldugumuza inaniyoruz; onlar da öyle."
\par 12 Bunun üzerine bütün topluluk sustu ve Barnaba'yla Pavlus'u dinlemeye basladi. Barnaba'yla Pavlus, Tanri'nin kendileri araciligiyla öteki uluslar arasinda yaptigi harikalarla belirtileri tek tek anlattilar.
\par 13 Onlar konusmalarini bitirince Yakup söz aldi: "Kardesler, beni dinleyin" dedi.
\par 14 "Simun, Tanri'nin öteki uluslardan kendine ait olacak bir halk çikarmak amaciyla onlara ilk kez nasil yaklastigini anlatmistir.
\par 15 Peygamberlerin sözleri de bunu dogrulamaktadir. Yazilmis oldugu gibi: 'Bundan sonra ben geri dönüp, Davut'un yikik konutunu yeniden kuracagim. Onun yikintilarini yeniden kurup Onu tekrar ayaga kaldiracagim.
\par 17 Öyle ki, geriye kalan insanlar, Bana ait olan bütün uluslar Rab'bi arasinlar. Bunlari ta baslangiçtan bildiren Rab, Iste böyle diyor.'
\par 19 "Bu nedenle, kanimca öteki uluslardan Tanri'ya dönenlere güçlük çikarmamaliyiz.
\par 20 Ancak putlara sunulup murdar* hale gelen etlerden, fuhustan, bogularak öldürülen hayvanlarin etinden ve kandan sakinmalari gerektigini onlara yazmaliyiz.
\par 21 Çünkü çok eski zamanlardan beri Musa'nin sözleri her kentte duyurulmakta, her Sabat Günü* havralarda okunmaktadir." Öteki Uluslardan Olan Imanlilara Mektup
\par 22 Bunun üzerine bütün inanlilar topluluguyla* elçiler ve ihtiyarlar*, kendi aralarindan seçtikleri adamlari Pavlus ve Barnaba'yla birlikte Antakya'ya göndermeye karar verdiler. Kardeslerin önde gelenlerinden Barsabba denilen Yahuda ile Silas'i seçtiler.
\par 23 Onlarin eliyle su mektubu yolladilar: "Kardesleriniz olan biz elçilerle ihtiyarlardan, öteki uluslardan olup Antakya, Suriye ve Kilikya'da bulunan siz kardeslere selam!
\par 24 Bizden bazi kisilerin yaniniza geldigini, sözleriyle sizi tedirgin edip aklinizi karistirdigini duyduk. Oysa onlari biz göndermedik.
\par 25 Bu nedenle aramizdan seçtigimiz bazi kisileri, sevgili kardeslerimiz Barnaba ve Pavlus'la birlikte size göndermeye oybirligiyle karar verdik.
\par 26 Bu ikisi, Rabbimiz Isa Mesih'in adi ugruna canlarini gözden çikarmis kisilerdir.
\par 27 Kararimiz uyarinca size Yahuda ile Silas'i gönderiyoruz. Onlar ayni seyleri sözlü olarak da aktaracaklar.
\par 28 Kutsal Ruh ve bizler, gerekli olan su kurallarin disinda size herhangi bir sey yüklememeyi uygun gördük: Putlara sunulan kurbanlarin etinden, kandan, bogularak öldürülen hayvanlarin etinden ve fuhustan sakinmalisiniz. Bunlardan kaçinirsaniz, iyi edersiniz. Esen kalin."
\par 30 Adamlar böylece yola koyulup Antakya'ya gittiler. Toplulugu bir araya getirerek onlara mektubu verdiler.
\par 31 Imanlilar, mektuptaki yüreklendirici sözleri okuyunca sevindiler.
\par 32 Kendileri peygamber olan Yahuda ile Silas, birçok konusmalar yaparak kardesleri yüreklendirip ruhça pekistirdiler.
\par 33 Bir süre orada kaldiktan sonra, kendilerini göndermis olanlarin yanina dönmek üzere kardesler tarafindan esenlikle yolcu edildiler.
\par 35 Pavlus'la Barnaba ise Antakya'da kaldilar, birçoklariyla birlikte ögretip Rab'bin sözünü müjdelediler.
\par 36 Bundan bir süre sonra Pavlus Barnaba'ya, "Rab'bin sözünü duyurdugumuz bütün kentlere dönüp kardesleri ziyaret edelim, nasil olduklarini görelim" dedi.
\par 37 Barnaba, Markos denilen Yuhanna'yi da yanlarinda götürmek istiyordu.
\par 38 Ama Pavlus, Pamfilya'da kendilerini yüzüstü birakip birlikte göreve devam etmeyen Markos'u yanlarinda götürmeyi uygun görmedi.
\par 39 Aralarinda öylesine keskin bir anlasmazlik çikti ki, birbirlerinden ayrildilar. Barnaba Markos'u alip Kibris'a dogru yelken açti.
\par 40 Silas'i seçen Pavlus ise, kardeslerce Rab'bin lütfuna emanet edildikten sonra yola çikti.
\par 41 Suriye ve Kilikya bölgelerini dolasarak inanli topluluklarini* pekistirdi.

\chapter{16}

\par 1 Pavlus, Derbe ve Listra'ya da ugradi. Listra'da Timoteos adinda bir Isa ögrencisi vardi. Annesi imanli bir Yahudi, babasi ise Grek'ti*.
\par 2 Listra ve Konya'daki kardesler ondan övgüyle söz ediyorlardi.
\par 3 Timoteos'u kendisiyle birlikte götürmek isteyen Pavlus, oralarda bulunan Yahudiler yüzünden onu sünnet ettirdi. Çünkü hepsi, babasinin Grek oldugunu biliyordu.
\par 4 Kent kent dolasarak Yerusalim'deki elçilerle ihtiyarlarin* aldigi kararlari imanlilara iletiyor, bunlara uymalarini istiyorlardi.
\par 5 Böylelikle topluluklarin* imani güçleniyor ve sayilari günden güne artiyordu.
\par 6 Kutsal Ruh'un, Tanri sözünü Asya Ili'nde* yaymalarini engellemesi üzerine Pavlus'la arkadaslari Frikya ve Galatya bölgesinden geçtiler.
\par 7 Misya sinirina geldiklerinde Bitinya bölgesine geçmek istediler. Ama Isa'nin Ruhu onlara izin vermedi.
\par 8 Bunun üzerine Misya'dan geçip Troas Kenti'ne gittiler.
\par 9 O gece Pavlus bir görüm gördü. Önünde Makedonyali bir adam durmus, ona yalvariyordu: "Makedonya'ya geçip bize yardim et" diyordu.
\par 10 Pavlus'un gördügü bu görümden sonra hemen Makedonya'ya gitmenin bir yolunu aradik. Çünkü Tanri'nin bizi, Müjde'yi oradakilere duyurmaya çagirdigi sonucuna varmistik.
\par 11 Troas'tan denize açilip dogru Semadirek Adasi'na, ertesi gün de Neapolis'e gittik.
\par 12 Oradan da Filipi'ye geçtik. Burasi bir Roma yerlesim merkezi ve Makedonya'nin o bölgesinde önemli bir kentti. Birkaç gün bu kentte kaldik.
\par 13 Sabat Günü* kent kapisindan çikip irmak kiyisina gittik. Orada bir dua yeri olacagini düsünüyorduk. Oturduk, orada toplanmis kadinlarla konusmaya basladik.
\par 14 Bizi dinleyenler arasinda Tiyatira Kenti'nden Lidya adinda bir kadin vardi. Mor kumas ticareti yapan Lidya, Tanri'ya tapan biriydi. Pavlus'un söylediklerine kulak vermesi için Rab onun yüregini açti.
\par 15 Lidya, ev halkiyla birlikte vaftiz* olduktan sonra bizi evine çagirdi. "Beni Rab'bin bir inanlisi kabul ediyorsaniz, gelin, evimde kalin" dedi ve bizi razietti.
\par 16 Bir gün biz dua yerine giderken, karsimiza, falcilik ruhuna tutulmus köle bir kiz çikti. Bu kiz, gelecekten haber vererek efendilerine bir hayli kazanç sagliyordu.
\par 17 Pavlus'u ve bizleri izleyerek, "Bu adamlar yüce Tanri'nin kullaridir, size kurtulus yolunu bildiriyorlar!" diye bagirip durdu.
\par 18 Ve günlerce sürdürdü bunu. Sonunda, bundan çok rahatsiz olan Pavlus arkasina dönerek ruha, "Isa Mesih'in adiyla, bu kizin içinden çikmani buyuruyorum" dedi. Ruh hemen kizin içinden çikti.
\par 19 Kizin efendileri, kazanç umutlarinin yok oldugunu görünce Pavlus'la Silas'i yakalayip çarsi meydanina, yetkililerin önüne sürüklediler.
\par 20 Onlari yargiçlarin karsisina çikartarak, "Bu adamlar Yahudi'dir" dediler, "Kentimizi altüst ettiler. Biz Romalilar için benimsenmesi ve uygulanmasi yasak birtakim töreler yayiyorlar."
\par 22 Halk da Pavlus'la Silas'a yapilan saldiriya katildi. Yargiçlar onlarin giysilerini yirtip siyirarak degnekle dövülmeleri için buyruk verdi.
\par 23 Onlari iyice dövdürdükten sonra hapse attilar. Zindanciya, onlari siki güvenlik altinda tutmasini buyurdular.
\par 24 Bu buyrugu alan zindanci onlari hapishanenin iç bölmesine atarak ayaklarini tomruga vurdu.
\par 25 Gece yarisina dogru Pavlus'la Silas dua ediyor, Tanri'yi ilahilerle yüceltiyorlardi. Öbür tutuklular da onlari dinliyordu.
\par 26 Birdenbire öyle siddetli bir deprem oldu ki, tutukevi temelden sarsildi. Bir anda bütün kapilar açildi, herkesin zincirleri çözüldü.
\par 27 Zindanci uyandi. Zindan kapilarini açik görünce kilicini çekip canina kiymak istedi. Çünkü tutuklularin kaçtigini sanmisti.
\par 28 Ama Pavlus yüksek sesle, "Canina kiyma, hepimiz buradayiz!" diye seslendi.
\par 29 Zindanci isik getirtip içeri daldi. Titreyerek Pavlus'la Silas'in önünde yere kapandi.
\par 30 Onlari disari çikararak, "Efendiler, kurtulmak için ne yapmam gerekir?" diye sordu.
\par 31 Onlar, "Rab Isa'ya iman et, sen de ev halkin da kurtulursunuz" dediler.
\par 32 Sonra kendisine ve ev halkinin hepsine Rab'bin sözünü bildirdiler.
\par 33 Gecenin o saatinde zindanci onlari götürüp yaralarini yikadi. Sonra hem kendisi hem ev halki hemen vaftiz* oldu.
\par 34 Pavlus'la Silas'i evine götürerek sofra kurdu. Tanri'ya inanmak, onu ve evindekilerin hepsini sevince bogmustu.
\par 35 Gün dogunca yargiçlar görevlileri göndererek, "O adamlari serbest birak" dediler.
\par 36 Zindanci bu sözleri Pavlus'a iletti. "Yargiçlar serbest birakilmaniz için haber gönderdi. Simdi çikabilirsiniz, esenlikle gidin" dedi.
\par 37 Ama Pavlus görevlilere söyle dedi: "Roma vatandasi* oldugumuz halde, bizi yargilamadan herkesin önünde dövüp hapse attilar. Simdi bizi gizlice mi kovacaklar? Olmaz böyle sey! Kendileri gelsinler, bizi alip çikarsinlar!"
\par 38 Görevliler bu sözleri yargiçlara iletti. Yargiçlar, Pavlus'la Silas'in Roma vatandasi oldugunu duyunca korktular.
\par 39 Gelip özür dilediler. Sonra onlari disari çikararak kentten ayrilmalarini rica ettiler.
\par 40 Pavlus'la Silas zindandan çikinca Lidya'nin evine gittiler. Kardeslerle görüsüp onlari yüreklendirdikten sonra oradan ayrildilar.

\chapter{17}

\par 1 Amfipolis ve Apollonya'dan geçerek Selanik'e geldiler. Burada Yahudiler'in bir havrasi vardi.
\par 2 Pavlus, her zamanki gibi Yahudiler'e giderek art arda üç Sabat Günü* onlarla Kutsal Yazilar üzerinde tartisti.
\par 3 Mesih'in* aci çekip ölümden dirilmesi gerektigine dair açiklamalarda bulunuyor, kanitlar gösteriyordu. "Size duyurmakta oldugum bu Isa, Mesih'tir" diyordu.
\par 4 Onlardan bazilari, Tanri'ya tapan Grekler'den büyük bir topluluk ve ileri gelen kadinlarin da birçogu ikna olup Pavlus'la Silas'a katildilar.
\par 5 Yahudiler bunu kiskandi. Çarsi pazardan topladiklari bazi kötü insanlardan bir kalabalik olusturup kentte kargasalik çikarttilar. Pavlus'la Silas'i bulmak ve halkin önünde yargilamak amaciyla Yason'un evine saldirdilar.
\par 6 Onlari bulamayinca, Yason ile bazi kardesleri kent yetkililerinin önüne sürüklediler. "Dünyayi altüst eden o adamlar buraya da geldiler" diye bagiriyorlardi.
\par 7 "Yason onlari evine aldi. Onlarin hepsi, Isa adinda baska bir kral oldugunu söyleyerek Sezar'in* buyruklarina karsi geliyorlar."
\par 8 Bu sözleri isiten kalabalik ve kentin yetkilileri telasa kapildi.
\par 9 Sonunda yetkililer Yason ve öbürlerini kefaletle serbest biraktilar.
\par 10 Kardesler hemen o gece Pavlus'la Silas'i Veriya Kenti'ne gönderdiler. Onlar oraya varinca Yahudiler'in havrasina gittiler.
\par 11 Veriya'daki Yahudiler Selanik'tekilerden daha açik fikirliydi. Tanri sözünü büyük ilgiyle karsilayarak her gün Kutsal Yazilar'i inceliyor, ögretilenlerin dogru olup olmadigini arastiriyorlardi.
\par 12 Böylelikle içlerinden birçoklari ve çok sayida saygin Grek kadin ve erkek iman etti.
\par 13 Selanik'teki Yahudiler Pavlus'un Veriya'da da Tanri'nin sözünü duyurdugunu ögrenince oraya gittiler, halki kiskirtip ayaga kaldirdilar.
\par 14 Bunun üzerine kardesler Pavlus'u hemen deniz kiyisina yolladilar. Silas ile Timoteos ise Veriya'da kaldilar.
\par 15 Pavlus'la birlikte gidenler onu Atina'ya kadar götürdüler. Sonra Pavlus'tan, Silas'la Timoteos'un bir an önce kendisine yetismeleri yolunda buyruk alarak geri döndüler.
\par 16 Onlari Atina'da bekleyen Pavlus, kenti putlarla dolu görünce yüreginde derin bir aci duydu.
\par 17 Bu nedenle, gerek havrada Yahudiler'le ve Tanri'ya tapan yabancilarla, gerek her gün çarsi meydaninda karsilastigi kisilerle tartisip durdu.
\par 18 Epikürcü ve Stoaci bazi filozoflar onunla atismaya basladilar. Kimi, "Bu lafebesi ne demek istiyor?" derken, kimi de, "Galiba yabanci ilahlarin haberciligini yapiyor" diyordu. Çünkü Pavlus, Isa'yla ve dirilisle ilgili Müjde'yi duyuruyordu.
\par 19 Onlar Pavlus'u alip Ares Tepesi Kurulu'na götürdüler. Ona, "Yaydigin bu yeni ögretinin ne oldugunu ögrenebilir miyiz?" dediler.
\par 20 "Kulagimiza yabanci gelen bazi konulardan söz ediyorsun. Bunlarin anlamini ögrenmek isteriz."
\par 21 Bütün Atinalilar ve kentte bulunan yabancilar, vakitlerini hep yeni düsünceleri anlatarak ve dinleyerek geçirirlerdi.
\par 22 Pavlus, Ares Tepesi Kurulu'nun önüne çikip sunlari söyledi: "Ey Atinalilar, sizin her bakimdan çok dindar oldugunuzu görüyorum.
\par 23 Ben çevrede dolasirken, tapindiginiz yerleri incelerken üzerinde, BILINMEYEN TANRI'YA diye yazilmis bir sunaga bile rastladim. Sizin bilmeden tapindiginiz bu Tanri'yi ben size tanitayim.
\par 24 "Dünyayi ve içindekilerin tümünü yaratan, yerin ve gögün Rabbi olan Tanri, elle yapilmis tapinaklarda oturmaz.
\par 25 Herkese yasam, soluk ve her seyi veren kendisi olduguna göre, bir seye gereksinmesi varmis gibi O'na insan eliyle hizmet edilmez. Tanri, bütün uluslari tek insandan türetti ve onlari yeryüzünün dört bucagina yerlestirdi.
\par 26 Uluslarin sürelerini ve yerlesecekleri bölgelerin sinirlarini önceden saptadi.
\par 27 Bunu, kendisini arasinlar ve el yordamiyla da olsa bulabilsinler diye yapti. Aslinda Tanri hiçbirimizden uzak degildir.
\par 28 Nitekim, 'O'nda yasiyor ve hareket ediyoruz; O'nda variz.' Bazi ozanlarinizin belirttigi gibi, 'Biz de O'nun soyundaniz.'
\par 29 "Tanri'nin soyundan oldugumuza göre, tanrisal özün, insan düsüncesi ve becerisiyle biçimlendirilmis altin, gümüs ya da tastan bir nesneye benzedigini düsünmemeliyiz.
\par 30 Tanri, geçmis dönemlerin bilgisizligini görmezlikten geldi; ama simdi her yerde herkesin tövbe etmesini buyuruyor.
\par 31 Çünkü dünyayi, atadigi Kisi araciligiyla adaletle yargilayacagi günü saptamistir. Bu Kisi'yi ölümden diriltmekle bunun güvencesini herkese vermistir."
\par 32 Ölülerin dirilmesiyle ilgili sözleri duyunca kimi alay etti, kimi de, "Seni bu konuda bir daha dinlemek isteriz" dedi.
\par 33 Bunun üzerine Pavlus aralarindan çikip gitti.
\par 34 Birkaç kisi ona katilip inandi. Bunlarin arasinda kurul üyesi Dionisios, Damaris adli bir kadin ve birkaç kisi daha vardi.

\chapter{18}

\par 1 Bundan sonra Pavlus Atina'dan ayrilip Korint'e gitti.
\par 2 Orada Pontus dogumlu, Akvila adinda bir Yahudi ile karisi Priskilla'yi buldu. Bunlar, Klavdius'un bütün Yahudiler'in Roma'yi terk etmesi yolundaki buyrugu üzerine, kisa süre önce Italya'dan gelmislerdi. Akvila ile Priskilla'nin yanina giden Pavlus, ayni meslekten oldugundan onlarla kalip çalisti. Çünkü meslekleri çadircilikti.
\par 4 Pavlus, her Sabat Günü* havrada tartisarak hem Yahudiler'i hem Grekler'i ikna etmeye çalisiyordu.
\par 5 Silas'la Timoteos Makedonya'dan gelince, Pavlus kendini tümüyle Tanri sözünü yaymaya verdi. Yahudiler'e, Isa'nin Mesih* olduguna dair taniklik ediyordu.
\par 6 Ama Yahudiler karsi gelip ona sövmeye baslayinca Pavlus, giysilerini silkerek, "Basiniza geleceklerin sorumlusu sizsiniz!" dedi. "Sorumluluk benden gitti. Bundan böyle öteki uluslara gidecegim."
\par 7 Pavlus oradan çikti, Tanri'ya tapan Titius Yustus adli birinin evine gitti. Yustus'un evi havranin bitisigindeydi.
\par 8 Havranin yöneticisi Krispus bütün ev halkiyla birlikte Rab'be inandi. Pavlus'u dinleyen Korintliler'den birçogu da inanip vaftiz* oldu.
\par 9 Bir gece Rab bir görümde Pavlus'a, "Korkma" dedi, "Konus, susma!
\par 10 Ben seninle birlikteyim; hiç kimse sana dokunmayacak, kötülük yapmayacak. Çünkü bu kentte benim halkim çoktur."
\par 11 Pavlus, orada bir buçuk yil kaldi ve halka sürekli Tanri'nin sözünü ögretti.
\par 12 Gallio'nun Ahaya Valisi oldugu siralarda, hep birlikte Pavlus'a karsi gelen Yahudiler onu mahkemeye çikardilar.
\par 13 "Bu adam Yasa'ya aykiri biçimde Tanri'ya tapinmalari için insanlari kandiriyor" dediler.
\par 14 Pavlus tam söze baslayacakken Gallio Yahudiler'e söyle dedi: "Ey Yahudiler, davaniz bir haksizlik ya da ciddi bir suçla ilgili olsaydi, sizleri sabirla dinlemem gerekirdi.
\par 15 Ama sorun bir ögreti, bazi adlar ve kendi yasanizla ilgili olduguna göre, bu davaya kendiniz bakin. Ben böyle seylere yargiçlik etmek istemem."
\par 16 Sonra Gallio onlari mahkemeden kovdu.
\par 17 Hep birlikte, havranin yöneticisi Sostenis'i yakalayip mahkemenin önünde dövdüler. Gallio ise olup bitenlere hiç aldirmadi.
\par 18 Pavlus Korint'teki kardeslerin yaninda bir süre daha kaldi. Sonra onlarla vedalasti, Priskilla ve Akvila ile birlikte Suriye'ye gitmek üzere gemiyle yola çikti. Adakta bulunmus oldugu için Kenhere'de saçlarini kestirmisti.
\par 19 Efes'e vardiklari zaman Priskilla ve Akvila'yi orada birakti. Kendisi havraya giderek Yahudiler'le tartismaya basladi.
\par 20 Bunlar daha uzun bir süre kalmasini istedilerse de, Pavlus kabul etmedi.
\par 21 Ama onlara veda ederken, "Tanri dilerse yaniniza yine dönecegim" dedi. Sonra Efes'ten denize açildi.
\par 22 Sezariye'ye vardiktan sonra Yerusalim'e gidip oradaki kiliseyi* ziyaret etti, oradan da Antakya'ya geçti.
\par 23 Bir süre orada kaldiktan sonra yola çikti; Galatya bölgesini ve Frikya'yi dolasarak bütün ögrencileri ruhça pekistirdi.
\par 24 Bu arada Iskenderiye dogumlu Apollos adinda bir Yahudi Efes'e geldi. Üstün bir konusma yetenegi olan Apollos, Kutsal Yazilar'i çok iyi biliyordu.
\par 25 Rab'bin yolunda egitilmis bir kisiydi. Atesli bir ruhla konusuyor ve sadece Yahya'nin vaftizini* bildigi halde Isa'yla ilgili gerçekleri dogru ögretiyordu.
\par 26 Havrada cesaretle konusmaya basladi. Kendisini dinleyen Priskilla ile Akvila, onu yanlarina alarak Tanri yolunu ona daha dogru biçimde açikladilar.
\par 27 Apollos Ahaya'ya gitmek isteyince kardesler onu cesaretlendirdiler. Onu iyi karsilamalari için oradaki ögrencilere mektup yazdilar. Apollos Ahaya'ya varinca Tanri'nin lütfuyla iman etmis olanlara çok yardim etti.
\par 28 Söyle ki Kutsal Yazilar'dan, Isa'nin Mesih* oldugunu kanitlayarak Yahudiler'in iddialarini açikça ve güçlü bir sekilde çürüttü.

\chapter{19}

\par 1 Apollos Korint'teyken Pavlus, iç bölgelerden geçerek Efes'e geldi. Orada bazi ögrencileri bularak onlara, "Iman ettiginiz zaman Kutsal Ruh'u aldiniz mi?" diye sordu. "Kutsal Ruh'un varligindan haberimiz yok ki!" dediler.
\par 3 "Öyleyse neye dayanarak vaftiz* oldunuz?" diye sordu. "Yahya'nin ögretisine dayanarak vaftiz olduk" dediler.
\par 4 Pavlus, "Yahya'nin yaptigi vaftiz, tövbeyle ilgili bir vaftizdi" dedi. "Halka, kendisinden sonra gelecek Olan'a, yani Isa'ya inanmalarini söyledi."
\par 5 Onlar bunu duyunca, Rab Isa'nin adiyla vaftiz oldular.
\par 6 Pavlus ellerini onlarin üzerine koyunca Kutsal Ruh üzerlerine indi ve bilmedikleri dillerle konusup peygamberlik etmeye basladilar.
\par 7 Asagi yukari on iki kisiydiler.
\par 8 Havraya giren Pavlus cesaretle konusmaya basladi. Üç ay boyunca oradakilerle tartisip durdu, onlari Tanri'nin Egemenligi* konusunda ikna etmeye çalisti.
\par 9 Ne var ki, bazilari sert bir tutum takinip ikna olmamakta direndiler ve Isa'nin yolunu halkin önünde kötülemeye basladilar. Bunun üzerine Pavlus onlardan ayrildi. Ögrencilerini de alip götürdü ve Tiranus'un dershanesinde her gün tartismalarini sürdürdü.
\par 10 Bu durum iki yil sürdü. Sonunda Yahudi olsun Grek* olsun, Asya Ili'nde* yasayan herkes Rab'bin sözünü isitti.
\par 11 Tanri, Pavlus'un eliyle olaganüstü mucizeler yaratiyordu.
\par 12 Söyle ki, Pavlus'un bedenine degen peskir ve pestamallar hasta olanlara götürüldügünde, hastaliklari yok oluyor, kötü ruhlar içlerinden çikiyordu.
\par 13 Çevrede dolasip kötü ruhlari kovmakla ugrasan bazi Yahudiler de kötü ruhlara tutsak olanlari Rab Isa'nin adini anarak kurtarmaya kalkistilar. "Pavlus'un tanittigi Isa'nin adiyla size emrediyoruz!" diyorlardi.
\par 14 Bunu yapanlar arasinda Skeva adli bir Yahudi baskâhinin yedi oglu da vardi.
\par 15 Kötü ruh ise onlara söyle karsilik verdi: "Isa'yi biliyor, Pavlus'u da taniyorum, ama siz kimsiniz?"
\par 16 Içinde kötü ruh bulunan adam onlara saldirdi, hepsini alt ederek bozguna ugratti. Öyle ki, o evden çiplak ve yarali olarak kaçtilar.
\par 17 Bu haber, Efes'te yasayan bütün Yahudiler'le Grekler'e ulasti. Hepsini bir korku aldi ve Rab Isa'nin adi büyük bir sayginlik kazandi.
\par 18 Iman edenlerin birçogu geliyor, yaptiklari kötülükleri itiraf edip anlatiyordu.
\par 19 Büyücülükle ugrasmis bir sürü kisi de kitaplarini toplayip herkesin önünde yaktilar. Kitaplarin degerini hesapladiklarinda toplam elli bin gümüs tuttugunu gördüler.
\par 20 Böylelikle Rab'bin sözü güçlü biçimde yayilip etkinlik kazaniyordu.
\par 21 Pavlus, bu olup bitenlerden sonra Makedonya ve Ahaya'dan geçip Yerusalim'e gitmeye karar verdi. "Oraya gittikten sonra Roma'yi da görmem gerek" diyordu.
\par 22 Yardimcilarindan ikisini, Timoteos ile Erastus'u Makedonya'ya göndererek kendisi bir süre daha Asya Ili'nde kaldi.
\par 23 O sirada Isa'nin yoluna iliskin büyük bir kargasalik çikti.
\par 24 Artemis Tapinagi'nin gümüsten maketlerini yapan Dimitrios adli bir kuyumcu, el sanatçilarina bir hayli is sagliyordu.
\par 25 Sanatçilari ve benzer islerle ugrasanlari bir araya toplayarak onlara söyle dedi: "Efendiler, bu isten büyük kazanç sagladigimizi biliyorsunuz.
\par 26 Ama Pavlus denen bu adamin, elle yapilan tanrilarin gerçek tanrilar olmadigini söyleyerek yalniz Efes'te degil, neredeyse bütün Asya Ili'nde* çok sayida kisiyi kandirip saptirdigini görüyor ve duyuyorsunuz.
\par 27 Hem bu sanatimiz sayginligini yitirmek tehlikesiyle karsi karsiyadir, hem de ulu tanriça Artemis'in Tapinagi'nin hiçe sayilmasi ve bütün Asya Ili'yle bütün dünyanin tapindigi tanriçanin, ululugundan yoksun kalmasi tehlikesi vardir."
\par 28 Oradakiler bunu duyunca öfkeyle doldular. "Efesliler'in Artemisi uludur!" diye bagirmaya basladilar.
\par 29 Kent büsbütün karisti. Halk, Pavlus'un yol arkadaslarindan Makedonyali Gayus ve Aristarhus'u yakalayip sürükleyerek birlikte tiyatroya kosustu.
\par 30 Pavlus halkin arasina girmek istediyse de, ögrenciler onu birakmadi.
\par 31 Hatta, Pavlus'un dostu olan bazi Asya Ili yöneticileri ona haber yollayarak tiyatroda görünmemesi için yalvardilar.
\par 32 Tiyatrodaki topluluk karisiklik içindeydi. Her kafadan bir ses çikiyordu. Çogu ne için toplandigini bile bilmiyordu.
\par 33 Yahudiler Iskender'i öne çikarinca kalabaliktan bazilari olayi ona bagladi. Eliyle bir isaret yapan Iskender, halka savunmasini yapmak istedi.
\par 34 Ama halk kendisinin Yahudi oldugunu anlayinca hep bir agizdan yaklasik iki saat boyunca, "Efesliler'in Artemisi uludur!" diye bagirip durdu.
\par 35 Kalabaligi yatistiran belediye yazmani, "Ey Efesliler" dedi, "Efes Kenti'nin, ulu Artemis Tapinagi'nin ve gökten düsen kutsal tasin bekçisi oldugunu bilmeyen var mi?
\par 36 Bunlari hiç kimse inkâr edemez. Bunun için sakin olmaniz ve düsüncesiz bir sey yapmamaniz gerekir.
\par 37 Buraya getirdiginiz bu adamlar, ne tapinaklari yagma ettiler, ne de tanriçamiza sövdüler.
\par 38 Dimitrios ve sanatçi arkadaslarinin herhangi birinden sikâyeti varsa, mahkemeler açik, yargiçlar da var. Karsilikli suçlamalarini orada yapsinlar.
\par 39 Sorusturacaginiz baska bir durum varsa, bunun yasal bir toplantida çözümlenmesi gerekir.
\par 40 Bugünkü olaylardan ötürü ayaklanma suçundan yargilanmak tehlikesindeyiz. Hiçbir gerekçesi olmayan bu kargasanin hesabini veremeyecegiz."
\par 41 Bunlari söyledikten sonra toplulugu dagitti.

\chapter{20}

\par 1 Pavlus, kargasalik yatistiktan sonra ögrencileri çagirtip onlari yüreklendirdi. Sonra kendilerine veda ederek Makedonya'ya gitmek üzere yola çikti.
\par 2 O yöreleri dolasarak imanlilari yüreklendiren birçok konusmalar yaptiktan sonra Yunanistan'a gitti.
\par 3 Orada üç ay kaldi. Suriye'ye deniz yoluyla gitmek üzereyken Yahudiler'in kendisine karsi bir düzen kurmasi nedeniyle dönüsü Makedonya üzerinden yapmaya karar verdi.
\par 4 Piros oglu Veriyali Sopater, Selanikliler'den Aristarhus ile Sekundus, Derbeli Gayus, Timoteos ve Asya Ili'nden* Tihikos ile Trofimos onunla birlikte gittiler.
\par 5 Bunlar önden gidip bizi Troas'ta beklediler.
\par 6 Biz de Mayasiz Ekmek Bayrami'ndan* sonra Filipi'den denize açilip bes günde Troas'a gelerek onlarla bulustuk. Orada yedi gün kaldik.
\par 7 Haftanin ilk günü* ekmek bölmek için bir araya toplandigimizda Pavlus imanlilara bir konusma yapti. Ertesi gün oradan ayrilacagi için konusmasini gece yarisina dek sürdürdü.
\par 8 Toplanmis oldugumuz üst kattaki odada birçok kandil yaniyordu.
\par 9 Eftihos adli bir delikanli pencerede oturuyordu. Pavlus konusmasini uzattikça Eftihos'u uyku basti. Uykuya dalinca da ikinci kattan asagi düstü ve yerden ölüsü kaldirildi.
\par 10 Asagi inen Pavlus delikanlinin üzerine kapanip onu kucakladi. "Telaslanmayin, yasiyor!" dedi.
\par 11 Sonra yukari çikip ekmek böldü ve yemek yedi. Gün doguncaya dek onlarla uzun uzun konustu, sonra oradan ayrildi.
\par 12 Çocugu diri olarak evine götüren imanlilar bu olaydan büyük cesaret aldilar.
\par 13 Biz önden giderek gemiye bindik ve Assos'a hareket ettik. Pavlus'u oradan alacaktik. Kendisi karadan gitmek istedigi için bunu böyle düzenlemisti.
\par 14 Bizi Assos'ta karsilayinca onu gemiye alip Midilli'ye geçtik.
\par 15 Oradan denize açilip ertesi gün Sakiz Adasi'nin karsisina geldik. Üçüncü gün Sisam'a ugradik ve bir gün sonra Milet'e vardik.
\par 16 Pavlus, Asya Ili'nde vakit kaybetmemek için Efes'e ugramamaya karar vermisti. Pentikost Günü Yerusalim'de olabilmek umuduyla acele ediyordu.
\par 17 Pavlus, Milet'ten Efes'e haber yollayarak kilisenin ihtiyarlarini* yanina çagirtti.
\par 18 Yanina geldikleri zaman onlara söyle dedi: "Asya Ili'ne* ayak bastigim ilk günden beri, sizinle bulundugum bütün süre boyunca, nasil davrandigimi biliyorsunuz.
\par 19 Yahudiler'in kurdugu düzenlerden çektigim sikintilarin ortasinda Rab'be tam bir alçakgönüllülükle, gözyaslari içinde kulluk ettim.
\par 20 Yararli olan herhangi bir seyi size duyurmaktan, gerek açikta gerek evden eve dolasarak size ögretmekten çekinmedim.
\par 21 Hem Yahudiler'i hem de Grekler'i*, tövbe edip Tanri'ya dönmeye ve Rabbimiz Isa'ya inanmaya çagirdim.
\par 22 "Simdi de Ruh'a boyun egerek Yerusalim'e gidiyorum. Orada basima neler gelecegini bilmiyorum.
\par 23 Ancak Kutsal Ruh, beni zincirler ve sikintilarin bekledigine dair her kentte beni uyariyor.
\par 24 Canimi hiç önemsemiyorum, ona deger vermiyorum. Yeter ki yarisi bitireyim ve Rab Isa'dan aldigim görevi, Tanri'nin lütfunu bildiren Müjde'ye taniklik etme görevini tamamlayayim.
\par 25 "Simdi aralarinda dolasip Tanri'nin Egemenligi'ni duyurdugum sizlerden hiçbirinin yüzümü bir daha görmeyecegini biliyorum.
\par 26 Bu yüzden bugün size sunu açikça söyleyeyim: Ben kimsenin ugrayacagi cezadan sorumlu degilim.
\par 27 Tanri'nin istegini size tam olarak bildirmekten çekinmedim.
\par 28 Kendinize ve Kutsal Ruh'un sizi gözetmen olarak görevlendirdigi bütün sürüye göz kulak olun. Rab'bin kendi kani pahasina sahip oldugu kiliseyi gütmek üzere atandiniz.
\par 29 Ben gittikten sonra sürüyü esirgemeyen yirtici kurtlarin araniza girecegini biliyorum.
\par 30 Hatta ögrencileri kendi peslerinden sürüklemek için sizin aranizdan da sapik sözler söyleyen kisiler çikacak.
\par 31 Bunun için uyanik durun. Üç yil boyunca, araliksiz, gece gündüz demeden, gözyasi dökerek her birinizi nasil uyardigimi hatirlayin.
\par 32 "Simdi sizi Tanri'ya ve O'nun lütfunu bildiren söze emanet ediyorum. Bu söz, sizi ruhça gelistirecek ve kutsal kilinmis olan bütün insanlar arasinda mirasa kavusturacak güçtedir.
\par 33 Ben hiç kimsenin altinina, gümüsüne ya da giysisine göz dikmedim.
\par 34 Siz de bilirsiniz ki, bu eller hem benim, hem de benimle birlikte olanlarin gereksinmelerini karsilamak için hizmet etmistir.
\par 35 Yaptigim her iste sizlere, böyle emek vererek güçsüzlere yardim etmemiz ve Rab Isa'nin, 'Vermek, almaktan daha büyük mutluluktur' diyen sözünü unutmamamiz gerektigini gösterdim."
\par 36 Pavlus bu sözleri söyledikten sonra diz çöküp onlarla birlikte dua etti.
\par 37 Sonra hepsi aci aci aglayarak Pavlus'un boynuna sarildilar, onu öptüler.
\par 38 Onlari en çok üzen, "Yüzümü bir daha görmeyeceksiniz" demesi oldu. Sonra onu gemiye kadar geçirdiler.

\chapter{21}

\par 1 Onlardan ayrilinca denize açilip dogru Istanköy'e gittik. Ertesi gün Rodos'a, oradan da Patara'ya geçtik.
\par 2 Fenike'ye gidecek bir gemi bulduk, buna binip denize açildik.
\par 3 Kibris'i görünce güneyinden geçerek Suriye'ye yöneldik ve Sur Kenti'nde karaya çiktik. Gemi, yükünü orada bosaltacakti.
\par 4 Isa'nin oradaki ögrencilerini arayip bulduk ve yanlarinda bir hafta kaldik. Ögrenciler Ruh'un yönlendirmesiyle Pavlus'u Yerusalim'e gitmemesi için uyardilar.
\par 5 Günümüz dolunca kentten ayrilip yolumuza devam ettik. Imanlilarin hepsi, esleri ve çocuklariyla birlikte bizi kentin disina kadar geçirdiler. Deniz kiyisinda diz çöküp dua ettik.
\par 6 Birbirimizle vedalastiktan sonra biz gemiye bindik, onlar da evlerine döndüler.
\par 7 Sur'dan deniz yolculugumuza devam ederek Batlamya Kenti'ne geldik. Oradaki kardesleri ziyaret edip bir gün yanlarinda kaldik.
\par 8 Ertesi gün ayrilip Sezariye'ye geldik. Yediler'den biri olan müjdeci Filipus'un evine giderek onun yaninda kaldik.
\par 9 Bu adamin peygamberlik eden, evlenmemis dört kizi vardi.
\par 10 Oraya varisimizdan birkaç gün sonra Yahudiye'den Hagavos adli bir peygamber geldi.
\par 11 Bu adam bize yaklasip Pavlus'un kusagini aldi, bununla kendi ellerini ayaklarini baglayarak dedi ki, "Kutsal Ruh söyle diyor: 'Yahudiler, bu kusagin sahibini Yerusalim'de böyle baglayip öteki uluslara teslim edecekler.'"
\par 12 Bu sözleri duyunca hem bizler hem de oralilar Yerusalim'e gitmemesi için Pavlus'a yalvardik.
\par 13 Bunun üzerine Pavlus söyle karsilik verdi: "Ne yapiyorsunuz, ne diye aglayip yüregimi sizlatiyorsunuz? Ben Rab Isa'nin adi ugruna Yerusalim'de yalniz baglanmaya degil, ölmeye de hazirim."
\par 14 Pavlus'u ikna edemeyince, "Rab'bin istedigi olsun" diyerek sustuk.
\par 15 Bir süre sonra hazirligimizi yapip Yerusalim'e dogru yola çiktik.
\par 16 Sezariye'deki ögrencilerden bazilari da bizimle birlikte geldiler. Bizi, evinde kalacagimiz adama, eski ögrencilerden Kibrisli Minason'a götürdüler.
\par 17 Yerusalim'e vardigimiz zaman kardesler bizi sevinçle karsiladilar.
\par 18 Ertesi gün Pavlus'la birlikte Yakup'u görmeye gittik. Ihtiyarlarin* hepsi orada toplanmisti.
\par 19 Pavlus, onlarin hal hatirini sorduktan sonra, hizmetinin araciligiyla Tanri'nin öteki uluslar arasinda yaptiklarini teker teker anlatti.
\par 20 Bunlari isitince Tanri'yi yücelttiler. Pavlus'a, "Görüyorsun kardes, Yahudiler arasinda binlerce imanli var ve hepsi Kutsal Yasa'nin candan savunucusudur" dediler.
\par 21 "Ne var ki, duyduklarina göre sen öteki uluslar arasinda yasayan bütün Yahudiler'e, çocuklarini sünnet etmemelerini, törelerimize uymamalarini söylüyor, Musa'nin Yasasi'na sirt çevirmeleri gerektigini ögretiyormussun.
\par 22 Simdi ne yapmali? Senin buraya geldigini mutlaka duyacaklar.
\par 23 Bunun için sana dedigimizi yap. Aramizda adak adamis dört kisi var.
\par 24 Bunlari yanina al, kendileriyle birlikte arinma törenine katil. Baslarini tiras edebilmeleri için kurban masraflarini sen öde. Böylelikle herkes, seninle ilgili duyduklarinin asilsiz oldugunu, senin de Kutsal Yasa'ya uygun olarak yasadigini anlasin.
\par 25 Öteki uluslardan olan imanlilara gelince, biz onlara, putlara sunulan kurbanlarin etinden, kandan, bogularak öldürülen hayvanlardan ve fuhustan sakinmalarini öngören kararimizi yazmistik."
\par 26 Bunun üzerine Pavlus o dört kisiyi yanina aldi, ertesi gün onlarla birlikte arinma törenine katildi. Sonra tapinaga girerek arinma günlerinin ne zaman tamamlanacagini, her birinin adina ne zaman kurban sunulacagini bildirdi.
\par 27 Yedi günlük süre bitmek üzereydi. Asya Ili'nden* bazi Yahudiler Pavlus'u tapinakta görünce bütün kalabaligi kiskirtarak onu yakaladilar.
\par 28 "Ey Israilliler, yardim edin!" diye bagirdilar. "Her yerde herkese, halkimiza, Kutsal Yasa'ya ve bu kutsal yere karsi ögretiler yayan adam budur. Üstelik tapinaga bazi Grekler'i sokarak bu kutsal yeri kirletti."
\par 29 Bu Yahudiler, daha önce kentte Pavlus'un yaninda gördükleri Efesli Trofimos'un, Pavlus tarafindan tapinaga sokuldugunu saniyorlardi.
\par 30 Bütün kent ayaga kalkmisti. Her taraftan kosusup gelen halk Pavlus'u tutup tapinaktan disari sürükledi. Arkasindan tapinagin kapilari hemen kapatildi.
\par 31 Onlar Pavlus'u öldürmeye çalisirken, bütün Yerusalim'in karistigi haberi Roma taburunun komutanina ulasti.
\par 32 Komutan hemen yüzbasilarla askerleri yanina alarak kalabaligin oldugu yere kostu. Komutanla askerleri gören halk Pavlus'u dövmeyi birakti.
\par 33 O zaman komutan yaklasip Pavlus'u yakaladi, çift zincirle baglanmasi için buyruk verdi. Sonra, "Kimdir bu adam, ne yapti?" diye sordu.
\par 34 Kalabaliktakilerin her biri ayri bir sey bagiriyordu. Kargasaliktan ötürü kesin bilgi edinemeyen komutan, Pavlus'un kaleye götürülmesini buyurdu.
\par 35 Pavlus merdivenlere geldiginde kalabalik öylesine azmisti ki, askerler onu tasimak zorunda kaldilar.
\par 36 Kalabalik, "Öldürün onu!" diye bagirarak onlari izliyordu.
\par 37 Kaleden içeri girmek üzereyken Pavlus komutana, "Sana bir sey söyleyebilir miyim?" dedi. Komutan, "Grekçe biliyor musun?" dedi.
\par 38 "Sen bundan bir süre önce bir ayaklanma baslatip dört bin tedhisçiyi çöle götüren Misirli degil misin?"
\par 39 Pavlus, "Ben Kilikya'dan Tarsuslu bir Yahudi, hiç de önemsiz olmayan bir kentin vatandasiyim" dedi. "Rica ederim, halka birkaç söz söylememe izin ver."
\par 40 Komutanin izin vermesi üzerine Pavlus merdivende dikilip eliyle halka bir isaret yapti. Derin bir sessizlik olunca, Ibrani* dilinde konusmaya basladi.

\chapter{22}

\par 1 "Kardesler ve babalar, size simdi yapacagim savunmayi dinleyin" dedi.
\par 2 Pavlus'un kendilerine Ibrani dilinde seslendigini duyduklarinda daha derin bir sessizlik oldu. Pavlus söyle devam etti: "Ben Yahudi'yim. Kilikya'nin Tarsus Kenti'nde dogdum ve burada, Yerusalim'de Gamaliel'in dizinin dibinde büyüdüm. Atalarimizin yasasiyla ilgili siki bir egitimden geçtim. Bugün hepinizin yaptigi gibi, ben de Tanri için gayretle çalisan biriydim.
\par 4 Isa'nin yolundan gidenlere öldüresiye zulmeder, kadin erkek demeden onlari baglayip hapse atardim.
\par 5 Baskâhin ile bütün kurul üyeleri söylediklerimi dogrulayabilirler. Onlardan Yahudi kardeslere yazilmis mektuplar alarak Sam'a dogru yola çikmistim. Amacim, oradaki Isa inanlilarini da cezalandirmak üzere baglayip Yerusalim'e getirmekti.
\par 6 "Ben ögleye dogru yol alip Sam'a yaklasirken, birdenbire gökten parlak bir isik çevremi aydinlatti.
\par 7 Yere yikildim. Bir sesin bana, 'Saul, Saul! Neden bana zulmediyorsun?' dedigini isittim.
\par 8 "'Ey Efendim, sen kimsin?' diye sordum. "Ses bana, 'Ben senin zulmettigin Nasirali Isa'yim' dedi.
\par 9 Yanimdakiler isigi gördülerse de, benimle konusanin söylediklerini anlamadilar.
\par 10 "'Rab, ne yapmaliyim?' diye sordum. "Rab bana, 'Kalk, Sam'a git' dedi, 'Yapmani tasarladigim her sey orada sana bildirilecek.'
\par 11 Parlayan isigin görkeminden gözlerim görmez oldugundan, yanimdakiler elimden tutup beni Sam'a götürdüler.
\par 12 "Orada Hananya adinda dindar, Kutsal Yasa'ya bagli biri vardi. Kentte yasayan bütün Yahudiler'in kendisinden övgüyle söz ettigi bu adam gelip yanimda durdu ve, 'Saul kardes, gözlerin görsün!' dedi. Ve ben o anda onu gördüm.
\par 14 "Hananya, 'Atalarimizin Tanrisi, kendisinin istegini bilmen ve Adil Olan'i görüp O'nun agzindan bir ses isitmen için seni seçmistir' dedi.
\par 15 'Görüp isittiklerini bütün insanlara duyurarak O'nun tanikligini yapacaksin.
\par 16 Haydi, ne bekliyorsun? Kalk, O'nun adini anarak vaftiz* ol ve günahlarindan arin!'
\par 17 "Ben Yerusalim'e döndükten sonra, tapinakta dua ettigim bir sirada, kendimden geçerek Rab'bi gördüm. Bana, 'Çabuk ol' dedi, 'Yerusalim'den hemen ayril. Çünkü benimle ilgili tanikligini kabul etmeyecekler.'
\par 19 "'Ya Rab' dedim, 'Benim havradan havraya giderek sana inananlari tutuklayip dövdügümü biliyorlar.
\par 20 Üstelik sana taniklik eden Istefanos'un kani döküldügü zaman, ben de oradaydim. Onu öldürenlerin kaftanlarina bekçilik ederek yapilanlari onayladim.'
\par 21 "Rab bana, 'Git' dedi, 'Seni uzaktaki uluslara gönderecegim.'"
\par 22 Pavlus'u buraya kadar dinleyenler, bu söz üzerine, "Böylesini yeryüzünden temizlemeli, yasamasi uygun degil!" diye seslerini yükselttiler.
\par 23 Onlar böyle bagirir, üstlüklerini sallayip havaya toz savururken komutan, Pavlus'un kalenin içine götürülmesini buyurdu. Halkin neden Pavlus'un aleyhine böyle bagirdigini ögrenmek için onun kamçilanarak sorguya çekilmesini istedi.
\par 25 Kendisini sirimlarla baglayip kollarini geriyorlardi ki, Pavlus orada duran yüzbasiya, "Mahkemesi yapilmamis bir Roma vatandasini* kamçilamaniz yasaya uygun mudur?" dedi.
\par 26 Yüzbasi bunu duyunca gidip komutana haber verdi. "Ne yapiyorsun?" dedi. "Bu adam Roma vatandasiymis."
\par 27 Komutan Pavlus'un yanina geldi, "Söyle bakayim, sen Romali misin?" diye sordu. Pavlus da, "Evet" dedi.
\par 28 Komutan, "Ben bu vatandasligi yüklü bir para ödeyerek elde ettim" diye karsilik verdi. Pavlus, "Ben ise dogustan Roma vatandasiyim" dedi.
\par 29 Onu sorguya çekecek olanlar hemen yanindan çekilip gittiler. Kendisini baglatan komutan da, onun Roma vatandasi oldugunu anlayinca korktu.
\par 30 Komutan ertesi gün, Yahudiler'in Pavlus'u tam olarak neyle suçladiklarini ögrenmek için onu hapisten getirtti, baskâhinlerle bütün Yüksek Kurul'un* toplanmasi için buyruk verdi ve onu asagi indirip Kurul'un önüne çikardi.

\chapter{23}

\par 1 Yüksek Kurul'u dikkatle süzen Pavlus, "Kardesler" dedi, "Ben bugüne dek Tanri'nin önünde tertemiz bir vicdanla yasadim."
\par 2 Baskâhin Hananya, Pavlus'un yaninda duranlara onun agzina vurmalari için buyruk verdi.
\par 3 Bunun üzerine Pavlus ona, "Seni badanali duvar, Tanri sana vuracaktir!" dedi. "Hem oturmus Kutsal Yasa'ya göre beni yargiliyorsun, hem de Yasa'yi çigneyerek beni dövdürüyorsun."
\par 4 Çevrede duranlar, "Tanri'nin baskâhinine hakaret mi ediyorsun?" dediler.
\par 5 Pavlus, "Kardesler, baskâhin oldugunu bilmiyordum" dedi. "Nitekim, 'Halkini yönetenleri kötüleme' diye yazilmistir."
\par 6 Oradakilerden bir bölümünün Saduki*, öbürlerinin de Ferisi* mezhebinden oldugunu anlayan Pavlus, Yüksek Kurul'a söyle seslendi: "Kardesler, ben özbeöz Ferisi'yim. Ölülerin dirilecegi umudunu besledigim için yargilanmaktayim."
\par 7 Pavlus'un bu sözü üzerine Ferisiler'le Sadukiler çekismeye basladilar, Kurul ikiye bölündü.
\par 8 Sadukiler, ölümden dirilis, melek ve ruh yoktur derler; Ferisiler ise bunlarin hepsine inanirlar.
\par 9 Kurul'da büyük bir kargasalik çikti. Ferisi mezhebinden bazi din bilginleri* kalkip atesli bir sekilde, "Bu adamda hiçbir suç görmüyoruz" diye bagirdilar. "Bir ruh ya da bir melek kendisiyle konusmussa, ne olmus?"
\par 10 Çekisme öyle siddetlendi ki komutan, Pavlus'u parçalayacaklar diye korktu. Askerlerin asagi inip onu zorla aralarindan alarak kaleye götürmelerini buyurdu.
\par 11 O gece Rab Pavlus'a görünüp, "Cesur ol" dedi, "Yerusalim'de benimle ilgili nasil taniklik ettinse, Roma'da da öyle taniklik etmen gerekir."
\par 12 Ertesi sabah Yahudiler aralarinda gizli bir anlasma yaptilar. "Pavlus'u öldürmeden bir sey yiyip içersek, bize lanet olsun!" diye ant içtiler.
\par 13 Bu anlasmaya katilanlarin sayisi kirki asiyordu.
\par 14 Bunlar baskâhinlerle ileri gelenlerin yanina gidip söyle dediler: "Biz, 'Pavlus'u öldürmeden agzimiza bir sey koyarsak, bize lanet olsun!' diye ant içtik.
\par 15 Simdi siz Yüksek Kurul'la* birlikte, Pavlus'a iliskin durumu daha ayrintili bir sekilde arastiracakmis gibi, komutanin onu size getirmesini rica edin. Biz de, Pavlus daha Kurul'a gelmeden onu öldürmeye hazir olacagiz."
\par 16 Ne var ki, Pavlus'un kizkardesinin oglu onlarin pusu kurdugunu duydu. Varip kaleye girdi ve haberi Pavlus'a iletti.
\par 17 Yüzbasilardan birini yanina çagiran Pavlus, "Bu genci komutana götür, kendisine iletecegi bir haber var" dedi.
\par 18 Yüzbasi, genci alip komutana götürdü. "Tutuklu Pavlus beni çagirip bu genci sana getirmemi rica etti. Sana bir söyleyecegi varmis" dedi.
\par 19 Komutan, genci elinden tutup bir yana çekti. "Bana bildirmek istedigin nedir?" diye sordu.
\par 20 "Yahudiler sözbirligi ettiler" dedi, "Pavlus'la ilgili durumu daha ayrintili bir sekilde arastirmak istiyorlarmis gibi, yarin onu Yüksek Kurul'a götürmeni rica edecekler.
\par 21 Ama sen onlara kanma! Aralarindan kirktan fazla kisi ona pusu kurmus bekliyor. 'Onu ortadan kaldirmadan bir sey yiyip içersek, bize lanet olsun!' diye ant içtiler. Simdi hazirlar, senden olumlu bir yanit gelmesini bekliyorlar."
\par 22 Komutan, "Bunlari bana açikladigini hiç kimseye söyleme" diye uyardiktan sonra genci saliverdi.
\par 23 Komutan, yüzbasilardan ikisini yanina çagirip söyle dedi: "Aksam saat* dokuzda Sezariye'ye hareket etmek üzere iki yüz piyade, yetmis atli ve iki yüz mizrakli hazirlayin.
\par 24 Ayrica Pavlus'u bindirip Vali Feliks'in yanina sag salim ulastirmak için hayvan saglayin."
\par 25 Sonra söyle bir mektup yazdi: "Klavdius Lisias'tan, Sayin Vali Feliks'e selam.
\par 27 Bu adami Yahudiler yakalamis öldürmek üzereydiler. Ne var ki, kendisinin Roma vatandasi oldugunu ögrenince askerlerle yetisip onu kurtardim.
\par 28 Kendisini neyle suçladiklarini bilmek istedigim için onu Yahudiler'in Yüksek Kurulu'nun önüne çikarttim.
\par 29 Suçlamanin, Yahudiler'in yasasina iliskin bazi sorunlarla ilgili oldugunu ögrendim. Ölüm ya da hapis cezasini gerektiren herhangi bir suçlama yoktu.
\par 30 Bana bu adama karsi bir tuzak kuruldugu bildirilince onu hemen sana gönderdim. Onu suçlayanlara da kendisiyle ilgili sikâyetlerini sana bildirmelerini buyurdum."
\par 31 Askerler, kendilerine verilen buyruk uyarinca Pavlus'u alip geceleyin Antipatris'e götürdüler.
\par 32 Ertesi gün, atlilari Pavlus'la birlikte yola devam etmek üzere birakarak kaleye döndüler.
\par 33 Atlilar Sezariye'ye varinca mektubu valiye verip Pavlus'u teslim ettiler.
\par 34 Vali mektubu okuduktan sonra Pavlus'un hangi ilden oldugunu sordu. Kilikyali oldugunu ögrenince, "Seni suçlayanlar da gelsin, o zaman seni dinlerim" dedi. Sonra Pavlus'un, Hirodes'in* sarayinda gözaltinda tutulmasi için buyruk verdi.

\chapter{24}

\par 1 Bundan bes gün sonra Baskâhin Hananya, bazi ileri gelenler ve Tertullus adli bir hatip Sezariye'ye gelip Pavlus'la ilgili sikâyetlerini valiye ilettiler.
\par 2 Pavlus çagrilinca Tertullus suçlamalarina basladi. "Ey erdemli Feliks!" dedi. "Senin sayende uzun süredir esenlik içinde yasamaktayiz. Aldigin önlemlerle de bu ulusun yararina olumlu gelismeler kaydedilmistir. Yaptiklarini, her zaman ve her yerde büyük bir sükranla aniyoruz.
\par 4 Seni fazla yormak istemiyorum; söyleyecegimiz birkaç sözü hosgörüyle dinlemeni rica ediyorum.
\par 5 "Biz sunu anladik ki, bu adam dünyanin her yaninda bütün Yahudiler arasinda kargasalik çikaran bir fesatçi ve Nasrani tarikatinin elebasilarindan biridir.
\par 6 Tapinagi bile kirletmeye kalkisti. Ama biz onu yakaladik. Onu sorguya çekersen, onunla ilgili bütün suçlamalarimizin dogrulugunu kendisinden ögrenebilirsin."
\par 9 Oradaki Yahudiler de anlatilanlarin dogru oldugunu söyleyerek bu suçlamalara katildilar.
\par 10 Valinin bir isareti üzerine Pavlus söyle karsilik verdi: "Senin yillardan beri bu ulusa yargiçlik ettigini bildigim için, kendi savunmami sevinçle yapiyorum.
\par 11 Sen kendin de ögrenebilirsin, tapinmak amaciyla Yerusalim'e gidisimden bu yana sadece on iki gün geçti.
\par 12 Beni ne tapinakta, ne havralarda, ne de kentin baska bir yerinde herhangi biriyle tartisirken ya da halki ayaklandirmaya çalisirken görmüslerdir.
\par 13 Su anda bana yönelttikleri suçlamalari da sana kanitlayamazlar.
\par 14 Bununla birlikte, sana sunu itiraf edeyim ki, kendilerinin tarikat dedikleri Yol'un bir izleyicisi olarak atalarimizin Tanrisi'na kulluk ediyorum. Kutsal Yasa'da ve peygamberlerin kitaplarinda yazili her seye inaniyorum.
\par 15 Ayni bu adamlarin kabul ettigi gibi, hem dogru kisilerin hem dogru olmayanlarin ölümden dirilecegine dair Tanri'ya umut bagladim.
\par 16 Bu nedenle ben gerek Tanri, gerek insanlar önünde vicdanimi temiz tutmaya her zaman özen gösteriyorum.
\par 17 "Uzun yillar sonra, ulusuma bagislar getirmek ve adaklar sunmak için Yerusalim'e geldim.
\par 18 Beni tapinakta adaklar sunarken bulduklari zaman arinmis durumdaydim. Çevremde ne bir kalabalik ne de karisiklik vardi. Ancak orada Asya Ili'nden* bazi Yahudiler bulunuyordu.
\par 19 Onlarin bana karsi bir diyecekleri varsa, senin önüne çikip suçlamalarini belirtmeleri gerekir.
\par 20 Buradakiler de, Yüksek Kurul'un* önündeki durusmam sirasinda bende ne suç bulduklarini açiklasinlar.
\par 21 Önlerine çikarildigimda, 'Bugün, ölülerin dirilisi konusunda tarafinizdan yargilanmaktayim' diye seslenmistim. Olsa olsa beni bu konuda suçlayabilirler."
\par 22 Isa'nin yoluna iliskin derin bilgisi olan Feliks durusmayi baska bir güne ertelerken, "Davanizla ilgili kararimi komutan Lisias gelince veririm" dedi.
\par 23 Oradaki yüzbasiya da Pavlus'u gözaltinda tutmasini, ama kendisine biraz serbestlik tanimasini, ona yardimda bulunmak isteyen dostlarindan hiçbirine engel olmamasini buyurdu.
\par 24 Birkaç gün sonra Feliks, Yahudi olan karisi Drusilla ile birlikte geldi, Pavlus'u çagirtarak Mesih Isa'ya olan inanci konusunda onu dinledi.
\par 25 Pavlus dogruluk, özdenetim ve gelecek olan yargi gününden söz edince Feliks korkuya kapildi. "Simdilik gidebilirsin" dedi, "Firsat bulunca seni yine çagirtirim."
\par 26 Bir yandan da Pavlus'un kendisine rüsvet verecegini umuyordu. Bu nedenle onu sik sik çagirtir, onunla sohbet ederdi.
\par 27 Iki yil dolunca görevini Porkius Festus'a devreden Feliks, Yahudiler'in gönlünü kazanmak amaciyla Pavlus'u hapiste birakti.

\chapter{25}

\par 1 Eyalete vardiktan üç gün sonra Festus, Sezariye'den Yerusalim'e gitti.
\par 2 Baskâhinlerle Yahudiler'in ileri gelenleri, Pavlus'la ilgili sikâyetlerini ona açikladilar. Festus'tan kendilerine bir iyilikte bulunmasini isteyerek Pavlus'u Yerusalim'e getirtmesi için yalvardilar. Bu arada pusu kurup Pavlus'u yolda öldüreceklerdi.
\par 4 Festus ise Pavlus'un Sezariye'de tutuklu bulundugunu, kendisinin de yakinda oraya gidecegini söyleyerek, "Aranizda yetkili olanlar benimle gelsinler; bu adam yanlis bir sey yapmissa, ona karsi suç duyurusunda bulunsunlar" dedi.
\par 6 Festus, onlarin arasinda sadece sekiz on gün kadar kaldi; sonra Sezariye'ye döndü. Ertesi gün yargi kürsüsüne oturarak Pavlus'un getirilmesini buyurdu.
\par 7 Pavlus içeri girince, Yerusalim'den gelen Yahudiler çevresini sardilar ve kanitlayamadiklari birçok agir suçlamada bulundular.
\par 8 Pavlus, "Ne Yahudiler'in yasasina, ne tapinaga, ne de Sezar'a* karsi hiçbir günah islemedim" diyerek kendini savundu.
\par 9 Yahudiler'in gönlünü kazanmak isteyen Festus, Pavlus'a söyle karsilik verdi: "Yerusalim'e gidip orada benim önümde bu konularda yargilanmak ister misin?"
\par 10 Pavlus, "Ben Sezar'in yargi kürsüsü önünde durmaktayim" dedi, "Burada yargilanmam gerekir. Sen de çok iyi biliyorsun ki, Yahudiler'e karsi hiçbir suç islemedim.
\par 11 Sayet suçum varsa, ölüm cezasini gerektirecek bir sey yapmissam, ölmekten çekinmem. Yok eger bunlarin bana karsi yaptigi suçlamalar asilsiz ise, hiç kimse beni onlarin eline teslim edemez. Davamin Sezar'a iletilmesini istiyorum."
\par 12 Festus, danisma kuruluyla görüstükten sonra su yaniti verdi: "Davani Sezar'a ilettin, Sezar'a gideceksin."
\par 13 Birkaç gün sonra Kral Agrippa ile Berniki, Festus'a bir nezaket ziyaretinde bulunmak üzere Sezariye'ye geldiler.
\par 14 Bir süre orada kaldilar. Bu arada Festus, Pavlus'la ilgili durumu krala anlatti. "Feliks'in tutuklu olarak biraktigi bir adam var" dedi.
\par 15 "Yerusalim'de bulundugum sirada Yahudiler'in baskâhinleriyle ileri gelenleri, onunla ilgili sikâyetlerini açikladilar, onu cezalandirmami istediler.
\par 16 "Ben onlara, 'Herhangi bir sanigi, kendisini suçlayanlarla yüzlestirmeden, kendisine yöneltilen ithamlarla ilgili olarak savunma firsati vermeden, onu suçlayanlarin eline teslim etmek Romalilar'in gelenegine aykiridir' dedim.
\par 17 Onlar benimle buraya gelince, hiç vakit kaybetmeden, ertesi gün yargi kürsüsüne oturup adamin getirilmesini buyurdum.
\par 18 Ne var ki, kalkip konusan davacilar ona, bekledigim türden kötülüklerle ilgili hiçbir suçlama yöneltmediler.
\par 19 Ancak onunla çekistikleri bazi sorunlar vardi. Bunlar, kendi dinlerine ve ölmüs de Pavlus'un iddiasina göre yasamakta olan Isa adindaki birine iliskin konulardi.
\par 20 Bunlari nasil sorusturacagimi bilemedigim için Pavlus'a, Yerusalim'e gidip orada bu konularda yargilanmaya razi olup olmayacagini sordum.
\par 21 Ama kendisi davasini Imparator'a iletti, Imparator'un kararina dek tutuklu kalmak istedi. Ben de onu Imparator'a gönderecegim zamana kadar tutuklu kalmasini buyurdum."
\par 22 Agrippa Festus'a, "Ben de bu adami dinlemek isterdim" dedi. Festus da, "Yarin onu dinlersin" dedi.
\par 23 Ertesi gün Agrippa ile Berniki büyük bir tantanayla gelip komutanlar ve kentin ileri gelenleriyle birlikte toplanti salonuna girdiler. Festus'un buyrugu üzerine Pavlus içeri getirildi.
\par 24 Festus, "Kral Agrippa ve burada bizimle bulunan bütün efendiler" dedi, "Yerusalim'de olsun, burada olsun, bütün Yahudi halkinin bana sikâyet ettigi bu adami görüyorsunuz. 'Onu artik yasatmamali!' diye haykiriyorlardi.
\par 25 Oysa ben, ölüm cezasini gerektiren hiçbir suç islemedigini anladim. Yine de, kendisi davasinin Imparator'a iletilmesini istediginden, onu göndermeye karar verdim.
\par 26 Ama Efendimiz'e bu adamla ilgili yazacak kesin bir seyim yok. Bu yüzden onu sizin önünüze ve özellikle, Kral Agrippa, senin önüne çikartmis bulunuyorum. Amacim, bu sorusturmanin sonucunda yazacak bir sey bulabilmektir.
\par 27 Bir tutukluyu Imparator'a gönderirken, kendisine yöneltilen suçlamalari belirtmemek bence anlamsiz."

\chapter{26}

\par 1 Agrippa Pavlus'a, "Kendini savunabilirsin" dedi. Bunun üzerine Pavlus elini uzatarak savunmasina söyle basladi: "Kral Agrippa! Yahudiler'in bana yönelttigi bütün suçlamalarla ilgili olarak savunmami bugün senin önünde yapacagim için kendimi mutlu sayiyorum.
\par 3 Özellikle suna seviniyorum ki, sen Yahudiler'in bütün törelerini ve sorunlarini yakindan bilen birisin. Bu nedenle beni sabirla dinlemeni rica ediyorum.
\par 4 "Bütün Yahudiler, gerek baslangiçta kendi memleketimde, gerek Yerusalim'de, gençligimden beri nasil yasadigimi bilirler.
\par 5 Beni eskiden beri tanirlar ve isteseler, geçmiste dinimizin en titiz mezhebi olan Ferisilige* bagli yasadigima taniklik edebilirler.
\par 6 Simdi ise, Tanri'nin atalarimiza olan vaadine umut bagladigim için burada bulunmakta ve yargilanmaktayim.
\par 7 Bu, on iki oymagimizin gece gündüz Tanri'ya canla basla kulluk ederek erismeyi umduklari vaattir. Ey kralim, Yahudiler'in bana yönelttikleri suçlamalar bu umutla ilgilidir.
\par 8 Sizler, Tanri'nin ölüleri diriltmesini neden 'inanilmaz' görüyorsunuz?
\par 9 "Dogrusu ben de, Nasirali Isa adina karsi elimden geleni yapmam gerektigi düsüncesindeydim.
\par 10 Ve Yerusalim'de bunu yaptim. Baskâhinlerden aldigim yetkiyle kutsallardan birçogunu hapse attirdim; ölüm cezasina çarptirildiklari zaman oyumu onlarin aleyhinde kullandim.
\par 11 Bütün havralari dolasip sik sik onlari cezalandirir, inandiklarina küfretmeye zorlardim. Öylesine kudurmustum ki, onlara zulmetmek için bulunduklari yabanci kentlere bile giderdim.
\par 12 "Bir keresinde baskâhinlerden aldigim yetki ve görevle Sam'a dogru yola çikmistim.
\par 13 Ey kralim, öglende yolda giderken, gökten gelip benim ve yol arkadaslarimin çevresini aydinlatan, günesten daha parlak bir isik gördüm.
\par 14 Hepimiz yere yikilmistik. Bir sesin bana Ibrani* dilinde seslendigini duydum. 'Saul, Saul, neden bana zulmediyorsun?' dedi. 'Üvendireye karsi tepmekle kendine zarar veriyorsun.'
\par 15 "Ben de, 'Ey Efendim, sen kimsin?' dedim. "'Ben senin zulmettigin Isa'yim' diye yanit verdi Rab.
\par 16 'Haydi, ayaga kalk. Seni hizmetimde görevlendirmek için sana göründüm. Hem gördüklerine, hem de kendimle ilgili sana göstereceklerime taniklik edeceksin.
\par 17 Seni kendi halkinin ve öteki uluslarin elinden kurtaracagim. Seni, uluslarin gözlerini açmak ve onlari karanliktan isiga, Seytan'in hükümranligindan Tanri'ya döndürmek için gönderiyorum. Öyle ki, bana iman ederek günahlarinin affina kavussunlar ve kutsal kilinanlarin arasinda yer alsinlar.'
\par 19 "Bunun için, ey Kral Agrippa, bu göksel görüme uymazlik etmedim.
\par 20 Önce Sam ve Yerusalim halkini, sonra bütün Yahudiye bölgesini ve öteki uluslari, tövbe edip Tanri'ya dönmeye ve bu tövbeye yarasir isler yapmaya çagirdim.
\par 21 Yahudiler'in beni tapinakta yakalayip öldürmeye kalkmalarinin nedeni buydu.
\par 22 Ama bugüne dek Tanri yardimcim oldu. Bu sayede burada duruyor, büyük küçük herkese taniklik ediyorum. Benim söylediklerim, peygamberlerin ve Musa'nin önceden haber verdigi olaylardan baska bir sey degildir.
\par 23 Onlar, Mesih'in* aci çekecegini ve ölümden dirilenlerin ilki olarak gerek kendi halkina, gerek öteki uluslara isigin dogusunu ilan edecegini bildirmislerdi."
\par 24 Pavlus bu sekilde savunmasini sürdürürken Festus yüksek sesle, "Pavlus, çildirmissin sen! Çok okumak seni delirtiyor!" dedi.
\par 25 Pavlus, "Sayin Festus" dedi, "Ben çildirmis degilim. Gerçek ve akla uygun sözler söylüyorum.
\par 26 Kral bu konularda bilgili oldugu için kendisiyle çekinmeden konusuyorum. Bu olaylardan hiçbirinin onun dikkatinden kaçmadigi kanisindayim. Çünkü bunlar ücra bir kösede yapilmis isler degildir.
\par 27 Kral Agrippa, sen peygamberlerin sözlerine inaniyor musun? Inandigini biliyorum."
\par 28 Agrippa Pavlus'a söyle dedi: "Bu kadar kisa bir sürede beni ikna edip Mesihçi mi yapacaksin?"
\par 29 "Ister kisa ister uzun sürede olsun" dedi Pavlus, "Tanri'dan dilerim ki yalniz sen degil, bugün beni dinleyen herkes, bu zincirler disinda benim gibi olsun!"
\par 30 Kral, vali, Berniki ve onlarla birlikte oturanlar kalkip disari çiktiktan sonra aralarinda söyle konustular: "Bu adamin, ölüm ya da hapis cezasini gerektiren bir sey yaptigi yok."
\par 32 Agrippa da Festus'a, "Bu adam davasini Sezar'a* iletmeseydi, serbest birakilabilirdi" dedi.

\chapter{27}

\par 1 Italya'ya dogru yelken açmamiza karar verilince, Pavlus'la öteki bazi tutuklulari Avgustus taburundan Yulius adli bir yüzbasiya teslim ettiler.
\par 2 Asya Ili'nin* kiyilarindaki limanlara ugrayacak olan bir Edremit gemisine binerek denize açildik. Selanik'ten Makedonyali Aristarhus da yanimizdaydi.
\par 3 Ertesi gün Sayda'ya ugradik. Pavlus'a dostça davranan Yulius, ihtiyaçlarini karsilamalari için dostlarinin yanina gitmesine izin verdi.
\par 4 Oradan yine denize açildik. Rüzgar ters yönden estigi için Kibris'in rüzgar altindan geçtik.
\par 5 Kilikya ve Pamfilya açiklarindan geçerek Likya'nin Mira Kenti'ne geldik.
\par 6 Orada, Italya'ya gidecek bir Iskenderiye gemisi bulan yüzbasi, bizi o gemiye bindirdi.
\par 7 Günlerce agir agir yol alarak Knidos Kenti'nin açiklarina güçlükle gelebildik. Rüzgar bize engel oldugundan Salmone burnundan dolanarak Girit'in rüzgar altindan geçtik.
\par 8 Kiyi boyunca güçlükle ilerleyerek Laseya Kenti'nin yakinlarinda bulunan ve Güzel Limanlar denilen bir yere geldik.
\par 9 Epey vakit kaybetmistik; oruç günü bile geçmisti. O mevsimde deniz yolculugu tehlikeli olacakti. Bu nedenle Pavlus onlari uyardi: "Efendiler" dedi, "Bu yolculugun yalniz yük ve gemiye degil, canlarimiza da çok zarar ve ziyan getirecegini görüyorum."
\par 11 Ama yüzbasi, Pavlus'un söylediklerini dinleyecegine, kaptanla gemi sahibinin sözüne uydu.
\par 12 Liman kislamaya elverisli olmadigindan gemidekilerin çogu, oradan tekrar denize açilmaya, mümkünse Feniks'e ulasip kisi orada geçirmeye karar verdiler.Feniks, Girit'in lodos ve karayele kapali bir limanidir.
\par 13 Güneyden hafif bir rüzgar esmeye baslayinca, bekledikleri anin geldigini sanarak demir aldilar; Girit kiyisini yakindan izleyerek ilerlemeye basladilar.
\par 14 Ne var ki, çok geçmeden karadan Evrakilon denen bir kasirga koptu.
\par 15 Kasirgaya tutulan gemi rüzgara karsi gidemeyince, kendimizi sürüklenmeye biraktik.
\par 16 Gavdos denen küçük bir adanin rüzgar altina siginarak geminin filikasini güçlükle saglama alabildik.
\par 17 Filikayi yukari çektikten sonra halatlar kullanarak gemiyi alttan kusattilar. Sirte Körfezi'nin sigliklarinda karaya oturmaktan korktuklari için yelken takimlarini indirip kendilerini sürüklenmeye biraktilar.
\par 18 Firtina bizi bir hayli hirpaladigi için ertesi gün gemiden yük atmaya basladilar.
\par 19 Üçüncü gün geminin takimlarini kendi elleriyle denize attilar.
\par 20 Günlerce ne günes ne de yildizlar göründü. Firtina da olanca siddetiyle sürdügünden, artik kurtulus umudunu tümden yitirmistik.
\par 21 Adamlar uzun zaman yemek yiyemeyince Pavlus ortaya çikip söyle dedi: "Efendiler, beni dinleyip Girit'ten ayrilmamaniz, bu zarar ve ziyana ugramamaniz gerekirdi.
\par 22 Simdi size ögüdüm su: Cesur olun! Gemi mahvolacak, ama aranizda hiçbir can kaybi olmayacak.
\par 23 Çünkü kendisine ait oldugum, kendisine kulluk ettigim Tanri'nin bir melegi bu gece yanima gelip dedi ki, 'Korkma Pavlus, Sezar'in* önüne çikman gerekiyor. Dahasi Tanri, seninle birlikte yolculuk edenlerin hepsini sana bagislamistir.'
\par 25 Bunun için efendiler, cesur olun! Tanri'ya inaniyorum ki, her sey tipki bana bildirildigi gibi olacak.
\par 26 Ancak bir adada karaya oturmamiz gerekiyor."
\par 27 On dördüncü gece Iyon Denizi'nde sürükleniyorduk. Gece yarisina dogru gemiciler karaya yaklastiklarini sezinlediler.
\par 28 Denizin derinligini ölçtüler ve yirmi kulaç oldugunu gördüler. Biraz ilerledikten sonra bir daha ölçtüler, on bes kulaç oldugunu gördüler.
\par 29 Kayaliklara bindirmekten korkarak kiçtan dört demir attilar ve günün tez dogmasi için dua ettiler.
\par 30 Bu sirada gemiciler gemiden kaçma girisiminde bulundular. Bas taraftan demir atacaklarmis gibi yapip filikayi denize indirdiler.
\par 31 Ama Pavlus yüzbasiyla askerlere, "Bunlar gemide kalmazsa, siz kurtulamazsiniz" dedi.
\par 32 Bunun üzerine askerler ipleri kesip filikayi denize düsürdüler.
\par 33 Gün dogmak üzereyken Pavlus herkesi yemek yemeye çagirdi. "Bugün on dört gündür kaygili bir bekleyis içindesiniz, hiçbir sey yemeyip aç kaldiniz" dedi.
\par 34 "Bunun için size rica ediyorum, yemek yiyin. Kurtulusunuz için bu gerekli.Hiçbirinizin basindan tek kil bile eksilmeyecektir."
\par 35 Pavlus bunlari söyledikten sonra ekmek aldi, hepsinin önünde Tanri'ya sükretti, ekmegi bölüp yemeye basladi.
\par 36 Hepsi bundan cesaret alarak yemek yedi.
\par 37 Gemide toplam iki yüz yetmis alti kisiydik.
\par 38 Herkes doyduktan sonra, bugdayi denize bosaltarak gemiyi hafiflettiler.
\par 39 Gündüz olunca gördükleri karayi taniyamadilar. Ama kumsali olan bir körfez farkederek, mümkünse gemiyi orada karaya oturtmaya karar verdiler.
\par 40 Demirleri kesip denizde biraktilar. Ayni anda dümenlerin iplerini çözüp ön yelkeni rüzgara vererek kumsala yöneldiler.
\par 41 Gemi bir kum yükseltisine çarpip karaya oturdu. Geminin basi kuma saplanip kimildamaz oldu, kiç tarafi ise dalgalarin siddetiyle dagilmaya basladi.
\par 42 Askerler, tutuklulardan hiçbiri yüzerek kaçmasin diye onlari öldürmek niyetindeydi.
\par 43 Ama Pavlus'u kurtarmak isteyen yüzbasi askerleri bu düsünceden vazgeçirdi. Önce yüzme bilenlerin denize atlayip karaya çikmalarini, sonra geriye kalanlarin, kiminin tahtalara kiminin de geminin öbür döküntülerine tutunarak onlari izlemesini buyurdu. Böylelikle herkes sag salim karaya çikti.

\chapter{28}

\par 1 Kurtulduktan sonra adanin Malta adini tasidigini ögrendik.
\par 2 Yerliler bize olaganüstü bir yakinlik gösterdiler. Hava yagisli ve soguk oldugu için ates yakip hepimizi dostça karsiladilar.
\par 3 Pavlus bir yigin çali çirpi toplayip atesin üzerine atti. O anda isidan kaçan bir engerek onun eline yapisti.
\par 4 Yerliler Pavlus'un eline asilan yilani görünce birbirlerine, "Bu adam kuskusuz bir katil" dediler. "Denizden kurtuldu, ama adalet onu yasatmadi."
\par 5 Ne var ki, elini silkip yilani atesin içine firlatan Pavlus hiçbir zarar görmedi.
\par 6 Halk, Pavlus'un bedeninin sismesini ya da birdenbire düsüp ölmesini bekliyordu. Ama uzun süre bekleyip de ona bir sey olmadigini görünce fikirlerini degistirdiler. "Bu bir ilahtir!" dediler.
\par 7 Bulundugumuz yerin yakininda adanin bas yetkilisi olan Publius adli birinin topraklari vardi. Bu adam bizi evine kabul ederek üç gün dostça agirladi.
\par 8 O sirada Publius'un babasi kanli ishale yakalanmis atesler içinde yatiyordu. Hastanin yanina giren Pavlus dua etti, ellerini üzerine koyup onu iyilestirdi.
\par 9 Bu olay üzerine adadaki öbür hastalar da gelip iyilestirildiler.
\par 10 Bizi bir sürü armaganla onurlandirdilar; denize açilacagimiz zaman gereksindigimiz malzemeleri gemiye yüklediler.
\par 11 Üç ay sonra, kisi adada geçiren ve ikiz tanrilar simgesini tasiyan bir Iskenderiye gemisiyle denize açildik.
\par 12 Sirakuza Kenti'ne ugrayip üç gün kaldik.
\par 13 Oradan da yolumuza devam ederek Regium'a geldik. Ertesi gün güneyden esmeye baslayan rüzgarin yardimiyla iki günde Puteoli'ye vardik.
\par 14 Orada buldugumuz kardesler, bizi yanlarinda bir hafta kalmaya çagirdilar. Sonunda Roma'ya vardik.
\par 15 Haberimizi alan Roma'daki kardesler, bizi karsilamak için Appius Çarsisi'na ve Üç Hanlar'a kadar geldiler. Pavlus onlari görünce Tanri'ya sükretti, yüreklendi.
\par 16 Roma'ya girdigimizde Pavlus'un, bir asker gözetiminde yalniz basina kalmasina izin verildi.
\par 17 Üç gün sonra Pavlus, Yahudiler'in ileri gelenlerini bir araya çagirdi. Bunlar toplandiklari zaman Pavlus kendilerine söyle dedi: "Kardesler, halkimiza ya da atalarimizin törelerine karsi hiçbir sey yapmadigim halde, Yerusalim'de tutuklanip Romalilar'in eline teslim edildim.
\par 18 Onlar beni sorguya çektikten sonra serbest birakmak istediler. Çünkü ölüm cezasini gerektiren hiçbir suç islememistim.
\par 19 Ama Yahudiler buna karsi çikinca, davami Sezar'a* iletmek zorunda kaldim.Bunu, kendi ulusumdan herhangi bir sikâyetim oldugu için yapmadim.
\par 20 Ben Israil'in umudu ugruna bu zincire vurulmus bulunuyorum. Sizi buraya,iste bu konuyu görüsmek ve konusmak için çagirdim."
\par 21 Onlar Pavlus'a, "Yahudiye'den seninle ilgili mektup almadik, oradan gelen kardeslerden hiçbiri de senin hakkinda kötü bir haber getirmedi, kötü bir sey söylemedi" dediler.
\par 22 "Biz senin fikirlerini senden duymak isteriz. Çünkü her yerde bu mezhebe karsi çikildigini biliyoruz."
\par 23 Pavlus'la bir gün kararlastirdilar ve o gün, daha büyük bir kalabalikla onun kaldigi yere geldiler. Pavlus sabahtan aksama dek onlara Tanri'nin Egemenligi'ne* iliskin açiklamalarda bulundu ve bu konuda taniklik etti. Gerek Musa'nin Yasasi'na, gerek peygamberlerin yazilarina dayanarak onlari Isa hakkinda ikna etmeye çalisti.
\par 24 Bazilari onun sözlerine inandi, bazilari ise inanmadi.
\par 25 Birbirleriyle anlasamayinca, Pavlus'un su son sözünden sonra ayrildilar: "Peygamber Yesaya araciligiyla atalariniza seslenen Kutsal Ruh dogru söyledi.
\par 26 Ruh dedi ki, 'Bu halka gidip sunu söyle: Duyacak duyacak, ama hiç anlamayacaksiniz, Bakacak bakacak, ama hiç görmeyeceksiniz.
\par 27 Çünkü bu halkin yüregi duygusuzlasti, Kulaklari agirlasti. Gözlerini de kapadilar. Öyle ki, gözleri görmesin, Kulaklari duymasin, yürekleri anlamasin, Ve bana dönmesinler. Dönselerdi, onlari iyilestirirdim.'
\par 28 "Sunu bilin ki, Tanri'nin sagladigi bu kurtulusun haberi öteki uluslara gönderilmistir. Ve onlar buna kulak vereceklerdir."
\par 30 Pavlus tam iki yil kendi kiraladigi evde kaldi ve ziyaretine gelen herkesi kabul etti.
\par 31 Hiçbir engelle karsilasmadan Tanri'nin Egemenligi'ni tam bir cesaretle duyuruyor, Rab Isa Mesih'le ilgili gerçekleri ögretiyordu.


\end{document}