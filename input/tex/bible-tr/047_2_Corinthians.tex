\begin{document}

\title{2 Korintliler}


\chapter{1}

\par 1 Tanri'nin istegiyle Mesih Isa'nin elçisi atanan ben Pavlus ve kardesimiz Timoteos'tan Ahaya'nin her yanindaki bütün kutsallara ve Tanri'nin Korint'teki kilisesine* selam!
\par 2 Babamiz Tanri'dan ve Rab Isa Mesih'ten sizlere lütuf ve esenlik olsun.
\par 3 Her türlü tesellinin kaynagi olan Tanri'ya, merhametli Baba'ya, Rabbimiz Isa Mesih'in Tanrisi ve Babasi'na övgüler olsun!
\par 4 Kendisinden aldigimiz teselliyle her türlü sikintida olanlari teselli edebilmemiz için bizi bütün sikintilarimizda teselli ediyor.
\par 5 Çünkü Mesih'in acilarini nasil büyük ölçüde çekiyorsak, Mesih sayesinde büyük teselli de buluyoruz.
\par 6 Sikinti çekiyorsak, bu sizin teselliniz ve kurtulusunuz içindir. Teselli buluyorsak bu, bizim çektigimiz acilarin aynisina dayanmanizda etkin olan bir teselli bulmaniz içindir.
\par 7 Size iliskin umudumuz sarsilmaz. Çünkü acilarimiza oldugu gibi, tesellimize de ortak oldugunuzu biliyoruz.
\par 8 Kardeslerim, Asya Ili'nde* çektigimiz sikintilardan habersiz kalmanizi istemiyoruz. Dayanabilecegimizden çok agir bir yük altindaydik. Öyle ki, yasamaktan bile umudumuzu kesmistik.
\par 9 Ölüme mahkûm oldugumuzu içimizde hissettik. Ama bu, kendimize degil, ölüleri dirilten Tanri'ya güvenmemiz için oldu.
\par 10 Tanri bizi böylesine büyük bir ölüm tehlikesinden kurtardi; daha da kurtaracaktir. Umudumuzu O'na bagladik. Siz de dualarinizla bize yardim ettikçe, bizi yine kurtaracaktir. Öyle ki, birçok kisinin dualariyla bize saglanan lütuftan ötürü birçoklarinin agzindan bizim için sükranlar sunulsun.
\par 12 Dünyaya ve özellikle size, insan bilgeligiyle degil, Tanri'nin lütfuyla, Tanri'dan gelen kutsallik ve içtenlikle davrandigimiza vicdanimiz taniktir. Ve biz bununla övünüyoruz.
\par 13 Okuyup anlayabileceginizden baska bir sey yazmiyoruz. Bizi bir ölçüde anladiginiz gibi, tümüyle anlayacaginizi umarim. Rabbimiz Isa'nin gününde bizim övüncümüz siz olacaginiz gibi, sizin övüncünüz de biz olalim.
\par 15 Bu güvenle, sizleri iki kez sevindirmek için önce size ugramak, sonra Makedonya'ya geçmek, Makedonya'dan yine size geri gelerek tarafinizdan Yahudiye'ye ugurlanmak niyetindeydim.
\par 17 Bunu isterken acaba kararsiz miydim? Ya da isteklerim benlikten mi doguyor ki, önce "Evet, evet", sonra "Hayir, hayir" diyeyim?
\par 18 Tanri'nin güvenilirligi hakki için diyorum ki, size ilettigimiz söz hem "evet" hem "hayir" degildir.
\par 19 Silvanus ve Timoteos'la birlikte size tanittigimiz Tanri'nin Oglu Isa Mesih hem "evet" hem "hayir" degildi. O'nda yalniz "evet" vardir.
\par 20 Çünkü Tanri'nin bütün vaatleri Mesih'te "evet"tir. Bu nedenle Tanri'nin yüceligi için Mesih araciligiyla Tanri'ya "Amin" deriz.
\par 21 Bizi sizinle birlikte Mesih'te pekistiren ve meshetmis* olan Tanri'dir.
\par 22 O bizi mühürledi, güvence olarak da yüreklerimize Kutsal Ruh'u yerlestirdi.
\par 23 Tanri'yi tanik tutarim ki, Korint'e dönmeyisimin nedeni sizi esirgemekti.
\par 24 Imaniniza egemen olmak istemiyoruz, sevinmeniz için sizinle birlikte çalisiyoruz. Çünkü imanda dimdik duruyorsunuz.

\chapter{2}

\par 1 Size tekrar keder dolu bir ziyaret yapmamaya karar verdim.
\par 2 Çünkü sizi kederlendirirsem, keder verdigim sizlerden baska beni kim sevindirecek?
\par 3 Bunu yazdim ki, geldigimde beni sevindirmesi gerekenler beni kederlendirmesin. Sevincimin hepinizin sevinci olduguna iliskin hepinize güvenim var.
\par 4 Kederlenesiniz diye degil, size besledigim derin sevgiyi anlayasiniz diye büyük bir sikinti ve yürek acisiyla gözyaslari içinde size yazdim.
\par 5 Eger biri bir baskasini kederlendirdiyse, beni degil abartmadan söyleyeyim bir dereceye kadar hepinizi kederlendirmis olur.
\par 6 Böyle birine çogunlugun verdigi bu ceza yeterlidir.
\par 7 Asiri kedere bogulmasin diye o kisiyi daha fazla cezalandirmayip bagislamali ve teselli etmelisiniz.
\par 8 Bunun için ona duydugunuz sevgiyi yenilemenizi rica ederim.
\par 9 Sizi sinamak ve her durumda söz dinleyenler olup olmadiginizi anlamak için yazdim size.
\par 10 Kimi bagislarsaniz, ben de onu bagislarim. Eger bir seyi bagisladimsa, bunu sizin için Mesih'in önünde bagisladim.
\par 11 Öyle ki, Seytan'in oyununa gelmeyelim. Çünkü onun düzenlerini bilmez degiliz.
\par 12 Mesih'in Müjdesi'ni yaymak amaciyla Troas'a geldigimde Rab'bin isi için bana bir kapi açildigi halde, kardesim Titus'u orada bulamadigim için iç huzurum yoktu. Bu nedenle oradakilere veda ederek Makedonya'ya gittim.
\par 14 Bizi her zaman Mesih'in zafer alayinda yürüten, O'nu tanimanin güzel kokusunu araciligimizla her yerde yayan Tanri'ya sükürler olsun!
\par 15 Çünkü biz hem kurtulanlar hem de mahvolanlar arasinda Tanri için Mesih'in güzel kokusuyuz.
\par 16 Mahvolanlar için ölüme götüren ölüm kokusu, kurtulanlar içinse yasama götüren yasam kokusuyuz. Böylesi bir ise kim yeterlidir?
\par 17 Birçoklari gibi, Tanri'nin sözünü ticaret araci yapanlar degiliz. Tanri tarafindan gönderilen ve Mesih'e ait olan kisiler olarak Tanri'nin önünde içtenlikle konusuyoruz.

\chapter{3}

\par 1 Kendimizi yine tavsiye etmeye mi basliyoruz? Yoksa bazilari gibi size ya da sizden tavsiye mektuplarina ihtiyacimiz mi var?
\par 2 Bütün insanlarca bilinen ve okunan, yüreklerimize yazilmis mektubumuz sizsiniz.
\par 3 Hizmetimizin sonucu olup mürekkeple degil, yasayan Tanri'nin Ruhu'yla, tas levhalara degil, insan yüreginin levhalarina yazilmis Mesih'in mektubu oldugunuz açiktir.
\par 4 Mesih sayesinde Tanri'ya böyle bir güvenimiz vardir.
\par 5 Herhangi bir seyi kendi basarimiz olarak saymaya yeterliyiz demek istemiyorum; bizi yeterli kilan Tanri'dir.
\par 6 O bizi yazili yasaya degil, Ruh'a dayali yeni bir antlasmanin hizmetkârlari olmaya yeterli kildi. Yazili yasa öldürür, Ruh ise yasatir.
\par 7 Ölümle sonuçlanan hizmet, yani tas üzerine harf harf kazilan yasa yücelik içinde geldiyse öyle ki, Israilogullari geçici olan parlakligindan ötürü Musa'nin yüzüne bakamadilar- Ruh'a dayali hizmetin yücelik içinde olacagi daha kesin degil mi?
\par 9 Insani suçlu çikaran hizmetin yüceligi varsa, aklanmayi saglayan hizmetin yüceligi çok daha askindir.
\par 10 Çünkü eskiden yüceltilmis olanin, simdi yücelikte askin olana göre yüceligi yoktur.
\par 11 Geçici olan, yücelik içinde geldiyse, kalici olanin yüceligi çok daha büyüktür.
\par 12 Böyle bir umuda sahip oldugumuz için büyük cesaretle konusabiliriz.
\par 13 Yüzündeki parlakligin giderek söndügünü Israilogullari görmesin diye yüzünü peçeyle örten Musa gibi degiliz.
\par 14 Israilogullari'nin zihinleri körelmisti. Bugün bile Eski Antlasma okunurken zihinleri ayni peçeyle örtülü kaliyor. Çünkü bu peçe ancak Mesih araciligiyla kalkar.
\par 15 Ne var ki, bugün bile Musa'nin yazilari okundugunda yüreklerini bir peçe örtüyor.
\par 16 Oysa ne zaman biri Rab'be dönerse, o peçe kaldirilir.
\par 17 Rab Ruh'tur, Rab'bin Ruhu neredeyse orada özgürlük vardir.
\par 18 Ve biz hepimiz peçesiz yüzle Rab'bin yüceligini görerek yücelik üstüne yücelikle O'na benzer olmak üzere degistiriliyoruz. Bu da Ruh olan Rab sayesinde oluyor.

\chapter{4}

\par 1 Bu hizmeti Tanri'nin merhametiyle üstlendigimiz için cesaretimizi yitirmeyiz.
\par 2 Utanç verici gizli yollari reddettik. Hileye basvurmayiz, Tanri'nin sözünü de çarpitmayiz. Gerçegi ortaya koyarak kendimizi Tanri'nin önünde her insanin vicdanina tavsiye ederiz.
\par 3 Yaydigimiz Müjde örtülüyse de, mahvolanlar için örtülüdür.
\par 4 Tanri'nin görünümü olan Mesih'in yüceligiyle ilgili Müjde'nin isigi imansizlarin üzerine dogmasin diye, bu çagin ilahi onlarin zihinlerini kör etmistir.
\par 5 Biz kendimizi ilan etmiyoruz; ama Mesih Isa'yi Rab, kendimizi de Isa ugruna kullariniz ilan ediyoruz.
\par 6 Çünkü, "Isik karanliktan parlayacak" diyen Tanri, Isa Mesih'in yüzünde parlayan kendi yüceligini tanimamizdan dogan isigi bize vermek için yüreklerimizi aydinlatti.
\par 7 Üstün gücün bizden degil, Tanri'dan kaynaklandigi bilinsin diye bu hazineye toprak kaplar içinde sahibiz.
\par 8 Her yönden sikistirilmisiz, ama ezilmis degiliz. Sasirmisiz, ama çaresiz degiliz.
\par 9 Kovalaniyoruz, ama terk edilmis degiliz. Yere yikilmisiz, ama yok olmus degiliz.
\par 10 Isa'nin yasami bedenimizde açikça görülsün diye Isa'nin ölümünü her an bedenimizde tasiyoruz.
\par 11 Çünkü Isa'nin yasami ölümlü bedenimizde açikça görülsün diye, biz yasayanlar Isa ugruna sürekli olarak ölüme teslim ediliyoruz.
\par 12 Böylece ölüm bizde, yasamsa sizde etkin olmaktadir.
\par 13 "Iman ettim, bu nedenle konustum" diye yazilmistir. Ayni iman ruhuna sahip olarak biz de iman ediyor ve bu nedenle konusuyoruz.
\par 14 Çünkü Rab Isa'yi dirilten Tanri'nin, bizi de Isa'yla diriltip sizinle birlikte kendi önüne çikaracagini biliyoruz.
\par 15 Bütün bunlar sizin yararinizadir. Böylelikle Tanri'nin lütfu çogalip daha çok insana ulastikça, Tanri'nin yüceligi için sükran da artsin.
\par 16 Bu nedenle cesaretimizi yitirmeyiz. Her ne kadar dis varligimiz harap oluyorsa da, iç varligimiz günden güne yenileniyor.
\par 17 Çünkü geçici, hafif sikintilarimiz bize, agirlikta hiçbir seyle karsilastirilamayacak kadar büyük, sonsuz bir yücelik kazandirmaktadir.
\par 18 Gözlerimizi görünen seylere degil, görünmeyenlere çeviriyoruz. Çünkü görünenler geçicidir, görünmeyenlerse sonsuza dek kalicidir.

\chapter{5}

\par 1 Biliyoruz ki, barindigimiz bu dünyasal çadir yikilirsa, göklerde Tanri'nin bize sagladigi bir konut -elle yapilmamis, sonsuza dek kalacak bir evimiz- vardir.
\par 2 Simdiyse göksel evimizi giyinmeyi özleyerek inliyoruz.
\par 3 Onu giyinirsek çiplak kalmayiz.
\par 4 Dünyasal çadirda yasayan bizler agir bir yük altinda inliyoruz. Asil istedigimiz soyunmak degil, giyinmektir. Öyle ki, ölümlü olan, yasam tarafindan yutulsun.
\par 5 Bizleri tam bu amaç için hazirlamis ve güvence olarak bize Ruh'u vermis olan Tanri'dir.
\par 6 Bu nedenle her zaman cesaretimiz vardir. Sunu biliyoruz ki, bu bedende yasadikça Rab'den uzaktayiz.
\par 7 Gözle görülene degil, imana dayanarak yasariz.
\par 8 Cesaretimiz vardir diyorum ve bedenden uzakta, Rab'bin yaninda olmayi yegleriz.
\par 9 Bunun için, ister bedende yasayalim ister bedenden uzak olalim, amacimiz Rab'bi hosnut etmektir.
\par 10 Çünkü bedende yasarken gerek iyi gerek kötü, yaptiklarimizin karsiligini almak için hepimiz Mesih'in yargi kürsüsü önüne çikmak zorundayiz.
\par 11 Rab'den korkmanin ne demek oldugunu bildigimizden insanlari ikna etmeye çalisiyoruz. Ne oldugumuzu Tanri biliyor; umarim siz de vicdaninizda biliyorsunuz.
\par 12 Kendimizi yine size tavsiye etmeye çalismiyoruz. Ama yürekle degil, dis görünüsle övünenleri yanitlayabilmeniz için bizimle övünmenize firsat veriyoruz.
\par 13 Eger kendimizde degilsek, bu Tanri içindir. Aklimiz basimizdaysa, bu sizin içindir.
\par 14 Bizi zorlayan, Mesih'in sevgisidir. Yargimiz su: Biri herkes için öldü; öyleyse hepsi öldü.
\par 15 Evet, Mesih herkes için öldü. Öyle ki, yasayanlar artik kendileri için degil, kendileri ugruna ölüp dirilen Mesih için yasasinlar.
\par 16 Bu nedenle, biz artik kimseyi insan ölçülerine göre tanimayiz. Mesih'i bu ölçülere göre tanidiksa da, artik öyle tanimiyoruz.
\par 17 Bir kimse Mesih'teyse, yeni yaratiktir; eski seyler geçmis, her sey yeni olmustur.
\par 18 Bunlarin hepsi Tanri'dandir. Tanri, Mesih araciligiyla bizi kendisiyle baristirdi ve bize baristirma görevini verdi.
\par 19 Söyle ki Tanri, insanlarin suçlarini saymayarak dünyayi Mesih'te kendisiyle baristirdi ve baristirma sözünü bize emanet etti.
\par 20 Böylece, Tanri araciligimizla çagrida bulunuyormus gibi Mesih'in adina elçilik ediyor, O'nun adina yalvariyoruz: Tanri'yla barisin.
\par 21 Tanri, günahi bilmeyen Mesih'i bizim için günah sunusu yapti. Öyle ki, Mesih sayesinde Tanri'nin dogrulugu olalim.

\chapter{6}

\par 1 Tanri'yla birlikte çalisan bizler, O'nun lütfunu bos yere kabul etmemenizi ayrica rica ediyoruz.
\par 2 Çünkü Tanri diyor ki, "Uygun zamanda seni duydum, Kurtulus günü sana yardim ettim." Uygun zaman iste simdidir, kurtulus günü iste simdidir.
\par 3 Hizmetimizin kötülenmemesi için hiçbir konuda hiç kimsenin sürçmesine neden olmadik.
\par 4 Tersine Tanri'nin hizmetkârlari olarak olaganüstü dayanmada, sikinti, güçlük ve elemlerde, dayak, hapis, karisiklik, emek, uykusuzluk ve açlikta; pak yasayista, bilgi, sabir, iyilik, Kutsal Ruh ve içten sevgide; gerçegin ilaninda ve Tanri'nin gücünde; sag ve sol ellerimizde dogrulugun silahlariyla, yücelikte ve onursuzlukta, iyi ünde ve kötü ünde, kendimizi her durumda örnek gösteriyoruz. Aldatanlar sayiliyorsak da dürüst kisileriz.
\par 9 Taninmiyor gibiyiz, ama iyi taniniyoruz. Ölümün agzindayiz, ama iste yasiyoruz. Dövülüyorsak bile öldürülmüs degiliz.
\par 10 Kederliyiz ama her zaman seviniyoruz. Yoksuluz ama birçoklarini zengin ediyoruz. Hiçbir seyimiz yok ama her seye sahibiz.
\par 11 Ey Korintliler, sizinle açikça konustuk, size yüregimizi açtik.
\par 12 Sizden sevgimizi esirgemedik, ama siz bizden sevginizi esirgediniz.
\par 13 Bize ayni karsiligi verebilmek için çocuklarima söyler gibi söylüyorum siz de yüreginizi açin.
\par 14 Imansizlarla ayni boyunduruga girmeyin. Çünkü dogrulukla fesadin ne ortakligi, isikla karanligin ne paydasligi olabilir?
\par 15 Mesih'le Beliyal uyum içinde olabilir mi? Iman edenle iman etmeyenin ortak yani olabilir mi?
\par 16 Tanri'nin tapinagiyla putlar uyusabilir mi? Çünkü biz yasayan Tanri'nin tapinagiyiz. Nitekim Tanri söyle diyor: "Aralarinda yasayacak, Aralarinda yürüyecegim. Onlarin Tanrisi olacagim, Onlar da benim halkim olacak."
\par 17 Bu nedenle, "Imansizlarin arasindan çikip ayrilin" diyor Rab. "Murdara* dokunmayin, Ben de sizi kabul edecegim."
\par 18 Her Seye Gücü Yeten Rab diyor ki, "Size Baba olacagim, Siz de ogullarim, kizlarim olacaksiniz."

\chapter{7}

\par 1 Sevgili kardesler, bu vaatlere sahip oldugumuza göre, bedeni ve ruhu lekeleyen her seyden kendimizi arindiralim; Tanri korkusuyla kutsallikta yetkinleselim.
\par 2 Yüreklerinizde bize yer verin. Kimseye haksizlik etmedik, kimseyi yoldan saptirmadik, kimseyi sömürmedik.
\par 3 Bunu sizi yargilamak için söylemiyorum. Daha önce de söyledigim gibi, yüregimizde öyle bir yeriniz var ki, sizinle ölürüz de yasariz da.
\par 4 Size çok güveniyor, sizinle çok övünüyorum. Teselliyle doluyum. Bütün sikintilar arasinda sevincim sonsuzdur.
\par 5 Makedonya'ya geldigimizde de hiç rahat yüzü görmedik. Her bakimdan sikinti çekiyorduk. Disarida kavgalar, yüregimizde korkular vardi.
\par 6 Ama yüregi ezik olanlari teselli eden Tanri, Titus'un yanimiza gelisiyle -yalniz gelisiyle degil, sizden aldigi teselliyle de- bizi teselli etti. Titus beni özlediginizi, benim için üzülüp gayret ettiginizi bize anlatinca sevincim bir kat daha artti.
\par 8 Mektubumla size aci verdiysem bile pisman degilim. Aslinda pisman olmustum -kisa bir süre için de olsa, o mektubun size aci verdigini görüyorum- ama simdi seviniyorum; aci duymaniza degil, bu acinizin sizi tövbeye yöneltmesine seviniyorum. Tanri'nin istegine uygun olarak aci çektiniz. Böylece hiçbir sekilde bizden zarar görmediniz.
\par 10 Tanri'nin istegiyle çekilen aci, kisiyi kurtulusla sonuçlanan ve pismanlik dogurmayan tövbeye götürür. Dünyanin acilariysa ölüm getirir.
\par 11 Bakin bu acilar, Tanri'nin istegiyle çektiginiz bu acilar sizde ne büyük ciddiyet, paklanmak için ne büyük istek yaratti! Sizde ne büyük öfke, korku, özlem, gayret ve suçluyu cezalandirma arzusu uyandirdi! Bu konuda her bakimdan masum oldugunuzu kanitladiniz.
\par 12 Size o mektubu yazdimsa da, haksizlik edeni ya da haksizlik göreni düsünerek yazmadim; bize ne denli adanmis oldugunuzu Tanri önünde açikça görmenizi istiyordum.
\par 13 Bütün bunlarla teselli buluyoruz. Tesellimize ek olarak Titus'un sevinci bizi daha da çok sevindirdi. Çünkü hepiniz onun yüregini ferahlattiniz.
\par 14 Sizleri ona övdüm, beni utandirmadiniz. Size söyledigimiz her sey nasil dogru idiyse, sizi Titus'a övmemiz de öylece dogru çikti.
\par 15 Hepinizin nasil söz dinledigini, kendisini nasil saygi ve korkuyla kabul ettiginizi animsadikça size olan sevgisi daha da artiyor.
\par 16 Size her bakimdan güvenebildigim için seviniyorum.

\chapter{8}

\par 1 Kardesler, sizlere Tanri'nin Makedonya'daki kiliselerine* sagladigi lütuftan söz etmek istiyoruz: Büyük sikintilarla denendiklerinde, coskun sevinçleri ve asiri yoksulluklari tam bir cömertlige dönüstü.
\par 3 Ellerinden geldigi kadarini, hatta daha fazlasini kendi istekleriyle verdiklerine taniklik ederim.
\par 4 Kutsallara yapilan yardima katkida bulunma ayricaliginin kendilerine verilmesi için bize yalvarip yakardilar.
\par 5 Umdugumuzdan da öte, kendilerini önce Rab'be, sonra Tanri'nin istegiyle bize adadilar.
\par 6 Bu nedenle, aranizda daha önce basladigi bu hayirli isi tamamlamasi için Titus'u isteklendirdik.
\par 7 Imanda, söz söylemekte, bilgide, her tür gayrette, bize beslediginiz sevgide, her seyde üstün oldugunuz gibi, bu hayirli iste de üstün olmaya bakin.
\par 8 Bunu buyruk olarak söylemiyorum, yalnizca sevginizin içtenligini ötekilerin gayretiyle karsilastirarak sinamak istiyorum.
\par 9 Rabbimiz Isa Mesih'in lütfunu bilirsiniz. O'nun yoksulluguyla siz zengin olasiniz diye, zengin oldugu halde sizin ugrunuza yoksul oldu.
\par 10 Bu konuda size yararli olani salik veriyorum. Geçen yil bagis toplamaya ilk girisen, hatta buna ilk heveslenen siz oldunuz.
\par 11 Simdi bu isi tamamlayin; bunu candan arzuladiginiz gibi, elinizden geldigince tamamlamaya bakin.
\par 12 Çünkü istek varsa, insanin elinde olmayana göre degil, elindekine göre yardimda bulunmasi uygundur.
\par 13 Amacimiz sizi sikintiya sokup baskalarini rahatlatmak degildir. Ama esitlik olsun diye, simdi elinizdeki fazlalik onlarin eksigini tamamladigi gibi, baska zaman onlarin elindeki fazlalik sizin eksiginizi tamamlasin. Öyle ki, "Çok toplayanin fazlasi, az toplayanin da eksigi yoktu" diye yazilmis oldugu gibi, esitlik olsun.
\par 16 Titus'un yüreginde sizin için ayni ilgiyi uyandiran Tanri'ya sükürler olsun!
\par 17 Çünkü Titus yalniz ricamizi kabul etmekle kalmadi, size derin ilgi duydugu için kendi istegiyle yaniniza geliyor.
\par 18 Müjde'yi yayma çabalarindan ötürü bütün kiliselerce* övülen bir kardesi de onunla birlikte gönderiyoruz.
\par 19 Üstelik bu kardes, Rab'bi yüceltmek ve yardima hazir oldugumuzu göstermek için yürüttügümüz bu hayirli hizmette yol arkadasimiz olmak üzere kiliseler tarafindan seçildi.
\par 20 Bu büyük bagisla ilgili hizmetimizde kimsenin elestirisine hedef olmamaya özen gösteriyoruz.
\par 21 Çünkü yalniz Rab'bin gözünde degil, insanlarin gözünde de dogru olani yapmaya dikkat ediyoruz.
\par 22 Birçok konuda defalarca deneyip gayretli buldugumuz, simdi size duydugu büyük güvenle çok daha gayretli olan kardesimizi de bu iki kisiyle birlikte gönderiyoruz.
\par 23 Titus'a gelince, o benim paydasim ve aranizdaki emektasimdir. Öbür kardeslerimizse kiliselerin elçileri, Mesih'in kivancidirlar.
\par 24 Bunun için onlara sevginizi kanitlayin, kiliselerin önünde sizinle övünmemizin nedenini gösterin.

\chapter{9}

\par 1 Kutsallara yapilacak bu yardimla ilgili olarak size yazmama gerek yok.
\par 2 Çünkü yardima hazir oldugunuzu biliyorum. Ahaya'daki sizlerin geçen yildan beri hazirlikli oldugunu söyleyerek Makedonyalilar karsisinda sizinle övünmekteyim. Gayretiniz onlarin çogunu harekete geçirdi.
\par 3 Bu konuda sizinle övünmemiz bosa çikmasin; dedigim gibi, hazirlikli olasiniz diye kardesleri yaniniza gönderiyorum.
\par 4 Öyle ki, bazi Makedonyalilar benimle birlikte gelir ve sizi hazirliksiz bulurlarsa, sizler bir yana, bizler duydugumuz güvenden ötürü utanmayalim.
\par 5 Bu nedenle önce yaniniza gelmeleri ve cömertçe vermeyi vaat ettiginiz armaganlari hazirlamalari için kardeslere ricada bulunmayi gerekli gördüm. Öyle ki, armaganiniz cimrilik degil, cömertlik örnegi olarak hazir olsun.
\par 6 Sunu unutmayin: Az eken az biçer, çok eken çok biçer.
\par 7 Herkes yüreginde niyet ettigi gibi versin; isteksizce ya da zorlanmis gibi degil. Çünkü Tanri sevinçle vereni sever.
\par 8 Her zaman, her yönden, her seye yeterli ölçüde sahip olarak her iyi ise cömertçe katkida bulunabilmeniz için, Tanri her nimeti size bol bol saglayacak güçtedir.
\par 9 Nitekim söyle yazilmistir: "Armaganlar dagitti, yoksullara verdi; Dogrulugu sonsuza dek kalicidir."
\par 10 Ekinciye tohum ve yiyecek ekmek saglayan Tanri, sizin de ekeceginizi saglayip çogaltacak, dogrulugunuzun ürünlerini artiracaktir.
\par 11 Her durumda cömert olmaniz için her bakimdan zenginlestiriliyorsunuz. Cömertliginiz bizim araciligimizla Tanri'ya sükran nedeni oluyor.
\par 12 Yaptiginiz bu hizmet yalniz kutsallarin eksiklerini gidermekle kalmiyor, birçoklarinin Tanri'ya sükretmesiyle de zenginlesiyor.
\par 13 Onlar, içtenliginizi kanitlayan bu hizmetten ötürü, açikça benimsediginiz Mesih Müjdesi'ne uyarak kendileriyle ve herkesle malinizi cömertçe paylastiginiz için Tanri'yi yüceltiyorlar.
\par 14 Tanri'nin size bagisladigi olaganüstü lütuftan dolayi sizler için dua ediyor, sizi özlüyorlar.
\par 15 Sözle anlatilamayan armagani için Tanri'ya sükürler olsun!

\chapter{10}

\par 1 Sizinle birlikteyken ürkek, ama aranizda degilken yigit kesilen ben Pavlus, Mesih'teki alçakgönüllülük ve yumusaklikla size rica ediyor, yalvariyorum: Yaniniza geldigim zaman, bizi olagan insanlar gibi yasayanlardan sayan bazilarina karsi güvenle takinmak niyetinde oldugum tavri ayni cesaretle size karsi takinmaya zorlamayin beni.
\par 3 Olagan insanlar gibi yasiyorsak da, insansal güce dayanarak savasmiyoruz.
\par 4 Çünkü savasimizin silahlari insansal silahlar degil, kaleleri yikan tanrisal güce sahip silahlardir.
\par 5 Safsatalari, Tanri bilgisine karsi diklenen her engeli yikiyor, her düsünceyi tutsak edip Mesih'e bagimli kiliyoruz.
\par 6 Mesih'e tümüyle bagimli oldugunuz zaman, O'na bagimli olmayan her eylemi cezalandirmaya hazir olacagiz.
\par 7 Gözünüzün önündekine bakin. Bir kimse Mesih'e ait olduguna güveniyorsa, yine düsünsün: Kendisi kadar biz de Mesih'e aitiz.
\par 8 Sizi yikmak için degil, gelistirmek için Rab'bin bize verdigi yetkiyle biraz fazla övünsem de utanmam.
\par 9 Mektuplarimla sizi korkutmaya çalisiyormus gibi görünmek istemiyorum.
\par 10 Çünkü bazilari, "Mektuplari agir ve etkilidir, ama kisisel varligi etkisiz, konusma yetenegi de sifir" diyormus.
\par 11 Böyle diyenler sunu bilsin ki, uzaktayken mektuplarimizda ne diyorsak, aranizdayken de öyle davraniyoruz.
\par 12 Kendilerini tavsiye eden bazilariyla kendimizi bir tutmaya ya da karsilastirmaya elbette cesaret edemeyiz! Onlar kendilerini kendileriyle ölçüp karsilastirmakla akilsizlik ediyorlar.
\par 13 Ama biz haddimizi asip fazla övünmeyiz; övünmemiz, Tanri'nin bizim için belirledigi, sizlere kadar da uzanan alanin sinirlari içinde kalir.
\par 14 Etkinlik alanimiz size kadar uzanmasaydi, sizinle ilgilenmekle sinirlarimizin disina çikmis sayilabilirdik. Oysa Mesih'in Müjdesi'ni size kadar ilk ulastiran biz olduk.
\par 15 Baskalarinin emegiyle övünüp haddimizi asmayiz. Umudumuz odur ki, sizin imaniniz büyüdükçe sayenizde etkinlik alanimiz alabildigine genisleyecek.
\par 16 Böylelikle Müjde'yi sizlerden daha ötelere yayabilecegiz. Çünkü baskasinin etkinlik alaninda basarilmis islerle övünmek istemiyoruz.
\par 17 "Övünen, Rab'le övünsün."
\par 18 Kabule deger kisi kendi kendini tavsiye eden degil, Rab'bin tavsiye ettigi kisidir.

\chapter{11}

\par 1 Umarim yapacagim küçük bir akilsizligi hos görürsünüz. Ne olur, beni hos görün!
\par 2 Sizler için tanrisal bir kiskançlik duyuyorum. Çünkü sizleri el degmemis kiz gibi tek ere, Mesih'e sunmak üzere nisanladim.
\par 3 Ne var ki, yilanin Havva'yi kurnazligiyla aldatmasi gibi, düsüncelerinizin Mesih'e olan içten ve pak adanmisliktan saptirilmasindan korkuyorum.
\par 4 Çünkü size gelen ve bizim tanittigimizdan degisik bir Isa'yi tanitanlari pekâlâ hos görüyorsunuz. Ayrica, aldiginiz ruhtan farkli bir ruhu ve kabul ettiginizden farkli bir müjdeyi kabul ederek bunlari hos görüyorsunuz.
\par 5 Sözüm ona üstün elçilerden hiç de asagi oldugumu sanmiyorum!
\par 6 Acemi bir konusmaci olabilirim, ama bilgiden yana acemi degilim. Bunu size her durumda, her bakimdan açikça gösterdik.
\par 7 Yücelmeniz için kendimi alçaltarak Tanri'nin Müjdesi'ni size karsiliksiz bildirmekle günah mi isledim?
\par 8 Size hizmet etmek için yardim aldigim baska kiliseleri* adeta soydum.
\par 9 Aranizdayken ihtiyacim oldugu halde hiçbirinize yük olmadim. Çünkü Makedonya'dan gelen kardesler eksiklerimi tamamladilar. Size yük olmamaya hep özen gösterdim, bundan böyle de özen gösterecegim.
\par 10 Mesih'in gerçegine sahip olarak kesinlikle diyebilirim ki, Ahaya Ili'nde hiç kimse beni böyle övünmekten alikoyamaz.
\par 11 Neden mi? Sizi sevmedigimden mi? Tanri biliyor ki, sizi seviyorum.
\par 12 Övündükleri konuda bize esit sayilmak isteyen firsatçilara firsat vermemek için, yaptigimi yapmaya devam edecegim.
\par 13 Bu tür adamlar sahte elçiler, düzenbaz isçiler, kendilerine Mesih'in elçisi süsü verenlerdir.
\par 14 Buna sasmamali. Seytan da kendisine isik melegi süsü verir.
\par 15 Ona hizmet edenlerin de kendilerine dogrulugun hizmetkârlari süsü vermesi sasirtici degildir. Onlarin sonu yaptiklarina göre olacaktir.
\par 16 Yine söylüyorum, kimse beni akilsiz sanmasin. Öyle saniyorsaniz, akilsiz birini kabul eder gibi de olsa beni kabul edin ki, ben de biraz övüneyim!
\par 17 Söylediklerimi Rab'bin söyleyecegi gibi degil, akilsiz biri gibi, bu övüngen tavirla söylüyorum.
\par 18 Mademki birçoklari ne olduklariyla övünüyorlar, ben de övünecegim.
\par 19 Sizler akilli oldugunuz için akilsizlara seve seve katlaniyorsunuz!
\par 20 Aslinda sizi köle edenlere, sömürenlere, sizden yararlananlara, büyüklük taslayanlara ya da sizi tokatlayanlara katlaniyorsunuz.
\par 21 Utanarak kabul ediyorum ki, biz bunu yapacak güçte degildik! Ama birinin övünmeye cesaret ettigi konuda -akilsiz biri gibi konusuyorum- ben de övünmeye cesaret ediyorum.
\par 22 Onlar Ibrani mi? Ben de Ibrani'yim. Israilli mi? Ben de Israilli'yim. Ibrahim'in soyundan midirlar? Ben de onun soyundanim.
\par 23 Mesih'in hizmetkârlari midirlar? Aklimi kaçirmis gibi konusuyorum. Ben O'nun daha üstün bir hizmetkâriyim. Ben daha çok emek verdim, hapse daha çok girdim, sayisiz dayak yedim, çok kez ölümle burun buruna geldim.
\par 24 Bes kez Yahudiler'den otuz dokuzar kirbaç yedim.
\par 25 Üç kez degnekle dövüldüm, bir kez taslandim, üç kez deniz kazasina ugradim. Bir gün bir gece açik denizde kaldim.
\par 26 Sik sik yolculuk ettim. Irmaklarda, haydutlar arasinda, gerek soydaslarimin gerekse öteki uluslarin* arasinda tehlikelere ugradim. Kentte, kirda, denizde, sahte kardesler arasinda tehlikelere düstüm.
\par 27 Emek verdim, sikinti çektim, çok kez uykusuz kaldim. Açligi, susuzlugu tattim. Çok kez yiyecek sikintisi çektim, sogukta çiplak kaldim.
\par 28 Öbür sorunlarin yanisira, bütün kiliseler* için her gün çektigim kayginin baskisi var üzerimde.
\par 29 Kim güçsüz olur da ben güçsüz olmam? Kim günaha düsürülür de ben onun için yanmam?
\par 30 Övünmem gerekiyorsa, güçsüzlügümü gösteren seylerle övünecegim.
\par 31 Rab Isa'nin sonsuza dek övülecek olan Tanrisi ve Babasi biliyor ki, yalan söylemiyorum.
\par 32 Sam'da Kral Aretas'in valisi beni yakalatmak için kenti denetim altina almisti.
\par 33 Ama beni küfe içinde surdaki bir pencereden sarkittilar; böylece onun elinden siyrilip kaçtim.

\chapter{12}

\par 1 Yararli olmasa da övünmek gereklidir. Simdi görümlere ve Rab'bin vahiylerine geleyim.
\par 2 On dört yil önce alinip üçüncü göge götürülmüs bir Mesih izleyicisi taniyorum. Bu, bedensel olarak mi, yoksa beden disinda mi oldu, bilmiyorum, Tanri bilir.
\par 3 Evet, bu adamin cennete götürüldügünü biliyorum; bu, bedensel olarak mi, yoksa bedenden ayri mi oldu, bilmiyorum, Tanri bilir. Orada, dille anlatilamaz, insanin söylemesi yasak olan sözler isitti.
\par 5 Böyle biriyle övünecegim. Ama kendimle ilgili olarak, güçsüzlüklerimden baska bir seyle övünmeyecegim.
\par 6 Övünmek istesem bile akilsiz olmayacagim. Çünkü gerçegi söylemis olacagim. Ama kimse beni gördügünden ya da isittiginden daha üstün görmesin diye övünmekten kaçiniyorum.
\par 7 Aldigim vahiylerin üstünlügüyle gururlanmayayim diye bana bedende bir diken, beni yumruklamak için Seytan'in bir melegi verildi, gururlanmayayim diye.
\par 8 Bundan kurtulmak için Rab'be üç kez yalvardim.
\par 9 Ama O bana, "Lütfum sana yeter. Çünkü gücüm, güçsüzlükte tamamlanir" dedi. Iste, Mesih'in gücü içimde bulunsun diye güçsüzlüklerimle sevinerek daha çok övünecegim.
\par 10 Bu nedenle Mesih ugruna güçsüzlükleri, hakaretleri, zorluklari, zulümleri ve darliklari sevinçle karsiliyorum. Çünkü ne zaman güçsüzsem, o zaman güçlüyüm.
\par 11 Akilsiz biri gibi davrandim, ama beni buna siz zorladiniz. Aslinda beni siz tavsiye etmeliydiniz. Çünkü bir hiç isem de, sözüm ona üstün elçilerden hiç de asagi degilim.
\par 12 Elçiligimin kanitlari aranizda büyük bir sabirla, belirtiler, harikalar ve mucizelerle gösterildi.
\par 13 Size yük olmayisimdan baska öbür kiliselerden ne eksiginiz var ki? Bu haksizligimi bagislayin!
\par 14 Iste, üçüncü kez yaniniza gelmeye hazirim ve size yük olmayacagim. Çünkü sizde olani degil, sizi istiyorum. Çocuklarin anne babalari için degil, anne babalarin çocuklari için para biriktirmesi gerekir.
\par 15 Ben de canlariniz ugruna malimi da kendimi de seve seve harcayacagim. Sizi daha çok seversem, daha az mi sevilecegim?
\par 16 Öyle olsun, ben size yük olmadim. Ama kurnaz biri oldugumdan sizi hileyle elde etmisim!
\par 17 Size gönderdigim adamlardan biri araciligiyla sizi sömürdüm mü?
\par 18 Titus'u size gelmeye isteklendirdim ve öbür kardesi de onunla birlikte gönderdim. Titus sizi sömürmedi, degil mi? Ayni ruhla davranmadik mi, ayni yolu izlemedik mi?
\par 19 Bunca zamandir önünüzde kendimizi savundugumuzu mu düsünüyorsunuz? Tanri'nin önünde, Mesih'e ait kisiler olarak konusuyoruz. Sevgili kardesler, yaptigimiz her sey sizin gelismeniz içindir.
\par 20 Çünkü geldigimde sizi istedigim durumda bulamayacagimdan korkuyorum. Sizler de beni istediginiz durumda bulamayabilirsiniz. Aranizda çekisme, kiskançlik, öfke, bencil tutkular, iftira, dedikodu, böbürlenme, kargasa olmasindan korkuyorum.
\par 21 Korkarim size tekrar geldigimde Tanrim beni önünüzde utandiracak; daha önce günah isleyip de kapildiklari pisliklerden, fuhus ve sefahatten tövbe etmeyen birçoklari için yas tutacagim.

\chapter{13}

\par 1 Bu, yaniniza üçüncü gelisim olacak. Her suçlama iki ya da üç tanigin tanikligiyla dogrulanmalidir.
\par 2 Daha önce, aranizda ikinci kez bulundugumda, geçmiste günah islemis olanlarla onlarin disinda kalanlarin hepsine söylemistim, simdi sizden uzaktayken de yineliyorum: Tekrar yaniniza gelirsem, hiç kimseyi esirgemeyecegim!
\par 3 Mesih'in benim araciligimla konustuguna iliskin kanit istiyorsunuz. Mesih size karsi güçsüz degildir; O'nun gücü sizde etkindir.
\par 4 Güçsüzlük içinde çarmiha gerildigi halde, simdi Tanri'nin gücüyle yasiyor. Biz de O'nda güçsüz oldugumuz halde, Tanri'nin gücü sayesinde O'nunla birlikte sizin yarariniza yasayacagiz.
\par 5 Iman yolunda olup olmadiginizi anlamak için kendinizi sinayip yoklayin. Isa Mesih'in içinizde oldugunu bilmiyor musunuz? Yoksa sinavdan basarisiz çikarsiniz.
\par 6 Umarim bizim basarisizliga ugramadigimizi anlayacaksiniz.
\par 7 Kötü bir sey yapmamaniz için Tanri'ya dua ediyoruz. Dilegimiz, bizim sinavi geçmis görünmemiz degil, biz sinavda basarisiz görünsek bile sizin iyi olani yapmanizdir.
\par 8 Çünkü gerçege karsi degil, ancak gerçek ugruna bir sey yapabiliriz.
\par 9 Biz güçsüz, sizse güçlüyken seviniyoruz. Yetkin olmaniz için de dua ediyoruz.
\par 10 Rab'bin yikmak degil, gelistirmek için bana verdigi yetkiyi yaniniza geldigimde sert biçimde kullanmak zorunda kalmayayim diye, bunlari aranizda degilken yaziyorum.
\par 11 Son olarak hosça kalin, kardeslerim. Yasantinizi düzeltin, çagrima kulak verin, düsüncelerinizde birlik olun, esenlik içinde yasayin. Sevgi ve esenlik kaynagi olan Tanri sizinle birlikte olacaktir.
\par 12 Birbirinizi kutsal öpüsle selamlayin.
\par 13 Bütün kutsallar size selam eder.
\par 14 Rab Isa Mesih'in lütfu, Tanri'nin sevgisi ve Kutsal Ruh'un paydasligi hepinizle birlikte olsun.


\end{document}