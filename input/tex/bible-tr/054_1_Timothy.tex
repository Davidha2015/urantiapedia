\begin{document}

\title{1 Timoteyus}


\chapter{1}

\par 1 Kurtaricimiz Tanri'nin ve umudumuz Mesih Isa'nin buyruguyla Mesih Isa'nin elçisi atanan ben Pavlus'tan imanda öz oglum Timoteos'a selam! Baba Tanri'dan ve Rabbimiz Mesih Isa'dan sana lütuf, merhamet ve esenlik olsun.
\par 3 Makedonya'ya giderken sana rica ettigim gibi, Efes'te kal ve bazi kisilerin farkli ögretiler yaymamasini, masallarla ve sonu gelmeyen soyagaçlariyla ugrasmamasini ögütle. Bu seyler, imana dayanan tanrisal düzene hizmet etmekten çok, tartismalara yol açar.
\par 5 Bu buyrugun amaci, pak yürekten, temiz vicdandan, içten imandan dogan sevgiyi uyandirmaktir.
\par 6 Bazi kisiler bunlardan saparak bos konusmalara daldilar.
\par 7 Kutsal Yasa* ögretmeni olmak istiyorlar, ama ne söyledikleri sözleri ne de iddiali olduklari konulari anliyorlar.
\par 8 Yasa'yi özüne uygun biçimde kullanan için Yasa'nin iyi oldugunu biliyoruz.
\par 9 Çünkü biliyoruz ki, Yasa dogrular için degil, yasa tanimayanlarla asiler, tanrisizlarla günahkârlar, kutsalliktan yoksunlarla kutsala karsi saygisiz olanlar, anne ya da babasini öldürenler, katiller, fuhus yapanlar, oglancilar, köle tüccarlari, yalancilar, yalan yere ant içenler ve saglam ögretiye karsit olan baska ne varsa onlar için konmustur.
\par 11 Mübarek Tanri'nin bana emanet edilen yüce Müjdesi'ne göre bu böyledir. Tanri Merhametlidir
\par 12 Beni güçlendiren Rabbimiz Mesih Isa'ya sükrederim. Çünkü beni güvenilir sayarak hizmetine aldi.
\par 13 Bir zamanlar O'na küfreden, zalim ve küstah biri oldugum halde bana merhamet edildi. Çünkü ne yaptiysam bilgisizlikten ve imansizliktan yaptim.
\par 14 Ama Rabbimiz'in lütfu, imanla ve Mesih Isa'da olan sevgiyle birlikte bol bol üzerime döküldü.
\par 15 "Mesih Isa günahkârlari kurtarmak için dünyaya geldi" sözü, güvenilir ve her bakimdan kabule layik bir sözdür. Günahkârlarin en kötüsü benim.
\par 16 Ama Mesih Isa, kendisine iman edip sonsuz yasama kavusacak olanlara örnek olayim diye sinirsiz sabrini öncelikle bende sergilemek için bana merhamet etti.
\par 17 Onur ve yücelik sonsuzlara dek bütün çaglarin Krali, ölümsüz ve görünmez tek Tanri'nin olsun! Amin.
\par 18 Oglum Timoteos, senin hakkinda önceden söylenen peygamberlik sözleri uyarinca, bu buyrugu sana emanet ediyorum. Öyle ki, bu sözlere dayanarak iyi savasi sürdüresin.
\par 19 Imana ve temiz vicdana saril. Bazilari temiz vicdani bir yana iterek iman konusunda battilar.
\par 20 Himeneos ve Iskender bunlardandir. Küfür etmemeyi ögrensinler diye onlari Seytan'a teslim ettim.

\chapter{2}

\par 1 Her seyden önce sunu ögütlerim: Tanri yoluna tam bir baglilik ve agirbaslilik içinde sakin ve huzurlu bir yasam sürelim diye, krallarla bütün üst yöneticiler dahil, bütün insanlar için dilekler, dualar, yakarislar ve sükürler sunulsun.
\par 3 Böyle yapmak iyidir ve Kurtaricimiz Tanri'yi hosnut eder.
\par 4 O bütün insanlarin kurtulup gerçegin bilincine erismesini ister.
\par 5 Çünkü tek Tanri ve Tanri'yla insanlar arasinda tek araci vardir. O da insan olan ve kendisini herkes için fidye olarak sunmus bulunan Mesih Isa'dir. Uygun zamanda verilen taniklik budur.
\par 7 Ben bunun habercisi ve elçisi atandim -gerçegi söylüyorum, yalan söylemiyorum- uluslara imani ve gerçegi ögretmeye atandim.
\par 8 Buna göre, erkeklerin öfkelenip çekismeden, her yerde pak eller yükselterek dua etmelerini isterim.
\par 9 Kadinlarin da saç örgüleriyle, altinlarla, incilerle ya da pahali giysilerle degil, sade giyimle, edepli ve ölçülü tutumla, Tanri yolunda yürüdüklerini ileri süren kadinlara yarasir biçimde, iyi islerle süslenmelerini isterim.
\par 11 Kadin sükûnet ve tam bir uysallik içinde ögrensin.
\par 12 Kadinin ögretmesine, erkege egemen olmasina izin vermiyorum; sakin olsun.
\par 13 Çünkü önce Adem, sonra Havva yaratildi; aldatilan da Adem degildi, kadin aldatilip suç isledi.
\par 15 Ama dogum yapip kurtulacaktir; yeter ki, sagduyuyla iman, sevgi ve kutsallikta yasasin.

\chapter{3}

\par 1 Iste güvenilir söz: Bir kimse gözetmen olmayi gönülden istiyorsa, iyi bir görev arzu etmis olur.
\par 2 Ancak gözetmen ayiplanacak bir yani olmayan, tek karili, ölçülü, sagduyulu, saygin, konuksever, ögretmeye yetenekli biri olmali.
\par 3 Sarap düskünü, zorba olmamali; uysal, kavgadan ve para sevgisinden uzak olmali.
\par 4 Evini iyi yönetmeli, çocuklarina söz dinletmeli, her yönden saygili olmalarini saglamali.
\par 5 Kendi evini yönetmesini bilmeyen, Tanri'nin toplulugunu* nasil kayirabilir?
\par 6 Gözetmen yeni iman etmis biri olmamali. Yoksa gurura kapilip Iblis'in ugradigi yargiya ugrayabilir.
\par 7 Toplulugun disindakiler tarafindan da iyi bir insan olarak taninmalidir. Öyle ki, ayiplanacak duruma ve Iblis'in tuzagina düsmesin.
\par 8 Ayni sekilde kilise* görevlileri, özü sözü ayri, sarap tutkunu, haksiz kazanç pesinde kosan kisiler degil, agirbasli kisiler olmali.
\par 9 Temiz vicdanla imanin sirrina sarilmalidirlar.
\par 10 Bunlar da önce denensin; elestirilecek bir yönleri yoksa görev alsinlar.
\par 11 Ayni sekilde kadinlar agirbasli olmali; iftiraci degil, ama ölçülü ve her bakimdan güvenilir olmali.
\par 12 Görevliler tek karili, çocuklarini ve evlerini iyi yöneten kisiler olsun.
\par 13 Görevlerini iyi yapanlar, kendileri için iyi bir yer edinir, Mesih Isa'ya imanda büyük cesaret kazanirlar.
\par 14 Yakinda yanina gelmeyi umuyorum. Ama gecikirsem, gerçegin diregi ve dayanagi olan Tanri'nin ev halki arasinda, yani yasayan Tanri'nin toplulugunda* nasil davranmak gerektigini bilesin diye sana bunlari yaziyorum.
\par 16 Kuskusuz Tanri yolunun sirri büyüktür. O, bedende göründü, Ruh'ça dogrulandi, Meleklerce görüldü, Uluslara tanitildi, Dünyada O'na iman edildi, Yücelik içinde yukari alindi.

\chapter{4}

\par 1 Ruh açikça diyor ki, son zamanlarda bazilari yalancilarin ikiyüzlülügü nedeniyle aldatici ruhlara ve cinlerin ögretilerine kulak vererek imandan dönecek. Vicdanlari adeta kizgin bir demirle daglanmis bu yalancilar evlenmeyi yasaklayacak, iman edip gerçegi bilenlerin sükranla yemesi için Tanri'nin yarattigi yiyeceklerden çekinmek gerektigini buyuracaklar.
\par 4 Oysa Tanri'nin yarattigi her sey iyidir, hiçbir sey reddedilmemeli; yeter ki, sükranla kabul edilsin.
\par 5 Çünkü her sey Tanri'nin sözüyle ve duayla kutsal kilinir.
\par 6 Bunlari kardeslere ögütlersen, imanin ve izledigin iyi ögretinin sözleriyle beslenmis olarak Mesih Isa'nin iyi bir görevlisi olursun.
\par 7 Kutsalliktan yoksun kocakari masallarini reddet. Kendini Tanri yolunda egit.
\par 8 Bedeni egitmenin biraz yarari var; ama simdiki ve gelecek yasamin vaadini içeren Tanri yolunda yürümek her yönden yararlidir.
\par 9 Bu güvenilir ve her bakimdan kabule layik bir sözdür.
\par 10 Bunun için emek veriyor, mücadele ediyoruz. Çünkü umudumuzu bütün insanlarin, özellikle iman edenlerin Kurtaricisi olan diri Tanri'ya bagladik.
\par 11 Bunlari buyur ve ögret.
\par 12 Gençsin diye kimse seni küçümsemesin. Konusmada, davranista, sevgide, imanda, paklikta imanlilara örnek ol.
\par 13 Ben yanina gelinceye dek kendini topluluga Kutsal Yazilar'i okumaya, ögüt vermeye, ögretmeye ada.
\par 14 Peygamberlik sözüyle, ihtiyarlar* kurulunun ellerini senin üzerine koymasiyla sana verilen ve hâlâ sende olan ruhsal armagani ihmal etme.
\par 15 Bu konularin üzerinde dur, kendini bunlara ver ki, ilerledigini herkes görsün.
\par 16 Kendine ve ögretine dikkat et, bu yolda yürümeye devam et. Çünkü bunu yapmakla hem kendini hem seni dinleyenleri kurtaracaksin.

\chapter{5}

\par 1 Yasli adama çikisma, babanmis gibi yol göster. Genç erkeklere kardesinmis gibi, yasli kadinlara annenmis gibi, genç kadinlara tam bir yürek temizligiyle kizkardesinmis gibi yol göster.
\par 3 Gerçekten kimsesiz dul kadinlara saygi göster.
\par 4 Ama dul kadinin çocuklari ya da torunlari varsa, bunlar öncelikle kendi ev halkina yardim ederek Tanri yolunda yürümeyi ve büyüklerine iyilik borcunu ödemeyi ögrensinler. Çünkü bu Tanri'yi hosnut eder.
\par 5 Gerçekten kimsesiz, yalniz kalmis dul kadin umudunu Tanri'ya baglamistir; gece gündüz O'na dilekte bulunmaya ve dua etmeye devam eder.
\par 6 Kendini zevke veren dul kadinsa daha yasarken ölmüstür.
\par 7 Ayiplanacak duruma düsmemeleri için onlari bu konularda uyar.
\par 8 Kendi yakinlarina, özellikle de ev halkina bakmayan kisi imani inkâr etmis, imansizdan beter olmustur.
\par 9 Yaptigi iyiliklerle taninmis, tek erkekle evlenmis, en az altmis yasinda olan dul kadin, eger çocuk büyütmüs, konuk agirlamis, kutsallarin ayaklarini yikamis, sikintida olanlara yardim etmis, kendini her tür iyi ise adamissa, adi dullar listesine yazilsin.
\par 11 Daha genç dullari listeye alma. Çünkü bedensel arzulari Mesih'e bagliliklarina baskin çikinca evlenmek isterler.
\par 12 Böylece verdikleri ilk sözü çigneyerek hüküm giyerler.
\par 13 Ayni zamanda ev ev gezerek tembellige alisirlar. Yalniz tembellige alismakla kalmazlar, üzerlerine düsmeyen sözler söyleyerek baskalarinin isine karisan bosbogazlar olurlar.
\par 14 Bu nedenle, daha genç dullarin evlenmelerini, çocuk yapmalarini, evlerini yönetmelerini, düsmana hiçbir iftira firsati vermemelerini isterim.
\par 15 Kimisi zaten sapmis, Seytan'in ardina düsmüstür.
\par 16 Imanli bir kadinin dul yakinlari varsa onlara yardim etsin. Inanlilar toplulugu* yük altina girmesin ki, gerçekten kimsesiz olan dullara yardim edebilsin.
\par 17 Toplulugu iyi yöneten ihtiyarlar*, özellikle Tanri sözünü duyurup ögretmeye emek verenler iki kat saygiya layik görülsün.
\par 18 Çünkü Kutsal Yazi'da söyle deniyor: "Harman döven öküzün agzini baglama" ve "Isçi ücretini hak eder."
\par 19 Iki ya da üç tanik olmadikça, bir ihtiyara yöneltilen suçlamayi kabul etme.
\par 20 Günah isleyenleri herkesin önünde azarla ki, öbürleri de korksun.
\par 21 Bu söylediklerimi yan tutmadan, kimseyi kayirmadan yerine getirmen için seni Tanri'nin, Mesih Isa'nin ve seçilmis meleklerin önünde uyariyorum.
\par 22 Birinin üzerine ellerini koymakta aceleci davranma, baskalarinin günahlarina ortak olma. Kendini temiz tut.
\par 23 Artik yalniz su içmekten vazgeç; miden ve sik sik bas gösteren rahatsizliklarin için biraz da sarap iç.
\par 24 Bazi kisilerin günahlari bellidir, kendilerinden önce yargi kürsüsüne ulasir. Bazilarinin günahlariysa sonradan ortaya çikar.
\par 25 Bunun gibi, iyi isler de bellidir; belli olmayanlar bile gizli kalamaz.

\chapter{6}

\par 1 Kölelik boyundurugu altinda olanlarin hepsi kendi efendilerini tam bir saygiya layik görsünler ki, Tanri'nin adi ve ögretisi kötülenmesin.
\par 2 Efendileri iman etmis olanlarsa, nasil olsa kardesiz deyip efendilerine saygisizlik etmesinler. Tersine, daha iyi hizmet etsinler. Çünkü bu iyi hizmetten yararlananlar, sevdikleri imanlilardir. Bu ilkeleri ögret ve ögütle.
\par 3 Eger biri farkli ögretiler yayar, dogru sözleri, yani Rabbimiz Isa Mesih'in sözlerini ve Tanri yoluna dayanan ögretiyi onaylamazsa, kendini begenmis, bilgisiz bir kisidir. Böyle biri tartismalari ve kelime kavgalarini hastalik derecesinde sever. Bu seyler kiskançliga, çekismeye, iftiraya, kötü kuskulara, düsünceleri yozlasmis ve gerçegi yitirmis kisilerin durmadan sürtüsmesine yol açar. Onlar Tanri yolunu kazanç yolu saniyorlar.
\par 6 Oysa eldekiyle yetinerek Tanri yolunda yürümek büyük kazançtir.
\par 7 Çünkü dünyaya ne bir sey getirdik, ne de ondan bir sey götürebiliriz.
\par 8 Yiyecegimiz, giyecegimiz varsa bunlarla yetiniriz.
\par 9 Zengin olmak isteyenler ayartilip tuzaga düserler, insani çöküse ve yikima götüren birçok saçma ve zararli arzulara kapilirlar.
\par 10 Çünkü her türlü kötülügün bir kökü de para sevgisidir. Kimileri zengin olma hevesiyle imandan saptilar, kendi kendilerine çok aci çektirdiler.
\par 11 Ama sen, ey Tanri adami, bu seylerden kaç! Dogrulugun, Tanri yolunun, imanin, sevginin, sabrin, uysalligin ardindan kos.
\par 12 Iman ugrunda yüce mücadeleyi sürdür. Sonsuz yasama simsiki saril. Bunun için çagrildin ve birçok tanik önünde yüce inanci açikça benimsedin.
\par 13 Her seye yasam veren Tanri'nin ve Pontius Pilatus önünde yüce inanca taniklik etmis olan Mesih Isa'nin huzurunda sana buyuruyorum: Rabbimiz Isa Mesih'in gelisine dek Tanri buyrugunu lekesiz ve kusursuz olarak koru.
\par 15 Mübarek ve tek Hükümdar, krallarin Krali, rablerin Rabbi, ölümsüzlügün tek sahibi, yaklasilmaz isikta yasayan, hiçbir insanin görmedigi ve göremeyecegi Tanri, Mesih'i belirlenen zamanda ortaya çikaracaktir. Onur ve kudret sonsuza dek O'nun olsun! Amin.
\par 17 Simdiki çagda zengin olanlara gururlanmamalarini, gelip geçici zenginlige umut baglamamalarini buyur. Zevk almamiz için bize her seyi bol bol veren Tanri'ya umut baglasinlar.
\par 18 Iyilik yapmalarini, iyilikten yana zengin, eliaçik ve paylasmaya istekli olmalarini buyur.
\par 19 Böylelikle gerçek yasama kavusmak üzere gelecek için kendilerine saglam temel olacak bir hazine biriktirmis olurlar.
\par 20 Ey Timoteos, sana emanet edileni koru! Kutsalliktan yoksun, bos sözlerden, yalan yere "bilgi" denen düsüncelerin çeliskilerinden sakin.
\par 21 Kimileri bu sözde bilgiye sahip olduklarini ileri sürerek imandan saptilar. Tanri'nin lütfu sizlerle birlikte olsun.


\end{document}