\begin{document}

\title{2 Timoteyus}


\chapter{1}

\par 1 Mesih Isa'daki yasam vaadi uyarinca Tanri'nin istegiyle Mesih Isa'nin elçisi atanan ben Pavlus'tan sevgili oglum Timoteos'a selam! Baba Tanri'dan ve Rabbimiz Mesih Isa'dan sana lütuf, merhamet ve esenlik olsun.
\par 3 Durmadan, gece gündüz dualarimda seni anarak atalarim gibi temiz vicdanla kulluk ettigim Tanri'ya sükrediyorum.
\par 4 Gözyaslarini animsiyor, sevinçle dolmak için seni görmeyi özlemle bekliyorum.
\par 5 Sendeki içten imani animsiyorum. Önce büyükannen Lois'in ve annen Evniki'nin sahip oldugu imana simdi senin de sahip olduguna eminim.
\par 6 Bu nedenle, ellerimi senin üzerine koymamla Tanri'nin sana verdigi armagani alevlendirmen gerektigini hatirlatiyorum.
\par 7 Çünkü Tanri bize korkaklik ruhu degil, güç, sevgi ve özdenetim ruhu vermistir.
\par 8 Bunun için Rabbimiz'e taniklik etmekten de O'nun ugruna tutuklu bulunan benden de utanma. Tanri'nin gücüyle Müjde ugruna benimle birlikte sikintiya gögüs ger.
\par 9 Tanri bizi yaptiklarimiza göre degil, kendi amacina ve lütfuna göre kurtarip kutsal bir yasama çagirdi. Bu lütuf bize zamanin baslangicindan önce Mesih Isa'da bagislanmis, simdi de O'nun gelisiyle açiga çikarilmistir. Kurtaricimiz Mesih Isa ölümü etkisiz kilmis, yasami ve ölümsüzlügü Müjde araciligiyla isiga çikarmistir.
\par 11 Ben Müjde'nin habercisi, elçisi ve ögretmeni atandim.
\par 12 Bu acilari çekmemin nedeni de budur. Ama bundan utanmiyorum. Çünkü kime inandigimi biliyorum. O'nun bana emanet ettigini o güne dek koruyacak güçte olduguna eminim.
\par 13 Benden isitmis oldugun dogru sözleri örnek alarak imanla ve Mesih Isa'da olan sevgiyle bunlara bagli kal.
\par 14 Sana emanet edilen iyi ögretileri içimizde yasayan Kutsal Ruh araciligiyla koru.
\par 15 Biliyorsun, Asya Ili'ndekilerin* hepsi beni terk edip gittiler. Figelos'la Hermogenis de bunlardandir.
\par 16 Rab, Onisiforos'un ev halkina merhamet etsin. Çünkü o çok kez içimi ferahlatti ve zincire vurulmus olmamdan utanmadi.
\par 17 Tersine, Roma'ya geldiginde beni gayretle arayip buldu.
\par 18 O gün Rab'den merhamet bulmasini dilerim. Efes'te onun bana ne kadar hizmet ettigini sen de çok iyi bilirsin.

\chapter{2}

\par 1 Oglum, Mesih Isa'da olan lütufla güçlen.
\par 2 Birçok tanik önünde benden isittigin sözleri, baskalarina da ögretmeye yeterli olacak güvenilir kisilere emanet et.
\par 3 Mesih Isa'nin iyi bir askeri olarak benimle birlikte sikintiya gögüs ger.
\par 4 Askerlik yapan kisi günlük yasamla ilgili islere karismaz; kendisini askerlige çagirani hosnut etmeye çalisir.
\par 5 Bunun gibi, spor yarismasina katilan kisi de kurallar uyarinca yarismazsa zafer tacini giyemez.
\par 6 Emek veren çiftçi üründen ilk payi alan kisi olmalidir.
\par 7 Dediklerimi iyi düsün. Rab sana her konuda anlayis verecektir.
\par 8 Yaydigim Müjde'de açiklandigi gibi, Davut'un soyundan olup ölümden dirilmis olan Isa Mesih'i animsa.
\par 9 Bu Müjde ugruna bir suçlu gibi zincire vurulmaya kadar varan sikintilara katlaniyorum. Ama Tanri'nin sözü zincire vurulmus degildir.
\par 10 Bunun içindir ki, seçilmisler ugruna her seye dayaniyorum. Öyle ki, onlar da sonsuz yüceligin yanisira Mesih Isa'da olan kurtulusa kavussunlar.
\par 11 Su güvenilir bir sözdür: "O'nunla birlikte öldüysek, O'nunla birlikte yasayacagiz.
\par 12 Dayanirsak, O'nunla birlikte egemenlik sürecegiz. O'nu inkâr edersek, O da bizi inkâr edecek.
\par 13 Biz sadik kalmasak da, O sadik kalacak. Çünkü kendi özüne aykiri davranamaz."
\par 14 Bu konulari imanlilara animsat. Dinleyenleri felakete sürüklemekten baska yarari olmayan kelime kavgalari çikarmamalari için onlari Tanri'nin önünde uyar.
\par 15 Kendini Tanri'ya makbul, gerçegin bildirisini dogru kullanan, alni ak bir isçi olarak sunmaya gayret et.
\par 16 Bayagi, bos sözlerden sakin. Çünkü bunlara dalanlar tanrisizlikta daha da ileri gidecekler.
\par 17 Sözleri kangren gibi yayilacak. Himeneos'la Filitos bunlardandir.
\par 18 Dirilis olup bitti diyerek gerçek yoldan saptilar. Simdi de bazilarinin imanini altüst ediyorlar.
\par 19 Ne var ki, Tanri'nin attigi saglam temel, "Rab kendine ait olanlari bilir" ve "Rab'bin adini anan herkes kötülükten uzak dursun" sözleriyle mühürlenmis olarak duruyor.
\par 20 Büyük bir evde yalniz altin ve gümüs kaplar bulunmaz; tahta ve toprak kaplar da vardir. Kimi onurlu, kimi bayagi is için kullanilir.
\par 21 Bunun gibi, kisi de kendini bayagi islerden aritirsa, onurlu amaçlara uygun, kutsal kilinmis, efendisine yararli, her iyi ise hazir bir kap olur.
\par 22 Gençlik arzularindan kaç. Temiz yürekle Rab'be yakaranlarla birlikte dogrulugun, imanin, sevginin ve esenligin ardindan kos.
\par 23 Saçma, cahilce tartismalara girmeyi reddet. Bunlarin kavga dogurdugunu bilirsin.
\par 24 Rab'bin kulu kavgaci olmamali. Tersine, herkese sefkatle davranmali, ögretme yetenegi olmali, haksizliklara sabirla dayanmalidir.
\par 25 Kendisine karsi olanlari yumusak huyla yola getirmeli. Gerçegi anlamalari için Tanri belki onlara bir tövbe yolu açar.
\par 26 Böylelikle ayilabilir, istegini yerine getirmeleri için kendilerini tutsak eden Iblis'in tuzagindan kurtulabilirler.

\chapter{3}

\par 1 Sunu bil ki, son günlerde çetin anlar olacaktir.
\par 2 Insanlar kendilerini seven, para düskünü, övüngen, kibirli, küfürbaz, anne baba sözü dinlemez, nankör, kutsalliktan ve sevgiden yoksun, uzlasmaz, iftiraci, özünü denetleyemeyen, azgin, iyilik düsmani olacaklar.
\par 4 Hain, aceleci, kendini begenmis, Tanri'dan çok eglenceyi seven, Tanri yolundaymis gibi görünüp bu yolun gücünü inkâr edenler olacaklar. Böylelerinden uzak dur.
\par 6 Bunlarin arasinda evlerin içine sokulup günahla yüklü, çesitli arzularla sürüklenen, her zaman ögrenen, ama gerçegin bilgisine bir türlü erisemeyen zayif iradeli kadinlari adeta tutsak eden adamlar var.
\par 8 Yannis'le Yambris nasil Musa'ya karsi geldilerse, bunlar da gerçege karsi gelirler. Düsünceleri yozlasmis, iman konusunda reddedilmis insanlardir.
\par 9 Ama daha ileri gidemeyecekler. Çünkü Yannis'le Yambris örnegindeki gibi, bunlarin da akilsizligini herkes açikça görecektir.
\par 10 Sense benim ögretimi, davranisimi, amacimi, imanimi, sabrimi, sevgimi, dayanma gücümü, çektigim zulüm ve acilari, örnegin Antakya'da, Konya'da ve Listra'da basima gelenleri yakindan izledin. Ne zulümlere katlandim! Ama Rab beni hepsinden kurtardi.
\par 12 Mesih Isa'ya ait olup Tanri yoluna yarasir bir yasam sürmek isteyenlerin hepsi zulüm görecek.
\par 13 Ama kötüler ve sahtekârlar, aldatarak ve aldanarak gittikçe daha beter olacaklar.
\par 14 Sense ögrendigin ve güvendigin ilkelere bagli kal. Çünkü bunlari kimlerden ögrendigini biliyorsun. Mesih Isa'ya iman araciligiyla seni bilge kilip kurtulusa kavusturacak güçte olan Kutsal Yazilar'i da çocuklugundan beri biliyorsun.
\par 16 Kutsal Yazilar'in tümü Tanri esinlemesidir ve ögretmek, azarlamak, yola getirmek, dogruluk konusunda egitmek için yararlidir.
\par 17 Bunlar sayesinde Tanri adami her iyi is için donatilmis olarak yetkin olur.

\chapter{4}

\par 1 Tanri'nin ve dirilerle ölüleri yargilayacak olan Mesih Isa'nin önünde, O'nun gelisi ve egemenligi hakki için sana buyuruyorum: Tanri sözünü duyur. Zaman uygun olsun olmasin, bu görevi sürdür. Insanlari tam bir sabirla egiterek ikna et, uyar, isteklendir.
\par 3 Çünkü öyle bir zaman gelecek ki, saglam ögretiye katlanamayacaklar. Kulaklarini oksayan sözler duymak için çevrelerine kendi arzularina uygun ögretmenler toplayacaklar.
\par 4 Kulaklarini gerçege tikayip masallara sapacaklar.
\par 5 Ama sen her durumda ayik ol, sikintiya gögüs ger, müjdeci olarak isini yap, görevini tamamla.
\par 6 Çünkü kanim adak sarabi gibi dökülmek üzere. Benim için ayrilma zamani geldi.
\par 7 Yüce mücadeleyi sürdürdüm, yarisi bitirdim, imani korudum.
\par 8 Bundan böyle dogruluk taci benim için hazir duruyor. Adil yargiç olan Rab o gün bu taci bana, yalniz bana degil, O'nun gelisini özlemle beklemis olanlarin hepsine verecektir.
\par 9 Yanima tez gelmeye gayret et.
\par 10 Çünkü Dimas bu dünyayi sevdigi için beni terk edip Selanik'e gitti. Kriskis Galatya'ya, Titus Dalmaçya'ya gitti.
\par 11 Yanimda yalniz Luka var. Markos'u alip beraberinde getir, yapacagim hizmette bana yardim eder.
\par 12 Tihikos'u Efes'e gönderdim.
\par 13 Troas'ta Karp'in yaninda biraktigim abayi, kitaplari, özellikle yazi derilerini gelirken beraberinde getir.
\par 14 Bakirci Iskender bana çok kötülük etti. Rab ona yaptiklarinin karsiligini verecektir.
\par 15 Sen de ondan sakin. Çünkü söylediklerimize siddetle karsi koydu.
\par 16 Ilk savunmamda benden yana çikan olmadi, hepsi beni terk etti. Bunun hesabi onlardan sorulmasin.
\par 17 Ama Tanri bildirisi araciligimla tam olarak açiklansin, bütün uluslar bunu duysun diye Rab yardimima gelip beni güçlendirdi. Aslanin agzindan böyle kurtuldum!
\par 18 Rab beni her kötülükten kurtarip güvenlik içinde göksel egemenligine ulastiracak. Sonsuzlara dek O'na yücelik olsun! Amin.
\par 19 Priska, Akvila ve Onisiforos'un ev halkina selam söyle.
\par 20 Erastus, Korint'te kaldi. Trofimos'u da Milet'te hasta biraktim.
\par 21 Kis bastirmadan gelmeye gayret et. Evvulus, Pudens, Linus, Klavdiya ve bütün kardesler sana selam ederler.
\par 22 Rab ruhunla birlikte olsun. Tanri'nin lütfu sizlerle olsun.


\end{document}