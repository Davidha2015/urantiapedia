\begin{document}

\title{Titus}


\chapter{1}

\par 1 Tanri'nin seçtigi kisilerin iman etmeleri, Tanri yoluna uygun gerçegi anlamalari için Tanri'nin kulu ve Isa Mesih'in elçisi atanan ben Pavlus'tan selam!
\par 2 Elçiligim, yalan söylemeyen Tanri'nin zamanin baslangicindan önce vaat ettigi sonsuz yasam umuduna dayanmaktadir.
\par 3 Kurtaricimiz Tanri'nin buyruguyla bana emanet edilen bildiride Tanri, kendi sözünü uygun zamanda açiklamistir.
\par 4 Ortak imanimiza göre öz oglum olan Titus'a Baba Tanri'dan ve Kurtaricimiz Mesih Isa'dan lütuf ve esenlik olsun.
\par 5 Geri kalan isleri düzene sokman ve sana buyurdugum gibi her kentte ihtiyarlar* ataman için seni Girit'te biraktim.
\par 6 Ihtiyar seçilecek kisi elestirilecek yönü olmayan, tek karili biri olsun. Çocuklari imanli olmali, sefahatle suçlanan ya da asi çocuklar olmamali.
\par 7 Gözetmen, Tanri evinin kâhyasi olduguna göre, elestirilecek yönü olmamali. Dikbasli, tez öfkelenen, sarap düskünü, zorba, haksiz kazanç pesinde kosan biri olmamali.
\par 8 Tersine, konuksever, iyiliksever, sagduyulu, adil, pak, kendini denetleyebilen biri olmali.
\par 9 Hem baskalarini saglam ögretiyle yüreklendirmek, hem de karsi çikanlari ikna edebilmek için imanlilara ögretilen güvenilir söze simsiki sarilmali.
\par 10 Çünkü asi, bosbogaz, aldatici birçok kisi vardir. Özellikle sünnet yanlilari bunlardandir.
\par 11 Onlarin agzini kapamak gerek. Haksiz kazanç ugruna, ögretmemeleri gerekeni ögreterek bazi aileleri tümüyle yikiyorlar.
\par 12 Kendilerinden biri, öz peygamberlerinden biri söyle demistir: "Giritliler hep yalancidir, azgin canavarlar, tembel oburlardir."
\par 13 Bu taniklik dogrudur. Bu nedenle, Yahudi masallarina, gerçegi reddedenlerin buyruklarina kulak vermeyip saglam imana sahip olmalari için onlari sert bir sekilde uyar.
\par 15 Yüregi temiz olanlar için her sey temizdir, ama yüregi kirli olanlar ve imansizlar için hiçbir sey temiz degildir. Çünkü onlarin zihinleri de vicdanlari da kirlenmistir.
\par 16 Tanri'yi tanidiklarini ileri sürer, ama yaptiklariyla O'nu yadsirlar. Söz dinlemez, hiçbir iyi ise yaramaz igrenç kisilerdir.

\chapter{2}

\par 1 Sana gelince, saglam ögretiye uygun olani ögret.
\par 2 Yasli erkeklere ölçülü, agirbasli, sagduyulu olmalarini buyur. Imanda, sevgide ve sabirda saglam olsunlar.
\par 3 Ayni sekilde yasli kadinlar saygin bir yasam sürmeli. Iftiraci, saraba tutsak olmamali; iyi olani ögretmeli.
\par 4 Öyle ki genç kadinlari, kocalarini ve çocuklarini seven, sagduyulu, temiz yürekli, iyi birer ev kadini ve kocalarina bagimli olmak üzere egitebilsinler. O zaman Tanri'nin sözü kötülenmez.
\par 6 Genç erkekleri de sagduyulu olmaya özendir.
\par 7 Iyi olani yaparak her konuda onlara örnek ol. Ögretisinde dürüst ve agirbasli ol, kimsenin kinayamayacagi dogru sözler söyle. Öyle ki bize karsi gelen, hakkimizda söyleyecek kötü bir söz bulamayip utansin.
\par 9 Köleleri, her konuda efendilerine bagimli olmaya özendir. Efendilerini hosnut etsinler. Ters yanit vermeden,
\par 10 hirsizlik yapmadan, tümüyle güvenilir olduklarini göstersinler. Böylece Kurtaricimiz Tanri'yla ilgili ögretiyi her yönden çekici kilsinlar.
\par 11 Çünkü Tanri'nin bütün insanlara kurtulus saglayan lütfu ortaya çikmistir.
\par 12 Bu lütuf, tanrisizligi ve dünyasal arzulari reddedip simdiki çagda sagduyulu, dogru, Tanri yoluna yarasir bir yasam sürebilmemiz için bizi egitiyor.
\par 13 Bu arada, mübarek umudumuzun gerçeklesmesini, ulu Tanri ve Kurtaricimiz Isa Mesih'in yücelik içinde gelmesini bekliyoruz.
\par 14 Mesih bizi her suçtan kurtarmak, aritip kendisine ait, iyilik etmekte gayretli bir halk yapmak üzere kendini bizim için feda etti.
\par 15 Bunlari tam bir yetkiyle bildir, dinleyenleri isteklendir, günahli olanlari ikna et. Hiç kimse seni küçümsemesin.

\chapter{3}

\par 1 Yöneticilerle yönetimlere bagli olmalari, söz dinlemeleri ve iyi olan her seyi yapmaya hazir olmalari gerektigini imanlilara animsat.
\par 2 Kimseyi kötülemesinler. Kavgaci degil, uysal olsunlar. Herkese her zaman yumusak davransinlar.
\par 3 Çünkü bir zamanlar biz de anlayissiz, söz dinlemez, kolay aldanan, türlü arzulara ve zevklere köle olan, kötülük ve kiskançlik içinde yasayan, nefret edilen ve birbirimizden nefret eden kisilerdik.
\par 4 Ama Kurtaricimiz Tanri iyiligini ve insana olan sevgisini açikça göstererek bizi kurtardi. Bunu dogrulukla yaptigimiz islerden dolayi degil, kendi merhametiyle, yeniden dogus yikamasiyla ve Kurtaricimiz Isa Mesih araciligiyla üzerimize bol bol döktügü Kutsal Ruh'un yenilemesiyle yapti.
\par 7 Öyle ki, O'nun lütfuyla aklanmis olarak umut içinde sonsuz yasamin mirasçilari olalim.
\par 8 Bu güvenilir bir sözdür. Tanri'ya iman etmis olanlarin, kendilerini iyi islere vermeye özen göstermeleri için bu konularda israrli olmani istiyorum. Bunlar insan için iyi ve yararlidir.
\par 9 Akilsiz tartismalardan, soyagaci didismelerinden, Kutsal Yasa'yla* ilgili çekisme ve kavgalardan sakin. Bunlar yararsiz ve bos seylerdir.
\par 10 Birinci ve ikinci uyaridan sonra bölücü kisiyle iliskini kes.
\par 11 Böyle birinin sapmis oldugundan ve günah islediginden emin olabilirsin; o kendi kendini mahkûm etmistir.
\par 12 Ben Artemas'i ya da Tihikos'u sana gönderir göndermez, Nikopolis'e, yanima gelmeye gayret et. Çünkü kisi orada geçirmeye karar verdim.
\par 13 Hukukçu Zenas'la Apollos'u yolcu ederken bir eksikleri olmamasina dikkat et.
\par 14 Bizimkiler de kendilerini iyi islere vermeyi ögrensinler. Böylelikle temel ihtiyaçlari karsilamis ve verimsiz bir yasam sürmemis olurlar.
\par 15 Yanimdakilerin hepsi sana selam eder. Bizi seven imanlilara selam söyle. Tanri'nin lütfu hepinizle birlikte olsun.


\end{document}