\begin{document}

\title{Yakup}


\chapter{1}

\par 1 Tanri'nin ve Rab Isa Mesih'in kulu ben Yakup, dagilmis olan on iki oymaga selam ederim.
\par 2 Kardeslerim, çesitli denemelerle yüz yüze geldiginizde bunu büyük sevinçle karsilayin.
\par 3 Çünkü bilirsiniz ki, imaninizin sinanmasi dayanma gücünü yaratir.
\par 4 Dayanma gücü de, hiçbir eksigi olmayan, olgun, yetkin kisiler olmaniz için tam bir etkinlige erissin.
\par 5 Içinizden birinin bilgelikte eksigi varsa, herkese cömertçe, azarlamadan veren Tanri'dan istesin; kendisine verilecektir.
\par 6 Yalniz hiç kusku duymadan, imanla istesin. Çünkü kusku duyan kisi rüzgarin sürükleyip savurdugu deniz dalgasina benzer.
\par 7 Her bakimdan degisken, kararsiz olan kisi Rab'den bir sey alacagini ummasin.
\par 9 Düskün olan kardes kendi yüksekligiyle, zengin olansa kendi düskünlügüyle övünsün. Çünkü zengin kisi kir çiçegi gibi solup gidecek.
\par 11 Günes yakici sicagiyla dogar ve otu kurutur. Otun çiçegi düser, görünüsünün güzelligi yok olur. Zengin de bunun gibi kendi ugraslari içinde kaybolup gidecektir.
\par 12 Ne mutlu denemeye dayanan kisiye! Denemeden basariyla çiktigi zaman Rab'bin kendisini sevenlere vaat ettigi yasam tacini alacaktir.
\par 13 Ayartilan kisi, "Tanri beni ayartiyor" demesin. Çünkü Tanri kötülükle ayartilmadigi gibi kendisi de kimseyi ayartmaz.
\par 14 Herkes kendi arzulariyla sürüklenip aldanarak ayartilir.
\par 15 Sonra arzu gebe kalir ve günah dogurur. Günah olgunlasinca da ölüm getirir.
\par 16 Sevgili kardeslerim, aldanmayin!
\par 17 Her nimet, her mükemmel armagan yukaridan, kendisinde degiskenlik ya da döneklik gölgesi olmayan Isiklar Babasi'ndan gelir.
\par 18 O, yarattiklarinin bir anlamda ilk meyveleri olmamiz için bizleri kendi istegi uyarinca, gerçegin bildirisiyle yasama kavusturdu.
\par 19 Sevgili kardeslerim, sunu aklinizda tutun: Herkes dinlemekte çabuk, konusmakta yavas, öfkelenmekte de yavas olsun.
\par 20 Çünkü insanin öfkesi Tanri'nin istedigi dogrulugu saglamaz.
\par 21 Bunun için, her türlü pisligi ve her tarafa yayilmis olan kötülügü üstünüzden siyirip atarak, içinize ekilmis, canlarinizi kurtaracak güçte olan sözü alçakgönüllülükle kabul edin.
\par 22 Tanri sözünü yalniz duymakla kalmayin, sözün uygulayicilari da olun. Yoksa kendinizi aldatmis olursunuz.
\par 23 Çünkü sözün dinleyicisi olup da uygulayicisi olmayan kisi, aynada kendi dogal yüzüne bakan kisiye benzer.
\par 24 Kendini görür, sonra gider ve nasil bir kisi oldugunu hemen unutur.
\par 25 Oysa mükemmel yasaya, özgürlük yasasina yakindan bakip ona bagli kalan, unutkan dinleyici degil de etkin uygulayici olan kisi, yaptiklariyla mutlu olacaktir.
\par 26 Dindar oldugunu sanip da dilini dizginlemeyen kisi kendini aldatir. Böylesinin dindarligi bostur.
\par 27 Baba Tanri'nin gözünde temiz ve kusursuz dindarlik, kisinin sikinti çeken öksüzler ve dullarla ilgilenmesi ve kendini dünyanin lekelemesinden korumasidir.

\chapter{2}

\par 1 Kardeslerim, yüce Rabbimiz Isa Mesih'e iman edenler olarak insanlar arasinda ayrim yapmayin.
\par 2 Toplandiginiz yere altin yüzüklü, sik giyimli bir adamla kirli giysiler içinde yoksul bir adam geldiginde, sik giyimliye ilgiyle, "Sen suraya, iyi yere otur", yoksula da, "Sen orada dur" ya da "Ayaklarimin dibine otur" derseniz, aranizda ayrim yapmis, kötü düsünceli yargiçlar gibi davranmis olmuyor musunuz?
\par 5 Dinleyin, sevgili kardeslerim: Tanri, bu dünyada yoksul olanlari imanda zenginlesmek ve kendisini sevenlere vaat ettigi egemenligin mirasçilari olmak üzere seçmedi mi?
\par 6 Ama siz yoksulun onurunu kirdiniz. Sizi sömüren zenginler degil mi? Sizi mahkemelere sürükleyen onlar degil mi?
\par 7 Ait oldugunuz Kisi'nin yüce adina küfreden onlar degil mi?
\par 8 "Komsunu kendin gibi seveceksin" diyen Kutsal Yazi'ya uyarak Kralimiz Tanri'nin Yasasi'ni gerçekten yerine getiriyorsaniz, iyi ediyorsunuz.
\par 9 Ama insanlar arasinda ayrim yaparsaniz, günah islemis olursunuz; Yasa tarafindan, Yasa'yi çignemekten suçlu bulunursunuz.
\par 10 Çünkü Yasa'nin her dedigini yerine getirse de tek konuda ondan sapan kisi bütün Yasa'ya karsi suçlu olur.
\par 11 Nitekim "Zina etmeyeceksin" diyen, ayni zamanda "Adam öldürmeyeceksin" demistir. Zina etmez, ama adam öldürürsen, Yasa'yi yine de çignemis olursun.
\par 12 Özgürlük Yasasi'yla yargilanacak olanlar gibi konusup davranin.
\par 13 Çünkü yargi merhamet göstermeyene karsi merhametsizdir. Merhamet yargiya galip gelir.
\par 14 Kardeslerim, bir kimse iyi eylemleri yokken imani oldugunu söylerse, bu neye yarar? Böylesi bir iman onu kurtarabilir mi?
\par 15 Bir erkek ya da kiz kardes çiplak ve günlük yiyecekten yoksunken, içinizden biri ona, "Esenlikle git, isinmani, doymani dilerim" der, ama bedenin gereksindiklerini vermezse, bu neye yarar?
\par 17 Bunun gibi, tek basina eylemsiz iman da ölüdür.
\par 18 Ama biri söyle diyebilir: "Senin imanin var, benimse eylemlerim." Eylemlerin olmadan sen bana imanini göster, ben de sana imanimi eylemlerimle göstereyim.
\par 19 Sen Tanri'nin bir olduguna inaniyorsun, iyi ediyorsun. Cinler bile buna inaniyor ve titriyorlar!
\par 20 Ey akilsiz adam, eylem olmadan imanin yararsiz olduguna kanit mi istiyorsun?
\par 21 Atamiz Ibrahim, oglu Ishak'i sunagin üzerinde Tanri'ya adama eylemiyle aklanmadi mi?
\par 22 Görüyorsun, onun imani eylemleriyle birlikte etkindi; imani eylemleriyle tamamlandi.
\par 23 Böylelikle, "Ibrahim Tanri'ya iman etti, böylece aklanmis sayildi" diyen Kutsal Yazi yerine gelmis oldu. Ibrahim'e de Tanri'nin dostu dendi.
\par 24 Görüyorsunuz, insan yalniz imanla degil, eylemle de aklanir.
\par 25 Ayni biçimde, ulaklari konuk edip degisik bir yoldan geri gönderen fahise Rahav da bu eylemiyle aklanmadi mi?
\par 26 Ruhsuz beden nasil ölüyse, eylemsiz iman da ölüdür.

\chapter{3}

\par 1 Kardeslerim, biz ögretmenlerin daha titiz bir yargilamadan geçecegini biliyorsunuz; bu nedenle çogunuz ögretmen olmayin.
\par 2 Çünkü hepimiz çok hata yapariz. Sözleriyle hata yapmayan kimse, bütün bedenini de dizginleyebilen yetkin bir kisidir.
\par 3 Bize boyun egmeleri için atlarin agzina gem vururuz, böylece bütün bedenlerini yönlendiririz.
\par 4 Düsünün, gemiler de o kadar büyük oldugu, güçlü rüzgarlar tarafindan sürüklendigi halde, dümencinin gönlü nereye isterse küçücük bir dümenle o yöne çevrilirler.
\par 5 Bunun gibi, dil de bedenin küçük bir üyesidir, ama büyük islerle övünür. Düsünün, küçücük bir kivilcim koca bir ormani tutusturabilir.
\par 6 Dil de bir ates, bedenimizin üyeleri arasinda bir kötülük dünyasidir. Bütün varligimizi kirletir. Cehennemden alevlenmis olarak yasamimizin gidisini alevlendirir.
\par 7 Insan soyu, her tür yabanil hayvani, kusu, sürüngeni ve deniz yaratigini evcillestirmis ve evcillestirmektedir.
\par 8 Ama dili hiçbir insan evcillestiremez. Dil öldürücü zehirle dolu, dinmeyen bir kötülüktür.
\par 9 Dilimizle Rab'bi, Baba'yi överiz. Yine dilimizle Tanri'ya benzer yaratilmis insana söveriz.
\par 10 Övgü ve sövgü ayni agizdan çikar. Kardeslerim, bu böyle olmamali.
\par 11 Bir pinar ayni gözden tatli ve aci su akitir mi?
\par 12 Kardeslerim, incir agaci zeytin ya da asma incir verebilir mi? Bunun gibi, tuzlu su kaynagi tatli su veremez.
\par 13 Aranizda bilge ve anlayisli olan kim? Olumlu yasayisiyla, bilgelikten dogan alçakgönüllülükle iyi eylemlerini göstersin.
\par 14 Ama yüreginizde kin, kiskançlik, bencillik varsa övünmeyin, gerçegi yadsimayin.
\par 15 Böylesi "bilgelik" gökten inen degil, dünyadan, insan dogasindan, cinlerden gelen bilgeliktir.
\par 16 Çünkü nerede kiskançlik, bencillik varsa, orada karisiklik ve her tür kötülük vardir.
\par 17 Ama gökten inen bilgelik her seyden önce paktir, sonra barisçildir, yumusaktir, uysaldir. Merhamet ve iyi meyvelerle doludur. Kayiriciligi, ikiyüzlülügü yoktur.
\par 18 Baris içinde eken baris yapicilari dogruluk ürününü biçerler.

\chapter{4}

\par 1 Aranizdaki kavgalarin, çekismelerin kaynagi nedir? Bedeninizin üyelerinde savasan tutkulariniz degil mi?
\par 2 Bir sey arzu ediyor, elde edemeyince adam öldürüyorsunuz. Kiskaniyorsunuz, isteginize erisemeyince çekisip kavga ediyorsunuz. Elde edemiyorsunuz, çünkü Tanri'dan dilemiyorsunuz.
\par 3 Dilediginiz zaman da dileginize kavusamiyorsunuz. Çünkü kötü amaçla, tutkulariniz ugruna kullanmak için diliyorsunuz.
\par 4 Ey vefasizlar, dünyayla dostlugun Tanri'ya düsmanlik oldugunu bilmiyor musunuz? Dünyayla dost olmak isteyen, kendini Tanri'ya düsman eder.
\par 5 Sizce Kutsal Yazi bos yere mi söyle diyor: "Tanri içimize koydugu ruhu kiskançlik derecesinde özler."
\par 6 Yine de bize daha çok lütfeder. Bu nedenle Yazi söyle diyor: "Tanri kibirlilere karsidir, Ama alçakgönüllülere lütfeder."
\par 7 Bunun için Tanri'ya bagimli olun. Iblis'e karsi direnin, sizden kaçacaktir.
\par 8 Tanri'ya yaklasin, O da size yaklasacaktir. Ey günahkârlar, ellerinizi günahtan temizleyin. Ey kararsizlar, yüreklerinizi paklayin.
\par 9 Kederlenin, yas tutup aglayin. Gülüsünüz yasa, sevinciniz üzüntüye dönüssün.
\par 10 Rab'bin önünde kendinizi alçaltin, sizi yüceltecektir.
\par 11 Kardeslerim, birbirinizi yermeyin. Kardesini yeren ya da yargilayan kisi,Yasa'yi yermis ve yargilamis olur. Yasa'yi yargilarsan, Yasa'nin uygulayicisi degil, yargilayicisi olursun.
\par 12 Oysa tek Yasa koyucu, tek Yargiç vardir; kurtarmaya da mahvetmeye de gücü yeten O'dur. Ya komsusunu yargilayan sen, kim oluyorsun?
\par 13 Dinleyin simdi, "Bugün ya da yarin filan kente gidecegiz, orada bir yil kalip ticaret yapacak, para kazanacagiz" diyen sizler, yarin ne olacagini bilmiyorsunuz. Yasaminiz nedir ki? Kisa süre görünen, sonra yitip giden bugu gibisiniz.
\par 15 Bunun yerine, "Rab dilerse yasayacak, sunu sunu yapacagiz" demelisiniz.
\par 16 Ne var ki, simdi küstahliklarinizla övünüyorsunuz. Bu tür övünmelerin hepsi kötüdür.
\par 17 Bu nedenle, yapilmasi gereken iyi seyi bilip de yapmayan, günah islemis olur.

\chapter{5}

\par 1 Dinleyin simdi ey zenginler, basiniza gelecek felaketlerden ötürü feryat edip aglayin.
\par 2 Servetiniz çürümüs, giysinizi güve yemistir.
\par 3 Altinlariniz, gümüsleriniz pas tutmustur. Onlarin pasi size karsi taniklik edecek, etinizi ates gibi yiyecek. Bu son çagda servetinize servet kattiniz.
\par 4 Iste, ekinlerinizi biçen isçilerin haksizca alikoydugunuz ücretleri size karsi haykiriyor. Orakçilarin feryadi Her Seye Egemen Rab'bin kulagina eristi.
\par 5 Yeryüzünde zevk ve bolluk içinde yasadiniz. Bogazlanacaginiz gün için kendinizi besiye çektiniz.
\par 6 Size karsi koymayan dogru kisiyi yargilayip öldürdünüz. Sabredin
\par 7 Öyleyse kardesler, Rab'bin gelisine dek sabredin. Bakin, çiftçi ilk ve son yagmurlari alincaya dek topragin degerli ürününü nasil sabirla bekliyor!
\par 8 Siz de sabredin. Yüreklerinizi güçlendirin. Çünkü Rab'bin gelisi yakindir.
\par 9 Kardesler, yargilanmamak için birbirinize karsi homurdanmayin. Iste, Yargiç kapinin önünde duruyor.
\par 10 Kardesler, Rab'bin adiyla konusmus olan peygamberleri sikintilarda sabir örnegi olarak alin.
\par 11 Sikintiya dayanmis olanlari mutlu sayariz. Eyüp'ün nasil dayandigini duydunuz. Rab'bin en sonunda onun için neler yaptigini bilirsiniz. Rab çok sefkatli ve merhametlidir.
\par 12 Kardeslerim, öncelikle sunu söyleyeyim: Ne gök üzerine, ne yer üzerine, ne de baska bir sey üzerine ant için. "Evet"iniz evet, "hayir"iniz hayir olsun ki, yargiya ugramayasiniz.
\par 13 Içinizden biri sikintida mi, dua etsin. Sevinçli mi, ilahi söylesin.
\par 14 Içinizden biri hasta mi, kilisenin* ihtiyarlarini* çagirtsin; Rab'bin adiyla üzerine yag sürüp onun için dua etsinler.
\par 15 Imanla edilen dua hastayi iyilestirecek ve Rab onu ayaga kaldiracaktir. Eger hasta günah islemisse, günahlari bagislanacaktir.
\par 16 Bu nedenle, sifa bulmak için günahlarinizi birbirinize itiraf edin ve birbiriniz için dua edin. Dogru kisinin yalvarisi çok güçlü ve etkilidir.
\par 17 Ilyas da tipki bizim gibi insandi. Yagmur yagmamasi için gayretle dua etti; üç yil alti ay ülkeye yagmur yagmadi.
\par 18 Yeniden dua etti; gök yagmurunu, toprak da ürününü verdi.
\par 19 Kardeslerim, içinizden biri gerçegin yolundan sapar da baska biri onu yine gerçege döndürürse, bilsin ki, günahkâri sapik yolundan döndüren, ölümden bir can kurtarmis, bir sürü günahi örtmüs olur.


\end{document}