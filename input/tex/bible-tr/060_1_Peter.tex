\begin{document}

\title{1 Petrus}


\chapter{1}

\par 1 Mesih Isa'nin elçisi ben Petrus'tan Pontus, Galatya, Kapadokya, Asya Ili* ve Bitinya'ya dagilmis ve buralarda yabanci olarak yasayan seçilmislere selam!
\par 2 Isa Mesih'in sözünü dinlemeniz ve O'nun kaninin üzerinize serpilmesi için, Baba Tanri'nin öngörüsü uyarinca Ruh tarafindan kutsal kilinarak seçildiniz. Lütuf ve esenlik artan ölçüde sizin olsun.
\par 3 Rabbimiz Isa Mesih'in Tanrisi ve Babasi'na övgüler olsun. Çünkü O büyük merhametiyle yeniden dogmamizi sagladi. Isa Mesih'i ölümden diriltmekle bizi yasayan bir umuda, çürümez, lekesiz, solmaz bir mirasa kavusturdu. Bu miras sizin için göklerde saklidir.
\par 5 Zaman sona ererken açiga çikarilmaya hazir olan kurtulusa kavusasiniz diye iman sayesinde Tanri'nin gücüyle korunuyorsunuz.
\par 6 Bu nedenle simdi kisa bir süre çesitli denemeler sonucu aci çekmeniz gerekiyorsa da, sevinçle cosmaktasiniz.
\par 7 Böylelikle içtenligi kanitlanan imaniniz, Isa Mesih göründügünde size övgü, yücelik, onur kazandiracak. Imaniniz, atesle aritildigi halde yok olup giden altindan daha degerlidir.
\par 8 Mesih'i görmemis olsaniz da O'nu seviyorsunuz. Su anda O'nu görmediginiz halde O'na iman ediyor, sözle anlatilmaz yüce bir sevinçle cosuyorsunuz.
\par 9 Çünkü imaninizin sonucu olarak canlarinizin kurtulusuna erisiyorsunuz.
\par 10 Size bagislanacak lütuftan söz etmis olan peygamberler, bu kurtulusla ilgili dikkatli incelemeler, arastirmalar yaptilar.
\par 11 Içlerinde olan Mesih Ruhu, Mesih'in çekecegi acilara ve bu acilarin ardindan gelecek yüceliklere taniklik ettiginde, Ruh'un hangi zamani ya da nasil bir dönemi belirttigini arastirdilar.
\par 12 Simdi size de bildirilen gerçeklerle kendilerine degil, size hizmet ettikleri onlara açikça gösterildi. Bu gerçekleri gökten gönderilen Kutsal Ruh'un gücüyle size Müjde'yi iletenler bildirdi. Melekler bu gerçekleri yakindan görmeye büyük özlem duyarlar.
\par 13 Bu nedenle zihinlerinizi eyleme hazirlayin, ayik olun. Umudunuzu tümüyle Isa Mesih'in görünmesiyle size saglanacak olan lütfa baglayin.
\par 14 Söz dinleyen çocuklar olarak, bilgisiz oldugunuz geçmis zamandaki tutkulariniza uymayin.
\par 15 Sizi çagiran Tanri kutsal olduguna göre, siz de her davranisinizda kutsal olun.
\par 16 Nitekim söyle yazilmistir: "Kutsal olun, çünkü ben kutsalim."
\par 17 Kimseyi kayirmadan, kisiyi yaptiklarina bakarak yargilayan Tanri'yi Baba diye çagirdiginiza göre, gurbeti andiran bu dünyadaki zamaninizi Tanri korkusuyla geçirin.
\par 18 Biliyorsunuz ki, atalarinizdan kalma bos yasayisinizdan altin ya da gümüs gibi geçici seylerle degil, kusursuz ve lekesiz kuzuyu andiran Mesih'in degerli kaninin fidyesiyle kurtuldunuz.
\par 20 Dünyanin kurulusundan önce bilinen Mesih, çaglarin sonunda sizin yarariniza ortaya çikti.
\par 21 O'nu ölümden diriltip yücelten Tanri'ya O'nun araciligiyla iman ediyorsunuz. Böylece imaniniz ve umudunuz Tanri'dadir.
\par 22 Gerçege uymakla kendinizi arittiniz, kardesler için içten bir sevgiye sahip oldunuz. Onun için birbirinizi candan, yürekten sevin.
\par 23 Çünkü ölümlü degil, ölümsüz bir tohumdan, yani Tanri'nin diri ve kalici sözü araciligiyla yeniden dogdunuz.
\par 24 Nitekim,"Insan soyu ota benzer, Bütün yüceligi kir çiçegi gibidir. Ot kurur, çiçek solar, Ama Rab'bin sözü sonsuza dek kalir." Iste size müjdelenmis olan söz budur.

\chapter{2}

\par 1 Bu nedenle her kötülügü, hileyi, ikiyüzlülügü, kiskançligi ve bütün iftiralari üzerinizden siyirip atin.
\par 2 Yeni dogmus bebekler gibi, hilesiz sütü andiran Tanri sözünü özleyin ki, bununla beslenip büyüyerek kurtulusa erisesiniz.
\par 3 Çünkü Rab'bin iyiligini tattiniz.
\par 4 Insanlarca reddedilmis, ama Tanri'ya göre seçkin ve degerli olan diri tasa, Rab'be gelin.
\par 5 O sizi diri taslar olarak ruhsal bir tapinagin yapiminda kullansin. Böylelikle, Isa Mesih araciligiyla Tanri'nin begenisini kazanan ruhsal kurbanlar sunmak üzere kutsal bir kâhinler* toplulugu olursunuz.
\par 6 Çünkü Kutsal Yazi'da söyle deniyor: "Iste, Siyon'a* bir tas, Seçkin, degerli bir köse tasi koyuyorum. O'na iman eden hiç utandirilmayacak."
\par 7 Iman eden sizler için bu tas degerlidir. Ama imansizlar için, "Yapicilarin reddettigi tas Kösenin bas tasi," "Sürçme tasi ve tökezleme kayasi oldu." Imansizlar Tanri'nin sözünü dinlemedikleri için sürçerler. Zaten sürçmek üzere belirlenmislerdir.
\par 9 Ama siz seçilmis soy, Kral'in kâhinleri, kutsal ulus, Tanri'nin öz halkisiniz. Sizi karanliktan sasilasi isigina çagiran Tanri'nin erdemlerini duyurmak için seçildiniz.
\par 10 Bir zamanlar halk degildiniz, ama simdi Tanri'nin halkisiniz. Bir zamanlar merhamete erismemistiniz, simdiyse merhamete eristiniz.
\par 11 Sevgili kardesler, size yalvaririm, cana karsi savasan benligin tutkularindan kaçinin. Çünkü bu dünyada yabanci ve konuksunuz.
\par 12 Inanmayanlar arasinda olumlu bir yasam sürün. Öyle ki, kötülük yapanlarmissiniz gibi size iftira etseler de, iyi islerinizi görerek Tanri'yi, kendilerine yaklastigi gün yüceltsinler.
\par 13 Insanlar arasinda yetkili kilinmis her kuruma -gerek her seyin üstünde olan krala gerekse kötülük yapanlarin cezalandirilmasi, iyilik edenlerin onurlandirilmasi için kral tarafindan gönderilen valilere Rab adina bagimli olun.
\par 15 Çünkü Tanri'nin istegi, iyilik yaparak akilsizlarin bilgisizligini susturmanizdir.
\par 16 Özgür insanlar olarak yasayin, ancak özgürlügünüzü kötülük yapmak için bahane etmeyin. Tanri'nin kullari olarak yasayin.
\par 17 Herkese saygi gösterin. Imanli kardeslerinizi sevin, Tanri'dan korkun, krala saygi gösterin.
\par 18 Ey hizmetkârlar, efendilerinizin yalniz iyi ve yumusak huylu olanlarina degil, ters huylu olanlarina da tam bir saygiyla bagimli olun.
\par 19 Haksiz yere aci çeken kisi, Tanri bilinciyle aciya katlanirsa, Tanri'yi hosnut eder.
\par 20 Çünkü günah isleyip dövüldügünüzde dayanirsaniz, bunda övülecek ne var? Ama iyilik edip aci çektiginizde dayanirsaniz, Tanri'yi hosnut edersiniz.
\par 21 Nitekim bunun için çagrildiniz. Mesih, izinden gidesiniz diye ugrunuza aci çekerek size örnek oldu.
\par 22 "O günah islemedi, agzindan hileli söz çikmadi."
\par 23 Kendisine sövüldügünde sövgüyle karsilik vermedi, aci çektiginde kimseyi tehdit etmedi; davasini, adaletle yargilayan Tanri'ya birakti.
\par 24 Bizler günah karsisinda ölelim, dogruluk ugruna yasayalim diye, günahlarimizi çarmihta kendi bedeninde yüklendi. O'nun yaralariyla sifa buldunuz.
\par 25 Çünkü yolunu sasirmis koyunlar gibiydiniz, simdiyse canlarinizin Çobani'na ve Gözetmeni'ne döndünüz.

\chapter{3}

\par 1 Bunun gibi, ey kadinlar, siz de kocalariniza bagimli olun. Öyle ki, kimileri Tanri sözüne inanmasa bile, Tanri korkusuna dayanan temiz yasayisinizi görerek söze gerek kalmadan karilarinin yasayisiyla kazanilsinlar.
\par 3 Süsünüz örgülü saçlar, altin takilar, güzel giysiler gibi disla ilgili olmasin.
\par 4 Gizli olan iç varliginiz, sakin ve yumusak bir ruhun solmayan güzelligiyle süsünüz olsun. Bu, Tanri'nin gözünde çok degerlidir.
\par 5 Çünkü geçmiste umudunu Tanri'ya baglamis olan kutsal kadinlar da kocalarina bagimli olarak böyle süslenirlerdi.
\par 6 Örnegin Sara Ibrahim'i "Efendim" diye çagirir, sözünü dinlerdi. Iyilik eder, hiçbir tehditten yilmazsaniz, siz de Sara'nin çocuklari olursunuz.
\par 7 Bunun gibi, ey kocalar, siz de daha zayif varliklar olan karilarinizla anlayis içinde yasayin. Tanri'nin lütfettigi yasamin ortak mirasçilari olduklari için onlara saygi gösterin. Öyle ki, dualariniza bir engel çikmasin.
\par 8 Sonuç olarak hepiniz ayni düsüncede birlesin. Baskalarinin duygularini paylasin. Birbirinizi kardesçe sevin. Sefkatli, alçakgönüllü olun.
\par 9 Kötülüge kötülükle, sövgüye sövgüyle degil, tersine, kutsamayla karsilik verin. Çünkü kutsanmayi miras almak için çagrildiniz.
\par 10 Söyle ki, "Yasamdan zevk almak, Iyi günler görmek isteyen, Dilini kötülükten, Dudaklarini yalandan uzak tutsun.
\par 11 Kötülükten sakinip iyilik yapsin. Esenligi amaçlasin, ardinca gitsin.
\par 12 Çünkü Rab'bin gözleri Dogru kisilerin üzerindedir. Kulaklari onlarin yakarisina açiktir. Ama Rab kötülük yapanlara karsidir."
\par 13 Iyilik yapmakta gayretli olursaniz, size kim kötülük edecek?
\par 14 Dogruluk ugruna aci çekseniz bile, ne mutlu size! Insanlarin "korktugundan korkmayin, ürkmeyin."
\par 15 Mesih'i Rab olarak yüreklerinizde kutsayin. Içinizdeki umudun nedenini soran herkese uygun bir yanit vermeye her zaman hazir olun.
\par 16 Yalniz bunu yumusak huyla, saygiyla yapin. Vicdaninizi temiz tutun. Öyle ki, Mesih'e ait olarak sürdürdügünüz olumlu yasami kinayanlar size ettikleri iftiradan utansinlar.
\par 17 Iyilik edip aci çekmek -eger Tanri'nin istegi buysa- kötülük yapip aci çekmekten daha iyidir.
\par 18 Nitekim Mesih de bizleri Tanri'ya ulastirmak amaciyla dogru kisi olarak dogru olmayanlar için günah sunusu* olarak ilk ve son kez öldü. Bedence öldürüldü, ama ruhça diriltildi.
\par 19 Ruhta gidip bunlari zindanda olan ruhlara da duyurdu.
\par 20 Bir zamanlar, Nuh'un günlerinde gemi yapilirken, Tanri'nin sabirla beklemesine karsin bu ruhlar söz dinlememislerdi. O gemide birkaç kisi, daha dogrusu sekiz kisi suyla kurtuldu.
\par 21 Bu olay vaftizi* simgeliyor. Bedenin kirden arinmasi degil, Tanri'ya yönelen temiz vicdanin dilegi olan vaftiz, Isa Mesih'in dirilisiyle simdi sizi de kurtariyor.
\par 22 Göge çikmis olan Mesih Tanri'nin sagindadir. Bütün melekler, yetkiler ve güçler O'na bagli kilinmistir.

\chapter{4}

\par 1 Mesih bedence aci çektigine göre, siz de ayni düsünceyle silahlanin. Çünkü bedence aci çekmis olan, günaha sirt çevirmistir.
\par 2 Sonuç olarak, dünyadaki yasaminin geri kalan bölümünü artik insan tutkularina göre degil, Tanri'nin istegine göre sürdürür.
\par 3 Inanmayanlarin hoslandiklarini yaparak sefahat, sehvet, sarhosluk, çilgin eglenceler, içki alemleri ve ilke tanimayan putperestlik içinde yasayarak geçmiste harcadiginiz günler yeter!
\par 4 Inanmayanlar, kendinizi onlarla birlikte ayni sefahat seline atmamanizi yadirgiyor, size sövüyorlar.
\par 5 Onlar, ölüleri de dirileri de yargilamaya hazir olan Tanri'ya hesap verecekler.
\par 6 Çünkü ölüler bedence öbür insanlar gibi yargilansin, ama ruhça Tanri gibi yasasin diye Müjde onlara da bildirildi.
\par 7 Her seyin sonu yakindir. Bu nedenle, sagduyulu olun ve dua etmek için ayik durun.
\par 8 Her seyden önce birbirinizi candan sevin. Çünkü sevgi birçok günahi örter.
\par 9 Söylenmeksizin birbirinize konukseverlik gösterin.
\par 10 Her biriniz hangi ruhsal armagani aldiysaniz, bunu Tanri'nin çok yönlü lütfunun iyi kâhyalari olarak birbirinize hizmet etmekte kullanin.
\par 11 Konusan, Tanri'nin sözlerini iletir gibi konussun. Baskalarina hizmet eden, Tanri'nin verdigi güçle hizmet etsin. Öyle ki, Isa Mesih araciligiyla Tanri her seyde yüceltilsin. Yücelik ve kudret sonsuzlara dek Mesih'indir! Amin.
\par 12 Sevgili kardeslerim, sinanmaniz için size giydirilen atesten gömlegi, size garip bir sey oluyormus gibi yadirgamayin.
\par 13 Tersine, Mesih'in acilarina ortak oldugunuz oranda sevinin ki, Mesih'in görkemi göründügünde de sevinçle cosasiniz.
\par 14 Mesih'in adindan ötürü hakarete ugrarsaniz, ne mutlu size! Çünkü Tanri'nin yüce Ruhu üzerinizde bulunuyor.
\par 15 Hiçbiriniz katil, hirsiz, kötülük yapan ya da baskalarinin isine karisan biri olarak aci çekmesin.
\par 16 Ama Mesih inanlisi oldugu için aci çeken, bundan utanç duymasin. Tasidigi bu adla Tanri'yi yüceltsin.
\par 17 Çünkü yarginin, Tanri'nin ev halkindan baslayacagi an gelmistir. Eger yargilama önce bizden baslarsa, Tanri'nin Müjdesi'ne kulak asmayanlarin sonu ne olacak?
\par 18 "Dogru kisi güçlükle kurtuluyorsa, Tanrisiz ve günahli kisiye ne olacak?"
\par 19 Bunun için, Tanri'nin istegi uyarinca aci çekenler, iyilik ederek canlarini güvenilir Yaradan'a emanet etsinler.

\chapter{5}

\par 1 Bu nedenle aranizdaki ihtiyarlara*, onlar gibi bir ihtiyar, Mesih'in çektigi acilarin tanigi, açiga çikacak olan yüceligin paydasi olarak rica ediyorum: Tanri'nin size verdigi sürüyü güdün. Zorunluymus gibi degil, Tanri'nin istedigi gibi gönüllü gözetmenlik yapin. Para hirsiyla degil, gönül rizasiyla, size emanet edilenlere egemenlik taslamadan, sürüye örnek olarak göreviniziyapin.
\par 4 Bas Çoban göründügü zaman yüceligin solmaz tacina kavusacaksiniz.
\par 5 Ey gençler, siz de ihtiyarlara bagimli olun. Hepiniz birbirinize karsi alçakgönüllülügü kusanin. Çünkü, "Tanri kibirlilere karsidir, Ama alçakgönüllülere lütfeder."
\par 6 Uygun zamanda sizi yüceltmesi için, Tanri'nin kudretli eli altinda kendinizi alçaltin.
\par 7 Bütün kaygilarinizi O'na yükleyin, çünkü O sizi kayirir.
\par 8 Ayik ve uyanik olun. Düsmaniniz Iblis kükreyen aslan gibi yutacak birini arayarak dolasiyor.
\par 9 Dünyanin her yerindeki kardeslerinizin de ayni acilari çektigini bilerek imanda sarsilmadan Iblis'e karsi direnin.
\par 10 Sizleri Mesih'te sonsuz yüceligine çagiran ve bütün lütfun kaynagi olan Tanri'nin kendisi kisa bir süre aci çekmenizden sonra sizi yetkinlestirip pekistirecek, güçlendirip temellendirecektir.
\par 11 Kudret sonsuzlara dek O'nun olsun! Amin.
\par 12 Kendisini güvenilir bir kardes saydigim Silvanus araciligiyla size kisaca yazmis bulunuyorum. Sizi yüreklendiriyor ve sözünü ettigim lütfun Tanri'nin gerçek lütfu olduguna taniklik ediyorum. Buna bagli kalin.
\par 13 Sizler gibi seçilmis olan Babil'deki kilise* ve oglum Markos size selam ederler.
\par 14 Birbirinizi sevgiyle öperek selamlayin. Sizlere, Mesih'e ait olan herkese esenlik olsun.


\end{document}