\begin{document}

\title{2 Petrus}


\chapter{1}

\par 1 Isa Mesih'in kulu ve elçisi ben Simun Petrus'tan Tanrimiz ve Kurtaricimiz Isa Mesih'in dogrulugu sayesinde bizimkiyle esdeger bir imana kavusmus olanlara selam!
\par 2 Tanri'yi ve Rabbimiz Isa'yi tanimakla lütuf ve esenlik artan ölçüde sizin olsun.
\par 3 Kendi yüceligi ve erdemiyle bizi çagiranin tanrisal gücü, kendisini tanimamiz sonucunda yasamamiz ve Tanri yolunda yürümemiz için gereken her seyi bize verdi.
\par 4 O'nun yüceligi ve erdemi sayesinde bize çok büyük ve degerli vaatler verildi. Öyle ki, dünyada kötü arzularin yol açtigi yozlasmadan kurtulmus olarak, bu vaatler araciligiyla tanrisal özyapiya ortak olasiniz.
\par 5 Iste bu nedenle her türlü gayreti göstererek imaniniza erdemi, erdeminize bilgiyi, bilginize özdenetimi, özdenetiminize dayanma gücünü, dayanma gücünüze Tanri yoluna bagliligi, bagliliginiza kardesseverligi, kardesseverliginize sevgiyi katin.
\par 8 Çünkü bu niteliklere artan ölçüde sahip olursaniz, Rabbimiz Isa Mesih'i tanimakta etkisiz ve verimsiz olmazsiniz.
\par 9 Bu niteliklere sahip olmayan uzagi göremez, kördür. Eski günahlarindan temizlendigini unutmustur.
\par 10 Bunun için, ey kardesler, çagrilmisliginizi ve seçilmisliginizi köklestirmeye daha çok gayret edin. Bunlari yaparsaniz, hiçbir zaman tökezlemezsiniz.
\par 11 Böylece Rabbimiz ve Kurtaricimiz Isa Mesih'in sonsuz egemenligine girme hakki size cömertçe saglanacaktir.
\par 12 Onun için, her ne kadar bunlari biliyorsaniz ve sahip oldugunuz gerçekle pekistirilmisseniz de, bunlari size her zaman animsatacagim.
\par 13 Bu bedende yasadigim sürece bunlari animsatarak sizi gayrete getirmeyi dogru buluyorum.
\par 14 Rabbimiz Isa Mesih'in bana bildirdigi gibi, bedenden ayrilisimin yakin oldugunu biliyorum.
\par 15 Ben bu dünyadan göçtükten sonra da bunlari sürekli animsayabilmeniz için simdi her gayreti gösterecegim.
\par 16 Rabbimiz Isa Mesih'in kudretini ve gelisini size bildirirken uydurma masallara basvurmadik. O'nun görkemini gözlerimizle gördük.
\par 17 Mesih, yüce ve görkemli Olan'dan kendisine ulasan sesle, "Sevgili Oglum budur, O'ndan hosnudum" diyen sesle Baba Tanri'dan onur ve yücelik aldi.
\par 18 Kutsal dagda O'nunla birlikte bulundugumuz için gökten gelen bu sesi biz de isittik.
\par 19 Peygamberlerin sözleri bizim için daha büyük kesinlik kazandi. Gün agarip sabah yildizi yüreklerinizde doguncaya dek, karanlik yerde isik saçan çiraya benzeyen bu sözlere kulak verirseniz, iyi edersiniz.
\par 20 Öncelikle sunu bilin ki, Kutsal Yazilar'daki hiçbir peygamberlik sözü kimsenin özel yorumu degildir.
\par 21 Çünkü hiçbir peygamberlik sözü insan isteginden kaynaklanmadi. Kutsal Ruh tarafindan yöneltilen insanlar Tanri'nin sözlerini ilettiler.

\chapter{2}

\par 1 Ama Israil halki arasinda sahte peygamberler vardi; tipki sizin de aranizda yanlis ögreti yayanlar olacagi gibi. Bunlar kendilerini satin alan Efendi'yi bile yadsiyarak gizlice araniza yikici ögretiler sokacaklar. Böyleleri kendi baslarina ani bir yikim getirecek.
\par 2 Birçoklari da onlarin sefahatine kapilacak. Onlarin yüzünden gerçegin yoluna sövülecek.
\par 3 Açgözlülüklerinden ötürü uydurma sözlerle sizi sömürecekler. Onlar için çoktan beri verilmis olan yargi gecikmez. Onlari bekleyen yikim da uyuklamaz.
\par 4 Tanri günah isleyen melekleri esirgemedi; onlari cehenneme atip karanlikta zincire vurdu. Yargilanincaya dek orada tutulacaklar.
\par 5 Tanri eski dünyayi da esirgemedi. Ama tanrisizlarin dünyasina tufani gönderdiginde, dogruluk yolunu bildiren Nuh'u ve yedi kisiyi daha korudu.
\par 6 Sodom ve Gomora kentlerini yakip yikarak yargiladi. Böylece tanrisizlarin basina geleceklere bir örnek verdi.
\par 7 Ama ilke tanimayan kisilerin sefih yasayisindan azap duyan dogru adam Lut'u kurtardi.
\par 8 Çünkü onlarin arasinda yasayan bu dogru adam, görüp isittigi yasa tanimaz davranislar yüzünden dogru yüreginde her gün istirap çekerdi.
\par 9 Görülüyor ki Rab kendi yolunda yürüyenleri karsilastiklari denemelerden nasil kurtaracagini bilir. Dogru olmayanlari, özellikle benligin yozlasmis tutkulari ardindan giden ve yetkisini hor görenleri cezalandirarak yargi gününe dek nasil alikoyacagini da bilir. Bu küstah, dikbasli kisiler yüce varliklara sövmekten korkmazlar.
\par 11 Oysa melekler bile, güç ve kudrette daha üstün olduklari halde bu varliklari Rab'bin önünde söverek yargilamazlar.
\par 12 Ama anlamadiklari konularda sövüp sayan bu kisiler, içgüdüleriyle yasayan, yakalanip bogazlanmak üzere dogan, akildan yoksun hayvanlar gibidir. Hayvanlar gibi onlar da yikima ugrayacaklar.
\par 13 Ettikleri haksizliga karsilik zarar görecekler. Gündüzün zevk alemlerine dalmayi eglence sayarlar. Birer leke ve yüzkarasidirlar. Sizinle yiyip içerken kendi hilelerinden zevk alirlar.
\par 14 Gözleri zinayla doludur, günaha doymazlar. Kararsiz kisileri ayartirlar. Yüregi açgözlülüge alistirilmis lanetli insanlardir.
\par 15 Haksizlikla elde ettigi kazanci seven Beor oglu Balam'in yolunu tutarak dogru yolu birakip saptilar.
\par 16 Balam isledigi suçtan ötürü azarlandi. Konusamayan esek, insan diliyle konusarak bu peygamberin çilginligina engel oldu.
\par 17 Bu kisiler, susuz pinarlar, firtinanin dagittigi sis gibidirler. Onlari koyu karanlik bekliyor.
\par 18 Çünkü yanlis yolda yürüyenlerden henüz kurtulanlari, bos ve kurumlu sözler söyleyerek benligin tutkulariyla, sefahatle ayartirlar.
\par 19 Onlara özgürlük vaat ederler, oysa kendileri yozlasmisligin kölesidirler. Çünkü insan neye yenilirse onun kölesi olur.
\par 20 Rab ve Kurtarici Isa Mesih'i tanimakla dünyanin çirkefliginden kurtulduktan sonra yine ayni islere karisip yenilirlerse, son durumlari ilk durumlarindan beter olur.
\par 21 Çünkü dogruluk yolunu bilip de kendilerine emanet edilen kutsal buyruktan geri dönmektense, bu yolu hiç bilmemis olmak onlar için daha iyi olurdu.
\par 22 Su gerçek özdeyis onlarin durumunu anlatiyor: "Köpek kendi kusmuguna döner", "Domuz da yikandiktan sonra çamurda yuvarlanmaya döner."

\chapter{3}

\par 1 Sevgili kardesler, simdi bu benim size yazdigim ikinci mektuptur. Her iki mektubumda da bu konulari animsatarak temiz düsüncelerinizi uyandirmaya çalistim.
\par 2 Öyle ki, kutsal peygamberlerin çok önceden söyledigi sözleri ve Kurtaricimiz Rab'bin elçileriniz araciligiyla verdigi buyrugu animsayasiniz.
\par 3 Öncelikle sunu bilmelisiniz: Dünyanin son günlerinde kendi tutkularinin ardindan giden alayci kisiler türeyecek. Bunlar, "Rab'bin gelisiyle ilgili vaat ne oldu? Atalarimizin ölümünden beri her sey yaratilisin baslangicinda oldugu gibi duruyor" diyerek alay edecekler.
\par 5 Ne var ki, göklerin çok önceden Tanri'nin sözüyle var oldugunu, yerin sudan ve su araciligiyla sekillendigini bile bile unutuyorlar.
\par 6 O zamanki dünya yine suyla, tufanla mahvolmustu.
\par 7 Simdiki yer ve göklerse atese verilmek üzere ayni sözle saklaniyor, tanrisizlarin yargilanarak mahvolacagi güne dek korunuyorlar.
\par 8 Sevgili kardeslerim, sunu unutmayin ki, Rab'bin gözünde bir gün bin yil, bin yil bir gün gibidir.
\par 9 Bazilarinin düsündügü gibi Rab vaadini yerine getirmekte gecikmez; ama size karsi sabrediyor. Çünkü kimsenin mahvolmasini istemiyor, herkesin tövbe etmesini istiyor.
\par 10 Ama Rab'bin günü hirsiz gibi gelecek. O gün gökler büyük bir gürültüyle ortadan kalkacak, maddesel ögeler yanarak yok olacak, yer ve yeryüzünde yapilmis olan her sey yanip tükenecek.
\par 11 Her sey böylece yok olacagina göre, sizin nasil kisiler olmaniz gerekir? Tanri'nin gününü bekleyip o günün gelisini çabuklastirarak kutsallik içinde yasamali, Tanri yolunu izlemelisiniz. O gün gökler yanarak yok olacak, maddesel ögeler siddetli ateste eriyip gidecek.
\par 13 Ama biz Tanri'nin vaadi uyarinca dogrulugun barinacagi yeni gökleri, yeni yeryüzünü bekliyoruz.
\par 14 Bunun için, sevgili kardeslerim, mademki bunlari bekliyorsunuz, Tanri'nin önünde lekesiz, kusursuz ve baris içinde olmaya gayret edin.
\par 15 Sevgili kardesimiz Pavlus'un da kendisine verilen bilgelikle size yazdigi gibi, Rabbimiz'in sabrini kurtulus firsati sayin.
\par 16 Pavlus bütün mektuplarinda bu konulardan böyle söz eder. Mektuplarinda güç anlasilan bazi yerler var ki, bilgisiz ve kararsiz kisiler, öbür Kutsal Yazilar'i oldugu gibi bunlari da çarpitarak kendi yikimlarini hazirliyorlar.
\par 17 Bu nedenle, sevgili kardeslerim, ilke tanimayan kisilerin aldatmasiyla sürüklenip kararliliginizdan sapmamak için bunlari önceden bilerek sakinin.
\par 18 Öte yandan Rabbimiz ve Kurtaricimiz Isa Mesih'in lütfunda ve O'nu tanimakta ilerleyin. Simdi ve sonsuza dek O'na yücelik olsun! Amin.


\end{document}