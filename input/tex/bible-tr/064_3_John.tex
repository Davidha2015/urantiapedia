\begin{document}

\title{3 Yuhanna}


\chapter{1}

\par 1 Ben ihtiyardan*, gerçekten sevdigim sevgili Gayus'a selam!
\par 2 Sevgili kardesim, canin gönenç içinde oldugu gibi, her bakimdan saglikli ve gönenç içinde olman için dua ediyorum.
\par 3 Bazi kardesler gelip senin gerçege bagli kaldigina, gerçegin izinden yürüdügüne taniklik edince çok sevindim.
\par 4 Benim için, çocuklarimin gerçegin izinden yürüdüklerini duymaktan daha büyük bir sevinç olamaz!
\par 5 Sevgili kardesim, sana yabanci olduklari halde, kardesler için yaptigin her seyi içten bir baglilikla yapiyorsun.
\par 6 Onlar kilise* önünde sevgine taniklik ettiler. Onlari Tanri'ya yarasir biçimde yardimlarinla birlikte ugurlarsan iyi edersin.
\par 7 Çünkü inanmayanlardan hiçbir yardim almadan, Mesih'in adi ugruna yola çiktilar.
\par 8 Bu nedenle, gerçek ugruna emektaslar olmak için böylelerini desteklemeliyiz.
\par 9 Kiliseye bazi seyler yazdim, ama aralarinda en üstün olma sevdasinda olan Diotrefis bizi kabul etmiyor.
\par 10 Bunun için, eger gelirsem, bize yönelttigi haksiz suçlamalarla yaptigi kötülükleri animsatacagim. Bununla yetinmeyerek kardesleri de kabul etmiyor, kabul etmek isteyenlere de engel olup onlari kiliseden disari atiyor.
\par 11 Sevgili kardesim, kötüyü degil, iyiyi örnek al. Iyilik yapan kisi Tanri'dandir. Kötülük yapansa Tanri'yi görmemistir.
\par 12 Herkesle birlikte gerçegin kendisi, Dimitrios'un degerli biri olduguna taniklik ediyor. Biz de taniklik ederiz. Tanikligimizin dogru oldugunu biliyorsun.
\par 13 Sana yazacak çok seyim var, ama mürekkeple, kalemle yazmak istemiyorum.
\par 14 Yakinda seni görmek umudundayim, o zaman yüz yüze konusuruz. Esen kal! Arkadaslar sana selam ederler. Sen de oradaki arkadaslara adli adinca selam söyle.


\end{document}