\begin{document}

\title{Esinleme}


\chapter{1}

\par 1 Isa Mesih'in vahyidir. Tanri yakin zamanda olmasi gereken olaylari kullarina göstermesi için O'na bu vahyi verdi. O da gönderdigi melegi araciligiyla bunu kulu Yuhanna'ya iletti.
\par 2 Yuhanna, Tanri'nin sözüne ve Isa Mesih'in tanikligina -gördügü her seye- taniklik etmektedir.
\par 3 Bu peygamberlik sözlerini okuyana, burada yazilanlari dinleyip yerine getirene ne mutlu! Çünkü beklenen zaman yakindir.
\par 4 Ben Yuhanna'dan, Asya Ili'ndeki* yedi kiliseye* selam! Var olan, var olmus ve gelecek olandan, O'nun tahtinin önünde bulunan yedi ruhtan ve ölüler arasindan ilk dogan, dünya krallarina egemen olan güvenilir tanik Isa Mesih'ten sizlere lütuf ve esenlik olsun. Yücelik ve güç sonsuzlara dek, bizi seven, kaniyla bizi günahlarimizdan özgür kilmis ve bizi bir krallik haline getirip Babasi Tanri'nin hizmetinde kâhinler* yapmis olan Mesih'in olsun! Amin.
\par 7 Iste bulutlarla geliyor! Her göz O'nu görecek, O'nun bedenini desmis olanlar bile. O'nun için dövünecek yeryüzünün bütün halklari. Evet, böyle olacak! Amin.
\par 8 Var olan, var olmus ve gelecek olan, Her Seye Gücü Yeten Rab Tanri, "Alfa* ve Omega* Ben'im" diyor.
\par 9 Isa'ya ait biri olarak sikintida, tanrisal egemenlikte ve sabirda ortaginiz ve kardesiniz olan ben Yuhanna, Tanri'nin sözü ve Isa'ya taniklik ugruna Patmos denilen adada bulunuyordum.
\par 10 Rab'bin gününde Ruh'un etkisinde kalarak arkamda borazan sesine benzer yüksek bir ses isittim.
\par 11 Ses, "Gördüklerini kitaba yaz ve yedi kiliseye*, yani Efes, Izmir, Bergama, Tiyatira, Sart, Filadelfya ve Laodikya'ya gönder" dedi.
\par 12 Bana sesleneni görmek için arkama döndüm. Döndügümde yedi altin kandillik ve bunlarin ortasinda, giysileri ayagina kadar uzanan, gögsüne altin kusak sarinmis, insanogluna benzer birini gördüm.
\par 14 Basi, saçi ak yapagi gibi beyaz, kar gibi bembeyazdi. Gözleri alev alev yanan atesti sanki.
\par 15 Ayaklari, ocakta kor haline gelmis parlak tunca benziyordu. Sesi, gürül gürül akan sularin sesi gibiydi.
\par 16 Sag elinde yedi yildiz vardi. Agzindan iki agizli keskin bir kiliç uzaniyordu. Yüzü bütün gücüyle parlayan günes gibiydi.
\par 17 O'nu görünce, ölü gibi ayaklarinin dibine yigildim. O ise sag elini üzerime koyup söyle dedi: "Korkma! Ilk ve son Ben'im.
\par 18 Diri Olan Ben'im. Ölmüstüm, ama iste sonsuzluklar boyunca diriyim. Ölümün ve ölüler diyarinin anahtarlari bendedir.
\par 19 Bunun için gördüklerini, simdi olanlari ve bundan sonra olacaklari yaz.
\par 20 Sag elimde gördügün yedi yildizla yedi altin kandilligin sirrina gelince, yedi yildiz yedi kilisenin melekleri, yedi kandillikse yedi kilisedir."

\chapter{2}

\par 1 "Efes'teki kilisenin* melegine yaz. Yedi yildizi sag elinde tutan, yedi altin kandilligin ortasinda yürüyen söyle diyor:
\par 2 `Yaptiklarini, çaliskanligini, sabrini biliyorum. Kötü kisilere katlanamadigini da biliyorum. Elçi olmadiklari halde kendilerini elçi diye tanitanlari sinadin ve onlari yalanci buldun.
\par 3 Evet, sabirlisin, adim ugruna acilara dayandin ve yilmadin.
\par 4 Ne var ki, bir konuda sana karsiyim: Baslangiçtaki sevginden uzaklastin.
\par 5 Bunun için, nereden düstügünü animsa! Tövbe et ve baslangiçta yaptiklarini sürdür. Tövbe etmezsen, gelip kandilligini yerinden kaldiririm.
\par 6 Yine de olumlu bir yanin var: Nikolas yanlilarinin yaptiklarindan nefret ediyorsun; ben de nefret ederim.
\par 7 Kulagi olan, Ruh'un kiliselere ne dedigini isitsin. Galip gelene Tanri'nin cennetinde bulunan yasam agacindan yeme hakkini verecegim.'"
\par 8 "Izmir'deki kilisenin* melegine yaz. Ölmüs ve yasama dönmüs, ilk ve son olan söyle diyor:
\par 9 `Sikintilarini, yoksullugunu biliyorum. Oysa zenginsin! Yahudi olduklarini söyleyen, ama Yahudi degil de Seytan'in havrasi durumunda olanlarin iftiralarini biliyorum.
\par 10 Çekmek üzere oldugun sikintilardan korkma! Bak, denenesiniz diye Iblis içinizden bazilarini yakinda zindana atacak. On gün sikinti çekeceksiniz. Ölüm pahasina da olsa sadik kal, sana yasam tacini verecegim.
\par 11 Kulagi olan, Ruh'un kiliselere ne dedigini isitsin. Galip gelen, ikinci ölümden hiçbir zarar görmeyecek.'"
\par 12 "Bergama'daki kilisenin* melegine yaz. Iki agizli keskin kilica sahip olan söyle diyor:
\par 13 `Nerede yasadigini biliyorum; Seytan'in tahti oradadir. Yine de adima simsiki baglisin. Aranizda, Seytan'in yasadigi yerde öldürülen sadik tanigim Antipa'nin günlerinde bile bana olan imanini yadsimadin.
\par 14 Ne var ki, birkaç konuda sana karsiyim: Aranizda Balam'in ögretisine bagli olanlar var. Putlara sunulan kurbanlarin etini yemeleri, fuhus yapmalari için Israilogullari'ni ayartmayi Balak'a ögreten Balam'di.
\par 15 Bunun gibi, sizin aranizda da Nikolas yanlilarinin ögretisine bagli olanlar var.
\par 16 Bunun için tövbe et! Yoksa yanina tez gelir, agzimdaki kiliçla onlara karsi savasirim.
\par 17 Kulagi olan, Ruh'un kiliselere ne dedigini isitsin. Galip gelene sakli mandan* verecegim. Ayrica, ona beyaz bir tas ve bu tasin üzerinde yazili olan yeni bir ad, alandan baska kimsenin bilmedigi bir ad verecegim.'"
\par 18 "Tiyatira'daki kilisenin* melegine yaz. Gözleri alev alev yanan atese, ayaklari parlak tunca benzeyen Tanri'nin Oglu söyle diyor:
\par 19 `Yaptiklarini, sevgini, imanini, hizmetini, sabrini biliyorum. Son yaptiklarinin ilk yaptiklarini astigini da biliyorum.
\par 20 Ne var ki, bir konuda sana karsiyim: Kendini peygamber diye tanitan Izebel adindaki kadini hosgörüyle karsiliyorsun. Bu kadin ögretisiyle kullarimi saptirip fuhus yapmaya, putlara sunulan kurbanlarin etini yemeye yöneltiyor.
\par 21 Tövbe etmesi için ona bir süre tanidim, ama fuhus yapmaktan tövbe etmek istemiyor.
\par 22 Bak, onu yataga düsürecegim; onun yaptiklarindan tövbe etmezlerse, onunla zina edenleri de büyük sikintilarin içine atacagim.
\par 23 Onun çocuklarini salgin hastalikla öldürecegim. O zaman bütün kiliseler, gönülleri ve yürekleri denetleyenin ben oldugumu bilecekler. Her birinize yaptiklarinizin karsiligini verecegim.
\par 24 "`Ama size, yani Tiyatira'da bulunan öbürlerine, bu ögretiyi benimsememis, Seytan'in sözde derin sirlarini ögrenmemis olanlarin hepsine sunu söylüyorum: Ben gelinceye dek sizde olana simsiki sarilin. Üzerinize bundan baska bir yük koymuyorum.
\par 26 Ben Babam'dan nasil yetki aldimsa, galip gelene, yaptigim isleri sonuna dek sürdürene uluslarin üzerinde yetki verecegim. Demir çomakla güdecek onlari, Çömlek gibi kirip parçalayacaktir. Galip gelene sabah yildizini da verecegim.
\par 29 Kulagi olan, Ruh'un kiliselere ne dedigini isitsin.'"

\chapter{3}

\par 1 "Sart'taki kilisenin* melegine yaz. Tanri'nin yedi ruhuna ve yedi yildiza sahip olan söyle diyor: `Yaptiklarini biliyorum. Yasiyorsun diye ad yapmissin, ama ölüsün.
\par 2 Uyan! Geriye kalan ve ölmek üzere olan ne varsa güçlendir. Çünkü yaptiklarinin Tanrim'in önünde tamamlanmamis oldugunu gördüm.
\par 3 Bu nedenle neler aldigini, neler isittigini animsa. Bunlari yerine getir, tövbe et! Eger uyanmazsan, hirsiz gibi gelecegim. Hangi saatte gelecegimi hiç bilemeyeceksin.
\par 4 Ama Sart'ta, aranizda giysilerini lekelememis birkaç kisi var ki, beyazlar içinde benimle birlikte yürüyecekler. Çünkü buna layiktirlar.
\par 5 Galip gelen böylece beyaz giysiler giyecek. Onun adini yasam kitabindan hiç silmeyecegim. Babam'in ve meleklerinin önünde o kisinin adini açikça anacagim.
\par 6 Kulagi olan, Ruh'un kiliselere ne dedigini isitsin.'"
\par 7 "Filadelfya'daki kilisenin* melegine yaz. Kutsal ve gerçek olan, Davut'un anahtarina sahip olan, açtigini kimsenin kapayamadigi, kapadigini kimsenin açamadigi kisi söyle diyor:
\par 8 `Yaptiklarini biliyorum. Iste önüne kimsenin kapayamayacagi açik bir kapi koydum. Gücünün az oldugunu biliyorum; yine de sözüme uydun, adimi yadsimadin.
\par 9 Bak, Seytan'in havrasindan olanlari, Yahudi olmadiklari halde Yahudi olduklarini ileri süren yalancilari öyle edecegim ki, gelip ayaklarina kapanacak, benim seni sevdigimi anlayacaklar.
\par 10 Sözüme uyarak sabirla dayandin. Ben de yeryüzünde yasayanlari denemek için bütün dünyanin üzerine gelecek olan denenme saatinden seni esirgeyecegim.
\par 11 Tez geliyorum. Tacini kimse elinden almasin diye sahip olduguna simsiki saril.
\par 12 Galip geleni Tanrim'in Tapinagi'nda sütun yapacagim. Böyle biri artik oradan hiç ayrilmayacak. Onun üzerine Tanrim'in adini, Tanrim'a ait kentin -gökten Tanrim'in yanindan inen yeni Yerusalim'in- adini ve benim yeni adimi yazacagim.
\par 13 Kulagi olan, Ruh'un kiliselere ne dedigini isitsin.'"
\par 14 "Laodikya'daki kilisenin* melegine yaz. Amin, sadik ve gerçek tanik, Tanri yaratilisinin kaynagi söyle diyor:
\par 15 `Yaptiklarini biliyorum. Ne soguksun, ne sicak. Keske ya soguk ya sicak olsaydin!
\par 16 Oysa ne sicak ne soguksun, iliksin. Bu yüzden seni agzimdan kusacagim.
\par 17 Zenginim, zenginlestim, hiçbir seye gereksinmem yok diyorsun; ama zavalli, acinacak durumda, yoksul, kör ve çiplak oldugunu bilmiyorsun.
\par 18 Zengin olmak için benden ateste aritilmis altin, giyinip çiplakliginin ayibini örtmek için beyaz giysiler, görmek için gözlerine sürmek üzere merhem satin almani salik veriyorum.
\par 19 Ben sevdiklerimi azarlayip terbiye ederim. Onun için gayrete gel, tövbe et.
\par 20 Iste kapida durmus, kapiyi çaliyorum. Biri sesimi isitir ve kapiyi açarsa, onun yanina girecegim; ben onunla, o da benimle, birlikte yemek yiyecegiz.
\par 21 Ben nasil galip gelerek Babam'la birlikte Babam'in tahtina oturdumsa, galip gelene de benimle birlikte tahtima oturma hakkini verecegim.
\par 22 Kulagi olan, Ruh'un kiliselere ne dedigini isitsin.'"

\chapter{4}

\par 1 Bundan sonra gökte açik duran bir kapi gördüm. Benimle konustugunu isittigim, borazan sesine benzeyen ilk ses söyle dedi: "Buraya çik! Bundan sonra olmasi gereken olaylari sana göstereyim."
\par 2 O anda Ruh'un etkisinde kalarak gökte bir taht ve tahtta oturan birini gördüm.
\par 3 Tahtta oturanin, yesim ve kirmizi akik tasina benzer bir görünüsü vardi. Zümrüdü andiran bir gökkusagi tahti çevreliyordu.
\par 4 Tahtin çevresinde yirmi dört ayri taht vardi. Bu tahtlara baslarinda altin taçlar olan, beyaz giysilere bürünmüs yirmi dört ihtiyar oturmustu.
\par 5 Tahttan simsekler çakiyor, ugultular, gök gürlemeleri isitiliyordu. Tahtin önünde alev alev yanan yedi mesale vardi. Bunlar Tanri'nin yedi ruhudur.
\par 6 Tahtin önünde billur gibi, sanki camdan bir deniz vardi. Tahtin ortasinda ve çevresinde, önü ve arkasi gözlerle kapli dört yaratik duruyordu.
\par 7 Birinci yaratik aslana, ikincisi danaya benziyordu. Üçüncü yaratigin yüzü insan yüzü gibiydi. Dördüncü yaratik uçan bir kartali andiriyordu.
\par 8 Dört yaratigin her birinin altisar kanadi vardi. Yaratiklarin her yani, kanatlarinin alt tarafi bile gözlerle kapliydi. Gece gündüz durup dinlenmeden söyle diyorlar: "Kutsal, kutsal, kutsaldir, Her Seye Gücü Yeten Rab Tanri, Var olmus, var olan ve gelecek olan."
\par 9 Yaratiklar tahtta oturani, sonsuzluklar boyunca yasayani yüceltip ona saygi ve sükran sundukça, yirmi dört ihtiyar tahtta oturanin, sonsuzluklar boyunca yasayanin önünde yere kapanarak O'na tapiniyorlar. Taçlarini tahtin önüne koyarak söyle diyorlar: "Rabbimiz ve Tanrimiz! Yüceligi, saygiyi, gücü almaya layiksin. Çünkü her seyi sen yarattin; Hepsi senin isteginle yaratilip var oldu."

\chapter{5}

\par 1 Tahtta oturanin sag elinde iki yani da yazili, yedi mühürle mühürlenmis bir tomar gördüm.
\par 2 Yüksek sesle, "Tomari açmaya, mühürlerini çözmeye kim layiktir?" diye seslenen güçlü bir melek de gördüm.
\par 3 Ama ne gökte, ne yeryüzünde, ne de yer altinda tomari açip içine bakabilecek kimse yoktu.
\par 4 Aci aci aglamaya basladim. Çünkü tomari açip içine bakmaya layik kimse bulunamadi.
\par 5 Bunun üzerine ihtiyarlardan biri bana, "Aglama!" dedi. "Iste, Yahuda oymagindan gelen Aslan, Davut'un Kökü galip geldi. Tomari ve yedi mührünü O açacak."
\par 6 Tahtin, dört yaratigin ve ihtiyarlarin ortasinda, bogazlanmis gibi duran bir Kuzu gördüm. Yedi boynuzu, yedi gözü vardi. Bunlar Tanri'nin bütün dünyaya gönderilmis yedi ruhudur.
\par 7 Kuzu gelip tahtta oturanin sag elinden tomari aldi.
\par 8 Tomari alinca, dört yaratikla yirmi dört ihtiyar O'nun önünde yere kapandilar. Her birinin elinde birer lir ve kutsallarin dualari olan buhur dolu altin taslar vardi.
\par 9 Yeni bir ezgi söylüyorlardi: "Tomari almaya, Mühürlerini açmaya layiksin! Çünkü bogazlandin Ve kaninla her oymaktan, her dilden, Her halktan, her ulustan Insanlari Tanri'ya satin aldin.
\par 10 Onlari Tanrimiz'in hizmetinde Bir krallik haline getirdin, Kâhinler* yaptin. Dünya üzerinde egemenlik sürecekler."
\par 11 Sonra tahtin, yaratiklarin ve ihtiyarlarin çevresinde çok sayida melek gördüm, seslerini isittim. Sayilari binlerce binler, onbinlerce onbinlerdi.
\par 12 Yüksek sesle söyle diyorlardi: "Bogazlanmis Kuzu Gücü, zenginligi, bilgeligi, kudreti, Saygiyi, yüceligi, övgüyü Almaya layiktir."
\par 13 Ardindan gökte, yeryüzünde, yer altinda ve denizlerdeki bütün yaratiklarin, bunlardaki bütün varliklarin söyle dedigini isittim: "Övgü, saygi, yücelik ve güç sonsuzlara dek Tahtta oturanin ve Kuzu'nun olsun!"
\par 14 Dört yaratik, "Amin" dediler. Ihtiyarlar da yere kapanip tapindilar.

\chapter{6}

\par 1 Sonra Kuzu'nun yedi mühürden birini açtigini gördüm. O anda dört yaratiktan birinin, gök gürültüsüne benzer bir sesle, "Gel!" dedigini isittim.
\par 2 Bakinca beyaz bir at gördüm. Binicisinin yayi vardi. Kendisine bir taç verildi ve galip gelen biri olarak zafer kazanmaya çikti.
\par 3 Kuzu ikinci mührü açinca, ikinci yaratigin "Gel!" dedigini isittim.
\par 4 O zaman kizil renkte baska bir at çikti ortaya. Binicisine dünyadan barisi kaldirma yetkisi verildi. Bunun sonucu olarak insanlar birbirlerini bogazlayacaklar. Atliya ayrica büyük bir kiliç verildi.
\par 5 Kuzu üçüncü mührü açinca, üçüncü yaratigin "Gel!" dedigini isittim. Bakinca siyah bir at gördüm. Binicisinin elinde bir terazi vardi.
\par 6 Dört yaratigin ortasinda sanki bir sesin söyle dedigini isittim: "Bir ölçek bugday bir dinara, üç ölçek arpa bir dinara. Ama zeytinyagina, saraba zarar verme!"
\par 7 Kuzu dördüncü mührü açinca, "Gel!" diyen dördüncü yaratigin sesini isittim.
\par 8 Bakinca soluk renkli bir at gördüm. Binicisinin adi Ölüm'dü. Ölüler diyari onun ardinca geliyordu. Bunlara kiliçla, kitlikla, salgin hastalikla, yeryüzünün yabanil hayvanlariyla ölüm saçmak için yeryüzünün dörtte biri üzerinde yetki verildi.
\par 9 Kuzu besinci mührü açinca, sunagin altinda, Tanri'nin sözü ve sürdürdükleri taniklik nedeniyle öldürülenlerin canlarini gördüm.
\par 10 Yüksek sesle feryat ederek söyle diyorlardi: "Kutsal ve gerçek olan Efendimiz! Yeryüzünde yasayanlari yargilayip onlardan kanimizin öcünü almak için daha ne kadar bekleyeceksin?"
\par 11 Onlarin her birine beyaz birer kaftan verildi. Kendileri gibi öldürülecek olan öbür Tanri kullarinin ve kardeslerinin sayisi tamamlanincaya dek kisa bir süre daha beklemeleri istendi.
\par 12 Kuzu altinci mührü açinca, büyük bir deprem oldugunu gördüm. Günes keçi kilindan yapilmis siyah bir çul gibi karardi. Ay bastan asagi kan rengine döndü.
\par 13 Incir agaci, güçlü bir rüzgarla sarsildiginda nasil ham incirlerini dökerse, gökteki yildizlar da öylece yeryüzüne düstü.
\par 14 Gökyüzü dürülen bir tomar gibi ortadan kalkti. Her dag, her ada yerinden sökülüp alindi.
\par 15 Dünya krallari, büyükleri, komutanlari, zenginleri, güçlüleri, özgürü kölesi herkes magaralara, daglardaki kayalarin arasina gizlendiler.
\par 16 Daglara, kayalara, "Üzerimize düsün!" dediler, "Tahtta oturanin yüzünden ve Kuzu'nun gazabindan saklayin bizi!
\par 17 Çünkü onlarin gazabinin büyük günü geldi. Buna kim dayanabilir?"

\chapter{7}

\par 1 Bundan sonra yeryüzünün dört kösesinde duran dört melek gördüm. Bunlar karaya, denize ya da herhangi bir agaç üzerine esmesin diye, yeryüzünün dört rüzgarini tutuyorlardi.
\par 2 Sonra gündogusundan yükselen baska bir melek gördüm. Yasayan Tanri'nin mührünü tasiyordu. Karaya, denize zarar vermek için yetki verilen dört melege yüksek sesle bagirdi:
\par 3 "Biz Tanrimiz'in kullarini alinlarindan mühürleyene dek karaya, denize ya da agaçlara zarar vermeyin!"
\par 4 Mühürlenmis olanlarin sayisini isittim. Israilogullari'nin bütün oymaklarindan 144 000 kisi mühürlenmisti:
\par 5 Yahuda oymagindan 12 000 kisi mühürlenmisti. Ruben oymagindan 12 000, Gad oymagindan 12 000,
\par 6 Aser oymagindan 12 000, Naftali oymagindan 12 000, Manasse oymagindan 12 000,
\par 7 Simon oymagindan 12 000, Levi oymagindan 12 000, Issakar oymagindan 12 000,
\par 8 Zevulun oymagindan 12 000, Yusuf oymagindan 12 000, Benyamin oymagindan 12 000 kisi mühürlenmisti.
\par 9 Bundan sonra gördüm ki, her ulustan, her oymaktan, her halktan, her dilden olusan, kimsenin sayamayacagi kadar büyük bir kalabalik tahtin ve Kuzu'nun önünde duruyordu. Hepsi de birer beyaz kaftan giymisti, ellerinde hurma dallari vardi.
\par 10 Yüksek sesle bagiriyorlardi: "Kurtaris, tahtta oturan Tanrimiz'a Ve Kuzu'ya özgüdür!"
\par 11 Bütün melekler tahtin, ihtiyarlarin ve dört yaratigin çevresinde duruyordu. Tahtin önünde yüzüstü yere kapanip Tanri'ya tapinarak söyle diyorlardi:
\par 12 "Amin! Övgü, yücelik, bilgelik, Sükran, saygi, güç, kudret, Sonsuzlara dek Tanrimiz'in olsun! Amin!"
\par 13 Bu sirada ihtiyarlardan biri bana sordu: "Beyaz kaftan giymis olan bu kisiler kim, nereden geldiler?"
\par 14 "Sen bunu biliyorsun, efendim" dedim. Bana dedi ki, "Bunlar o büyük sikintidan geçip gelenlerdir. Kaftanlarini Kuzu'nun kaniyla yikamis, bembeyaz etmislerdir.
\par 15 Bunun için, "Tanri'nin tahti önünde duruyor, Tapinaginda gece gündüz O'na tapiniyorlar. Tahtta oturan, çadirini onlarin üzerine gerecek.
\par 16 Artik acikmayacak, Artik susamayacaklar. Ne günes ne kavurucu sicak Çarpacak onlari.
\par 17 Çünkü tahtin ortasinda olan Kuzu onlari güdecek Ve yasam sularinin pinarlarina götürecek. Tanri gözlerinden bütün yaslari silecek."

\chapter{8}

\par 1 Kuzu yedinci mührü açinca, gökte yarim saat kadar sessizlik oldu.
\par 2 Tanri'nin önünde duran yedi melegi gördüm. Onlara yedi borazan verildi.
\par 3 Altin bir buhurdan tasiyan baska bir melek gelip sunagin önünde durdu. Tahtin önündeki altin sunakta bütün kutsallarin dualariyla birlikte sunmak üzere kendisine çok miktarda buhur verildi.
\par 4 Kutsallarin dualariyla buhurun dumani, Tanri'nin önünde melegin elinden yükseldi.
\par 5 Melek buhurdani aldi, sunagin atesiyle doldurup yeryüzüne atti. Gök gürlemeleri, ugultular isitildi, simsekler çakti, yer sarsildi.
\par 6 Yedi melek ellerindeki yedi borazani çalmaya hazirlandi.
\par 7 Birinci melek borazanini çaldi. Kanla karisik dolu ve ates olustu, yeryüzüne yagdi. Yerin üçte biri, agaçlarin üçte biri ve bütün yesil otlar yandi.
\par 8 Ikinci melek borazanini çaldi. Alev alev yanan, dag gibi büyük bir kütle denize atildi. Denizin üçte biri kana dönüstü.
\par 9 Denizdeki yaratiklarin üçte biri öldü, gemilerin üçte biri yok oldu.
\par 10 Üçüncü melek borazanini çaldi. Gökten mesale gibi yanan büyük bir yildiz irmaklarin üçte biri üzerine ve su pinarlarinin üzerine düstü.
\par 11 Bu yildizin adi Pelin'dir. Sularin üçte biri pelin gibi acilasti. Acilasan sulardan içen birçok insan öldü.
\par 12 Dördüncü melek borazanini çaldi. Günesin üçte biri, ayin üçte biri, yildizlarin üçte biri vuruldu. Sonuç olarak isiklarinin üçte biri söndü, gündüzün ve gecenin üçte biri isiksiz kaldi.
\par 13 Sonra gögün ortasinda uçan bir kartal gördüm. Yüksek sesle söyle bagirdigini isittim: "Borazanlarini çalacak olan öbür üç melegin borazan seslerinden yeryüzünde yasayanlarin vay, vay, vay haline!"

\chapter{9}

\par 1 Besinci melek borazanini çaldi. Gökten yere düsmüs bir yildiz gördüm. Dipsiz derinliklere açilan kuyunun anahtari ona verildi.
\par 2 Dipsiz derinliklerin kuyusunu açinca, kuyudan büyük bir ocagin dumani gibi bir duman çikti. Kuyunun dumanindan günes ve hava karardi.
\par 3 Dumanin içinden yeryüzüne çekirgeler yagdi. Bunlara yeryüzündeki akreplerin gücüne benzer bir güç verilmisti.
\par 4 Çekirgelere yeryüzündeki otlara, herhangi bir bitki ya da agaca degil de, yalniz alinlarinda Tanri'nin mührü bulunmayan insanlara zarar vermeleri söylendi.
\par 5 Bu insanlari öldürmelerine degil, bes ay süreyle iskence etmelerine izin verildi. Yaptiklari iskence akrebin insani soktugu zaman verdigi aciya benziyordu.
\par 6 O günlerde insanlar ölümü arayacak, ama bulamayacaklar. Ölümü özleyecekler, ama ölüm onlardan kaçacak.
\par 7 Çekirgelerin görünümü, savasa hazirlanmis atlara benziyordu. Baslarinda altin taçlara benzer basliklar vardi. Yüzleri insan yüzleri gibiydi.
\par 8 Saçlari kadin saçina, disleri aslan disine benziyordu.
\par 9 Demir zirhlara benzer gögüs zirhlari vardi. Kanatlarinin sesi savasa kosan çok sayida atli arabanin sesine benziyordu.
\par 10 Akrebinkine benzer kuyruklari ve igneleri vardi. Kuyruklarinda, insanlara bes ay zarar verecek güce sahiptiler.
\par 11 Baslarinda kral olarak dipsiz derinliklerin melegi vardi. Bu melegin Ibranice* adi Avaddon, Grekçe adiysa Apolyon'dur.
\par 12 Birinci "vay" geçti, iste bundan sonra iki "vay" daha geliyor.
\par 13 Altinci melek borazanini çaldi. Tanri'nin önündeki altin sunagin dört boynuzundan gelen bir ses isittim.
\par 14 Ses, elinde borazan olan altinci melege, "Büyük Firat Irmagi'nin yaninda bagli duran dört melegi çöz" dedi.
\par 15 Tam o saat, o gün, o ay, o yil için hazir tutulan dört melek, insanlarin üçte birini öldürmek üzere çözüldü.
\par 16 Atli ordularinin sayisi iki yüz milyondu, sayilarini duydum.
\par 17 Görümümde atlari ve binicilerini gördüm. Ates, gökyakut ve kükürt renginde gögüs zirhlari kusanmislardi. Atlarin baslari aslan basina benziyordu. Agizlarindan ates, duman, kükürt fiskiriyordu.
\par 18 Insanlarin üçte biri bunlarin agzindan fiskiran ates, duman ve kükürtten, bu üç beladan öldü.
\par 19 Atlarin gücü agizlarinda ve kuyruklarindadir. Yilani andiran kuyruklarinin basiyla zarar verirler.
\par 20 Geriye kalan insanlar, yani bu belalardan ölmemis olanlar, kendi elleriyle yaptiklari putlardan dönüp tövbe etmediler. Cinlere ve göremeyen, isitemeyen, yürüyemeyen altin, gümüs, tunç, tas, tahta putlara tapmaktan vazgeçmediler.
\par 21 Adam öldürmekten, büyü, fuhus, hirsizlik yapmaktan da tövbe etmediler.

\chapter{10}

\par 1 Sonra gökten inen güçlü baska bir melek gördüm. Buluta sarinmisti, basinin üzerinde gökkusagi vardi. Yüzü günese, ayaklari atesten sütunlara benziyordu.
\par 2 Elinde açilmis küçük bir tomar vardi. Sag ayagini denize, sol ayagini karaya koyarak aslanin kükremesini andiran yüksek sesle bagirdi. O bagirinca, yedi gök gürlemesi dile gelip seslendiler.
\par 4 Yedi gök gürlemesi seslendiginde yazmak üzereydim ki, gökten, "Yedi gök gürlemesinin söylediklerini mühürle, yazma!" diyen bir ses isittim.
\par 5 Denizle karanin üzerinde durdugunu gördügüm melek, sag elini göge kaldirdi.
\par 6 Gögü ve göktekileri, yeri ve yerdekileri, denizi ve denizdekileri yaratanin, sonsuzluklar boyunca yasayanin hakki için ant içip dedi ki, "Artik gecikme olmayacak.
\par 7 Yedinci melek borazanini çaldigi zaman, Tanri'nin sir olan tasarisi tamamlanacak. Nitekim Tanri bunu, kullari peygamberlere müjdelemisti."
\par 8 Gökten isittigim ses benimle yine konusmaya basladi: "Git, denizle karanin üzerinde duran melegin elindeki açik tomari al" dedi.
\par 9 Melegin yanina gidip küçük tomari bana vermesini istedim. "Al, bunu ye!" dedi. "Midende bir acilik yapacak, ama agzina bal gibi tatli gelecek."
\par 10 Küçük tomari melegin elinden alip yedim, agzimda bal gibi tatliydi. Ama yutunca midem acilasti.
\par 11 Sonra bana söyle dendi: "Yine birçok halk, ulus, dil ve kralla ilgili olarak peygamberlikte bulunmalisin."

\chapter{11}

\par 1 Bana degnege benzer bir ölçü kamisi verilip söyle dendi: "Git, Tanri'nin Tapinagi'ni ve sunagi ölç, orada tapinanlari say!
\par 2 Tapinagin dis avlusunu birak, orayi ölçme. Çünkü orasi, kutsal kenti* kirk iki ay ayaklariyla çigneyecek olan uluslara verildi.
\par 3 Iki tanigima güç verecegim; çul giysiler içinde bin iki yüz altmis gün peygamberlik edecekler."
\par 4 Bunlar yeryüzünün Rabbi önünde duran iki zeytin agaciyla iki kandilliktir.
\par 5 Biri onlara zarar vermeye kalkisirsa, agizlarindan ates fiskiracak ve düsmanlarini yiyip bitirecek. Onlara zarar vermek isteyen herkesin böyle öldürülmesi gerekir.
\par 6 Peygamberlik ettikleri sürece yagmur yagmasin diye gögü kapamaya yetkileri vardir. Sulari kana dönüstürme ve yeryüzünü, kaç kez isterlerse, her türlü belayla vurma yetkisine sahiptirler.
\par 7 Taniklik görevleri sona erince dipsiz derinliklerden çikan canavar onlarla savasacak, onlari yenip öldürecek.
\par 8 Cesetleri, simgesel olarak Sodom ve Misir diye adlandirilan büyük kentin anayoluna serilecek. Onlarin Rabbi de orada çarmiha gerilmisti.
\par 9 Her halktan, oymaktan, dilden, ulustan insan üç buçuk gün cesetlerini seyredecek, cesetlerinin mezara konulmasina izin vermeyecekler.
\par 10 Yeryüzünde yasayanlar onlarin bu durumuna sevinip bayram edecek, birbirlerine armaganlar gönderecekler. Çünkü bu iki peygamber yeryüzünde yasayanlara çok eziyet etmisti.
\par 11 Üç buçuk gün sonra iki peygamber, Tanri'dan gelen yasam solugunu alinca ayaga kalktilar. Onlari görenler dehsete kapildi.
\par 12 Iki peygamber gökten gelen yüksek bir sesin, "Buraya çikin!" dedigini isittiler. Sonra düsmanlarinin gözü önünde bir bulut içinde göge yükseldiler.
\par 13 Tam o saatte siddetli bir deprem oldu, kentin onda biri yikildi. Depremde yedi bin kisi can verdi. Geriye kalanlar dehsete kapilip gökteki Tanri'yi yücelttiler.
\par 14 Ikinci "vay" geçti. Iste, üçüncü "vay" tez geliyor.
\par 15 Yedinci melek borazanini çaldi. Gökte yüksek sesler duyuldu: "Dünyanin egemenligi Rabbimiz'in ve Mesihi'nin oldu. O sonsuzlara dek egemenlik sürecek."
\par 16 Tanri'nin önünde tahtlarinda oturan yirmi dört ihtiyar yüzüstü yere kapandi. Tanri'ya tapinarak söyle dediler: "Her Seye Gücü Yeten, Var olan, var olmus olan Rab Tanri! Sana sükrediyoruz. Çünkü büyük gücünü kusanip Egemenlik sürmeye basladin.
\par 18 Uluslar gazaba gelmislerdi. Simdiyse senin gazabin üzerlerine geldi. Ölüleri yargilamak, Kullarin olan peygamberleri, kutsallari, Küçük olsun büyük olsun, Senin adindan korkanlari ödüllendirmek Ve yeryüzünü mahvedenleri mahvetmek zamani da geldi."
\par 19 Ardindan Tanri'nin gökteki tapinagi açildi, tapinakta O'nun Antlasma Sandigi* göründü. O anda simsekler çakti, ugultular, gök gürlemeleri isitildi. Yer sarsildi, siddetli bir dolu firtinasi koptu.

\chapter{12}

\par 1 Gökte olaganüstü bir belirti, günese sarinmis bir kadin göründü. Ay ayaklarinin altindaydi, basinda on iki yildizdan olusan bir taç vardi.
\par 2 Kadin gebeydi. Dogum sancilari içinde kivraniyor, feryat ediyordu.
\par 3 Ardindan gökte baska bir belirti göründü: Yedi basli, on boynuzlu, kizil renkli büyük bir ejderhaydi bu. Yedi basinda yedi taç vardi.
\par 4 Kuyruguyla gökteki yildizlarin üçte birini sürükleyip yeryüzüne atti. Sonra dogum yapmak üzere olan kadinin önünde durdu; kadin dogurur dogurmaz Ejderha çocugu yutacakti.
\par 5 Kadin bir ogul, bütün uluslari demir çomakla güdecek bir erkek çocuk dogurdu. Çocuk hemen alinip Tanri'ya, Tanri'nin tahtina götürüldü.
\par 6 Kadinsa çöle kaçti. Orada bin iki yüz altmis gün beslenmesi için Tanri tarafindan hazirlanmis bir yeri vardi.
\par 7 Gökte savas oldu. Mikail'le melekleri ejderhayla savastilar. Ejderha kendi melekleriyle birlikte karsi koydu, ama gücü yetmedi. Bu yüzden gökteki yerlerini yitirdiler.
\par 9 Büyük ejderha -Iblis ya da Seytan denen, bütün dünyayi saptiran o eski yilan- melekleriyle birlikte yeryüzüne atildi.
\par 10 Bundan sonra gökte yüksek bir sesin söyle dedigini duydum: "Tanrimiz'in kurtarisi, gücü, egemenligi Ve Mesihi'nin yetkisi simdi gerçeklesti. Çünkü kardeslerimizin suçlayicisi, Onlari Tanrimiz'in önünde gece gündüz suçlayan Asagi atildi.
\par 11 Kardeslerimiz Kuzu'nun kaniyla Ve ettikleri taniklik bildirisiyle Onu yendiler. Ölümü göze alacak kadar Vazgeçmislerdi can sevgisinden.
\par 12 Bunun için, ey gökler ve orada yasayanlar, Sevinin! Vay halinize, yer ve deniz! Çünkü Iblis zamaninin az oldugunu bilerek Büyük bir öfkeyle üzerinize indi."
\par 13 Ejderha yeryüzüne atildigini görünce, erkek çocugu doguran kadini kovalamaya basladi.
\par 14 Yilanin önünden çöle, üç buçuk yil beslenecegi yere uçup kaçabilmesi için kadina büyük kartal kanatlari verildi.
\par 15 Yilan agzindan, kadini selle süpürüp götürmek için onun ardindan irmak gibi su akitti.
\par 16 Ama yeryüzü, agzini açip ejderhanin agzindan akittigi irmagi yutarak kadina yardim etti.
\par 17 Bunun üzerine ejderha kadina öfkelendi. Kadinin soyundan geriye kalanlarla, Tanri'nin buyruklarini yerine getirip Isa'ya tanikliklarini sürdürenlerle savasmaya gitti.
\par 18 Denizin kiyisinda dikilip durdu.

\chapter{13}

\par 1 Sonra on boynuzlu, yedi basli bir canavarin denizden çiktigini gördüm. Boynuzlarinin üzerinde on taç vardi, baslarinin üzerinde küfür niteliginde adlar yaziliydi.
\par 2 Gördügüm canavar parsa benziyordu. Ayaklari ayi ayagi, agzi aslan agzi gibiydi. Ejderha canavara kendi gücü ve tahtiyla birlikte büyük yetki verdi.
\par 3 Canavarin baslarindan biri ölümcül bir yara almisa benziyordu. Ne var ki, bu ölümcül yara iyilesmisti. Bütün dünya saskinlik içinde canavarin ardindan gitti.
\par 4 Insanlar canavara yetki veren ejderhaya taptilar. "Canavar gibisi var mi? Onunla kim savasabilir?" diyerek canavara da taptilar.
\par 5 Canavara, kurumlu sözler söyleyen, küfürler savuran bir agiz ve kirk iki ay süreyle kullanabilecegi bir yetki verildi.
\par 6 Tanri'ya küfretmek, O'nun adina ve konutuna, yani gökte yasayanlara küfretmek için agzini açti.
\par 7 Kutsallarla savasip onlari yenmesine izin verildi. Canavar her oymak, her halk, her dil, her ulus üzerinde yetkili kilindi.
\par 8 Yeryüzünde yasayan ve dünya kurulali beri bogazlanmis Kuzu'nun yasam kitabina adi yazilmamis olan herkes ona tapacak.
\par 9 Kulagi olan isitsin!
\par 10 Tutsak düsecek olan Tutsak düsecek. Kiliçla öldürülecek olan Kiliçla öldürülecek. Bu, kutsallarin sabrini ve imanini gerektirir.
\par 11 Bundan sonra baska bir canavar gördüm. Yerden çikan bu canavarin kuzu gibi iki boynuzu vardi, ama ejderha gibi ses çikariyordu.
\par 12 Ilk canavarin bütün yetkisini onun adina kullaniyor, yeryüzünü ve orada yasayanlari ölümcül yarasi iyilesen ilk canavara tapmaya zorluyordu.
\par 13 Insanlarin gözü önünde, gökten yere ates yagdiracak kadar büyük belirtiler gerçeklestiriyordu.
\par 14 Ilk canavarin adina gerçeklestirmesine izin verilen belirtiler sayesinde, yeryüzünde yasayanlari saptirdi. Onlara kiliçla yaralanan, ama sag kalan canavarin onuruna bir heykel yapmalarini buyurdu.
\par 15 Canavarin heykeline yasam solugu vermesi için kendisine güç verildi. Öyle ki, heykel konusabilsin ve kendisine tapmayan herkesi öldürebilsin.
\par 16 Küçük büyük, zengin yoksul, özgür köle, herkesin sag eline ya da alnina bir isaret vurduruyordu.
\par 17 Öyle ki, bu isareti, yani canavarin adini ya da adini simgeleyen sayiyi tasimayan ne bir sey satin alabilsin, ne de satabilsin.
\par 18 Bu konu bilgelik gerektirir. Anlayabilen, canavara ait sayiyi hesaplasin. Çünkü bu sayi insani simgeler. Sayisi 666'dir.

\chapter{14}

\par 1 Sonra Kuzu'nun Siyon* Dagi'nda durdugunu gördüm. O'nunla birlikte 144 000 kisi vardi. Alinlarinda kendisinin ve Babasi'nin adlari yaziliydi.
\par 2 Gökten, gürül gürül akan sularin sesini, güçlü gök gürlemesini andiran bir ses isittim. Isittigim ses, lir çalanlarin çikardigi sese benziyordu.
\par 3 Bu 144 000 kisi, tahtin önünde, dört yaratigin ve ihtiyarlarin önünde yeni bir ezgi söylüyordu. Yeryüzünden satin alinmis olan bu kisilerden baska kimse o ezgiyi ögrenemedi.
\par 4 Kendilerini kadinlarla lekelememis olanlar bunlardir. Pak kisilerdir. Kuzu nereye giderse ardisira giderler. Tanri'ya ve Kuzu'ya ait olacaklarin ilk bölümü olmak üzere insanlar arasindan satin alinmislardir.
\par 5 Agizlarindan hiç yalan çikmamistir. Kusursuzdurlar.
\par 6 Bundan sonra gögün ortasinda uçan baska bir melek gördüm. Yeryüzünde yasayanlara -her ulusa, her oymaga, her dile, her halka- iletmek üzere sonsuza dek kalici olan Müjde'yi getiriyordu.
\par 7 Yüksek sesle söyle diyordu: "Tanri'dan korkun! O'nu yüceltin! Çünkü O'nun yargilama saati geldi. Gögü, yeri, denizi, su pinarlarini yaratana tapinin!"
\par 8 Ardindan gelen ikinci bir melek, "Yikildi! Kendi azgin fuhus sarabini bütün uluslara içiren büyük Babil yikildi!" diyordu.
\par 9 Onlari üçüncü bir melek izledi. Yüksek sesle söyle diyordu: "Bir kimse canavara ve heykeline taparsa, alnina ya da eline canavarin isaretini koydurursa, Tanri gazabinin kâsesinde saf olarak hazirlanmis Tanri öfkesinin sarabindan içecektir. Böylelerine kutsal meleklerin ve Kuzu'nun önünde ates ve kükürtle iskence edilecek.
\par 11 Çektikleri iskencenin dumani sonsuzlara dek tütecek. Canavara ve heykeline tapip onun adinin isaretini alanlar gece gündüz rahat yüzü görmeyecekler.
\par 12 Bu da, Tanri'nin buyruklarini yerine getiren, Isa'ya imanlarini sürdüren kutsallarin sabrini gerektirir."
\par 13 Gökten bir ses isittim. "Yaz! Bundan böyle Rab'be ait olarak ölenlere ne mutlu!" diyordu. Ruh, "Evet" diyor, "Ugraslarindan dinlenecekler. Çünkü yaptiklari onlari izleyecek."
\par 14 Sonra beyaz bir bulut gördüm. Bulutun üzerinde "insanogluna benzer biri" oturuyordu. Basinda altin bir taç, elinde keskin bir orak vardi.
\par 15 Tapinaktan çikan baska bir melek bulutun üzerinde oturana yüksek sesle bagirdi: "Oragini uzat ve biç! Biçme saati geldi. Çünkü yerin ekini olgunlasmis bulunuyor."
\par 16 Bulutun üzerinde oturan, oragini yerin üzerine salladi, yerin ekini biçildi.
\par 17 Gökteki tapinaktan baska bir melek çikti. Onun da keskin bir oragi vardi.
\par 18 Ates üzerinde yetkili olan baska bir melek de sunaktan çikip geldi. Keskin oragi olana yüksek sesle, "Keskin oragini uzat!" dedi. "Yerin asmasinin salkimlarini topla. Çünkü üzümleri olgunlasti."
\par 19 Bunun üzerine melek oragini yerin üzerine salladi. Yerin asmasinin ürününü toplayip Tanri öfkesinin büyük masarasina atti.
\par 20 Kentin disinda çignenen masaradan kan akti. Kan, 1 600 ok atimi kadar yayilip atlarin gemlerine dek yükseldi.

\chapter{15}

\par 1 Gökte büyük ve sasilasi baska bir belirti gördüm: Son yedi belayi tasiyan yedi melekti. Çünkü Tanri'nin öfkesi bu belalarla son buluyordu.
\par 2 Atesle karisik camdan deniz gibi bir sey gördüm. Canavara, heykeline ve adini simgeleyen sayiya karsi zafer kazananlar, ellerinde Tanri'nin verdigi lirlerle cam denizin üzerinde durmuslardi.
\par 3 Tanri kulu Musa'nin ve Kuzu'nun ezgisini söylüyorlardi: "Her Seye Gücü Yeten Rab Tanri, Senin islerin büyük ve sasilasi islerdir. Ey uluslarin krali, Senin yollarin dogru ve adildir. Ya Rab, senden kim korkmaz, Adini kim yüceltmez? Çünkü kutsal olan yalniz sensin. Bütün uluslar gelip sana tapinacaklar. Çünkü adil islerin açikça görüldü."
\par 5 Bundan sonra gökteki tapinagin, yani Taniklik Çadiri'nin açildigini gördüm.
\par 6 Yedi belayi tasiyan yedi melek temiz, parlak keten giysiler giymis, gögüslerine altin kusaklar sarinmis olarak tapinaktan çikti.
\par 7 Dört yaratiktan biri yedi melege, sonsuzluklar boyunca yasayan Tanri'nin öfkesiyle dolu yedi altin tas verdi.
\par 8 Tapinak Tanri'nin yüceliginden ve gücünden ötürü dumanla doldu. Yedi melegin yedi belasi sona erinceye dek kimse tapinaga giremedi.

\chapter{16}

\par 1 Sonra tapinaktan yükselen gür bir sesin yedi melege, "Gidin, Tanri'nin öfkesiyle dolu yedi tasi yeryüzüne bosaltin!" dedigini isittim.
\par 2 Birinci melek gidip tasini yeryüzüne bosaltti. Canavarin isaretini tasiyip heykeline tapanlarin üzerinde aci veren igrenç yaralar olustu.
\par 3 Ikinci melek tasini denize bosaltti. Deniz ölü kanina benzer kana dönüstü, içindeki bütün canlilar öldü.
\par 4 Üçüncü melek tasini irmaklara, su pinarlarina bosaltti; bunlar da kana dönüstü.
\par 5 Sulardan sorumlu melegin söyle dedigini isittim: "Var olan, var olmus olan kutsal Tanri! Bu yargilarinda adilsin.
\par 6 Kutsallarin ve peygamberlerin kanini döktükleri için, Içecek olarak sen de onlara kan verdin. Bunu hak ettiler."
\par 7 Sunaktan gelen bir sesin, "Evet, Her Seye Gücü Yeten Rab Tanri, Yargilarin dogru ve adildir" dedigini isittim.
\par 8 Dördüncü melek tasini günese bosaltti. Bununla günese insanlari yakma gücü verildi.
\par 9 Insanlar korkunç bir isiyla kavruldular. Tövbe edip bu belalara egemen olan Tanri'yi yücelteceklerine, O'nun adina küfrettiler.
\par 10 Besinci melek tasini canavarin tahtina bosaltti. Canavarin egemenligi karanliga gömüldü. Insanlar istiraptan dillerini isirdilar.
\par 11 Istirap ve yaralarindan ötürü Gögün Tanrisi'na küfrettiler. Yaptiklarindan tövbe etmediler.
\par 12 Altinci melek tasini büyük Firat Irmagi'na bosaltti. Gündogusundan gelen krallarin yolu açilsin diye irmagin sulari kurudu.
\par 13 Bundan sonra ejderhanin agzindan, canavarin agzindan ve sahte peygamberin agzindan kurbagaya benzer üç kötü ruhun çiktigini gördüm.
\par 14 Bunlar dogaüstü belirtiler gerçeklestiren cinlerin ruhlaridir. Her Seye Gücü Yeten Tanri'nin büyük gününde olacak savas için bütün dünyanin krallarini toplamaya gidiyorlar.
\par 15 "Iste hirsiz gibi geliyorum! Çiplak dolasmamak ve utanç içinde kalmamak için uyanik durup giysilerini üstünde bulundurana ne mutlu!"
\par 16 Üç kötü ruh, krallari Ibranice* Armagedon denilen yere topladilar.
\par 17 Yedinci melek tasini havaya bosaltti. Tapinaktaki tahttan yükselen gür bir ses, "Tamam!" dedi.
\par 18 O anda simsekler çakti, ugultular, gök gürlemeleri isitildi. Öyle büyük bir deprem oldu ki, yeryüzünde insan oldu olali bu kadar büyük bir deprem olmamisti.
\par 19 Büyük kent üçe bölündü. Uluslarin kentleri yerle bir oldu. Tanri büyük Babil'i animsadi, ona atesli gazabinin sarabini içeren kâseyi verdi.
\par 20 Bütün adalar ortadan kalkti, daglar yok oldu.
\par 21 Insanlarin üzerine gökten tanesi yaklasik kirk kilo agirliginda iri dolu yagdi. Dolu belasi öyle korkunçtu ki, insanlar bu yüzden Tanri'ya küfrettiler.

\chapter{17}

\par 1 Yedi tasi alan yedi melekten biri gelip benimle konustu: "Gel!" dedi. "Sana engin sularin kenarinda oturan büyük fahisenin çarptirilacagi cezayi göstereyim.
\par 2 Dünya krallari onunla fuhus yaptilar. Yeryüzünde yasayanlar onun fuhsunun sarabiyla sarhos oldular."
\par 3 Bundan sonra melek beni Ruh'un yönetiminde çöle götürdü. Orada yedi basli, on boynuzlu, üzeri küfür niteliginde adlarla kapli kirmizi bir canavarin üstüne oturmus bir kadin gördüm.
\par 4 Kadin, mor ve kirmizi giysilere bürünmüs, altinlar, degerli taslar, incilerle süslenmisti. Elinde igrenç seylerle, fuhsunun çirkeflikleriyle dolu altin bir kâse vardi.
\par 5 Alnina su gizemli ad yazilmisti:
\par 6 Kadinin, kutsallarin ve Isa'ya taniklik etmis olanlarin kaniyla sarhos oldugunu gördüm. Onu görünce büyük bir saskinliga düstüm.
\par 7 Melek bana, "Neden sastin?" diye sordu. "Kadinin ve onu tasiyan yedi basli, on boynuzlu canavarin sirrini ben sana açiklayayim.
\par 8 Gördügün canavar bir zamanlar vardi, ama simdi yok. Biraz sonra dipsiz derinliklerden çikacak ve yikima gidecek. Yeryüzünde yasayan ve dünya kurulali beri adlari yasam kitabina yazilmamis olanlar canavari görünce sasacaklar. Çünkü o bir zamanlar vardi, simdi yok, ama yine gelecek.
\par 9 "Bunu anlamak için bilgelik gerek. Yedi bas, kadinin üzerinde oturdugu yedi tepedir; ayni zamanda yedi kraldir.
\par 10 Bunlarin besi düstü, biri duruyor, ötekiyse henüz gelmedi. Gelince kisa süre kalmasi gerek.
\par 11 Yasamis, ama simdi yok olan canavarin kendisi sekizinci kraldir. O da yedilerden biridir ve yikima gitmektedir.
\par 12 Gördügün on boynuz henüz egemenlik sürmemis on kraldir; canavarla birlikte bir saat egemenlik sürmek üzere yetki alacaklar.
\par 13 Düsünce birligi içinde olan bu krallar güçlerini ve yetkilerini canavara verecekler.
\par 14 Kuzu'ya karsi savasacaklar, ama Kuzu onlari yenecek. Çünkü Kuzu, rablerin Rabbi, krallarin Krali'dir. O'nunla birlikte olanlar, çagrilmis, seçilmis ve O'na sadik kalmis olanlardir."
\par 15 Bundan sonra melek bana, "Su gördügün sular -fahisenin kenarinda oturdugu sular- halklar, toplumlar, uluslar ve dillerdir" dedi.
\par 16 "Gördügün canavarla on boynuz fahiseden nefret edecek, onu perisan edip çiplak birakacaklar. Etini yiyip kendisini ateste yakacaklar.
\par 17 Çünkü Tanri, amacini gerçeklestirme istegini onlarin yüregine koymustur. Öyle ki, Tanri'nin sözleri yerine gelinceye dek kralliklarini canavara devretmekte sözbirligi edecekler.
\par 18 Gördügün kadin dünya krallari üzerinde egemenlik süren büyük kenttir."

\chapter{18}

\par 1 Bundan sonra büyük yetkiye sahip baska bir melegin gökten indigini gördüm. Yeryüzü onun görkemiyle aydinlandi.
\par 2 Melek gür bir sesle bagirdi: "Yikildi! Büyük Babil yikildi! Cinlerin barinagi, Her kötü ruhun ugragi, Her murdar* ve igrenç kusun siginagi oldu.
\par 3 Çünkü bütün uluslar Azgin fuhsunun sarabindan içtiler. Dünya krallari da Onunla fuhus yaptilar. Dünya tüccarlari Onun asiri sefahatiyle zenginlestiler."
\par 4 Gökten baska bir ses isittim: "Ey halkim!" diyordu. "Onun günahlarina ortak olmamak, Ugradigi belalara ugramamak için çik oradan!
\par 5 Çünkü üst üste yigilan günahlari göge eristi, Ve Tanri onun suçlarini animsadi.
\par 6 Babil nasil davrandiysa, karsiligini ona aynen verin, Yaptiklarinin iki katini ödeyin. Hazirladigi kâsedeki içkinin Iki katini hazirlayip ona içirin.
\par 7 Kendini yücelttigi, sefahate verdigi oranda Istirap ve keder verin ona. Çünkü içinden diyor ki, `Tahtinda oturan bir kraliçeyim, dul degilim. Asla yas tutmayacagim!'
\par 8 Bu nedenle basina gelecek belalar Ölüm, yas ve kitlik- Bir gün içinde gelecek. Ates onu yiyip bitirecek. Çünkü onu yargilayan Rab Tanri güçlüdür.
\par 9 "Kendisiyle fuhus yapan ve sefahatte yasayan dünya krallari onu yakan atesin dumanini görünce onun için aglayip dövünecekler.
\par 10 Çektigi istiraptan dehsete düsecek, uzakta durup, `Vay basina koca kent, Vay basina güçlü kent Babil! Bir saat içinde cezani buldun' diyecekler.
\par 11 "Dünya tüccarlari onun için aglayip yas tutuyor. Çünkü mallarini satin alacak kimse yok artik.
\par 12 Altini, gümüsü, degerli taslari, incileri, ince keteni, ipegi, mor ve kirmizi kumaslari, her çesit kokulu agaci, fildisinden yapilmis her çesit esyayi, en pahali agaçlardan, tunç, demir ve mermerden yapilmis her çesit mali, tarçin ve kakule, buhur, güzel kokulu yag, günnük, sarap, zeytinyagi, ince un ve bugdayi, sigirlari, koyunlari, atlari, arabalari ve köleleri, insanlarin canini satin alacak kimse yok artik.
\par 14 "Diyecekler ki, `Caninin çektigi meyveler elinden gitti, Bütün degerli ve göz alici mallarin yok oldu. Insanlar bunlari bir daha göremeyecek.'
\par 15 Babil'de bu mallari satarak zenginlesen tüccarlar, kentin çektigi istiraptan dehsete düsecekler. Uzakta durup aglayacak, yas tutacaklar.
\par 16 "`Vay basina, vay!' diyecekler. `Ince keten, mor ve kirmizi kumas kusanmis, Altin, degerli tas ve incilerle süslenmis Koca kent!
\par 17 Onca büyük zenginlik Bir saat içinde yok oldu.' "Gemi kaptanlari, yolcular, tayfalar, denizde çalisanlarin hepsi, onu yakan atesin dumanini görünce uzakta durup, `Koca kent gibisi var mi?' diye feryat ettiler.
\par 19 Baslarina toprak döktüler, yas tutup aglayarak feryat ettiler: `Vay basina koca kent, vay! Denizde gemileri olanlarin hepsi Onun sayesinde, onun degerli mallariyla Zengin olmuslardi. Kent bir saat içinde viraneye döndü.'
\par 20 Ey gök, kutsallar, elçiler, peygamberler! Onun basina gelenlere sevinin! Çünkü Tanri onu yargilayip hakkinizi aldi."
\par 21 Sonra güçlü bir melek degirmen tasina benzer büyük bir tasi kaldirip denize atarak söyle dedi: "Koca kent Babil de Iste böyle siddetle atilacak Ve bir daha görülmeyecek.
\par 22 Artik sende lir çalanlarin, ezgi okuyanlarin, Kaval ve borazan çalanlarin sesi Hiç isitilmeyecek. Artik sende hiçbir el sanatinin ustasi bulunmayacak. Sende artik degirmen sesi duyulmayacak.
\par 23 Artik sende hiç kandil isigi parlamayacak. Sende artik gelin güvey sesi duyulmayacak. Senin tüccarlarin dünyanin büyükleriydi. Bütün uluslar senin büyücülügünle yoldan sapmisti.
\par 24 Peygamberlerin, kutsallarin Ve yeryüzünde bogazlanan herkesin kani Sende bulundu."

\chapter{19}

\par 1 Bundan sonra gökte büyük bir kalabaligin sesini andiran yüksek bir ses isittim. "Haleluya!" diyorlardi. "Kurtaris, yücelik ve güç Tanrimiz'a özgüdür.
\par 2 Çünkü O'nun yargilari dogru ve adildir. Yeryüzünü fuhsuyla yozlastiran Büyük fahiseyi yargilayip Kendi kullarinin kaninin öcünü aldi."
\par 3 Ikinci kez, "Haleluya! Onun dumani sonsuzlara dek tütecek" dediler.
\par 4 Yirmi dört ihtiyarla dört yaratik yere kapanip, "Amin! Haleluya!" diyerek tahtta oturan Tanri'ya tapindilar.
\par 5 Sonra tahttan bir ses yükseldi: "Ey Tanrimiz'in bütün kullari! Küçük büyük, O'ndan korkan hepiniz, O'nu övün!"
\par 6 Ardindan büyük bir kalabaligin, gürül gürül akan sularin, güçlü gök gürlemelerinin sesine benzer sesler isittim. "Haleluya!" diyorlardi. "Çünkü Her Seye Gücü Yeten Rab Tanrimiz Egemenlik sürüyor.
\par 7 Sevinelim, cosalim! O'nu yüceltelim! Çünkü Kuzu'nun dügünü basliyor, Gelini hazirlandi.
\par 8 Giymesi için ona temiz ve parlak Ince keten giysiler verildi." Ince keten kutsallarin adil islerini simgeler.
\par 9 Sonra melek bana, "Yaz!" dedi. "Ne mutlu Kuzu'nun dügün sölenine çagrilmis olanlara!" Ardindan ekledi: "Bunlar gerçek sözlerdir, Tanri'nin sözleridir."
\par 10 Ona tapinmak üzere ayaklarina kapandim. Ama o, "Sakin yapma!" dedi. "Ben de senin ve Isa'ya tanikligini sürdüren kardeslerin gibi bir Tanri kuluyum. Tanri'ya tap! Çünkü Isa'ya taniklik, peygamberlik ruhunun özüdür."
\par 11 Bundan sonra gögün açilmis oldugunu, beyaz bir atin orada durdugunu gördüm. Binicisinin adi Sadik ve Gerçek'tir. Adaletle yargilar, savasir.
\par 12 Gözleri alev alev yanan ates gibidir. Basinda çok sayida taç var. Üzerinde kendisinden baska kimsenin bilmedigi bir ad yazilidir.
\par 13 Kana batirilmis bir kaftan giymisti. Tanri'nin Sözü adiyla anilir.
\par 14 Beyaz, temiz, ince ketene bürünmüs olan gökteki ordular, beyaz atlara binmis O'nu izliyorlardi.
\par 15 Agzindan uluslari vuracak keskin bir kiliç uzaniyor. Onlari demir çomakla güdecek. Her Seye Gücü Yeten Tanri'nin atesli gazabinin sarabini üreten masarayi kendisi çigneyecek.
\par 16 Kaftaninin ve kalçasinin üzerinde su ad yaziliydi:
\par 17 Bundan sonra güneste duran bir melek gördüm. Gögün ortasinda uçan bütün kuslari yüksek sesle çagirdi: "Krallarin, komutanlarin, güçlü adamlarin, atlarla binicilerinin, özgür köle, küçük büyük, hepsinin etini yemek için toplanin, Tanri'nin büyük sölenine gelin!"
\par 19 Sonra canavari, dünya krallarini ve onlarin ordularini, ata binmis Olan'la O'nun ordusuna karsi savasmak üzere toplanmis gördüm.
\par 20 Canavarla onun önünde dogaüstü belirtiler gerçeklestiren sahte peygamber yakalandi. Sahte peygamber, canavarin isaretini alip heykeline tapanlari bu belirtilerle saptirmisti. Her ikisi de kükürtle yanan ates gölüne diri diri atildi.
\par 21 Geriye kalanlar, ata binmis Olan'in agzindan uzanan kiliçla öldürüldü. Bütün kuslar bunlarin etiyle doydu.

\chapter{20}

\par 1 Sonra bir melegin gökten indigini gördüm. Elinde dipsiz derinliklerin anahtari ve büyük bir zincir vardi.
\par 2 Melek ejderhayi -Iblis ya da Seytan denen o eski yilani- yakalayip bin yil için bagladi.
\par 3 Bin yil tamamlanincaya dek uluslari bir daha saptirmasin diye onu dipsiz derinliklere atti, oraya kapayip girisi mühürledi. Bin yil geçtikten sonra kisa bir süre için serbest birakilmasi gerekiyor.
\par 4 Bazi tahtlar ve bunlara oturanlari gördüm. Onlara yargilama yetkisi verilmisti. Isa'ya taniklik ve Tanri'nin sözü ugruna basi kesilenlerin canlarini da gördüm. Bunlar, canavara ve heykeline tapmamis, alinlarina ve ellerine onun isaretini almamis olanlardi. Hepsi dirilip Mesih'le birlikte bin yil egemenlik sürdüler.
\par 5 Ilk dirilis budur. Ölülerin geri kalani bin yil tamamlanmadan dirilmedi.
\par 6 Ilk dirilise dahil olanlar mutlu ve kutsaldir. Ikinci ölümün bunlarin üzerinde yetkisi yoktur. Onlar Tanri'nin ve Mesih'in kâhinleri* olacak, O'nunla birlikte bin yil egemenlik sürecekler.
\par 7 Bin yil tamamlaninca Seytan atildigi zindandan serbest birakilacak.
\par 8 Yeryüzünün dört bucagindaki uluslari -Gog'la Magog'u- saptirmak, savas için bir araya toplamak üzere zindandan çikacak. Toplananlarin sayisi deniz kumu kadar çoktur.
\par 9 Yeryüzünün dört bir yanindan gelerek kutsallarin ordugahini ve sevilen kenti kusattilar. Ama gökten ates yagdi, onlari yakip yok etti.
\par 10 Onlari saptiran Iblis ise canavarla sahte peygamberin de içinde bulundugu ates ve kükürt gölüne atildi. Gece gündüz, sonsuzlara dek iskence çekeceklerdir.
\par 11 Sonra büyük, beyaz bir taht ve tahtta oturani gördüm. Yerle gök önünden kaçtilar, yok olup gittiler.
\par 12 Tahtin önünde duran küçük büyük, ölüleri gördüm. Sonra kitaplar açildi. Yasam kitabi denen baska bir kitap daha açildi. Ölüler kitaplarda yazilanlara bakilarak yaptiklarina göre yargilandi.
\par 13 Deniz kendisinde olan ölüleri, ölüm ve ölüler diyari da kendilerinde olan ölüleri teslim ettiler. Her biri yaptiklarina göre yargilandi.
\par 14 Ölüm ve ölüler diyari ates gölüne atildi. Iste bu ates gölü ikinci ölümdür.
\par 15 Adi yasam kitabina yazilmamis olanlar ates gölüne atildi.

\chapter{21}

\par 1 Bundan sonra yeni bir gökle yeni bir yeryüzü gördüm. Çünkü önceki gökle yeryüzü ortadan kalkmisti. Deniz de yoktu artik.
\par 2 Kutsal kentin, yeni Yerusalim'in gökten, Tanri'nin yanindan indigini gördüm. Güveyi için hazirlanmis süslü bir gelin gibiydi.
\par 3 Tahttan yükselen gür bir sesin söyle dedigini isittim: "Iste, Tanri'nin konutu insanlarin arasindadir. Tanri onlarin arasinda yasayacak. Onlar O'nun halki olacaklar, Tanri'nin kendisi de onlarin arasinda bulunacak.
\par 4 Onlarin gözlerinden bütün yaslari silecek. Artik ölüm olmayacak. Artik ne yas, ne aglayis, ne de istirap olacak. Çünkü önceki düzen ortadan kalkti."
\par 5 Tahtta oturan, "Iste her seyi yeniliyorum" dedi. Sonra, "Yaz!" diye ekledi, "Çünkü bu sözler güvenilir ve gerçektir."
\par 6 Bana, "Tamam!" dedi, "Alfa* ve Omega*, baslangiç ve son Ben'im. Susayana yasam suyunun pinarindan karsiliksiz su verecegim.
\par 7 Galip gelen bunlari miras alacak. Ben onun Tanrisi olacagim, o da bana ogul olacak.
\par 8 Ama korkak, imansiz, igrenç, adam öldüren, fuhus yapan, büyücü, putperest ve bütün yalancilara gelince, onlarin yeri, kükürtle yanan ates gölüdür. Ikinci ölüm budur."
\par 9 Son yedi belayla dolu yedi tasi tasiyan yedi melekten biri gelip benimle konustu. "Gel!" dedi, "Kuzu'ya es olacak gelini sana göstereyim."
\par 10 Sonra melek beni Ruh'un yönetiminde büyük, yüksek bir daga götürdü. Oradan bana gökten, Tanri'nin yanindan inen ve O'nun görkemiyle isildayan kutsal kenti, Yerusalim'i gösterdi. Kentin isiltisi çok degerli bir tasin, billur gibi parildayan yesim tasinin isiltisina benziyordu.
\par 12 Büyük ve yüksek surlari ve on iki kapisi vardi. Kapilari on iki melek bekliyordu. Kapilarin üzerine Israilogullari'nin on iki oymaginin adlari yazilmisti.
\par 13 Doguda üç kapi, kuzeyde üç kapi, güneyde üç kapi, batida üç kapi vardi.
\par 14 Kenti çevreleyen surlarin on iki temel tasi bulunuyordu. Bunlarin üzerinde Kuzu'nun on iki elçisinin adlari yaziliydi.
\par 15 Benimle konusan melegin elinde kenti ve kent kapilariyla surlari ölçmek için altin bir ölçü kamisi vardi.
\par 16 Kent kare biçimindeydi, uzunlugu enine esitti. Melek kenti kamisla ölçtü, her bir yani 12 000 ok atimi geldi. Uzunlugu, eni ve yüksekligi birbirine esitti.
\par 17 Melek surlari da ölçtü. Kullandigi insan ölçüsüne göre 144 arsindi.
\par 18 Surlar yesimden yapilmisti. Kent ise, cam durulugunda saf altindandi.
\par 19 Kent surlarinin temelleri her tür degerli tasla bezenmisti. Birinci temel tasi yesim, ikincisi laciverttasi, üçüncüsü akik, dördüncüsü zümrüt, besincisi damarli akik, altincisi kirmizi akik, yedincisi sari yakut, sekizincisi beril, dokuzuncusu topaz, onuncusu sarica zümrüt, onbirincisi gökyakut, onikincisi ametistti.
\par 21 On iki kapi on iki inciydi; kapilarin her biri birer inciden yapilmisti. Kentin anayolu cam saydamliginda saf altindandi.
\par 22 Kentte tapinak görmedim. Çünkü Her Seye Gücü Yeten Rab Tanri ve Kuzu, kentin tapinagidir.
\par 23 Aydinlanmak için kentin günes ya da aya gereksinimi yoktur. Çünkü Tanri'nin görkemi onu aydinlatiyor. Kuzu da onun çirasidir.
\par 24 Uluslar kentin isiginda yürüyecekler. Dünya krallari servetlerini oraya getirecekler.
\par 25 Kentin kapilari gündüz hiç kapanmayacak, orada gece olmayacak.
\par 26 Uluslarin görkemi ve zenginligi oraya tasinacak.
\par 27 Oraya murdar* hiçbir sey, igrenç ve aldatici isler yapan hiç kimse asla girmeyecek; yalniz adlari Kuzu'nun yasam kitabinda yazili olanlar girecek.

\chapter{22}

\par 1 Melek bana Tanri'nin ve Kuzu'nun tahtindan çikan billur gibi berrak yasam suyu irmagini gösterdi.
\par 2 Kentin anayolunun ortasinda akan irmagin iki yaninda on iki çesit meyve üreten ve her ay meyvesini veren yasam agaci bulunuyordu. Agacin yapraklari uluslara sifa vermek içindir.
\par 3 Artik hiçbir lanet kalmayacak. Tanri'nin ve Kuzu'nun tahti kentin içinde olacak, kullari O'na tapinacak.
\par 4 O'nun yüzünü görecek, alinlarinda O'nun adini tasiyacaklar.
\par 5 Artik gece olmayacak. Çira isigina da günes isigina da gereksinmeleri olmayacak. Çünkü Rab Tanri onlara isik verecek ve sonsuzlara dek egemenlik sürecekler.
\par 6 Melek bana, "Bu sözler güvenilir ve gerçektir" dedi. "Peygamberlerin ruhlarinin Tanrisi olan Rab, yakin zamanda olmasi gereken olaylari kullarina göstermek için melegini gönderdi."
\par 7 "Iste tez geliyorum! Bu kitaptaki peygamberlik sözlerine uyana ne mutlu!"
\par 8 Bunlari isiten ve gören ben Yuhanna'yim. Isitip gördügümde bunlari bana gösteren melege tapmak için ayaklarina kapandim.
\par 9 Ama o bana, "Sakin yapma!" dedi, "Ben senin, peygamber kardeslerin ve bu kitabin sözlerine uyanlar gibi bir Tanri kuluyum. Tanri'ya tap!"
\par 10 Sonra bana, "Bu kitabin peygamberlik sözlerini mühürleme" dedi, "Çünkü beklenen zaman yakindir.
\par 11 Kötülük yapan, yine kötülük yapsin. Kirli olan, kirli islerini sürdürsün. Dogru olan, yine dogruyu yapsin. Kutsal olan kutsal kalsin."
\par 12 "Iste tez geliyorum! Verecegim ödüller yanimdadir. Herkese yaptiginin karsiligini verecegim.
\par 13 Alfa* ve Omega*, birinci ve sonuncu, baslangiç ve son Ben'im.
\par 14 "Kaftanlarini yikayan, böylelikle yasam agacindan yemeye hak kazanarak kapilardan geçip kente girenlere ne mutlu!
\par 15 Köpekler, büyücüler, fuhus yapanlar, adam öldürenler, putperestler, yalani sevip hile yapanlarin hepsi disarida kalacaklar.
\par 16 "Ben Isa, kiliselerle* ilgili bu tanikligi sizlere iletsin diye melegimi gönderdim. Davut'un kökü ve soyu Ben'im, parlak sabah yildizi Ben'im."
\par 17 Ruh ve Gelin, "Gel!" diyorlar. Isiten, "Gel!" desin. Susayan gelsin. Dileyen, yasam suyundan karsiliksiz alsin.
\par 18 Bu kitaptaki peygamberlik sözlerini duyan herkesi uyariyorum! Her kim bu sözlere bir sey katarsa, Tanri da bu kitapta yazili belalari ona katacaktir.
\par 19 Her kim bu peygamberlik kitabinin sözlerinden bir sey çikarirsa, Tanri da bu kitapta yazili yasam agacindan ve kutsal kentten ona düsen payi çikaracaktir.
\par 20 Bunlara taniklik eden, "Evet, tez geliyorum!" diyor. Amin! Gel, ya Rab Isa!
\par 21 Rab Isa'nin lütfu kutsallarla birlikte olsun! Amin.


\end{document}