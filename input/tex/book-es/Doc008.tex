\chapter{Documento 8. El Espíritu Infinito}
\par
%\textsuperscript{(90.1)}
\textsuperscript{8:0.1} ALLÁ por la eternidad, cuando el «primer» pensamiento infinito y absoluto del Padre Universal encuentra en el Hijo Eterno un verbo tan perfecto y adecuado para su expresión divina, se produce a continuación tanto en el Dios-Pensamiento como en el Dios-Verbo el deseo supremo de tener un agente universal e infinito que los exprese mutuamente y actúe de manera combinada.

\par
%\textsuperscript{(90.2)}
\textsuperscript{8:0.2} En los albores de la eternidad, tanto el Padre como el Hijo se vuelven infinitamente conscientes de su mutua interdependencia, de su unidad eterna y absoluta; por consiguiente, establecen una alianza infinita y perpetua de asociación divina. Este acuerdo sin fin se efectúa para llevar a cabo sus conceptos unidos a lo largo de todo el círculo de la eternidad; y desde este acontecimiento sucedido en la eternidad, el Padre y el Hijo continúan con esta unión divina.

\par
%\textsuperscript{(90.3)}
\textsuperscript{8:0.3} Nos encontramos ahora frente a frente con el origen en la eternidad del Espíritu Infinito, la Tercera Persona de la Deidad. En el mismo instante en que Dios Padre y Dios Hijo conciben conjuntamente una acción idéntica e infinita ---la ejecución de un plan-pensamiento absoluto--- en ese mismo momento el Espíritu Infinito surge en toda su plenitud a la existencia.

\par
%\textsuperscript{(90.4)}
\textsuperscript{8:0.4} Al exponer de esta manera el orden del origen de las Deidades, lo hago así solamente para permitiros pensar en sus relaciones. En realidad, las tres existen desde la eternidad; son existenciales. No tienen ni principio ni fin en el tiempo; están coordinadas y son supremas, últimas, absolutas e infinitas. Existen, han existido siempre y siempre existirán. Son tres personas claramente individualizadas pero eternamente asociadas: Dios Padre, Dios Hijo y Dios Espíritu.

\section*{1. El Dios de acción}
\par
%\textsuperscript{(90.5)}
\textsuperscript{8:1.1} En la eternidad del pasado, con la personalización del Espíritu Infinito el ciclo divino de la personalidad se vuelve perfecto y completo. El Dios de Acción existe, y el inmenso escenario del espacio está preparado para el prodigioso drama de la creación ---para la aventura universal--- para el panorama divino de las eras eternas.

\par
%\textsuperscript{(90.6)}
\textsuperscript{8:1.2} El primer acto del Espíritu Infinito consiste en examinar y reconocer a sus padres divinos, el Padre-Padre y el Hijo-Madre. Él, el Espíritu, los identifica a los dos sin reserva. Es plenamente consciente de sus personalidades distintas y de sus atributos infinitos, así como de su naturaleza combinada y de su acción unida. Luego, voluntariamente, con una buena disposición trascendente y una espontaneidad inspiradora, la Tercera Persona de la Deidad, a pesar de su igualdad con la Primera y Segunda Personas, promete una lealtad eterna a Dios Padre y reconoce su dependencia perpetua de Dios Hijo.

\par
%\textsuperscript{(90.7)}
\textsuperscript{8:1.3} El ciclo de la eternidad queda establecido; es inherente a la naturaleza de esta operación, al reconocimiento mutuo de la independencia de la personalidad de cada una de las Deidades, y a la unión ejecutiva de las tres. La Trinidad del Paraíso ya existe. El escenario del espacio universal está preparado para el múltiple panorama sin fin donde el propósito del Padre Universal se despliega de forma creativa a través de la personalidad del Hijo Eterno y gracias a la actividad ejecutiva del Dios de Acción, el agente que ejecuta las acciones, en la realidad, de la asociación creadora Padre-Hijo.

\par
%\textsuperscript{(91.1)}
\textsuperscript{8:1.4} El Dios de Acción actúa y las bóvedas inertes del espacio se ponen en movimiento. Mil millones de esferas perfectas surgen de inmediato a la existencia. Antes de este momento hipotético de la eternidad, las energías espaciales inherentes al Paraíso ya existen y están potencialmente operativas, pero aún no se han manifestado; la gravedad física tampoco se puede medir si no es mediante la reacción de las realidades materiales a su atracción incesante. No existe ningún universo material en este (supuesto) momento eternamente lejano, pero en el mismo instante en que se materializan mil millones de mundos, se pone de manifiesto una gravedad suficiente y adecuada para mantenerlos bajo la atracción perpetua del Paraíso.

\par
%\textsuperscript{(91.2)}
\textsuperscript{8:1.5} Ahora centellea por toda la creación de los Dioses la segunda forma de energía, y este espíritu que mana a raudales es atraído instantáneamente por la gravedad espiritual del Hijo Eterno. Y así, el universo dos veces abrazado por la gravedad es tocado por la energía de la infinidad y sumergido en el espíritu de la divinidad. De esta forma, el terreno de la vida está preparado para la conciencia de la mente, puesta de manifiesto en los circuitos de inteligencia asociados del Espíritu Infinito.

\par
%\textsuperscript{(91.3)}
\textsuperscript{8:1.6} Sobre estas semillas de existencia potencial, difundidas por toda la creación central de los Dioses, el Padre actúa, y la personalidad de las criaturas aparece. Luego, la presencia de las Deidades del Paraíso ocupa todo el espacio organizado y empieza a atraer eficazmente a todas las cosas y a todos los seres hacia el Paraíso.

\par
%\textsuperscript{(91.4)}
\textsuperscript{8:1.7} El Espíritu Infinito se eterniza al mismo tiempo que nacen los mundos de Havona, siendo este universo central creado por él, con él y en él, de conformidad con los conceptos combinados y las voluntades unidas del Padre y del Hijo. La Tercera Persona se deifica mediante este acto mismo de creación conjunta, convirtiéndose así para siempre en el Creador Conjunto.

\par
%\textsuperscript{(91.5)}
\textsuperscript{8:1.8} Son los tiempos grandiosos e impresionantes de la expansión creadora del Padre y del Hijo por medio de, y en la acción de, su asociado conjunto y ejecutivo exclusivo, la Fuente-Centro Tercera. No existe ningún archivo de estos tiempos agitados. Sólo disponemos de las escasas revelaciones del Espíritu Infinito para justificar estas poderosas operaciones, y él se limita a confirmar el hecho de que el universo central y todo lo relacionado con éste se eternizó al mismo tiempo que él conseguía la personalidad y la existencia consciente.

\par
%\textsuperscript{(91.6)}
\textsuperscript{8:1.9} En resumen, el Espíritu Infinito declara que, puesto que él es eterno, el universo central también lo es. Éste es el punto de partida tradicional de la historia del universo de universos. No se sabe absolutamente nada, y no existen archivos, respecto a cualquier acontecimiento o actividad anterior a esta prodigiosa erupción de energía creativa y de sabiduría administrativa que cristalizó el inmenso universo que existe y que funciona tan exquisitamente en el centro de todas las cosas. Más allá de este acontecimiento se extienden las operaciones impenetrables de la eternidad y las profundidades de la infinidad ---misterio absoluto.

\par
%\textsuperscript{(91.7)}
\textsuperscript{8:1.10} Describimos de esta forma el origen secuencial de la Fuente-Centro Tercera como una condescendencia interpretativa hacia la mente de las criaturas mortales, atada al tiempo y condicionada por el espacio. La mente del hombre necesita tener un punto de partida para poder imaginarse la historia del universo, y se me ha ordenado que proporcione esta técnica para que pueda acceder al concepto histórico de la eternidad. Para la mente material, la coherencia exige que exista una Causa Primera; por eso consideramos como un postulado que el Padre Universal es la Fuente Primera y el Centro Absoluto de toda la creación, al mismo tiempo que enseñamos a la mente de todas las criaturas que el Hijo y el Espíritu son coeternos con el Padre en todas las fases de la historia universal y en todos los ámbitos de la actividad creadora. Y hacemos esto sin descuidar de ninguna manera la realidad y la eternidad de la Isla del Paraíso y de los Absolutos Incalificado, Universal y de la Deidad.

\par
%\textsuperscript{(92.1)}
\textsuperscript{8:1.11} Ya es suficiente con que la mente material de los hijos del tiempo sea capaz de concebir al Padre en la eternidad. Sabemos que todo niño puede relacionarse mejor con la realidad dominando primero las relaciones de la situación padre-hijo, y ampliando después este concepto hasta abarcar a la familia como un todo. Posteriormente, la mente en desarrollo del niño será capaz de ajustarse al concepto de las relaciones familiares, de las relaciones de la comunidad, la raza y el mundo, y luego a las del universo, del superuniverso e incluso del universo de universos.

\section*{2. La naturaleza del Espíritu Infinito}
\par
%\textsuperscript{(92.2)}
\textsuperscript{8:2.1} El Creador Conjunto existe desde la eternidad y es uno, de manera total y sin restricción, con el Padre Universal y el Hijo Eterno. El Espíritu Infinito refleja a la perfección no solamente la naturaleza del Padre Paradisiaco, sino también la del Hijo Original.

\par
%\textsuperscript{(92.3)}
\textsuperscript{8:2.2} A la Fuente-Centro Tercera se le conoce por numerosos títulos: el Espíritu Universal, el Guía Supremo, el Creador Conjunto, el Ejecutivo Divino, la Mente Infinita, el Espíritu de los Espíritus, el Espíritu Madre Paradisiaco, el Actor Conjunto, el Coordinador Final, el Espíritu Omnipresente, la Inteligencia Absoluta, la Acción Divina; y en Urantia se le confunde a veces con la mente cósmica.

\par
%\textsuperscript{(92.4)}
\textsuperscript{8:2.3} Es totalmente adecuado denominar Espíritu Infinito a la Tercera Persona de la Deidad, porque Dios es espíritu\footnote{\textit{Dios es espíritu}: Jn 4:24.}. Pero las criaturas materiales, que tienden a cometer el error de considerar la materia como la realidad fundamental, y la mente, así como el espíritu, como postulados enraizados en la materia, comprenderían mejor a la Fuente-Centro Tercera si lo llamaran la Realidad Infinita, el Organizador Universal o el Coordinador de la Personalidad.

\par
%\textsuperscript{(92.5)}
\textsuperscript{8:2.4} El Espíritu Infinito, como revelación universal de la divinidad, es insondable y está totalmente fuera de la comprensión humana. Para percibir la absolutidad del Espíritu, sólo necesitáis contemplar la infinidad del Padre Universal y sentiros asombrados por la eternidad del Hijo Original.

\par
%\textsuperscript{(92.6)}
\textsuperscript{8:2.5} Hay misterio en verdad en la persona del Espíritu Infinito, pero no tanto como en el Padre y el Hijo. De todos los aspectos de la naturaleza del Padre, es su infinidad la que el Creador Conjunto revela de manera más notable. Aunque el universo maestro se extienda finalmente hasta la infinidad, la presencia espiritual, el control energético y el potencial mental del Actor Conjunto serán adecuados para satisfacer las exigencias de esa creación ilimitada.

\par
%\textsuperscript{(92.7)}
\textsuperscript{8:2.6} Aunque el Espíritu Infinito comparte en todos los sentidos la perfección, la rectitud y el amor del Padre Universal, siente inclinación hacia los atributos de misericordia del Hijo Eterno, convirtiéndose así en el ministro de la misericordia de las Deidades del Paraíso para el gran universo. Para siempre jamás ---de manera universal y eterna--- el Espíritu es un ministro de misericordia, porque al igual que los Hijos divinos revelan el amor de Dios, el Espíritu divino describe la misericordia de Dios.

\par
%\textsuperscript{(93.1)}
\textsuperscript{8:2.7} No es posible que el Espíritu pueda tener más bondad que el Padre, puesto que toda bondad tiene su origen en el Padre, pero esta bondad la podemos comprender mejor en los actos del Espíritu. La fidelidad del Padre y la constancia del Hijo se hacen muy reales para los seres espirituales y las criaturas materiales de las esferas gracias al ministerio amoroso y al servicio incesante de las personalidades del Espíritu Infinito.

\par
%\textsuperscript{(93.2)}
\textsuperscript{8:2.8} El Creador Conjunto hereda toda la belleza de pensamiento y todo el carácter veraz del Padre. Estas características sublimes de la divinidad están coordinadas en los niveles casi supremos de la mente cósmica, la cual está subordinada a la sabiduría eterna e infinita de la mente incondicionada e ilimitada de la Fuente-Centro Tercera.

\section*{3. Las relaciones del Espíritu con el Padre y el Hijo}
\par
%\textsuperscript{(93.3)}
\textsuperscript{8:3.1} Al igual que el Hijo Eterno es la expresión verbal del «primer» pensamiento absoluto e infinito del Padre Universal, el Actor Conjunto es la ejecución perfecta del «primer» concepto, o plan creador completo, para efectuar la acción combinada de la asociación entre las personalidades del Padre y del Hijo, compuesta por la unión absoluta entre el pensamiento y el verbo. La Fuente-Centro Tercera se eterniza al mismo tiempo que la creación central, hecha por decreto, y sólo esta creación central tiene una existencia eterna entre los universos.

\par
%\textsuperscript{(93.4)}
\textsuperscript{8:3.2} Desde la personalización de la Fuente Tercera, la Fuente Primera ya no participa personalmente en la creación del universo. El Padre Universal delega todo aquello que es posible a su Hijo Eterno; de igual manera, el Hijo Eterno deposita toda la autoridad y todo el poder posibles en el Creador Conjunto.

\par
%\textsuperscript{(93.5)}
\textsuperscript{8:3.3} El Hijo Eterno y el Creador Conjunto han planeado y formado, como asociados y por medio de sus personalidades coordinadas, todos los universos que han sido traídos a la existencia después de Havona. En todas las creaciones posteriores, el Espíritu mantiene con el Hijo la misma relación personal que el Hijo mantiene con el Padre en la primera creación central.

\par
%\textsuperscript{(93.6)}
\textsuperscript{8:3.4} Un Hijo Creador del Hijo Eterno y un Espíritu Creativo del Espíritu Infinito os han creado, a vosotros y a vuestro universo; y aunque el Padre sostiene fielmente aquello que han organizado, a este Hijo Universal y a este Espíritu Universal les incumbe fomentar y sostener su obra, así como aportar su ministerio a las criaturas creadas por ellos mismos.

\par
%\textsuperscript{(93.7)}
\textsuperscript{8:3.5} El Espíritu Infinito es el agente eficaz del Padre amoroso y del Hijo misericordioso que ejecuta su proyecto conjunto de atraer hacia ellos a todas las almas que aman la verdad en todos los mundos del tiempo y del espacio. En el mismo instante en que el Hijo Eterno aceptó el plan de su Padre consistente en que las criaturas de los universos alcanzaran la perfección, en el momento en que el proyecto de ascensión se convirtió en un plan del Padre y del Hijo, en ese instante el Espíritu Infinito se convirtió en el administrador conjunto del Padre y del Hijo para llevar a cabo su propósito eterno y unido. Al hacer esto, el Espíritu Infinito prometió al Padre y al Hijo todos los recursos de su presencia divina y de sus personalidades espirituales; lo ha dedicado \textit{todo} al prodigioso plan de elevar a las criaturas volitivas sobrevivientes a las alturas divinas de la perfección paradisiaca.

\par
%\textsuperscript{(93.8)}
\textsuperscript{8:3.6} El Espíritu Infinito es una revelación completa, exclusiva y universal del Padre Universal y de su Hijo Eterno. Todo conocimiento relacionado con la asociación Padre-Hijo ha de adquirirse a través del Espíritu Infinito, el representante conjunto de la unión divina entre el pensamiento y el verbo.

\par
%\textsuperscript{(93.9)}
\textsuperscript{8:3.7} El Hijo Eterno es el único camino de acceso al Padre Universal\footnote{\textit{El Hijo: el único camino}: Mt 11:27; Lc 10:22; Jn 1:18; 6:44-46; 14:6-11,20.}, y el Espíritu Infinito es el único medio de alcanzar al Hijo Eterno. Los seres ascendentes del tiempo sólo pueden descubrir al Hijo por medio del paciente ministerio del Espíritu.

\par
%\textsuperscript{(94.1)}
\textsuperscript{8:3.8} El Espíritu Infinito es la primera de las Deidades del Paraíso que alcanzan los peregrinos ascendentes en el centro de todas las cosas. La Tercera Persona envuelve a la Segunda y a la Primera Personas, y por eso siempre ha de ser reconocida primero por todos los candidatos que desean ser presentados al Hijo y a su Padre.

\par
%\textsuperscript{(94.2)}
\textsuperscript{8:3.9} El Espíritu representa igualmente y sirve de forma similar al Padre y al Hijo de otras muchas maneras.

\section*{4. El espíritu del ministerio divino}
\par
%\textsuperscript{(94.3)}
\textsuperscript{8:4.1} Paralelamente al universo físico donde la gravedad del Paraíso mantiene unidas todas las cosas, existe el universo espiritual donde la palabra del Hijo interpreta el pensamiento de Dios, y cuando este verbo «se hace carne»\footnote{\textit{El Verbo hecho carne}: Jn 1:14.}, demuestra la misericordia amorosa de la naturaleza combinada de los Creadores asociados. Pero en toda esta creación material y espiritual, y a través de ella, existe un inmenso escenario en el que el Espíritu Infinito y su progenitura espiritual dan a conocer la misericordia, la paciencia y el afecto perpetuo combinados de los padres divinos hacia los hijos inteligentes que han concebido y creado en cooperación. La esencia del carácter divino del Espíritu es servir perpetuamente a la mente. Y toda la descendencia espiritual del Actor Conjunto participa en este deseo de ofrecer su ministerio, en este impulso divino a servir.

\par
%\textsuperscript{(94.4)}
\textsuperscript{8:4.2} Dios es amor\footnote{\textit{Dios es amor}: 1 Jn 4:8,16.}, el Hijo es misericordia, el Espíritu es ministerio ---el ministerio del amor divino y de la misericordia sin fin para toda la creación inteligente. El Espíritu es la personificación del amor del Padre y de la misericordia del Hijo; en él están los dos eternamente unidos para el servicio universal. El Espíritu es el \textit{amor aplicado} para la creación compuesta de criaturas, el amor combinado del Padre y del Hijo.

\par
%\textsuperscript{(94.5)}
\textsuperscript{8:4.3} En Urantia, el Espíritu Infinito es conocido como una influencia omnipresente, una presencia universal, pero en Havona lo conoceréis como una presencia personal de verdadero servicio. Aquí, el ministerio del Espíritu del Paraíso es el modelo ejemplar e inspirador para cada uno de sus Espíritus coordinados y de sus personalidades subordinadas que sirven a los seres creados en los mundos del tiempo y del espacio. En este universo divino, el Espíritu Infinito participó plenamente en las siete apariciones trascendentales del Hijo Eterno; también participó con el Hijo Miguel original en las siete donaciones sobre los circuitos de Havona, convirtiéndose así en el ministro espiritual compasivo y comprensivo para cada peregrino del tiempo que atraviesa estos círculos perfectos de las alturas.

\par
%\textsuperscript{(94.6)}
\textsuperscript{8:4.4} Cuando un Hijo de Dios Creador acepta la responsabilidad de crear un universo local en proyecto, las personalidades del Espíritu Infinito se comprometen a ser los ministros incansables de este Hijo Miguel cuando emprende su misión de aventura creadora. Al Espíritu Infinito lo encontramos especialmente en las personas de las Hijas Creativas, los Espíritus Madres de los universos locales, y lo encontramos dedicado a la tarea de fomentar la ascensión de las criaturas materiales hacia unos niveles cada vez más elevados de consecución espiritual. Todo este trabajo de servicio hacia las criaturas es efectuado en perfecta armonía con los objetivos, y en estrecha asociación con las personalidades, de los Hijos Creadores de estos universos locales.

\par
%\textsuperscript{(94.7)}
\textsuperscript{8:4.5} Al igual que los Hijos de Dios se ocupan de la gigantesca tarea de revelar a un universo la personalidad amorosa del Padre, el Espíritu Infinito se dedica al ministerio interminable de revelar el amor combinado del Padre y del Hijo a las mentes individuales de todos los hijos de cada universo. En estas creaciones locales, el Espíritu no desciende hasta las razas materiales en la similitud de la carne mortal como lo hacen algunos Hijos de Dios, sino que el Espíritu Infinito y sus Espíritus coordinados rebajan su categoría, experimentan alegremente una serie asombrosa de atenuaciones de su divinidad, hasta que aparecen como ángeles para estar a vuestro lado y guiaros por los humildes caminos de la existencia terrestre.

\par
%\textsuperscript{(95.1)}
\textsuperscript{8:4.6} Mediante esta misma serie decreciente, el Espíritu Infinito se acerca realmente mucho, como persona, a cada ser de las esferas de origen animal. Y el Espíritu hace todo esto sin invalidar en lo más mínimo su existencia como Tercera Persona de la Deidad en el centro de todas las cosas\footnote{\textit{Atracción espiritual}: Heb 7:19; Stg 4:8.}.

\par
%\textsuperscript{(95.2)}
\textsuperscript{8:4.7} El Creador Conjunto es verdaderamente y para siempre la gran personalidad ministrante, el ministro universal de la misericordia. Para comprender el ministerio del Espíritu, reflexionad sobre la verdad de que él es el retrato combinado del amor interminable del Padre y de la misericordia eterna del Hijo. Sin embargo, el ministerio del Espíritu no está únicamente limitado a representar al Hijo Eterno y al Padre Universal. El Espíritu Infinito posee también el poder de servir a las criaturas del universo en su propio nombre y derecho; la Tercera Persona tiene una dignidad divina y dispensa también el ministerio universal de la misericordia por su propia cuenta.

\par
%\textsuperscript{(95.3)}
\textsuperscript{8:4.8} A medida que el hombre aprenda más cosas sobre el ministerio amoroso e infatigable de las órdenes inferiores de la familia de criaturas de este Espíritu Infinito, admirará y adorará más la naturaleza trascendente y el carácter incomparable de esta Acción combinada del Padre Universal y del Hijo Eterno. En verdad, este Espíritu es «los ojos del Señor que están siempre sobre los justos»\footnote{\textit{Los ojos del Señor están sobre los justos}: Sal 34:15; 1 P 3:12.} y «los oídos divinos que siempre están abiertos a sus oraciones»\footnote{\textit{Los oídos divinos siempre están abiertos}: Sal 34:15; 1 P 3:12.}.

\section*{5. La presencia de Dios}
\par
%\textsuperscript{(95.4)}
\textsuperscript{8:5.1} El atributo sobresaliente del Espíritu Infinito es su omnipresencia. En todo el universo de universos está presente en todas partes este espíritu que lo impregna todo, y que es tan semejante a la presencia de una mente universal y divina. Tanto la Segunda Persona como la Tercera Persona de la Deidad están representadas en todos los mundos por sus espíritus siempre presentes.

\par
%\textsuperscript{(95.5)}
\textsuperscript{8:5.2} El Padre es \textit{infinito} y, por consiguiente, sólo está limitado por su volición. En la concesión de los Ajustadores y en la incorporación de la personalidad a su circuito, el Padre actúa solo, pero en el contacto de las fuerzas espirituales con los seres inteligentes utiliza los espíritus y las personalidades del Hijo Eterno y del Espíritu Infinito. Está espiritualmente presente a voluntad, y de igual manera, con el Hijo o con el Actor Conjunto; está presente \textit{con} el Hijo y \textit{en} el Espíritu. El Padre está presente con toda seguridad en todas partes, y nosotros discernimos su presencia por y a través de todas estas fuerzas, influencias y presencias diversas pero asociadas.

\par
%\textsuperscript{(95.6)}
\textsuperscript{8:5.3} En vuestras escrituras sagradas, el término \textit{Espíritu de Dios}\footnote{\textit{Espíritu de Dios (Espíritu Santo)}: Gn 1:2; Ex 31:3; 35:31; Job 33:4; Sal 51:10-11; 139:7; Pr 1:23; Is 44:3; 59:21; 61:1; 63:10-11; Lc 4:1; 11:13; Jn 1:33; 3:5; 2 Ti 1:14.} parece haber sido empleado para designar indistintamente tanto al Espíritu Infinito del Paraíso como al Espíritu Creativo de vuestro universo local. El Espíritu Santo es el circuito espiritual de esta Hija Creativa del Espíritu Infinito del Paraíso. El Espíritu Santo es un circuito autóctono de cada universo local y está limitado al ámbito espiritual de esa creación; pero el Espíritu Infinito es omnipresente.

\par
%\textsuperscript{(95.7)}
\textsuperscript{8:5.4} Existen muchas influencias espirituales, y todas funcionan como \textit{una sola.} Incluso el trabajo de los Ajustadores del Pensamiento, aunque es independiente de todas las otras influencias, coincide invariablemente con el ministerio espiritual de las influencias combinadas del Espíritu Infinito y del Espíritu Madre de un universo local. Estas presencias espirituales, tal como funcionan en la vida de los urantianos, no se pueden separar. Actúan en vuestra mente y sobre vuestra alma como un solo espíritu, a pesar de sus orígenes diversos. Y a medida que experimentáis este ministerio espiritual unido, para vosotros se convierte en la influencia del Supremo, «que siempre es capaz de evitar que falléis y de presentaros irreprochables ante vuestro Padre en las alturas»\footnote{\textit{Capaz de sosteneros y presentaros}: Jud 1:24.}.

\par
%\textsuperscript{(96.1)}
\textsuperscript{8:5.5} Recordad siempre que el Espíritu Infinito es el Actor \textit{Conjunto;} tanto el Padre como el Hijo actúan en él y a través de él; está presente no sólo como él mismo, sino también como Padre y como Hijo, y como Padre-Hijo. En reconocimiento de este hecho y por muchas razones adicionales, a la presencia espiritual del Espíritu Infinito se la califica a menudo de «el espíritu de Dios»\footnote{\textit{Espíritu de Dios (Espíritu Santo)}: Gn 1:2; Ex 31:3; 35:31; Job 33:4; Sal 51:10-11; 139:7; Pr 1:23; Is 44:3; 59:21; 61:1; 63:10-11; Lc 4:1; 11:13; Jn 1:33; 3:5; 2 Ti 1:14.}.

\par
%\textsuperscript{(96.2)}
\textsuperscript{8:5.6} También sería coherente referirse a la coordinación de todo el ministerio espiritual como el espíritu de Dios\footnote{\textit{Dios es un espíritu}: Jn 4:24.}, porque esta coordinación es realmente la unión de los espíritus de Dios Padre, Dios Hijo, Dios Espíritu y Dios Séptuple ---el espíritu mismo de Dios Supremo.

\section*{6. La personalidad del Espíritu Infinito}
\par
%\textsuperscript{(96.3)}
\textsuperscript{8:6.1} No permitáis que la donación tan difundida y la extensa distribución de la Fuente-Centro Tercera oscurezcan o disminuyan de otra manera el hecho de su personalidad. El Espíritu Infinito es una presencia universal, una acción eterna, un poder cósmico, una influencia sagrada y una mente universal; es todo esto e infinitamente más, pero es también una verdadera personalidad divina.

\par
%\textsuperscript{(96.4)}
\textsuperscript{8:6.2} El Espíritu Infinito es una personalidad completa y perfecta, el coordinado y el igual divino del Padre Universal y del Hijo Eterno. El Creador Conjunto es tan real y visible para las inteligencias superiores de los universos como el Padre y el Hijo; en verdad lo es más, porque es el Espíritu el que todos los ascendentes deben alcanzar antes de poder acercarse al Padre a través del Hijo.

\par
%\textsuperscript{(96.5)}
\textsuperscript{8:6.3} El Espíritu Infinito, la Tercera Persona de la Deidad, posee todos los atributos que vosotros asociáis con la personalidad. El Espíritu está dotado de una mente absoluta: «El Espíritu sondea todas las cosas, incluso las cosas profundas de Dios»\footnote{\textit{El Espíritu sondea todas las cosas}: 1 Cr 28:9; 1 Co 2:10.}. El Espíritu no sólo está dotado de mente, sino también de voluntad. A propósito de la concesión de sus dones, está escrito: «Pero todas estas cosas las hace el solo y mismo Espíritu, repartiendo a cada cual individualmente y como él quiere»\footnote{\textit{El mismo Espíritu repartido al hombre}: 1 Co 12:11.}.

\par
%\textsuperscript{(96.6)}
\textsuperscript{8:6.4} «El amor del Espíritu»\footnote{\textit{El amor del Espíritu}: Ro 15:30.} es real, como lo son también sus tristezas; por ello, «no aflijáis al Espíritu de Dios»\footnote{\textit{No aflijáis al Espíritu de Dios}: Ef 4:30.}. Cuando observamos al Espíritu Infinito, ya sea como una Deidad del Paraíso o como el Espíritu Creativo de un universo local, descubrimos que el Creador Conjunto no es solamente la Fuente-Centro Tercera sino también una persona divina. Esta personalidad divina reacciona también ante el universo como una persona. El Espíritu os dice: «Aquel que tenga oídos, que escuche lo que dice el Espíritu»\footnote{\textit{Quien tenga oídos que oiga al Espíritu}: Ap 2:7,11,17,29; 3:6,13,22.}. «El Espíritu mismo intercede por vosotros»\footnote{\textit{El Espíritu intercede}: Ro 8:26-27.}. El Espíritu ejerce una influencia personal y directa sobre los seres creados, «porque todos aquellos que son conducidos por el Espíritu de Dios, son hijos de Dios»\footnote{\textit{Hijos de Dios}: 1 Cr 22:10; Sal 2:7; Is 56:5; Mt 5:9,16,45; Lc 20:36; Jn 1:12-13; 11:52; Hch 17:28-29; Ro 9:26; 2 Co 6:18; Gl 3:26; 4:5-7; Ef 1:5; Flp 2:15; Heb 12:5-8; 1 Jn 3:1-2,10; 5:2; Ap 21:7; 2 Sam 7:14.}\footnote{\textit{Los conducidos por el Espíritu son hijos de Dios}: Ro 8:14-17,19,21.}.

\par
%\textsuperscript{(96.7)}
\textsuperscript{8:6.5} Aunque contemplemos el fenómeno del ministerio del Espíritu Infinito en los mundos lejanos del universo de universos, aunque imaginemos a esta misma Deidad coordinadora actuando en, y por medio de, las legiones incalculables de los múltiples seres que tienen su origen en la Fuente-Centro Tercera, aunque reconozcamos la omnipresencia del Espíritu, sin embargo seguimos afirmando que esta misma Fuente-Centro Tercera es una persona, el Creador Conjunto de todas las cosas, de todos los seres y de todos los universos.

\par
%\textsuperscript{(96.8)}
\textsuperscript{8:6.6} En la administración de los universos, el Padre, el Hijo y el Espíritu están perfecta y eternamente interasociados. Aunque cada uno de ellos está consagrado a un ministerio personal hacia toda la creación, los tres están divina y absolutamente entrelazados en un servicio de creación y de control que los convierte para siempre en \textit{uno solo.}

\par
%\textsuperscript{(97.1)}
\textsuperscript{8:6.7} En la persona del Espíritu Infinito, el Padre y el Hijo están siempre mutuamente presentes con una perfección incalificada, porque el Espíritu se parece al Padre y se parece al Hijo, y también se parece al Padre y al Hijo, ya que los dos son eternamente uno solo.

\par
%\textsuperscript{(97.2)}
\textsuperscript{8:6.8} [Presentado en Urantia por un Consejero Divino de Uversa, encargado por los Ancianos de los Días de describir la naturaleza y el trabajo del Espíritu Infinito.]