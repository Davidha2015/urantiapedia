\chapter{Documento 120. La donación de Miguel en Urantia}
\par
%\textsuperscript{(1323.1)}
\textsuperscript{120:0.1} DESIGNADO por Gabriel para supervisar la nueva exposición de la vida de Miguel cuando estuvo en Urantia en la similitud de la carne mortal, yo, el Melquisedek director de la comisión reveladora encargada de esta tarea, estoy autorizado a presentar esta narración sobre ciertos acontecimientos que precedieron de inmediato la llegada del Hijo Creador a Urantia para emprender la fase final de su experiencia de donación en el universo. Vivir esas vidas idénticas que él impone a los seres inteligentes de su propia creación, donarse así en la similitud de sus diversas órdenes de seres creados, es una parte del precio que cada Hijo Creador debe pagar para conseguir la soberanía completa y suprema sobre el universo de cosas y de seres creado por él\footnote{\textit{La donación de Miguel en Urantia}: Jn 1:1-18.}.

\par
%\textsuperscript{(1323.2)}
\textsuperscript{120:0.2} Antes de los acontecimientos que estoy a punto de describir, Miguel de Nebadon se había donado seis veces en la similitud de seis órdenes diferentes de su variada creación de seres inteligentes. Luego se preparó para descender a Urantia en la similitud de los mortales, la orden más humilde de sus criaturas volitivas inteligentes y, como tal humano del reino material, ejecutar el acto final del drama consistente en conseguir la soberanía sobre su universo de acuerdo con los mandatos de los divinos Gobernantes Paradisiacos del universo de universos.

\par
%\textsuperscript{(1323.3)}
\textsuperscript{120:0.3} En el transcurso de cada una de las donaciones anteriores, Miguel no sólo había adquirido la experiencia finita de un grupo de sus seres creados, sino que también había adquirido una experiencia esencial de cooperación con el Paraíso que, en sí misma y por sí misma, contribuiría además a establecerlo como soberano del universo creado por él. En cualquier momento durante todas las épocas pasadas del universo local, Miguel podía haber afirmado su soberanía personal como Hijo Creador, y, como Hijo Creador, podía haber gobernado su universo de la manera que hubiera escogido. En ese caso, Emmanuel y los Hijos Paradisiacos asociados se habrían marchado del universo. Pero Miguel no deseaba gobernar Nebadon simplemente por su propio derecho aislado, como Hijo Creador. Deseaba ascender, mediante una experiencia efectiva de subordinación cooperativa a la Trinidad del Paraíso, hasta esa elevada posición en el estatus universal en la que estaría cualificado para gobernar su universo y administrar sus asuntos con esa perfección de perspicacia y esa sabiduría de ejecución que algún día caracterizarán al gobierno sublime del Ser Supremo. No aspiraba a la perfección de gobierno como Hijo Creador, sino a la supremacía administrativa como personificación de la sabiduría universal y de la experiencia divina del Ser Supremo.

\par
%\textsuperscript{(1324.1)}
\textsuperscript{120:0.4} Miguel tenía, por tanto, una doble finalidad al efectuar estas siete donaciones a las diversas órdenes de criaturas de su universo: en primer lugar, completaba la experiencia obligatoria de comprender a las criaturas, que se exige a todos los Hijos Creadores antes de que asuman la soberanía completa. En cualquier momento, un Hijo Creador puede gobernar su universo por su propio derecho, pero sólo puede gobernar como representante supremo de la Trinidad del Paraíso después de pasar por las siete donaciones a las criaturas de su universo. En segundo lugar, aspiraba al privilegio de representar la máxima autoridad de la Trinidad del Paraíso que se puede ejercer en la administración directa y personal de un universo local. En consecuencia, durante la experiencia de cada una de sus donaciones en el universo, Miguel se subordinó voluntariamente, de manera satisfactoria y aceptable, a las voluntades combinadas de las diversas asociaciones de las personas de la Trinidad del Paraíso. Es decir, en la primera donación se sometió a la voluntad combinada del Padre, del Hijo y del Espíritu; en la segunda, a la voluntad del Padre y del Hijo; en la tercera, a la voluntad del Padre y del Espíritu; en la cuarta, a la voluntad del Hijo y del Espíritu; en la quinta, a la voluntad del Espíritu Infinito; en la sexta, a la voluntad del Hijo Eterno; y durante la séptima y última donación en Urantia, a la voluntad del Padre Universal\footnote{\textit{Revelar la voluntad del Padre Universal}: Mt 26:39,42,44; Mc 14:36,39; Lc 22:42; Jn 4:34; 5:30; 6:38-40; 15:10; 17:4.}.

\par
%\textsuperscript{(1324.2)}
\textsuperscript{120:0.5} Miguel combina pues, en su soberanía personal, la voluntad divina de las fases séptuples de los Creadores universales con la experiencia comprensiva de las criaturas de su universo local. Su administración se ha vuelto así representativa del máximo poder y autoridad, pero desprovista de toda apropiación arbitraria. Su poder es ilimitado, pues procede de una asociación experimentada con las Deidades del Paraíso; su autoridad es incuestionable, ya que fue conseguida mediante una experiencia real en la similitud de las criaturas del universo; su soberanía es suprema, puesto que expresa al mismo tiempo el punto de vista séptuple de la Deidad del Paraíso y el punto de vista de las criaturas del tiempo y del espacio\footnote{\textit{Todas las cosas entregadas en manos de Miguel}: Mt 11:27a; Lc 10:22a.}.

\par
%\textsuperscript{(1324.3)}
\textsuperscript{120:0.6} Después de determinar el momento de su donación final y después de elegir el planeta donde tendría lugar este acontecimiento extraordinario, Miguel mantuvo con Gabriel la conferencia habitual que precede a una donación, y luego se presentó ante Emmanuel, su hermano mayor y consejero paradisiaco. Miguel entregó entonces a la custodia de Emmanuel todos los poderes de la administración del universo que no habían sido conferidos previamente a Gabriel. Y justo antes de la partida de Miguel para encarnarse en Urantia, Emmanuel aceptó la custodia del universo durante el período de la donación en Urantia, y procedió a dar los consejos para la donación que servirían de guía a Miguel durante su encarnación cuando dentro de poco creciera en Urantia como un mortal del reino.

\par
%\textsuperscript{(1324.4)}
\textsuperscript{120:0.7} A este respecto se debe tener en cuenta que Miguel había elegido efectuar esta donación en la similitud de la carne mortal sometido a la voluntad del Padre Paradisiaco. El Hijo Creador no necesitaba instrucciones de nadie para llevar a cabo esta encarnación si hubiera tenido el único propósito de conseguir la soberanía sobre su universo, pero había emprendido un programa de revelación del Supremo que implicaba un funcionamiento cooperativo con las diversas voluntades de las Deidades del Paraíso. Y así, cuando consiguiera final y personalmente su soberanía, englobaría realmente la voluntad séptuple de la Deidad tal como ésta culmina en el Supremo. Por ello, anteriormente había sido instruido seis veces por los representantes personales de las diversas Deidades del Paraíso y sus asociaciones; y ahora recibía las instrucciones del Unión de los Días, el embajador de la Trinidad del Paraíso en el universo local de Nebadon, que actuaba en nombre del Padre Universal.

\par
%\textsuperscript{(1325.1)}
\textsuperscript{120:0.8} La buena disposición con que este poderoso Hijo Creador se subordinaba voluntariamente una vez más a la voluntad de las Deidades del Paraíso, en esta ocasión a la del Padre Universal, había de producir ventajas inmediatas y enormes compensaciones. Mediante esta decisión de efectuar un acto así de subordinación asociativa, Miguel iba a experimentar en esta encarnación no solamente la naturaleza del hombre mortal, sino también la voluntad del Padre Paradisiaco de todos. Además, podía emprender esta donación única con la completa seguridad de que no solamente Emmanuel ejercería la plena autoridad del Padre Paradisiaco en la administración de su universo durante su ausencia debida a la donación en Urantia, sino también con el conocimiento reconfortante de que los Ancianos de los Días del superuniverso habían decretado que su creación estaría segura durante todo el período de la donación.

\par
%\textsuperscript{(1325.2)}
\textsuperscript{120:0.9} Éste era el escenario en el momento importante en que Emmanuel presentó el cometido de la séptima donación. Tengo permiso para exponer los extractos siguientes de las instrucciones de Emmanuel, antes de la donación, al gobernante del universo que después se convirtió en Jesús de Nazaret (Cristo Miguel) en Urantia:

\section*{1. Misión de la séptima donación}
\par
%\textsuperscript{(1325.3)}
\textsuperscript{120:1.1} «Mi hermano Creador, estoy a punto de presenciar tu séptima y última donación universal. Has ejecutado con gran fidelidad y perfección las seis misiones anteriores, y sólo puedo pensar que saldrás igualmente triunfante de ésta, tu donación final camino de la soberanía. Hasta ahora has aparecido en las esferas de tus donaciones como un ser plenamente desarrollado de la orden que habías escogido. Ahora estás a punto de aparecer en Urantia, el planeta desordenado y perturbado que has elegido, no como un mortal plenamente desarrollado, sino como un bebé indefenso. Esto, compañero mío, va a ser para ti una experiencia nueva y no probada. Estás a punto de pagar todo el precio de la donación y de experimentar la iluminación completa de la encarnación de un Creador en la similitud de una criatura».

\par
%\textsuperscript{(1325.4)}
\textsuperscript{120:1.2} «Durante cada una de tus donaciones anteriores, elegiste someterte voluntariamente a la voluntad de las tres Deidades del Paraíso y de sus interasociaciones divinas. De las siete fases de la voluntad del Supremo, has estado sometido a todas ellas en tus anteriores donaciones, salvo a la voluntad personal de tu Padre Paradisiaco. Ahora que has decidido someterte por completo a la voluntad de tu Padre durante toda tu séptima donación, yo, como representante personal de nuestro Padre, asumo la jurisdicción incondicional sobre tu universo durante el período de tu encarnación».

\par
%\textsuperscript{(1325.5)}
\textsuperscript{120:1.3} «Al emprender la donación en Urantia, te has despojado voluntariamente de todo apoyo extraplanetario y de toda ayuda especial que hubiera podido prestarte cualquier criatura de tu propia creación. Al igual que tus hijos creados de Nebadon dependen totalmente de ti para conducirse con seguridad durante toda su carrera universal, ahora deberás depender enteramente y sin reservas de tu Padre Paradisiaco para conducirte con seguridad a través de las vicisitudes no reveladas de tu próxima carrera como mortal. Y cuando hayas terminado esta experiencia donadora, conocerás en toda su verdad el pleno sentido y el rico significado de esa confianza por la fe cuyo dominio exiges tan invariablemente a todas tus criaturas como parte de sus relaciones íntimas contigo, como Creador y Padre de su universo local».

\par
%\textsuperscript{(1326.1)}
\textsuperscript{120:1.4} «Durante toda tu donación en Urantia sólo tienes que preocuparte de una sola cosa, de la comunión ininterrumpida entre tú y tu Padre Paradisiaco; la perfección de esa relación permitirá que el mundo de tu donación, e incluso todo el universo creado por ti, contemplen una revelación nueva y más comprensible de tu Padre y de mi Padre, del Padre Universal de todos. Sólo tienes que preocuparte, pues, de tu vida personal en Urantia. Yo me haré plena y eficazmente responsable de la seguridad y de la administración ininterrumpida de tu universo desde el momento en que renuncies voluntariamente a tu autoridad hasta que regreses a nosotros como Soberano del Universo, confirmado por el Paraíso, y recibas nuevamente de mis manos, no la autoridad de vicegerente que ahora me entregas, sino en lugar de ella, el poder supremo y la jurisdicción sobre tu universo».

\par
%\textsuperscript{(1326.2)}
\textsuperscript{120:1.5} «Y para que puedas saber con seguridad que tengo la facultad de hacer todo lo que te prometo en este momento (sabiendo muy bien que soy la garantía de todo el Paraíso para el fiel cumplimiento de mi palabra), te anuncio que acaban de comunicarme un mandato de los Ancianos de los Días de Uversa que impedirá todo peligro espiritual en Nebadon durante el período de tu donación voluntaria. Desde el momento en que abandones tu conciencia, al principio de tu encarnación como mortal, hasta que regreses a nosotros como soberano supremo e incondicional de este universo que tú mismo has creado y organizado, nada grave podrá ocurrir en todo Nebadon. Durante el ínterin de tu encarnación, poseo las instrucciones de los Ancianos de los Días que ordenan inequívocamente la destrucción instantánea y automática de cualquier ser culpable de rebelión o que se atreva a instigar una insurrección en el universo de Nebadon mientras estés ausente durante esta donación. Hermano mío, a la vista de la autoridad del Paraíso inherente a mi presencia y acrecentada por el mandato judicial de Uversa, tu universo y todas sus criaturas leales estarán a salvo durante tu donación. Puedes emprender tu misión con un solo pensamiento ---ampliar la revelación de nuestro Padre a los seres inteligentes de tu universo».

\par
%\textsuperscript{(1326.3)}
\textsuperscript{120:1.6} «Como en cada una de tus donaciones anteriores, quisiera recordarte que recibo la jurisdicción sobre tu universo en calidad de hermano fideicomisario. Ejerzo toda la autoridad y uso todo el poder en tu nombre. Actúo como lo haría nuestro Padre Paradisiaco y de acuerdo con tu petición explícita de que actúe así en tu lugar. Así las cosas, toda esta autoridad delegada podrás ejercerla de nuevo en cualquier momento que estimes oportuno solicitar su restitución. Tu donación es totalmente voluntaria en todas sus fases. Como mortal encarnado en el mundo, estarás desprovisto de facultades celestiales, pero podrás recuperar todo el poder abandonado en cualquier momento que decidas reasumir tu autoridad universal. Si eligieras reinstalarte en tu poder y tu autoridad, recuerda que sería enteramente por razones \textit{personales}, puesto que soy la garantía viviente y suprema cuya presencia y promesa aseguran la administración intacta de tu universo de acuerdo con la voluntad de tu Padre. Una rebelión como ya ha ocurrido tres veces en Nebadon no puede producirse durante tu ausencia de Salvington para esta donación. Para el período de tu donación en Urantia, los Ancianos de los Días han decretado que toda rebelión en Nebadon contendrá la semilla automática de su propia aniquilación».

\par
%\textsuperscript{(1326.4)}
\textsuperscript{120:1.7} «Mientras estés ausente debido a esta donación final y extraordinaria, me comprometo (con la cooperación de Gabriel) a administrar fielmente tu universo; al encomendarte que emprendas este ministerio de revelación divina y que pases por esta experiencia de comprensión perfeccionada de los humanos, actúo en nombre de mi Padre y tu Padre, y te ofrezco los consejos siguientes que deberían guiarte para vivir tu vida terrestre a medida que tomes conciencia progresivamente de la misión divina de tu estancia continuada en la carne:»

\section*{2. Las limitaciones de la donación}
\par
%\textsuperscript{(1327.1)}
\textsuperscript{120:2.1} «1. De acuerdo con las costumbres y en conformidad con la técnica de Sonarington ---de acuerdo con los mandatos del Hijo Eterno del Paraíso--- lo he previsto todo para que puedas emprender inmediatamente esta donación como mortal en armonía con los planes formulados por ti y que Gabriel me ha entregado para su custodia. Crecerás en Urantia como un hijo del planeta, completarás tu educación humana ---sometido en todo momento a la voluntad de tu Padre Paradisiaco--- vivirás tu vida en Urantia como lo has determinado, terminarás tu estancia planetaria y te prepararás para ascender hasta tu Padre y recibir de él la soberanía suprema sobre tu universo»\footnote{\textit{Jesús vive sometido a la voluntad del Padre}: Mt 26:39,42,44; Mc 14:36,39; Lc 22:42; Jn 4:34; 5:30; 6:38-40; 15:10; 17:4.}.

\par
%\textsuperscript{(1327.2)}
\textsuperscript{120:2.2} «2. Aparte de tu misión en la Tierra y de tu revelación al universo, pero inherente a las dos, te aconsejo que, una vez que seas suficientemente consciente de tu identidad divina, asumas la tarea adicional de poner fin técnicamente a la rebelión de Lucifer en el sistema de Satania, y que hagas todo esto como \textit{Hijo del Hombre}. Así pues, como una criatura mortal del mundo que en su debilidad se ha hecho poderosa porque se ha sometido por la fe a la voluntad de su Padre, te sugiero que lleves a cabo con benevolencia todo lo que tantas veces te has negado a realizar arbitrariamente por la fuerza y el poder cuando disponías de estos atributos en la época en que empezó esta rebelión pecaminosa e injustificada. Yo consideraría como una digna culminación de tu donación como mortal que volvieras entre nosotros como Hijo del Hombre, Príncipe Planetario de Urantia, a la vez que como Hijo de Dios, soberano supremo de tu universo. Como hombre mortal, el tipo más inferior de criatura inteligente en Nebadon, haz frente y juzga las pretensiones blasfemas de Caligastia y de Lucifer, y en el humilde estado que habrás asumido, pon fin para siempre a las tergiversaciones vergonzosas de estos hijos de la luz caídos\footnote{\textit{«Hijos de la luz»}: Lc 16:8; Jn 12:36; Ef 5:8; 1 Ts 5:5.}. Ya que has rehusado continuamente desacreditar a estos rebeldes mediante el ejercicio de tus prerrogativas como creador, sería conveniente que ahora, en la similitud de las criaturas más humildes de tu creación, arrebates el poder de las manos de estos Hijos caídos; y así todo tu universo local reconocerá con toda equidad, claramente y para siempre, que has sido justo al hacer, en la forma de la carne mortal, aquellas cosas que la misericordia no te exhortó a hacer con el poder de una autoridad arbitraria. Habiendo establecido así, por medio de tu donación, la posibilidad de la soberanía del Supremo en Nebadon, habrás llevado efectivamente a su término los asuntos pendientes de todas las insurrecciones anteriores, a pesar de la mayor o menor cantidad de tiempo que te lleve realizar esta tarea. Esta acción eliminará lo más esencial de las disensiones pendientes en tu universo. Cuando recibas posteriormente la soberanía suprema sobre tu universo, en ninguna parte de tu gran creación personal podrán producirse desafíos similares a tu autoridad»\footnote{\textit{Finalización de la rebelión}: Is 14:12-20; Mt 4:1-11; Mc 1:12-13; Lc 4:1-14; 10:18; 2 P 2:4; Ap 12:7-9.}.

\par
%\textsuperscript{(1327.3)}
\textsuperscript{120:2.3} «3. Cuando hayas logrado poner fin a la secesión en Urantia, cosa que harás indudablemente, te aconsejo que aceptes que Gabriel te confiera el título de `Príncipe Planetario de Urantia' como reconocimiento eterno de tu universo por tu experiencia final de donación, y que además hagas todo lo posible, que sea consecuente con el significado de tu donación, por reparar la aflicción y la confusión causadas en Urantia por la traición de Caligastia y la falta adámica posterior».

\par
%\textsuperscript{(1328.1)}
\textsuperscript{120:2.4} «4. De acuerdo con tu petición, Gabriel y todos los interesados cooperarán contigo en el deseo que has expresado de terminar tu donación en Urantia con la declaración de un juicio dispensacional\footnote{\textit{Juicio dispensacional}: Mt 27:52-53; Jn 5:25-29.} del planeta, acompañado por el final de una era, la resurrección de los supervivientes\footnote{\textit{Resurrección de los muertos}: Mt 27:52-53.} mortales dormidos y el establecimiento de la dispensación del Espíritu de la Verdad\footnote{\textit{Espíritu de la Verdad}: Ez 11:19; 18:31; 36:26-27; Jl 2:28-29; Lc 24:49; Jn 7:39; 14:16-18,23,26; 15:4,26; 16:7-8,13-14; 17:21-23; Hch 1:5,8a; 2:1-4,16-18; 2:33; 2 Co 13:5; Gl 2:20; 4:6; Ef 1:13; 4:30; 1 Jn 4:12-15.} otorgado».

\par
%\textsuperscript{(1328.2)}
\textsuperscript{120:2.5} «5. En lo que se refiere al planeta de tu donación y a la generación inmediata de hombres que vivirán allí en la época de tu estancia como mortal, te aconsejo que desempeñes principalmente el papel de instructor. Concede tu atención, en primer lugar, a la liberación y a la inspiración de la naturaleza espiritual del hombre. A continuación, ilumina el intelecto ensombrecido de los hombres, cura sus almas y libera sus mentes de los temores seculares. Y luego, de acuerdo con tu sabiduría humana, contribuye al bienestar físico y a la comodidad material de tus hermanos en la carne. Vive la vida religiosa ideal para inspirar y edificar a todo tu universo».

\par
%\textsuperscript{(1328.3)}
\textsuperscript{120:2.6} «6. En el planeta de tu donación, libera espiritualmente al hombre aislado por la rebelión\footnote{\textit{Libertad a los cautivos espirituales}: Is 42:5-7; 49:9; 61:1; Lc 4:18; Gl 5:1,13.}. En Urantia, haz una contribución adicional a la soberanía del Supremo, extendiendo así el establecimiento de esta soberanía por todos los amplios dominios de tu creación personal. En esta donación material en la similitud de la carne, estás a punto de experimentar la iluminación final de un Creador espacio-temporal, la doble experiencia de trabajar dentro de la naturaleza del hombre con la voluntad de tu Padre Paradisiaco. En tu vida temporal, la voluntad de la criatura finita y la voluntad del Creador infinito han de convertirse en una sola, tal como se están uniendo también en la Deidad evolutiva del Ser Supremo. Derrama sobre el planeta de tu donación el Espíritu de la Verdad para que todos los mortales normales de esa esfera aislada tengan así un acceso inmediato y completo al ministerio de la presencia separada de nuestro Padre Paradisiaco, los Ajustadores del Pensamiento de los mundos».

\par
%\textsuperscript{(1328.4)}
\textsuperscript{120:2.7} «7. En todo lo que vayas a hacer en el mundo de tu donación, recuerda siempre que estás viviendo una vida para la instrucción y la edificación de todo tu universo. Vas a \textit{donar} esta vida de encarnación mortal en Urantia, pero debes \textit{vivir} dicha vida para inspirar espiritualmente a todas las inteligencias humanas y superhumanas que han vivido, existen ahora o puedan vivir en cada mundo habitado que ha formado, forma ahora o pueda formar parte de la inmensa galaxia de tu dominio administrativo. Tu vida terrestre en la similitud de la carne mortal no debes vivirla para que sirva de \textit{ejemplo} a los mortales de Urantia de la época de tu estancia en la Tierra, ni para ninguna generación posterior de seres humanos de Urantia o de cualquier otro mundo. En lugar de eso, tu vida encarnada en Urantia será una \textit{inspiración} para todas las vidas de todos los mundos de Nebadon de todas las generaciones de los tiempos por venir».

\par
%\textsuperscript{(1328.5)}
\textsuperscript{120:2.8} «8. La gran misión que debes realizar y experimentar en la encarnación mortal está contenida en tu decisión de vivir una vida totalmente dedicada a hacer la voluntad de tu Padre Paradisiaco\footnote{\textit{Jesús dedicado a vivir la voluntad del Padre}: Mt 26:39,42,44; Mc 14:36,39; Lc 22:42; Jn 4:34; 5:30; 6:38-40; 15:10; 17:4.}, y así \textit{revelar a Dios}, tu Padre, en la carne y especialmente a las criaturas de carne. Al mismo tiempo, \textit{interpretarás} también con un nuevo realce a nuestro Padre para los seres supermortales de todo Nebadon. Junto con este ministerio de nueva revelación y de interpretación ampliada del Padre Paradisiaco\footnote{\textit{Jesús dedicado a revelar a Dios a la humanidad}: Mt 5:45,48; 6:1,4,6; 11:25-27; Mc 11:25-26; Lc 6:35-36; 10:22; Jn 1:18; 3:31-34; 4:21-23; 6:45-46; 14:6-11,20; 15:15; 16:25; 17:8,25-26.} para los tipos de mente humana y superhumana, actuarás también de tal manera que efectuarás una nueva revelación del hombre a Dios. Demuestra en tu corta y única vida en la carne, como nunca antes se ha visto en todo Nebadon, las posibilidades trascendentes que puede alcanzar un humano que conoce a Dios durante la breve carrera de la existencia mortal, y efectúa una \textit{interpretación} nueva y reveladora del hombre y de las vicisitudes de su vida planetaria a todas las inteligencias superhumanas de todo Nebadon y para todos los tiempos. Vas a descender a Urantia en la similitud de la carne mortal, y al vivir como un hombre de tu tiempo y de tu generación, actuarás de tal manera que mostrarás a todo tu universo el ideal de una técnica perfeccionada en el compromiso supremo de los asuntos de tu inmensa creación: la hazaña de Dios que busca al hombre y lo encuentra, y el fenómeno del hombre que busca a Dios y lo encuentra; hacer todo esto para su satisfacción mutua, y hacerlo durante una corta vida en la carne».

\par
%\textsuperscript{(1329.1)}
\textsuperscript{120:2.9} «9. Te advierto que tengas siempre presente que, aunque de hecho te vas a convertir en un hombre normal del mundo, seguirás siendo en potencia un Hijo Creador del Padre Paradisiaco. Durante toda esta encarnación, aunque vas a vivir y actuar como Hijo del Hombre, los atributos creativos de tu divinidad personal irán contigo desde Salvington a Urantia. Tu voluntad siempre tendrá el poder de dar por terminada la encarnación en cualquier momento posterior a la llegada de tu Ajustador del Pensamiento. Antes de la llegada y de la recepción del Ajustador, yo garantizaré la integridad de tu personalidad. Pero después de la llegada de tu Ajustador, y a medida que reconozcas progresivamente la naturaleza y la importancia de tu misión donadora, deberías abstenerte de formular cualquier deseo superhumano de obtener, de conseguir o de poder, debido al hecho de que tus prerrogativas como creador permanecerán asociadas a tu personalidad mortal, porque estos atributos son inseparables de tu presencia personal. Pero, aparte de la voluntad del Padre Paradisiaco, ninguna repercusión superhumana acompañará tu carrera terrestre, a menos que tú, mediante un acto de voluntad consciente y deliberada, tomes una decisión indivisa que conduzca a la elección de toda tu personalidad».

\section*{3. Consejos y advertencias adicionales}
\par
%\textsuperscript{(1329.2)}
\textsuperscript{120:3.1} «Y ahora, hermano mío, al despedirme de ti mientras te preparas para partir hacia Urantia, y después de haberte aconsejado sobre la conducta general de tu donación, permíteme presentarte algunas advertencias que son el resultado de una deliberación con Gabriel y que se refieren a aspectos menores de tu vida como mortal. Así pues, te sugerimos además que:»

\par
%\textsuperscript{(1329.3)}
\textsuperscript{120:3.2} «1. En la búsqueda del ideal de tu vida mortal en la Tierra, concedas también alguna atención a la realización y ejemplificación de algunas cosas prácticas e inmediatamente útiles para tus compañeros humanos».

\par
%\textsuperscript{(1329.4)}
\textsuperscript{120:3.3} «2. En lo que concierne a las relaciones familiares, da prioridad a las costumbres aceptadas de la vida familiar tal como las encuentres establecidas en la época y en la generación de tu donación. Vive tu vida familiar y comunitaria de acuerdo con las prácticas del pueblo en el que has elegido aparecer».

\par
%\textsuperscript{(1329.5)}
\textsuperscript{120:3.4} «3. En tus relaciones con el orden social, te aconsejamos que limites tus esfuerzos principalmente a la regeneración espiritual y a la emancipación intelectual. Evita todo enredo con la estructura económica y los compromisos políticos de tu época. Conságrate en especial a vivir la vida religiosa ideal en Urantia».

\par
%\textsuperscript{(1329.6)}
\textsuperscript{120:3.5} «4. En ninguna circunstancia, ni siquiera en el más mínimo detalle, debes interferir en la evolución progresiva, normal y ordenada de las razas de Urantia. Pero no se debe interpretar que esta prohibición limita tus esfuerzos por dejar detrás de ti, en Urantia, un sistema duradero y mejorado de \textit{ética religiosa positiva}. Como Hijo dispensacional, se te han concedido ciertos privilegios relacionados con el avance del estado \textit{espiritual} y \textit{religioso} de los pueblos del mundo».

\par
%\textsuperscript{(1330.1)}
\textsuperscript{120:3.6} «5. Si lo consideras conveniente, puedes identificarte con los movimientos religiosos y espirituales existentes que puedan encontrarse en Urantia, pero trata de evitar, de todas las maneras posibles, el establecimiento formal de un culto organizado, de una religión cristalizada o de una agrupación ética separada de seres humanos. Tu vida y tus enseñanzas deben convertirse en la herencia común de todas las religiones y de todos los pueblos».

\par
%\textsuperscript{(1330.2)}
\textsuperscript{120:3.7} «6. Con el fin de que no contribuyas innecesariamente a la creación de sistemas estereotipados posteriores de creencias religiosas en Urantia, o de otros tipos de lealtades religiosas no progresivas, te aconsejamos además: No dejes ningún escrito detrás de ti en el planeta. Abstente de escribir en materiales permanentes; ordena a tus asociados que no hagan imágenes u otros retratos de tu aspecto físico. Asegúrate de que nada potencialmente idólatra se quede en el planeta en el momento de tu partida».

\par
%\textsuperscript{(1330.3)}
\textsuperscript{120:3.8} «7. Aunque vivirás la vida social normal y corriente del planeta, y serás un individuo normal del sexo masculino, es probable que no entres en la relación del matrimonio, una relación que sería perfectamente honorable y compatible con tu donación; pero debo recordarte que uno de los mandatos de Sonarington, relativos a la encarnación, prohíbe que un Hijo donador originario del Paraíso deje tras de sí una descendencia humana en un planeta cualquiera».

\par
%\textsuperscript{(1330.4)}
\textsuperscript{120:3.9} «8. Para todos los demás detalles de tu próxima donación, quisiéramos confiarte a la dirección de tu Ajustador interior, a las enseñanzas del espíritu divino siempre presente que guía a los hombres, y al juicio razonable de tu mente humana de origen hereditario y en expansión. Una asociación así de atributos de criatura y de Creador te permitirá vivir para nosotros la vida perfecta del hombre en las esferas planetarias, no necesariamente perfecta tal como la pueda considerar cualquier hombre de cualquier generación en cualquier mundo (y mucho menos en Urantia), pero será evaluada como total y supremamente plena por los mundos más perfeccionados y en vías de perfeccionarse de tu extenso universo».

\par
%\textsuperscript{(1330.5)}
\textsuperscript{120:3.10} «Y ahora, que tu Padre y mi Padre, que siempre nos ha sostenido en todas las actividades pasadas, te guíe, te sostenga y esté contigo desde el momento en que nos dejes y lleves a cabo el abandono de la conciencia de tu personalidad, durante tu reconocimiento gradual de tu identidad divina encarnada en una forma humana, y luego durante toda tu experiencia de donación en Urantia, hasta tu liberación de la carne y tu ascensión a la derecha de la soberanía de nuestro Padre. Cuando vuelva a verte en Salvington, te acogeremos a tu regreso como soberano supremo e incondicional de este universo que tú mismo has creado, servido y comprendido por completo».

\par
%\textsuperscript{(1330.6)}
\textsuperscript{120:3.11} «Ahora reino en tu lugar. Asumo la jurisdicción sobre todo Nebadon como soberano en funciones durante el ínterin de tu séptima donación, la de un mortal en Urantia. A ti, Gabriel, te encomiendo la salvaguardia del que está a punto de ser el Hijo del Hombre, hasta que pronto regrese a mí envuelto en poder y gloria como Hijo del Hombre e Hijo de Dios. Gabriel, y yo soy tu soberano hasta que Miguel regrese así».

\par
%\textsuperscript{(1330.7)}
\textsuperscript{120:3.12} Luego, en presencia de todo Salvington reunido, Miguel se retiró inmediatamente de entre nosotros, y ya no volvimos a verlo en su sitio de costumbre hasta que regresó como soberano supremo y personal del universo, después de finalizar su carrera de donación en Urantia.

\section*{4. La encarnación --- la unión de dos en uno}
\par
%\textsuperscript{(1331.1)}
\textsuperscript{120:4.1} Así pues, ciertos hijos indignos de Miguel, que habían acusado a su padre-Creador de buscar egoístamente la soberanía y que se habían permitido insinuar que el Hijo Creador se mantenía en el poder de manera arbitraria y autocrática debido a la lealtad irracional de las criaturas sumisas de un universo engañado, iban a ser silenciados para siempre y a quedarse confundidos y desilusionados por la vida de servicio altruista que el Hijo de Dios empezaba ahora como Hijo del Hombre ---todo el tiempo sometido a «la voluntad del Padre Paradisiaco»\footnote{\textit{Vivir sometido a la voluntad del Padre}: Mt 26:39,42,44; Mc 14:36,39; Lc 22:42; Jn 4:34; 5:30; 6:38-40; 15:10; 17:4.}.

\par
%\textsuperscript{(1331.2)}
\textsuperscript{120:4.2} Pero no os equivoquéis; aunque Cristo Miguel era verdaderamente un ser de origen dual, no era una personalidad doble. No era Dios en asociación \textit{con} el hombre, sino más bien Dios \textit{encarnado} en el hombre. Y siempre fue exactamente este ser combinado. El único factor progresivo en esta relación incomprensible fue la comprensión y el reconocimiento conscientes y graduales (por parte de su mente humana) de este hecho de ser Dios y hombre.

\par
%\textsuperscript{(1331.3)}
\textsuperscript{120:4.3} Cristo Miguel no se volvió progresivamente Dios. Dios no se volvió hombre en algún momento vital de la vida terrestre de Jesús. Jesús fue Dios \textit{y} hombre ---siempre e incluso para siempre jamás\footnote{\textit{Jesús, Dios y hombre}: Jn 1:1,14; 10:30; 14:9-11; 17:11b,21-23a.}. Este Dios y este hombre eran, y son ahora, \textit{uno solo}, al igual que la Trinidad del Paraíso compuesta por tres seres es en realidad \textit{una} Deidad.

\par
%\textsuperscript{(1331.4)}
\textsuperscript{120:4.4} Nunca perdáis de vista el hecho de que el propósito espiritual supremo de la donación de Miguel era realzar la \textit{revelación de Dios}\footnote{\textit{Realzar la revelación de Dios}: Mt 11:27; Lc 10:22; Jn 1:18; 6:45-46; 8:26-27; 12:43-44,49-50; 14:7-11; 17:7-9,25-26.}.

\par
%\textsuperscript{(1331.5)}
\textsuperscript{120:4.5} Los mortales de Urantia tienen conceptos variables de lo milagroso, pero para nosotros, que vivimos como ciudadanos del universo local, hay pocos milagros, y entre éstos, las donaciones encarnadas de los Hijos Paradisiacos son con mucho los más misteriosos. La aparición de un Hijo divino en vuestro mundo por un proceso aparentemente natural, nosotros la consideramos como un milagro ---el funcionamiento de unas leyes universales que sobrepasan nuestra comprensión. Jesús de Nazaret era una persona milagrosa.

\par
%\textsuperscript{(1331.6)}
\textsuperscript{120:4.6} A lo largo de toda esta experiencia extraordinaria, Dios Padre escogió manifestarse como siempre lo hace \textit{{}---de la manera habitual---} de la manera normal, natural y fiable de la actuación divina.