\chapter{Documento 130. En el camino a Roma}
\par
%\textsuperscript{(1427.1)}
\textsuperscript{130:0.1} El viaje por el mundo romano consumió la mayor parte del año veintiocho y todo el año veintinueve de la vida de Jesús en la Tierra. Jesús y los dos nativos de la India ---Gonod y su hijo Ganid--- salieron de Jerusalén el domingo por la mañana 26 de abril del año 22. Llevaron a cabo su viaje tal como lo habían programado, y Jesús se despidió del padre y del hijo en la ciudad de Charax, en el Golfo Pérsico, el 10 de diciembre del año siguiente, el año 23.

\par
%\textsuperscript{(1427.2)}
\textsuperscript{130:0.2} Desde Jerusalén se dirigieron a Cesarea pasando por Jope. En Cesarea cogieron un barco para Alejandría. Desde Alejandría navegaron hasta Lasea, en Creta. Desde Creta siguieron por mar hasta Cartago, haciendo escala en Cirene. En Cartago tomaron un barco para Nápoles, deteniéndose en Malta, Siracusa y Mesina. Desde Nápoles fueron a Capua, y desde allí viajaron por la Vía Apia hasta Roma.

\par
%\textsuperscript{(1427.3)}
\textsuperscript{130:0.3} Al terminar su estancia en Roma se dirigieron por vía terrestre a Tarento, donde se hicieron a la mar para Atenas en Grecia, deteniéndose en Nicópolis y Corinto. Desde Atenas fueron a Éfeso por la ruta de Troade. Desde Éfeso navegaron hacia Chipre, haciendo escala en Rodas. Pasaron mucho tiempo visitando Chipre y descansando, y luego se embarcaron para Antioquía en Siria. Desde Antioquía fueron hacia el sur hasta Sidón y pasaron después por Damasco. Desde Damasco viajaron en caravana hasta Mesopotamia, pasando por Tapsacos y Larisa. Permanecieron algún tiempo en Babilonia, visitaron Ur y otros lugares, y luego fueron a Susa. Desde Susa viajaron a Charax, donde Gonod y Ganid embarcaron para la India.

\par
%\textsuperscript{(1427.4)}
\textsuperscript{130:0.4} Jesús había aprendido los rudimentos del idioma que hablaban Gonod y Ganid cuando estuvo trabajando cuatro meses en Damasco. Mientras estuvo allí, pasó la mayoría del tiempo haciendo traducciones del griego a una de las lenguas de la India, con la ayuda de un nativo de la región donde vivía Gonod.

\par
%\textsuperscript{(1427.5)}
\textsuperscript{130:0.5} Durante su viaje por el Mediterráneo, Jesús pasó aproximadamente la mitad del día enseñando a Ganid y sirviendo de intérprete a Gonod en sus entrevistas de negocios y en sus relaciones sociales. El resto del día lo tenía a su disposición, y lo dedicaba a entablar esos estrechos contactos personales con sus semejantes, esas íntimas relaciones con los mortales de este mundo, que tanto caracterizaron sus actividades de estos años inmediatamente anteriores a su ministerio público.

\par
%\textsuperscript{(1427.6)}
\textsuperscript{130:0.6} Gracias a estas observaciones de primera mano y a estos contactos reales, Jesús trabó conocimiento con la civilización material e intelectual superior de Occidente y del Levante. De Gonod y de su brillante hijo aprendió mucho sobre la civilización y la cultura de la India y de China, ya que Gonod, que era ciudadano de la India, había hecho tres grandes viajes al imperio de la raza amarilla.

\par
%\textsuperscript{(1427.7)}
\textsuperscript{130:0.7} El joven Ganid aprendió mucho de Jesús durante esta larga e íntima asociación. Llegaron a tenerse un gran afecto mutuo, y el padre del muchacho trató muchas veces de persuadir a Jesús para que los acompañara a la India, pero él siempre declinó la invitación, alegando la necesidad de regresar con su familia de Palestina.

\section*{1. En Jope --- discurso sobre Jonás}
\par
%\textsuperscript{(1428.1)}
\textsuperscript{130:1.1} Durante su estancia en Jope, Jesús conoció a Gadía\footnote{\textit{Gadía y Simón}: Hch 10:5-6.}, un intérprete filisteo que trabajaba para un curtidor llamado Simón. Los agentes de Gonod en Mesopotamia habían hecho muchos negocios con este Simón; por eso Gonod y su hijo deseaban visitarlo camino de Cesarea. Mientras permanecieron en Jope, Jesús y Gadía se hicieron buenos amigos. El joven filisteo era un buscador de la verdad. Jesús era un dador de la verdad; él \textit{era} la verdad para esa generación en Urantia. Cuando un gran buscador y un gran dador de la verdad se encuentran, se produce una gran iluminación liberadora surgida de la experiencia de la nueva verdad.

\par
%\textsuperscript{(1428.2)}
\textsuperscript{130:1.2} Un día, después de la cena, Jesús y el joven filisteo paseaban por la orilla del mar y Gadía, sin saber que este «escriba de Damasco» estaba tan bien versado en las tradiciones hebreas, mostró a Jesús el lugar donde Jonás supuestamente había embarcado\footnote{\textit{Embarco de Jonás}: Jon 1:3.} para su funesto viaje a Tarsis. Cuando concluyó sus comentarios, hizo a Jesús la pregunta siguiente: «¿Tú crees que el gran pez se tragó realmente a Jonás?»\footnote{\textit{¿Tú crees que el gran pez se tragó realmente a Jonás?}: Jon 1:17.}. Jesús percibió que la vida del joven había estado enormemente influida por esta tradición, y que sus reflexiones al respecto le habían inculcado la locura de intentar huir del deber. Por lo tanto, Jesús no dijo nada que pudiera destruir repentinamente las motivaciones fundamentales que guiaban a Gadía en su vida práctica. En respuesta a la pregunta, Jesús dijo: «Amigo mío, todos somos como Jonás, con una vida que vivir de acuerdo con la voluntad de Dios. Cada vez que tratamos de esquivar el deber de la vida diaria para ir en busca de tentaciones lejanas, nos ponemos inmediatamente bajo el dominio de influencias que no están dirigidas por los poderes de la verdad ni por las fuerzas de la rectitud. Huir del deber es sacrificar la verdad. Evadirse del servicio de la luz y la vida sólo puede llevar a esos conflictos angustiosos con las temibles ballenas del egoísmo, que al final conducen a las tinieblas y a la muerte, a menos que esos Jonases que han abandonado a Dios deseen, incluso estando en lo más profundo de su desesperación, volver su corazón hacia la búsqueda de Dios y su bondad. Cuando estas almas desalentadas buscan sinceramente a Dios ---con hambre de verdad y sed de rectitud--- no hay nada que pueda retenerlas por más tiempo en cautiverio. Por muy profundos que sean los abismos donde puedan haber caído, cuando buscan la luz de todo corazón, el espíritu del Señor Dios de los cielos las libera de sus cadenas; las tribulaciones de la vida las arrojan a la tierra firme de las nuevas oportunidades para un servicio renovado y una vida más sabia».

\par
%\textsuperscript{(1428.3)}
\textsuperscript{130:1.3} Gadía se sintió muy conmovido por la enseñanza de Jesús. Siguieron conversando a la orilla del mar hasta muy entrada la noche, y antes de regresar a sus alojamientos, rezaron juntos y el uno por el otro. Este mismo Gadía escuchó las predicaciones posteriores de Pedro, se convirtió en un profundo creyente en Jesús de Nazaret, y tuvo una noche una memorable controversia con Pedro en casa de Dorcas\footnote{\textit{Casa de Dorcas en Jope}: Hch 9:36-42.}. Gadía también contribuyó mucho a que Simón, el rico mercader de cuero\footnote{\textit{Conversión de Simón el curtidor}: Hch 9:43.}, se decidiera a abrazar el cristianismo.

\par
%\textsuperscript{(1428.4)}
\textsuperscript{130:1.4} (En este relato de la obra personal de Jesús con sus semejantes mortales durante su viaje por el Mediterráneo, y de acuerdo con el permiso que hemos recibido, traduciremos libremente sus palabras a la terminología moderna que se emplea en Urantia en el momento de esta presentación).

\par
%\textsuperscript{(1429.1)}
\textsuperscript{130:1.5} La última conversación de Jesús con Gadía trató sobre el bien y el mal. Este joven filisteo estaba bastante desconcertado por el sentimiento de injusticia que le producía la presencia del mal conviviendo con el bien en el mundo. Dijo: «Si Dios es infinitamente bueno, ¿cómo puede permitir que suframos las penas del mal?. Después de todo, ¿quién crea el mal?» En aquellos tiempos, mucha gente creía todavía que Dios creaba a la vez el bien y el mal, pero Jesús nunca enseñó un error semejante. Al responder a esta pregunta, Jesús dijo: «Hermano mío, Dios es amor, por lo tanto debe ser bueno, y su bondad es tan grande y real que no puede contener las cosas pequeñas e irreales del mal. Dios es tan positivamente bueno que no hay absolutamente ninguna cabida en él para el mal negativo. El mal es la elección inmadura y el paso en falso irreflexivo de los que se resisten a la bondad, rechazan la belleza y traicionan la verdad. El mal sólo es la inadaptación de la inmadurez o la influencia desintegradora y deformadora de la ignorancia. El mal es la inevitable oscuridad que sigue de cerca al rechazo imprudente de la luz. El mal es lo tenebroso y lo falso; cuando se abraza conscientemente y se aprueba voluntariamente, se convierte en pecado»\footnote{\textit{Dios es amor}: 1 Jn 4:8,16.}.

\par
%\textsuperscript{(1429.2)}
\textsuperscript{130:1.6} «Al dotarte de la facultad de escoger entre la verdad y el error, tu Padre celestial ha creado el potencial negativo de la vía positiva de la luz y la vida; pero los errores del mal no existen realmente hasta el momento en que una criatura inteligente quiere que existan, por una mala elección de su manera de vivir. Estos males se elevan posteriormente a la categoría de pecado mediante la elección consciente y deliberada de esa misma criatura obstinada y rebelde. Por eso, nuestro Padre que está en los cielos permite que el bien y el mal continúen juntos su camino hasta el final de la vida, al igual que la naturaleza permite que el trigo y la cizaña crezcan juntos hasta el momento de la siega»\footnote{\textit{El trigo y la cizaña}: Mt 13:24-30.}. Gadía quedó plenamente satisfecho con la respuesta de Jesús a su pregunta, después de que la discusión posterior clarificara en su mente el verdadero significado de estas importantes declaraciones.

\section*{2. En Cesarea}
\par
%\textsuperscript{(1429.3)}
\textsuperscript{130:2.1} Jesús y sus amigos permanecieron en Cesarea más tiempo del que habían previsto, porque se descubrió que uno de los enormes remos que gobernaban la nave en la que pensaban embarcarse corría peligro de romperse. El capitán decidió permanecer en el puerto mientras fabricaban uno nuevo. Había escasez de carpinteros cualificados para esta tarea, y Jesús se ofreció voluntariamente para ayudar. Por las noches, Jesús y sus amigos paseaban por la hermosa muralla que servía de paseo alrededor del puerto. A Ganid le interesó mucho la explicación de Jesús sobre el sistema de canalización de las aguas de la ciudad y la técnica que utilizaban al emplear las mareas para lavar las calles y alcantarillas de la ciudad. El joven indio se sintió muy impresionado por el templo de Augusto, situado en una elevación y rematado con una estatua colosal del emperador romano. La segunda tarde de su estancia, los tres asistieron a un espectáculo en el enorme anfiteatro que podía contener veinte mil personas sentadas, y aquella misma noche fueron a ver una obra griega en el teatro. Estos eran los primeros espectáculos de este tipo que Ganid había visto en su vida, e hizo muchas preguntas a Jesús acerca de ellos. El tercer día por la mañana hicieron una visita oficial al palacio del gobernador, porque Cesarea era la capital de Palestina y la residencia del procurador romano.

\par
%\textsuperscript{(1429.4)}
\textsuperscript{130:2.2} En su posada también estaba alojado un mercader de Mongolia, y como este oriental hablaba bastante bien el griego, Jesús mantuvo varias largas conversaciones con él. Este hombre se quedó muy impresionado con la filosofía de vida de Jesús y no olvidó nunca sus sabias palabras sobre «la manera de vivir la vida celestial en la Tierra, sometiéndose diariamente a la voluntad del Padre celestial». Este mercader era taoísta, y por ello se había convertido en un firme creyente en la doctrina de una Deidad universal. Al regresar a Mongolia, empezó a enseñar estas verdades avanzadas a sus vecinos y a sus asociados en los negocios, y como resultado directo de estas actividades, su hijo mayor decidió hacerse sacerdote taoísta. Durante toda su vida, este joven ejerció una gran influencia en favor de la verdad avanzada; fue sucedido en esta vía por un hijo y un nieto, que también se consagraron fielmente a la doctrina del Dios Único ---el Soberano Supremo del Cielo.

\par
%\textsuperscript{(1430.1)}
\textsuperscript{130:2.3} Aunque la rama oriental de la iglesia cristiana primitiva, que tenía su centro en Filadelfia, permaneció más fiel a las enseñanzas de Jesús que la hermandad de Jerusalén, es lamentable que no hubiera nadie como Pedro que fuera a China, o como Pablo que viajara a la India, donde el terreno espiritual era entonces tan favorable para plantar la semilla del nuevo evangelio del reino\footnote{\textit{Evangelio del reino}: Mt 4:3; 9:35; 24:14; Mc 1:14-15.}. Estas mismas enseñanzas de Jesús, tal como las sostenían los filadelfianos, hubieran suscitado en las mentes de los pueblos asiáticos espiritualmente hambrientos el mismo interés inmediato y efectivo que las predicaciones de Pedro y de Pablo suscitaron en occidente.

\par
%\textsuperscript{(1430.2)}
\textsuperscript{130:2.4} Un día, uno de los jóvenes que trabajaban con Jesús en el remo del timón se mostró muy interesado por las palabras que este último dejaba caer de vez en cuando mientras trabajaban en el astillero. Cuando Jesús sugirió que el Padre que está en los cielos se interesaba por el bienestar de sus hijos en la Tierra, este joven griego llamado Anaxando dijo: «Si los Dioses se interesan por mí, entonces ¿por qué no quitan al capataz cruel e injusto que dirige este taller?». Se quedó sorprendido cuando Jesús replicó: «Puesto que conoces los caminos de la bondad y valoras la justicia, tal vez los Dioses han puesto a este hombre equivocado cerca de ti para que puedas guiarlo por ese camino mejor. Quizás tú eres la sal que puede hacer a este hermano más agradable para todos los demás hombres, es decir, si no has perdido tu sabor. Tal como están las cosas, este hombre es tu amo porque sus malos procedimientos te influyen desfavorablemente. ¿Por qué no afirmar tu dominio sobre el mal mediante el poder de la bondad, convirtiéndote así en el amo de todas las relaciones entre vosotros dos?. Puedo predecir que el bien que hay en ti podría vencer al mal que hay en él, si le dieras una oportunidad honrada y vivificante. En el transcurso de la existencia mortal no hay aventura más apasionante que la alegría de asociarse, en la vida material, con la energía espiritual y la verdad divina en una de sus luchas victoriosas contra el error y el mal. Es una experiencia maravillosa y transformadora la de convertirse en el canal viviente de la luz espiritual para los mortales que permanecen en las tinieblas espirituales. Si estás más favorecido por la verdad que este hombre, su necesidad debería ser un desafío para ti. ¡Seguramente no serás un cobarde, capaz de permanecer en la orilla del mar mirando cómo perece un compañero que no sabe nadar!. ¡Cuánto más valiosa es el alma de este hombre que se debate en las tinieblas, comparada con su cuerpo que se ahoga en el mar!».

\par
%\textsuperscript{(1430.3)}
\textsuperscript{130:2.5} Anaxando se sintió profundamente conmovido por las palabras de Jesús. No tardó en contar a su superior lo que Jesús le había dicho, y aquella misma noche los dos pidieron a Jesús que les aconsejara sobre el bienestar de sus almas. Mucho más tarde, después de haberse proclamado en Cesarea el mensaje cristiano, estos dos hombres, uno griego y el otro romano, creyeron en la predicación de Felipe\footnote{\textit{Predicación de Felipe}: Hch 8:40.} y se convirtieron en miembros influyentes de la iglesia fundada por él. Posteriormente, este joven griego fue nombrado intendente de un centurión romano llamado Cornelio\footnote{\textit{Conversión de Cornelio}: Hch 10:1-48.}, que se hizo creyente a través del ministerio de Pedro. Anaxando continuó aportando la luz a los que estaban en las tinieblas hasta la época en que Pablo fue encarcelado en Cesarea\footnote{\textit{Encarcelamiento de Pablo}: Hch 23:31-24:27.}. Pereció accidentalmente mientras socorría a los heridos y moribundos, durante la gran masacre en la que murieron veinte mil judíos.

\par
%\textsuperscript{(1431.1)}
\textsuperscript{130:2.6} Por esta época, Ganid empezó a darse cuenta de que su tutor empleaba sus ratos libres en este ministerio personal poco común hacia sus semejantes, y el joven indio decidió descubrir el motivo de estas actividades incesantes. Preguntó: «¿Por qué te ocupas continuamente en hablar con extraños?» Y Jesús respondió: «Ganid, ningún hombre es un extraño para el que conoce a Dios. En la experiencia de encontrar al Padre que está en los cielos, descubres que todos los hombres son tus hermanos, y ¿no es normal que uno sienta alegría al encontrarse con un hermano recién descubierto?. Conocer a nuestros hermanos y hermanas, comprender sus problemas y aprender a amarlos, es la experiencia suprema de la vida».

\par
%\textsuperscript{(1431.2)}
\textsuperscript{130:2.7} Fue una conversación que duró hasta bien entrada la noche, en el transcurso de la cual el joven pidió a Jesús que le explicara la diferencia entre la voluntad de Dios y el acto mental humano de elegir, que también se llama voluntad. En sustancia, Jesús dijo: La voluntad de Dios es el camino de Dios, el asociarse con la elección de Dios frente a cualquier alternativa potencial. En consecuencia, hacer la voluntad de Dios es la experiencia progresiva de parecerse cada vez más a Dios, y Dios es la fuente y el destino de todo lo que es bueno, bello y verdadero. La voluntad del hombre es el camino del hombre, la suma y la sustancia de lo que el mortal escoge ser y hacer. La voluntad es la elección deliberada de un ser auto-consciente, que conduce a una decisión y a un comportamiento basados en una reflexión inteligente.

\par
%\textsuperscript{(1431.3)}
\textsuperscript{130:2.8} Aquella tarde, Jesús y Ganid habían disfrutado jugando con un perro pastor muy inteligente, y Ganid quiso saber si el perro tenía alma, si tenía voluntad. En respuesta a sus preguntas, Jesús dijo: «El perro tiene una mente que puede conocer al hombre material, su dueño, pero no puede conocer a Dios, que es espíritu. Así pues, el perro no posee una naturaleza espiritual y no puede disfrutar de una experiencia espiritual. El perro puede tener una voluntad derivada de la naturaleza y acrecentada por el adiestramiento, pero este poder de la mente no es una fuerza espiritual, ni tampoco es comparable con la voluntad humana, porque no es \textit{reflexiva ---} no es el resultado de la discriminación de los significados superiores y morales, o de la elección de los valores espirituales y eternos. La posesión de estos poderes de discriminación espiritual y de elección de la verdad es lo que convierte al hombre mortal en un ser moral, en una criatura dotada de los atributos de la responsabilidad espiritual y del potencial de la supervivencia eterna». Jesús siguió explicando que la ausencia de estos poderes mentales en los animales es lo que hace imposible para siempre que el mundo animal pueda desarrollar un lenguaje en el tiempo, o experimentar algo equivalente a la supervivencia de la personalidad en la eternidad. Como consecuencia de la lección de este día, Ganid no creyó nunca más en la transmigración de las almas humanas a los cuerpos de los animales.

\par
%\textsuperscript{(1431.4)}
\textsuperscript{130:2.9} Al día siguiente, Ganid discutió de todo esto con su padre, y en respuesta a una pregunta de Gonod, Jesús explicó que «las voluntades humanas que se dedican exclusivamente a tomar decisiones temporales relacionadas con los problemas materiales de la existencia animal, están condenadas a perecer con el tiempo. Las que toman decisiones morales sinceras y efectúan elecciones espirituales incondicionales, se identifican así progresivamente con el espíritu interior y divino, y se van transformando cada vez más en valores de supervivencia eterna: una progresión sin fin de servicio divino».

\par
%\textsuperscript{(1431.5)}
\textsuperscript{130:2.10} Fue este mismo día cuando oímos por primera vez esa verdad capital que, enunciada en términos modernos, significaría: «La voluntad es esa manifestación de la mente humana que permite a la conciencia subjetiva expresarse objetivamente y experimentar el fenómeno de aspirar a ser semejante a Dios». Es en este mismo sentido como todo ser humano reflexivo e inclinado hacia el espíritu puede volverse \textit{creativo}.

\section*{3. En Alejandría}
\par
%\textsuperscript{(1432.1)}
\textsuperscript{130:3.1} La estancia en Cesarea había estado llena de acontecimientos; cuando el barco estuvo listo, Jesús y sus dos amigos zarparon un día al mediodía hacia Alejandría en Egipto.

\par
%\textsuperscript{(1432.2)}
\textsuperscript{130:3.2} La travesía fue sumamente agradable para los tres. Ganid estaba encantado con el viaje y mantenía ocupado a Jesús contestando a sus preguntas. Al acercarse al puerto de la ciudad, el joven se emocionó al ver el gran faro de Faros, situado en la isla que Alejandro había unido con la tierra firme a través de un dique, creando así dos magníficas ensenadas que hicieron de Alejandría la encrucijada comercial marítima de África, Asia y Europa. Este gran faro era una de las siete maravillas del mundo y el precursor de todos los faros posteriores. Por la mañana se levantaron temprano para contemplar este magnífico dispositivo salvavidas creado por el hombre, y en medio de las exclamaciones de Ganid, Jesús dijo: «Y tú, hijo mío, te parecerás a este faro cuando regreses a la India, incluso cuando tu padre descanse en paz. Serás como la luz de la vida para los que estén a tu alrededor en las tinieblas, mostrando a todos los que lo deseen el camino seguro para llegar al puerto de la salvación». Estrechando la mano de Jesús, Ganid le dijo: «Lo seré».

\par
%\textsuperscript{(1432.3)}
\textsuperscript{130:3.3} Subrayamos de nuevo que los primeros maestros de la religión cristiana cometieron un grave error al concentrarse exclusivamente en la civilización occidental del mundo romano. Las enseñanzas de Jesús, tal como las conservaban los creyentes mesopotámicos del siglo primero, hubieran sido recibidas de buena gana por los diversos grupos religiosos de Asia.

\par
%\textsuperscript{(1432.4)}
\textsuperscript{130:3.4} A las cuatro horas de desembarcar, ya estaban instalados cerca del extremo oriental de la gran avenida, de treinta metros de ancha y ocho kilómetros de larga, que llegaba hasta los límites occidentales de esta ciudad de un millón de habitantes. Después de echar una primera ojeada a las principales atracciones de la ciudad ---la universidad (museo), la biblioteca, el mausoleo real de Alejandro, el palacio, el templo de Neptuno, el teatro y el gimnasio--- Gonod se dedicó a sus negocios mientras que Jesús y Ganid se fueron a la biblioteca, la más grande del mundo. Aquí había cerca de un millón de manuscritos de todos los países civilizados: Grecia, Roma, Palestina, Partia, India, China e incluso Japón. En esta biblioteca, Ganid vio la mayor colección de literatura india de todo el mundo, y durante su estancia en Alejandría pasaron en este lugar un rato cada día. Jesús contó a Ganid que las escrituras hebreas habían sido traducidas al griego en este lugar. Discutieron una y otra vez de todas las religiones del mundo, y Jesús se esforzó en enseñar a esta mente joven la verdad que contenía cada una de ellas, añadiendo siempre: «Pero Yahvé es el Dios que surgió de las revelaciones de Melquisedek y del pacto con Abraham. Los judíos eran los descendientes de Abraham y ocuparon posteriormente la misma tierra en la que Melquisedek había vivido y enseñado, y desde la cual envió maestros a todo el mundo; y su religión acabó describiendo al Señor Dios de Israel como Padre Universal que está en los cielos, reconociéndolo de manera más clara que cualquier otra religión del mundo».

\par
%\textsuperscript{(1432.5)}
\textsuperscript{130:3.5} Bajo la dirección de Jesús, Ganid hizo una recopilación de las enseñanzas de todas las religiones del mundo que reconocían a una Deidad Universal, aunque pudieran admitir también otras deidades subordinadas. Después de muchas discusiones, Jesús y Ganid decidieron que los romanos no tenían ningún verdadero Dios en su religión, la cual no era mucho más que un culto al emperador. Llegaron a la conclusión de que los griegos tenían una filosofía, pero difícilmente una religión con un Dios personal. Descartaron los cultos de misterio debido a la confusión de su multiplicidad, y a que sus conceptos variados sobre la Deidad parecían derivarse de otras religiones y de religiones más antiguas.

\par
%\textsuperscript{(1433.1)}
\textsuperscript{130:3.6} Aunque estas traducciones se hicieron en Alejandría, Ganid no arregló definitivamente esta selección y añadió sus propias conclusiones personales hasta finales de su estancia en Roma. Se sorprendió mucho al descubrir que los mejores autores de literatura sagrada del mundo reconocían todos, más o menos claramente, la existencia de un Dios eterno, y estaban en gran parte de acuerdo en cuanto al carácter de este Dios y sus relaciones con el hombre mortal.

\par
%\textsuperscript{(1433.2)}
\textsuperscript{130:3.7} Jesús y Ganid pasaron mucho tiempo en el museo durante su estancia en Alejandría. Este museo no era una colección de objetos raros, sino más bien una universidad de bellas artes, ciencia y literatura. Profesores eruditos daban allí conferencias diarias, y en aquellos tiempos era el centro intelectual del mundo occidental. Día tras día, Jesús interpretaba las conferencias para Ganid. Cierto día, durante la segunda semana, el joven exclamó: «Maestro Josué, tú sabes más que todos estos profesores; deberías levantarte y decirles las grandes cosas que me has enseñado. Están confundidos porque piensan demasiado. Hablaré con mi padre para que arregle esto». Jesús sonrió y le dijo: «Eres un alumno admirativo, pero estos maestros no están dispuestos a que tú y yo les enseñemos nada. El orgullo de la erudición no espiritualizada es una trampa en la experiencia humana. El verdadero maestro mantiene su integridad intelectual permaneciendo siempre como un alumno».

\par
%\textsuperscript{(1433.3)}
\textsuperscript{130:3.8} Alejandría era el lugar donde se mezclaban las culturas occidentales, y la ciudad más grande y magnífica del mundo después de Roma. Aquí se encontraba la sinagoga judía más grande del mundo, con la sede administrativa del sanedrín de Alejandría, los setenta ancianos dirigentes.

\par
%\textsuperscript{(1433.4)}
\textsuperscript{130:3.9} Entre las muchas personas con quienes Gonod hizo transacciones mercantiles, había cierto banquero judío llamado Alejandro, cuyo hermano Filón era un famoso filósofo religioso de esta época. Filón había emprendido la tarea elogiable, pero extremadamente difícil, de armonizar la filosofía griega con la teología hebrea. Ganid y Jesús conversaron mucho sobre las enseñanzas de Filón y esperaban asistir a algunas de sus conferencias, pero durante toda su estancia en Alejandría este famoso judío helenista estuvo enfermo en la cama.

\par
%\textsuperscript{(1433.5)}
\textsuperscript{130:3.10} Jesús elogió a Ganid muchos aspectos de la filosofía griega y de la doctrina de los estoicos, pero le inculcó la verdad de que estos sistemas de creencias, así como las enseñanzas imprecisas de algunos compatriotas de Ganid, sólo eran religiones en el sentido de que conducían a los hombres a encontrar a Dios y a disfrutar la experiencia viviente de conocer al Eterno.

\section*{4. El discurso sobre la realidad}
\par
%\textsuperscript{(1433.6)}
\textsuperscript{130:4.1} La noche antes de partir de Alejandría, Ganid y Jesús tuvieron una larga conversación con uno de los profesores nombrados por el gobierno en la universidad, que daba una conferencia sobre las enseñanzas de Platón. Jesús hizo de intérprete para el erudito maestro griego, pero no insertó ninguna enseñanza propia que refutara la filosofía griega. Aquella noche, Gonod había salido para asuntos de negocios; por eso, después de la partida del profesor, el maestro y su alumno tuvieron una larga e íntima conversación sobre las doctrinas de Platón. Jesús aprobó de manera moderada algunas de las enseñanzas griegas sobre la teoría de que las cosas materiales del mundo eran vagos reflejos de las realidades espirituales invisibles, pero más sustanciales. Sin embargo, trató de establecer cimientos más sólidos para las reflexiones del joven, y por eso se embarcó en una larga disertación sobre la naturaleza de la realidad en el universo. He aquí en esencia y en lenguaje moderno lo que Jesús dijo a Ganid:

\par
%\textsuperscript{(1434.1)}
\textsuperscript{130:4.2} La fuente de la realidad universal es el Infinito. Las cosas materiales de la creación finita son las repercusiones espacio-temporales del Arquetipo Paradisíaco y de la Mente Universal del Dios eterno. La causalidad en el mundo físico, la conciencia de sí en el mundo intelectual y el yo progresivo en el mundo espiritual ---estas realidades, proyectadas a escala universal, combinadas en una conexión eterna y experimentadas con cualidades perfectas y valores divinos--- constituyen \textit{la realidad del Supremo}. Pero en el universo siempre cambiante, la Personalidad Original de la causalidad, de la inteligencia y de la experiencia espiritual permanece inmutable, absoluta. Incluso en un universo eterno de valores ilimitados y de cualidades divinas, todas las cosas pueden cambiar y cambian con frecuencia, excepto los Absolutos y aquello que ha alcanzado el estado físico, el contenido intelectual o la identidad espiritual que sean absolutos.

\par
%\textsuperscript{(1434.2)}
\textsuperscript{130:4.3} El nivel más alto que pueden alcanzar las criaturas finitas es el reconocimiento del Padre Universal y el conocimiento del Supremo. Incluso entonces, estos seres destinados a la finalidad continúan experimentando cambios en los movimientos del mundo físico y en sus fenómenos materiales. Asimismo, siguen siendo conscientes del progreso del yo en su continua ascensión por el universo espiritual, y experimentan una conciencia creciente de su apreciación cada vez más profunda del cosmos intelectual y de su reacción al mismo. La criatura solamente puede unificarse con el Creador mediante la perfección, la armonía y la unanimidad de la voluntad; este estado de divinidad sólo se puede alcanzar y mantener si la criatura continúa viviendo en el tiempo y en la eternidad conformando constantemente su voluntad personal finita a la voluntad divina del Creador. El deseo de hacer la voluntad del Padre siempre ha de ser supremo en el alma y debe dominar la mente de un hijo ascendente de Dios.

\par
%\textsuperscript{(1434.3)}
\textsuperscript{130:4.4} Un tuerto nunca podrá percibir la profundidad de una perspectiva. De la misma manera, los científicos materialistas tuertos y los místicos y alegoristas espirituales tuertos tampoco pueden tener una visión correcta, ni pueden comprender adecuadamente las verdaderas profundidades de la realidad universal. Todos los valores auténticos de la experiencia de la criatura están ocultos en la profundidad del reconocimiento.

\par
%\textsuperscript{(1434.4)}
\textsuperscript{130:4.5} Una causación desprovista de mente no puede transformar lo rudimentario y lo simple en elementos refinados y complejos; la experiencia sin el espíritu tampoco puede hacer que las mentes materiales de los mortales del tiempo se conviertan en caracteres divinos de supervivencia eterna. El único atributo del universo que caracteriza tan exclusivamente a la Deidad infinita es la perpetua donación creativa de la personalidad, que puede sobrevivir alcanzando progresivamente a la Deidad.

\par
%\textsuperscript{(1434.5)}
\textsuperscript{130:4.6} La personalidad es esa dotación cósmica, esa fase de la realidad universal, que puede coexistir con unos cambios ilimitados y al mismo tiempo conservar su identidad en presencia misma de todos esos cambios, e indefinidamente después de ellos.

\par
%\textsuperscript{(1434.6)}
\textsuperscript{130:4.7} La vida es una adaptación de la causalidad cósmica original a las exigencias y posibilidades de las situaciones universales; surge a la existencia mediante la acción de la Mente Universal y la activación de la chispa espiritual del Dios que es espíritu. El significado de la vida es su adaptabilidad; el valor de la vida es su capacidad para el progreso ---incluso hasta las alturas de la conciencia de Dios.

\par
%\textsuperscript{(1434.7)}
\textsuperscript{130:4.8} La mala adaptación de la vida autoconsciente al universo produce la desarmonía cósmica. Si la voluntad de la personalidad diverge definitivamente de la tendencia de los universos, termina en el aislamiento intelectual, en la segregación de la personalidad. La pérdida del piloto espiritual interior sobreviene con el cese espiritual de la existencia. Así pues, la vida inteligente y progresiva es, en sí misma y por sí misma, una prueba incontrovertible de la existencia de un universo intencional que expresa la voluntad de un Creador divino. Y esta vida, en su conjunto, lucha por alcanzar los valores superiores, teniendo como meta final al Padre Universal.

\par
%\textsuperscript{(1435.1)}
\textsuperscript{130:4.9} Aparte de los servicios superiores y casi espirituales del intelecto, la mente del hombre sólo sobrepasa el nivel animal en cuestión de grados. Por eso, los animales (que carecen de culto y de sabiduría) no pueden experimentar la superconciencia, la conciencia de la conciencia. La mente animal sólo es consciente del universo objetivo.

\par
%\textsuperscript{(1435.2)}
\textsuperscript{130:4.10} El conocimiento es la esfera de la mente material, la que discierne los hechos. La verdad es el dominio del intelecto espiritualmente dotado que es consciente de conocer a Dios. El conocimiento se puede demostrar; la verdad se experimenta. El conocimiento es una posesión de la mente; la verdad una experiencia del alma, del yo que progresa. El conocimiento es una función del nivel no espiritual; la verdad es una fase del nivel mental-espiritual de los universos. La visión de la mente material percibe un mundo de conocimiento basado en hechos; la visión del intelecto espiritualizado discierne un mundo de valores verdaderos. Estos dos puntos de vista, sincronizados y armonizados, revelan el mundo de la realidad, en el cual la sabiduría interpreta los fenómenos del universo en términos de experiencia personal progresiva.

\par
%\textsuperscript{(1435.3)}
\textsuperscript{130:4.11} El error (el mal) es la consecuencia de la imperfección. Las características de la imperfección, o los hechos de la mala adaptación, se revelan en el nivel material mediante la observación crítica y el análisis científico; en el nivel moral se revelan mediante la experiencia humana. La presencia del mal constituye la prueba de las inexactitudes de la mente y de la inmadurez del yo en evolución. Así pues, el mal es también una medida de la imperfección con que se interpreta el universo. La posibilidad de cometer errores es inherente a la adquisición de la sabiduría, el plan según el cual se progresa desde lo parcial y temporal a lo completo y eterno, desde lo relativo e imperfecto a lo definitivo y perfeccionado. El error es la sombra del estado incompleto relativo, que necesariamente debe proyectarse en medio del camino universal ascendente del hombre hacia la perfección del Paraíso. El error (el mal) no es una peculiaridad real del universo; es simplemente la observación de una relatividad en las relaciones entre la imperfección de lo finito incompleto y los niveles ascendentes del Supremo y del Último.

\par
%\textsuperscript{(1435.4)}
\textsuperscript{130:4.12} Aunque Jesús expuso todo esto al joven en el lenguaje más apropiado para su comprensión, Ganid tenía los párpados pesados al final de la explicación y pronto cayó presa del sueño. A la mañana siguiente, se levantaron temprano para subir a bordo del barco con rumbo a Lasea, en la isla de Creta. Pero antes de embarcarse, el muchacho aún tenía que hacer más preguntas sobre el mal, a las cuales Jesús respondió:

\par
%\textsuperscript{(1435.5)}
\textsuperscript{130:4.13} El mal es un concepto de la relatividad. Surge al observarse las imperfecciones que aparecen en la sombra proyectada por un universo finito de cosas y de seres, cuando este cosmos oscurece la luz viviente de la expresión universal de las realidades eternas del Uno Infinito.

\par
%\textsuperscript{(1435.6)}
\textsuperscript{130:4.14} El mal potencial es inherente al estado necesariamente incompleto de la revelación de Dios como expresión, limitada por el espacio-tiempo, de la infinidad y de la eternidad. El hecho de lo parcial en presencia de lo completo constituye la relatividad de la realidad; crea la necesidad de escoger intelectualmente, y establece unos niveles de valores en nuestra capacidad para reconocer y responder al espíritu. El concepto incompleto y finito que la mente temporal y limitada de la criatura posee del Infinito es, en sí mismo y por sí mismo, \textit{el mal potencial}. Pero el error cada vez mayor de no efectuar, injustificadamente, una rectificación espiritual razonable de estas desarmonías intelectuales e insuficiencias espirituales, originalmente inherentes, equivale a cometer \textit{el mal efectivo}.

\par
%\textsuperscript{(1436.1)}
\textsuperscript{130:4.15} Todos los conceptos estáticos y muertos son potencialmente malos. La sombra finita de la verdad relativa y viviente está en continuo movimiento. Los conceptos estáticos retrasan invariablemente la ciencia, la política, la sociedad y la religión. Los conceptos estáticos pueden representar cierto conocimiento, pero les falta sabiduría y están desprovistos de verdad. Sin embargo, no permitáis que el concepto de la relatividad os desoriente tanto que no podáis reconocer la coordinación del universo bajo la dirección de la mente cósmica, y su control estabilizado mediante la energía y el espíritu del Supremo.

\section*{5. En la isla de Creta}
\par
%\textsuperscript{(1436.2)}
\textsuperscript{130:5.1} Al ir a Creta, los viajeros no tenían otra intención que la de distraerse, pasear por la isla y escalar las montañas. Los cretenses de esta época no disfrutaban de una reputación envidiable entre los pueblos vecinos. Sin embargo, Jesús y Ganid consiguieron que muchas almas realzaran sus niveles de pensamiento y de vida, estableciendo así las bases para la rápida aceptación de las enseñanzas evangélicas posteriores, cuando llegaron los primeros predicadores de Jerusalén. Jesús amaba a estos cretenses, a pesar de las duras palabras que Pablo pronunció más tarde sobre ellos, cuando envió a Tito a la isla para reorganizar sus iglesias\footnote{\textit{Duras palabras de Pablo}: Tit 1:10-16.}.

\par
%\textsuperscript{(1436.3)}
\textsuperscript{130:5.2} En la ladera de una montaña de Creta, Jesús tuvo su primera larga conversación con Gonod sobre la religión. El padre se quedó muy impresionado y dijo: «No me extraña que el chico se crea todo lo que le dices; pero yo no sabía que tuvieran una religión así en Jerusalén, y mucho menos en Damasco». Fue durante la estancia en esta isla cuando Gonod propuso por primera vez a Jesús que fuera con ellos a la India, y Ganid estuvo encantado con la idea de que Jesús pudiera aceptar este arreglo.

\par
%\textsuperscript{(1436.4)}
\textsuperscript{130:5.3} Cierto día, Ganid preguntó a Jesús por qué no se había dedicado a enseñar públicamente, y éste le respondió: «Hijo mío, todo debe aguardar su hora. Has nacido en el mundo, pero ninguna cantidad de ansiedad y ninguna manifestación de impaciencia te ayudarán a crecer. En todos estos asuntos hay que darle tiempo al tiempo. Sólo el tiempo hace que la fruta verde madure en el árbol. Una estación sucede a la otra y el atardecer sigue al amanecer únicamente con el paso del tiempo. Ahora estoy camino de Roma con tu padre y contigo, y esto es suficiente por hoy. Mi mañana esta enteramente en las manos de mi Padre celestial». Entonces contó a Ganid la historia de Moisés y de sus cuarenta años de espera vigilante y de preparación continua.

\par
%\textsuperscript{(1436.5)}
\textsuperscript{130:5.4} Durante la visita a Buenos Puertos\footnote{\textit{Buenos Puertos}: Hch 27:8.} se produjo un incidente que Ganid no olvidó nunca. El recuerdo de este episodio siempre le despertó el deseo de hacer algo para cambiar el sistema de castas de su India natal. Un borracho degenerado estaba atacando a una joven esclava en la vía pública. Cuando Jesús vio el apuro de la chica, se abalanzó y alejó a la doncella del asalto del perturbado. Mientras la niña aterrorizada se agarraba a él, Jesús mantuvo al hombre enfurecido a una distancia prudencial con su poderoso brazo derecho extendido, hasta que el pobre tipo se agotó de tanto lanzar golpes furiosos en el aire. Ganid sintió el fuerte impulso de ayudar a Jesús a manejar este incidente, pero su padre se lo prohibió. Aunque no hablaban el idioma de la joven, ésta podía entender su acto de misericordia y les manifestó su profunda gratitud mientras los tres la acompañaban hasta su casa. En toda su vida encarnada, probablemente Jesús nunca estuvo tan cerca de pelearse con uno de sus contemporáneos como en esta ocasión. Aquella tarde le costó trabajo hacer entender a Ganid por qué no había golpeado al borracho. Ganid pensaba que este hombre debería haber recibido por lo menos tantos golpes como había dado a la joven.

\section*{6. El joven que tenía miedo}
\par
%\textsuperscript{(1437.1)}
\textsuperscript{130:6.1} Mientras estaban en las montañas, Jesús tuvo una larga conversación con un joven que estaba temeroso y abatido. No pudiendo encontrar ánimo y consuelo en la relación con sus semejantes, este joven había buscado la soledad de las colinas; había crecido con un sentimiento de desamparo e inferioridad. Estas tendencias naturales se habían visto acrecentadas por las numerosas circunstancias difíciles que el muchacho había sufrido a medida que crecía, principalmente la pérdida de su padre cuando tenía doce años. Al encontrarse con él, Jesús le dijo: «¡Saludos, amigo mío!, ¿por qué estás tan triste en un día tan hermoso?. Si ha sucedido algo que te aflija, quizás pueda ayudarte de alguna manera. En todo caso, es para mi un placer ofrecerte mis servicios».

\par
%\textsuperscript{(1437.2)}
\textsuperscript{130:6.2} El joven estaba poco dispuesto a hablar, por lo que Jesús intentó otra manera de acercarse a su alma, diciendo: «Comprendo que subas a estos montes para huir de la gente; por eso es natural que no quieras conversar conmigo, pero me gustaría saber si te son familiares estas colinas. ¿Conoces la dirección de estos senderos?. ¿Y podrías quizás indicarme cuál es el mejor camino para ir a Fénix?». El joven conocía muy bien aquellas montañas, y se interesó tanto en mostrar a Jesús el camino de Fénix, que dibujó en la tierra todos los senderos, explicándolos con todo detalle. Pero se quedó sorprendido y lleno de curiosidad cuando Jesús, después de decirle adiós y de hacer como el que se iba, se volvió repentinamente hacia él diciendo: «Sé muy bien que deseas quedarte a solas con tu desconsuelo; pero no sería ni amable ni justo por mi parte recibir de ti una ayuda tan generosa para encontrar el mejor camino de llegar a Fénix, y luego alejarme despreocupadamente sin hacer el menor esfuerzo por responder a tu petición de ayuda y orientación para encontrar el mejor camino hacia el destino que buscas en tu corazón mientras permaneces aquí en la ladera de la montaña. Al igual que tú conoces muy bien los senderos que conducen a Fénix, por haberlos recorrido muchas veces, yo conozco bien el camino de la ciudad de tus esperanzas frustradas y de tus ambiciones contrariadas. Y puesto que me has pedido ayuda, no te decepcionaré». El joven se quedó prácticamente atónito, y apenas logró balbucear: «Pero... si no te he pedido nada». Entonces Jesús, poniéndole suavemente la mano en el hombro, le dijo: «No, hijo, no con palabras, pero apelaste a mi corazón con tu mirada anhelante. Hijo mío, para el que ama a sus semejantes hay una elocuente petición de ayuda en tu actitud de desaliento y desesperación. Siéntate a mi lado mientras te hablo de los senderos del servicio y de los caminos de la felicidad, que conducen desde las penas del yo a las alegrías de las actividades afectuosas en la fraternidad de los hombres y en el servicio del Dios del cielo».

\par
%\textsuperscript{(1437.3)}
\textsuperscript{130:6.3} En aquel momento el joven sentía muchos deseos de hablar con Jesús, y se arrodilló a sus pies suplicándole que lo ayudara, que le mostrara el camino para escapar de su mundo de penas y fracasos personales. Jesús le dijo: «Amigo mío, ¡levántate!. ¡Ponte de pie como un hombre!. Puedes estar rodeado de enemigos mezquinos y muchos obstáculos pueden retrasar tu marcha, pero las cosas importantes y reales de este mundo y del universo están de tu parte. El Sol sale todas las mañanas para saludarte exactamente igual que lo hace para el hombre más poderoso y próspero de la Tierra. Mira ---tienes un cuerpo fuerte y músculos poderosos--- tus facultades físicas son superiores a la media. Por supuesto, todo eso no sirve prácticamente para nada mientras te quedes aquí sentado en la ladera de la montaña lamentándote de tus desgracias, reales o imaginarias. Pero podrías hacer grandes cosas con tu cuerpo si quisieras apresurarte hacia los lugares donde hay grandes cosas por hacerse. Tratas de huir de tu yo infeliz, pero eso no es posible. Tú y los problemas de tu vida son reales; no puedes huir de ellos mientras estés vivo. Pero mira además, tu mente es clara y capaz. Tu cuerpo robusto tiene una mente inteligente que lo dirige. Pon tu mente a trabajar para resolver sus problemas; enseña a tu intelecto a trabajar para ti. No te dejes dominar por el miedo como un animal sin discernimiento. Tu mente debería ser tu valiente aliada en la resolución de los problemas de tu vida, en lugar de ser tú, como lo has sido, su abyecto esclavo atemorizado y el siervo de la depresión y de la derrota. Pero lo más valioso de todo, tu verdadero potencial de realización, es el espíritu que vive dentro de ti; él estimulará e inspirará tu mente para que se controle a sí misma y active a tu cuerpo si deseas liberarlo de las cadenas del miedo; así permitirás que tu naturaleza espiritual comience a liberarte de los males de la indolencia, gracias a la presencia y al poder de la fe viviente. Verás entonces cómo esta fe vencerá tu miedo a los hombres mediante la presencia irresistible de ese nuevo y predominante \textit{amor por tus semejantes} que pronto llenará tu alma hasta rebosar, porque en tu corazón habrá nacido la conciencia de que eres un hijo de Dios».

\par
%\textsuperscript{(1438.1)}
\textsuperscript{130:6.4} «En este día, hijo mío, has de nacer de nuevo, restablecido como un hombre de fe, de valor y de servicio consagrado a los hombres por amor a Dios. Cuando te hayas reajustado así a la vida, dentro de ti mismo, también te habrás reajustado con el universo; habrás nacido de nuevo ---nacido del espíritu--- y en adelante toda tu vida será una consecución victoriosa. Los problemas te fortificarán, las decepciones te espolearán, las dificultades serán un desafío y los obstáculos, un estímulo. ¡Levántate, joven!. Di adiós a la vida de temores serviles y de huidas cobardes. Regresa rápidamente a tu deber y vive tu vida en la carne como un hijo de Dios, como un mortal dedicado al servicio ennoblecedor del hombre en la Tierra, y destinado al magnífico y perpetuo servicio de Dios en la eternidad».

\par
%\textsuperscript{(1438.2)}
\textsuperscript{130:6.5} Este joven, llamado Fortunato, se convirtió más tarde en el jefe de los cristianos de Creta y en el íntimo asociado de Tito en sus esfuerzos por elevar a los creyentes cretenses\footnote{\textit{Creyentes cretenses}: Tit 1:5.}.

\par
%\textsuperscript{(1438.3)}
\textsuperscript{130:6.6} Los viajeros estaban realmente descansados y dispuestos cuando un buen día, a mediodía, se prepararon para zarpar hacia Cartago, en el norte de África, deteniéndose dos días en Cirene. Es aquí donde Jesús y Ganid prestaron sus primeros auxilios a un muchacho llamado Rufo, que había resultado herido al desplomarse una carreta de bueyes cargada. Lo llevaron a la casa de su madre, y en cuanto a su padre, Simón, jamás podía imaginar que el hombre cuya cruz llevaría más tarde, por orden de un soldado romano, era el mismo extranjero que en otro tiempo había socorrido a su hijo\footnote{\textit{Simón el portador de la cruz}: Mt 27:32; Mc 15:21; Lc 23:26.}.

\section*{7. En Cartago --- el discurso sobre el tiempo y el espacio}
\par
%\textsuperscript{(1438.4)}
\textsuperscript{130:7.1} Durante la ruta hacia Cartago, Jesús pasó la mayoría del tiempo conversando con sus compañeros de viaje sobre temas sociales, políticos y comerciales, pero no se dijo casi nada sobre religión. Por primera vez, Gonod y Ganid descubrieron que Jesús era un buen narrador, y lo mantuvieron ocupado contando anécdotas de sus primeros años de vida en Galilea. También se enteraron de que se había criado en Galilea y no en Jerusalén ni en Damasco.

\par
%\textsuperscript{(1438.5)}
\textsuperscript{130:7.2} Ganid había notado que la mayoría de las personas que habían encontrado por casualidad se sentían atraídas por Jesús, y por ello preguntó qué tenía uno que hacer para ganar amigos. Su maestro le dijo: «Interésate por tus semejantes; aprende a amarlos y vigila la oportunidad de hacer algo por ellos que estás seguro que desean»; luego citó el antiguo proverbio judío: «Un hombre que quiere tener amigos debe mostrarse amistoso»\footnote{\textit{Un hombre que quiere amigos debe ser amistoso}: Pr 18:24.}.

\par
%\textsuperscript{(1439.1)}
\textsuperscript{130:7.3} En Cartago, Jesús tuvo una larga conversación memorable con un sacerdote mitríaco sobre la inmortalidad, el tiempo y la eternidad. Este persa se había educado en Alejandría y deseaba realmente aprender de Jesús. En respuesta a sus numerosas preguntas, y traducido a terminología moderna, Jesús dijo en sustancia lo siguiente:

\par
%\textsuperscript{(1439.2)}
\textsuperscript{130:7.4} El tiempo es la corriente de los acontecimientos temporales que fluyen, percibidos por la conciencia de la criatura. El tiempo es un nombre que se ha dado al orden en que suceden los acontecimientos, que permite reconocerlos y separarlos. El universo del espacio es un fenómeno relacionado con el tiempo cuando es observado desde cualquier posición interior fuera de la morada fija del Paraíso. El movimiento del tiempo sólo se revela en relación con algo que no se mueve en el espacio como un fenómeno del tiempo. En el universo de universos, el Paraíso y sus Deidades trascienden tanto el tiempo como el espacio. En los mundos habitados, la personalidad humana (habitada y orientada por el espíritu del Padre Paradisiaco) es la única realidad relacionada con lo físico que puede trascender la secuencia material de los acontecimientos temporales.

\par
%\textsuperscript{(1439.3)}
\textsuperscript{130:7.5} Los animales no perciben el tiempo como el hombre, e incluso para el hombre, debido a su punto de vista fragmentario y circunscrito, el tiempo aparece como una sucesión de acontecimientos; pero a medida que el hombre asciende, que progresa interiormente, su visión de esta procesión de acontecimientos aumenta de tal manera que la discierne cada vez más en su totalidad. Lo que anteriormente aparecía como una sucesión de acontecimientos se verá ahora como un ciclo completo y perfectamente relacionado; de esta manera, la simultaneidad circular desplazará cada vez más a la antigua conciencia de la secuencia lineal de los acontecimientos.

\par
%\textsuperscript{(1439.4)}
\textsuperscript{130:7.6} Hay siete conceptos diferentes del espacio tal como está condicionado por el tiempo. El espacio se mide por el tiempo y no el tiempo por el espacio. La confusión de los científicos surge de que no logran reconocer la realidad del espacio. El espacio no es simplemente un concepto intelectual de la variación en la conexión de los objetos del universo. El espacio no está vacío, y la mente es la única cosa que el hombre conoce que puede trascender, aunque sea parcialmente, el espacio. La mente puede funcionar independientemente del concepto de la conexión espacial de los objetos materiales. El espacio es relativa y comparativamente finito para todos los seres con estatus de criatura. Cuanto más se aproxima la conciencia a la noción de las siete dimensiones cósmicas, el concepto de espacio potencial se aproxima más a la ultimidad. Pero el potencial del espacio sólo es realmente último en el nivel absoluto.

\par
%\textsuperscript{(1439.5)}
\textsuperscript{130:7.7} Debe ser evidente que la realidad universal tiene un significado siempre relativo y en expansión en los niveles ascendentes y en vías de perfeccionamiento del cosmos. A fin de cuentas, los mortales sobrevivientes alcanzan la identidad en un universo de siete dimensiones.

\par
%\textsuperscript{(1439.6)}
\textsuperscript{130:7.8} El concepto espacio-temporal de una mente de origen material está destinado a sufrir ampliaciones sucesivas a medida que la personalidad consciente que lo concibe asciende los niveles del universo. Cuando el hombre alcanza la mente que media entre los planos material y espiritual de existencia, sus ideas del espacio-tiempo se amplían enormemente en cuanto a la calidad de percepción y a la cantidad de experiencia. Los conceptos cósmicos crecientes de una personalidad espiritual que progresa se deben al aumento tanto de la profundidad de la perspicacia como del campo de la conciencia. A medida que la personalidad continúa su camino hacia arriba y hacia el interior hasta los niveles trascendentales de semejanza con la Deidad, el concepto del espacio-tiempo se acercará cada vez más a los conceptos sin tiempo y sin espacio de los Absolutos. Relativamente, y según sus logros trascendentales, los hijos con destino último llegarán a percibir estos conceptos del nivel absoluto.

\section*{8. En el camino a Neápolis y Roma}
\par
%\textsuperscript{(1440.1)}
\textsuperscript{130:8.1} La primera escala en el camino de Italia era la isla de Malta. Jesús tuvo aquí una larga conversación con un joven abatido y desanimado llamado Claudo. Este muchacho había pensado en quitarse la vida, pero cuando terminó de conversar con el escriba de Damasco, dijo: «Voy a afrontar la vida como un hombre; basta ya de hacer el cobarde. Voy a volver con mi gente y empezar de nuevo». Poco tiempo después se convirtió en un predicador entusiasta de los cínicos, y más tarde aún se unió a Pedro para proclamar el cristianismo en Roma y en Nápoles. Después de la muerte de Pedro fue a España a predicar el evangelio, pero no supo nunca que el hombre que lo había inspirado en Malta era el mismo Jesús a quien posteriormente proclamó como Liberador del mundo.

\par
%\textsuperscript{(1440.2)}
\textsuperscript{130:8.2} En Siracusa pasaron una semana completa. El acontecimiento más notable de esta escala fue la rehabilitación de Esdras, el judío descarriado, que tenía la taberna donde Jesús y sus compañeros se habían hospedado. A Esdras le encantó la facilidad de trato de Jesús y le pidió que lo ayudara a volver a la fe de Israel. Expresó su desesperanza diciendo: «Quiero ser un verdadero hijo de Abraham, pero no consigo encontrar a Dios». Jesús le dijo: «Si quieres realmente encontrar a Dios, ese deseo es en sí mismo la prueba de que ya lo has encontrado\footnote{\textit{Busca y encontrarás}: Mt 7:7-8; Lc 11:9-10.}. Tu problema no es que no puedas encontrar a Dios, porque el Padre ya te ha encontrado; tu problema es simplemente que no conoces a Dios. ¿Acaso no has leído en el profeta Jeremías: `Me buscarás y me encontrarás cuando me busques con todo tu corazón'?\footnote{\textit{Me buscarás y me encontrarás}: Jer 29:13.}. Y además, ¿no dice también este mismo profeta: `Te daré un corazón para que me conozcas, que yo soy el Señor, y tú pertenecerás a mi pueblo, y yo seré tu Dios'?\footnote{\textit{Te daré un corazón para que me conozcas}: Jer 24:7.}. ¿Y no has leído también en las escrituras donde dice: `Él mira a los hombres, y si alguno dijera: He pecado y he pervertido lo que era justo, y no me ha aprovechado, entonces Dios liberará de las tinieblas el alma de ese hombre, y verá la luz'?\footnote{\textit{Él mira a los hombres}: Job 33:27-28.}. Entonces Esdras encontró a Dios para satisfacción de su alma. Posteriormente, en asociación con un próspero prosélito griego, este judío construyó la primera iglesia cristiana de Siracusa.

\par
%\textsuperscript{(1440.3)}
\textsuperscript{130:8.3} En Mesina se detuvieron un solo día, pero lo suficiente como para cambiar la vida de un muchacho, un vendedor de frutas; Jesús le compró frutas y a su vez lo alimentó con el pan de la vida. El muchacho no olvidó nunca las palabras de Jesús y la bondadosa mirada que las acompañó cuando, apoyando la mano sobre su hombro, le dijo: «Adiós, hijo mío, sé valiente mientras te haces hombre, y después de alimentar el cuerpo, aprende también a alimentar el alma. Mi Padre que está en los cielos estará contigo y te guiará». El muchacho se hizo devoto de la religión mitríaca, y posteriormente se convirtió a la fe cristiana.

\par
%\textsuperscript{(1440.4)}
\textsuperscript{130:8.4} Por fin llegaron a Nápoles, y tuvieron el sentimiento de que ya no estaban lejos de su destino final, Roma. Gonod tenía muchos negocios que tratar en Nápoles; aparte del tiempo en que Jesús era necesario como intérprete, él y Ganid dedicaron sus ratos libres a visitar y explorar la ciudad. Ganid se estaba haciendo experto en detectar a aquellos que parecían necesitar ayuda. Encontraron mucha pobreza en esta ciudad y distribuyeron muchas limosnas. Pero Ganid nunca comprendió el significado de las palabras de Jesús cuando le vio dar, en la calle, una moneda a un mendigo, y se negó a detenerse y consolar al hombre. Jesús dijo: «¿Por qué malgastar palabras con alguien que no puede percibir el significado de lo que dices? El espíritu del Padre no puede enseñar y salvar a alguien que no tiene capacidad para la filiación». Jesús quería decir que el hombre no tenía una mente normal, que carecía de la facultad de responder a la guía del espíritu.

\par
%\textsuperscript{(1441.1)}
\textsuperscript{130:8.5} En Nápoles no tuvo lugar ninguna experiencia sobresaliente; Jesús y el joven recorrieron toda la ciudad y repartieron buen ánimo con muchas sonrisas a centenares de hombres, mujeres y niños.

\par
%\textsuperscript{(1441.2)}
\textsuperscript{130:8.6} Desde aquí siguieron hacia Roma por el camino de Capua, donde permanecieron tres días. Por la Vía Apia continuaron su viaje en dirección a Roma junto a sus animales de carga, ansiosos los tres por ver a esta dueña del imperio, la ciudad más grande del mundo entero.