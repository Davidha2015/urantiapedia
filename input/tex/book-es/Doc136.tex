\chapter{Documento 136. El bautismo y los cuarenta días}
\par
%\textsuperscript{(1509.1)}
\textsuperscript{136:0.1} JESÚS comenzó su ministerio público cuando el interés popular por la predicación de Juan estaba en su apogeo y en la época en que el pueblo judío de Palestina esperaba ansiosamente la aparición del Mesías\footnote{\textit{El pueblo expectante del Mesías}: Lc 3:15; Jn 1:25.}. Había un gran contraste entre Juan y Jesús. Juan era un obrero ardiente y severo, mientras que Jesús era un trabajador tranquilo y feliz; en toda su vida, sólo unas pocas veces se le vio apresurarse. Jesús era un consuelo reconfortante para el mundo y en cierto modo un ejemplo. Juan apenas era un consuelo o un ejemplo; predicaba el reino de los cielos, pero no participaba mucho de su felicidad. Aunque Jesús se refería a Juan como el más grande de los profetas\footnote{\textit{Juan, el mayor profeta}: Mt 11:11; Lc 7:28.} del antiguo orden, también decía que el más humilde de los que vieran la gran luz del nuevo camino, y entrara por allí en el reino de los cielos, era en verdad más grande que Juan.

\par
%\textsuperscript{(1509.2)}
\textsuperscript{136:0.2} Cuando Juan predicaba el reino venidero, lo esencial de su mensaje era: «¡Arrepentíos!. Huid de la cólera inminente». Cuando Jesús empezó a predicar, mantuvo la exhortación al arrepentimiento, pero este mensaje estaba siempre ligado al evangelio, a la buena nueva de la alegría y de la libertad del nuevo reino\footnote{\textit{El mensaje del evangelio}: Lc 8:1.}.

\section*{1. Los conceptos del Mesías esperado}
\par
%\textsuperscript{(1509.3)}
\textsuperscript{136:1.1} Los judíos poseían diversas ideas sobre el libertador esperado, y cada una de estas diferentes escuelas de enseñanza mesiánica podía citar pasajes de las escrituras hebreas como prueba de sus argumentos. De manera general, los judíos consideraban que su historia nacional empezaba con Abraham y culminaría con el Mesías y la nueva era del reino de Dios. En los siglos anteriores habían concebido a este libertador como «el siervo del Señor», luego como «el Hijo del Hombre», mientras que más recientemente algunos incluso habían llegado a referirse al Mesías como el «Hijo de Dios». Pero, sin importar que le llamaran «la semilla de Abraham» o «el hijo de David», todos estaban de acuerdo en que tenía que ser el Mesías, el «ungido». Así pues, el concepto evolucionó desde «siervo del Señor» a «hijo de David», y de «Hijo del Hombre» a «Hijo de Dios»\footnote{\textit{El Siervo del Señor}: Is 42:1; 53:11; Ez 34:22-31. \textit{El Hijo del Hombre}: Ez 2:1,3,6,8; 3:1-4,10,17; Mt 8:20; Dn 7:13,14; Mc 2:10; Lc 5:24; Jn 1:51; Ap 1:13; 14:14; 1 Hen 46:1-6; 48:1-7; 60:10; 62:1-14; 63:11; 69:26-29; 70:1-2; 71:14-16. \textit{Semilla de Abraham}: Gn 18:18; Hch 3:25; Ro 4:13. \textit{Hijo de David}: Mt 1:1; 2 Sam 7:12-13. \textit{El Mesías}: Dn 9:25-26; Jn 1:41; 4:25-26. \textit{El Ungido}: Is 61:1; Lc 4:18; Hch 10:38. \textit{Jesús, Hijo de Dios}: Mt 8:29; 14:33; 16:15-16; 27:54; Mc 1:1; 3:11; 15:39; Lc 1:35; 4:41; Jn 1:34,49; 3:16-18; 10:36; 20:31; Hch 8:37.}.

\par
%\textsuperscript{(1509.4)}
\textsuperscript{136:1.2} En los tiempos de Juan y de Jesús, los judíos más cultos habían desarrollado la idea del Mesías venidero como que sería un israelita perfeccionado y representativo, que reuniría en sí mismo como «siervo del Señor» el triple cargo de profeta, sacerdote y rey.

\par
%\textsuperscript{(1509.5)}
\textsuperscript{136:1.3} Los judíos creían devotamente que, al igual que Moisés había liberado a sus padres de la esclavitud egipcia mediante prodigios milagrosos, el Mesías esperado liberaría al pueblo judío de la dominación romana mediante milagros de poder aún más grandes y maravillas de triunfo racial. Los rabinos habían reunido casi quinientos pasajes de las Escrituras que, a pesar de sus contradicciones aparentes, eran profecías, según ellos, del advenimiento del Mesías. En medio de todos estos detalles de tiempo, de técnicas y de funciones, casi perdieron de vista por completo la \textit{personalidad} del Mesías prometido. Esperaban el restablecimiento de la gloria nacional judía ---la exaltación temporal de Israel--- en lugar de la salvación del mundo. Es evidente pues que Jesús de Nazaret no podría satisfacer nunca este concepto mesiánico materialista de la mente judía. Si los judíos hubieran sabido ver estos pronunciamientos proféticos bajo una luz diferente, muchas de sus predicciones supuestamente mesiánicas hubieran preparado sus mentes de manera muy natural para reconocer en Jesús a aquel que cerraría una era e inauguraría una dispensación de misericordia y de salvación nueva y mejor para todas las naciones.

\par
%\textsuperscript{(1510.1)}
\textsuperscript{136:1.4} Los judíos habían sido educados en la creencia de la doctrina de la \textit{Shekinah}. Pero este pretendido símbolo de la Presencia Divina no estaba visible en el templo. Creían que la venida del Mesías efectuaría su restablecimiento. Tenían ideas confusas sobre el pecado racial y la supuesta naturaleza maligna del hombre. Algunos enseñaban que el pecado de Adán había causado la maldición de la raza humana, y que el Mesías destruiría esa maldición y restituiría al hombre en el favor divino. Otros enseñaban que al crear al hombre, Dios había puesto dentro de su ser una naturaleza buena y otra mala; que cuando observó el resultado de esta combinación, se había desilusionado mucho, y que «se arrepintió de haber creado así al hombre»\footnote{\textit{Dios arrepentido de crear al hombre}: Gn 6:6.}. Los que enseñaban esto creían que el Mesías tenía que venir para redimir al hombre de esta naturaleza maligna innata.

\par
%\textsuperscript{(1510.2)}
\textsuperscript{136:1.5} La mayoría de los judíos creía que continuaban languideciendo bajo el poder romano debido a sus pecados nacionales y a la frialdad de los prosélitos gentiles. La nación judía no se había \textit{arrepentido} de todo corazón; por eso el Mesías retrasaba su llegada. Se hablaba mucho de arrepentimiento, lo que explica la atracción poderosa e inmediata de la predicación de Juan: «Arrepentíos y sed bautizados, porque el reino de los cielos está cerca»\footnote{\textit{Arrepentíos y bautizaos}: Mt 3:2.}. Y para cualquier judío piadoso, el reino de los cielos sólo podía significar una cosa: la venida del Mesías.

\par
%\textsuperscript{(1510.3)}
\textsuperscript{136:1.6} La donación de Miguel contenía una característica que era completamente ajena al concepto judío del Mesías; esta característica era la \textit{unión} de las dos naturalezas: la humana y la divina. Los judíos habían concebido al Mesías de distintas maneras: como humano perfeccionado, como superhumano e incluso como divino, pero nunca habían pensado en el concepto de la \textit{unión} de lo humano y lo divino. Este fue el gran escollo de los primeros discípulos de Jesús. Captaban el concepto humano del Mesías como hijo de David\footnote{\textit{El Mesías como hijo de David}: 2 Sam 7:12-13.}, tal como había sido presentado por los primeros profetas; también comprendían al Mesías como Hijo del Hombre\footnote{\textit{El Mesías como Hijo del Hombre}: Dn 7:13-14.}, la idea superhumana de Daniel y de algunos de los últimos profetas, e incluso como Hijo de Dios, tal como lo habían descrito el autor del Libro de Enoc y algunos de sus contemporáneos. Pero nunca llegaron a considerar, ni por un solo instante, el verdadero concepto de la unión, en una sola personalidad terrestre, de las dos naturalezas: la humana y la divina. La encarnación del Creador en forma de criatura no había sido revelada de antemano. Sólo fue revelada en Jesús\footnote{\textit{Dios revelado cuando se encarnó}: Jn 1:1-5.}. El mundo no sabía nada de estas cosas hasta que el Hijo Creador se hizo carne y habitó entre los mortales del planeta.

\section*{2. El bautismo de Jesús}
\par
%\textsuperscript{(1510.4)}
\textsuperscript{136:2.1} Jesús fue bautizado\footnote{\textit{El bautismo de Jesús}: Mt 3:13-17; Mc 1:9; Lc 3:21.} en el apogeo de la predicación de Juan, cuando Palestina estaba inflamada con la esperanza de su mensaje ---«el reino de Dios está cerca»\footnote{\textit{El reino de Dios está cerca}: Dn 2:44; Mt 3:2.}--- y todo el pueblo judío se dedicaba a un análisis de sí mismo serio y solemne. El sentido judío de la solidaridad racial era muy profundo. Los judíos no sólo creían que los pecados de un padre podían afectar a sus hijos, sino que también creían firmemente que el pecado de un individuo podía maldecir a la nación. Por consiguiente, no todos los que se sometían al bautismo de Juan se consideraban culpables de los pecados específicos que Juan denunciaba. Muchas almas piadosas eran bautizadas por Juan para el bien de Israel; temían que un pecado de ignorancia por su parte pudiera retrasar la venida del Mesías. Sentían que pertenecían a una nación culpable y maldita por el pecado\footnote{\textit{Se sentían como una nación culpable}: Dn 9:11.}, y se sometían al bautismo para manifestar de este modo los frutos de una penitencia racial. Por lo tanto, es evidente que Jesús no recibió de ninguna manera el bautismo de Juan como rito de arrepentimiento o para la remisión de los pecados. Al aceptar el bautismo de manos de Juan, Jesús no hacía más que seguir el ejemplo de muchos israelitas piadosos.

\par
%\textsuperscript{(1511.1)}
\textsuperscript{136:2.2} Cuando Jesús de Nazaret bajó al Jordán para ser bautizado, era un mortal del mundo que había alcanzado el pináculo de la ascensión evolutiva humana en todos los aspectos relacionados con la conquista de la mente y la identificación del yo con el espíritu. Ese día, estuvo de pie en el Jordán como un mortal perfeccionado de los mundos evolutivos del tiempo y del espacio. Una sincronía perfecta y una comunicación plena se habían establecido entre la mente mortal de Jesús y su Ajustador espiritual interior, el don divino de su Padre Paradisiaco. Desde la ascensión de Miguel a la jefatura de su universo, un Ajustador como éste reside en todos los seres normales que viven en Urantia, excepto que el Ajustador de Jesús había sido preparado previamente para esta misión especial, habiendo habitado de manera similar en Maquiventa Melquisedek, otro superhumano encarnado en la similitud de la carne mortal.

\par
%\textsuperscript{(1511.2)}
\textsuperscript{136:2.3} Ordinariamente, cuando un mortal del planeta alcanza estos altos niveles de perfección de la personalidad, se producen esos fenómenos preliminares de elevación espiritual que culminan finalmente en la fusión definitiva del alma madura del mortal con su Ajustador divino asociado. Aparentemente, un cambio de esta naturaleza debía producirse en la experiencia de la personalidad de Jesús de Nazaret el mismo día que descendió al Jordán con sus dos hermanos para ser bautizado por Juan. Esta ceremonia era el acto final de su vida puramente humana en Urantia, y muchos observadores superhumanos esperaban presenciar la fusión del Ajustador con la mente que habitaba, pero todos estaban destinados a sufrir una desilusión. Ocurrió algo nuevo y mucho más grandioso. Mientras Juan imponía sus manos sobre Jesús para bautizarlo, el Ajustador residente se despidió para siempre del alma humana perfeccionada de Josué ben José. Unos instantes después, esta entidad divina regresó de Divinington como Ajustador Personalizado\footnote{\textit{Ajustador Personalizado}: Mt 3:16-17; Mc 1:10-11; Lc 3:22; Jn 1:32-33.} y jefe de sus semejantes en todo el universo local de Nebadon. Jesús pudo así observar a su propio espíritu divino anterior regresar y descender sobre él de forma personalizada. Y entonces oyó hablar a este mismo espíritu originario del Paraíso, que decía: «Éste es mi Hijo amado en quien tengo complacencia»\footnote{\textit{Éste es mi Hijo amado}: Mt 3:17; Mc 1:11; Lc 3:22.}. Juan y los dos hermanos de Jesús también oyeron estas palabras. Los discípulos de Juan, que estaban al borde del agua, no las oyeron ni tampoco vieron la aparición del Ajustador Personalizado. Sólo los ojos de Jesús contemplaron al Ajustador Personalizado.

\par
%\textsuperscript{(1511.3)}
\textsuperscript{136:2.4} Cuando el Ajustador Personalizado ahora ensalzado que había regresado hubo hablado así, todo fue silencio. Y mientras los cuatro interesados permanecían en el agua, Jesús levantó la mirada hacia el cercano Ajustador y oró: «Padre mío que reinas en el cielo, santificado sea tu nombre. ¡Que venga tu reino!. Que tu voluntad se haga en la Tierra, así como se hace en el cielo»\footnote{\textit{La oración de Jesús}: Mt 6:9-10; Lc 11:2.}. Cuando terminó de orar, «se abrieron los cielos»\footnote{\textit{Se abrieron los cielos}: Mt 3:16; Mc 1:10; Lc 3:21.}, y el Hijo del Hombre contempló la imagen de sí mismo como Hijo de Dios, presentada por el Ajustador ahora Personalizado, tal como era antes de venir a la Tierra en la similitud de la carne mortal, y tal como volvería a ser cuando terminara su vida encarnada. Jesús fue el único que presenció esta visión celestial.

\par
%\textsuperscript{(1512.1)}
\textsuperscript{136:2.5} Lo que Juan y Jesús oyeron fue la voz del Ajustador Personalizado\footnote{\textit{Voz del Ajustador Personalizado}: Mt 3:17; Mc 1:11; Lc 3:22.}, hablando en nombre del Padre Universal, porque el Ajustador proviene del Padre Paradisiaco y es semejante a él. Durante el resto de la vida terrenal de Jesús, este Ajustador Personalizado estuvo asociado con él en todas sus obras; Jesús permaneció en constante comunión con este Ajustador ensalzado.

\par
%\textsuperscript{(1512.2)}
\textsuperscript{136:2.6} Cuando Jesús fue bautizado, no se arrepintió de ninguna mala acción y no hizo ninguna confesión de pecado. Se trataba de un bautismo de consagración a la realización de la voluntad del Padre celestial. En su bautismo escuchó la llamada inequívoca de su Padre, la citación final para que se ocupara de los asuntos de su Padre, y se retiró a solas durante cuarenta días para meditar sobre estos múltiples problemas. Al retirarse así durante cierto tiempo de todo contacto personal activo con sus asociados terrenales, Jesús, tal como era y en Urantia, estaba siguiendo el mismo procedimiento que prevalece en los mundos morontiales, cuando un mortal ascendente fusiona con la presencia interior del Padre Universal.

\par
%\textsuperscript{(1512.3)}
\textsuperscript{136:2.7} Este día de bautismo marcó el final de la vida puramente humana de Jesús. El Hijo divino ha encontrado a su Padre, el Padre Universal ha encontrado a su Hijo encarnado, y hablan el uno con el otro.

\par
%\textsuperscript{(1512.4)}
\textsuperscript{136:2.8} (Jesús tenía casi treinta y un años y medio cuando fue bautizado. Aunque Lucas dice que fue bautizado en el decimoquinto año del reinado de Tiberio César, lo que nos daría el año 29 puesto que Augusto murió en el año 14, hay que recordar que Tiberio fue coemperador con Augusto durante dos años y medio antes de la muerte de este último, habiéndose acuñado monedas en su honor en octubre del año 11. El decimoquinto año de su reinado efectivo fue, por tanto, este mismo año 26, el del bautismo de Jesús. Éste fue también el año en que Poncio Pilatos empezó a mandar como gobernador de Judea.)\footnote{\textit{Cronología de eventos}: Lc 3:1.}

\section*{3. Los cuarenta días}
\par
%\textsuperscript{(1512.5)}
\textsuperscript{136:3.1} Jesús había soportado la gran tentación de su donación como mortal antes de su bautismo cuando el rocío del Monte Hermón lo había mojado durante seis semanas. Allá en el Monte Hermón, como un mortal del planeta sin ayuda ninguna, se había enfrentado con Caligastia, el pretendiente de Urantia, el príncipe de este mundo, y lo había derrotado. En este día memorable, según los archivos del universo, Jesús de Nazaret se convirtió en el Príncipe Planetario de Urantia. Este Príncipe de Urantia, que muy pronto sería proclamado Soberano supremo de Nebadon, iniciaba ahora cuarenta días\footnote{\textit{Los «cuarenta días»}: Mt 4:1-11; Mc 1:12-13; Lc 4:1-13.} de retiro para elaborar los planes y determinar la técnica que utilizaría para proclamar el nuevo reino de Dios en el corazón de los hombres.

\par
%\textsuperscript{(1512.6)}
\textsuperscript{136:3.2} Después de su bautismo, consagró estos cuarenta días a adaptarse a los cambios de relaciones con el mundo y el universo, ocasionados por la personalización de su Ajustador. Durante su aislamiento en las colinas de Perea, Jesús determinó la política a seguir y los métodos que emplearía en la nueva fase modificada de la vida terrenal que estaba a punto de inaugurar.

\par
%\textsuperscript{(1512.7)}
\textsuperscript{136:3.3} Jesús no efectuó este retiro para ayunar ni tampoco para afligir su alma. No era un asceta, y había venido para destruir definitivamente todas estas ideas sobre cómo acercarse a Dios. Sus razones para buscar esta soledad eran totalmente diferentes de las que habían motivado a Moisés y a Elías, e incluso a Juan el Bautista. Jesús estaba entonces plenamente consciente de sus relaciones con el universo creado por él, así como con el universo de universos supervisado por el Padre Paradisiaco, su Padre celestial. Ahora recordaba plenamente su misión de donación y las instrucciones que le diera su hermano mayor Emmanuel antes de empezar su encarnación en Urantia. Ahora comprendía clara y plenamente todas estas vastas relaciones y deseaba encontrarse a solas durante un período de meditación tranquila, para poder elaborar los planes y decidir el procedimiento a seguir en la continuación de su obra pública a favor de este mundo y de todos los demás mundos de su universo local.

\par
%\textsuperscript{(1513.1)}
\textsuperscript{136:3.4} Mientras deambulaba por las colinas en busca de un refugio apropiado, Jesús se encontró con el jefe ejecutivo de su universo, Gabriel, la Radiante Estrella Matutina de Nebadon. Gabriel restableció ahora sus comunicaciones personales con el Hijo Creador del universo; era su primer contacto directo desde que Miguel se despidió de sus asociados en Salvington para ir a Edentia con objeto de prepararse para su donación en Urantia. Siguiendo las instrucciones de Emmanuel, y autorizado por los Ancianos de los Días de Uversa, Gabriel mostró ahora a Jesús la información que indicaba que la experiencia de su donación en Urantia estaba prácticamente terminada en lo referente a la adquisición de la soberanía perfeccionada de su universo y a la finalización de la rebelión de Lucifer. Lo primero lo había conseguido el día de su bautismo, cuando la personalización de su Ajustador demostró la perfección y la plenitud de su donación en la similitud de la carne mortal, y lo segundo se volvió un hecho histórico el día que descendió del Monte Hermón para reunirse con el joven Tiglat que lo esperaba. Jesús recibió ahora la noticia, proveniente de la autoridad más alta del universo local y del superuniverso, de que su obra donadora había terminado en lo que afectaba a su estado personal en relación con la soberanía y la rebelión. Ya había recibido esta garantía directamente del Paraíso en su visión bautismal y en el fenómeno de la personalización de su Ajustador del Pensamiento interior.

\par
%\textsuperscript{(1513.2)}
\textsuperscript{136:3.5} Mientras permanecía en la montaña conversando con Gabriel, el Padre de Edentia, el de la Constelación, apareció en persona ante Jesús y Gabriel, diciendo: «Los registros han finalizado. La soberanía del Miguel n{\textordmasculine} 611.121 sobre su universo de Nebadon descansa consumada a la diestra del Padre Universal. Te libero de tu donación de parte de Emmanuel, tu hermano y patrocinador de tu encarnación en Urantia. Eres libre de dar por terminada tu donación de encarnación ahora o en cualquier otro momento, de la manera que tú mismo escojas, ascender a la diestra de tu Padre, recibir tu soberanía y asumir el gobierno incondicional bien merecido de todo Nebadon. También doy fe de que por autorización de los Ancianos de los Días, se han completado las formalidades superuniversales relacionadas con la terminación de toda rebelión pecaminosa en tu universo; se te ha otorgado una autoridad plena e ilimitada para intervenir en cualquier posible sublevación de este tipo en el futuro. Tu obra en Urantia y en la carne de una criatura mortal está formalmente terminada. De ahora en adelante, todo lo que hagas dependerá de tu propia elección».

\par
%\textsuperscript{(1513.3)}
\textsuperscript{136:3.6} Cuando el Altísimo Padre de Edentia se hubo despedido, Jesús conversó largo rato con Gabriel sobre el bienestar del universo y, al enviar sus saludos a Emmanuel, le aseguró que en la obra que estaba por emprender en Urantia, siempre recordaría los consejos recibidos en Salvington antes de comenzar su misión donadora.

\par
%\textsuperscript{(1514.1)}
\textsuperscript{136:3.7} Durante estos cuarenta días de aislamiento, Santiago y Juan, los hijos de Zebedeo, estuvieron ocupados buscando a Jesús. Muchas veces estuvieron a poca distancia del lugar donde residía, pero nunca llegaron a encontrarlo.

\section*{4. Los planes para la obra pública}
\par
%\textsuperscript{(1514.2)}
\textsuperscript{136:4.1} Día tras día, en las colinas, Jesús elaboró los planes para el resto de su donación en Urantia. En primer lugar decidió que no enseñaría al mismo tiempo que Juan. Proyectó permanecer en un retiro relativo hasta que la obra de Juan consiguiera su propósito, o fuera interrumpida súbitamente por su encarcelamiento. Jesús sabía muy bien que los sermones de Juan, intrépidos y desprovistos de tacto, pronto suscitarían el temor y la enemistad de los gobernantes civiles. En vista de la situación precaria de Juan, Jesús empezó definitivamente a preparar su programa de trabajo público a favor de su pueblo y del mundo, a favor de cada mundo habitado de todo su vasto universo. La donación de Miguel como mortal tuvo lugar \textit{en} Urantia, pero \textit{para} todos los mundos de Nebadon.

\par
%\textsuperscript{(1514.3)}
\textsuperscript{136:4.2} Después de concebir el plan general para coordinar su programa con el movimiento de Juan, lo primero que hizo Jesús fue repasar mentalmente las instrucciones de Emmanuel. Reflexionó profundamente sobre los consejos que le habían dado relativos a sus métodos de trabajo, y a que no dejara escritos perdurables en el planeta. Jesús nunca más volvió a escribir, salvo en la arena. En su visita posterior a Nazaret, y con gran pena por parte de su hermano José, Jesús destruyó todos los escritos suyos que se conservaban en las tablillas del taller de carpintería, o estaban colgados en las paredes de la vieja casa. Jesús también reflexionó mucho sobre los consejos de Emmanuel relacionados con su comportamiento en materia económica, social y política hacia el mundo que encontraría en esta época.

\par
%\textsuperscript{(1514.4)}
\textsuperscript{136:4.3} Jesús no ayunó durante estos cuarenta días de aislamiento\footnote{\textit{Jesús no ayunó}: Mt 4:2; Mc 1:13; Lc 4:2.}. El período más largo que estuvo sin alimentarse fue los dos primeros días que pasó en las colinas, pues estaba tan ensimismado en sus pensamientos que se olvidó por completo de comer. Pero al tercer día se puso a buscar alimentos. Durante este período, tampoco fue \textit{tentado} por espíritus malignos ni por personalidades rebeldes estacionadas en este mundo o procedentes de cualquier otro mundo\footnote{\textit{Las «tentaciones» ocurrieron anteriormente}: Mt 4:3-11; Mc 1:12-13; Lc 4:13.}.

\par
%\textsuperscript{(1514.5)}
\textsuperscript{136:4.4} Estos cuarenta días fueron la ocasión para el diálogo final entre su mente humana y su mente divina, o más bien para el primer funcionamiento real de estas dos mentes reunidas ahora en una sola. Los resultados de este importante período de meditación demostraban de manera concluyente que su mente divina había dominado triunfal y espiritualmente a su intelecto humano. De ahora en adelante, la mente del hombre se ha convertido en la mente de Dios, y aunque la individualidad de la mente del hombre está siempre presente, esta mente humana espiritualizada dice siempre: «Que no se haga mi voluntad sino la tuya»\footnote{\textit{Que no se haga mi voluntad sino la tuya}: Mt 26:39,42,44; Mc 14:36,39; Lc 22:42; Jn 4:34; Jn 5:30; Jn 6:38-40; Jn 15:10; Jn 17:4.}.

\par
%\textsuperscript{(1514.6)}
\textsuperscript{136:4.5} Los acontecimientos de este período extraordinario no fueron las visiones fantásticas de una mente hambrienta y debilitada, ni tampoco fueron los simbolismos confusos y pueriles que más tarde se transmitieron como las «tentaciones de Jesús en el desierto»\footnote{\textit{Razón de la idea de los «cuarenta días»}: Mt 4:1-11; Mc 1:12-13; Lc 4:1-13.}. Fue más bien un período para meditar sobre toda la carrera memorable y variada de la donación en Urantia, y para preparar cuidadosamente los planes del ministerio ulterior que fuera más útil para este mundo, y a la vez contribuyera también un poco al mejoramiento de todas las otras esferas aisladas por la rebelión. Jesús examinó toda la historia de la vida humana en Urantia, desde los días de Andón y Fonta, pasando por la falta de Adán, hasta el ministerio de Melquisedek de Salem.

\par
%\textsuperscript{(1514.7)}
\textsuperscript{136:4.6} Gabriel había recordado a Jesús que podía manifestarse al mundo de dos maneras diferentes, en el caso de que decidiera permanecer algún tiempo en Urantia. También se le indicó claramente a Jesús que su elección en esta materia no tendría nada que ver con su soberanía universal ni con el final de la rebelión de Lucifer. Las dos maneras de servir al mundo eran las siguientes:

\par
%\textsuperscript{(1515.1)}
\textsuperscript{136:4.7} 1. Su propia vía ---La vía que pudiera parecerle más agradable y útil, desde el punto de vista de las necesidades inmediatas de este mundo y de la edificación en curso de su propio universo.

\par
%\textsuperscript{(1515.2)}
\textsuperscript{136:4.8} 2. La vía del Padre ---La demostración con el ejemplo de un ideal, a largo plazo, de vida como criatura, según lo ven las altas personalidades de la administración paradisíaca del universo de universos.

\par
%\textsuperscript{(1515.3)}
\textsuperscript{136:4.9} Se le indicó claramente a Jesús que tenía dos maneras de ordenar el resto de su vida terrestre. Tal como se podían observar a la luz de la situación inmediata, cada una de estas vías tenía puntos a favor. El Hijo del Hombre vio claramente que su elección entre estas dos líneas de conducta no tendría ninguna repercusión sobre la recepción de la soberanía de su universo; éste era un asunto que ya estaba arreglado y sellado en los archivos del universo de universos y sólo estaba pendiente de su petición personal. Pero se le indicó a Jesús que su hermano paradisíaco, Emmanuel, sentiría una gran satisfacción si Jesús juzgara conveniente terminar su carrera terrenal de encarnación tan noblemente como la había empezado, siempre sometido a la voluntad del Padre. Al tercer día de este aislamiento, Jesús se prometió a sí mismo que volvería al mundo para terminar su carrera terrenal, y que en cualquier situación que implicara los dos caminos, siempre escogería la voluntad del Padre. Y vivió el resto de su vida terrestre permaneciendo siempre fiel a esta resolución. Incluso hasta el amargo final, subordinó invariablemente su voluntad soberana a la de su Padre celestial.

\par
%\textsuperscript{(1515.4)}
\textsuperscript{136:4.10} Los cuarenta días en el desierto montañoso no fueron un período de grandes tentaciones, sino más bien el período de las \textit{grandes decisiones}\footnote{\textit{Las grandes decisiones}: Mt 4:1; Mc 1:12-13; Lc 4:1-2.} del Maestro. Durante estos días de solitaria comunión consigo mismo y con la presencia inmediata de su Padre ---el Ajustador Personalizado (pues ya no tenía un guardián seráfico personal)--- tomó una tras otra las grandes decisiones que regirían su política y su conducta durante el resto de su carrera terrenal. La tradición de una gran tentación fue conectada posteriormente con este período de aislamiento debido a una confusión con los relatos fragmentarios de las luchas en el Monte Hermón, y además porque era costumbre que todos los grandes profetas y líderes humanos empezaran su carrera pública sometiéndose a estos supuestos períodos de ayuno y oración. Cada vez que Jesús se enfrentaba con una decisión nueva o importante, siempre tenía la costumbre de retirarse para comulgar con su propio espíritu y tratar así de conocer la voluntad de Dios.

\par
%\textsuperscript{(1515.5)}
\textsuperscript{136:4.11} En todos estos proyectos para el resto de su vida terrenal, Jesús siempre estuvo dividido, en su corazón humano, entre dos líneas opuestas de conducta:

\par
%\textsuperscript{(1515.6)}
\textsuperscript{136:4.12} 1. Albergaba un intenso deseo de conseguir que su pueblo ---y el mundo entero--- creyera en él y aceptara su nuevo reino espiritual. Y conocía muy bien las ideas de sus compatriotas sobre el Mesías venidero.

\par
%\textsuperscript{(1515.7)}
\textsuperscript{136:4.13} 2. Vivir y actuar de la manera que sabía que su Padre aprobaría, llevar a cabo su trabajo a favor de otros mundos necesitados, y continuar, en el establecimiento del reino, revelando al Padre y manifestando su divino carácter de amor.

\par
%\textsuperscript{(1515.8)}
\textsuperscript{136:4.14} Durante estos días extraordinarios, Jesús vivió en una antigua caverna rocosa, un refugio en la ladera de las colinas, cerca de una aldea llamada en otro tiempo Beit Adis. Bebía en el pequeño manantial que brotaba en la falda de la colina cerca de este refugio rocoso.

\section*{5. La primera gran decisión}
\par
%\textsuperscript{(1516.1)}
\textsuperscript{136:5.1} Al tercer día de empezar esta conversación consigo mismo y con su Ajustador Personalizado, Jesús fue gratificado con la visión de las huestes celestiales de Nebadon, reunidas y enviadas por sus comandantes para aguardar los mandatos de su amado Soberano. Este ejército poderoso comprendía doce legiones de serafines\footnote{\textit{Doce legiones de ángeles y otros}: Mt 26:53.} y cantidades proporcionales de todas las órdenes de inteligencias del universo. La primera gran decisión de Jesús en su aislamiento consistió en determinar si utilizaría o no estas poderosas personalidades en el programa posterior de su obra pública en Urantia.

\par
%\textsuperscript{(1516.2)}
\textsuperscript{136:5.2} Jesús decidió que \textit{no} utilizaría ni una sola personalidad de esta vasta asamblea, a menos que resultara evidente que se trataba de la \textit{voluntad de su Padre}. A pesar de esta decisión de tipo general, este enorme ejército permaneció con él durante el resto de su vida terrestre, siempre dispuesto a obedecer a la menor expresión de la voluntad de su Soberano. Jesús no contemplaba constantemente, con sus ojos humanos, estas personalidades acompañantes, pero su Ajustador Personalizado asociado las veía permanentemente y podía comunicarse con todas ellas.

\par
%\textsuperscript{(1516.3)}
\textsuperscript{136:5.3} Antes de descender de su retiro de cuarenta días en las montañas, Jesús confió el mando inmediato de este ejército acompañante de personalidades universales a su Ajustador recientemente Personalizado. Durante más de cuatro años del tiempo de Urantia, estas personalidades seleccionadas de todas las divisiones de inteligencias universales funcionaron con obediencia y respeto bajo la sabia dirección de este Monitor de Misterio Personalizado, ensalzado y experimentado. Al asumir el mando de esta poderosa asamblea, el Ajustador, que había sido en otro tiempo parte y esencia del Padre Paradisiaco, aseguró a Jesús que en ningún caso se permitiría a estos agentes superhumanos servir o manifestarse en conexión con su carrera terrestre, o a favor de ella, a menos que fuera patente que el Padre deseaba dicha intervención. Así pues, mediante una sola gran decisión, Jesús se privó voluntariamente de toda cooperación sobrehumana en todos los asuntos relacionados con el resto de su carrera como mortal, a menos que el Padre eligiera por su cuenta participar en un acto o episodio determinado de los trabajos terrestres del Hijo.

\par
%\textsuperscript{(1516.4)}
\textsuperscript{136:5.4} Al aceptar el mando de las huestes universales al servicio de Cristo Miguel, el Ajustador Personalizado se esmeró en señalar a Jesús que, aunque las actividades \textit{espaciales} de esta asamblea de criaturas universales podían ser limitadas por la autoridad delegada de su Creador, estas restricciones no tendrían efecto en cuanto a las funciones de estas criaturas en el \textit{tiempo}. Esta limitación se debía al hecho de que los Ajustadores son seres independientes del tiempo una vez que han sido personalizados. Por consiguiente, a Jesús se le advirtió que, aunque el control de todas las inteligencias vivientes colocadas bajo el mando del Ajustador sería completo y perfecto en todo lo relacionado con el \textit{espacio}, no se podrían imponer unas limitaciones tan perfectas en lo concerniente al \textit{tiempo}. El Ajustador le dijo: «Tal como has ordenado, impediré que este ejército acompañante de inteligencias universales intervenga en cualquier cuestión relacionada con tu carrera terrestre, excepto en los casos en que el Padre Paradisiaco me ordene dejar actuar a estos agentes para que se cumpla su voluntad divina, tal como tú la hayas elegido, y en aquellos otros casos en que tu voluntad divina y humana pueda emprender una elección o una acción que implique desviaciones del orden terrestre natural, relacionadas exclusivamente con el \textit{tiempo}. En todos estos casos soy impotente, y tus criaturas aquí reunidas en perfección y unidad de poder son igualmente impotentes. Si tus dos naturalezas unidas albergan alguna vez tales deseos, esos mandatos tuyos serán ejecutados inmediatamente. En todos esos asuntos, tu deseo constituirá la abreviación del tiempo, y la cosa proyectada \textit{existirá}. Bajo mi autoridad, esto constituye la mayor limitación que puede imponerse a tu soberanía potencial. En mi propia conciencia el tiempo no existe, y por esta razón no puedo limitar a tus criaturas en ninguna cuestión relacionada con el tiempo».

\par
%\textsuperscript{(1517.1)}
\textsuperscript{136:5.5} Jesús fue así informado de las consecuencias de su decisión de seguir viviendo como un hombre entre los hombres. Mediante una sola decisión, había excluido a todas sus huestes universales presentes de inteligencias diversas de participar en su próximo ministerio público, excepto en los asuntos relacionados exclusivamente con el \textit{tiempo}. Es pues evidente que cualquier posible manifestación sobrenatural o supuestamente superhumana que acompañara al ministerio de Jesús sólo concerniría a la eliminación del tiempo, a menos que el Padre celestial dictaminara específicamente lo contrario. Ningún milagro, ningún ministerio de misericordia, ningún otro acontecimiento posible que ocurriera en relación con el resto de la obra terrestre de Jesús, podría tener la naturaleza o el carácter de una acción que trascendiera las leyes naturales establecidas, que rigen normalmente los asuntos de los hombres tal como viven en Urantia, \textit{excepto} en esta cuestión expresamente mencionada del \textit{tiempo}. Por supuesto, ningún límite podía ser impuesto a las manifestaciones de «la voluntad del Padre». La eliminación del tiempo, en conexión con el deseo expreso de este Soberano potencial de un universo, sólo podía evitarse mediante la acción directa y explícita de la \textit{voluntad} de este hombre-Dios en el sentido de que el tiempo, relacionado con el acto o el acontecimiento en cuestión, \textit{no debía ser acortado o eliminado}. A fin de impedir la aparición de \textit{milagros temporales} aparentes, Jesús tenía que permanecer constantemente consciente del tiempo. Cualquier lapsus en su conciencia del tiempo, en conexión con el mantenimiento de un deseo concreto, equivaldría a hacer efectiva la cosa concebida en la mente de este Hijo Creador, y todo ello sin la intervención del tiempo.

\par
%\textsuperscript{(1517.2)}
\textsuperscript{136:5.6} Gracias al control supervisor de su Ajustador Personalizado y asociado, Miguel podía limitar perfectamente sus actividades terrestres personales en lo relacionado con el espacio, pero no le era posible al Hijo del Hombre limitar así su nuevo estado terrestre como Soberano potencial de Nebadon en lo referente al \textit{tiempo}. Este era el estado real de Jesús de Nazaret cuando salió para comenzar su ministerio público en Urantia.

\section*{6. La segunda decisión}
\par
%\textsuperscript{(1517.3)}
\textsuperscript{136:6.1} Habiendo fijado su política respecto a todas las personalidades de todas las clases de inteligencias por él creadas, en la medida en que esto podía determinarse a la vista del potencial inherente a su nuevo estado de divinidad, Jesús orientó luego sus pensamientos sobre sí mismo. Ahora que era plenamente consciente de ser el creador de todas las cosas y de todos los seres existentes en este universo, ¿qué iba a hacer con estas prerrogativas de creador en las situaciones recurrentes de la vida que tendría que afrontar en cuanto regresara a Galilea para reanudar su trabajo entre los hombres?. De hecho, allí mismo donde se encontraba, en estas colinas solitarias, ya se le había presentado poderosamente este problema mediante la necesidad de conseguir comida. Al tercer día de sus meditaciones solitarias, su cuerpo humano sintió hambre. ¿Debía ir en busca de alimento como cualquier hombre común, o debía ejercer simplemente sus poderes creadores normales y producir un alimento corporal apropiado y al alcance de la mano?. Esta gran decisión del Maestro os ha sido descrita como una tentación ---como un reto de unos supuestos enemigos para que «mande que estas piedras se conviertan en panes»\footnote{\textit{La «tentación» para obtener comida}: Mt 4:3; Lc 4:3.}.

\par
%\textsuperscript{(1518.1)}
\textsuperscript{136:6.2} Jesús estableció pues una nueva política coherente para el resto de su obra terrenal. En lo que se refería a sus necesidades personales, e incluso en general en sus relaciones con otras personalidades, eligió deliberadamente en ese momento seguir el camino de la existencia terrestre normal; se pronunció firmemente contra una línea de conducta que trascendiera, violara o ultrajara las leyes naturales establecidas por él. Pero tal como ya le había advertido su Ajustador Personalizado, no podía asegurar que en ciertas circunstancias concebibles, estas leyes naturales no pudieran resultar considerablemente \textit{aceleradas}. En principio, Jesús decidió que la obra de su vida sería organizada y continuada conforme a las leyes de la naturaleza y en armonía con la organización social existente. El Maestro eligió así un programa de vida que equivalía a la decisión de estar en contra de los milagros y de los prodigios. Una vez más se pronunció a favor de «la voluntad del Padre»; una vez más puso todas las cosas entre las manos de su Padre Paradisiaco.

\par
%\textsuperscript{(1518.2)}
\textsuperscript{136:6.3} La naturaleza humana de Jesús le dictaba que su primer deber era preservar su vida; es el comportamiento normal del hombre físico en los mundos del tiempo y del espacio, y por consiguiente, la reacción legítima de un mortal de Urantia. Pero las preocupaciones de Jesús no se limitaban sólo a este mundo y a sus criaturas; estaba viviendo una vida destinada a instruir e inspirar a las múltiples criaturas de un vastísimo universo.

\par
%\textsuperscript{(1518.3)}
\textsuperscript{136:6.4} Antes de la iluminación de su bautismo, había vivido en perfecta sumisión a la voluntad y a la orientación de su Padre celestial. Tomó la enérgica decisión de continuar viviendo con la misma dependencia implícita y humana de la voluntad del Padre. Se propuso seguir una línea de conducta antinatural ---decidió que no trataría de preservar su vida. Escogió continuar su política de negarse a defenderse. Expresó sus conclusiones con las palabras de las Escrituras, familiares para su mente humana: «No sólo de pan vivirá el hombre, sino de toda palabra que sale de la boca de Dios»\footnote{\textit{No sólo de pan vivirá el hombre}: Dt 8:3; Mt 4:3-4; Lc 4:3-4.}. Al llegar a esta conclusión sobre el apetito de la naturaleza física que se manifiesta como hambre, el Hijo del Hombre efectuó su declaración final sobre todas las demás necesidades de la carne y de los impulsos naturales de la naturaleza humana.

\par
%\textsuperscript{(1518.4)}
\textsuperscript{136:6.5} Quizás podría utilizar su poder sobrehumano para ayudar a otros, pero nunca para sí mismo. Y se mantuvo fiel a esta línea de conducta hasta el final, cuando dijeron mofándose de él: «Ha salvado a los demás, pero no puede salvarse a sí mismo»\footnote{\textit{Salvó a otros pero no a sí mismo}: Mt 27:42a; Mc 15:31; Lc 23:35a.} ---porque no quiso hacerlo.

\par
%\textsuperscript{(1518.5)}
\textsuperscript{136:6.6} Los judíos esperaban a un Mesías que realizara maravillas aún más grandes que Moisés, de quien se decía que había hecho manar agua de la roca en un lugar árido y que había alimentado con maná a sus antepasados en el desierto. Jesús conocía la clase de Mesías que esperaban sus compatriotas, y disponía de todos los poderes y prerrogativas para estar a la altura de sus más ardientes esperanzas, pero tomó la decisión de ponerse en contra de este magnífico programa de poder y de gloria. Jesús consideraba esta conducta de esperar acciones milagrosas como un retroceso a los antiguos tiempos de la magia ignorante y de las prácticas degeneradas de los curanderos salvajes. Quizás, para la salvación de sus criaturas, consintiera en acelerar la ley natural, pero trascender sus propias leyes, ya sea en su propio beneficio o para deslumbrar a sus semejantes, eso no lo haría. Y esta decisión del Maestro fue definitiva.

\par
%\textsuperscript{(1518.6)}
\textsuperscript{136:6.7} Jesús se entristecía por su pueblo; comprendía plenamente cómo habían llegado a esperar al Mesías venidero, la época en que «la tierra producirá diez mil veces más frutos, y una vid tendrá mil sarmientos, y cada sarmiento producirá mil racimos, y cada racimo producirá mil uvas, y cada uva producirá un barril de vino»\footnote{\textit{La tierra dará frutos}: Is 4:2; Is 5:10; 1 Hen 10:18-20a.}. Los judíos creían que el Mesías inauguraría una era de abundancia milagrosa. Los hebreos se habían alimentado durante mucho tiempo de tradiciones de milagros y de leyendas de prodigios.

\par
%\textsuperscript{(1519.1)}
\textsuperscript{136:6.8} Jesús no era un Mesías que venía para multiplicar el pan y el vino. No venía para abastecer exclusivamente las necesidades temporales; venía para hacer una revelación de su Padre celestial a sus hijos terrestres, mientras intentaba que sus hijos terrestres se unieran a él en un esfuerzo sincero por vivir según la voluntad del Padre que está en los cielos.

\par
%\textsuperscript{(1519.2)}
\textsuperscript{136:6.9} Con esta decisión, Jesús de Nazaret describía a los espectadores de un universo la locura y el pecado de prostituir los talentos divinos y las aptitudes dadas por Dios para el engrandecimiento personal o para el beneficio y la glorificación puramente egoístas. Éste era el pecado de Lucifer y Caligastia.

\par
%\textsuperscript{(1519.3)}
\textsuperscript{136:6.10} Esta gran decisión de Jesús ilustra dramáticamente la verdad de que la satisfacción egoísta y la gratificación sensual, solas y por sí mismas, son incapaces de aportar la felicidad a los seres humanos que evolucionan. En la existencia mortal, existen valores más elevados ---la maestría intelectual y el perfeccionamiento espiritual--- que trascienden con mucho la gratificación necesaria de los apetitos e impulsos puramente físicos del hombre. Los dones naturales del hombre, sus talentos y aptitudes, deberían emplearse principalmente para desarrollar y ennoblecer los poderes superiores de la mente y del espíritu.

\par
%\textsuperscript{(1519.4)}
\textsuperscript{136:6.11} Jesús reveló así, a las criaturas de su universo, la técnica del camino nuevo y mejor, los valores morales superiores de la vida, y las satisfacciones espirituales más profundas de la existencia humana evolutiva en los mundos del espacio.

\section*{7. La tercera decisión}
\par
%\textsuperscript{(1519.5)}
\textsuperscript{136:7.1} Después de tomar sus decisiones respecto a los asuntos relacionados con el alimento y el suministro físico para las necesidades de su cuerpo material, el cuidado de su salud y la de sus asociados, aún quedaban otros problemas por resolver. ¿Cómo se comportaría ante un peligro personal?. Decidió ejercer una vigilancia normal sobre su seguridad física, y tomar precauciones razonables para evitar el fin prematuro de su carrera en la carne, pero decidió abstenerse de toda intervención superhumana cuando sobreviniera la crisis de su vida en la carne. Mientras tomaba esta decisión, Jesús estaba sentado a la sombra de un árbol en un saliente rocoso, con un precipicio que se abría ante él. Se daba perfectamente cuenta que desde este saliente podía arrojarse al vacío sin sufrir ningún daño\footnote{\textit{La «tentación» de saltar al vacío}: Mt 4:5-6; Lc 4:9-11.}, siempre que revocara su primera gran decisión de no invocar la intervención de sus inteligencias celestiales para continuar la obra de su vida en Urantia, y siempre que anulara su segunda decisión sobre su comportamiento respecto a la preservación de su vida.

\par
%\textsuperscript{(1519.6)}
\textsuperscript{136:7.2} Jesús sabía que sus compatriotas esperaban un Mesías que estuviera por encima de las leyes naturales. Le habían enseñado bien aquel pasaje de las Escrituras: «No te sucederá ningún mal, y ninguna plaga se acercará a tu morada. Pues te confiará al cuidado de sus ángeles para que te guarden en todos tus caminos. Te llevarán en sus manos, para que tu pie no tropiece contra una piedra»\footnote{\textit{No te sucederá ningún mal}: Sal 91:10-12.}. Esta especie de presunción, este desafío a las leyes de la gravedad de su Padre, ¿estarían justificados para protegerse de un posible daño o quizás para ganarse la confianza de su pueblo mal enseñado y desorientado?. Esta línea de conducta, por muy satisfactoria que fuera para los judíos en busca de signos, no sería una revelación de su Padre, sino una dudosa manipulación de las leyes establecidas en el universo de universos.

\par
%\textsuperscript{(1519.7)}
\textsuperscript{136:7.3} Comprendiendo todo esto y sabiendo que el Maestro se negaba a trabajar desafiando sus leyes establecidas de la naturaleza en lo que concernía a su conducta personal, sabéis con certidumbre que nunca caminó sobre las aguas\footnote{\textit{Jesús nunca caminó sobre las aguas}: Mt 14:25-27; Mc 6:48-50; Jn 6:19-21.} y que nunca hizo nada que violara su orden material de administrar el mundo\footnote{\textit{Ningún signo sobrenatural}: Mt 12:38; Mt 16:1; Mc 8:11.}. Por supuesto, recordad siempre que aún no se había encontrado la manera de librarlo por completo de la falta de control sobre el elemento tiempo en conexión con los asuntos entregados a la jurisdicción del Ajustador Personalizado.

\par
%\textsuperscript{(1520.1)}
\textsuperscript{136:7.4} Durante toda su vida terrenal, Jesús permaneció constantemente fiel a esta decisión. Aunque los fariseos le provocaron pidiéndole un signo, y los espectadores en el Calvario le desafiaron a que descendiera de la cruz, mantuvo firmemente la decisión que tomó en esta hora en la ladera de la montaña\footnote{\textit{Incluso cuando quienes le veían le provocaron}: Mt 27:39-44; Mc 15:29-32; Lc 23:35-37.}.

\section*{8. La cuarta decisión}
\par
%\textsuperscript{(1520.2)}
\textsuperscript{136:8.1} El gran problema siguiente con el que tuvo que luchar este hombre-Dios y que pronto resolvió de acuerdo con la voluntad del Padre celestial consistía en saber si debía o no emplear algunos de sus poderes sobrehumanos para atraer la atención y conseguir la adhesión de sus semejantes. ¿Debía emplear, de alguna manera, sus poderes universales para satisfacer la inclinación de los judíos por lo espectacular y lo maravilloso?. Decidió que no haría nada semejante. Se ratificó en una línea de conducta que eliminaba todas estas prácticas como método para llevar su misión al conocimiento de los hombres. Y vivió constantemente de acuerdo con esta gran decisión. Incluso en los numerosos casos en que permitió manifestaciones de misericordia que comportaron un acortamiento del tiempo\footnote{\textit{Que no contaran nada sobre el acortamiento del tiempo}: Mt 8:4; Mt 9:30; Mt 12:16; Mc 1:44; Mc 5:43; Mc 7:36; Mc 8:26; Lc 5:14; Lc 8:56.}, casi invariablemente recomendó a los que recibieron su ministerio curativo que no contaran a nadie los beneficios que habían recibido. Siempre rechazó el desafío sarcástico de sus enemigos cuando le pedían «muéstranos un signo»\footnote{\textit{Jesús rehusó mostrar signos}: Mt 12:38-39; Mt 16:1-4; Mc 8:11-12; Lc 11:16,29-30; Jn 2:18-20; 6:30.} como prueba y demostración de su divinidad.

\par
%\textsuperscript{(1520.3)}
\textsuperscript{136:8.2} Jesús preveía muy sabiamente que la realización de milagros y la ejecución de prodigios sólo produciría una lealtad superficial mediante la intimidación de la mente material; tales acciones no revelarían a Dios ni salvarían a los hombres. Se negó a ser simplemente un hacedor de prodigios. Resolvió que se ocuparía de una sola tarea: el establecimiento del reino de los cielos.

\par
%\textsuperscript{(1520.4)}
\textsuperscript{136:8.3} Durante todo este importante diálogo de Jesús en comunión consigo mismo, el elemento humano que interroga y casi duda estaba presente, porque Jesús era hombre a la vez que Dios. Era evidente que los judíos nunca lo aceptarían como Mesías si no hacía prodigios. Además, si consentía en hacer una sola cosa no natural, la mente humana sabría con certidumbre que era por subordinación a una mente verdaderamente divina. Para la mente divina, ¿sería compatible con «la voluntad del Padre»\footnote{\textit{Jesús decide vivir la voluntad del Padre}: Mt 26:39,42,44; Mc 14:36,39; Lc 22:42; Jn 4:34; 5:30; 6:38-40; 15:10; 17:4.} hacer esta concesión a la naturaleza dubitativa de la mente humana?. Jesús decidió que sería incompatible, y citó la presencia del Ajustador Personalizado como prueba suficiente de la divinidad asociada con la humanidad.

\par
%\textsuperscript{(1520.5)}
\textsuperscript{136:8.4} Jesús había viajado mucho; recordaba Roma, Alejandría y Damasco. Conocía los modos de obrar del mundo ---cómo la gente conseguía sus propósitos en la política y en el comercio por medio de compromisos y diplomacia. ¿Utilizaría este conocimiento para hacer avanzar su misión en la Tierra?. ¡No!. Se pronunció igualmente contra todo compromiso con la sabiduría del mundo y la influencia de las riquezas para establecer el reino. De nuevo escogió depender exclusivamente de la voluntad del Padre.

\par
%\textsuperscript{(1520.6)}
\textsuperscript{136:8.5} Jesús se daba perfectamente cuenta de los atajos que se abrían para alguien con sus poderes. Conocía muchas maneras de atraer la atención inmediata de la nación y del mundo entero sobre su persona. Pronto se celebraría la Pascua en Jerusalén; la ciudad estaría llena de visitantes. Podía ascender al pináculo del templo y, ante las multitudes asombradas, caminar por el aire\footnote{\textit{Decide no caminar en el aire desde el templo}: Mt 4:5-6; Lc 4:9-11.}; éste era el tipo de Mesías que la gente esperaba. Pero después los desilusionaría, puesto que no había venido para volver a establecer el trono de David. Y conocía la futilidad del método de Caligastia, consistente en tratar de adelantarse a la manera natural, lenta y segura de llevar a cabo el propósito divino. Una vez más, el Hijo del Hombre se inclinó con obediencia ante la vía del Padre, la voluntad del Padre.

\par
%\textsuperscript{(1521.1)}
\textsuperscript{136:8.6} Jesús escogió establecer el reino de los cielos en el corazón de los hombres mediante métodos naturales, normales, difíciles y penosos, los mismos procedimientos que sus hijos terrestres tendrían que seguir posteriormente en su trabajo de ampliar y extender este reino celestial. El Hijo del Hombre sabía muy bien que sería «a través de muchas tribulaciones como muchos hijos de todos los tiempos entrarían en el reino»\footnote{\textit{Entrar al reino a través de muchas tribulaciones}: Hch 14:22.}. Jesús estaba pasando ahora por la gran prueba de los hombres civilizados, la de tener el poder y negarse firmemente a utilizarlo para fines puramente egoístas o personales.

\par
%\textsuperscript{(1521.2)}
\textsuperscript{136:8.7} Al estudiar la vida y la experiencia del Hijo del Hombre, deberíais tener siempre presente el hecho de que el Hijo de Dios estaba encarnado en la mente de un ser humano del siglo primero, y no en la mente de un mortal del siglo veinte o de otro siglo. Con esto deseamos transmitiros la idea de que los dones humanos de Jesús habían sido adquiridos por vía natural. Él era el producto de los factores hereditarios y ambientales de su época, unidos a la influencia de su instrucción y de su educación. Su humanidad era auténtica, natural, totalmente derivada y alimentada por los antecedentes de la situación intelectual real y de las condiciones económicas y sociales de aquella época y de aquella generación. Aunque en la experiencia de este hombre-Dios siempre existía la posibilidad de que la mente divina trascendiera al intelecto humano, sin embargo, siempre que funcionaba su mente humana, lo hacía como lo haría una verdadera mente mortal en las condiciones del entorno humano de aquella época.

\par
%\textsuperscript{(1521.3)}
\textsuperscript{136:8.8} Jesús ilustró para todos los mundos de su vasto universo la locura de crear situaciones artificiales con el propósito de mostrar una autoridad arbitraria, o de permitirse un poder excepcional, para realzar los valores morales o acelerar el progreso espiritual. Jesús decidió que, durante su misión en la Tierra, no se prestaría a repetir la decepción del reinado de los Macabeos. Se negó a prostituir sus atributos divinos para adquirir una popularidad no merecida o para conseguir un prestigio político. No consentiría en transmutar la energía divina y creativa en poder nacional o en prestigio internacional. Jesús de Nazaret se negó a hacer compromisos con el \textit{mal}, y mucho menos a asociarse con el pecado. El Maestro colocó triunfalmente la fidelidad a la voluntad de su Padre por encima de cualquier otra consideración terrestre y temporal.

\section*{9. La quinta decisión}
\par
%\textsuperscript{(1521.4)}
\textsuperscript{136:9.1} Habiendo establecido el criterio a seguir en lo referente a sus relaciones individuales con las leyes naturales y el poder espiritual, dirigió su atención hacia la elección de los métodos que emplearía para proclamar y establecer el reino de Dios. Juan ya había iniciado este trabajo; ¿cómo podría Jesús continuar el mensaje?. ¿Cómo debería seguir con la misión de Juan?. ¿Cómo debería organizar a sus seguidores para que el esfuerzo resultara eficaz y la cooperación inteligente?. Jesús estaba llegando ahora a la decisión final que le impediría seguir considerándose el Mesías judío, al menos tal como la población concebía al Mesías en aquella época.

\par
%\textsuperscript{(1522.1)}
\textsuperscript{136:9.2} Los judíos imaginaban a un libertador que llegaría con un poder milagroso para derribar a los enemigos de Israel y establecer a los judíos como gobernantes del mundo, libres de la miseria y de la opresión. Jesús sabía que esta esperanza no se materializaría nunca. Sabía que el reino de los cielos concernía a la victoria sobre el mal en el corazón de los hombres, y que se trataba de un asunto puramente espiritual. Reflexionó sobre la conveniencia de inaugurar el reino espiritual con una brillante y deslumbrante demostración de poder ---esta línea de conducta hubiera sido permisible y estaba totalmente dentro de la jurisdicción de Miguel--- pero adoptó una posición totalmente contraria a este plan. No transigiría con las técnicas revolucionarias de Caligastia. Había ganado potencialmente el mundo sometiéndose a la voluntad del Padre, y se propuso terminar su obra como la había empezado, y como Hijo del Hombre.

\par
%\textsuperscript{(1522.2)}
\textsuperscript{136:9.3} ¡Es difícil que podáis imaginar lo que hubiera sucedido en Urantia si este hombre-Dios, ahora en posesión potencial de todos los poderes en el cielo y en la Tierra, hubiera decidido desplegar una sola vez el estandarte de la soberanía, formar su prodigioso ejército en orden de batalla!. Pero no transigiría. No serviría al mal para que se pudiera suponer que la adoración de Dios provenía de ello. Permaneció fiel a la voluntad del Padre. Proclamaría a un universo que lo observaba: «Adoraréis al Señor vuestro Dios, y a él solo serviréis»\footnote{\textit{Servir sólo al Señor Dios}: Ex 20:3-5a; Dt 5:7-9a; 6:13-14; 10:20; Mt 4:10; Lc 4:8.}.

\par
%\textsuperscript{(1522.3)}
\textsuperscript{136:9.4} A medida que pasaban los días, Jesús percibía con mayor claridad la clase de revelador de la verdad que iba a ser. Discernía que el camino de Dios no iba a ser un camino fácil. Empezó a darse cuenta de que el resto de su experiencia humana podría ser un amargo cáliz\footnote{\textit{Beber el cáliz amargo}: Mt 20:22-23; Mt 26:39,42; Mc 10:38-39; Mc 14:36; Lc 22:42; Jn 18:11.}, pero decidió beberlo.

\par
%\textsuperscript{(1522.4)}
\textsuperscript{136:9.5} Incluso su mente humana dice adiós al trono de David. Paso a paso, esta mente humana se mueve en el sendero de lo divino. La mente humana todavía hace preguntas, pero acepta invariablemente las respuestas divinas como regla final, en esta existencia combinada de vivir como un hombre en el mundo mientras se somete todo el tiempo, de forma incondicional, a hacer la voluntad eterna y divina del Padre.

\par
%\textsuperscript{(1522.5)}
\textsuperscript{136:9.6} Roma era la dueña del mundo occidental. El Hijo del Hombre, ahora en su aislamiento, tomando estas importantes decisiones, con las huestes del cielo a sus órdenes, representaba la última oportunidad de los judíos para conseguir el dominio del mundo; pero este judío de nacimiento, dotado de una sabiduría y de un poder tan extraordinarios, no quiso emplear sus dones universales para encumbrarse personalmente ni para entronizar a su pueblo\footnote{\textit{«Tentación» para controlar los reinos}: Mt 4:8; Lc 4:5.}. Veía, por decirlo así, «los reinos de este mundo», y poseía el poder para apoderarse de ellos. Los Altísimos de Edentia habían puesto estos poderes en sus manos, pero no los quería. Los reinos de la Tierra eran cosas mezquinas, indignas del interés del Creador y Soberano de un universo. Sólo tenía un objetivo: la revelación posterior de Dios al hombre, el establecimiento del reino, la soberanía del Padre celestial en el corazón de los hombres.

\par
%\textsuperscript{(1522.6)}
\textsuperscript{136:9.7} Las ideas de batallas, contiendas y masacres repugnaban a Jesús; no quería nada de eso. Aparecería en la Tierra como el Príncipe de la Paz para revelar a un Dios de amor. Antes de su bautismo había rechazado de nuevo otra oferta de los celotes para encabezar su rebelión contra los opresores romanos. Ahora, tomó la decisión final con respecto a los pasajes de las Escrituras que su madre le había enseñado, tales como: «El Señor me ha dicho: `Tú eres mi Hijo; te he engendrado hoy. Pídeme, y te daré a los paganos por herencia y los confines de la Tierra como posesión. Los quebrantarás con mano de hierro; los despedazarás como una vasija de alfarero'»\footnote{\textit{Tú eres mi Hijo; te he engendrado hoy}: Sal 2:7-9; Hch 13:33.}.

\par
%\textsuperscript{(1522.7)}
\textsuperscript{136:9.8} Jesús de Nazaret llegó a la conclusión de que estas citas no se referían a él. Por último, y de una vez por todas, la mente humana del Hijo del Hombre barrió por completo todas estas dificultades y contradicciones mesiánicas ---las escrituras hebreas, la educación de los padres, la enseñanza del chazan, las expectativas de los judíos y los ambiciosos deseos humanos. Decidió su línea de conducta de manera definitiva. Regresaría a Galilea y empezaría tranquilamente la proclamación del reino, confiando en su Padre (el Ajustador Personalizado) para elaborar los detalles cotidianos de actuación.

\par
%\textsuperscript{(1523.1)}
\textsuperscript{136:9.9} Con estas decisiones, Jesús sentó un digno ejemplo para todas las personas de todos los mundos de un vasto universo al negarse a aplicar pruebas materiales para demostrar los problemas espirituales, al negarse a desafiar presuntuosamente las leyes naturales. Y dio un ejemplo inspirador de lealtad universal y de nobleza moral cuando se negó a coger el poder temporal como preludio de la gloria espiritual.

\par
%\textsuperscript{(1523.2)}
\textsuperscript{136:9.10} Si el Hijo del Hombre tenía dudas acerca de su misión y de la naturaleza de ésta cuando subió a las colinas después de su bautismo, ya no tenía ninguna cuando volvió entre sus compañeros después de los cuarenta días de aislamiento y de decisiones.

\par
%\textsuperscript{(1523.3)}
\textsuperscript{136:9.11} Jesús ha elaborado un programa para establecer el reino del Padre. No alimentará las satisfacciones físicas de la gente. No distribuirá pan a las multitudes como vio hacer tan recientemente en Roma. No atraerá la atención sobre sí mismo haciendo prodigios, a pesar de que los judíos esperan precisamente un libertador de esta índole. Tampoco intentará que acepten su mensaje espiritual mediante una exhibición de autoridad política o de poder temporal.

\par
%\textsuperscript{(1523.4)}
\textsuperscript{136:9.12} Al rechazar estos métodos que realzarían el reino venidero a los ojos de los judíos que lo esperaban, Jesús contaba con que estos mismos judíos rechazarían a fin de cuentas y con seguridad todos sus derechos a la autoridad y a la divinidad. Sabiendo todo esto, Jesús trató de evitar durante mucho tiempo que sus primeros discípulos hablaran de él como si fuera el Mesías.

\par
%\textsuperscript{(1523.5)}
\textsuperscript{136:9.13} Durante todo su ministerio público, tuvo que enfrentarse constantemente con tres situaciones recurrentes: el clamor para ser alimentados, la insistencia en ver milagros, y la petición final de que permitiera a sus seguidores coronarlo rey. Pero Jesús no se apartó nunca de las decisiones que había tomado durante estos días de aislamiento en las colinas de Perea.

\section*{10. La sexta decisión}
\par
%\textsuperscript{(1523.6)}
\textsuperscript{136:10.1} El último día de este retiro memorable, antes de bajar de la montaña para reunirse con Juan y sus discípulos, el Hijo del Hombre tomó su decisión final. Y la comunicó al Ajustador Personalizado en estos términos: «En todas las demás cuestiones, al igual que en estas decisiones ya registradas, te prometo que me someteré a la voluntad de mi Padre»\footnote{\textit{Someterse a la voluntad del Padre}: Mt 26:39,42,44; Mc 14:36,39; Lc 22:42; Jn 4:34; 5:30; 6:38-40; 15:10; 17:4.}. Después de haber dicho esto, descendió de la montaña. Y su faz resplandecía con la gloria de las victorias espirituales y de las proezas morales.