\chapter{Documento 140. La ordenación de los doce}
\par
%\textsuperscript{(1568.1)}
\textsuperscript{140:0.1} El domingo 12 de enero del año 27, un poco antes del mediodía, Jesús reunió a los apóstoles para su ordenación\footnote{\textit{Sermón de la ordenación}: Mt 5:1; Lc 6:17.} como predicadores públicos del evangelio del reino\footnote{\textit{El evangelio del reino}: Mt 4:23; 9:35; 24:14; Mc 1:14-15.}. Los doce esperaban ser llamados de un día a otro; por eso, aquella mañana no se alejaron mucho de la costa para pescar. Algunos de ellos se habían quedado cerca de la orilla reparando sus redes y remendando sus atavíos de pesca.

\par
%\textsuperscript{(1568.2)}
\textsuperscript{140:0.2} Cuando Jesús bajó a la playa para convocar a los apóstoles, primero llamó a Andrés y Pedro, que estaban pescando cerca de la orilla; luego hizo señas a Santiago y Juan, que se encontraban cerca en una barca charlando con su padre Zebedeo y reparando sus redes. Reunió de dos en dos a los otros apóstoles, y cuando los doce estuvieron congregados, se dirigió con ellos hacia las tierras montañosas del norte de Cafarnaúm, donde procedió a instruirlos como preparación para su ordenación formal.

\par
%\textsuperscript{(1568.3)}
\textsuperscript{140:0.3} Por una vez, los doce apóstoles estaban silenciosos; incluso Pedro se hallaba pensativo. ¡Por fin había llegado la hora tanto tiempo esperada! Partían a solas con el Maestro para participar en algún tipo de ceremonia solemne de consagración personal y de dedicación colectiva al trabajo sagrado de representar a su Maestro en la proclamación del advenimiento del reino de su Padre.

\section*{1. La instrucción preliminar}
\par
%\textsuperscript{(1568.4)}
\textsuperscript{140:1.1} Antes del servicio formal de ordenación, Jesús dijo a los doce que estaban sentados a su alrededor: «Hermanos míos, la hora del reino ha llegado. Os he traído aquí, a solas conmigo, para presentaros al Padre como embajadores del reino. Algunos de vosotros me habéis oído hablar de este reino en la sinagoga cuando fuisteis llamados por primera vez. Cada uno de vosotros ha aprendido más sobre el reino del Padre desde que habéis estado trabajando conmigo en las ciudades cercanas al Mar de Galilea. Pero en este momento tengo algo más que deciros con respecto a este reino».

\par
%\textsuperscript{(1568.5)}
\textsuperscript{140:1.2} «El nuevo reino que mi Padre está a punto de establecer en el corazón de sus hijos terrestres está destinado a ser un dominio eterno. Este gobierno de mi Padre en el corazón de aquellos que desean hacer su voluntad divina no tendrá fin. Os declaro que mi Padre no es el Dios de los judíos o de los gentiles. Muchos vendrán del este y del oeste para sentarse con nosotros en el reino del Padre, mientras que muchos hijos de Abraham se negarán a entrar en esta nueva fraternidad, en la que el espíritu del Padre reina en el corazón de los hijos de los hombres»\footnote{\textit{El reino eterno}: Lc 1:33. \textit{No es el Dios de los judíos o de los gentiles}: 2 Cr 19:7; Job 34:19; Eclo 35:12; Hch 10:34; Ro 2:9-11; 9:24; 10:12; Gl 2:6; 3:28; Ef 6:9; Col 3:11. \textit{Muchos vendrán}: Mt 8:11-12; Lc 13:28-29.}.

\par
%\textsuperscript{(1568.6)}
\textsuperscript{140:1.3} «El poder de este reino no consistirá en la fuerza de los ejércitos ni en la importancia de las riquezas, sino más bien en la gloria del espíritu divino que vendrá a enseñar la mente y dirigir el corazón de los ciudadanos renacidos de este reino celestial ---los hijos de Dios. Ésta es la fraternidad del amor donde reina la rectitud y cuyo grito de guerra será: Paz en la Tierra y buena voluntad entre todos los hombres. Este reino, que muy pronto vais a proclamar, es el deseo de los hombres de bien de todos los tiempos, la esperanza de toda la Tierra y el cumplimiento de las sabias promesas de todos los profetas»\footnote{\textit{Paz en la Tierra}: Lc 2:14.}.

\par
%\textsuperscript{(1569.1)}
\textsuperscript{140:1.4} «Pero para vosotros, hijos míos, y para todos los demás que quieran seguiros en este reino, una dura prueba se prepara. Sólo la fe os permitirá atravesar sus puertas, pero tendréis que producir los frutos del espíritu de mi Padre si queréis continuar ascendiendo en la vida progresiva de la comunidad divina. En verdad, en verdad os digo que no todo el que dice `Señor, Señor' entrará en el reino de los cielos, sino más bien aquel que hace la voluntad de mi Padre que está en los cielos»\footnote{\textit{Sólo la fe os atravesará la puerta}: Ro 1:17; Ro 3:28; Ef 2:8. \textit{Producir los frutos del espíritu para continuar}: Gl 5:22-23; Ef 5:9; Stg 2:17,20,26. \textit{El que hace la voluntad de mi Padre}: Mt 7:21.}.

\par
%\textsuperscript{(1569.2)}
\textsuperscript{140:1.5} «Vuestro mensaje para el mundo será: Buscad primero el reino de Dios y su rectitud, y cuando los hayáis encontrado, todas las demás cosas esenciales para la supervivencia eterna estarán aseguradas por añadidura. Ahora quisiera dejar claro para vosotros que este reino de mi Padre no vendrá con una exhibición exterior de poder ni con una demostración indecorosa. No debéis salir de aquí para proclamar el reino diciendo: `está aquí' o `está allí', porque este reino que predicaréis es Dios dentro de vosotros»\footnote{\textit{Buscad primero el reino}: Mt 6:33; Lc 12:31. \textit{No con una demostración indecorosa}: Mt 24:30-31; Mc 13:26; Lc 21:21-27; Hch 1:7-8. \textit{El reino es Dios dentro de vosotros}: Job 32:8,18; Is 63:10-11; Ez 37:14; Mt 10:29; Lc 17:21; Jn 17:21-23; Ro 8:9-11; 1 Co 3:16-17; 6:19; 2 Co 6:16; Gl 2:20; 1 Jn 3:24; 4:12-15; Ap 21:3.}.

\par
%\textsuperscript{(1569.3)}
\textsuperscript{140:1.6} «Quien quiera ser grande en el reino de mi Padre, deberá volverse un ministro para todos; y si alguien quiere ser el primero entre vosotros, que se convierta en el servidor de sus hermanos. Una vez que hayáis sido recibidos realmente como ciudadanos del reino celestial, ya no seréis servidores, sino hijos, hijos del Dios viviente. Así es como este reino progresará en el mundo, hasta que destruya todas las barreras y conduzca a todos los hombres a conocer a mi Padre y a creer en la verdad salvadora que he venido a proclamar. Incluso ahora mismo el reino está cerca, y algunos de vosotros no moriréis hasta que hayáis visto llegar el reino de Dios con gran poder»\footnote{\textit{Ser grande por el ministerio}: Mt 20:26-27; 23:11; Mc 9:35; 10:43-44; Lc 22:26. \textit{Hijos del Dios viviente}: 1 Cr 22:10; Sal 2:7; Is 56:5; Mt 5:9,16,45; Lc 20:36; Jn 1:12-13; 11:52; Hch 17:28-29; Ro 8:14-17,19,21; 9:26; 2 Co 6:18; Gl 3:26; 4:5-7; Ef 1:5; Flp 2:15; Heb 12:5-8; 1 Jn 3:1-2,10; 5:2; Ap 21:7; 2 Sam 7:14. \textit{El reino está cerca}: Mt 3:2; 4:17; 10:7; Mc 1:15; Lc 10:9-11; 17:20-21; 21:31. \textit{El reino de Dios con gran poder}: Hch 2:1-4.}.

\par
%\textsuperscript{(1569.4)}
\textsuperscript{140:1.7} «Esto que vuestros ojos contemplan ahora, este pequeño comienzo de doce hombres comunes, se multiplicará y crecerá hasta que, finalmente, toda la Tierra se llene con las alabanzas de mi Padre. Y no será tanto por las palabras que diréis, sino más bien por la vida que viviréis, como los hombres sabrán que habéis estado conmigo y que habéis aprendido las realidades del reino. Aunque no quisiera colocar ninguna carga pesada sobre vuestra mente, estoy a punto de depositar sobre vuestra alma la solemne responsabilidad de representarme en el mundo cuando os deje dentro de poco, como yo represento ahora a mi Padre en esta vida que estoy viviendo en la carne». Cuando Jesús terminó de hablar, se levantó.

\section*{2. La ordenación}
\par
%\textsuperscript{(1569.5)}
\textsuperscript{140:2.1} Jesús indicó entonces a los doce mortales que acababan de escuchar su declaración sobre el reino que se arrodillaran en círculo alrededor de él. Luego, el Maestro puso sus manos sobre la cabeza de cada apóstol, empezando por Judas Iscariote y terminando por Andrés. Después de haberlos bendecido, extendió las manos y oró:

\par
%\textsuperscript{(1569.6)}
\textsuperscript{140:2.2} «Padre mío, aquí te traigo a estos hombres, mis mensajeros. Entre nuestros hijos de la Tierra, he escogido a estos doce para que vayan a representarme como yo he venido para representarte. Ámalos y acompáñalos como tú me has amado y acompañado. Y ahora, Padre mío, concédeles sabiduría a estos hombres, mientras deposito todos los asuntos del reino venidero entre sus manos. Desearía, si es tu voluntad, permanecer algún tiempo en la Tierra para ayudarlos en sus trabajos por el reino. De nuevo, Padre mío, te doy las gracias por estos hombres, y los confío a tu cuidado mientras me dedico a terminar la obra que me has encomendado».

\par
%\textsuperscript{(1570.1)}
\textsuperscript{140:2.3} Cuando Jesús terminó de orar, cada uno de los apóstoles permaneció inclinado en su sitio. Transcurrieron muchos minutos antes de que el mismo Pedro se atreviera a levantar los ojos para mirar al Maestro. Uno tras otro abrazaron a Jesús, pero nadie dijo nada. Un gran silencio invadió el lugar, mientras que una multitud de seres celestiales contemplaba desde arriba esta escena solemne y sagrada ---el Creador de un universo poniendo los asuntos de la fraternidad divina de los hombres bajo la dirección de unas mentes humanas.

\section*{3. El sermón de ordenación}
\par
%\textsuperscript{(1570.2)}
\textsuperscript{140:3.1} Jesús tomó entonces la palabra y dijo: «Ahora que sois embajadores del reino de mi Padre, os habéis convertido así en una clase de hombres separada y distinta de todos los demás hombres de la Tierra. Ahora ya no sois como unos hombres entre los hombres, sino como unos ciudadanos iluminados de otro país celestial entre las criaturas ignorantes de este mundo tenebroso. Ya no es suficiente con que viváis como habéis hecho hasta ahora, sino que de aquí en adelante deberéis de vivir como aquellos que han saboreado las glorias de una vida mejor, y han sido enviados de vuelta a la Tierra como embajadores del Soberano de ese mundo nuevo y mejor. Se espera más del profesor que del alumno; al amo se le exige más que al servidor. A los ciudadanos del reino celestial se les pide más que a los ciudadanos del gobierno terrestre. Algunas de las cosas que estoy a punto de deciros os parecerán duras, pero habéis elegido representarme en el mundo como yo ahora represento al Padre. Y como agentes míos en la Tierra, estaréis obligados a acatar las enseñanzas y las prácticas que reflejan mis ideales de vida mortal en los mundos del espacio, lo que ejemplifico en mi vida terrestre revelando al Padre que está en los cielos»\footnote{\textit{Jesús comienza el sermón de la ordenación}: Mt 5:2; Lc 6:20. \textit{Sois embajadores}: 2 Co 5:20. \textit{Ciudadanos del cielo}: Heb 11:16. \textit{Se te pedirá más}: Lc 12:48.}.

\par
%\textsuperscript{(1570.3)}
\textsuperscript{140:3.2} «Os envío para proclamar la libertad a los cautivos espirituales, la alegría a los esclavos del miedo, y para curar a los enfermos de acuerdo con la voluntad de mi Padre que está en los cielos. Cuando encontréis a mis hijos en la aflicción, decidles palabras de estímulo como éstas:»\footnote{\textit{Liberar a los cautivos espirituales}: Is 61:1; Lc 4:18.}

\par
%\textsuperscript{(1570.4)}
\textsuperscript{140:3.3} «Bienaventurados los pobres de espíritu, los humildes, porque de ellos son los tesoros del reino de los cielos»\footnote{\textit{Bienaventurados los pobres de espíritu}: Mt 5:3; Lc 6:20b.}.

\par
%\textsuperscript{(1570.5)}
\textsuperscript{140:3.4} «Bienaventurados los que tienen hambre y sed de rectitud, porque ellos serán saciados»\footnote{\textit{Bienaventurados los que tienen hambre y sed de rectitud}: Mt 5:6; Lc 6:21a.}.

\par
%\textsuperscript{(1570.6)}
\textsuperscript{140:3.5} «Bienaventurados los mansos, porque ellos heredarán la Tierra»\footnote{\textit{Bienaventurados los mansos}: Mt 5:5.}.

\par
%\textsuperscript{(1570.7)}
\textsuperscript{140:3.6} «Bienaventurados los limpios de corazón, porque ellos verán a Dios»\footnote{\textit{Bienaventurados los limpios de corazón}: Mt 5:8.}.

\par
%\textsuperscript{(1570.8)}
\textsuperscript{140:3.7} «Y decid también a mis hijos estas palabras adicionales de consuelo espiritual y de promesa:»

\par
%\textsuperscript{(1570.9)}
\textsuperscript{140:3.8} «Bienaventurados los afligidos, porque ellos serán consolados. Bienaventurados los que lloran, porque ellos recibirán el espíritu de la alegría»\footnote{\textit{Bienaventurados los afligidos}: Mt 5:4. \textit{Bienaventurados los que lloran}: Lc 6:21b.}.

\par
%\textsuperscript{(1570.10)}
\textsuperscript{140:3.9} «Bienaventurados los misericordiosos, porque ellos alcanzarán misericordia»\footnote{\textit{Bienaventurados los misericordiosos}: Mt 5:7.}.

\par
%\textsuperscript{(1570.11)}
\textsuperscript{140:3.10} «Bienaventurados los pacificadores, porque ellos serán llamados hijos de Dios»\footnote{\textit{Bienaventurados los pacificadores}: Mt 5:9.}.

\par
%\textsuperscript{(1570.12)}
\textsuperscript{140:3.11} «Bienaventurados los perseguidos a causa de su rectitud, porque de ellos es el reino de los cielos. Consideraos bienaventurados cuando los hombres os injurien y os persigan, y digan falsamente toda clase de mal contra vosotros. Regocijaos y alegraos en extremo, porque vuestra recompensa será grande en los cielos»\footnote{\textit{Bienaventurados los perseguidos}: Mt 5:10-12; Lc 6:22-23a.}.

\par
%\textsuperscript{(1570.13)}
\textsuperscript{140:3.12} «Hermanos míos, mientras os envío fuera, vosotros sois la sal de la Tierra, una sal con sabor de salvación. Pero si esta sal ha perdido su sabor, ¿con qué se sazonará? En lo sucesivo ya no sirve más que para ser arrojada y pisoteada por los hombres»\footnote{\textit{Sois la sal de la Tierra}: Mt 5:13; Mc 9:50; Lc 14:34-35.}.

\par
%\textsuperscript{(1570.14)}
\textsuperscript{140:3.13} «Vosotros sois la luz del mundo. Una ciudad situada en una colina no se puede ocultar. Los hombres tampoco encienden una luz para ponerla debajo de un almud, sino en un candelero; y da luz a todos los que están en la casa. Que vuestra luz brille ante los hombres de tal manera que puedan ver vuestras buenas obras y sean inducidos a glorificar a vuestro Padre que está en los cielos»\footnote{\textit{Vosotros sois la luz del mundo}: Mt 5:14-16; Mc 4:21; Lc 8:16; 11:33.}.

\par
%\textsuperscript{(1571.1)}
\textsuperscript{140:3.14} «Os envío al mundo para que me representéis y actuéis como embajadores del reino de mi Padre. Cuando salgáis a proclamar la buena nueva, poned vuestra confianza en el Padre, de quien sois mensajeros. No resistáis a la injusticia por medio de la fuerza; no pongáis vuestra confianza en el vigor corporal. Si vuestro prójimo os golpea en la mejilla derecha, ofrecedle también la izquierda. Estad dispuestos a sufrir una injusticia en lugar de acudir a la ley entre vosotros. Atended con bondad y misericordia a todos los que están afligidos y necesitados»\footnote{\textit{Embajadores del reino}: 2 Co 5:20. \textit{Estad dispuestos a la injusticia}: Mt 5:39; Lc 6:29. \textit{No litiguéis unos con otros}: 1 Co 6:1-7.}.

\par
%\textsuperscript{(1571.2)}
\textsuperscript{140:3.15} «Os lo digo: amad a vuestros enemigos, haced el bien a los que os odian, bendecid a los que os maldicen, y orad por los que os utilizan con malicia. Haced por los hombres todo lo que creéis que yo haría por ellos»\footnote{\textit{Ama a tus enemigos}: Mt 5:44; Lc 6:27-28. \textit{La regla de oro}: Mt 7:12; Lc 6:31.}.

\par
%\textsuperscript{(1571.3)}
\textsuperscript{140:3.16} «Vuestro Padre que está en los cielos hace que el Sol brille sobre los malos al igual que sobre los buenos; asimismo, envía la lluvia sobre los justos y los injustos. Vosotros sois los hijos de Dios; aún más, ahora sois los embajadores del reino de mi Padre. Sed misericordiosos como Dios es misericordioso, y en el eterno futuro del reino, seréis perfectos como vuestro Padre celestial es perfecto»\footnote{\textit{El Padre trae el sol y la lluvia}: Mt 5:45. \textit{Sed misericordiosos como Dios lo es}: Lc 6:36. \textit{Sed perfectos}: Gn 17:1; 1 Re 8:61; Lv 19:2; Dt 18:13; Mt 5:48; 2 Co 13:11; Stg 1:4; 1 P 1:16.}.

\par
%\textsuperscript{(1571.4)}
\textsuperscript{140:3.17} «Estáis encargados de salvar a los hombres, no de juzgarlos. Al final de vuestra vida terrestre, todos esperaréis misericordia; por eso os pido que durante vuestra vida mortal mostréis misericordia a todos vuestros hermanos en la carne. No cometáis el error de intentar quitar una mota del ojo de vuestro hermano, cuando hay una viga en el vuestro. Después de sacar primero la viga de vuestro propio ojo, podréis ver mejor para quitar la mota del ojo de vuestro hermano»\footnote{\textit{Salvad a los hombres, no los juzguéis}: Mt 7:1-2; Lc 6:37. \textit{La mota en el ojo ajeno y la viga en el propio}: Mt 7:3-5; Lc 6:41-42.}.

\par
%\textsuperscript{(1571.5)}
\textsuperscript{140:3.18} «Discernid claramente la verdad; vivid con audacia la vida recta; así seréis mis apóstoles y los embajadores de mi Padre. Habéis oído decir que: `Si el ciego conduce al ciego, los dos se caerán al precipicio'. Si queréis guiar a otras personas hacia el reino, vosotros mismos tenéis que caminar en la clara luz de la verdad viviente. En todos los asuntos del reino, os exhorto a que mostréis un juicio justo y una sabiduría penetrante. No ofrezcáis las cosas santas a los perros, ni arrojéis vuestras perlas delante de los cerdos, no sea que pisoteen vuestras joyas y se vuelvan para despedazaros»\footnote{\textit{Si el ciego conduce al ciego}: Mt 15:14; Lc 6:39. \textit{No arrojéis vuestras perlas a los cerdos}: Mt 7:6.}.

\par
%\textsuperscript{(1571.6)}
\textsuperscript{140:3.19} «Os pongo en guardia contra los falsos profetas que vendrán hacia vosotros vestidos de cordero, mientras que por dentro son como lobos voraces. Por sus frutos los conoceréis. ¿Recogen los hombres uvas de las espinas o higos de los cardos? Así pues, todo buen árbol produce buenos frutos, pero el árbol corrompido da malos frutos. Un buen árbol no puede producir malos frutos, ni un árbol corrompido buenos frutos. Todo árbol que no da buenos frutos pronto es derribado y arrojado al fuego. Para conseguir entrar en el reino de los cielos, lo que cuenta es el móvil. Mi Padre mira dentro del corazón de los hombres y los juzga por sus deseos internos y sus intenciones sinceras»\footnote{\textit{Advertencia contra los falsos profetas}: Mt 7:15,22-23; Mt 24:11,24; Mc 13:22; 1 Jn 4:1. \textit{Por sus frutos los conoceréis}: Mt 7:16-20; Lc 6:43-44. \textit{Frutos del espíritu}: Gl 5:22-23; Ef 5:9. \textit{Requerimientos para dar buenos frutos}: Mt 3:10; Mt 12:33; Lc 3:9; Lc 6:43-44; Lc 13:6,9; Jn 15:7-8,16. \textit{Dios juzga el corazón}: Lc 16:15; Hch 1:24; 1 Ts 2:4.}.

\par
%\textsuperscript{(1571.7)}
\textsuperscript{140:3.20} «En el gran día del juicio del reino, muchos me dirán: `¿No hemos profetizado en tu nombre y hemos hecho muchas obras maravillosas por tu nombre?' Pero yo me veré obligado a decirles, `Nunca os he conocido; apartaos de mí, vosotros que sois unos falsos educadores'. Pero todo el que escuche esta instrucción y ejecute sinceramente su misión de representarme ante los hombres, como yo he representado a mi Padre ante vosotros, encontrará una entrada abundante a mi servicio y en el reino del Padre celestial»\footnote{\textit{Falsas presunciones de fe}: Mt 7:22-23. \textit{Recompensa de los creyentes}: Mt 7:24; Lc 6:47.}.

\par
%\textsuperscript{(1571.8)}
\textsuperscript{140:3.21} Los apóstoles nunca habían escuchado antes a Jesús expresarse de esta manera, pues les había hablado como alguien que posee una autoridad suprema\footnote{\textit{Jesús hablaba con autoridad}: Mt 7:28-29; Mc 1:22.}. Descendieron de la montaña casi al ponerse el Sol, pero ninguno le hizo preguntas a Jesús.

\section*{4. Vosotros sois la sal de la Tierra}
\par
%\textsuperscript{(1572.1)}
\textsuperscript{140:4.1} El llamado «Sermón de la Montaña» no es el evangelio de Jesús. Contiene de hecho muchas enseñanzas útiles, pero eran las instrucciones de ordenación de Jesús a los doce apóstoles. Era el encargo personal del Maestro a los que iban a continuar predicando el evangelio y que aspiraban a representarlo en el mundo de los hombres, como él representaba a su Padre con tanta elocuencia y perfección.

\par
%\textsuperscript{(1572.2)}
\textsuperscript{140:4.2} \textit{«Vosotros sois la sal de la Tierra, una sal con sabor de salvación. Pero si esta sal ha perdido su sabor, ¿con qué se sazonará? En lo sucesivo ya no sirve más que para ser arrojada y pisoteada por los hombres».}

\par
%\textsuperscript{(1572.3)}
\textsuperscript{140:4.3} En los tiempos de Jesús, la sal era un elemento precioso. Se utilizaba incluso como moneda. La palabra moderna «salario» se deriva de sal. La sal no sólo condimenta los alimentos, sino que también los conserva. Hace que otras cosas sean más sabrosas, y sirve así a medida que se gasta.

\par
%\textsuperscript{(1572.4)}
\textsuperscript{140:4.4} \textit{«Vosotros sois la luz del mundo. Una ciudad situada en una colina no se puede ocultar. Los hombres tampoco encienden una luz para ponerla debajo de un almud, sino en un candelero; y da luz a todos los que están en la casa. Que vuestra luz brille ante los hombres de tal manera que puedan ver vuestras buenas obras y sean inducidos a glorificar a vuestro Padre que está en los cielos».}\footnote{\textit{Vosotros sois la luz del mundo}: Mt 5:14-16; Mc 4:21; Lc 8:16; 11:33.}

\par
%\textsuperscript{(1572.5)}
\textsuperscript{140:4.5} Aunque la luz disipa las tinieblas, también puede ser tan «cegadora» como para confundir y frustrar. Se nos exhorta a que dejemos brillar nuestra luz \textit{de tal manera} que nuestros semejantes se sientan guiados hacia unos caminos nuevos y divinos de vida realzada. Nuestra luz no debe brillar como para atraer la atención sobre nosotros mismos. También podemos utilizar nuestra propia profesión como un «reflector» eficaz para diseminar esta luz de la vida.

\par
%\textsuperscript{(1572.6)}
\textsuperscript{140:4.6} Los caracteres fuertes no se forman \textit{evitando} hacer el mal, sino más bien haciendo realmente el bien. El altruismo es la insignia de la grandeza humana. Los niveles más altos de autorrealización se alcanzan mediante la adoración y el servicio. La persona feliz y eficaz está motivada por el amor de hacer el bien, y no por el temor de hacer el mal.

\par
%\textsuperscript{(1572.7)}
\textsuperscript{140:4.7} \textit{«Por sus frutos los conoceréis».}\footnote{\textit{Por sus frutos los conoceréis}: Mt 7:16-20; Lc 6:43-44. \textit{Frutos del espíritu}: Gl 5:22-23; Ef 5:9.} La personalidad es básicamente invariable. Lo que cambia ---lo que crece--- es el carácter moral. El error principal de las religiones modernas es el negativismo. El árbol que no produce frutos es «derribado y arrojado al fuego»\footnote{\textit{El árbol estéril será talado}: Mt 7:19.}. El valor moral no puede provenir de la simple represión ---de la obediencia al mandato «No harás». El miedo y la verg\"uenza son motivaciones sin valor para la vida religiosa. La religión solamente es válida cuando revela la paternidad de Dios y realza la fraternidad de los hombres.

\par
%\textsuperscript{(1572.8)}
\textsuperscript{140:4.8} Una persona se forma una filosofía eficaz de la vida combinando la perspicacia cósmica con la suma de sus propias reacciones emocionales ante el entorno social y económico. Recordad: aunque los impulsos hereditarios no se pueden modificar fundamentalmente, las reacciones emocionales a esos impulsos sí se pueden cambiar; por consiguiente, la naturaleza moral se puede modificar, el carácter se puede mejorar. En un carácter fuerte, las reacciones emocionales están integradas y coordinadas, generando así una personalidad unificada. La falta de unificación debilita la naturaleza moral y engendra la desdicha.

\par
%\textsuperscript{(1572.9)}
\textsuperscript{140:4.9} Sin una meta que merezca la pena, la vida carece de objetivo y de provecho, lo que ocasiona mucha infelicidad. El discurso de Jesús en la ordenación de los doce constituye una filosofía magistral de la vida. Jesús exhortó a sus seguidores a que ejercitaran una fe experiencial. Les advirtió que no se limitaran a depender de un asentimiento intelectual, de la credulidad o de la autoridad establecida.

\par
%\textsuperscript{(1573.1)}
\textsuperscript{140:4.10} La educación debería ser una técnica para aprender (para descubrir) los mejores métodos de satisfacer nuestros impulsos naturales y hereditarios, y la felicidad es el resultado final de estas técnicas mejores de satisfacción emocional. La felicidad depende poco del entorno, aunque un ambiente agradable puede contribuir mucho a ella.

\par
%\textsuperscript{(1573.2)}
\textsuperscript{140:4.11} Todo mortal ansía realmente ser una persona completa, ser perfecto\footnote{\textit{Sed perfectos}: Gn 17:1; 1 Re 8:61; Lv 19:2; Dt 18:13; Mt 5:48; 2 Co 13:11; Stg 1:4; 1 P 1:16.} como el Padre que está en los cielos es perfecto, y este logro es posible porque, a fin de cuentas, el «universo es verdaderamente paternal».

\section*{5. Amor paternal y amor fraternal}
\par
%\textsuperscript{(1573.3)}
\textsuperscript{140:5.1} Desde el Sermón de la Montaña hasta el discurso de la Última Cena, Jesús enseñó a sus discípulos a manifestar un amor \textit{paternal}\footnote{\textit{Amor paternal}: Jn 14:21,23; 15:9-12; 17:23,26; 1 Jn 3:1.} en lugar de un amor \textit{fraternal}. El amor fraternal consiste en amar al prójimo como a sí mismo, lo que sería una aplicación adecuada de la «regla de oro». Pero el afecto paternal exige que améis a vuestros compañeros mortales como Jesús os ama.

\par
%\textsuperscript{(1573.4)}
\textsuperscript{140:5.2} Jesús ama a la humanidad con un afecto doble. Vivió en la Tierra bajo una doble personalidad ---humana y divina. Como Hijo de Dios, ama al hombre con un amor paternal ---es el Creador del hombre, su Padre en el universo. Como Hijo del Hombre, Jesús ama a los mortales como un hermano ---fue realmente un hombre entre los hombres.

\par
%\textsuperscript{(1573.5)}
\textsuperscript{140:5.3} Jesús no esperaba que sus discípulos consiguieran una manifestación imposible de amor fraternal, pero sí contaba con que se esforzarían tanto por parecerse a Dios ---por ser perfectos como el Padre que está en los cielos es perfecto--- que podrían empezar a considerar a los hombres como Dios considera a sus criaturas, y así podrían empezar a amar a los hombres como Dios los ama ---a manifestar los principios de un afecto paternal. En el transcurso de estas exhortaciones a los doce apóstoles, Jesús trató de revelar este nuevo concepto de \textit{amor paternal}, tal como está relacionado con ciertas actitudes emocionales involucradas cuando se efectúan numerosos ajustes sociales al entorno.

\par
%\textsuperscript{(1573.6)}
\textsuperscript{140:5.4} El Maestro inició este importante discurso llamando la atención sobre cuatro actitudes de \textit{fe}, como preludio a la descripción posterior de sus cuatro reacciones trascendentales y supremas de amor paternal, en contraste con las limitaciones del simple amor fraternal.

\par
%\textsuperscript{(1573.7)}
\textsuperscript{140:5.5} Primero habló de los que eran pobres de espíritu, de los que tenían hambre de rectitud, de los que perseveraban en la mansedumbre y de los limpios de corazón. Se podría esperar que estos mortales que disciernen el espíritu alcanzarían los niveles suficientes de desinterés divino como para ser capaces de intentar el extraordinario ejercicio del afecto \textit{paternal;} que, incluso en la aflicción, estarían facultados para mostrar misericordia, promover la paz y soportar las persecuciones. Y que a lo largo de todas estas penosas situaciones, amarían con un amor paternal incluso a una humanidad poco amable. El afecto de un padre puede alcanzar unos niveles de devoción que trascienden inmensamente el afecto de un hermano.

\par
%\textsuperscript{(1573.8)}
\textsuperscript{140:5.6} La fe y el amor de estas beatitudes fortalecen el carácter moral y crean la felicidad. El miedo y la ira debilitan el carácter y destruyen la felicidad. Este sermón importante se inició con una nota de felicidad.

\par
%\textsuperscript{(1573.9)}
\textsuperscript{140:5.7} 1. \textit{«Bienaventurados los pobres de espíritu ---los humildes»}\footnote{\textit{Los pobres de espíritu, los humildes}: Mt 5:3; Lc 6:20b.}. Para un niño, la felicidad es la satisfacción de una ansia inmediata de placer. El adulto está dispuesto a sembrar las semillas de la abnegación, con el fin de obtener las cosechas posteriores de una felicidad mayor. En los tiempos de Jesús y después de ellos, la felicidad ha sido asociada demasiado a menudo con la idea de poseer riquezas. En la historia del fariseo y del publicano que oraban en el templo\footnote{\textit{La oración del publicano}: Lc 18:10-14.}, uno se sentía rico de espíritu ---egotista; el otro se sentía «pobre de espíritu» ---humilde. Uno era autosuficiente; el otro era enseñable y buscaba la verdad. Los pobres de espíritu buscan metas de riqueza espiritual ---buscan a Dios. Estos buscadores de la verdad no tienen que esperar sus recompensas en un futuro lejano; son recompensados \textit{ahora}. Encuentran el reino de los cielos en su propio corazón, y experimentan esa felicidad \textit{ahora}.

\par
%\textsuperscript{(1574.1)}
\textsuperscript{140:5.8} 2. \textit{«Bienaventurados los que tienen hambre y sed de rectitud, porque ellos serán saciados»}\footnote{\textit{Los hambrientos y sedientos de rectitud}: Mt 5:6; Lc 6:21a.}. Sólo aquellos que se sienten pobres de espíritu tienen sed de rectitud. Sólo los humildes buscan la fuerza divina y anhelan el poder espiritual. Sin embargo, es sumamente peligroso practicar a sabiendas el ayuno espiritual con el fin de aumentar nuestro apetito de los dones espirituales. El ayuno físico se vuelve peligroso después de cuatro o cinco días; uno puede perder todo deseo de alimentarse. El ayuno prolongado, tanto físico como espiritual, tiende a destruir el apetito.

\par
%\textsuperscript{(1574.2)}
\textsuperscript{140:5.9} La rectitud experiencial es un placer, no un deber. La rectitud de Jesús es un amor dinámico ---un afecto paterno-fraternal. No es una rectitud negativa del tipo «no harás». ¿Cómo podría alguien tener hambre de algo negativo ---de algo a «no hacer»?

\par
%\textsuperscript{(1574.3)}
\textsuperscript{140:5.10} No es fácil enseñar estas dos primeras beatitudes a una mente infantil, pero la mente madura debería captar su significado.

\par
%\textsuperscript{(1574.4)}
\textsuperscript{140:5.11} 3. \textit{«Bienaventurados los mansos, porque ellos heredarán la Tierra»}\footnote{\textit{Los mansos heredarán la Tierra}: Mt 5:5.}. La mansedumbre auténtica no tiene ninguna relación con el miedo. Es más bien una actitud del hombre cooperando con Dios ---«Hágase tu voluntad»\footnote{\textit{Hágase tu voluntad}: Mt 6:10; 26:39,42,44; Mc 14:36,39; Lc 11:2; 22:42.}. Engloba la paciencia y la indulgencia, y está motivada por una fe imperturbable en un universo justo y amistoso. Domina todas las tentaciones de rebelarse contra el gobierno divino. Jesús fue el hombre manso ideal de Urantia, y heredó un vasto universo.

\par
%\textsuperscript{(1574.5)}
\textsuperscript{140:5.12} 4. \textit{«Bienaventurados los limpios de corazón, porque ellos verán a Dios»}\footnote{\textit{Los puros de corazón}: Mt 5:8.}. La pureza espiritual no es una cualidad negativa, salvo que carece de recelo y de revancha. Al hablar de la pureza, Jesús no tenía la intención de tratar exclusivamente de las actitudes sexuales humanas. Se refería más bien a esa fe que los hombres deberían tener en sus semejantes; a esa fe que los padres tienen en sus hijos, y que les permite amar a sus semejantes como un padre los amaría. El amor de un padre no tiene necesidad de mimar, y no perdona el mal, pero siempre se opone al cinismo. El amor paternal tiene una única finalidad, y siempre busca lo mejor que hay en el hombre; ésta es la actitud de un verdadero padre.

\par
%\textsuperscript{(1574.6)}
\textsuperscript{140:5.13} Ver a Dios ---por la fe--- significa adquirir la verdadera perspicacia espiritual. La perspicacia espiritual intensifica el gobierno del Ajustador, y los dos reunidos terminan por aumentar la conciencia de Dios. Cuando conocéis al Padre, os sentís confirmados en la seguridad de vuestra filiación divina, y podéis amar cada vez más a vuestros hermanos en la carne, no sólo como un hermano ---con un amor fraternal--- sino también como un padre ---con un afecto paternal\footnote{\textit{Amor paternal}: Jn 3:16; Jn 17:23,26; 1 Jn 3:1,16; 1 Jn 4:9-11,19.}.

\par
%\textsuperscript{(1574.7)}
\textsuperscript{140:5.14} Esta exhortación es fácil de enseñar incluso a un niño. Los niños son confiados por naturaleza, y los padres deberían cuidar de que no pierdan esta fe sencilla. Al tratar con los niños, evitad todo engaño y absteneos de sugerir la desconfianza. Ayudadlos juiciosamente a escoger a sus héroes y a seleccionar el trabajo de su vida.

\par
%\textsuperscript{(1574.8)}
\textsuperscript{140:5.15} Luego, Jesús continuó instruyendo a sus discípulos sobre cómo conseguir el objetivo principal de todas las luchas humanas ---la perfección--- e incluso la consecución divina. Siempre les recomendaba: «Sed perfectos como vuestro Padre que está en los cielos es perfecto»\footnote{\textit{Sed perfectos}: Gn 17:1; 1 Re 8:61; Lv 19:2; Dt 18:13; Mt 5:48; 2 Co 13:11; Stg 1:4; 1 P 1:16.}. No exhortaba a los doce a que amaran al prójimo como se amaban a sí mismos. Esto hubiera sido un logro meritorio, que hubiera indicado la realización del amor fraternal. Recomendaba más bien a sus apóstoles que amaran a los hombres como él los había amado ---con un afecto \textit{paternal} así como fraternal. Y esto lo ilustró indicando cuatro reacciones supremas de amor paternal:

\par
%\textsuperscript{(1575.1)}
\textsuperscript{140:5.16} 1. \textit{«Bienaventurados los afligidos, porque ellos serán consolados»}\footnote{\textit{Los afligidos serán confortados}: Mt 5:4; Lc 6:21b.}. El llamado sentido común o la lógica más superior nunca sugerirían que la felicidad puede surgir de la aflicción. Pero Jesús no se refería a la aflicción externa u ostentatoria. Hacía alusión a una actitud emotiva de ternura de corazón. Es un gran error enseñar a los niños y a los jóvenes que no es varonil mostrar ternura\footnote{\textit{La ternura es varonil}: Mc 14:72; Lc 7:32; Lc 19:41; Lc 22:62; Jn 11:35; Hch 20:37; Ap 5:4.} o, por otra parte, dar testimonio de sentimientos emotivos o de sufrimientos físicos. La compasión es un atributo valioso tanto en el hombre como en la mujer. No es necesario ser insensible para ser varonil. Ésta es la manera equivocada de crear hombres valientes. Los grandes hombres de este mundo no han tenido miedo de afligirse. Moisés, el afligido\footnote{\textit{Moisés, el afligido}: Nm 12:3.}, fue un hombre más grande que Sansón o Goliat. Moisés fue un guía extraordinario, pero también estaba lleno de mansedumbre. Ser sensible y reaccionar antes las necesidades humanas crea una felicidad auténtica y duradera, y al mismo tiempo estas actitudes benévolas protegen el alma contra las influencias destructivas de la ira, el odio y la desconfianza.

\par
%\textsuperscript{(1575.2)}
\textsuperscript{140:5.17} 2. \textit{«Bienaventurados los misericordiosos, porque ellos conseguirán misericordia»}\footnote{\textit{Los misericordiosos obtendrán misericordia}: Mt 5:7.}. La misericordia denota aquí la altura, la profundidad y la anchura de la amistad más sincera ---la bondad. A veces, la misericordia puede ser pasiva, pero aquí es activa y dinámica--- la ternura paternal suprema. Un padre amoroso tiene pocas dificultades para perdonar a su hijo, incluso muchas veces. En un niño no mimado, el impulso de aliviar el sufrimiento es natural. Los niños son normalmente bondadosos y compasivos cuando tienen la edad suficiente para apreciar las situaciones reales.

\par
%\textsuperscript{(1575.3)}
\textsuperscript{140:5.18} 3. \textit{«Bienaventurados los pacificadores, porque ellos serán llamados hijos de Dios»}\footnote{\textit{Los pacificadores serán llamados hijos de Dios}: Mt 5:9.}. Los oyentes de Jesús deseaban ardientemente una liberación militar, no unos pacificadores. Pero la paz de Jesús\footnote{\textit{La paz de Jesús}: Jn 14:27a.} no es de tipo pacífico y negativo. En presencia de las pruebas y de las persecuciones, decía: «Mi paz os dejo». «Que vuestro corazón no se perturbe, y no tengáis miedo»\footnote{\textit{La paz cura el corazón perturbado}: Jn 14:1. \textit{No tengáis miedo}: Jn 14:27b.}. Ésta es la paz que impide los conflictos ruinosos. La paz personal integra la personalidad. La paz social impide el miedo, la codicia y la ira. La paz política impide los antagonismos raciales, las desconfianzas nacionales y la guerra. La pacificación es el remedio para la desconfianza y la sospecha.

\par
%\textsuperscript{(1575.4)}
\textsuperscript{140:5.19} Es fácil enseñar a los niños a trabajar como pacificadores. Disfrutan con las actividades de equipo; les gusta jugar juntos. El Maestro dijo en otra ocasión: «Quien quiera salvar su vida la perderá, pero quien esté dispuesto a perderla, la encontrará»\footnote{\textit{Quien quiera salvar su vida, la perderá}: Mt 10:39; 16:25; Mc 8:35; 9:24; Lc 17:33; Jn 12:25.}.

\par
%\textsuperscript{(1575.5)}
\textsuperscript{140:5.20} 4. \textit{«Bienaventurados los perseguidos a causa de su rectitud, porque de ellos es el reino de los cielos. Consideraos bienaventurados cuando los hombres os injurien y os persigan, y digan falsamente toda clase de mal contra vosotros. Regocijaos y alegraos en extremo, porque vuestra recompensa será grande en los cielos»}\footnote{\textit{Los perseguidos por su rectitud}: Mt 5:10-12a; Lc 6:22-23a.}.

\par
%\textsuperscript{(1575.6)}
\textsuperscript{140:5.21} Muy a menudo, la persecución sigue de hecho a la paz. Pero los jóvenes y los adultos valientes no huyen nunca de las dificultades o del peligro. «No existe un amor más grande que el de dar la vida por sus amigos»\footnote{\textit{No existe un amor más grande}: Jn 15:13.}. Un amor paternal puede hacer libremente todas estas cosas ---unas cosas que el amor fraternal difícilmente puede abarcar. El progreso ha sido siempre la cosecha final de la persecución.

\par
%\textsuperscript{(1575.7)}
\textsuperscript{140:5.22} Los niños responden siempre al desafío de la valentía. La juventud siempre está dispuesta a «aceptar un desafío». Todos los niños deberían aprender pronto a sacrificarse.

\par
%\textsuperscript{(1575.8)}
\textsuperscript{140:5.23} Se descubre pues que las bienaventuranzas del Sermón de la Montaña están basadas en la fe y el amor, y no en la ley ---en la ética y el deber.

\par
%\textsuperscript{(1575.9)}
\textsuperscript{140:5.24} El amor paternal se complace en devolver el bien por el mal ---en hacer el bien como pago a la injusticia.

\section*{6. La noche de la ordenación}
\par
%\textsuperscript{(1576.1)}
\textsuperscript{140:6.1} El domingo por la noche, al llegar de las tierras altas del norte de Cafarnaúm a la casa de Zebedeo, Jesús y los doce compartieron una cena sencilla. Más tarde, mientras Jesús se fue a pasear por la playa, los doce hablaron entre ellos. Después de una breve conversación, mientras los gemelos encendían un pequeño fuego para calentarse y tener más luz, Andrés salió a buscar a Jesús; cuando le dio alcance, le dijo: «Maestro, mis hermanos son incapaces de comprender lo que has dicho sobre el reino. No nos sentimos en condiciones de empezar este trabajo hasta que nos hayas dado más enseñanzas. He venido para pedirte que te reúnas con nosotros en el jardín y nos ayudes a comprender el significado de tus palabras». Y Jesús fue con Andrés para reunirse con los apóstoles.

\par
%\textsuperscript{(1576.2)}
\textsuperscript{140:6.2} Cuando hubo entrado en el jardín, congregó a los apóstoles a su alrededor y continuó enseñándoles, diciendo: «Encontráis difícil recibir mi mensaje porque quisierais construir la nueva enseñanza directamente sobre la antigua, pero os afirmo que tenéis que renacer. Tenéis que comenzar de nuevo como niños pequeños y estar dispuestos a confiar en mi enseñanza y a creer en Dios. El nuevo evangelio del reino no se puede amoldar a lo que existe. Tenéis ideas equivocadas sobre el Hijo del Hombre y su misión en la Tierra. Pero no cometáis el error de pensar que he venido para rechazar la ley y los profetas; no he venido para destruir, sino para completar, ampliar e iluminar. No he venido para transgredir la ley, sino más bien para escribir estos nuevos mandamientos en las tablas de vuestro corazón»\footnote{\textit{Tenéis que renacer}: Jn 3:3,7; 1 P 1:23. \textit{Comenzar como niños pequeños}: Mt 18:2-4; 19:13-14; Mc 9:36-37; 10:13-15; Lc 9:47-48; 18:16-17. \textit{No vengo a transgredir la ley, sino a completarla}: Mt 5:17.}.

\par
%\textsuperscript{(1576.3)}
\textsuperscript{140:6.3} «Exijo de vosotros una rectitud que sobrepasará a la de aquellos que intentan obtener el favor del Padre con la limosna, la oración y el ayuno. Si queréis entrar en el reino, habréis de tener una rectitud que consiste en el amor, la misericordia y la verdad ---el deseo sincero de hacer la voluntad de mi Padre que está en los cielos»\footnote{\textit{Una rectitud mayor}: Mt 5:20.}.

\par
%\textsuperscript{(1576.4)}
\textsuperscript{140:6.4} Entonces, Simón Pedro dijo: «Maestro, si tienes un nuevo mandamiento, quisiéramos oírlo. Revélanos el nuevo camino»\footnote{\textit{Un nuevo mandamiento}: Mt 5:21-22.}. Jesús le contestó a Pedro: «Habéis oído decir a los que enseñan la ley: `No matarás; y cualquiera que mate estará sujeto a juicio'. Pero yo miro más allá del acto para descubrir el móvil. Os declaro que todo aquel que está irritado contra su hermano está en peligro de ser condenado. El que alimenta el odio en su corazón y planea la venganza en su mente, corre el peligro de ser juzgado. Tenéis que juzgar a vuestros compañeros por sus actos; el Padre que está en los cielos juzga según las intenciones»\footnote{\textit{Habéis oído «No matarás»}: Ex 20:13; Dt 5:17; Mt 5:21. \textit{Judgaz los motivos, no los actos}: 1 Sam 16:7. \textit{Corre peligro de ser juzgado}: Lv 24:17,21; Nm 35:30.}.

\par
%\textsuperscript{(1576.5)}
\textsuperscript{140:6.5} «Habéis oído decir a los maestros de la ley: `No cometerás adulterio'. Pero yo os digo que todo hombre que mira a una mujer con intenciones de lujuria, ya ha cometido adulterio con ella en su corazón. Sólo podéis juzgar a los hombres por sus actos, pero mi Padre mira dentro del corazón de sus hijos y los juzga con misericordia según sus intenciones y deseos reales»\footnote{\textit{No cometáis adulterio}: Ex 20:14; Dt 5:18. \textit{Pero yo os digo «La lujuria es adulterio»}: Mt 5:27-28.}.

\par
%\textsuperscript{(1576.6)}
\textsuperscript{140:6.6} Jesús estaba dispuesto a continuar examinando los otros mandamientos, cuando Santiago Zebedeo le interrumpió para preguntar: «Maestro, ¿qué vamos a enseñar a la gente sobre el divorcio? ¿Hemos de permitir que un hombre se divorcie de su mujer como Moisés lo ordenó?»\footnote{\textit{Respecto al divorcio}: Mt 5:31-32; 19:3-9; Mc 10:2-12; Lc 16:18.} Cuando Jesús escuchó esta pregunta, dijo: «No he venido para legislar, sino para iluminar. No he venido para reformar los reinos de este mundo, sino más bien para establecer el reino de los cielos. No es voluntad del Padre que ceda a la tentación de enseñaros reglas de gobierno, de comercio o de conducta social; aunque pudieran ser buenas para hoy, estarían lejos de ser convenientes para la sociedad de otra época. Estoy en la Tierra únicamente para confortar la mente, liberar el espíritu y salvar el alma de los hombres. Pero sobre esta cuestión del divorcio os diré que, aunque Moisés consideraba favorablemente estas cosas, no era así en los tiempos de Adán ni en el Jardín».

\par
%\textsuperscript{(1577.1)}
\textsuperscript{140:6.7} Después de que los apóstoles hubieron hablado entre ellos durante unos momentos, Jesús continuó diciendo: «Tenéis que reconocer siempre los dos puntos de vista de toda conducta de los mortales ---el humano y el divino; los caminos de la carne y la senda del espíritu; la opinión del tiempo y el punto de vista de la eternidad»\footnote{\textit{Dos visiones, la carne y el espíritu}: Jn 3:6; Ro 8:4-5; Gl 5:16-17.}. Aunque los doce no podían comprender todo lo que les enseñaba, esta instrucción les ayudó realmente mucho.

\par
%\textsuperscript{(1577.2)}
\textsuperscript{140:6.8} Entonces Jesús dijo: «Pero vais a tropezar con mis enseñanzas porque estáis acostumbrados a interpretar mi mensaje literalmente; sois lentos en discernir el espíritu de mi enseñanza. Debéis recordar otra vez que sois mis mensajeros; estáis obligados a vivir vuestra vida como yo he vivido la mía en espíritu. Sois mis representantes personales; pero no cometáis el error de esperar que todos los hombres vivan como vosotros en todos los aspectos. También debéis recordar que tengo ovejas que no pertenecen a este rebaño, y que también estoy en deuda con ellos, ya que he de proporcionarles el modelo para hacer la voluntad de Dios, mientras vivo la vida de la naturaleza mortal»\footnote{\textit{Tengo ovejas de otro rebaño}: Jn 10:16.}.

\par
%\textsuperscript{(1577.3)}
\textsuperscript{140:6.9} Entonces Natanael preguntó: «Maestro, ¿no vamos a dejar ningún lugar para la justicia? La ley de Moisés dice: `ojo por ojo y diente por diente'. ¿Qué vamos a decir nosotros?»\footnote{\textit{Ojo por ojo y diente por diente}: Ex 21:24; Lv 24:20; Dt 19:21; Mt 5:38.} Y Jesús contestó: «Vosotros devolveréis el bien por el mal. Mis mensajeros no deben luchar con los hombres, sino ser dulces con todos. Vuestra regla no será medida por medida. Los gobernantes de los hombres pueden tener tales leyes, pero no es así en el reino; la misericordia determinará siempre vuestro juicio, y el amor vuestra conducta. Y si estas afirmaciones os parecen duras, aun podéis echaros atrás. Si los requisitos del apostolado los encontráis demasiado duros, podéis volver al camino menos riguroso de los discípulos»\footnote{\textit{Devolver el bien por el mal}: Mt 5:38-42. \textit{Mayor exigencia a los apóstoles}: Lc 14:25-72; Jn 6:60-69.}.

\par
%\textsuperscript{(1577.4)}
\textsuperscript{140:6.10} Al escuchar estas palabras sorprendentes, los apóstoles se alejaron entre ellos un momento, pero no tardaron en volver, y Pedro dijo: «Maestro, queremos seguir contigo; ninguno de nosotros quiere volverse atrás. Estamos plenamente preparados para pagar el precio adicional; beberemos la copa. Queremos ser apóstoles, no simplemente discípulos».

\par
%\textsuperscript{(1577.5)}
\textsuperscript{140:6.11} Cuando Jesús oyó esto, dijo: «Estad dispuestos entonces a asumir vuestras responsabilidades y a seguirme. Haced vuestras buenas acciones en secreto; cuando deis una limosna, que la mano izquierda no sepa lo que hace la mano derecha. Cuando oréis, hacedlo a solas y no utilicéis vanas repeticiones y frases sin sentido. Recordad siempre que el Padre sabe lo que necesitáis incluso antes de que se lo pidáis. Y no os pongáis a ayunar con un aspecto triste para que os vean los hombres. Como mis apóstoles escogidos, reservados ahora para el servicio del reino, no acumuléis tesoros en la Tierra, sino que, mediante vuestro servicio desinteresado, guardad tesoros en el cielo, porque allí donde estén vuestros tesoros estará también vuestro corazón»\footnote{\textit{Asumid vuestras responsabilidades y seguidme}: Mt 16:24; Mc 8:34; 10:21; Lc 9:23. \textit{Haced buenas acciones en secreto}: Mt 6:1-4. \textit{Orad a solas}: Mt 6:5-7. \textit{Dios sabe lo que necesitamos antes de pedírselo}: Is 65:24; Mt 6:32. \textit{No ayunéis con aspecto triste}: Mt 6:16-18. \textit{Guardad tesoros en el cielo}: Mt 6:19-21; Lc 12:33-34.}.

\par
%\textsuperscript{(1577.6)}
\textsuperscript{140:6.12} «El ojo es la lámpara del cuerpo; por lo tanto, si vuestro ojo es generoso, todo vuestro cuerpo estará lleno de luz. Pero si vuestro ojo es egoísta, todo vuestro cuerpo estará lleno de tinieblas. Si la luz misma que está en vosotros se convierte en tinieblas, ¡cuán profundas serán esas tinieblas!»\footnote{\textit{El ojo es la lámpara del cuerpo}: Mt 6:22-23; Lc 11:34-36.}

\par
%\textsuperscript{(1577.7)}
\textsuperscript{140:6.13} Entonces Tomás preguntó a Jesús si debían «continuar teniéndolo todo en común». El Maestro contestó: «Sí, hermanos míos, quisiera que viviéramos juntos como una familia comprensiva. Una gran obra se os ha confiado, y deseo vuestro servicio indiviso. Sabéis que se ha dicho muy bien: `Nadie puede servir a dos señores a la vez'. No podéis adorar sinceramente a Dios, y al mismo tiempo servir al Dinero de todo corazón. Ahora que os habéis enrolado sin reservas en el trabajo del reino, no os inquietéis por vuestra vida, y preocupaos mucho menos por lo que vais a comer o beber, o con qué vestiréis vuestro cuerpo. Ya habéis aprendido que unas manos serviciales y unos corazones diligentes no pasan hambre. Ahora que os estáis preparando para consagrar todas vuestras energías al trabajo del reino, estad seguros de que el Padre no se olvidará de vuestras necesidades. Buscad primero el reino de Dios, y cuando hayáis encontrado la entrada, todas las cosas necesarias las recibiréis por añadidura. Por eso, no os preocupéis indebidamente por el mañana. A cada día le basta su propio afán»\footnote{\textit{Ningún hombre puede servir a dos señores}: Mt 6:24; Lc 16:13. \textit{No os angustiéis por el mañana}: Mt 6:25,31-32; Lc 12:22-23. \textit{Buscad primero el reino}: Mt 6:33-34.}.

\par
%\textsuperscript{(1578.1)}
\textsuperscript{140:6.14} Cuando vio que estaban dispuestos a permanecer levantados toda la noche para hacerle preguntas, Jesús les dijo: «Hermanos míos, sois vasijas de barro; es mejor que vayáis a descansar con el fin de estar dispuestos para el trabajo de mañana». Pero el sueño se había alejado de sus párpados. Pedro se atrevió a pedir a su Maestro «sólo una breve conversación privada contigo. No es que yo tenga secretos para mis hermanos, pero estoy confundido y, si acaso mereciera una reprimenda de mi Maestro, podría soportarla mejor a solas contigo». Jesús le dijo: «Ven conmigo, Pedro» ---mostrando el camino hacia la casa. Cuando Pedro regresó de encontrarse con su Maestro, muy alentado y bastante estimulado, Santiago decidió entrar para hablar con Jesús. Y así sucesivamente, hasta las primeras horas de la mañana, los demás apóstoles entraron de uno en uno para hablar con el Maestro. Cuando todos hubieron conversado personalmente con él, salvo los gemelos, que se habían dormido, Andrés entró a ver a Jesús y le dijo: «Maestro, los gemelos se han dormido cerca del fuego en el jardín; ¿debo despertarlos para preguntarles si quieren hablar también contigo?» Y Jesús le dijo a Andrés, sonriendo: «Hacen bien ---no los molestes». La noche ya había pasado y despuntaba la luz de un nuevo día.

\section*{7. La semana después de la ordenación}
\par
%\textsuperscript{(1578.2)}
\textsuperscript{140:7.1} Después de unas horas de sueño, cuando los doce estaban reunidos tomando un desayuno tardío con Jesús, éste les dijo: «Ahora debéis empezar vuestro trabajo de predicación de la buena nueva y de instrucción de los creyentes. Preparaos para ir a Jerusalén». Después de que Jesús hubiera hablado, Tomás reunió el valor suficiente para decir: «Ya sé, Maestro, que deberíamos estar preparados para emprender el trabajo, pero me temo que aún no somos capaces de llevar a cabo esta gran empresa. ¿Permitirías que nos quedáramos por aquí cerca unos días más, antes de empezar el trabajo del reino?» Cuando vio que todos sus apóstoles estaban dominados por el mismo temor, Jesús dijo: «Será como habéis pedido; permaneceremos aquí hasta después del sábado».

\par
%\textsuperscript{(1578.3)}
\textsuperscript{140:7.2} Durante semanas y semanas, pequeños grupos de activos buscadores de la verdad, así como espectadores curiosos, habían venido a Betsaida para ver a Jesús. Las noticias sobre él ya se habían difundido más allá de la región; habían venido grupos de investigadores desde ciudades tan lejanas como Tiro, Sidón, Damasco, Cesarea y Jerusalén. Hasta ese momento, Jesús había acogido a esta gente y los había instruido sobre el reino, pero el Maestro traspasó ahora esta tarea a los doce. Andrés escogía a uno de los apóstoles y le asignaba un grupo de visitantes; a veces, los doce estaban todos ocupados con esta misión.

\par
%\textsuperscript{(1578.4)}
\textsuperscript{140:7.3} Trabajaron durante dos días, enseñando de día y manteniendo sus conversaciones privadas hasta horas avanzadas de la noche. Al tercer día, Jesús se fue a charlar con Zebedeo y Salomé, después de despedir a sus apóstoles diciendo: «Id a pescar, tratad de hacer algo distinto sin preocupaciones, o visitad quizás a vuestras familias». El jueves regresaron para tres días más de enseñanza.

\par
%\textsuperscript{(1578.5)}
\textsuperscript{140:7.4} Durante esta semana de repaso, Jesús repitió muchas veces a sus apóstoles los dos grandes motivos de su misión en la Tierra después de su bautismo:

\par
%\textsuperscript{(1578.6)}
\textsuperscript{140:7.5} 1. Revelar el Padre a los hombres.

\par
%\textsuperscript{(1578.7)}
\textsuperscript{140:7.6} 2. Conducir a los hombres a hacerse conscientes de su filiación ---a comprender por la fe que son los hijos del Altísimo.

\par
%\textsuperscript{(1579.1)}
\textsuperscript{140:7.7} Una semana así de experiencias variadas hizo mucho bien a los doce; algunos incluso empezaron a tener demasiada confianza en sí mismos. En la última conferencia, la noche después del sábado, Pedro y Santiago se acercaron a Jesús, diciendo: «Estamos preparados; salgamos ahora para conquistar el reino». A lo cual Jesús replicó: «Que vuestra sabiduría iguale a vuestro entusiasmo y vuestra valentía compense vuestra ignorancia».

\par
%\textsuperscript{(1579.2)}
\textsuperscript{140:7.8} Aunque los apóstoles no lograban comprender muchas de sus enseñanzas, no dejaban de captar el significado de la vida maravillosamente hermosa que vivía con ellos.

\section*{8. El jueves por la tarde, en el lago}
\par
%\textsuperscript{(1579.3)}
\textsuperscript{140:8.1} Jesús sabía muy bien que sus apóstoles no asimilaban plenamente sus enseñanzas. Decidió impartir una instrucción especial a Pedro, Santiago y Juan, con la esperanza de que fueran capaces de clarificar las ideas de sus compañeros. Veía que los doce captaban algunas características de la idea de un reino espiritual, pero persistían con obstinación en relacionar directamente estas nuevas enseñanzas espirituales con sus antiguos conceptos literales y arraigados del reino de los cielos como restauración del trono de David y restablecimiento de Israel como potencia temporal en la Tierra. En consecuencia, el jueves por la tarde, Jesús se alejó de la costa en una barca con Pedro, Santiago y Juan, para hablarles de los asuntos del reino. Fue una conversación educativa de cuatro horas que abarcó decenas de preguntas y respuestas, y se puede incluir de manera muy provechosa en este relato, reorganizando el resumen de esta tarde importante que Simón Pedro ofreció a su hermano Andrés a la mañana siguiente:

\par
%\textsuperscript{(1579.4)}
\textsuperscript{140:8.2} 1. \textit{Hacer la voluntad del Padre}. La enseñanza de Jesús sobre confiar en los cuidados del Padre celestial no era un fatalismo ciego y pasivo. Aquella tarde citó, dándolo por bueno, un viejo refrán hebreo: «El que no trabaje no comerá»\footnote{\textit{El que no trabaje no comerá}: 2 Ts 3:10.}. Señaló su propia experiencia como comentario suficiente sobre sus enseñanzas. Sus preceptos sobre la confianza en el Padre no deben juzgarse según las condiciones sociales o económicas de los tiempos modernos o de cualquier otra época. Su enseñanza abarca los principios ideales de una vida cercana a Dios, en todas las épocas y en todos los mundos.

\par
%\textsuperscript{(1579.5)}
\textsuperscript{140:8.3} Jesús aclaró a los tres la diferencia que había entre las exigencias de ser apóstol y las de ser discípulo. Incluso entonces no prohibió a los doce que ejercitaran la prudencia y la previsión. Él no predicaba contra la previsión, sino contra la ansiedad y la preocupación. Enseñaba la sumisión activa y alerta a la voluntad de Dios. En respuesta a las numerosas preguntas sobre la frugalidad y el ahorro, simplemente llamó la atención sobre su vida de carpintero, de fabricante de barcas y de pescador, y sobre su cuidadosa organización de los doce. Trató de aclararles que el mundo no debe ser considerado como un enemigo; que las circunstancias de la vida constituyen un designio divino que trabaja con los hijos de Dios.

\par
%\textsuperscript{(1579.6)}
\textsuperscript{140:8.4} Jesús tuvo grandes dificultades para hacerles comprender su práctica personal de la no resistencia. Se negaba absolutamente a defenderse, y a los apóstoles les pareció que le hubiera gustado que ellos hubieran seguido la misma política. Les enseñó que no se opusieran al mal, que no combatieran las injusticias o las injurias, pero no les enseñó que toleraran pasivamente la maldad. Aquella tarde dejó muy claro que aprobaba el castigo social para los malhechores y los criminales, y que a veces el gobierno civil tiene que emplear la fuerza para mantener el orden social y aplicar la justicia.

\par
%\textsuperscript{(1579.7)}
\textsuperscript{140:8.5} Nunca dejó de prevenir a sus discípulos contra la práctica perniciosa de las \textit{represalias}; no soportaba la revancha, la idea de desquitarse. Deploraba que se guardara rencor. Rechazaba la idea del ojo por ojo y diente por diente\footnote{\textit{Ojo por ojo y diente por diente}: Ex 21:24; Lv 24:20; Dt 19:21; Mt 5:38.}. Desaprobaba todo el concepto de la revancha privada y personal\footnote{\textit{Advertencia contra la venganza}: Pr 20:22; Dt 32:35; Ro 12:19.}, dejando estas cuestiones al gobierno civil, por un lado, y al juicio de Dios, por otro. Aclaró a los tres que sus enseñanzas se aplicaban al \textit{individuo}, no al Estado. Las instrucciones que había dado hasta ese momento sobre estas cuestiones las resumió como sigue:

\par
%\textsuperscript{(1580.1)}
\textsuperscript{140:8.6} Amad a vuestros enemigos\footnote{\textit{Amad a vuestros enemigos}: Mt 5:43-44; Lc 6:27,35.} ---recordad las demandas morales de la fraternidad humana.

\par
%\textsuperscript{(1580.2)}
\textsuperscript{140:8.7} La futilidad del mal: un agravio no se repara con la venganza\footnote{\textit{Futilidad de devolver mal con mal}: Mt 5:39-42; Ro 12:17-21; 1 Ts 5:15; 1 P 3:9-12.}. No cometáis el error de combatir el mal con sus propias armas.

\par
%\textsuperscript{(1580.3)}
\textsuperscript{140:8.8} Tened fe\footnote{\textit{Tened fe}: Mt 6:25-34; Lc 12:22-32.} ---tened confianza en el triunfo final de la justicia divina y de la bondad eterna.

\par
%\textsuperscript{(1580.4)}
\textsuperscript{140:8.9} 2. \textit{Actitud política}. Advirtió a sus apóstoles que fueran discretos en sus comentarios sobre las tensas relaciones que existían entonces entre el pueblo judío y el gobierno romano; les prohibió que se enredaran de alguna manera en estas dificultades. Siempre tenía el cuidado de evitar las trampas políticas de sus enemigos, respondiendo siempre: «Dad al César lo que es del César, y a Dios lo que es de Dios»\footnote{\textit{Dad al César lo que es del César y a Dios lo que es de Dios}: Mt 22:21; Mc 12:17; Lc 20:25.}. Se negaba a desviar su atención de su misión, que era la de establecer un nuevo camino de salvación; no se permitía a sí mismo preocuparse por otra cosa. En su vida personal, siempre acataba debidamente todas las leyes y reglas civiles; en todas sus enseñanzas públicas, hacía caso omiso de las cuestiones cívicas, sociales y económicas. Dijo a los tres apóstoles que sólo se preocupaba por los principios de la vida espiritual interior y personal del hombre.

\par
%\textsuperscript{(1580.5)}
\textsuperscript{140:8.10} Jesús no era pues un reformador político. No venía para reorganizar el mundo; aunque lo hubiera hecho, sólo hubiera sido aplicable a aquella época y a aquella generación. Sin embargo, mostró al hombre la mejor manera de vivir, y ninguna generación está exenta de la tarea de descubrir la mejor manera de adaptar la vida de Jesús a sus propios problemas. Pero no cometáis nunca el error de identificar las enseñanzas de Jesús con alguna teoría política o económica, con algún sistema social o industrial.

\par
%\textsuperscript{(1580.6)}
\textsuperscript{140:8.11} 3. \textit{Actitud social}. Durante mucho tiempo, los rabinos judíos habían debatido la cuestión: ¿Quién es mi prójimo?\footnote{\textit{¿Quién es mi prójimo?}: Lc 10:29-37.} Jesús vino a presentar la idea de una bondad activa y espontánea, de un amor tan sincero por los semejantes, que ampliaba el concepto de vecindad hasta incluir al mundo entero, convirtiendo así en prójimos a todos los hombres. Pero a pesar de todo esto, Jesús se interesaba únicamente por el individuo, no por la masa. Jesús no era un sociólogo, pero trabajó para destruir todas las formas de aislamiento egoísta. Enseñaba la simpatía pura, la compasión. Miguel de Nebadon es un Hijo dominado por la misericordia; la compasión es su verdadera naturaleza.

\par
%\textsuperscript{(1580.7)}
\textsuperscript{140:8.12} El Maestro no dijo que los hombres nunca debían convidar a comer a sus amigos, pero sí dijo que sus discípulos deberían organizar festines para los pobres y los desgraciados\footnote{\textit{Convidar a los pobres}: Lc 14:12-14.}. Jesús tenía un sólido sentido de la justicia, pero siempre estaba templada por la misericordia. No enseñó a sus apóstoles que se dejaran engañar por los parásitos sociales o los mendigos profesionales. El momento en que estuvo más cerca de efectuar unas declaraciones sociológicas fue cuando dijo: «No juzguéis, para no ser juzgados»\footnote{\textit{No juzguéis, para no ser juzgados}: Mt 7:1; Lc 6:37.}.

\par
%\textsuperscript{(1580.8)}
\textsuperscript{140:8.13} Indicó claramente que la beneficencia sin distinción puede ser acusada de muchos males sociales. Al día siguiente, Jesús ordenó definitivamente a Judas que no se entregara ningún fondo apostólico como limosna, a menos que él lo pidiera o que dos de los apóstoles lo solicitaran conjuntamente. En todas estas cuestiones, Jesús siempre tenía la costumbre de decir: «Sed tan prudentes como las serpientes, pero tan inofensivos como las palomas»\footnote{\textit{Sed sabios como serpientes}: Mt 10:16b.}. En todas las situaciones sociales, parecía tener el propósito de enseñar la paciencia, la tolerancia y el perdón.

\par
%\textsuperscript{(1581.1)}
\textsuperscript{140:8.14} Para Jesús, la familia ocupaba el centro mismo de la filosofía de la vida ---aquí y en el más allá. Sus enseñanzas sobre Dios las basó en la familia, tratando al mismo tiempo de corregir la tendencia de los judíos a honrar excesivamente a sus antepasados. Alabó la vida familiar como el deber humano más alto, pero indicó claramente que las relaciones familiares no deben interferir con las obligaciones religiosas. Llamó la atención sobre el hecho de que la familia es una institución temporal que no sobrevive a la muerte. Jesús no dudó en abandonar a su familia cuando ésta se opuso a la voluntad del Padre. Enseñó la nueva y más amplia fraternidad de los hombres ---los hijos de Dios. En la época de Jesús, las costumbres relacionadas con el divorcio eran relajadas en Palestina y en todo el imperio romano. Se negó repetidas veces a establecer leyes sobre el matrimonio y el divorcio, pero muchos de los primeros seguidores de Jesús tenían opiniones arraigadas sobre el divorcio, y no dudaron en atribuírselas a él. Todos los escritores del Nuevo Testamento, exceptuando a Juan Marcos, se adhirieron a estas ideas más estrictas y avanzadas sobre el divorcio.

\par
%\textsuperscript{(1581.2)}
\textsuperscript{140:8.15} 4. \textit{Actitud económica}. Jesús trabajó, vivió y comerció en el mundo tal como lo encontró. No era un reformador económico, aunque llamó frecuentemente la atención sobre la injusticia de la distribución desigual de la riqueza; pero no ofreció ninguna sugerencia como remedio. Indicó claramente a los tres que, aunque sus apóstoles no debían poseer bienes, no predicaba contra la riqueza y la propiedad, sino únicamente contra su distribución desigual e injusta. Reconocía la necesidad de la justicia social y de la equidad industrial, pero no ofreció ninguna regla para conseguirlas.

\par
%\textsuperscript{(1581.3)}
\textsuperscript{140:8.16} Nunca enseñó a sus discípulos que evitaran las posesiones terrenales; sólo a sus doce apóstoles. Lucas, el médico, creía firmemente en la igualdad social, y contribuyó mucho a interpretar las palabras de Jesús en consonancia con sus creencias personales. Jesús nunca ordenó personalmente a sus seguidores que adoptaran un modo de vida comunitario; no hizo ninguna declaración de ningún tipo sobre estas cuestiones.

\par
%\textsuperscript{(1581.4)}
\textsuperscript{140:8.17} Jesús previno con frecuencia a sus oyentes contra la codicia, declarando que «la felicidad de un hombre no consiste en la abundancia de sus posesiones materiales»\footnote{\textit{La felicidad no consiste en posesiones}: Lc 12:15.}. Reiteraba constantemente: «¿De qué le sirve a un hombre ganar el mundo entero, si pierde su propia alma?»\footnote{\textit{¿De qué le sirve al hombre ganar el mundo entero?}: Mt 16:26; Mc 8:36; Lc 9:25.} No lanzó ataques directos contra la posesión de bienes, pero sí insistió en que es eternamente esencial el dar la prioridad a los valores espirituales. En sus enseñanzas posteriores trató de corregir muchas opiniones erróneas urantianas sobre la vida, contando numerosas parábolas que dio a conocer en el transcurso de su ministerio público. Jesús nunca tuvo la intención de formular teorías económicas; sabía muy bien que cada época debe desarrollar sus propios remedios para los problemas existentes. Si Jesús estuviera hoy en la Tierra, viviendo su vida en la carne, desilusionaría mucho a la mayoría de los hombres y mujeres de bien, por la sencilla razón de que no tomaría partido en los debates políticos, sociales o económicos del día. Permanecería sublimemente al margen, mientras que os enseñaría a perfeccionar vuestra vida espiritual interior, con el fin de haceros mucho más competentes para atacar la solución de vuestros problemas puramente humanos.

\par
%\textsuperscript{(1581.5)}
\textsuperscript{140:8.18} Jesús quería hacer a todos los hombres semejantes a Dios, y luego permanecer cerca con simpatía mientras estos hijos de Dios resuelven sus propios problemas políticos, sociales y económicos. No era la riqueza lo que denunciaba, sino lo que hace la riqueza a la mayoría de sus adictos. Este jueves por la tarde, Jesús dijo por primera vez a sus discípulos que «es más bienaventurado dar que recibir»\footnote{\textit{Es más bienaventurado dar que recibir}: Hch 20:35.}.

\par
%\textsuperscript{(1581.6)}
\textsuperscript{140:8.19} 5. \textit{Religión personal}. Vosotros, al igual que hicieron sus apóstoles, deberíais comprender mejor las enseñanzas de Jesús a través de su vida. Vivió una vida perfeccionada en Urantia, y sus enseñanzas excepcionales sólo se pueden comprender cuando se visualiza esa vida en su trasfondo inmediato. Es su vida, y no sus lecciones a los doce o sus sermones a las multitudes, lo que ayudará mejor a revelar el carácter divino y la personalidad amorosa del Padre.

\par
%\textsuperscript{(1582.1)}
\textsuperscript{140:8.20} Jesús no atacó las enseñanzas de los profetas hebreos o de los moralistas griegos. El Maestro reconocía las numerosas cosas buenas que defendían estos grandes pensadores, pero había venido a la Tierra para enseñar algo \textit{adicional:} «la conformidad voluntaria de la voluntad del hombre a la voluntad de Dios»\footnote{\textit{Haced voluntariamente la «voluntad» de Dios}: Sal 143:10; Eclo 15:11-20; Mt 6:10; 7:21; 12:50; Mc 3:35; Lc 8:21; 11:2; Jn 7:16-17; 9:31; 14:21-24; 15:10,14-16.}. Jesús no quería limitarse a producir \textit{hombres religiosos}, unos mortales enteramente ocupados en sentimientos religiosos y animados exclusivamente por impulsos espirituales. Si hubierais podido echar una sola mirada sobre él, hubierais sabido que Jesús era realmente un hombre de gran experiencia en las cosas de este mundo. Las enseñanzas de Jesús en este sentido han sido groseramente falseadas y muy mal presentadas a lo largo de todos los siglos de la era cristiana; también habéis tenido ideas tergiversadas sobre la mansedumbre y la humildad del Maestro. La meta que perseguía en su vida parece haber sido un \textit{magnífico respeto de sí mismo}. Sólo aconsejaba a los hombres que se humillaran para que pudieran ser verdaderamente ensalzados; lo que en realidad perseguía era una humildad auténtica ante Dios. Atribuía un gran valor a la sinceridad ---al corazón puro. La fidelidad era una virtud cardinal en su evaluación del carácter, mientras que la \textit{valentía} era el centro mismo de sus enseñanzas. Su consigna era «No temáis»\footnote{\textit{Jesús decía «No temáis»}: Mt 10:28,31; 14:27; 17:7; 28:10; Mc 5:36; 6:50; Lc 5:10; 8:50; 12:4,4,7,32; Jn 6:20; 14:27.}, y el aguante paciente era su ideal de la fuerza de carácter. Las enseñanzas de Jesús constituyen una religión de valor, de coraje y de heroísmo. Y precisamente por eso escogió, como representantes personales suyos, a doce hombres corrientes que eran en su mayoría pescadores toscos, viriles y valerosos.

\par
%\textsuperscript{(1582.2)}
\textsuperscript{140:8.21} Jesús tenía poco que decir sobre los vicios sociales de su época; rara vez se refirió a la delincuencia moral. Era un educador positivo de la verdadera virtud. Evitó cuidadosamente el método negativo de impartir la enseñanza; rehusó darle publicidad al mal. No era siquiera un reformador moral. Sabía muy bien, y así lo enseñó a sus apóstoles, que los impulsos sensuales de la humanidad no se suprimen con los reproches religiosos ni con las prohibiciones legales. Sus pocas denuncias estaban dirigidas sobre todo contra el orgullo, la crueldad, la opresión y la hipocresía.

\par
%\textsuperscript{(1582.3)}
\textsuperscript{140:8.22} Jesús no denunció con vehemencia ni siquiera a los fariseos, como había hecho Juan. Sabía que muchos escribas y fariseos tenían un corazón honesto; comprendía que eran esclavos serviles de las tradiciones religiosas. Jesús insistía mucho en «empezar por sanar el árbol»\footnote{\textit{Empezad sanando el árbol}: Mt 12:33.}. Fijó en el ánimo de los tres que valoraba la vida en su totalidad, y no sólo algunas virtudes particulares.

\par
%\textsuperscript{(1582.4)}
\textsuperscript{140:8.23} La única lección que Juan aprendió de la enseñanza de este día fue que el fondo de la religión de Jesús consistía en adquirir un carácter compasivo, acoplado con una personalidad motivada por hacer la voluntad del Padre que está en los cielos.

\par
%\textsuperscript{(1582.5)}
\textsuperscript{140:8.24} Pedro captó la idea de que el evangelio que estaban a punto de proclamar era realmente un nuevo punto de partida para toda la raza humana. Más tarde transmitió esta impresión a Pablo, que la utilizó para formular su doctrina de Cristo como «el segundo Adán»\footnote{\textit{Cristo como segundo Adán}: 1 Co 15:45-49.}.

\par
%\textsuperscript{(1582.6)}
\textsuperscript{140:8.25} Santiago comprendió la emocionante verdad de que Jesús deseaba que sus hijos de la Tierra vivieran como si ya fueran ciudadanos del reino celestial acabado.

\par
%\textsuperscript{(1582.7)}
\textsuperscript{140:8.26} Jesús sabía que los hombres son diferentes, y así lo enseñó a sus apóstoles. Los exhortaba constantemente a que se abstuvieran de intentar moldear a los discípulos y a los creyentes según un modelo predeterminado. Intentaba dejar que cada alma se desarrollara según su propia manera, como un individuo distinto que se perfecciona ante Dios. En respuesta a una de las numerosas preguntas de Pedro, el Maestro dijo: «Quiero liberar a los hombres para que puedan empezar de nuevo como niños pequeños en una vida nueva y mejor». Jesús insistía siempre en que la verdadera bondad debe ser inconsciente\footnote{\textit{La verdadera bondad es inconsciente}: Mt 6:1-3.}, que al hacer caridad no hay que dejar que la mano izquierda se entere de lo que hace la derecha.

\par
%\textsuperscript{(1583.1)}
\textsuperscript{140:8.27} Aquella tarde, los tres apóstoles se escandalizaron cuando se dieron cuenta de que la religión de su Maestro no preveía el examen espiritual de sí mismo. Todas las religiones anteriores y posteriores a los tiempos de Jesús, incluido el cristianismo, prevén cuidadosamente un examen concienzudo de sí mismo. Pero no es así con la religión de Jesús de Nazaret; su filosofía de la vida carece de introspección religiosa. El hijo del carpintero nunca enseñó la \textit{formación} del carácter; enseñó el \textit{crecimiento} del carácter\footnote{\textit{Crecimiento del carácter}: Mt 13:31-32; Mc 4:31-32; Lc 13:18-19.}, declarando que el reino de los cielos se parece a un grano de mostaza. Pero Jesús no dijo nada que proscribiera el análisis de sí mismo como medio de prevenir el egotismo presuntuoso.

\par
%\textsuperscript{(1583.2)}
\textsuperscript{140:8.28} El derecho a entrar en el reino está condicionado por la fe, por la creencia personal. Lo que hay que pagar para permanecer en la ascensión progresiva del reino es la perla de gran precio\footnote{\textit{La perla de gran precio}: Mt 13:45-46.}; para poseerla, el hombre vende todo lo que tiene.

\par
%\textsuperscript{(1583.3)}
\textsuperscript{140:8.29} La enseñanza de Jesús es una religión para todos, no solamente para los débiles y los esclavos. Su religión nunca se cristalizó (en su época) en credos y en leyes teológicas; no dejó ni una línea escrita detrás de él. Su vida y sus enseñanzas fueron legadas al universo como una herencia inspiradora e ideal, adecuada para la orientación espiritual y la instrucción moral de todas las épocas en todos los mundos. Incluso hoy en día, las enseñanzas de Jesús se mantienen apartadas de todas las religiones, como tales, aunque son la esperanza viviente de cada una de ellas.

\par
%\textsuperscript{(1583.4)}
\textsuperscript{140:8.30} Jesús no enseñó a sus apóstoles que la religión es la única ocupación del hombre en la Tierra; ésta era la idea que tenían los judíos del servicio de Dios. Pero sí insistió en que la religión sería la ocupación exclusiva de los doce. Jesús no enseñó nada que desviara a sus creyentes de la búsqueda de una cultura auténtica; sólo le quitó mérito a las escuelas religiosas de Jerusalén, atadas a la tradición. Era liberal, generoso, culto y tolerante. La piedad retraída no ocupaba ningún lugar en su filosofía de la manera recta de vivir.

\par
%\textsuperscript{(1583.5)}
\textsuperscript{140:8.31} El Maestro no ofreció soluciones para los problemas no religiosos de su propia época ni de ninguna época posterior. Jesús deseaba desarrollar la comprensión espiritual de las realidades eternas y estimular la iniciativa en la originalidad de la manera de vivir; se ocupó exclusivamente de las necesidades espirituales subyacentes y permanentes de la raza humana. Reveló una bondad igual a la de Dios. Exaltó el amor ---la verdad, la belleza y la bondad--- como el ideal divino y la realidad eterna.

\par
%\textsuperscript{(1583.6)}
\textsuperscript{140:8.32} El Maestro vino para crear un nuevo espíritu en el hombre, una nueva voluntad ---para conferirle una capacidad nueva para conocer la verdad, experimentar la compasión y escoger la bondad--- la voluntad de estar en armonía con la voluntad de Dios, unida al impulso eterno de volverse perfecto\footnote{\textit{Sed perfectos}: Gn 17:1; 1 Re 8:61; Lv 19:2; Dt 18:13; Mt 5:48; 2 Co 13:11; Stg 1:4; 1 P 1:16.} como el Padre que está en los cielos es perfecto.

\section*{9. El día de la consagración}
\par
%\textsuperscript{(1583.7)}
\textsuperscript{140:9.1} Jesús dedicó el sábado siguiente a sus apóstoles, regresando a las tierras altas donde los había ordenado. Allí, después de un largo mensaje personal de estímulo, hermosamente conmovedor, emprendió el acto solemne de la consagración de los doce. Aquel sábado por la tarde, Jesús reunió a los apóstoles a su alrededor, en la ladera de la colina, y los puso en manos de su Padre celestial como preparación para el día en que se vería obligado a dejarlos solos en el mundo. No hubo ninguna enseñanza nueva en esta ocasión, sólo conversación y comunión.

\par
%\textsuperscript{(1584.1)}
\textsuperscript{140:9.2} Jesús analizó muchos aspectos del sermón de ordenación, pronunciado en este mismo lugar; luego los llamó ante él, uno a uno, y les encargó que salieran al mundo como sus representantes. La misión de consagración\footnote{\textit{Consagración y comisión}: Mt 10:1; Mc 3:13-14; Lc 9:1-2.} del Maestro fue: «Id por todo el mundo y predicad la buena nueva del reino. Liberad a los cautivos espirituales, confortad a los oprimidos y ayudad a los afligidos. Habéis recibido gratuitamente, dad gratuitamente»\footnote{\textit{Encargo de predicar por el mundo}: Mt 24:14; 28:19-20a; Mc 13:10; 16:15; Lc 24:47; Jn 17:18; Hch 1:8b. \textit{Liberad a los cautivos espirituales}: Is 61:1; Lc 4:18. \textit{Confortad a los oprimidos}: Is 40:1; 2 Co 1:3-4. \textit{Ayudad a los afligios}: Mt 20:26; Mc 10:43. \textit{Habéis recibido gratuitamente, dad gratuitamente}: Mt 10:8.}.

\par
%\textsuperscript{(1584.2)}
\textsuperscript{140:9.3} Jesús les aconsejó que no llevaran dinero ni ropa adicional\footnote{\textit{No llevéis dinero ni ropa adicional}: Mt 10:9-10; Mc 6:8-9; Lc 9:3; Lc 10:4.}, diciendo: «El obrero merece su salario»\footnote{\textit{El obrero merece su salario}: Lc 10:7.}. Y finalmente dijo: «Mirad, os envío como ovejas en medio de los lobos; sed pues tan prudentes como las serpientes y tan inofensivos como las palomas. Pero prestad atención, porque vuestros enemigos os llevarán ante sus consejos, y os criticarán severamente en sus sinagogas. Seréis llevados ante los gobernadores y los jefes porque creéis en este evangelio, y vuestro testimonio mismo será mi propio testimonio ante ellos. Cuando os lleven a juicio, no os inquietéis por lo que tendréis que decir, porque el espíritu de mi Padre vive en vosotros y en esos momentos hablará por vosotros. Algunos de vosotros seréis ejecutados, y antes de que establezcáis el reino en la Tierra, seréis odiados por muchos pueblos a causa de este evangelio; pero no temáis, yo estaré con vosotros y mi espíritu os precederá en el mundo entero. La presencia de mi Padre permanecerá en vosotros mientras que os dirigís primero hacia los judíos y luego hacia los gentiles»\footnote{\textit{Os envío como ovejas entre lobos}: Lc 10:3. \textit{Sed sabios como serpientes}: Mt 10:16. \textit{Os llevarán ante los consejos}: Lc 21:12. \textit{Os criticarán en las sinagogas}: Mt 10:17-21; Mc 13:9. \textit{Seréis vilipendiados y ejecutados}: Mt 24:9. \textit{Testimonio del espíritu interior}: Mt 10:19-20; Mc 13:11; Lc 12:11-12; Lc 21:13-15. \textit{Primero a los judíos, luego a los gentiles}: Ro 1:16.}.

\par
%\textsuperscript{(1584.3)}
\textsuperscript{140:9.4} Cuando descendieron de la montaña, regresaron a su hogar en la casa de Zebedeo.

\section*{10. La noche después de la consagración}
\par
%\textsuperscript{(1584.4)}
\textsuperscript{140:10.1} Aquella noche, Jesús enseñó dentro de la casa porque había empezado a llover; habló muy extensamente a los doce para tratar de mostrarles lo que debían \textit{ser}, y no lo que debían \textit{hacer}. Sólo conocían una religión que imponía \textit{hacer} ciertas cosas para poder alcanzar la rectitud ---la salvación. Pero Jesús les repetía: «En el reino, tenéis que \textit{ser} rectos para hacer el trabajo». Muchas veces reiteró: «Sed» pues perfectos, como vuestro Padre que está en los cielos es perfecto»\footnote{\textit{Sed perfectos}: Gn 17:1; 1 Re 8:61; Lv 19:2; Dt 18:13; Mt 5:48; 2 Co 13:11; Stg 1:4; 1 P 1:16.}. El Maestro explicaba todo el tiempo a sus apóstoles aturdidos que la salvación que había venido a traer al mundo sólo se podía obtener \textit{creyendo}, con una fe simple y sincera. Jesús dijo: «Juan ha predicado un bautismo de arrepentimiento, de aflicción por la vieja manera de vivir. Vosotros vais a proclamar el bautismo de la comunión con Dios. Predicad el arrepentimiento a los que necesitan esa enseñanza, pero a los que ya buscan entrar sinceramente en el reino, abridles las puertas de par en par y pedidles que entren en la jubilosa hermandad de los hijos de Dios»\footnote{\textit{Juan os dió un bautismo de arrepentimiento}: Mt 3:2; Lc 3:3; Hch 13:24. \textit{La hermandad de los hijos de Dios}: Mt 12:50; Mc 3:35; Lc 8:21; Hch 2:42; 1 Co 1:9; Ef 3:9.}. Pero era una tarea difícil la de persuadir a estos pescadores galileos de que, en el reino, primero hay que \textit{ser} recto por la fe, antes de \textit{obrar} con rectitud en la vida cotidiana de los mortales de la Tierra.

\par
%\textsuperscript{(1584.5)}
\textsuperscript{140:10.2} Otro gran obstáculo en este trabajo de enseñar a los doce era su tendencia a aceptar los principios altamente idealistas y espirituales de la verdad religiosa, y transformarlos en reglas concretas de conducta personal. Jesús les presentaba el hermoso espíritu de la actitud del alma, pero ellos insistían en traducir estas enseñanzas a reglas de comportamiento personal. Muchas veces, cuando estaban seguros de recordar lo que el Maestro había dicho, casi no podían dejar de olvidar lo que \textit{no} había dicho. Pero asimilaron lentamente su enseñanza, porque Jesús \textit{era} todo lo que enseñaba. Lo que no pudieron obtener con sus instrucciones verbales, lo adquirieron paulatinamente viviendo con él.

\par
%\textsuperscript{(1585.1)}
\textsuperscript{140:10.3} Los apóstoles no percibían que su Maestro estaba ocupado en vivir una vida de inspiración espiritual para todas las personas de todas las épocas en todos los mundos de un vasto universo. A pesar de lo que Jesús les decía de vez en cuando, los apóstoles no captaban la idea de que estaba efectuando una labor \textit{en} este mundo, pero \textit{para} todos los otros mundos de su inmensa creación. Jesús vivió su vida terrestre en Urantia, no para establecer un ejemplo personal de vida mortal para los hombres y mujeres de este mundo, sino más bien para crear \textit{un ideal altamente espiritual e inspirador} para todos los seres mortales de todos los mundos.

\par
%\textsuperscript{(1585.2)}
\textsuperscript{140:10.4} Esta misma noche, Tomás le preguntó a Jesús: «Maestro, tú dices que debemos volvernos como niños pequeños antes de poder entrar en el reino del Padre, y sin embargo nos has advertido que no nos dejemos engañar por los falsos profetas, ni que nos hagamos culpables de arrojar nuestras perlas a los cerdos. Pues bien, estoy francamente desconcertado. No consigo comprender tu enseñanza»\footnote{\textit{Volverse como un niño}: Mt 18:2-4; Mt 19:13-14; Mc 9:36-37; Mc 10:13-15; Lc 9:47-48; Lc 18:16-17. \textit{Los falsos profetas}: Mt 7:15,22-23; Mt 24:11; Mc 13:22; 1 Jn 4:1. \textit{Arrojar nuestras perlas a los cerdos}: Mt 7:6.}. Jesús le contestó a Tomás: «¡Cuánto tiempo seré indulgente con vosotros! Siempre insistís en entender literalmente todo lo que enseño. Cuando os he pedido que os volváis como niños pequeños, como precio de entrada en el reino, no me refería a la facilidad de dejarse engañar, a la simple buena voluntad de creer, ni a la rapidez para confiar en los extraños agradables. Lo que deseaba que pudierais deducir con este ejemplo era la relación entre un niño y su padre. Tú eres el hijo, y es en el reino de \textit{tu} padre donde pretendes entrar. Entre todo niño normal y su padre existe ese afecto natural que asegura una relación comprensiva y amorosa, y que excluye para siempre toda tendencia al regateo para obtener el amor y la misericordia del Padre. Y el evangelio que vais a predicar tiene que ver con una salvación que se origina cuando se comprende, por la fe, esta misma relación eterna entre el niño y su padre».

\par
%\textsuperscript{(1585.3)}
\textsuperscript{140:10.5} La característica principal de la enseñanza de Jesús consistía en que la \textit{moralidad} de su filosofía se originaba en la relación personal del individuo con Dios ---la misma relación que entre el niño y su padre. Jesús hacía hincapié en el \textit{individuo}, y no en la raza o en la nación. Mientras cenaban, Jesús tuvo una conversación con Mateo en la que le explicó que la moralidad de un acto cualquiera está determinada por el móvil del individuo. La moralidad de Jesús era siempre positiva. La regla de oro\footnote{\textit{La regla de oro de Jesús, en modo positivo}: Mt 7:12; Lc 6:31.}, tal como Jesús la expuso de nuevo con más claridad, exige un contacto social activo; la antigua regla negativa\footnote{\textit{La antigua regla de oro, en modo negativo}: Tb 4:15.} podía ser obedecida en la soledad. Jesús despojó a la moralidad de todas las reglas y ceremonias, y la elevó a los niveles majestuosos del pensamiento espiritual y de la vida verdaderamente recta.

\par
%\textsuperscript{(1585.4)}
\textsuperscript{140:10.6} Esta nueva religión de Jesús no estaba desprovista de implicaciones prácticas, pero todo lo que se puede encontrar en su enseñanza con un valor práctico, en el aspecto político, social o económico, es la consecuencia natural de esta experiencia interior del alma, que manifiesta los frutos del espíritu en el ministerio diario espontáneo de una experiencia religiosa personal auténtica.

\par
%\textsuperscript{(1585.5)}
\textsuperscript{140:10.7} Después de que Jesús y Mateo terminaran de hablar, Simón Celotes preguntó: «Pero, Maestro, ¿\textit{todos} los hombres son hijos de Dios?» Y Jesús contestó: «Sí, Simón, todos los hombres son hijos de Dios, y ésa es la buena nueva que vais a proclamar». Pero los apóstoles no conseguían comprender esta doctrina; era una declaración nueva, extraña y sorprendente. A causa de su deseo de inculcar esta verdad a sus discípulos, Jesús les enseñó a tratar a todos los hombres como hermanos.

\par
%\textsuperscript{(1585.6)}
\textsuperscript{140:10.8} En respuesta a una pregunta de Andrés, el Maestro indicó claramente que la moralidad implícita en su enseñanza era inseparable de la religión implícita en su manera de vivir. Enseñaba la moralidad, no partiendo de la \textit{naturaleza} del hombre, sino partiendo de la \textit{relación} del hombre con Dios.

\par
%\textsuperscript{(1585.7)}
\textsuperscript{140:10.9} Juan le preguntó a Jesús: «Maestro, ¿qué es el reino de los cielos?» Y Jesús respondió: «El reino de los cielos consiste en estas tres cosas esenciales: primero, el reconocimiento del hecho de la soberanía de Dios; segundo, la creencia en la verdad de la filiación con Dios; y tercero, la fe en la eficacia del deseo supremo humano de hacer la voluntad de Dios ---de ser semejante a Dios. Y he aquí la buena nueva del evangelio: por medio de la fe, cada mortal puede poseer todas estas cosas esenciales para la salvación».

\par
%\textsuperscript{(1586.1)}
\textsuperscript{140:10.10} Ahora que la semana de espera había terminado, se prepararon para partir al día siguiente hacia Jerusalén.