\chapter{Documento 142. La pascua en Jerusalén}
\par 
%\textsuperscript{(1596.1)}
\textsuperscript{142:0.1} DURANTE el mes de abril, Jesús y los apóstoles trabajaron en Jerusalén, saliendo de la ciudad todas las tardes para pasar la noche en Betania. El mismo Jesús pasó una o dos noches por semana en Jerusalén en la casa de Flavio, un judío griego, donde muchos judíos eminentes venían en secreto para entrevistarse con él.

\par 
%\textsuperscript{(1596.2)}
\textsuperscript{142:0.2} El primer día en Jerusalén, Jesús visitó al antiguo sumo sacerdote Anás, su amigo de años atrás y pariente de Salomé, la esposa de Zebedeo. Anás había oído hablar de Jesús y de sus enseñanzas, y cuando Jesús llamó a la casa del sumo sacerdote, fue recibido con mucha reserva. Cuando Jesús percibió la frialdad de Anás, se despidió inmediatamente, diciéndole al marcharse: «El miedo es el principal tirano del hombre, y el orgullo, su mayor debilidad; ¿te entregarás tú mismo a la esclavitud de estos dos destructores de la alegría y de la libertad?» Pero Anás no respondió. El Maestro no lo volvió a ver hasta el momento en que Anás se sentó con su yerno para juzgar al Hijo del Hombre.

\section*{1. La enseñanza en el templo}
\par 
%\textsuperscript{(1596.3)}
\textsuperscript{142:1.1} Durante todo este mes, Jesús o uno de los apóstoles enseñaron diariamente en el templo. Cuando el gentío pascual era demasiado numeroso como para entrar en el templo y escuchar la enseñanza, los apóstoles dirigían muchos grupos educativos fuera de los recintos sagrados. Lo esencial de su mensaje era:

\par 
%\textsuperscript{(1596.4)}
\textsuperscript{142:1.2} 1. El reino de los cielos está cerca\footnote{\textit{El Reino de Dios está cerca}: Mt 3:2; 4:17; 10:7; Mc 1:15; Lc 10:9,11; 17:20-21; 21:31.}.

\par 
%\textsuperscript{(1596.5)}
\textsuperscript{142:1.3} 2. Podéis entrar en el reino de los cielos mediante vuestra fe en la paternidad de Dios, convirtiéndoos así en los hijos de Dios\footnote{\textit{Hijos de Dios por la fe}: 1 Cr 22:10; Sal 2:7; Is 56:5; Mt 5:9,16,45; Lc 20:36; Jn 1:12-13; 11:52; Hch 17:28-29; Ro 8:14-17,19,21; 9:26; 2 Co 6:18; Gl 3:26; 4:5-7; Ef 1:5; Flp 2:15; Heb 12:5-8; 1 Jn 3:1-2,10; 5:2; Ap 21:7; 2 Sam 7:14.}.

\par 
%\textsuperscript{(1596.6)}
\textsuperscript{142:1.4} 3. El amor es la regla de vida dentro del reino ---la suprema devoción a Dios mientras que amáis a vuestro prójimo como a vosotros mismos\footnote{\textit{Amar a Dios y al prójimo como a uno mismo}: Mt 5:43-44; 19:19; 22:36-39; Mc 12:28-33; Lc 10:25-27.}.

\par 
%\textsuperscript{(1596.7)}
\textsuperscript{142:1.5} 4. La ley del reino es la obediencia a la voluntad del Padre, la cual produce los frutos del espíritu en vuestra vida personal\footnote{\textit{Frutos del espíritu}: Gl 5:22-23; Ef 5:9.}.

\par 
%\textsuperscript{(1596.8)}
\textsuperscript{142:1.6} Las multitudes que vinieron a celebrar la Pascua escucharon esta enseñanza de Jesús, y centenares de ellos se regocijaron con la buena nueva\footnote{\textit{Las multitudes se regocijan con la buena nueva}: Jn 2:23.}. Los principales sacerdotes y dirigentes de los judíos empezaron a interesarse mucho por Jesús y sus apóstoles, y discutieron entre sí sobre lo que debían hacer con ellos.

\par 
%\textsuperscript{(1596.9)}
\textsuperscript{142:1.7} Además de enseñar dentro y fuera del templo, los apóstoles y otros creyentes se ocupaban de hacer mucho trabajo personal entre las multitudes de la Pascua. Estos hombres y mujeres interesados en el mensaje de Jesús llevaron las nuevas que escucharon durante esta celebración pascual hasta los lugares más alejados del imperio romano, y también a oriente. Éste fue el principio de la difusión del evangelio del reino en el mundo exterior. El trabajo de Jesús ya no iba a limitarse a Palestina.

\section*{2. La ira de Dios}
\par 
%\textsuperscript{(1597.1)}
\textsuperscript{142:2.1} Se encontraba en Jerusalén, asistiendo a las festividades de la Pascua, un rico negociante judío de Creta llamado Jacobo, que fue hasta Andrés para pedirle ver a Jesús en privado. Andrés arregló este encuentro secreto con Jesús en la casa de Flavio para el día siguiente al anochecer. Este hombre no podía comprender las enseñanzas del Maestro, y venía porque deseaba indagar más plenamente sobre el reino de Dios. Jacobo le dijo a Jesús: «Pero, Rabino, Moisés y los antiguos profetas nos dicen que Yahvé es un Dios celoso, un Dios con una gran ira y un intenso furor. Los profetas dicen que odia a los malhechores y que se venga de los que no obedecen su ley. Tú y tus discípulos nos enseñáis que Dios es un Padre benévolo y compasivo que ama tanto a todos los hombres que los acogería con agrado en este nuevo reino de los cielos que tú proclamas tan cercano»\footnote{\textit{Moisés dijo que Dios es celoso}: Ex 20:5; 34:14; Nah 1:2; Dt 6:15; 32:16,21; Jos 24:19. \textit{Un Dios de gran ira e intenso furor}: Ex 22:24; 2 Re 22:13; Lv 10:6; Nm 11:33; Dt 9:7-8,22. \textit{Que se venga de los pecadores}: Sal 58:10; Is 34:8; Dt 32:35,41,43; Jue 11:36.}.

\par 
%\textsuperscript{(1597.2)}
\textsuperscript{142:2.2} Cuando Jacobo terminó de hablar, Jesús contestó: «Jacobo, has expuesto muy bien las enseñanzas de los antiguos profetas, que instruyeron a los hijos de su generación de acuerdo con las luces de su tiempo. Nuestro Padre que está en el Paraíso es invariable. Pero el concepto de su naturaleza se ha ampliado y ha crecido desde la época de Moisés hasta los tiempos de Amós, e incluso hasta la generación del profeta Isaías. Ahora, yo he venido en forma carnal para revelar el Padre con una nueva gloria y dar a conocer su amor y su misericordia a todos los hombres de todos los mundos. A medida que el evangelio de este reino se divulgue por el mundo con su mensaje de felicidad y de buena voluntad para todos los hombres, nacerán unas relaciones mejores y superiores entre las familias de todas las naciones. A medida que pase el tiempo, los padres y sus hijos se amarán más los unos a los otros, y esto producirá una mayor comprensión del amor del Padre que está en los cielos por sus hijos de la Tierra. Recuerda, Jacobo, que un padre bueno y verdadero no solamente ama a su familia como un todo ---como una familia--- sino que también ama de verdad y cuida con afecto a cada miembro \textit{individual}.

\par 
%\textsuperscript{(1597.3)}
\textsuperscript{142:2.3} Después de mucho discutir sobre el carácter del Padre celestial, Jesús se detuvo para decir: «Tú, Jacobo, como eres padre de una familia numerosa, conoces bien la verdad de mis palabras». Y Jacobo dijo: «Pero Maestro, ¿quién te ha dicho que soy padre de seis hijos? ¿Cómo sabías esto de mí?» Y el Maestro contestó: «Basta con decir que el Padre y el Hijo conocen todas las cosas, porque en verdad lo ven todo. Puesto que amas a tus hijos como un padre terrestre, ahora debes aceptar como una realidad el amor del Padre celestial por \textit{ti} ---no solamente por todos los hijos de Abraham, sino por ti, por tu alma individual».

\par 
%\textsuperscript{(1597.4)}
\textsuperscript{142:2.4} Jesús continuó diciendo: «Cuando tus hijos son muy jóvenes e inmaduros, y has de castigarlos, pueden pensar que su padre está enojado y lleno de ira resentida. Su inmadurez no les permite penetrar más allá del castigo para discernir el afecto previsor y correctivo del padre. Pero cuando estos mismos hijos se vuelven hombres y mujeres adultos, ¿no sería insensato por su parte agarrarse a estos conceptos antiguos y equivocados sobre su padre? Como hombres y mujeres, deberían discernir ahora el amor de su padre en todas estas correcciones de los primeros años. A medida que transcurren los siglos, ¿no debería la humanidad llegar a comprender mejor la verdadera naturaleza y el carácter amoroso del Padre que está en los cielos? ¿Qué provecho sacáis de la iluminación espiritual de las generaciones sucesivas, si persistís en ver a Dios como lo veían Moisés y los profetas? Te digo, Jacobo, que a la brillante luz de esta hora, deberías ver al Padre como ninguno de tus antecesores lo han contemplado nunca. Al verlo de esta manera, deberías regocijarte por entrar en un reino donde gobierna un Padre tan misericordioso, y deberías procurar que su voluntad de amor domine tu vida de aquí en adelante».

\par 
%\textsuperscript{(1598.1)}
\textsuperscript{142:2.5} Y Jacobo contestó: «Rabino, yo creo; deseo que me conduzcas al reino del Padre».

\section*{3. El concepto de Dios}
\par 
%\textsuperscript{(1598.2)}
\textsuperscript{142:3.1} La mayoría de los doce apóstoles habían escuchado este debate sobre el carácter de Dios, y aquella noche hicieron muchas preguntas a Jesús sobre el Padre que está en los cielos. La mejor manera de presentar las respuestas del Maestro a estas preguntas consiste en resumirlas de la manera siguiente con un lenguaje moderno:

\par 
%\textsuperscript{(1598.3)}
\textsuperscript{142:3.2} Jesús reprendió suavemente a los doce, diciéndoles en esencia: ¿No conocéis las tradiciones de Israel relacionadas con el crecimiento de la idea de Yahvé, e ignoráis la enseñanza de las Escrituras sobre la doctrina de Dios? Luego el Maestro empezó a instruir a los apóstoles sobre la evolución del concepto de la Deidad a lo largo de todo el desarrollo del pueblo judío. Llamó su atención sobre las siguientes fases del crecimiento de la idea de Dios:

\par 
%\textsuperscript{(1598.4)}
\textsuperscript{142:3.3} 1. \textit{Yahvé} ---El dios de los clanes del Sinaí\footnote{\textit{Yahvé, el dios de los clanes del Sinaí}: Gn 22:14; Ex 6:3; Sal 83:18; Is 12:2; 26:4.}. Éste era el concepto primitivo de la Deidad, que Moisés elevó al nivel superior de Señor Dios de Israel. El Padre que está en los cielos nunca deja de aceptar la adoración sincera de sus hijos de la Tierra, por muy tosco que sea su concepto de la Deidad o el nombre con que simbolizan su naturaleza divina.

\par 
%\textsuperscript{(1598.5)}
\textsuperscript{142:3.4} 2. \textit{El Altísimo}. Este concepto del Padre que está en los cielos fue proclamado por Melquisedek a Abraham, y desde Salem fue llevado muy lejos por aquellos que creyeron posteriormente en esta idea ampliada y expandida de la Deidad. Abraham y su hermano se fueron de Ur\footnote{\textit{Abraham abandona Ur}: Gn 11:31.} porque se había establecido allí la adoración del Sol, y se volvieron creyentes en las enseñanzas de Melquisedek sobre El Elyón ---el Dios Altísimo\footnote{\textit{El Elyón, el Dios Altísimo}: Gn 14:18-22; Heb 7:1.}. Tenían un concepto compuesto de Dios, consistente en una mezcla de sus antiguas ideas mesopotámicas y de la doctrina del Altísimo.

\par 
%\textsuperscript{(1598.6)}
\textsuperscript{142:3.5} 3. \textit{El Shaddai}\footnote{\textit{El Shaddai, Deidad creadora}: Gn 17:1; 28:3; Ex 6:3.}. Durante aquellos tiempos primitivos, muchos hebreos adoraban a El Shaddai, el concepto egipcio del Dios del cielo, que habían aprendido durante su cautiverio en la tierra del Nilo. Mucho tiempo después de la época de Melquisedek, estos tres conceptos de Dios se fundieron en uno solo para formar la doctrina de la Deidad creadora, el Señor Dios de Israel.

\par 
%\textsuperscript{(1598.7)}
\textsuperscript{142:3.6} 4. \textit{Elohim}\footnote{\textit{Elohim (los Dioses) crearon}: Gn 1:1; Ex 6:2.}. La enseñanza sobre la Trinidad del Paraíso ha sobrevivido desde los tiempos de Adán. ¿No recordáis que las Escrituras empiezan afirmando que «En el principio, los Dioses crearon los cielos y la Tierra»? Esto indica que cuando se escribió este pasaje, el concepto trinitario de tres Dioses en uno había encontrado su lugar en la religión de nuestros antepasados.

\par 
%\textsuperscript{(1598.8)}
\textsuperscript{142:3.7} 5. \textit{El Yahvé Supremo}. En los tiempos de Isaías, estas creencias sobre Dios se habían ampliado hasta el concepto de un Creador Universal\footnote{\textit{Creador Universal}: Gn 1:1-27; 2:4-23; 5:1-2; Ex 20:11; 31:17; 2 Re 19:15; 2 Cr 2:12; Neh 9:6; Sal 115:15; 121:2; 124:8; 134:3; 146:6; Eclo 1:1-4; 33:10; Is 37:16; 40:26,28; 42:5; 45:12,18; Jer 10:11-12; 32:17; 51:15; Bar 3:32-36; Am 4:13; Mal 2:10; Mc 13:19; Jn 1:10-3; Hch 4:24; 14:15; Ef 3:9; Col 1:16; Heb 1:2; 1 P 4:19; Ap 4:11; 10:6; 14:7.} que era a la vez todopoderoso\footnote{\textit{Todopoderoso}: Ex 9:16; 15:6; 1 Cr 29:11-12; Neh 1:10; Job 36:22; 37:23; Sal 59:16; 106:8; 111:6; 147:5; Jer 10:12; 27:5; 32:17; 51:15; Nm 14:17; Nah 1:3; Dt 9:29; Mt 28:18; 2 Sam 22:33.} y totalmente misericordioso\footnote{\textit{Totalmente misericordioso}: Ex 20:6; 1 Cr 16:34,41; 2 Cr 5:13; 7:3,6; 30:9; Esd 3:11; Sal 25:6; 36:o5; 86:5,13,15; 100:5; 103:8,11,17; 107:1; 116:5; 117:2; 118:1,4; 136:1-26; 145:8; Is 54:8; 55:7; Jer 3:12; Nm 14:18-19; Miq 7:18; Dt 4:31; 5:10; Heb 8:12.}. Este concepto de Dios, en vías de evolución y ampliación, suplantó en la práctica todas las ideas anteriores que la religión de nuestros padres tenía sobre la Deidad.

\par 
%\textsuperscript{(1598.9)}
\textsuperscript{142:3.8} 6. \textit{El Padre que está en los cielos}\footnote{\textit{El Padre que está en los Cielos}: Mt 5:9,16,45,48; 6:1,9,14; 6:26,32; 7:11,21; 10:32-33; 11:25; 12:50; 15:13; 16:17; 18:10,14,19,35; 23:9; Mc 11:25-26; Lc 10:21; 11:2,13.}. Y ahora, conocemos a Dios como nuestro Padre que está en los cielos. Nuestra enseñanza proporciona una religión en la que el creyente \textit{es} un hijo de Dios. Ésta es la buena nueva del evangelio del reino de los cielos. El Hijo y el Espíritu coexisten con el Padre, y la revelación de la naturaleza y del ministerio de estas Deidades del Paraíso continuará ampliándose y clarificándose a lo largo de las eras sin fin de la progresión espiritual eterna de los hijos ascendentes de Dios. En todos los tiempos y durante todas las épocas, la adoración verdadera de cualquier ser humano ---respecto al progreso espiritual individual--- es reconocida por el espíritu interior como un homenaje que se rinde al Padre que está en los cielos.

\par 
%\textsuperscript{(1599.1)}
\textsuperscript{142:3.9} Los apóstoles nunca se habían sentido antes tan conmocionados como al escuchar este relato del crecimiento del concepto de Dios en la mente judía de las generaciones anteriores; estaban demasiado aturdidos como para hacer preguntas. Mientras permanecían sentados en silencio delante de Jesús, el Maestro continuó: «Habríais conocido estas verdades si hubierais leído las Escrituras. ¿No habéis leído lo que se dice en Samuel: `Y la ira del Señor se encendió contra Israel, de tal manera que incitó a David contra ellos, diciéndole que fuera a contar a Israel y a Judá'? Esto no era de extrañar, porque en la época de Samuel, los hijos de Abraham creían realmente que Yahvé creaba tanto el bien como el mal. Pero cuando un escritor posterior narró estos acontecimientos, después de la ampliación del concepto judío sobre la naturaleza de Dios, no se atrevió a atribuir el mal a Yahvé, y por esta razón dijo: `Y Satanás se levantó contra Israel, e incitó a David para que contara a Israel'. ¿No podéis discernir que estos relatos de las Escrituras muestran claramente cómo continuó creciendo el concepto de la naturaleza de Dios de una generación a la siguiente?»\footnote{\textit{El Señor pidió a David hacer recuento del pueblo}: 2 Sam 24:1. \textit{Satanás pidió a David hacer recuento del pueblo}: 1 Cr 21:1.}

\par 
%\textsuperscript{(1599.2)}
\textsuperscript{142:3.10} «También deberíais haber percibido el crecimiento de la comprensión de la ley divina, en perfecta congruencia con estos conceptos ampliados de la divinidad. Cuando los hijos de Israel salieron de Egipto, en una fecha anterior a la revelación ampliada de Yahvé, tenían diez mandamientos que les sirvieron de ley hasta la época en que acamparon delante del Sinaí. Estos diez mandamientos eran:»

\par 
%\textsuperscript{(1599.3)}
\textsuperscript{142:3.11} «1. No adoraréis a ningún otro dios, porque el Señor es un Dios celoso»\footnote{\textit{No adorarás a ningún otro dios}: Ex 20:3; Ex 34:14; Dt 5:7.}.

\par 
%\textsuperscript{(1599.4)}
\textsuperscript{142:3.12} «2. No fundiréis imágenes de dioses»\footnote{\textit{No fundiréis imágenes de los dioses}: Ex 20:4; Ex 34:17; Dt 5:8.}.

\par 
%\textsuperscript{(1599.5)}
\textsuperscript{142:3.13} «3. No dejaréis de guardar la fiesta del pan ázimo»\footnote{\textit{Guardaréis la fiesta de los ácimos}: Ex 12:6-11; Ex 23:15; Ex 34:18; Dt 16:1-4.}.

\par 
%\textsuperscript{(1599.6)}
\textsuperscript{142:3.14} «4. Todos los machos primogénitos de los hombres y de los animales me pertenecen, dice el Señor»\footnote{\textit{Todos los machos primogénitos son del Señor}: Ex 13:2,12; 34:19-20; Nm 18:15.}.

\par 
%\textsuperscript{(1599.7)}
\textsuperscript{142:3.15} «5. Podéis trabajar seis días, pero el séptimo descansaréis»\footnote{\textit{El séptimo día descansaréis}: Ex 20:8-11; 34:21; Dt 5:12-15.}.

\par 
%\textsuperscript{(1599.8)}
\textsuperscript{142:3.16} «6. No dejaréis de guardar la fiesta de las primeras frutas y la fiesta de la cosecha a final de año»\footnote{\textit{Celebraréis dos fiestas}: Ex 23:16; 34:22.}.

\par 
%\textsuperscript{(1599.9)}
\textsuperscript{142:3.17} «7. No ofreceréis la sangre de ningún sacrificio con pan fermentado»\footnote{\textit{No sacrificaréis con pan fermentado}: Ex 23:18a; 34:25a.}.

\par 
%\textsuperscript{(1599.10)}
\textsuperscript{142:3.18} «8. El sacrificio de la fiesta de la Pascua no se dejará allí hasta por la mañana»\footnote{\textit{No dejáreis comida pascual para el día siguiente}: Ex 23:18b; 34:25b.}.

\par 
%\textsuperscript{(1599.11)}
\textsuperscript{142:3.19} «9. Llevaréis a la casa del Señor vuestro Dios las primicias de los primeros frutos de la tierra»\footnote{\textit{Llevaréis las primicias de los frutos}: Ex 23:16,19a; 34:22,26a.}.

\par 
%\textsuperscript{(1599.12)}
\textsuperscript{142:3.20} «10. No herviréis un cabrito en la leche de su madre»\footnote{\textit{No herviréis un cabrito en la leche de su madre}: Ex 23:19b; 34:26b; Dt 14:21.}.

\par 
%\textsuperscript{(1599.13)}
\textsuperscript{142:3.21} «Luego, en medio de los truenos y los relámpagos del Sinaí, Moisés les dio los nuevos diez mandamientos, y todos admitiréis que son unas expresiones más dignas de acompañar los conceptos ampliados de la Deidad, representados como Yahvé. ¿No habéis observado nunca que estos mandamientos están registrados dos veces en las Escrituras? En el primer caso, la liberación de Egipto se señala como razón para guardar el sábado, mientras que en un escrito posterior, las creencias religiosas en progreso de nuestros antepasados exigieron que este texto fuera cambiado para reconocer el hecho de la creación como motivo para respetar el sábado»\footnote{\textit{El monte Sinaí}: Ex 19:16-18; Dt 5:4-5. \textit{Los diez mandamientos}: Ex 20:1-17; Dt 5:6-21. \textit{Salida de Egipto}: Dt 5:15. \textit{Creación en seis días}: Ex 20:11.}.

\par 
%\textsuperscript{(1599.14)}
\textsuperscript{142:3.22} «Y luego, recordaréis que una vez más ---en la época de Isaías, cuando había una mayor iluminación espiritual--- estos diez mandamientos negativos fueron cambiados por la gran ley positiva del amor, por el precepto de amar a Dios de manera suprema y a vuestro prójimo como a vosotros mismos. Yo también os declaro que esta ley suprema del amor a Dios y a los hombres constituye todo el deber de los hombres»\footnote{\textit{Cambiados por la ley del amor}: Is 38:17; 55:4-7; 63:7-9. \textit{Amar a Dios y al prójimo}: Mt 5:43-44; 19:19; 22:36-40; Mc 12:28-33; Lc 10:25-27.}.

\par 
%\textsuperscript{(1600.1)}
\textsuperscript{142:3.23} Cuando terminó de hablar, nadie le hizo ninguna pregunta. Y cada uno de ellos se retiró para descansar.

\section*{4. Flavio y la cultura griega}
\par 
%\textsuperscript{(1600.2)}
\textsuperscript{142:4.1} Flavio, el judío griego, era un prosélito sin acceso al templo, pues no había sido circuncidado ni bautizado. Como apreciaba mucho la belleza en el arte y la escultura, la casa que ocupaba durante su estancia en Jerusalén era un hermoso edificio. Este hogar estaba exquisitamente adornado con tesoros inapreciables que había rebuscado aquí y allá en sus viajes por el mundo. Cuando pensó por primera vez en invitar a Jesús a su casa, temía que el Maestro pudiera ofenderse al ver aquellas pretendidas imágenes. Pero cuando Jesús entró en la casa, Flavio se quedó agradablemente sorprendido ya que, en lugar de reprenderle por tener aquellos objetos supuestamente idólatras esparcidos por toda la casa, manifestó un gran interés por toda la colección, y mostró su aprecio haciendo muchas preguntas sobre cada objeto, mientras que Flavio lo acompañaba de una habitación a otra, mostrándole sus estatuas favoritas.

\par 
%\textsuperscript{(1600.3)}
\textsuperscript{142:4.2} El Maestro vio que su anfitrión estaba aturdido por su actitud favorable hacia el arte; por consiguiente, cuando terminaron de examinar toda la colección, Jesús dijo: «Puesto que sabes apreciar la belleza de las cosas creadas por mi Padre y modeladas por las manos artísticas del hombre, ¿por qué esperabas recibir una reprimenda? Porque Moisés intentó en otra época combatir la idolatría y la adoración de los falsos dioses, ¿por qué todos los hombres han de rechazar la reproducción de la gracia y de la belleza? Te digo, Flavio, que los hijos de Moisés lo han comprendido mal, y ahora convierten en falsos dioses hasta sus prohibiciones de las imágenes y de los retratos de las cosas del cielo y de la tierra. Pero, aunque Moisés enseñara estas restricciones a las mentes ignorantes de aquellos tiempos, ¿qué tienen que ver con nuestra época, en la que el Padre que está en los cielos es revelado como el Soberano Espiritual universal por encima de todo? Flavio, te aseguro que en el reino venidero ya no continuarán enseñando `No adoréis esto y no adoréis aquello'; ya no se ocuparán de ordenar que os abstengáis de esto y que tengáis cuidado de no hacer aquello, sino que todos se ocuparán más bien de un solo deber supremo. Y este deber de los hombres está expresado en dos grandes privilegios: la adoración sincera del Creador infinito, el Padre Paradisiaco, y el servicio amoroso otorgado a nuestros semejantes. Si amas a tu prójimo como a ti mismo, sabes realmente que eres un hijo de Dios»\footnote{\textit{Moisés combatió la idolatría}: Ex 20:4; Dt 5:8. \textit{Cambiados por la ley del amor}: Is 38:17; 55:4-7; 63:7-9. \textit{Amar a Dios y al prójimo}: Mt 5:43-44; 19:19; 22:36-40; Mc 12:28-33; Lc 10:25-27.}.

\par 
%\textsuperscript{(1600.4)}
\textsuperscript{142:4.3} «En una época en que mi Padre no era bien comprendido, las tentativas de Moisés por oponerse a la idolatría estaban justificadas, pero en la era por venir, el Padre habrá sido revelado en la vida del Hijo; y esta nueva revelación de Dios hará que sea perpetuamente inútil confundir al Padre Creador con los ídolos de piedra o las imágenes de oro y plata. En lo sucesivo, los hombres inteligentes podrán disfrutar de los tesoros del arte, sin confundir esta apreciación material de la belleza con la adoración y el servicio del Padre Paradisiaco, el Dios de todas las cosas y de todos los seres».

\par 
%\textsuperscript{(1600.5)}
\textsuperscript{142:4.4} Flavio creyó todo lo que Jesús le enseñó. Al día siguiente se dirigió a Betania más allá del Jordán y fue bautizado por los discípulos de Juan. Hizo esto porque los apóstoles de Jesús aún no bautizaban a los creyentes. Cuando Flavio regresó a Jerusalén, dio una gran fiesta para Jesús e invitó a sesenta de sus amigos. Muchos de estos convidados también se hicieron creyentes en el mensaje del reino venidero.

\section*{5. El discurso sobre la seguridad}
\par 
%\textsuperscript{(1601.1)}
\textsuperscript{142:5.1} Uno de los grandes sermones que Jesús predicó en el templo, durante esta semana de la Pascua, fue en respuesta a una pregunta que hizo uno de sus oyentes, un hombre de Damasco. Este hombre preguntó a Jesús: «Pero, Rabino, ¿cómo sabremos con certidumbre que has sido enviado por Dios, y que podemos entrar realmente en ese reino que tú y tus discípulos afirmáis que está cerca?» Y Jesús contestó:

\par 
%\textsuperscript{(1601.2)}
\textsuperscript{142:5.2} «En cuanto a mi mensaje y a las enseñanzas de mis discípulos, debéis juzgarlos por sus frutos. Si os proclamamos las verdades del espíritu, el espíritu atestiguará en vuestro corazón que nuestro mensaje es auténtico. En lo referente al reino y a vuestra seguridad de que seréis aceptados por el Padre celestial, permitidme preguntaros ¿habría entre vosotros algún padre, digno de ese nombre y de buen corazón, que mantuviera a su hijo en la ansiedad o la duda en cuanto a su posición dentro de la familia o a su grado de seguridad en el afecto del corazón de su padre? ¿Acaso vosotros, los padres terrestres, disfrutáis torturando a vuestros hijos con incertidumbres sobre el lugar que ocupan en el amor permanente de vuestro corazón humano? Vuestro Padre que está en los cielos tampoco deja a sus hijos, nacidos del espíritu por la fe, en una ambigua incertidumbre sobre su posición en el reino. Si recibís a Dios como vuestro Padre, entonces sí que sois en verdad los hijos de Dios. Y si sois sus hijos, entonces estáis seguros de la posición y del lugar de todo lo que concierne a la filiación eterna y divina. Si creéis en mis palabras, creéis de ese modo en Aquel que me ha enviado, y al creer así en el Padre, os habéis asegurado vuestra posición en la ciudadanía celestial. Si hacéis la voluntad del Padre que está en los cielos, nunca dejaréis de conseguir la vida eterna de progreso en el reino divino»\footnote{\textit{Judgad por los frutos del espíritu}: Mt 7:15-19; Mt 12:33; Lc 3:9; Lc 6:43-45; Gl 5:22-23; Ef 5:9. \textit{Hijos de Dios}: 1 Cr 22:10; Sal 2:7; Is 56:5; Mt 5:9,16,45; Lc 20:36; Jn 1:12-13; 11:52; Hch 17:28-29; Ro 8:14-17,19,21; 9:26; 2 Co 6:18; Gl 3:26; 4:5-7; Ef 1:5; Flp 2:15; Heb 12:5-8; 1 Jn 3:1-2,10; 5:2; Ap 21:7; 2 Sam 7:14.}.

\par 
%\textsuperscript{(1601.3)}
\textsuperscript{142:5.3} «El Espíritu Supremo dará testimonio con vuestro espíritu de que sois realmente los hijos de Dios. Si sois los hijos de Dios, entonces habéis nacido del espíritu de Dios; y cualquiera que ha nacido del espíritu, tiene dentro de sí el poder de vencer todas las dudas, y ésta es la victoria que supera todas las incertidumbres, vuestra propia fe»\footnote{\textit{El espíritu da testimonio del espíritu}: Ro 8:14-17; 1 Jn 5:6,8. \textit{Sobreponerse a la duda con la fe}: 1 Jn 5:4.}.

\par 
%\textsuperscript{(1601.4)}
\textsuperscript{142:5.4} «El profeta Isaías ha dicho, al hablar de esta época: `Cuando el espíritu se derrame sobre nosotros desde arriba, entonces la labor de la rectitud se convertirá en paz, tranquilidad y seguridad para siempre'. Para todos los que creen de verdad en este evangelio, yo seré la garantía de su admisión en la felicidad eterna y en la vida perpetua del reino de mi Padre. Así pues, vosotros que oís este mensaje y creéis en este evangelio del reino, sois los hijos de Dios y tenéis la vida eterna. La prueba para el mundo entero de que habéis nacido del espíritu es que os amáis sinceramente los unos a los otros»\footnote{\textit{Cuando el espíritu se derrame vendrá la paz}: Is 32:15-17. \textit{Evidencia del espíritu: el amor}: Jn 13:35.}.

\par 
%\textsuperscript{(1601.5)}
\textsuperscript{142:5.5} La multitud de oyentes permaneció muchas horas con Jesús, haciéndole preguntas y escuchando atentamente sus respuestas confortantes. La enseñanza de Jesús también animó a los apóstoles a predicar el evangelio del reino con más fuerza y seguridad. Esta experiencia en Jerusalén fue una gran inspiración para los doce. Era su primer contacto con un gentío tan enorme, y aprendieron muchas lecciones valiosas que les resultaron de gran ayuda en su trabajo posterior.

\section*{6. La conversación con Nicodemo}
\par 
%\textsuperscript{(1601.6)}
\textsuperscript{142:6.1} Una tarde, en la casa de Flavio, un tal Nicodemo vino a ver a Jesús\footnote{\textit{Nicodemo visita a Jesús}: Jn 3:1-2a.}; era un miembro rico y anciano del sanedrín judío. Había oído hablar mucho de las enseñanzas de este galileo, y por eso fue a escucharlo una tarde mientras enseñaba en los patios del templo. Hubiera querido ir a menudo a escuchar las lecciones de Jesús, pero temía ser visto por la gente que asistía a su enseñanza, porque los dirigentes de los judíos estaban ya tan en desacuerdo con Jesús, que ningún miembro del sanedrín quería que se le identificara abiertamente de alguna manera con él. En consecuencia, Nicodemo había convenido con Andrés que vería a Jesús aquella tarde concreta, en privado y después del anochecer. Pedro, Santiago y Juan se encontraban en el jardín de Flavio cuando empezó la entrevista, pero más tarde todos entraron en la casa, donde continuó la conversación.

\par 
%\textsuperscript{(1602.1)}
\textsuperscript{142:6.2} Al recibir a Nicodemo, Jesús no mostró ninguna deferencia especial; al hablar con él, no hubo concesiones ni intentos indebidos de persuasión. El Maestro no trató de rechazar a su clandestino visitante, ni fue sarcástico con él. En todo su trato con el distinguido visitante, Jesús se mostró tranquilo, serio y digno. Nicodemo no era un delegado oficial del sanedrín; vino a ver a Jesús solamente debido a su interés personal y sincero por las enseñanzas del Maestro.

\par 
%\textsuperscript{(1602.2)}
\textsuperscript{142:6.3} Después de ser presentado por Flavio, Nicodemo dijo: «Rabino, sabemos que eres un instructor enviado por Dios, porque ningún simple hombre podría enseñar así a menos que Dios estuviera con él. Y estoy deseoso de saber más cosas sobre tus enseñanzas relacionadas con el reino venidero»\footnote{\textit{La pregunta de Nicodemo}: Jn 3:2b.}.

\par 
%\textsuperscript{(1602.3)}
\textsuperscript{142:6.4} Jesús respondió a Nicodemo: «En verdad, en verdad te digo, Nicodemo, que a menos que un hombre nazca de lo alto, no puede ver el reino de Dios»\footnote{\textit{Jesús responde «Naciendo de nuevo»}: Jn 3:3.}. Entonces Nicodemo contestó: «Pero, ¿cómo puede un hombre nacer de nuevo cuando es viejo? No puede entrar por segunda vez en el seno de su madre para nacer»\footnote{\textit{¿Cómo «nacer de nuevo»?}: Jn 3:4.}.

\par 
%\textsuperscript{(1602.4)}
\textsuperscript{142:6.5} Jesús dijo: «Sin embargo, te aseguro que a menos que un hombre nazca del espíritu, no podrá entrar en el reino de Dios. Lo que ha nacido de la carne, es carne, y lo que ha nacido del espíritu, es espíritu. Pero no deberías asombrarte porque he dicho que debes nacer de lo alto. Cuando sopla el viento, oyes el susurro de las hojas, pero no ves el viento ---de dónde viene o adónde va--- y lo mismo sucede con todo aquel que ha nacido del espíritu. Con los ojos de la carne puedes contemplar las manifestaciones del espíritu, pero no puedes discernir realmente al espíritu»\footnote{\textit{Nacido del espíritu}: Jn 3:5-8.}.

\par 
%\textsuperscript{(1602.5)}
\textsuperscript{142:6.6} Nicodemo respondió: «Pero no comprendo ---¿cómo puede ser eso?»\footnote{\textit{Nicodemo repite su pregunta}: Jn 3:9.} Jesús dijo: «¿Es posible que seas un educador de Israel y que sin embargo ignores todo esto? Los que conocen las realidades del espíritu tienen pues el deber de revelar estas cosas a los que disciernen solamente las manifestaciones del mundo material. Pero ¿nos creerás si te hablamos de las verdades celestiales? ¿Tienes el coraje de creer, Nicodemo, en alguien que ha descendido del cielo, en el mismo Hijo del Hombre?»\footnote{\textit{Realidades del espíritu}: Jn 3:10-13.}

\par 
%\textsuperscript{(1602.6)}
\textsuperscript{142:6.7} Y Nicodemo dijo: «Pero ¿cómo puedo empezar a captar ese espíritu que ha de rehacerme como preparación para entrar en el reino?» Jesús respondió: «El espíritu del Padre que está en los cielos ya reside dentro de ti. Si quieres dejarte conducir por este espíritu que viene de arriba, muy pronto empezarás a ver con los ojos del espíritu; a continuación, si escoges de todo corazón seguir la orientación del espíritu, nacerás del espíritu, porque el único propósito de tu vida será hacer la voluntad de tu Padre que está en los cielos. Al encontrarte así, nacido del espíritu y feliz en el reino de Dios, empezarás a producir en tu vida diaria los frutos abundantes del espíritu».

\par 
%\textsuperscript{(1602.7)}
\textsuperscript{142:6.8} Nicodemo era completamente sincero. Estaba profundamente impresionado, pero se fue desconcertado. Era un hombre realizado en cuanto al desarrollo personal, al dominio de sí mismo e incluso a las altas cualidades morales. Era refinado, egoísta y altruista, pero no sabía cómo \textit{someter} su voluntad a la voluntad del Padre divino, como un niño pequeño está dispuesto a someterse a la guía y dirección de un padre terrestre sabio y amoroso, convirtiéndose así en realidad en un hijo de Dios, en un heredero progresivo del reino eterno.

\par 
%\textsuperscript{(1603.1)}
\textsuperscript{142:6.9} Pero Nicodemo supo reunir la suficiente fe como para apoderarse del reino. Protestó tímidamente cuando sus colegas del sanedrín intentaron condenar a Jesús sin juicio. Más tarde, con José de Arimatea, reconoció audazmente su fe\footnote{\textit{Reconocimiento público}: Jn 19:38-42.} y reclamó el cuerpo de Jesús, incluso cuando la mayoría de los discípulos habían huido atemorizados del escenario del sufrimiento y de la muerte final de su Maestro.

\section*{7. La lección sobre la familia}
\par 
%\textsuperscript{(1603.2)}
\textsuperscript{142:7.1} Después del activo período de enseñanza y de trabajo personal durante la semana pascual en Jerusalén, Jesús pasó el miércoles siguiente descansando con sus apóstoles en Betania. Aquella tarde, Tomás hizo una pregunta que atrajo una respuesta larga e instructiva. Tomás dijo: «Maestro, el día que fuimos seleccionados como embajadores del reino, nos dijiste muchas cosas; nos instruiste sobre nuestra manera personal de vivir, pero, ¿qué le enseñaremos a la multitud? ¿Cómo deberá vivir esa gente después de que el reino llegue más plenamente? ¿Tus discípulos poseerán esclavos? ¿Tus fieles buscarán la pobreza y huirán de la riqueza? ¿Prevalecerá solamente la misericordia, de tal manera que ya no tendremos ni ley ni justicia?» Jesús y los doce pasaron toda la tarde y toda aquella noche, después de la cena, discutiendo las preguntas de Tomás. Para los propósitos de esta narración, presentamos el siguiente resumen de las instrucciones del Maestro:

\par 
%\textsuperscript{(1603.3)}
\textsuperscript{142:7.2} En primer lugar, Jesús intentó aclarar a sus apóstoles que él mismo estaba en la Tierra viviendo una vida excepcional en la carne, y que ellos doce habían sido llamados para participar en esta experiencia donadora del Hijo del Hombre; como tales colaboradores, también tenían que compartir muchas de las restricciones y obligaciones especiales de toda esta experiencia de donación. Hubo una insinuación velada a que el Hijo del Hombre era la única persona que había vivido en la Tierra, capaz de ver simultáneamente dentro del corazón mismo de Dios y en las profundidades del alma humana.

\par 
%\textsuperscript{(1603.4)}
\textsuperscript{142:7.3} Jesús explicó muy claramente que el reino de los cielos era una experiencia evolutiva que empezaba aquí, en la Tierra, y progresaba por medio de etapas sucesivas de vida hasta el Paraíso. En el transcurso de la noche indicó con precisión que en alguna fase futura del desarrollo del reino, volvería a visitar este mundo con poder espiritual y gloria divina.

\par 
%\textsuperscript{(1603.5)}
\textsuperscript{142:7.4} Luego explicó que la «idea del reino» no era la mejor manera de ilustrar la relación del hombre con Dios; que empleaba esta metáfora porque el pueblo judío estaba esperando el reino, y porque Juan había predicado refiriéndose al reino por venir. Jesús dijo: «La gente de otra época comprenderá mejor el evangelio del reino cuando éste sea presentado en unos términos que expresen la relación familiar ---cuando el hombre comprenda la religión como la enseñanza de la paternidad de Dios y la fraternidad de los hombres, la filiación con Dios»\footnote{\textit{Analogía de «la familia»}: 1 Jn 3:1-2a.}. Después, el Maestro disertó con cierta amplitud sobre la familia terrenal, como una ilustración de la familia celestial, exponiendo de nuevo las dos leyes fundamentales de la vida: el primer mandamiento de amor por el padre, el cabeza de familia, y el segundo mandamiento de amor mutuo entre los hijos, el de amar al hermano como a sí mismo. Luego explicó que esta cualidad del afecto fraternal se manifestaría invariablemente en el servicio social desinteresado y amoroso.

\par 
%\textsuperscript{(1603.6)}
\textsuperscript{142:7.5} A esto le siguió el debate memorable sobre las características fundamentales de la vida familiar, y su aplicación a la relación existente entre Dios y el hombre. Jesús declaró que una verdadera familia está fundada en los siete hechos siguientes:

\par 
%\textsuperscript{(1604.1)}
\textsuperscript{142:7.6} 1. \textit{El hecho de la existencia}. Las relaciones de la naturaleza y los fenómenos del parecido físico están ligados en la familia: los hijos heredan ciertas características parentales. Los hijos tienen su origen en sus padres; la existencia de su personalidad depende del acto de los padres. La relación de padre a hijo es inherente a toda la naturaleza e impregna todas las existencias vivientes.

\par 
%\textsuperscript{(1604.2)}
\textsuperscript{142:7.7} 2. \textit{La seguridad y el placer}. Los padres auténticos experimentan un gran placer satisfaciendo las necesidades de sus hijos. Muchos padres no se contentan con abastecer simplemente las necesidades de sus hijos, sino que disfrutan también asegurándoles sus placeres.

\par 
%\textsuperscript{(1604.3)}
\textsuperscript{142:7.8} 3. \textit{La educación y la preparación}. Los padres sabios planean cuidosamente la educación y la preparación adecuada de sus hijos e hijas. Se les prepara desde que son jóvenes para las responsabilidades mayores de la vida adulta.

\par 
%\textsuperscript{(1604.4)}
\textsuperscript{142:7.9} 4. \textit{La disciplina y la restricción}. Los padres previsores también toman medidas para la disciplina, la dirección, la corrección y a veces la restricción necesarias de sus descendientes jóvenes e inmaduros.

\par 
%\textsuperscript{(1604.5)}
\textsuperscript{142:7.10} 5. \textit{La camaradería y la lealtad}. El padre afectuoso mantiene una relación íntima y amorosa con sus hijos. Siempre está dispuesto a escuchar sus peticiones; siempre está preparado para compartir sus penalidades y ayudarlos en sus dificultades. El padre se interesa de manera suprema por el bienestar progresivo de su descendencia.

\par 
%\textsuperscript{(1604.6)}
\textsuperscript{142:7.11} 6. \textit{El amor y la misericordia}. Un padre compasivo perdona espontáneamente; los padres no alimentan ideas de venganza contra sus hijos. Los padres no son como los jueces, los enemigos o los acreedores. Las familias verdaderas están construidas sobre la tolerancia, la paciencia y el perdón.

\par 
%\textsuperscript{(1604.7)}
\textsuperscript{142:7.12} 7. \textit{Las disposiciones para el futuro}. A los padres temporales les gusta dejar una herencia para sus hijos. La familia continúa de una generación a la siguiente. La muerte sólo acaba con una generación para marcar el comienzo de la siguiente. La muerte pone término a una vida individual, pero no necesariamente a la familia.

\par 
%\textsuperscript{(1604.8)}
\textsuperscript{142:7.13} El Maestro examinó durante horas la aplicación de estas características de la vida familiar a las relaciones del hombre ---el hijo terrestre--- con Dios ---el Padre Paradisiaco. Y ésta fue su conclusión: «Conozco a la perfección la totalidad de esta relación de un hijo con el Padre, porque ya he alcanzado ahora, en el terreno de la filiación, todo lo que tendréis que alcanzar en el eterno futuro. El Hijo del Hombre está preparado para ascender a la diestra del Padre, de manera que, en mí, el camino está ahora aún más abierto para que todos vosotros veáis a Dios y, antes de que hayáis terminado la gloriosa progresión, os volváis perfectos como vuestro Padre que está en los cielos es perfecto»\footnote{\textit{Sed perfectos}: Gn 17:1; 1 Re 8:61; Lv 19:2; Dt 18:13; Mt 5:48; 2 Co 13:11; Stg 1:4; 1 P 1:16.}.

\par 
%\textsuperscript{(1604.9)}
\textsuperscript{142:7.14} Cuando los apóstoles escucharon estas palabras sorprendentes, recordaron las declaraciones que Juan había hecho en la época del bautismo de Jesús; también se acordaron vívidamente de esta experiencia en conexión con sus predicaciones y enseñanzas, después de la muerte y resurrección del Maestro.

\par 
%\textsuperscript{(1604.10)}
\textsuperscript{142:7.15} Jesús es un Hijo divino que cuenta con toda la confianza del Padre Universal. Había estado con el Padre y lo comprendía plenamente. Ahora había vivido su vida terrestre a la entera satisfacción del Padre, y esta encarnación en la carne le había permitido comprender plenamente al hombre. Jesús era la perfección del hombre; había alcanzado la misma perfección que todos los verdaderos creyentes están destinados a alcanzar en él y a través de él. Jesús reveló al hombre un Dios de perfección, y presentó a Dios, en su propia persona, al hijo perfeccionado de los mundos.

\par 
%\textsuperscript{(1605.1)}
\textsuperscript{142:7.16} Aunque Jesús estuvo hablando durante varias horas, Tomás aún no estaba satisfecho, puesto que dijo: «Pero, Maestro, no nos parece que el Padre que está en los cielos nos trate siempre con bondad y misericordia. Muchas veces sufrimos enormemente en la Tierra, y nuestras oraciones no siempre son contestadas. ¿En qué punto no conseguimos captar el significado de tu enseñanza?»

\par 
%\textsuperscript{(1605.2)}
\textsuperscript{142:7.17} Jesús replicó: «Tomás, Tomás, ¿cuánto tiempo necesitarás para adquirir la aptitud de escuchar con el oído del espíritu? ¿Cuánto tiempo pasará antes de que disciernas que este reino es un reino espiritual, y que mi Padre es también un ser espiritual? ¿No comprendes que os enseño como hijos espirituales de la familia espiritual del cielo, cuyo jefe paterno es un espíritu infinito y eterno? ¿No me permitiréis que utilice la familia terrestre para ilustrar las relaciones divinas, sin aplicar mi enseñanza tan literalmente a los asuntos materiales? ¿No podéis separar en vuestra mente las realidades espirituales del reino, de los problemas materiales, sociales, económicos y políticos de esta época? Cuando hablo el lenguaje del espíritu, ¿por qué insistís en traducir mi intención al lenguaje de la carne, simplemente porque me tomo la libertad de emplear las relaciones vulgares y literales con una finalidad ilustrativa? Hijos míos, os ruego que dejéis de aplicar la enseñanza del reino del espíritu a los sórdidos asuntos de la esclavitud, la pobreza, las casas y las tierras, y a los problemas materiales de la equidad y la justicia humanas. Estas cuestiones temporales interesan a los hombres de este mundo, y aunque en cierto modo afectan a todos los hombres, habéis sido llamados para representarme en el mundo como yo represento a mi Padre. Sois los embajadores espirituales de un reino espiritual, los representantes especiales del Padre del espíritu. A estas alturas, ya debería poder instruiros como hombres maduros del reino del espíritu. ¿Tendré que seguir hablándoos como si fuerais niños? ¿No creceréis nunca en percepción espiritual? Sin embargo, os amo y seré indulgente con vosotros hasta el fin de nuestra asociación en la carne. E incluso entonces, mi espíritu os precederá en el mundo entero».

\section*{8. En Judea del sur}
\par 
%\textsuperscript{(1605.3)}
\textsuperscript{142:8.1} A finales de abril, la oposición de los fariseos y saduceos se había vuelto tan pronunciada contra Jesús, que el Maestro y sus apóstoles decidieron dejar Jerusalén por un tiempo, y se dirigieron hacia el sur para trabajar en Belén y Hebrón\footnote{\textit{Regreso a Judea}: Jn 3:22.}. Pasaron todo el mes de mayo efectuando un trabajo personal en estas ciudades y entre la gente de los pueblos vecinos. Durante este viaje no hicieron ninguna predicación pública, sino solamente visitas de casa en casa. Mientras los apóstoles enseñaban el evangelio y cuidaban a los enfermos, Jesús y Abner pasaron una parte de este tiempo en En-Gedi, visitando la colonia nazarea. Juan el Bautista había salido de este lugar, y Abner había sido jefe de este grupo. Muchos miembros de la fraternidad nazarea se hicieron creyentes en Jesús, pero la mayoría de estos hombres ascéticos y extravagantes rehusó aceptarlo como un instructor enviado del cielo, porque no enseñaba el ayuno ni otras formas de abnegación.

\par 
%\textsuperscript{(1605.4)}
\textsuperscript{142:8.2} La gente que vivía en esta región no sabía que Jesús había nacido en Belén. Al igual que la gran mayoría de sus discípulos, siempre habían supuesto que el Maestro había nacido en Nazaret, pero los doce conocían la verdad.

\par 
%\textsuperscript{(1605.5)}
\textsuperscript{142:8.3} Esta estancia en el sur de Judea fue un período de trabajo reposado y fructífero; muchas almas se añadieron al reino. A primeros de junio, la agitación contra Jesús se había calmado tanto en Jerusalén, que el Maestro y los apóstoles regresaron para instruir y alentar a los creyentes.

\par 
%\textsuperscript{(1606.1)}
\textsuperscript{142:8.4} Aunque Jesús y los apóstoles pasaron todo el mes de junio en Jerusalén o en las proximidades, no efectuaron ninguna enseñanza pública durante este período. Vivieron la mayor parte del tiempo en las tiendas que montaron en un parque o jardín sombreado conocido en aquella época con el nombre de Getsemaní. Este parque estaba situado en la ladera occidental del Monte de los Olivos, no lejos del arroyo Cedrón. Los sábados del fin de semana los pasaban habitualmente con Lázaro y sus hermanas en Betania. Jesús entró pocas veces dentro de los muros de Jerusalén, pero un gran número de investigadores interesados fueron hasta Getsemaní para charlar con él. Un viernes por la noche, Nicodemo y un tal José de Arimatea se atrevieron a salir para visitar a Jesús, pero cuando estaban delante de la entrada de la tienda del Maestro, se volvieron atrás por miedo. Por supuesto, no se percataban de que Jesús conocía todo lo que hacían.

\par 
%\textsuperscript{(1606.2)}
\textsuperscript{142:8.5} Cuando los dirigentes de los judíos se enteraron de que Jesús había regresado a Jerusalén, se prepararon para arrestarlo; pero al observar que no predicaba en público, concluyeron que se había asustado con el alboroto que habían causado anteriormente, y decidieron permitirle que continuara enseñando de esta manera privada, sin molestarlo más. Así es como las cosas siguieron desarrollándose tranquilamente hasta los últimos días de junio, cuando un tal Simón, miembro del sanedrín, abrazó públicamente las enseñanzas de Jesús, después de decírselo en persona a los jefes de los judíos. Inmediatamente se produjo un nuevo alboroto para capturar a Jesús, y tomó tal importancia, que el Maestro decidió retirarse a las ciudades de Samaria y la Decápolis\footnote{\textit{Regreso a la Decápolis}: Jn 4:1-3.}.