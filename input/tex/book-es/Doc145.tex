\chapter{Documento 145. Cuatro días memorables en Cafarnaúm}
\par 
%\textsuperscript{(1628.1)}
\textsuperscript{145:0.1} JESÚS y los apóstoles llegaron a Cafarnaúm\footnote{\textit{Jesús y sus apóstoles en Cafarnaúm}: Mc 1:21a; Lc 4:31.} el martes 13 de enero al anochecer. Como de costumbre, establecieron su cuartel general en la casa de Zebedeo, en Betsaida. Ahora que Juan el Bautista había sido ejecutado, Jesús se preparó para lanzarse abiertamente a su primera gira de predicación pública en Galilea. La noticia del regreso de Jesús se difundió rápidamente por toda la ciudad, y a primeras horas del día siguiente, María, la madre de Jesús, salió apresuradamente hacia Nazaret para visitar a su hijo José.

\par 
%\textsuperscript{(1628.2)}
\textsuperscript{145:0.2} Jesús pasó el miércoles, el jueves y el viernes en la casa de Zebedeo instruyendo a sus apóstoles como preparación para su primera gran gira de predicación pública. También recibió y enseñó, tanto individualmente como en grupo, a muchos investigadores serios. Por medio de Andrés, arregló las cosas para hablar en la sinagoga el sábado siguiente.

\par 
%\textsuperscript{(1628.3)}
\textsuperscript{145:0.3} Al final de la tarde del viernes, Rut, la hermana menor de Jesús, le hizo una visita en secreto. Pasaron casi una hora juntos en una barca anclada a poca distancia de la costa. Ningún ser humano se enteró nunca de esta visita, salvo Juan Zebedeo, a quien se le recomendó que no se lo dijera a nadie. Rut era el único miembro de la familia de Jesús que creía, de manera firme y constante, en la divinidad de la misión terrestre de su hermano; y lo creyó desde su más temprana conciencia espiritual, pasando por todo el ministerio extraordinario de Jesús, su muerte, su resurrección y su ascensión. Finalmente, Rut pasó a los mundos del más allá sin haber dudado nunca del carácter sobrenatural de la misión en la carne de su hermano-padre. En lo que respecta a su familia terrestre, la pequeña Rut fue el principal consuelo de Jesús durante las penosas pruebas de su juicio, su rechazo y su crucifixión.

\section*{l. La redada de peces}
\par 
%\textsuperscript{(1628.4)}
\textsuperscript{145:1.1} El viernes por la mañana de esta misma semana, cuando Jesús estaba enseñando al lado de la playa, la gente se apiñó junto a él tan cerca del borde del agua, que hizo señas a unos pescadores que estaban en una barca cercana para que vinieran a rescatarlo. Subió a la barca y continuó enseñando durante más de dos horas a la multitud reunida\footnote{\textit{Jesús habla desde una barca}: Lc 5:1-3.}. Esta barca tenía el nombre de «Simón»; era la antigua embarcación de pesca de Simón Pedro y había sido construida por las mismas manos de Jesús. Aquella precisa mañana, la barca estaba siendo utilizada por David Zebedeo y dos socios\footnote{\textit{Simón y David}: Mt 4:18; Mc 1:16.}, que acababan de volver a la costa después de una noche de pesca infructuosa en el lago. Estaban limpiando y reparando sus redes cuando Jesús les pidió que vinieran en su ayuda.

\par 
%\textsuperscript{(1628.5)}
\textsuperscript{145:1.2} Después de que Jesús hubo terminado de enseñar a la gente, dijo a David: «Como os habéis retrasado por venir a ayudarme, permitidme ahora trabajar con vosotros. Vamos a pescar. Dirigíos hacia esa parte profunda y dejad caer vuestras redes para hacer una captura». Pero Simón, uno de los ayudantes de David, respondió: «Maestro, es inútil. Hemos faenado toda la noche y no hemos cogido nada; sin embargo, puesto que tú lo ordenas, vamos a salir y arrojaremos las redes». Simón consintió en seguir las instrucciones de Jesús porque David, su patrón, se lo indicó con un gesto. Cuando llegaron al lugar señalado por Jesús, lanzaron sus redes y reunieron tal cantidad de peces que tuvieron miedo de que se rompieran las redes; tanto fue así que hicieron señas a sus asociados de la costa para que vinieran a ayudarlos. Cuando llenaron totalmente las tres barcas de peces, casi hasta el punto de hundirse, el tal Simón se postró a los pies de Jesús, diciendo: «Apártate de mí, Maestro, porque soy un pecador». Simón y todos los implicados en este episodio se quedaron atónitos con esta redada de peces\footnote{\textit{La redada de peces}: Lc 5:4-11.}. A partir de aquel día, David Zebedeo, este Simón, y sus asociados abandonaron sus redes y siguieron a Jesús\footnote{\textit{Pescadores de hombres}: Mt 4:19-20; Mc 1:17-18.}.

\par 
%\textsuperscript{(1629.1)}
\textsuperscript{145:1.3} Pero ésta no fue en ningún sentido una pesca milagrosa. Jesús era un atento observador de la naturaleza; era un pescador experto y conocía las costumbres de los peces en el Mar de Galilea. En esta ocasión, se limitó a dirigir a estos hombres hacia el lugar donde los peces se encontraban a aquella hora del día. Pero los seguidores de Jesús siempre consideraron este suceso como un milagro.

\section*{2. La tarde en la sinagoga}
\par 
%\textsuperscript{(1629.2)}
\textsuperscript{145:2.1} El sábado siguiente, en los oficios de la tarde en la sinagoga\footnote{\textit{Jesús predica en la sinagoga}: Mc 1:21; Lc 4:31.}, Jesús predicó su sermón sobre «La voluntad del Padre que está en los cielos». Por la mañana, Simón Pedro había predicado sobre «El reino». En la reunión del jueves por la noche en la sinagoga, Andrés había enseñado sobre el tema «El nuevo camino». En aquel momento concreto, la gente que creía en Jesús era más numerosa en Cafarnaúm que en cualquier otra ciudad de la Tierra.

\par 
%\textsuperscript{(1629.3)}
\textsuperscript{145:2.2} Cuando Jesús enseñó en la sinagoga aquel sábado por la tarde, siguiendo la costumbre cogió su primer texto en la ley y leyó en el Libro del Éxodo: «Servirás al Señor tu Dios, y él bendecirá tu pan y tu agua, y toda enfermedad será apartada de ti»\footnote{\textit{Servirás al Señor, tu Dios}: Ex 23:25.}. El segundo texto lo escogió en los Profetas, leyendo en Isaías: «Levántate y resplandece, porque ha venido tu luz, y la gloria del Señor se ha levantado sobre ti. La oscuridad puede cubrir la Tierra y las profundas tinieblas envolver a la gente, pero el espíritu del Señor se levantará sobre ti y verán que la gloria divina te acompaña. Incluso los gentiles vendrán hacia esta luz, y muchos grandes pensadores se rendirán ante su resplandor»\footnote{\textit{Levántate, resplandece, ha venido la luz}: Is 60:1-3.}.

\par 
%\textsuperscript{(1629.4)}
\textsuperscript{145:2.3} Este sermón fue un esfuerzo por parte de Jesús para exponer claramente el hecho de que la religión es una \textit{experiencia personal}. Entre otras cosas, el Maestro dijo:

\par 
%\textsuperscript{(1629.5)}
\textsuperscript{145:2.4} «Sabéis bien que, aunque un padre de buen corazón ama a su familia como un todo, los considera así como grupo a causa de su sólido afecto por cada miembro individual de esa familia. Hay que dejar de acercarse al Padre que está en los cielos como un hijo de Israel, y hacerlo como un \textit{hijo de Dios}. Como grupo, sois en efecto los hijos de Israel, pero como individuos, cada uno de vosotros es un hijo de Dios. He venido, no para revelar el Padre a los hijos de Israel, sino más bien para traer al creyente individual este conocimiento de Dios y la revelación de su amor y de su misericordia como una experiencia personal auténtica. Todos los profetas os han enseñado que Yahvé cuida a su pueblo, que Dios ama a Israel. Pero yo he venido en medio de vosotros para proclamar una verdad más grande, una verdad que muchos de los últimos profetas también captaron, la verdad de que Dios \textit{os} ama ---a cada uno de vosotros--- como individuos. Durante todas estas generaciones, habéis tenido una religión nacional o racial; yo he venido ahora para daros una religión personal».

\par 
%\textsuperscript{(1630.1)}
\textsuperscript{145:2.5} «Pero incluso esto no es una idea nueva. Muchos de los que tenéis inclinaciones espirituales habéis conocido esta verdad, puesto que algunos profetas así os lo han enseñado. ¿No habéis leído en las Escrituras lo que dice el profeta Jeremías?: `En aquellos días ya no volverán a decir: los padres han comido uvas verdes y son los hijos los que tienen la dentera. Cada cual morirá por su propia iniquidad; todo hombre que coma uvas verdes tendrá dentera. Mirad, se acercan los días en que haré un nuevo pacto con mi pueblo, no según el pacto que hice con sus padres cuando los saqué de la tierra de Egipto, sino según el nuevo camino. Incluso escribiré mi ley en sus corazones. Yo seré su Dios, y ellos serán mi pueblo. Cuando llegue ese día, los hombres ya no dirán a sus vecinos: ¿conoces al Señor? ¡No! Porque todos me conocerán personalmente, desde el más humilde hasta el más grande'.»\footnote{\textit{Uvas verdes, hijos con dentera}: Jer 31:29-34.}

\par 
%\textsuperscript{(1630.2)}
\textsuperscript{145:2.6} «¿No habéis leído estas promesas? ¿No creéis en las Escrituras? ¿No comprendéis que las palabras del profeta se están cumpliendo en lo que contempláis hoy mismo? ¿No os ha exhortado Jeremías a que hagáis de la religión un asunto del corazón, a que os relacionéis con Dios como individuos? ¿No os ha dicho el profeta que el Dios de los cielos sondearía vuestros corazones individuales? ¿Y no se os ha advertido que el corazón humano es, por naturaleza, más engañoso que nada, y con mucha frecuencia desesperadamente perverso?»\footnote{\textit{La religión un asunto del corazón}: Jer 24:7. \textit{Sondeará vuestros corazones individuales}: Jer 17:9-10.}

\par 
%\textsuperscript{(1630.3)}
\textsuperscript{145:2.7} «¿No habéis leído también el pasaje donde Ezequiel enseñó a vuestros padres que la religión debe convertirse en una realidad en vuestra experiencia individual? Ya no utilizaréis el proverbio que dice: `Los padres han comido uvas verdes y son los hijos los que tienen la dentera'. `Tan cierto como que estoy vivo', dice el Señor Dios, `he aquí que todas las almas me pertenecen; tanto el alma del padre como el alma del hijo. Sólo el alma que peque morirá'. Y luego, Ezequiel predijo incluso el día de hoy cuando habló en nombre de Dios, diciendo: `Os daré también un nuevo corazón, y pondré dentro de vosotros un espíritu nuevo'.»\footnote{\textit{Sólo el alma pecadora morirá}: Jer 31:29-30; Ez 18:2-4. \textit{Dios dará un nuevo corazón, un nuevo espíritu}: Ez 36:26.}

\par 
%\textsuperscript{(1630.4)}
\textsuperscript{145:2.8} «Debéis dejar de temer que Dios castiga a una nación por el pecado de un individuo. El Padre que está en los cielos tampoco castigará a uno de sus hijos creyentes por los pecados de una nación, aunque un miembro individual de una familia pueda sufrir a menudo las consecuencias materiales de los errores familiares y de las transgresiones colectivas. ¿No os dais cuenta de que la esperanza de tener una nación mejor ---o un mundo mejor--- está ligada al progreso y a la iluminación del individuo?»

\par 
%\textsuperscript{(1630.5)}
\textsuperscript{145:2.9} Luego el Maestro describió que, una vez que los hombres disciernen esta libertad espiritual, el Padre que está en los cielos quiere que sus hijos de la Tierra empiecen la ascensión eterna de la carrera hacia el Paraíso, que consiste en una respuesta consciente de la criatura al impulso divino del espíritu interior por encontrar al Creador, conocer a Dios y tratar de volverse semejante a él.

\par 
%\textsuperscript{(1630.6)}
\textsuperscript{145:2.10} Este sermón fue de una gran ayuda para los apóstoles. Todos comprendieron mucho mejor que el evangelio del reino es un mensaje destinado al individuo, no a la nación.

\par 
%\textsuperscript{(1630.7)}
\textsuperscript{145:2.11} Aunque los habitantes de Cafarnaúm estaban familiarizados con las enseñanzas de Jesús, se quedaron asombrados con su sermón de este sábado. Enseñó, en verdad, como alguien que tiene autoridad, y no como los escribas\footnote{\textit{Jesús enseñó con autoridad}: Mt 7:28-29; Mc 1:21-22; Lc 4:31-32.}.

\par 
%\textsuperscript{(1630.8)}
\textsuperscript{145:2.12} En el preciso momento en que Jesús terminaba de hablar, un joven de la asamblea que se había perturbado mucho con sus palabras cayó víctima de un violento ataque epiléptico, acompañado de grandes gritos. Al final de la crisis, cuando estaba recobrando la conciencia, habló en un estado de ensueño, diciendo: «¿Qué vamos a hacer contigo, Jesús de Nazaret? Eres el santo de Dios; ¿has venido para destruirnos?» Jesús pidió a la gente que permaneciera tranquila, cogió al joven por la mano, y le dijo: «Sal de ese estado»; y se despertó inmediatamente\footnote{\textit{Curación del epiléptico}: Mc 1:23-26; Lc 4:33-35.}.

\par 
%\textsuperscript{(1631.1)}
\textsuperscript{145:2.13} Este joven no estaba poseído por un espíritu impuro o un demonio; era víctima de una epilepsia corriente. Pero le habían enseñado que su afección se debía a que estaba poseído por un espíritu maligno. Creía en lo que le habían dicho y se comportaba de acuerdo con ello en todo lo que pensaba o decía sobre su enfermedad. Toda la gente creía que estos fenómenos estaban causados directamente por la presencia de los espíritus impuros. En consecuencia, creyeron que Jesús había echado un demonio de este hombre. Pero Jesús no lo curó de su epilepsia en aquel momento. Este joven no se curó realmente hasta más tarde, aquel mismo día, después de la puesta del Sol. Mucho después del día de Pentecostés, el apóstol Juan, que fue el último que escribió sobre las actividades de Jesús, evitó toda referencia a estas pretendidas «expulsiones de demonios», y lo hizo así debido al hecho de que estos casos de posesión demoníaca no volvieron a producirse después de Pentecostés.

\par 
%\textsuperscript{(1631.2)}
\textsuperscript{145:2.14} Como resultado de este vulgar incidente, por todo Cafarnaúm se divulgó rápidamente la noticia de que Jesús había echado un demonio de un hombre\footnote{\textit{Noticias de la curación}: Mc 1:27-28; Lc 4:36-37.}, y que lo había curado milagrosamente en la sinagoga al final de su sermón de la tarde. El sábado era el momento propicio para que este rumor sorprendente se propagara de manera rápida y eficaz. Esta noticia llegó también a todas las poblaciones más pequeñas que rodeaban a Cafarnaúm, y mucha gente se la creyó.

\par 
%\textsuperscript{(1631.3)}
\textsuperscript{145:2.15} La esposa y la suegra de Simón Pedro hacían la mayor parte de la cocina y del trabajo doméstico en la gran casa de Zebedeo, donde Jesús y los doce habían establecido su cuartel general. La casa de Pedro estaba cerca de la de Zebedeo. Jesús y sus amigos se detuvieron allí al regresar de la sinagoga porque la madre de la esposa de Pedro llevaba varios días enferma con fiebre y escalofríos. Sucedió por casualidad que la fiebre se le quitó mientras Jesús estaba de pie al lado de la enferma, sosteniendo su mano, acariciándole la frente y diciéndole palabras de consuelo y de aliento\footnote{\textit{La suegra de Pedro, enferma}: Mt 8:14-15; Mc 8:30-31; Lc 4:38-39.}. Jesús aún no había tenido tiempo de explicar a sus apóstoles que no se había producido ningún milagro en la sinagoga; con este incidente tan reciente y vívido en su memoria, y al recordar el agua y el vino de Caná, tomaron esta coincidencia como otro milagro, y algunos de ellos salieron precipitadamente para difundir la noticia por toda la ciudad.

\par 
%\textsuperscript{(1631.4)}
\textsuperscript{145:2.16} Amata, la suegra de Pedro, padecía de paludismo. En aquel momento no fue curada milagrosamente por Jesús. Su curación no se realizó hasta varias horas más tarde, después de la puesta del Sol, en conexión con el extraordinario acontecimiento que se produjo en el patio delantero de la casa de Zebedeo.

\par 
%\textsuperscript{(1631.5)}
\textsuperscript{145:2.17} Estos casos son típicos de la manera en que una generación en busca de prodigios, y un pueblo propenso a ver milagros, se aferraban indefectiblemente a todas estas coincidencias como pretexto para proclamar que Jesús había efectuado otro milagro.

\section*{3. La curación a la puesta del Sol}
\par 
%\textsuperscript{(1631.6)}
\textsuperscript{145:3.1} En el momento en que Jesús y sus apóstoles se disponían a compartir su cena, casi al final de este sábado memorable, todo Cafarnaúm y sus alrededores estaban alborotados a causa de estas pretendidas curaciones milagrosas; y todos los que estaban enfermos o afligidos empezaron a prepararse para ir a ver a Jesús, o para que sus amigos los transportaran hasta allí, en cuanto se pusiera el Sol. Según las enseñanzas judías, ni siquiera estaba permitido buscar la salud durante las horas sagradas del sábado.

\par 
%\textsuperscript{(1632.1)}
\textsuperscript{145:3.2} Así pues, tan pronto como el Sol desapareció por el horizonte, decenas de hombres, mujeres y niños afligidos empezaron a dirigirse hacia la casa de Zebedeo en Betsaida. Un hombre salió con su hija paralítica en cuanto el Sol se ocultó por detrás de la casa de su vecino\footnote{\textit{La multitud a la puesta de sol}: Mt 8:16; Mc 1:32-33; Lc 4:40.}.

\par 
%\textsuperscript{(1632.2)}
\textsuperscript{145:3.3} Los acontecimientos de todo este día habían preparado el escenario para este espectáculo extraordinario a la puesta del Sol. Incluso el texto que Jesús había utilizado en su sermón de la tarde daba a entender que la enfermedad debía ser desterrada; ¡y había hablado con un poder y una autoridad sin precedentes! ¡Su mensaje era tan apremiante! Aunque no había apelado a la autoridad humana, había hablado directamente a la conciencia y al alma de los hombres. Aún cuando no había recurrido a la lógica, a las argucias legales o a las aserciones ingeniosas, había efectuado un poderoso llamamiento directo, claro y personal al corazón de sus oyentes.

\par 
%\textsuperscript{(1632.3)}
\textsuperscript{145:3.4} Este sábado fue un gran día en la vida terrestre de Jesús, y en la vida de un universo. Para todo el universo local, la pequeña ciudad judía de Cafarnaúm fue, en todos los sentidos, la verdadera capital de Nebadon. El puñado de judíos de la sinagoga de Cafarnaúm no eran los únicos seres que escucharon esta importante declaración con la que Jesús concluyó su sermón: «El odio es la sombra del miedo, y la venganza, la máscara de la cobardía». Sus oyentes tampoco podrían olvidar sus palabras benditas, cuando declaró: «El hombre es el hijo de Dios, y no un hijo del diablo».

\par 
%\textsuperscript{(1632.4)}
\textsuperscript{145:3.5} Poco después de la puesta del Sol, mientras Jesús y los apóstoles permanecían todavía alrededor de la mesa de la cena, la esposa de Pedro escuchó voces en el patio delantero y, al acercarse a la puerta, vio que se estaba congregando un gran número de enfermos, y que el camino de Cafarnaúm estaba atestado de gente que venía a buscar la curación de manos de Jesús. Al contemplar este espectáculo, fue inmediatamente a informar a su marido, el cual se lo dijo a Jesús.

\par 
%\textsuperscript{(1632.5)}
\textsuperscript{145:3.6} Cuando el Maestro salió a la entrada principal de la casa de Zebedeo, sus ojos se encontraron con una masa humana aquejada y afligida. Contempló a casi mil seres humanos enfermos y doloridos; al menos éste era el número de personas reunidas delante de él. Pero no todos los presentes estaban afligidos; algunos habían venido para ayudar a sus seres queridos en este esfuerzo por conseguir la curación.

\par 
%\textsuperscript{(1632.6)}
\textsuperscript{145:3.7} El espectáculo de estos mortales afligidos, hombres, mujeres y niños, que sufrían en gran parte a consecuencia de las equivocaciones y transgresiones de sus propios Hijos a quienes había confiado la administración del universo, conmovió particularmente el corazón humano de Jesús y puso a prueba la misericordia divina de este benévolo Hijo Creador. Pero Jesús sabía bien que nunca podría construir un movimiento espiritual duradero sobre la base de unos prodigios puramente materiales. Había seguido la conducta permanente de abstenerse de exhibir sus prerrogativas de creador. Lo sobrenatural o lo milagroso no habían acompañado a su enseñanza desde el episodio de Caná; sin embargo, esta multitud afligida conmovió su corazón compasivo y apeló poderosamente a su afecto comprensivo.

\par 
%\textsuperscript{(1632.7)}
\textsuperscript{145:3.8} Una voz procedente del patio delantero exclamó: «Maestro, pronuncia la palabra, devuélvenos la salud, cura nuestras enfermedades y salva nuestras almas». Apenas se habían pronunciado estas palabras cuando una inmensa comitiva de serafines, controladores físicos, Portadores de Vida e intermedios, que siempre acompañaban a este Creador encarnado de un universo, se prepararon para actuar con poder creativo si su Soberano daba la señal. Éste fue uno de esos momentos, en la carrera terrestre de Jesús, en los que la sabiduría divina y la compasión humana estaban tan entrelazadas en el juicio del Hijo del Hombre, que buscó refugio recurriendo a la voluntad de su Padre.

\par 
%\textsuperscript{(1632.8)}
\textsuperscript{145:3.9} Cuando Pedro imploró al Maestro que atendiera aquellas peticiones de ayuda, Jesús paseó su mirada sobre la muchedumbre de afligidos, y contestó: «He venido al mundo para revelar al Padre y establecer su reino. He vivido mi vida hasta este momento con esa finalidad. Por lo tanto, si fuera la voluntad de Aquel que me ha enviado, y si no es incompatible con mi dedicación a proclamar el evangelio del reino de los cielos, desearía que mis hijos se curaran... y..». pero las demás palabras de Jesús se perdieron en el alboroto.

\par 
%\textsuperscript{(1633.1)}
\textsuperscript{145:3.10} Jesús había transferido la responsabilidad de esta decisión curativa a la autoridad de su Padre. Es evidente que la voluntad del Padre no interpuso ninguna objeción, pues apenas había pronunciado el Maestro estas palabras, el conjunto de personalidades celestiales que servían bajo las órdenes del Ajustador del Pensamiento Personalizado de Jesús se puso poderosamente en movimiento. La enorme comitiva descendió en medio de aquella multitud abigarrada de mortales afligidos, y en unos instantes, 683 hombres, mujeres y niños recuperaron la salud, fueron perfectamente curados de todas sus enfermedades físicas y de otros desórdenes materiales\footnote{\textit{La curación a la puesta de sol}: Mt 8:6b; Mc 1:34; Lc 4:40-41.}. Una escena semejante no se había visto nunca en la Tierra antes de aquel día, ni tampoco después. Para aquellos de nosotros que estaban presentes y contemplaron esta oleada creativa de curaciones, fue en verdad un espectáculo conmovedor.

\par 
%\textsuperscript{(1633.2)}
\textsuperscript{145:3.11} Pero de todos los seres que se quedaron asombrados con esta explosión repentina e inesperada de curación sobrenatural, Jesús era el más sorprendido. En el instante en que su interés y su compasión humanos estaban centrados en la escena de sufrimiento y aflicción desplegada allí ante sus ojos, olvidó tener en cuenta en su mente humana las advertencias exhortatorias de su Ajustador Personalizado; éste le había advertido que, bajo ciertas condiciones y en ciertas circunstancias, era imposible limitar el elemento tiempo en las prerrogativas creadoras de un Hijo Creador. Jesús deseaba que estos mortales que sufrían se curaran, si no se infringía con ello la voluntad de su Padre. El Ajustador Personalizado de Jesús decidió instantáneamente que un acto así de energía creativa no transgrediría en aquel momento la voluntad del Padre Paradisiaco; con esta decisión ---y teniendo en cuenta que Jesús había expresado previamente el deseo curativo--- el acto creativo \textit{existió}. Aquello que un \textit{Hijo Creador} desea y su Padre lo \textit{quiere}, EXISTE. Una curación física y masiva de mortales como ésta no volvió a producirse en toda la vida posterior de Jesús en la Tierra.

\par 
%\textsuperscript{(1633.3)}
\textsuperscript{145:3.12} Como era de esperar, la noticia de esta curación a la puesta del Sol, en Betsaida de Cafarnaúm, se difundió por toda Galilea y Judea, y por regiones más lejanas. Los temores de Herodes se despertaron una vez más; envió a unos observadores para que le informaran sobre la obra y las enseñanzas de Jesús, y para que averiguaran si se trataba del antiguo carpintero de Nazaret o de Juan el Bautista resucitado de entre los muertos.

\par 
%\textsuperscript{(1633.4)}
\textsuperscript{145:3.13} Durante el resto de su carrera terrestre, y a causa principalmente de esta demostración involuntaria de curación física, Jesús se convirtió en lo sucesivo tanto en médico como en predicador. Es cierto que continuó enseñando, pero su trabajo personal consistía sobre todo en ayudar a los enfermos y a los afligidos, mientras que sus apóstoles se ocupaban de predicar en público y de bautizar a los creyentes.

\par 
%\textsuperscript{(1633.5)}
\textsuperscript{145:3.14} Pero la mayoría de los que recibieron la curación física sobrenatural, o creativa, durante esta demostración de energía divina después de ponerse el Sol, no obtuvieron un beneficio espiritual permanente de esta extraordinaria manifestación de misericordia. Unos pocos fueron edificados realmente gracias a este ministerio físico, pero esta asombrosa erupción de curación creativa, independiente del tiempo, no hizo avanzar el reino espiritual en el corazón de los hombres.

\par 
%\textsuperscript{(1633.6)}
\textsuperscript{145:3.15} Las curaciones milagrosas que acompañaron de vez en cuando la misión de Jesús en la Tierra no formaban parte de su plan para proclamar el reino. Fueron accidentalmente inherentes a la presencia en la Tierra de un ser divino con unas prerrogativas creadoras casi ilimitadas, en asociación con una combinación sin precedentes de misericordia divina y de compasión humana. Pero estos pretendidos milagros dieron muchos problemas a Jesús, en el sentido de que le proporcionaron una publicidad que ocasionaba prejuicios y le aportaron una notoriedad que no deseaba.

\section*{4. La noche siguiente}
\par 
%\textsuperscript{(1634.1)}
\textsuperscript{145:4.1} Durante toda la noche que siguió a esta gran explosión de curaciones, la multitud alegre y feliz invadió la casa de Zebedeo, y el entusiasmo emotivo de los apóstoles de Jesús alcanzó los niveles más altos. Desde el punto de vista humano, éste fue probablemente el día más grande de todos los días inolvidables de su asociación con Jesús. En ningún momento anterior ni posterior se elevaron sus esperanzas hasta tales alturas de expectativa confiada. Sólo unos días antes, cuando aún se encontraban en el interior de las fronteras de Samaria, Jesús les había dicho que había llegado la hora en que el reino debía ser proclamado con \textit{poderío}, y ahora sus ojos habían contemplado lo que suponían que era la realización de esta promesa. Estaban emocionados con la idea de lo que vendría después si esta asombrosa manifestación de poder curativo no era más que el principio. Habían desterrado sus dudas prolongadas sobre la divinidad de Jesús. Estaban literalmente embriagados con el éxtasis de su aturdido encantamiento.

\par 
%\textsuperscript{(1634.2)}
\textsuperscript{145:4.2} Pero cuando buscaron a Jesús, no pudieron encontrarlo. El Maestro estaba muy perturbado por lo que había sucedido. Estos hombres, mujeres y niños que habían sido curados de diversas enfermedades se quedaron hasta horas avanzadas de la noche, esperando que Jesús regresara para poder expresarle su gratitud. A medida que pasaban las horas y el Maestro permanecía recluido, los apóstoles no podían comprender su conducta; su alegría hubiera sido completa y perfecta si no hubiera sido por esta ausencia continuada. Cuando Jesús regresó entre ellos, ya era tarde, y prácticamente todos los beneficiarios del episodio curativo se habían ido a sus casas. Jesús rehusó las felicitaciones y la adoración de los doce y de los demás que se habían quedado para saludarlo, limitándose a decir: «No os regocijéis porque mi Padre tenga el poder de curar el cuerpo, sino más bien porque tiene la fuerza de salvar el alma. Vamos a descansar, pues mañana tenemos que ocuparnos de los asuntos del Padre».

\par 
%\textsuperscript{(1634.3)}
\textsuperscript{145:4.3} Una vez más, doce hombres decepcionados, perplejos y con el corazón entristecido se fueron a descansar; pocos de ellos, exceptuando a los gemelos, durmieron mucho aquella noche. Tan pronto como el Maestro hacía algo que alegraba el alma y regocijaba el corazón de sus apóstoles, parecía que inmediatamente hacía añicos sus esperanzas y demolía completamente los fundamentos de su coraje y entusiasmo. Cuando estos pescadores desconcertados se miraban entre sí a los ojos, sólo tenían un pensamiento: «No podemos comprenderlo. ¿Qué significa todo esto?»

\section*{5. El domingo por la mañana temprano}
\par 
%\textsuperscript{(1634.4)}
\textsuperscript{145:5.1} Jesús tampoco durmió mucho aquel sábado por la noche. Se dio cuenta de que el mundo estaba lleno de sufrimiento físico y repleto de dificultades materiales. Meditaba sobre el grave peligro de verse obligado a consagrar tal cantidad de su tiempo al cuidado de los enfermos y afligidos, que su misión de establecer el reino espiritual en el corazón de los hombres se vería obstaculizada por el ministerio de las cosas físicas, o al menos subordinada a dicho ministerio. Debido a que estos pensamientos y otros similares ocuparon la mente mortal de Jesús durante la noche, aquel domingo por la mañana se levantó mucho antes del amanecer y se fue solo\footnote{\textit{El retiro de Jesús}: Mc 1:35; Lc 4:42a.} a uno de sus lugares favoritos para comulgar con el Padre. En esta mañana temprano, Jesús escogió como tema de oración la sabiduría y el juicio para impedir que su compasión humana, unida a su misericordia divina, se sintieran tan influidas en presencia del sufrimiento humano, que todo su tiempo estuviera ocupado con el ministerio físico, descuidando el ministerio espiritual. Aunque no deseaba evitar por completo ayudar a los enfermos, sabía que también tenía que hacer un trabajo más importante, el de la enseñanza espiritual y la educación religiosa.

\par 
%\textsuperscript{(1635.1)}
\textsuperscript{145:5.2} Jesús salía tan a menudo a orar en las colinas porque no había habitaciones privadas donde poder llevar a cabo sus devociones personales.

\par 
%\textsuperscript{(1635.2)}
\textsuperscript{145:5.3} Pedro no pudo dormir aquella noche; por eso, poco después de que Jesús se hubiera ido a orar, despertó muy temprano a Santiago y a Juan, y los tres salieron para buscar a su Maestro\footnote{\textit{Buscando a Jesús}: Mc 1:36-37a.}. Después de buscarlo durante más de una hora, encontraron a Jesús y le suplicaron que les contara la razón de su extraña conducta. Deseaban saber por qué parecía estar disgustado por la poderosa efusión del espíritu de curación, cuando toda la gente estaba encantada y sus apóstoles tan llenos de alegría.

\par 
%\textsuperscript{(1635.3)}
\textsuperscript{145:5.4} Durante más de cuatro horas, Jesús se esforzó por explicar a estos tres apóstoles lo que había sucedido. Les enseñó lo que había acontecido y les explicó los peligros de este tipo de manifestaciones. Jesús les confió el motivo por el que había salido a orar. Intentó indicar claramente a sus asociados personales las verdaderas razones por las cuales el reino del Padre no se podía construir sobre la realización de prodigios y las curaciones físicas. Pero no podían comprender su enseñanza.

\par 
%\textsuperscript{(1635.4)}
\textsuperscript{145:5.5} Mientras tanto, el domingo por la mañana temprano, otra multitud de almas afligidas y muchos curiosos empezaron a congregarse alrededor de la casa de Zebedeo. Gritaban que querían ver a Jesús. Andrés y los apóstoles estaban tan perplejos que, mientras Simón Celotes hablaba a la asamblea, Andrés salió a buscar a Jesús con algunos de sus compañeros. Cuando hubo localizado a Jesús en compañía de los tres, Andrés dijo: «Maestro, ¿por qué nos dejas solos con la multitud? Mira, todo el mundo te busca; nunca han buscado antes tantas personas tu enseñanza. En este mismo momento, la casa está rodeada de gente que ha venido de cerca y de lejos a causa de tus obras poderosas. ¿No vas a volver con nosotros para aportarles tu ministerio?»\footnote{\textit{La multitud busca a Jesús para curarse}: Mc 1:37b; Lc 4:42b.}

\par 
%\textsuperscript{(1635.5)}
\textsuperscript{145:5.6} Cuando Jesús escuchó esto, contestó: «Andrés, ¿no te he enseñado a ti y a los demás que mi misión en la Tierra es revelar al Padre, y que mi mensaje es proclamar el reino de los cielos? ¿Entonces cómo puede ser que quieras que me desvíe de mi trabajo para contentar a los curiosos y satisfacer a los que buscan signos y prodigios? ¿No hemos estado entre esa gente todos estos meses? ¿Y se han congregado en multitudes para escuchar la buena nueva del reino? ¿Por qué vienen ahora a acosarnos? ¿No es para buscar la curación de su cuerpo físico, en vez de venir porque han recibido la verdad espiritual para la salvación de su alma? Cuando los hombres se sienten atraídos hacia nosotros a causa de las manifestaciones extraordinarias, muchos no vienen buscando la verdad y la salvación sino más bien la curación de sus dolencias físicas, y para conseguir la liberación de sus dificultades materiales».

\par 
%\textsuperscript{(1635.6)}
\textsuperscript{145:5.7} «Todo este tiempo he estado en Cafarnaúm, y tanto en la sinagoga como al lado del mar, he proclamado la buena nueva del reino a todos los que tenían oídos para oír y un corazón para recibir la verdad. No es voluntad de mi Padre que vuelva con vosotros para entretener a esos curiosos y dedicarme al ministerio de las cosas materiales, con exclusión de las espirituales. Os he ordenado para que prediquéis el evangelio y ayudéis a los enfermos, pero no debo dejarme absorber por las curaciones, dejando de lado mi enseñanza. No, Andrés, no voy a volver con vosotros. Id y decidle a la gente que crean en lo que les hemos enseñado, y que se regocijen en la libertad de los hijos de Dios. Y preparaos para nuestra partida hacia las otras ciudades de Galilea, donde el camino ya ha sido preparado para la predicación de la buena nueva del reino. Ésta es la finalidad para la que he venido desde donde se encuentra el Padre. Así pues, id y preparad nuestra partida inmediata, mientras espero aquí vuestro regreso»\footnote{\textit{Jesús viene a predicar, no a curar}: Mc 1:38; Lc 4:43.}.

\par 
%\textsuperscript{(1636.1)}
\textsuperscript{145:5.8} Cuando Jesús hubo hablado, Andrés y sus compañeros apóstoles emprendieron tristemente el camino de vuelta a la casa de Zebedeo, despidieron a la multitud reunida y se prepararon rápidamente para el viaje, como Jesús les había ordenado. Así pues, el domingo por la tarde 18 de enero del año 28, Jesús y los apóstoles empezaron su primera gira de predicación realmente pública y manifiesta en las ciudades de Galilea. Durante este primer periplo, predicaron el evangelio del reino en muchas ciudades, pero no visitaron Nazaret\footnote{\textit{Comienzo de la primera gira de predicación}: Mt 4:23; 9:35b; Mc 1:39; Lc 4:44.}.

\par 
%\textsuperscript{(1636.2)}
\textsuperscript{145:5.9} Aquel domingo por la tarde, poco después de que Jesús y sus apóstoles hubieran salido para Rimón, sus hermanos Santiago y Judá se presentaron en la casa de Zebedeo para verlo. Hacia el mediodía de aquel día, Judá había buscado por todas partes a su hermano Santiago y le había insistido para que fueran a ver a Jesús. Pero cuando Santiago consintió por fin en acompañar a Judá, Jesús ya se había marchado.

\par 
%\textsuperscript{(1636.3)}
\textsuperscript{145:5.10} Los apóstoles eran reacios a abandonar el gran interés que se había despertado en Cafarnaúm. Pedro calculó que no menos de mil creyentes podían haber sido bautizados en el reino. Jesús los escuchó con paciencia, pero no consintió en volver. Durante un rato prevaleció el silencio, y luego Tomás se dirigió a sus compañeros apóstoles diciendo: «¡Vamos! El Maestro ha hablado. No importa que no podamos comprender plenamente los misterios del reino de los cielos, pues de una cosa estamos seguros: Seguimos a un instructor que no busca ninguna gloria para sí mismo». Y, a regañadientes, salieron a predicar la buena nueva en las ciudades de Galilea.