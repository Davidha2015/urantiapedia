\chapter{Documento 152. Los acontecimientos que condujeron a la crisis de Cafarnaúm}
\par 
%\textsuperscript{(1698.1)}
\textsuperscript{152:0.1} LA HISTORIA de la curación de Amós, el lunático de Jeresa, ya había llegado hasta Betsaida y Cafarnaúm, de manera que una gran multitud esperaba a Jesús cuando su barca arribó aquel martes por la mañana. En esta multitud se encontraban los nuevos observadores enviados por el sanedrín de Jerusalén, que habían bajado a Cafarnaúm con el fin de encontrar un pretexto para arrestar e inculpar al Maestro. Mientras Jesús hablaba con la gente que se había reunido para saludarle, Jairo, uno de los jefes de la sinagoga, se abrió paso entre la muchedumbre, cayó a sus pies, lo cogió de la mano y le suplicó que se apresurara a ir con él, diciendo: «Maestro, mi hijita, mi única hija, yace en mi casa a punto de morir. Te ruego que vengas a curarla». Cuando Jesús escuchó la petición de este padre, dijo: «Iré contigo».

\par 
%\textsuperscript{(1698.2)}
\textsuperscript{152:0.2} Mientras Jesús acompañaba a Jairo, la gran multitud, que había escuchado la súplica del padre, los siguió para ver qué iba a suceder. Poco antes de llegar a la casa del jefe, mientras pasaban rápidamente por una calle estrecha con la muchedumbre empujándolo, Jesús se detuvo de pronto y exclamó: «Alguien me ha tocado». Y cuando aquellos que estaban cerca de él negaron haberle tocado, Pedro dijo: «Maestro, puedes ver que este gentío te apretuja, amenaza con aplastarnos, y sin embargo dices que `alguien me ha tocado'. ¿Qué quieres decir?» Entonces Jesús dijo: «He preguntado quién me ha tocado, porque he percibido que una energía viviente ha salido de mí». Jesús miró a su alrededor, y sus ojos se posaron en una mujer cercana, que se adelantó, se arrodilló a sus pies y dijo: «Durante años he estado afligida con una hemorragia mortificante. Muchos médicos me han hecho sufrir mucho; he gastado todos mis bienes, pero ninguno ha podido curarme. Entonces oí hablar de ti, y pensé que si pudiera tocar solamente el borde de tu manto, seguramente me curaría. Así pues, apreté el paso con la gente a medida que caminaban hasta que, al estar cerca de ti, Maestro, he tocado el borde de tu manto, y he recuperado la salud; sé que me he curado de mi aflicción»\footnote{\textit{Curación de la mujer hemorroísa}: Mt 9:20-21; Mc 5:25-33; Lc 8:43-47.}.

\par 
%\textsuperscript{(1698.3)}
\textsuperscript{152:0.3} Cuando Jesús escuchó esto, cogió a la mujer de la mano, la levantó y le dijo: «Hija, tu fe te ha curado; ve en paz». Era su \textit{fe}, y no su \textit{contacto}, lo que la había curado\footnote{\textit{La fe la había curado}: Mt 9:22; Mc 5:34; Lc 8:48.}. Este caso es un buen ejemplo de las muchas curaciones aparentemente milagrosas que acompañaron la carrera terrestre de Jesús, pero que él, en ningún sentido, deseó conscientemente. El paso del tiempo demostró que esta mujer se había curado realmente de su enfermedad. Su fe era del tipo que atrapaba directamente el poder creativo que residía en la persona del Maestro. Con la fe que tenía, sólo necesitaba acercarse a la persona del Maestro. No era necesario en absoluto que tocara su manto; eso era simplemente la parte supersticiosa de su creencia. Jesús llamó a su presencia a esta mujer de Cesarea de Filipo, llamada Verónica, para corregir dos errores que podrían haber permanecido en su mente, o que podrían haber perdurado en la mente de los que habían presenciado esta curación: No quería que Verónica se marchara pensando que su miedo por intentar robar su curación había sido premiado, o que su superstición de asociar el toque del vestido de Jesús con su curación había sido eficaz. Deseaba que todos supieran que era su \textit{fe} pura y viviente la que había efectuado la curación.

\section*{1. En la casa de Jairo}
\par 
%\textsuperscript{(1699.1)}
\textsuperscript{152:1.1} Jairo estaba, por supuesto, enormemente impaciente por esta demora en llegar a su casa; por eso ahora siguieron caminando con paso acelerado. Incluso antes de que entraran en el patio del jefe, uno de sus sirvientes salió diciendo: «No molestes al Maestro; tu hija ha muerto». Pero Jesús no pareció prestar atención a las palabras del sirviente, porque, llevándose consigo a Pedro, Santiago y Juan, se volvió hacia el padre desconsolado y le dijo: «No temas; limítate a creer». Cuando entró en la casa, encontró que los flautistas ya estaban allí con las plañideras formando un alboroto indecente; los parientes ya se habían puesto a llorar y a lamentarse. Después de echar a todas las plañideras de la habitación, entró con el padre, la madre y sus tres apóstoles. Había dicho a las plañideras que la doncella no estaba muerta, pero se rieron de él con desprecio. Jesús se volvió entonces hacia la madre, diciéndole: «Tu hija no está muerta; sólo está dormida». Cuando la casa recuperó la tranquilidad, Jesús se acercó al lecho de la niña, la cogió de la mano y le dijo: «Hija, yo te lo digo, ¡despierta y levántate!» Cuando la chica escuchó estas palabras, se levantó inmediatamente y caminó por la habitación. Luego, cuando se hubo recuperado de su aturdimiento, Jesús ordenó que le dieran algo de comer, pues había estado mucho tiempo sin tomar alimento\footnote{\textit{Curación de la hija de Jairo}: Mt 9:23-25; Mc 5:35-43; Lc 8:49-55.}.

\par 
%\textsuperscript{(1699.2)}
\textsuperscript{152:1.2} Como había mucha agitación en Cafarnaúm en contra de Jesús, éste reunió a la familia y les explicó que la joven había caído en un estado de coma después de una fiebre prolongada, y que él se había limitado a despertarla, que no la había resucitado de entre los muertos. También explicó todo esto a sus apóstoles, pero fue en vano; todos creían que había resucitado a la chiquilla de entre los muertos. Todo lo que Jesús decía para explicar muchos de estos milagros aparentes, tenía poco efecto sobre sus seguidores. Eran propensos a ver milagros, y no perdían ni una oportunidad para atribuirle un nuevo prodigio a Jesús. Jesús y los apóstoles regresaron a Betsaida, después de haber encargado específicamente a todos que no se lo contaran a nadie\footnote{\textit{No se lo digáis a nadie}: Mc 5:43; Lc 8:56.}.

\par 
%\textsuperscript{(1699.3)}
\textsuperscript{152:1.3} Cuando salió de la casa de Jairo, dos ciegos, guiados por un niño mudo, lo siguieron dando gritos para que los curara\footnote{\textit{Los ciegos pidiendo curación}: Mt 9:27.}. Aproximadamente por esta época, la reputación de Jesús como sanador estaba en su apogeo. Por todas partes donde iba, los enfermos y los afligidos lo estaban esperando\footnote{\textit{La reputación como sanador de Jesús}: Mt 9:26.}. El Maestro parecía ahora muy cansado, y todos sus amigos empezaban a preocuparse, pues si continuaba con su labor de enseñanza y de curación, acabaría por desplomarse.

\par 
%\textsuperscript{(1699.4)}
\textsuperscript{152:1.4} Los apóstoles de Jesús, sin contar a la gente común y corriente, no podían comprender la naturaleza y los atributos de este Dios-hombre. Ninguna generación posterior tampoco ha sido capaz de evaluar lo que sucedió en la Tierra en la persona de Jesús de Nazaret. Y la ciencia o la religión nunca tendrán la oportunidad de examinar estos acontecimientos notables, por la sencilla razón de que una situación así de extraordinaria no volverá a producirse nunca más en este mundo ni en ningún otro mundo de Nebadon. Nunca más volverá a aparecer, en ningún mundo de todo este universo, un ser en la similitud de la carne mortal que incorpore al mismo tiempo todos los atributos de la energía creativa, combinados con los dones espirituales que trascienden el tiempo y la mayoría de las otras limitaciones materiales.

\par 
%\textsuperscript{(1700.1)}
\textsuperscript{152:1.5} Antes de que Jesús estuviera en la Tierra, o después de entonces, nunca ha sido posible obtener de manera tan directa y gráfica los resultados que acompañan a la fe sólida y viviente de los hombres y las mujeres mortales. Para repetir estos fenómenos, tendríamos que ir a la presencia inmediata de Miguel, el Creador, y encontrarlo tal como era en aquella época ---el Hijo del Hombre. Asimismo, aunque su ausencia impide que estas manifestaciones materiales se produzcan hoy en día, deberíais absteneros de fijar cualquier tipo de limitación a la posible manifestación de su \textit{poder espiritual}. Aunque el Maestro está ausente como ser material, se encuentra presente como influencia espiritual en el corazón de los hombres. Al marcharse de este mundo, Jesús ha hecho posible que su espíritu viva al lado del de su Padre, que reside en la mente de todo el género humano.

\section*{2. La alimentación de los cinco mil}
\par 
%\textsuperscript{(1700.2)}
\textsuperscript{152:2.1} Jesús continuó enseñando a la gente durante el día, e instruyendo a los apóstoles y a los evangelistas por la noche. El viernes decretó una semana de vacaciones para que todos sus seguidores pudieran pasar unos días en sus casas o con sus amigos, antes de prepararse a subir a Jerusalén para la Pascua. Pero más de la mitad de sus discípulos se negaron a abandonarlo, y la multitud aumentaba diariamente hasta tal punto que David Zebedeo deseaba establecer un nuevo campamento, pero Jesús se negó a darle su consentimiento. El Maestro había descansado tan poco durante el sábado, que el domingo 27 de marzo por la mañana intentó alejarse de la gente. Algunos evangelistas se quedaron allí para hablarle a la multitud, mientras que Jesús y los doce planeaban escaparse, sin ser vistos, a la orilla opuesta del lago, donde pensaban encontrar el descanso que tanto necesitaban en un hermoso parque al sur de Betsaida-Julias. Esta región era un lugar de recreo favorito para los habitantes de Cafarnaúm; todos conocían bien estos parques de la costa oriental.

\par 
%\textsuperscript{(1700.3)}
\textsuperscript{152:2.2} Pero la gente no les dejó salirse con la suya. Vieron la dirección que tomaba la barca de Jesús, alquilaron todas las embarcaciones disponibles y salieron en su persecución\footnote{\textit{La gente les sigue}: Mt 14:13b-14; Mc 6:33-34; Lc 9:11; Jn 6:2.}. Los que no pudieron conseguir una barca se pusieron en camino para rodear a pie el extremo septentrional del lago.

\par 
%\textsuperscript{(1700.4)}
\textsuperscript{152:2.3} Al caer la tarde, más de mil personas habían localizado al Maestro en uno de los parques; él les habló brevemente, y Pedro lo hizo después. Mucha de esta gente había traído su comida, y después de cenar, se reunieron en pequeños grupos mientras los apóstoles y los discípulos de Jesús les enseñaban.

\par 
%\textsuperscript{(1700.5)}
\textsuperscript{152:2.4} El lunes por la tarde, la multitud había aumentado a más de tres mil personas. Y además ---ya entrada la noche--- la gente continuaba afluyendo, trayendo con ellos todo tipo de enfermos. Cientos de personas interesadas habían planeado detenerse en Cafarnaúm, en su camino hacia la Pascua, para ver y escuchar a Jesús, y se negaban sencillamente a sufrir un desengaño. El miércoles a mediodía, unos cinco mil hombres, mujeres y niños se habían congregado aquí, en este parque al sur de Betsaida-Julias. El tiempo era agradable, pues se acercaba el final de la estación de las lluvias en esta región.

\par 
%\textsuperscript{(1700.6)}
\textsuperscript{152:2.5} Felipe había traído provisiones para alimentar a Jesús y a los doce durante tres días, y estaban al cuidado del joven Marcos, su recadero. Este día por la tarde, el tercero para casi la mitad de esta multitud, los víveres que la gente había traído consigo estaban a punto de agotarse. David Zebedeo no contaba aquí con una ciudad de tiendas para alimentar y alojar a las multitudes. Felipe tampoco había previsto alimentos para una muchedumbre tan grande. Pero aunque la gente tenía hambre, no quería irse. Se cuchicheaba en voz baja que, como Jesús deseaba evitar dificultades tanto con Herodes como con los dirigentes de Jerusalén, había elegido este sitio tranquilo, fuera de la jurisdicción de todos sus enemigos, como el lugar adecuado para ser coronado rey. El entusiasmo de la gente aumentaba de hora en hora. A Jesús no le decían ni una palabra, aunque, por supuesto, sabía todo lo que estaba pasando. Incluso los doce apóstoles también estaban contaminados con estas ideas, y en especial los evangelistas más jóvenes. Los apóstoles que estaban a favor de esta tentativa para proclamar rey a Jesús eran Pedro, Juan, Simón Celotes y Judas Iscariote. Andrés, Santiago, Natanael y Tomás se oponían a este proyecto. Mateo, Felipe y los gemelos Alfeo no opinaban. El cabecilla de esta conspiración para hacerlo rey era Joab, uno de los jóvenes evangelistas.

\par 
%\textsuperscript{(1701.1)}
\textsuperscript{152:2.6} Ésta era la situación el miércoles hacia las cinco de la tarde, cuando Jesús le pidió a Santiago Alfeo que llamara a Andrés y a Felipe. Jesús dijo: «¿Qué vamos a hacer con la multitud? Hace ya tres días que están con nosotros, y muchos de ellos tienen hambre. No tienen comida»\footnote{\textit{No hay comida para la multitud}: Mt 14:15-16; Mc 6:35-37; Lc 9:12; Jn 6:5-7.}. Felipe y Andrés intercambiaron una mirada, y luego Felipe contestó: «Maestro, deberías despedir a esta gente para que fueran a los pueblos de los alrededores a comprar comida». Andrés temía que se materializara la intriga para coronarlo rey, por lo que apoyó rápidamente a Felipe, diciendo: «Sí, Maestro, creo que es mejor que despidas a la multitud para que se vayan por su camino y compren comida, y así consigues descansar algún tiempo». Mientras tanto, otros apóstoles se habían unido a la conversación. Jesús dijo entonces: «Pero no deseo despedirlos hambrientos; ¿no podéis alimentarlos?» Esto fue demasiado para Felipe, que dijo inmediatamente: «Maestro, aquí en pleno campo, ¿dónde podemos comprar pan para esta multitud? Con doscientos denarios no tendríamos suficiente para un almuerzo».

\par 
%\textsuperscript{(1701.2)}
\textsuperscript{152:2.7} Antes de que los apóstoles tuvieran la posibilidad de expresarse, Jesús se volvió hacia Andrés y Felipe, diciendo: «No quiero despedir a esta gente. Están aquí como ovejas sin pastor. Me gustaría alimentarlos. ¿De cuánta comida disponemos?» Mientras Felipe conversaba con Mateo y Judas, Andrés buscó al joven Marcos para averiguar cuántas provisiones quedaban. Volvió hacia Jesús, diciendo: «Al muchacho sólo le quedan cinco panes de cebada y dos pescados secos»\footnote{\textit{Cinco panes y dos peces}: Mt 14:17; Mc 6:38; Lc 9:13; Jn 6:8-10.} ---y Pedro añadió inmediatamente: «Y aún tenemos que comer esta noche».

\par 
%\textsuperscript{(1701.3)}
\textsuperscript{152:2.8} Jesús permaneció en silencio durante un momento. Había en sus ojos una mirada lejana. Los apóstoles no decían nada. Jesús se volvió repentinamente hacia Andrés y dijo: «Tráeme los panes y los peces». Cuando Andrés le trajo la canasta, el Maestro dijo: «Ordenad a la gente que se siente en la hierba en grupos de cien, y que designen a un jefe para cada grupo, mientras traéis a todos los evangelistas aquí con nosotros»\footnote{\textit{Traed la comida y sentad a la multitud}: Mt 14:18-19a; Mc 6:39-40; Lc 9:14-15; Jn 6:10.}.

\par 
%\textsuperscript{(1701.4)}
\textsuperscript{152:2.9} Jesús cogió los panes en sus manos y, después de dar las gracias, partió el pan y lo dio a sus apóstoles, que lo pasaron a sus compañeros, quienes a su vez lo llevaron a la multitud. Jesús partió y distribuyó los peces de la misma manera. Y aquella multitud comió hasta saciarse. Cuando hubieron terminado de comer, Jesús dijo a los discípulos: «Recoged los trozos que quedan para que no se pierda nada». Cuando terminaron de recoger los pedazos, tenían doce canastas llenas. Unos cinco mil hombres, mujeres y niños habían comido en este banquete extraordinario\footnote{\textit{Alimentación de 4.000}: Mt 15:32-38; Mc 8:1-9. \textit{Alimentación de 5.000}: Mt 14:19b-21; Mc 6:41-44; Lc 9:16-17; Jn 6:11-13.}.

\par 
%\textsuperscript{(1702.1)}
\textsuperscript{152:2.10} Éste fue el primero y el único milagro natural que Jesús efectuó después de haberlo planeado conscientemente. Es verdad que sus discípulos tenían tendencia a calificar de milagros muchas cosas que no lo eran, pero éste fue un auténtico ministerio sobrenatural. Se nos ha enseñado que, en este caso, Miguel multiplicó los elementos nutritivos como siempre lo hace, salvo que eliminó el factor tiempo y el encauzamiento vital observable.

\section*{3. El episodio de la coronación}
\par 
%\textsuperscript{(1702.2)}
\textsuperscript{152:3.1} La alimentación de los cinco mil por medio de la energía sobrenatural fue otro de esos casos en los que la compasión humana unida al poder creativo dieron como resultado lo que sucedió. Ahora que la multitud había sido saciada, y puesto que la fama de Jesús había aumentado aquí y ahora debido a este prodigio asombroso, el proyecto de apoderarse del Maestro y proclamarlo rey ya no necesitaba la dirección de nadie. La idea pareció propagarse entre la muchedumbre como un contagio. La reacción de la multitud ante esta satisfacción repentina y espectacular de sus necesidades físicas fue profunda e irresistible. A los judíos se les había enseñado durante mucho tiempo que cuando viniera el Mesías, el hijo de David, haría que la leche y la miel fluyeran de nuevo por la tierra, y que el pan de la vida les sería otorgado, tal como se suponía que el maná del cielo había caído sobre sus antepasados en el desierto. Todas estas expectativas, ¿no se habían cumplido ahora precisamente delante de sus ojos? Cuando esta multitud hambrienta y desnutrida hubo terminado de saciarse con el alimento milagroso, sólo tuvo una reacción unánime: «Éste es nuestro rey»\footnote{\textit{«Es nuestro rey»}: Jn 6:14-15.}. El libertador de Israel, obrador de prodigios, había llegado. A los ojos de esta gente sencilla, el poder de alimentar llevaba consigo el derecho a gobernar. Así pues, no es de extrañar que en cuanto la multitud hubo terminado de comer opíparamente, se levantara como un solo hombre, vociferando: «¡Hacedlo rey!»

\par 
%\textsuperscript{(1702.3)}
\textsuperscript{152:3.2} Este griterío poderoso entusiasmó a Pedro y a aquellos apóstoles que aún conservaban la esperanza de que Jesús afirmara su derecho a gobernar. Pero estas falsas esperanzas no iban a durar mucho tiempo. Apenas había dejado de resonar este poderoso griterío de la multitud en las rocas cercanas, cuando Jesús subió a una enorme piedra, levantó su mano derecha para atraer la atención, y dijo: «Hijos míos, vuestras intenciones son buenas, pero tenéis la vista corta y tendencias materialistas»\footnote{\textit{Por razones erróneas}: Jn 6:26.}. Hubo una breve pausa; este fornido galileo estaba allí plantado de manera majestuosa en el resplandor encantador de aquel crepúsculo oriental. Parecía un rey de pies a cabeza mientras continuó hablándole a esta multitud que retenía el aliento: «Queréis hacerme rey, no porque vuestras almas hayan sido iluminadas por una gran verdad, sino porque vuestros estómagos han sido llenados de pan. ¿Cuántas veces os he dicho que mi reino no es de este mundo?\footnote{\textit{Mi reino no es de este mundo}: Jn 18:36.} El reino de los cielos que nosotros proclamamos es una fraternidad espiritual, y ningún hombre lo gobierna sentado en un trono material. Mi Padre que está en los cielos es el Soberano omnisapiente y todopoderoso de esta fraternidad espiritual de los hijos de Dios en la Tierra. ¿De tal manera he fallado en revelaros al Padre de los espíritus que queréis hacer rey a su Hijo en la carne? Ahora iros todos de aquí a vuestras propias casas. Si necesitáis a un rey, que el Padre de las luces\footnote{\textit{Padre de las luces}: Stg 1:17.} sea entronizado en el corazón de cada uno de vosotros como Soberano espiritual de todas las cosas».

\par 
%\textsuperscript{(1702.4)}
\textsuperscript{152:3.3} Estas palabras de Jesús despidieron a la multitud atónita y descorazonada. Muchos de los que habían creído en él cambiaron de parecer y a partir de aquel día dejaron de seguirlo. Los apóstoles permanecían mudos, reunidos en silencio alrededor de las doce canastas con los restos de comida; sólo el joven Marcos, el chico de los recados, dijo: «Y se negó a ser nuestro rey». Antes de marcharse para estar solo en las colinas, Jesús se volvió hacia Andrés y le dijo: «Lleva a tus hermanos de regreso a la casa de Zebedeo y reza con ellos, especialmente por tu hermano Simón Pedro»\footnote{\textit{La multitud marcha, Jesús reza}: Mt 14:23; Mc 6:46.}.

\section*{4. La visión nocturna de Simón Pedro}
\par 
%\textsuperscript{(1703.1)}
\textsuperscript{152:4.1} Los apóstoles sin su Maestro ---que los había hecho partir solos--- se montaron en la barca\footnote{\textit{Los discípulos suben a la barca}: Mt 14:22; Mc 6:45; Jn 6:16-17a.} y empezaron a remar en silencio hacia Betsaida, en la orilla occidental del lago. Ninguno de los doce estaba tan abrumado y abatido como Simón Pedro. Apenas si pronunciaron una palabra; todos estaban pensando en el Maestro que se encontraba solo en las colinas. ¿Los había abandonado? Nunca antes los había despedido a todos, negándose a ir con ellos. ¿Qué podía significar todo esto?

\par 
%\textsuperscript{(1703.2)}
\textsuperscript{152:4.2} Se había levantado un fuerte viento contrario que casi les impedía avanzar, y la oscuridad cayó sobre ellos. A medida que pasaban las horas de oscuridad remando penosamente, Pedro, cada vez más cansado, cayó en un profundo sueño de agotamiento. Andrés y Santiago lo pusieron a descansar en el asiento acolchado de la popa de la barca. Mientras los otros apóstoles luchaban contra el viento y las olas, Pedro tuvo un sueño, una visión de Jesús que venía hacia ellos caminando por el mar. Cuando el Maestro pareció pasar cerca de la barca, Pedro gritó: «Sálvanos, Maestro, sálvanos». Los que se encontraban en la parte posterior de la barca le oyeron decir algunas de estas palabras. Mientras esta aparición nocturna continuaba en la mente de Pedro, soñó que Jesús decía: «Tened buen ánimo; soy yo; no temáis». Esto fue como un bálsamo de Galaad para el alma perturbada de Pedro; calmó su espíritu confuso, de manera que (en su sueño) gritó al Maestro: «Señor, si eres tú realmente, ordéname venir y caminar contigo por el agua». Y cuando Pedro empezó a caminar por el agua, las olas turbulentas lo asustaron, y cuando estaba a punto de hundirse, gritó: «Señor, ¡sálvame!» La mayor parte de los doce lo escucharon proferir este grito. Entonces Pedro soñó que Jesús venía a rescatarlo, alargaba su mano, lo agarraba y lo levantaba, diciendo: «Oh, hombre de poca fe, ¿por qué has dudado?»\footnote{\textit{Jesús «camina sobre el agua»}: Mt 14:24-31; Mc 6:47-51a; Jn 6:17b-21.}

\par 
%\textsuperscript{(1703.3)}
\textsuperscript{152:4.3} En conexión con la última parte de su sueño, Pedro se levantó del asiento donde dormía, salió de la barca y cayó realmente al agua. Y se despertó de su sueño en el momento en que Andrés, Santiago y Juan se inclinaban y lo sacaban del mar.

\par 
%\textsuperscript{(1703.4)}
\textsuperscript{152:4.4} Para Pedro esta experiencia siempre fue real. Creía sinceramente que Jesús había venido hacia ellos aquella noche. Sólo convenció parcialmente a Juan Marcos, lo que explica por qué Marcos omitió una parte de la historia en su narración. Lucas, el médico, investigó cuidadosamente este asunto, y concluyó que el episodio era una visión de Pedro; por consiguiente, rehusó incorporar esta historia en el relato que estaba preparando.

\section*{5. De regreso en Betsaida}
\par 
%\textsuperscript{(1703.5)}
\textsuperscript{152:5.1} El jueves por la mañana, antes del amanecer, anclaron su barca cerca de la casa de Zebedeo\footnote{\textit{Llegan a la orilla}: Mt 14:34; Mc 6:53; Jn 6:22-25a.} y procuraron dormir hasta alrededor del mediodía. Andrés fue el primero que se levantó; se fue a dar un paseo cerca del mar, y encontró a Jesús en compañía del chico de los recados, sentado en una piedra al borde del agua. Un gran número de gente y de jóvenes evangelistas pasaron toda la noche y gran parte del día siguiente buscando a Jesús por las colinas orientales; pero poco después de la medianoche, Jesús y el joven Marcos habían partido a pie para rodear el lago y cruzar el río de regreso a Betsaida.

\par 
%\textsuperscript{(1704.1)}
\textsuperscript{152:5.2} De las cinco mil personas que habían sido alimentadas milagrosamente y que, con el estómago lleno y el corazón vacío, habían querido proclamarlo rey, sólo unas quinientas insistieron en seguirlo. Pero antes de que se enteraran de que había regresado a Betsaida, Jesús le pidió a Andrés que congregara a los doce apóstoles y a sus asociados, incluyendo a las mujeres, diciendo: «Deseo hablar con ellos». Cuando todos estuvieron dispuestos, Jesús dijo:

\par 
%\textsuperscript{(1704.2)}
\textsuperscript{152:5.3} «¿Cuánto tiempo seré indulgente con vosotros? ¿Sois todos torpes en comprender espiritualmente y estáis faltos de fe viviente? Todos estos meses os he enseñado las verdades del reino, y sin embargo estáis dominados por los móviles materiales en lugar de estarlo por las consideraciones espirituales. ¿No habéis leído siquiera en las Escrituras el pasaje donde Moisés exhorta a los hijos incrédulos de Israel, diciendo: `No temáis, permaneced tranquilos y contemplad la salvación del Señor'?\footnote{\textit{No temáis, permaneced tranquilos y ved la salvación}: Ex 14:13.} El cantor dijo: `Poned vuestra confianza en el Señor'\footnote{\textit{Poned vuestra confianza en el Señor}: Sal 4:5.}. `Sed pacientes, esperad al Señor y tened buen ánimo\footnote{\textit{Sed pacientes, esperad al Señor}: Sal 27:14.}. Él fortalecerá vuestro corazón'. `Echad vuestra carga sobre el Señor\footnote{\textit{Echad vuestra carga sobre el Señor}: Sal 55:22.}, y él os sostendrá. Confiad en él en todo momento\footnote{\textit{Confiad en él en todo momento}: Sal 62:8.} y desahogaos con él, porque Dios es vuestro refugio'. `El que reside en el lugar secreto del Altísimo\footnote{\textit{Dios habita en lugar secreto}: Sal 91:1.}, permanecerá a la sombra del Todopoderoso'. `Es mejor fiarse del Señor que poner la confianza en los príncipes humanos'\footnote{\textit{Es mejor fiarse del Señor que de los príncipes}: Sal 118:9.}.»

\par 
%\textsuperscript{(1704.3)}
\textsuperscript{152:5.4} «¿Comprendéis todos ahora que la producción de milagros y la ejecución de prodigios materiales no conquistarán almas para el reino espiritual? Hemos alimentado a la multitud, pero eso no los ha inducido a tener hambre del pan de la vida ni sed de las aguas de la rectitud espiritual. Una vez satisfecha su hambre, no trataron de entrar en el reino de los cielos, sino que intentaron proclamar rey al Hijo del Hombre a la manera de los reyes de este mundo, sólo para poder seguir comiendo pan sin tener que trabajar para ganarlo. Todo esto, en lo que muchos de vosotros habéis más o menos participado, no contribuye en nada a revelar el Padre celestial ni a hacer avanzar su reino en la Tierra. ¿No tenemos enemigos suficientes entre los jefes religiosos del país como para hacer lo posible por indisponer también a los gobernantes civiles? Ruego al Padre que unja vuestros ojos para que podáis ver y abra vuestros oídos para que podáis oír, a fin de que tengáis una fe plena en el evangelio que os he enseñado»\footnote{\textit{Jesús discierne los motivos}: Jn 6:26-27.}.

\par 
%\textsuperscript{(1704.4)}
\textsuperscript{152:5.5} Jesús anunció después que deseaba retirarse unos días para descansar con sus apóstoles, antes de que se prepararan a subir a Jerusalén para la Pascua, y a todos los discípulos y a la multitud les prohibió que lo siguieran. En consecuencia, salieron en barca hacia la región de Genesaret para descansar y dormir durante dos o tres días. Jesús se estaba preparando para una gran crisis de su vida en la Tierra, y por esta razón pasó mucho tiempo en comunión con el Padre que está en los cielos.

\par 
%\textsuperscript{(1704.5)}
\textsuperscript{152:5.6} La noticia de la alimentación de los cinco mil y del intento de convertir a Jesús en rey despertó una amplia curiosidad y suscitó los temores de los jefes religiosos y de los gobernantes civiles de toda Galilea y Judea. Este gran milagro no contribuyó en nada a fomentar el evangelio del reino en el alma de los creyentes propensos al materialismo y poco entusiastas, pero sí cumplió el objetivo de poner fin a las tendencias de la familia inmediata de Jesús, compuesta por los apóstoles y los discípulos íntimos, consistentes en buscar milagros y en desear ardientemente un rey. Este episodio espectacular puso fin a la primera época de enseñanza, instrucción y curación, preparando así el camino para la inauguración de este último año de proclamación de las fases superiores y más espirituales del nuevo evangelio del reino ---la filiación divina, la libertad espiritual y la salvación eterna.

\section*{6. En Genesaret}
\par 
%\textsuperscript{(1705.1)}
\textsuperscript{152:6.1} Mientras descansaba en la casa de un rico creyente de la región de Genesaret, Jesús mantuvo conversaciones informales con los doce todas las tardes. Los embajadores del reino formaban un grupo serio, sobrio y escarmentado de hombres desilusionados. Pero incluso después de todo lo que había sucedido, los acontecimientos posteriores revelaron que estos doce hombres no estaban todavía completamente liberados de sus ideas innatas y largo tiempo acariciadas sobre la venida del Mesías judío. Los acontecimientos de algunas semanas antes se habían desarrollado demasiado rápidamente como para que estos pescadores asombrados pudieran comprender todo su significado. Los hombres y las mujeres necesitan tiempo para efectuar cambios radicales y amplios en sus conceptos básicos y fundamentales sobre la conducta social, las actitudes filosóficas y las convicciones religiosas.

\par 
%\textsuperscript{(1705.2)}
\textsuperscript{152:6.2} Mientras Jesús y los doce descansaban en Genesaret, las multitudes se dispersaron; algunos regresaron a sus casas y otros se fueron a Jerusalén para la Pascua. En menos de un mes, los seguidores entusiastas y declarados de Jesús, que ascendían a más de cincuenta mil solamente en Galilea, se redujeron a menos de quinientos. Jesús deseaba que sus apóstoles pasaran por esta experiencia con la inconstancia de las aclamaciones populares, para que no se sintieran tentados a fiarse de estas manifestaciones de histeria religiosa transitoria después de que los hubiera dejado solos con el trabajo del reino; pero sólo consiguió un éxito parcial en este esfuerzo.

\par 
%\textsuperscript{(1705.3)}
\textsuperscript{152:6.3} La segunda noche de su estancia en Genesaret, el Maestro contó de nuevo a los apóstoles la parábola del sembrador y añadió estas palabras: «Ya veis, hijos míos, que recurrir a los sentimientos humanos es transitorio y totalmente decepcionante; apelar exclusivamente al intelecto del hombre es igualmente vacío y estéril; sólo dirigiendo vuestro llamamiento al espíritu que vive dentro de la mente humana, podéis esperar conseguir un éxito duradero y efectuar esas maravillosas transformaciones del carácter humano que pronto se manifiestan mediante la producción abundante de los auténticos frutos del espíritu en la vida diaria de todos aquellos que se encuentran liberados así de las tinieblas de la duda mediante el nacimiento del espíritu en la luz de la fe ---el reino de los cielos».

\par 
%\textsuperscript{(1705.4)}
\textsuperscript{152:6.4} Jesús enseñó el recurso a las emociones como técnica para detener y concentrar la atención intelectual. A esa mente así despierta y avivada la calificó de puerta de entrada al alma, donde reside esa naturaleza espiritual del hombre que debe reconocer la verdad y responder al llamamiento espiritual del evangelio, a fin de producir los resultados permanentes de las verdaderas transformaciones del carácter.

\par 
%\textsuperscript{(1705.5)}
\textsuperscript{152:6.5} Jesús se esforzó así por preparar a los apóstoles para la conmoción inminente ---la crisis de la actitud del público hacia él, que iba a producirse pocos días después. Explicó a los doce que los dirigentes religiosos de Jerusalén conspirarían con Herodes Antipas para destruirlos. Los doce empezaron a comprender más plenamente (aunque no de manera definitiva) que Jesús no iba a sentarse en el trono de David. Percibieron más plenamente que los prodigios materiales no harían progresar la verdad espiritual. Empezaron a darse cuenta de que la alimentación de los cinco mil y el movimiento popular para hacer rey a Jesús fueron el apogeo de las expectativas del pueblo, que buscaba milagros y esperaba prodigios, y el punto culminante de las aclamaciones que Jesús recibía de la plebe. Discernían vagamente y entreveían débilmente los tiempos de la criba espiritual y de la cruel adversidad que se acercaban. Estos doce hombres se despertaban lentamente a la comprensión de la verdadera naturaleza de su tarea como embajadores del reino, y empezaron a prepararse para las pruebas difíciles y severas del último año del ministerio del Maestro en la Tierra.

\par 
%\textsuperscript{(1706.1)}
\textsuperscript{152:6.6} Antes de salir de Genesaret, Jesús les informó respecto a la alimentación milagrosa de los cinco mil, diciéndoles exactamente por qué había emprendido esta manifestación extraordinaria de poder creativo, y también les aseguró que no había cedido a su compasión por la multitud hasta que no hubo averiguado que aquello era «conforme a la voluntad del Padre».

\section*{7. En Jerusalén}
\par 
%\textsuperscript{(1706.2)}
\textsuperscript{152:7.1} El domingo 3 de abril, Jesús partió de Betsaida para dirigirse a Jerusalén, acompañado únicamente por los doce apóstoles. Para evitar las multitudes y atraer el mínimo de atención posible, viajaron por el camino de Gerasa y Filadelfia. Les prohibió que hicieran cualquier tipo de enseñanza pública durante este viaje; tampoco les permitió que enseñaran o predicaran mientras estuvieran en Jerusalén. Llegaron a Betania, cerca de Jerusalén, el miércoles 6 de abril al anochecer. Aquella fue la única noche que se detuvieron en la casa de Lázaro, Marta y María, pues al día siguiente se separaron. Jesús se hospedó con Juan en la casa de un creyente llamado Simón, cerca de la casa de Lázaro en Betania. Judas Iscariote y Simón Celotes se quedaron con unos amigos en Jerusalén, mientras que el resto de los apóstoles residió, de dos en dos, en diferentes hogares.

\par 
%\textsuperscript{(1706.3)}
\textsuperscript{152:7.2} Durante esta Pascua, Jesús sólo entró una vez en Jerusalén, y lo hizo el gran día de la fiesta. Abner llevó a muchos creyentes de Jerusalén para que se reunieran con Jesús en Betania. Durante esta estancia en Jerusalén, los doce aprendieron cuán amargos se estaban volviendo los sentimientos hacia su Maestro. Todos partieron de Jerusalén convencidos de que una crisis era inminente.

\par 
%\textsuperscript{(1706.4)}
\textsuperscript{152:7.3} El domingo 24 de abril, Jesús y los apóstoles salieron de Jerusalén hacia Betsaida, pasando por las ciudades costeras de Jope, Cesarea y Tolemaida. Desde allí fueron por el interior a Ramá y Corazín, llegando a Betsaida el viernes 29 de abril. En cuanto estuvieron en casa, Jesús envió a Andrés a pedirle permiso al jefe de la sinagoga para hablar al día siguiente, sábado, en los oficios de la tarde. Jesús sabía muy bien que ésta era la última vez que le permitirían hablar en la sinagoga de Cafarnaúm.