\chapter{Documento 153. La crisis en Cafarnaúm}
\par
%\textsuperscript{(1707.1)}
\textsuperscript{153:0.1} EL VIERNES por la noche, día de su llegada a Betsaida, y el sábado por la mañana, los apóstoles observaron que Jesús estaba seriamente ocupado en algún problema de gran importancia; se daban cuenta de que el Maestro reflexionaba de manera poco habitual en algún asunto importante. No tomó su desayuno y comió poco al mediodía. Todo el sábado por la mañana y la noche anterior, los doce y sus compañeros se habían reunido en pequeños grupos alrededor de la casa, en el jardín y a lo largo de la playa. Pesaba sobre todos ellos la tensión de la incertidumbre y la ansiedad del temor. Jesús les había dicho poca cosa desde que salieron de Jerusalén.

\par
%\textsuperscript{(1707.2)}
\textsuperscript{153:0.2} Hacía meses que no veían al Maestro tan preocupado y tan poco comunicativo. Incluso Simón Pedro estaba deprimido, si no abatido. Andrés no sabía qué hacer por sus asociados desanimados. Natanael dijo que estaban en medio de la «calma antes de la tormenta». Tomás expresó la opinión de que «algo fuera de lo común está a punto de suceder». Felipe aconsejó a David Zebedeo que «se olvidara de los planes para alimentar y alojar a la multitud, hasta que sepamos en qué está pensando el Maestro». Mateo se ocupaba con renovado esfuerzo en reaprovisionar la tesorería. Santiago y Juan conversaban sobre el próximo sermón en la sinagoga y hacían muchas especulaciones sobre su probable naturaleza y alcance. Simón Celotes expresaba la creencia, en realidad la esperanza, de que «el Padre que está en los cielos puede estar a punto de intervenir de manera inesperada para justificar y sostener a su Hijo», mientras que Judas Iscariote se atrevía a abrigar el pensamiento de que Jesús estaba posiblemente abrumado por los remordimientos, por «no haber tenido el coraje y la osadía de permitir a los cinco mil que lo proclamaran rey de los judíos».

\par
%\textsuperscript{(1707.3)}
\textsuperscript{153:0.3} Aquella hermosa tarde de sábado, Jesús salió de este grupo de seguidores deprimidos y apesadumbrados para predicar su memorable sermón en la sinagoga de Cafarnaúm\footnote{\textit{Predica en la sinagoga}: Jn 6:59.}. Las únicas palabras de saludo jovial o buenos deseos que recibió de sus discípulos inmediatos provinieron de uno de los confiados gemelos Alfeo, que, cuando Jesús salía de la casa camino de la sinagoga, lo saludó alegremente, diciendo: «Oramos para que el Padre te ayude, y para que podamos tener unas multitudes más grandes que nunca».

\section*{1. La preparación del escenario}
\par
%\textsuperscript{(1707.4)}
\textsuperscript{153:1.1} Una asamblea distinguida recibió a Jesús a las tres de la tarde de este precioso sábado en la nueva sinagoga de Cafarnaúm. Jairo presidía y entregó las Escrituras a Jesús para la lectura. El día anterior, cincuenta y tres fariseos y saduceos habían llegado de Jerusalén; también estaban presentes más de treinta jefes y dirigentes de las sinagogas vecinas. Estos jefes religiosos judíos actuaban directamente bajo las órdenes del sanedrín de Jerusalén, y constituían la vanguardia ortodoxa que había venido para iniciar una guerra abierta contra Jesús y sus discípulos. Al lado de estos dirigentes judíos, en los asientos de honor de la sinagoga, estaban sentados los observadores oficiales de Herodes Antipas, el cual les había ordenado que averiguaran la verdad sobre los inquietantes rumores de que el pueblo había intentado proclamar a Jesús rey de los judíos en los dominios de su hermano Felipe.

\par
%\textsuperscript{(1708.1)}
\textsuperscript{153:1.2} Jesús comprendía que iba a enfrentarse con la declaración inmediata de una guerra manifiesta y abierta por parte de sus enemigos cada vez más numerosos, y eligió audazmente emprender la ofensiva. Cuando alimentó a los cinco mil, había desafiado sus ideas sobre el Mesías material; ahora, decidió de nuevo atacar abiertamente sus conceptos del libertador judío. Esta crisis, que comenzó con la alimentación de los cinco mil y terminó con el sermón de este sábado por la tarde, marcó el momento en que se redujo la corriente de la fama y de las aclamaciones populares. De ahora en adelante, el trabajo del reino iba a ocuparse cada vez más de la tarea más importante de ganar conversos espirituales duraderos para la fraternidad verdaderamente religiosa de la humanidad. Este sermón marcó la crisis de transición entre el período de discusión, controversia y decisión, y el de la guerra abierta, con la aceptación final o el rechazo definitivo.

\par
%\textsuperscript{(1708.2)}
\textsuperscript{153:1.3} El Maestro sabía muy bien que muchos de sus seguidores estaban preparándose mentalmente, de manera lenta pero segura, para rechazarlo definitivamente. También sabía que muchos de sus discípulos estaban pasando, de manera lenta pero segura, por esa preparación de la mente y esa disciplina del alma que les permitiría triunfar sobre las dudas y afirmar valientemente su fe completa en el evangelio del reino. Jesús comprendía plenamente cómo se preparan los hombres para las decisiones de una crisis y para llevar a cabo acciones repentinas basadas en elecciones valientes, mediante el lento proceso de elegir reiteradamente entre el bien y el mal en las situaciones recurrentes. A sus mensajeros elegidos los sometió a repetidas desilusiones y les proporcionó frecuentes oportunidades de pruebas para que escogieran entre la buena y la mala manera de enfrentarse a las dificultades espirituales. Sabía que podía confiar en sus seguidores, que cuando se enfrentaran con la prueba final, tomarían sus decisiones esenciales de acuerdo con las actitudes mentales y las reacciones espirituales habituales adquiridas anteriormente.

\par
%\textsuperscript{(1708.3)}
\textsuperscript{153:1.4} Esta crisis en la vida terrestre de Jesús empezó con la alimentación de los cinco mil y terminó con este sermón en la sinagoga; la crisis en la vida de los apóstoles empezó con este sermón en la sinagoga y continuó durante un año entero, terminando solamente con el juicio y la crucifixión del Maestro.

\par
%\textsuperscript{(1708.4)}
\textsuperscript{153:1.5} Aquella tarde, mientras estaban sentados allí en la sinagoga, antes de que Jesús empezara a hablar, en la mente de todos sólo había un gran misterio, una pregunta suprema. Tanto sus amigos como sus enemigos tenían un solo pensamiento: «¿Por qué él mismo hizo retroceder tan deliberada y eficazmente la corriente del entusiasmo popular?» Fue inmediatamente antes y después de este sermón cuando las dudas y las decepciones de sus partidarios descontentos se convirtieron en una oposición inconsciente que finalmente se transformó en un verdadero odio. Fue después de este sermón en la sinagoga cuando Judas Iscariote pensó conscientemente por primera vez en desertar. Pero, por el momento, supo dominar eficazmente todas estas inclinaciones.

\par
%\textsuperscript{(1708.5)}
\textsuperscript{153:1.6} Todos estaban perplejos. Jesús los había dejado confundidos y desconcertados. Recientemente había emprendido la mayor demostración de poder sobrenatural de toda su carrera. La alimentación de los cinco mil fue el único acontecimiento de su vida terrestre que más se acercó al concepto judío del Mesías esperado. Pero esta ventaja extraordinaria fue contrarrestada de manera inmediata e inexplicable por su negativa resuelta e inequívoca a ser proclamado rey.

\par
%\textsuperscript{(1709.1)}
\textsuperscript{153:1.7} El viernes por la noche, y de nuevo el sábado por la mañana, los dirigentes de Jerusalén le habían insistido a Jairo larga y encarecidamente que impidiera que Jesús hablara en la sinagoga, pero fue en vano. La única respuesta de Jairo a todos sus argumentos fue: «He concedido esta petición, y no faltaré a mi palabra».

\section*{2. El sermón memorable}
\par
%\textsuperscript{(1709.2)}
\textsuperscript{153:2.1} Jesús dio comienzo a este sermón leyendo en la ley el pasaje que se encuentra en el Deuteronomio: «Pero sucederá que, si este pueblo no escucha la voz de Dios, las maldiciones de la transgresión le alcanzarán con seguridad. El Señor hará que tus enemigos te golpeen; serás llevado por todos los reinos de la Tierra. El Señor te pondrá, junto con el rey que hayas establecido por encima de ti, en las manos de una nación extranjera. Te convertirás en un motivo de asombro, de proverbio y de burla entre todas las naciones\footnote{\textit{Predicción de la degradación del pueblo judío}: Dt 28:52-53.}. Tus hijos e hijas irán al cautiverio\footnote{\textit{Tus hijos e hijas serán llevados cautivos}: Dt 28:41.}. Los extranjeros que viven contigo adquirirán una alta autoridad y tú descenderás muy bajo\footnote{\textit{Los extranjeros se alzarán en autoridad}: Dt 28:43.}. Estas cosas te sucederán para siempre, a ti y a tu descendencia, porque no has querido escuchar la palabra del Señor\footnote{\textit{Sobre tu descendencia, por no escuchar al Señor}: Dt 28:45-50.}. Por eso servirás a tus enemigos que vendrán contra ti. Sufrirás el hambre y la sed y llevarás este yugo extranjero de hierro. El Señor traerá contra ti a una nación venida de lejos, de los confines de la Tierra, una nación cuya lengua no comprenderás, una nación de aspecto feroz, una nación que tendrá pocas consideraciones contigo. Te asediará en todas tus ciudades hasta que los altos muros fortificados en los que has puesto tu confianza se vengan abajo; y todo el país caerá en sus manos. Y sucederá que te verás obligado a comer el fruto de tu propio cuerpo, la carne de tus hijos e hijas, durante ese tiempo de asedio, a causa de la penuria con que te oprimirán tus enemigos».

\par
%\textsuperscript{(1709.3)}
\textsuperscript{153:2.2} Cuando Jesús hubo terminado esta lectura, pasó a los Profetas y leyó en Jeremías: «`Si no queréis escuchar las palabras de mis servidores, los profetas que os he enviado, entonces pondré a esta casa como Silo, y haré de esta ciudad una maldición para todas las naciones de la Tierra'. Los sacerdotes y los educadores oyeron a Jeremías pronunciar estas palabras en la casa del Señor. Y sucedió que, cuando Jeremías terminó de decir todo lo que el Señor le había ordenado que proclamara a todo el pueblo, los sacerdotes y los educadores lo agarraron, diciendo: `Es seguro que morirás'. Y todo el pueblo se apiñó alrededor de Jeremías en la casa del Señor. Cuando los príncipes de Judá oyeron estas cosas, se sentaron para juzgar a Jeremías. Entonces, los sacerdotes y los educadores hablaron a los príncipes y a todo el pueblo, diciendo: `Este hombre merece la muerte porque ha profetizado en contra de nuestra ciudad, y lo habéis escuchado con vuestros propios oídos'. Entonces Jeremías dijo a todos los príncipes y a todo el pueblo: `El Señor me ha enviado a profetizar contra esta casa y contra esta ciudad todas las palabras que habéis oído. Corregid pues vuestra conducta y reformad vuestras acciones, y obedeced la voz del Señor vuestro Dios, para que podáis escapar de los males que se han pronunciado contra vosotros. En cuanto a mí, heme aquí en vuestras manos. Haced conmigo lo que a vuestro entender os parezca bueno y justo. Pero tened por seguro que, si me quitáis la vida, atraeréis una sangre inocente sobre vosotros y sobre este pueblo, porque en verdad el Señor me ha enviado para decir todas estas palabras en vuestros oídos'.»\footnote{\textit{Advertencias de Jeremías en el reinado de Joacim}: Jer 26:4-15.}

\par
%\textsuperscript{(1710.1)}
\textsuperscript{153:2.3} «Los sacerdotes y los educadores de aquella época intentaron matar a Jeremías, pero los jueces no lo consintieron; sin embargo, debido a sus palabras de advertencia, permitieron que lo bajaran con unas cuerdas a una mazmorra inmunda, donde se hundió en el lodo hasta las axilas. Esto es lo que este pueblo le hizo al profeta Jeremías cuando obedeció la orden del Señor de prevenir a sus hermanos sobre su inminente caída política\footnote{\textit{Jeremías en la mazmorra}: Jer 38:5-6.}. Hoy deseo preguntaros: ¿Qué harán los principales sacerdotes y los jefes religiosos de este pueblo con el hombre que se atreve a advertirles del día de su condena espiritual? ¿Trataréis también de quitarle la vida al instructor que se atreve a proclamar la palabra del Señor, y que no tiene miedo de indicar cómo os negáis a caminar en la senda de la luz que conduce a la entrada del reino de los cielos?»

\par
%\textsuperscript{(1710.2)}
\textsuperscript{153:2.4} «¿Qué buscáis como prueba de mi misión en la Tierra? Os hemos dejado tranquilos en vuestras posiciones de influencia y de poder, mientras predicábamos la buena nueva a los pobres y a los proscritos. No hemos lanzado ningún ataque hostil contra aquello que veneráis, sino que hemos proclamado una nueva libertad para el alma del hombre dominada por el miedo. He venido al mundo para revelar a mi Padre\footnote{\textit{Jesús ha venido para revelar al Padre}: Mt 5:45,48; Mt 6:1,4,6; Mt 11:25-27; Mc 11:25-26; Lc 6:36; Lc 10:22; Jn 1:18; Jn 3:31-34; Jn 4:21-23; Jn 6:46; Jn 14:6-13,20; Jn 15:15; Jn 16:25; Jn 17:8,25-26.} y para establecer en la Tierra la fraternidad espiritual de los hijos de Dios\footnote{\textit{Fraternidad de los hijos de Dios}: 1 Cr 22:10; Sal 2:7; Is 56:5; Mt 5:9,16,45; Lc 20:36; Jn 1:12-13; 11:52; 20:17b; Hch 17:28-29; Ro 8:14-17,19,21; 9:26; 2 Co 6:18; Gl 3:26; 4:5-7; Ef 1:5; Flp 2:15; Heb 12:5-8; 1 Jn 3:1-2,10; 5:2; Ap 21:7; 2 Sam 7:14.}, el reino de los cielos. Aunque os he recordado muchas veces que mi reino no es de este mundo\footnote{\textit{Mi reino no es de este mundo}: Jn 18:36. \textit{Reino espiritual}: Mt 9:35; 24:14; 25:1,14. \textit{Reino de Dios}: Mt 6:33; 12:28; 19:24; 21:31,43; Mc 1:14-15,23-25; 4:30; 9:1,47; 10:14-15,23-25; 12:34; 14;25; 15:43; Lc 4:43; 6:20; 7:28; 8:1,10; 9:2,11,27; 9:60,62; 10:9-11; 11:20; 12:31-32; 13:18,20,28-29; 14:15; 16:16; 17:20-21; 18:16-17,24-25; 19:11; 21:29-32; 22:16,18; 23:51; Jn 3:3-5; Ro 14:17; 1 Co 4:20; 6:9.}, sin embargo mi Padre os ha otorgado muchas manifestaciones de prodigios materiales, además de las transformaciones y regeneraciones espirituales más evidentes».

\par
%\textsuperscript{(1710.3)}
\textsuperscript{153:2.5} «¿Qué nuevo signo esperáis de mí? Os aseguro que ya tenéis pruebas suficientes como para poder tomar vuestras decisiones. En verdad, en verdad les digo a muchos de los que hoy están sentados delante de mí, que os enfrentáis con la necesidad de escoger el camino que vais a seguir. A vosotros os digo, como Josué se lo dijo a vuestros antepasados: `escoged en este día a quién queréis servir'\footnote{\textit{Escoged en este día a quién servir}: Jos 24:15.}. Muchos de vosotros os encontráis hoy en el cruce de los caminos».

\par
%\textsuperscript{(1710.4)}
\textsuperscript{153:2.6} «Cuando no pudisteis encontrarme después del banquete de la multitud en la otra orilla, algunos de vosotros alquilasteis la flota pesquera de Tiberiades, que una semana antes se había refugiado en las cercanías durante una tormenta, para salir en mi persecución\footnote{\textit{A Cafarnaúm en bote}: Jn 6:23-24,59.}, y ¿para qué? ¡No para buscar la verdad y la rectitud, ni para aprender a servir y a ayudar mejor a vuestros semejantes! No, sino más bien para conseguir más pan sin haber trabajado para obtenerlo. No era para llenar vuestra alma con la palabra de la vida, sino solamente para llenaros la barriga con el pan de la facilidad\footnote{\textit{Los hombres buscan el pan físico}: Jn 6:26.}. Os han enseñando desde hace mucho tiempo que cuando llegara el Mesías realizaría aquellos prodigios que harían la vida agradable y fácil para todo el pueblo elegido. Así pues, no es de extrañar que vosotros, que habéis recibido esta educación, deseéis con vehemencia los panes y los peces. Pero os afirmo que ésa no es la misión del Hijo del Hombre. He venido para proclamar la libertad espiritual, enseñar la verdad eterna y fomentar la fe viviente».

\par
%\textsuperscript{(1710.5)}
\textsuperscript{153:2.7} «Hermanos míos, no anheléis la comida perecedera, sino buscad más bien el alimento espiritual que nutre incluso en la vida eterna\footnote{\textit{Deberíais buscar el pan espiritual}: Jn 6:27.}; éste es el pan de la vida que el Hijo da a todos los que quieran cogerlo y comerlo, porque el Padre ha dado esta vida al Hijo sin restricción. Cuando me habéis preguntado: `¿Qué debemos hacer para realizar las obras de Dios?'\footnote{\textit{¿Cómo debemos hacer las obras de Dios?}: Jn 6:27-28.}, os he dicho claramente: `La obra de Dios consiste en creer en aquel que él ha enviado.'»

\par
%\textsuperscript{(1710.6)}
\textsuperscript{153:2.8} Luego, Jesús señaló el emblema de una vasija de maná adornada con racimos de uva, que decoraba el dintel de esta nueva sinagoga, y dijo: «Habéis creído que vuestros antepasados comieron en el desierto el maná\footnote{\textit{El maná en el desierto}: Ex 16:14-15.} ---el pan del cielo--- pero yo os digo que aquello era el pan de la tierra\footnote{\textit{El maná y el pan de la vida}: Jn 6:30-33.}. Aunque Moisés no dio a vuestros padres el pan procedente del cielo, mi Padre está ahora dispuesto a daros el verdadero pan de la vida. El pan del cielo es lo que desciende de Dios y da la vida eterna a los hombres del mundo. Cuando me digáis: Danos de ese pan viviente, yo contestaré: Yo soy ese pan de la vida\footnote{\textit{Yo soy el pan de la vida}: Jn 6:34-37.}. El que viene a mí no tendrá hambre, y el que cree en mí nunca tendrá sed. Me habéis visto, habéis vivido conmigo, habéis contemplado mis obras, y sin embargo no creéis que yo haya salido del Padre. Pero a aquellos que sí creen ---no temáis. Todos los que son dirigidos por el Padre vendrán a mí, y el que venga a mí de ninguna manera será rechazado».

\par
%\textsuperscript{(1711.1)}
\textsuperscript{153:2.9} «Ahora, dejad que os afirme de una vez por todas que he descendido a la Tierra no para hacer mi propia voluntad, sino la voluntad de Aquél que me ha enviado. Y la voluntad final de Aquél que me ha enviado es que yo no pierda ni uno solo de todos los que me ha dado\footnote{\textit{La voluntad de Dios es la vida eterna}: Jn 6:38-40.}. Y ésta es la voluntad del Padre: Que todo el que contemple al Hijo y crea en él, tenga la vida eterna\footnote{\textit{Todo el que crea tendrá vida eterna}: Jn 3:16.}. Ayer mismo os dí de comer pan para vuestro cuerpo; hoy os ofrezco el pan de la vida para vuestras almas hambrientas. ¿Queréis coger ahora el pan del espíritu, como entonces comisteis de tan buena gana el pan de este mundo?»

\par
%\textsuperscript{(1711.2)}
\textsuperscript{153:2.10} Mientras Jesús se detenía un momento para echar una mirada a la asamblea, uno de los educadores de Jerusalén (miembro del sanedrín) se levantó y preguntó: «¿He comprendido bien cuando has dicho que eres el pan que ha bajado del cielo, y que el maná que Moisés dio a nuestros padres en el desierto no lo era?» Jesús respondió al fariseo: «Has comprendido bien». Entonces dijo el fariseo: «Pero, ¿no eres Jesús de Nazaret, el hijo de José el carpintero? ¿Tu padre y tu madre, así como tus hermanos y hermanas, no son bien conocidos para muchos de nosotros? ¿Cómo puede ser entonces que aparezcas aquí en la casa de Dios afirmando que has descendido del cielo?»\footnote{\textit{¿No eres el hijo del carpintero?}: Mt 13:55-56; Mc 6:3; Lc 4:22. \textit{¿No eres humano?}: Jn 6:41-42.}

\par
%\textsuperscript{(1711.3)}
\textsuperscript{153:2.11} En aquellos momentos había mucho murmullo en la sinagoga, y amenazaba con producirse tal alboroto, que Jesús se puso de pie y dijo: «Seamos pacientes; la verdad nunca teme un examen honesto. Soy todo lo que dices y aun más. El Padre y yo somos uno\footnote{\textit{El Padre y yo somos uno}: Jn 1:1; 5:17-18; 10:30,38; 12:44-45; 14:7-11,20; 17:11,21-22.}; el Hijo hace solamente lo que el Padre le enseña\footnote{\textit{El Padre enseña al hijo}: Mt 11:27; Lc 10:22; Jn 5:19-23; 8:28.}, y todos aquellos que son dados al Hijo por el Padre, el Hijo los recibirá en sí mismo. Habéis leído lo que está escrito en los Profetas: `Todos seréis enseñados por Dios'\footnote{\textit{Todos serán enseñados por Dios}: Is 54:13.}, y `Aquellos que son enseñados por el Padre también escucharán a su Hijo'. Cualquiera que se abandona a la enseñanza del espíritu interior del Padre acabará por venir a mí. Ningún hombre ha visto al Padre, pero el espíritu del Padre vive dentro del hombre. El Hijo que ha descendido del cielo ha visto ciertamente al Padre. Y aquellos que creen sinceramente en este Hijo, ya tienen la vida eterna»\footnote{\textit{Los que creen tienen vida eterna}: Dn 12:2; Mt 19:16,29; 25:46; Mc 10:17,30; Lc 10:25; 18:18,30; Jn 3:15-16,36; 4:14,36; 5:24,39; 6:27,40,47; 6:54:68; 8:51-52; 10:28; 11:25-26; 12:25,50; 17:2-3; Hch 13:46-48; Ro 2:7; 5:21; 6:22-23; Gl 6:8; 1 Ti 1:16; 6:12,19; Tit 1:2; 3:7; 1 Jn 1:2; 2:25; 3:15; 5:11,13,20; Jud 1:21; Ap 22:5.}.

\par
%\textsuperscript{(1711.4)}
\textsuperscript{153:2.12} «Yo soy el pan de la vida\footnote{\textit{El pan de vida}: Jn 6:48-51.}. Vuestros padres comieron el maná en el desierto y han muerto. En cuanto a este pan que desciende de Dios, si un hombre lo come, nunca morirá en espíritu\footnote{\textit{Buscar el alimento espiritual}: Jn 6:58b.}. Repito que soy este pan viviente, y toda alma que consigue obtener esta naturaleza unida de Dios y hombre vivirá para siempre. Este pan de vida que doy a todos los que quieren recibirlo es mi propia naturaleza viviente y combinada. El Padre está en el Hijo y el Hijo es uno con el Padre ---ésta es mi revelación donadora de vida al mundo y mi don de salvación para todas las naciones».

\par
%\textsuperscript{(1711.5)}
\textsuperscript{153:2.13} Cuando Jesús terminó de hablar, el jefe de la sinagoga disolvió la asamblea, pero no querían irse. Se agolparon alrededor de Jesús para hacerle más preguntas, mientras que otros murmuraban y discutían entre ellos\footnote{\textit{La gente discutía entre sí}: Jn 6:52.}. Este estado de cosas continuó durante más de tres horas. Finalmente, el auditorio se dispersó mucho después de las siete de la tarde.

\section*{3. Después de la reunión}
\par
%\textsuperscript{(1712.1)}
\textsuperscript{153:3.1} A Jesús le hicieron muchas preguntas durante esta reunión después del sermón. Algunas fueron formuladas por sus discípulos perplejos, pero la mayoría la realizaron los incrédulos sofistas que sólo intentaban ponerlo en evidencia y hacerlo caer en una trampa.

\par
%\textsuperscript{(1712.2)}
\textsuperscript{153:3.2} Uno de los fariseos visitantes se subió en un pedestal y gritó esta pregunta: «Nos dices que eres el pan de la vida. ¿Cómo puedes darnos tu carne para comer o tu sangre para beber? ¿Para qué sirve tu enseñanza si no se puede llevar a cabo?» Jesús contestó a esta pregunta diciendo: «Yo no os he enseñado que mi carne sea el pan de la vida ni mi sangre el agua viva. Pero os he dicho que mi vida en la carne es una donación del pan del cielo\footnote{\textit{Yo soy el pan del cielo}: Jn 6:53-58.}. El hecho de la Palabra de Dios donada en la carne\footnote{\textit{La Palabra hecha carne}: Jn 1:14.} y el fenómeno del Hijo del Hombre sujeto a la voluntad de Dios, constituyen una realidad de experiencia que equivale al alimento divino. No podéis comer mi carne ni beber mi sangre, pero podéis volveros uno conmigo, en espíritu, como yo soy uno en espíritu con el Padre. Podéis ser alimentados con la palabra eterna de Dios, que es en verdad el pan de la vida, y que ha sido donada en la similitud de la carne mortal; y vuestra alma puede ser regada con el espíritu divino, que es verdaderamente el agua de la vida. El Padre me ha enviado al mundo para mostrar cómo desea habitar y dirigir a todos los hombres; y he vivido esta vida en la carne de tal manera que pueda inspirar también a todos los hombres para que intenten siempre conocer y hacer la voluntad del Padre celestial que reside en ellos».

\par
%\textsuperscript{(1712.3)}
\textsuperscript{153:3.3} Entonces, uno de los espías de Jerusalén que había estado observando a Jesús y a sus apóstoles, dijo: «Observamos que ni tú ni tus apóstoles os laváis las manos convenientemente antes de comer pan\footnote{\textit{Ceremonial de lavarse las manos}: Mt 15:1-9; Mc 7:1-13.}. Debéis saber muy bien que la práctica de comer con las manos sucias y sin lavar es una transgresión de la ley de los ancianos. Tampoco laváis correctamente vuestras copas para beber ni vuestros recipientes para comer. ¿Por qué mostráis tan poco respeto por las tradiciones de los padres y las leyes de nuestros ancianos?»\footnote{\textit{Respeto por los padres}: Ex 20:12; Dt 5:16.} Después de haberlo escuchado, Jesús respondió: «¿Por qué transgredís los mandamientos de Dios con las leyes de vuestra tradición? El mandamiento dice: `Honra a tu padre y a tu madre', y ordena que compartáis con ellos vuestros bienes si es necesario; pero promulgáis una ley basada en la tradición, que permite que los hijos desobedientes digan que el dinero que podría haber ayudado a los padres ha sido `entregado a Dios'. La ley de los ancianos libera así de sus responsabilidades a estos hijos astutos, aunque utilicen posteriormente todo ese dinero para su propio bienestar. ¿Cómo puede ser que anuléis el mandamiento de esta manera con vuestra propia tradición? Hipócritas, Isaías profetizó bien acerca de vosotros cuando dijo: `Este pueblo me honra con sus labios, pero su corazón está lejos de mí. Me adoran en vano, pues enseñan como doctrinas los preceptos de los hombres'\footnote{\textit{Me honráis con los labios pero no con el corazón}: Is 29:13.}.»

\par
%\textsuperscript{(1712.4)}
\textsuperscript{153:3.4} «Podéis ver cómo abandonáis el mandamiento para aferraros a las tradiciones de los hombres. Estáis totalmente dispuestos a rechazar la palabra de Dios para mantener vuestras propias tradiciones. Y os atrevéis a ensalzar de otras muchas maneras vuestras propias enseñanzas por encima de la ley y los profetas».

\par
%\textsuperscript{(1712.5)}
\textsuperscript{153:3.5} Jesús dirigió entonces sus comentarios a todos los presentes, diciendo: «Oídme todos con atención. El hombre no se contamina espiritualmente con lo que entra en su boca, sino más bien con lo que sale de su boca y procede del corazón»\footnote{\textit{La contaminación procede del corazón}: Mt 12:33-37; 15:10-20; Mc 7:14-23.}. Pero ni siquiera los apóstoles lograron captar plenamente el significado de sus palabras, porque Simón Pedro también le preguntó: «Para que algunos de tus oyentes no se sientan ofendidos innecesariamente, ¿podrías explicarnos el significado de estas palabras?» Entonces Jesús le dijo a Pedro: «¿También tú eres duro de entendimiento? ¿No sabes que toda planta que mi Padre celestial no haya sembrado será arrancada? Vuelve ahora tu atención hacia aquellos que quisieran conocer la verdad. No puedes obligar a los hombres a amar la verdad. Muchos de estos educadores son guías ciegos. Y ya sabes que si un ciego conduce a otro ciego, los dos se caerán al precipicio\footnote{\textit{Si un ciego guía a un ciego, ambos caen al precipicio}: Mt 15:14b; Lc 6:39.}. Pero, escucha con atención mientras te digo la verdad acerca de las cosas que manchan moralmente y que contaminan espiritualmente a los hombres. Declaro que lo que entra en el cuerpo por la boca o penetra en la mente a través de los ojos y los oídos, no es lo que mancha al hombre. El hombre sólo se mancha con el mal que se puede originar en su corazón, y que se expresa en las palabras y en los actos de esas personas impías. ¿No sabes que es del corazón de donde provienen los malos pensamientos, los proyectos perversos de asesinato, robo y adulterio, junto con la envidia, el orgullo, la ira, la venganza, las injurias y los falsos testimonios? Éstas son exactamente las cosas que manchan a los hombres, y no el hecho de comer pan con las manos ceremonialmente sucias».

\par
%\textsuperscript{(1713.1)}
\textsuperscript{153:3.6} Los delegados fariseos del sanedrín de Jerusalén estaban ahora casi convencidos de que había que detener a Jesús bajo la acusación de blasfemia o por mofarse de la ley sagrada de los judíos; de ahí sus esfuerzos por implicarlo en una discusión sobre algunas tradiciones de los ancianos, las llamadas leyes orales de la nación, para que tuviera la posibilidad de atacarlas. Por mucha escasez que hubiera de agua, estos judíos esclavizados por la tradición nunca dejaban de ejecutar la ceremonia exigida de lavarse las manos antes de cada comida. Tenían la creencia de que «es mejor morir que transgredir los mandamientos de los ancianos». Los espías habían hecho esta pregunta porque se decía que Jesús había afirmado: «La salvación es una cuestión de corazón limpio, más bien que de manos limpias». Estas creencias son difíciles de eliminar una vez que se han vuelto parte de vuestra religión. Incluso muchos años después de esto, el apóstol Pedro continuaba siendo esclavo del miedo a muchas de estas tradiciones sobre las cosas puras e impuras, y sólo se liberó finalmente después de experimentar un sueño vívido y extraordinario\footnote{\textit{El sueño vívido de Pedro}: Hch 10:9-16.}. Todo esto se puede comprender mejor cuando se recuerda que estos judíos consideraban con los mismos ojos el comer sin lavarse las manos que el comerciar con una prostituta; ambas acciones eran igualmente castigables con la excomunión.

\par
%\textsuperscript{(1713.2)}
\textsuperscript{153:3.7} Por eso el Maestro eligió debatir y exponer la insensatez de todo el sistema rabínico de reglas y reglamentos representado por la ley oral ---las tradiciones de los ancianos, pues todas eran consideradas como más sagradas y más obligatorias para los judíos que las mismas enseñanzas de las Escrituras. Y Jesús se expresó con menos reserva porque sabía que había llegado la hora en que no podía hacer nada más por impedir una ruptura de relaciones clara y abierta con estos dirigentes religiosos.

\section*{4. Las últimas palabras en la sinagoga}
\par
%\textsuperscript{(1713.3)}
\textsuperscript{153:4.1} En medio de las discusiones de esta reunión después de los oficios, uno de los fariseos de Jerusalén trajo ante Jesús a un joven trastornado que estaba poseído por un espíritu indómito y rebelde. Al conducir a este muchacho demente delante Jesús, dijo: «¿Qué puedes hacer por una aflicción como ésta? ¿Puedes echar fuera a los demonios?» Cuando el Maestro contempló al joven, se sintió conmovido por la compasión y, haciéndole una señal al muchacho para que se acercara, lo cogió de la mano y dijo: «Tú sabes quién soy; sal de él; y encargo a uno de tus compañeros leales que procure que no vuelvas». Inmediatamente, el joven se sintió normal y en su pleno juicio\footnote{\textit{Curación de un endemoniado}: Mt 9:32-33a; Mt 12:22; Lc 11:14.}. Éste es el primer caso en el que Jesús echó realmente a un «espíritu maligno» fuera de un ser humano. Todos los casos anteriores habían sido solamente supuestas posesiones del diablo; pero éste era un auténtico caso de posesión demoníaca, como a veces se producían en aquella época hasta el día de Pentecostés, en que el espíritu del Maestro fue derramado sobre todo el género humano, haciendo imposible para siempre que estos pocos rebeldes celestiales se aprovecharan de ciertos tipos inestables de seres humanos.

\par
%\textsuperscript{(1714.1)}
\textsuperscript{153:4.2} Como el pueblo se maravillaba, uno de los fariseos se levantó y acusó a Jesús de que podía hacer estas cosas porque estaba aliado con los demonios; que gracias al lenguaje que había empleado para echar fuera a este diablo, Jesús admitía que se conocían mutuamente; y continuó exponiendo que los educadores y los dirigentes religiosos de Jerusalén habían concluido que Jesús realizaba todos sus supuestos milagros por el poder de Belcebú, el príncipe de los demonios\footnote{\textit{Un fariseo dice que Belcebú lo curó}: Mt 9:33b-34; 12:23-24; Mc 3:22,30; Lc 11:15.}. El fariseo dijo: «No tengáis nada en común con este hombre; está asociado con Satanás».

\par
%\textsuperscript{(1714.2)}
\textsuperscript{153:4.3} Entonces Jesús dijo: «¿Cómo puede Satanás echar fuera a Satanás? Un reino dividido contra sí mismo no puede subsistir; si una casa está dividida contra sí misma, pronto cae en la desolación\footnote{\textit{Una casa dividida no puede prevalecer}: Mt 12:25-32; Mc 3:23-30; Lc 11:17-23.}. ¿Puede una ciudad resistir el asedio si está desunida? Si Satanás echa a Satanás, está dividido contra sí mismo; ¿cómo podrá entonces subsistir su reino? Pero deberíais saber que nadie puede entrar en la casa de un hombre fuerte y despojarlo de sus bienes, a menos que primero lo haya vencido y atado. Así pues, si echo fuera a los demonios por el poder de Belcebú, ¿por quién los echan vuestros hijos? Por eso ellos serán vuestros jueces. Pero si echo fuera a los demonios por el espíritu de Dios, entonces el reino de Dios ha venido realmente hasta vosotros. Si no estuvierais cegados por los prejuicios y descarriados por el miedo y el orgullo, percibiríais fácilmente que alguien más grande que los demonios está en medio de vosotros. Me obligáis a proclamar que el que no está conmigo está contra mí, y que el que no recoge conmigo desparrama en todas direcciones. ¡Dejad que os haga una advertencia solemne, a vosotros que, con los ojos abiertos y una malicia premeditada, os atrevéis a atribuir a sabiendas las obras de Dios a las acciones de los demonios! En verdad, en verdad os digo que todos vuestros pecados serán perdonados, e incluso todas vuestras blasfemias, pero cualquiera que blasfeme contra Dios de manera deliberada y con una intención perversa, nunca será perdonado. Puesto que esos autores permanentes de la iniquidad nunca buscarán ni recibirán el perdón, son culpables del pecado de rechazar eternamente el perdón divino».

\par
%\textsuperscript{(1714.3)}
\textsuperscript{153:4.4} «Muchos de vosotros habéis llegado hoy al cruce de los caminos; habéis llegado al punto en que tenéis que efectuar la elección inevitable entre la voluntad del Padre y los caminos de las tinieblas escogidos por vosotros mismos\footnote{\textit{Elegir entre el bien o el mal}: Mt 12:33-36; Jos 24:25.}. Según lo que escojáis ahora, eso mismo llegaréis a ser con el tiempo. O bien tenéis que mejorar el árbol y su fruto, o de otro modo el árbol y su fruto se corromperán. Declaro que en el reino eterno de mi Padre, el árbol se conoce por sus frutos\footnote{\textit{El árbol bueno se conoce por sus frutos}: Mt 7:16-20; Lc 6:43-45; Gl 5:22-23; Ef 5:9.}. Pero algunos de vosotros, que sois como víboras, ¿cómo podéis producir buenos frutos si ya habéis escogido el mal? Después de todo, vuestra boca expresa claramente la abundancia de mal que hay en vuestro corazón».

\par
%\textsuperscript{(1714.4)}
\textsuperscript{153:4.5} Entonces se levantó otro fariseo, diciendo: «Maestro, quisiéramos que nos dieras un signo predeterminado que nosotros aceptaríamos como demostración de tu autoridad y de tu derecho a enseñar. ¿Estás de acuerdo con este arreglo?»\footnote{\textit{Los fariseos piden un signo}: Mt 12:38-39; 16:1-4; Mc 8:11-12; Lc 11:16,29-30.} Cuando Jesús escuchó esto, dijo: «Esta generación sin fe y en busca de signos desea una señal, pero no se os dará más signo que el que ya tenéis, y aquel que veréis cuando el Hijo del Hombre se separe de vosotros».

\par
%\textsuperscript{(1714.5)}
\textsuperscript{153:4.6} Cuando hubo terminado de hablar, sus apóstoles lo rodearon y lo condujeron fuera de la sinagoga. Recorrieron el trayecto con él, en silencio, hasta la casa de Betsaida. Todos estaban asombrados y un poco aterrorizados por el cambio repentino en la táctica de enseñanza del Maestro. No estaban acostumbrados en absoluto a verlo actuar de una manera tan militante.

\section*{5. El sábado por la tarde}
\par
%\textsuperscript{(1715.1)}
\textsuperscript{153:5.1} Una y otra vez, Jesús había hecho añicos las esperanzas de sus apóstoles, y había destruido repetidas veces sus expectativas más acariciadas, pero nunca habían pasado por unos momentos de decepción ni por unos períodos de tristeza equivalentes a los que ahora estaban sufriendo. Además, un miedo real por su seguridad se mezclaba ahora con su depresión. Todos estaban sorprendidos y alarmados por la deserción tan repentina y completa del pueblo\footnote{\textit{Muchos seguidores se marchan}: Jn 6:60.}. También estaban un poco asustados y desconcertados por la audacia inesperada y la resolución afirmativa que mostraban los fariseos que habían venido de Jerusalén. Pero por encima de todo, estaban aturdidos a causa del repentino cambio de táctica de Jesús. En circunstancias normales, habrían acogido bien la aparición de esta actitud más militante, pero al producirse como se había producido, unida a tantas cosas inesperadas, esto les asustó.

\par
%\textsuperscript{(1715.2)}
\textsuperscript{153:5.2} Y ahora, para colmo de todas estas inquietudes, cuando llegaron a casa, Jesús se negó a comer. Se aisló durante horas en una de las habitaciones de arriba. Era cerca de la medianoche cuando Joab, el jefe de los evangelistas, regresó con la noticia de que aproximadamente un tercio de sus asociados habían abandonado la causa. Durante toda la noche, los discípulos leales estuvieron yendo y viniendo para informar de que el cambio súbito de sentimientos hacia el Maestro era general en Cafarnaúm. Los dirigentes de Jerusalén se apresuraron a alimentar este sentimiento de desafecto y, de todas las maneras posibles, procuraron fomentar un movimiento para que la gente se alejara de Jesús y sus enseñanzas. Durante estas horas difíciles, las doce mujeres mantenían una reunión en la casa de Pedro. Estaban enormemente trastornadas, pero ninguna de ellas desertó.

\par
%\textsuperscript{(1715.3)}
\textsuperscript{153:5.3} Poco después de la medianoche, Jesús bajó de la habitación de arriba y se mezcló con los doce y sus compañeros, unos treinta en total. Dijo: «Reconozco que esta criba del reino os aflige, pero es inevitable. Sin embargo, después de toda la preparación que habéis recibido, ¿había alguna buena razón para que tropezarais con mis palabras? ¿Cómo puede ser que estéis llenos de miedo y de consternación cuando veis que el reino se está despojando de esas multitudes tibias\footnote{\textit{Esas multitudes tibias}: Ap 3:16.} y de esos discípulos indiferentes? ¿Por qué os afligís cuando está despuntando un nuevo día en el que las enseñanzas espirituales del reino de los cielos van a brillar con una nueva gloria? Si encontráis difícil soportar esta prueba, ¿qué haréis entonces cuando el Hijo del Hombre deba regresar al Padre? ¿Cuándo y cómo os prepararéis para el momento en que ascenderé al lugar de donde vine a este mundo?»\footnote{\textit{Jesús consuela a los discípulos}: Jn 6:61-62.}

\par
%\textsuperscript{(1715.4)}
\textsuperscript{153:5.4} «Amados míos, debéis recordar que es el espíritu el que vivifica; la carne y todo lo relacionado con ella es de poco provecho. Las palabras que os he dicho son espíritu y vida. ¡Tened buen ánimo!\footnote{\textit{Tened buen ánimo}: Jn 6:63-70.} No os he abandonado. Mucha gente se ofenderá por la claridad de mis palabras de estos días. Ya habéis oído que muchos de mis discípulos se han vuelto atrás; ya no caminan conmigo. Sabía desde el principio que estos creyentes sin entusiasmo se quedarían por el camino. ¿No os escogí a vosotros doce y os aparté como embajadores del reino? Y ahora, en un momento como éste, ¿desertaréis vosotros también? Que cada uno de vosotros vele por su propia fe, porque uno de vosotros corre un grave peligro». Cuando Jesús hubo terminado de hablar, Simón Pedro dijo: «Sí, Señor, estamos tristes y perplejos, pero nunca te abandonaremos. Tú nos has enseñado las palabras de la vida eterna. Hemos creído en ti y te hemos seguido todo este tiempo. No nos volveremos atrás, porque sabemos que has sido enviado por Dios». Cuando Pedro terminó de hablar, todos asintieron unánimemente con la cabeza, aprobando su promesa de lealtad.

\par
%\textsuperscript{(1716.1)}
\textsuperscript{153:5.5} Entonces Jesús dijo: «Id a descansar, porque se acercan momentos de mucho trabajo; los próximos días van a ser muy activos».