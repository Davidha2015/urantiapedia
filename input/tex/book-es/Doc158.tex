\chapter{Documento 158. El monte de la transfiguración}
\par
%\textsuperscript{(1752.1)}
\textsuperscript{158:0.1} El viernes por la tarde 12 de agosto del año 29, el Sol iba a ponerse cuando Jesús y sus compañeros llegaron al pie del Monte Hermón, cerca del mismo lugar donde el joven Tiglat había aguardado en otro tiempo mientras el Maestro subía solo a la montaña para asegurar los destinos espirituales de Urantia y poner fin técnicamente a la rebelión de Lucifer. Permanecieron aquí durante dos días, preparándose espiritualmente para los acontecimientos que se iban a producir en breve.

\par
%\textsuperscript{(1752.2)}
\textsuperscript{158:0.2} De una manera general, Jesús sabía de antemano lo que iba a suceder en la montaña, y deseaba vivamente que todos sus apóstoles pudieran compartir esta experiencia. Se detuvo con ellos al pie de la montaña con el fin de prepararlos para esta revelación de sí mismo. Pero no pudieron alcanzar los niveles espirituales que hubieran justificado el hecho de exponerlos a la experiencia completa de la visita de los seres celestiales que pronto iban a aparecer sobre la Tierra. Y como no podía llevar a todos sus compañeros con él, decidió llevarse únicamente a los tres que lo acompañaban habitualmente en estas vigilias especiales. En consecuencia, solamente Pedro, Santiago y Juan compartieron, aunque de forma parcial, esta experiencia única con el Maestro.\footnote{\textit{Los tres que compartieron la experiencia}: Mt 17:1; Mc 9:2a; Lc 9:28.}

\section*{1. La transfiguración}
\par
%\textsuperscript{(1752.3)}
\textsuperscript{158:1.1} El lunes 15 de agosto por la mañana temprano, seis días después de la memorable confesión de Pedro, realizada un mediodía al borde del camino debajo de las moreras, Jesús y los tres apóstoles empezaron la ascensión del Monte Hermón.

\par
%\textsuperscript{(1752.4)}
\textsuperscript{158:1.2} Jesús había sido llamado para que subiera solo a la montaña con el fin de gestionar unos asuntos importantes que tenían que ver con el desarrollo de su donación en la carne, ya que esta experiencia estaba relacionada con el universo creado por él mismo. Es significativo que este acontecimiento extraordinario estuviera calculado para que ocurriera mientras Jesús y los apóstoles se encontraban en las tierras de los gentiles, y que se produjera efectivamente en una montaña de los gentiles.

\par
%\textsuperscript{(1752.5)}
\textsuperscript{158:1.3} Llegaron a su destino, casi a mitad de camino de la cima, un poco antes del mediodía. Mientras almorzaban, Jesús contó a los tres apóstoles una parte de la experiencia que había tenido, poco después de su bautismo, en las colinas al este del Jordán, y también les dijo algo más sobre su experiencia en el Monte Hermón durante su visita anterior a este retiro solitario.

\par
%\textsuperscript{(1752.6)}
\textsuperscript{158:1.4} Cuando era niño, Jesús tenía la costumbre de subir a la colina que estaba cerca de su casa, y soñar con las batallas que los ejércitos de los imperios habían librado en la planicie de Esdraelón; ahora, subía al Monte Hermón para recibir la dotación que lo prepararía para descender a las llanuras del Jordán y representar las escenas finales del drama de su donación en Urantia. Este día, en el Monte Hermón, el Maestro hubiera podido abandonar la lucha y volver a gobernar sus dominios universales, pero no solamente escogió satisfacer las exigencias de su orden de filiación divina, contenidas en el mandato del Hijo Eterno que está en el Paraíso, sino que también escogió satisfacer plenamente y hasta el fin la presente voluntad de su Padre Paradisiaco. Este día de agosto, tres de sus apóstoles vieron cómo rehusaba ser investido con la plena autoridad sobre su universo. Observaron aterrados la partida de los mensajeros celestiales, dejándolo solo para que terminara su vida terrestre como Hijo del Hombre e Hijo de Dios.

\par
%\textsuperscript{(1753.1)}
\textsuperscript{158:1.5} La fe de los apóstoles alcanzó su punto culminante en el momento de la alimentación de los cinco mil, y luego cayó rápidamente casi hasta el punto cero. Ahora, debido a que el Maestro había admitido su divinidad, la fe rezagada de los doce se elevó hasta su apogeo en las pocas semanas que siguieron, para sufrir después un declive progresivo. El tercer resurgimiento de su fe no se produjo hasta después de la resurrección del Maestro.

\par
%\textsuperscript{(1753.2)}
\textsuperscript{158:1.6} Hacia las tres de esta hermosa tarde, Jesús se despidió de los tres apóstoles, diciendo: «Me voy solo durante un tiempo para comulgar con el Padre y sus mensajeros; os ruego que os quedéis aquí, y mientras esperáis mi regreso, orad para que se haga la voluntad del Padre en toda vuestra experiencia relacionada con el resto de la misión donadora del Hijo del Hombre». Después de haberles dicho esto, Jesús se retiró para celebrar una larga conferencia con Gabriel y el Padre Melquisedek, y no regresó hasta cerca de las seis. Cuando Jesús observó la ansiedad de sus apóstoles debido a su ausencia prolongada, dijo: «¿Por qué teníais miedo? Sabéis muy bien que debo ocuparme de los asuntos de mi Padre; ¿por qué dudáis cuando no estoy con vosotros? Os declaro ahora que el Hijo del Hombre ha optado por pasar toda su vida en medio de vosotros y como uno de vosotros. Estad alegres; no os abandonaré hasta que haya terminado mi obra».

\par
%\textsuperscript{(1753.3)}
\textsuperscript{158:1.7} Mientras compartían una cena frugal, Pedro le preguntó al Maestro: «¿Cuánto tiempo vamos a permanecer en esta montaña, lejos de nuestros hermanos?» Jesús contestó: «Hasta que hayáis visto la gloria del Hijo del Hombre y sepáis que todo lo que os he declarado es verdad». Y hablaron de los asuntos de la rebelión de Lucifer, mientras estaban sentados cerca del rescoldo encendido de su fuego, hasta que la oscuridad los envolvió y los párpados de los apóstoles se hicieron pesados, pues habían emprendido su viaje muy temprano aquella mañana.

\par
%\textsuperscript{(1753.4)}
\textsuperscript{158:1.8} Los tres dormían profundamente desde hacía una media hora, cuando fueron despertados repentinamente por un crujido cercano\footnote{\textit{Los tres dormidos y despertados}: Lc 9:32.}; al mirar a su alrededor, para su gran sorpresa y consternación, vieron a Jesús conversando íntimamente con dos seres brillantes vestidos con las vestiduras de luz del mundo celestial. El rostro y la silueta de Jesús brillaban con la luminosidad de una luz celestial. Los tres hablaban en un lenguaje extraño, pero por ciertas cosas dichas, Pedro supuso erróneamente que los seres que estaban con Jesús eran Moisés y Elías; en realidad se trataba de Gabriel y del Padre Melquisedek. A petición de Jesús, los controladores físicos habían dispuesto lo necesario para que los apóstoles presenciaran esta escena.\footnote{\textit{La transfiguración}: Mt 17:2-3; Mc 9:2b-4; Lc 9:29-31.}

\par
%\textsuperscript{(1753.5)}
\textsuperscript{158:1.9} Los tres apóstoles estaban tan enormemente asustados que tardaron en recuperarse; mientras la deslumbrante visión se desvanecía delante de ellos y observaban que Jesús se quedaba solo, Pedro, que fue el primero en recuperarse, dijo\footnote{\textit{La sugerencia de Pedro}: Mt 17:4; Mc 9:5-6; Lc 9:33.}: «Jesús, Maestro, es provechoso haber estado aquí. Nos alegramos de ver esta gloria. Nos disgusta tener que regresar al mundo ignominioso. Si te parece bien, quedémonos aquí, y levantaremos tres tiendas, una para ti, otra para Moisés y otra para Elías». Pedro dijo esto a causa de su confusión, y porque no se le ocurrió ninguna otra cosa en ese momento.

\par
%\textsuperscript{(1753.6)}
\textsuperscript{158:1.10} Mientras Pedro aún estaba hablando, una nube plateada se les acercó y ensombreció a los cuatro. Ahora los apóstoles se asustaron mucho, y cuando caían de bruces para adorar, oyeron una voz, la misma que había hablado en el momento del bautismo de Jesús, que decía\footnote{\textit{Una nube y una voz}: Mt 17:5-8; Mc 9:7-8; Lc 9:34-36a.}: «Éste es mi Hijo amado; prestadle atención». Cuando la nube se desvaneció, Jesús estaba de nuevo solo con los tres; se inclinó y los tocó, diciendo: «Levantaos y no temáis; veréis cosas más grandes que ésta». Pero los apóstoles estaban realmente aterrorizados; mientras se preparaban para bajar de la montaña, poco antes de la medianoche, formaban un trío silencioso y pensativo.

\section*{2. El descenso de la montaña}
\par
%\textsuperscript{(1754.1)}
\textsuperscript{158:2.1} Durante cerca de la primera mitad del descenso de la montaña, no se dijo ni una palabra. Jesús empezó entonces la conversación, comentando: «Aseguraos de que no le contáis a nadie\footnote{\textit{No le contéis a nadie}: Mt 17:9; Mc 9:9; Lc 9:36b.}, ni siquiera a vuestros hermanos, lo que habéis visto y oído en esta montaña, hasta que el Hijo del Hombre haya resucitado de entre los muertos»\footnote{\textit{¿Resucitado de entre los muertos?}: Mc 9:10.}. Los tres apóstoles se quedaron anonadados y desconcertados por las palabras del Maestro «hasta que el Hijo del Hombre haya resucitado de entre los muertos». Habían reafirmado tan recientemente su fe en él como Libertador, el Hijo de Dios, y acababan de verlo transfigurado en gloria delante de sus propios ojos, ¡y ahora empezaba a hablar de «resurrección de entre los muertos»!

\par
%\textsuperscript{(1754.2)}
\textsuperscript{158:2.2} Pedro se estremeció con el pensamiento de la muerte del Maestro ---era una idea demasiado desagradable de soportar--- y temiendo que Santiago o Juan pudieran hacer alguna pregunta relacionada con esta declaración, pensó que sería mejor iniciar una conversación sobre otro tema; al no saber de qué hablar, expresó el primer pensamiento que le pasó por la cabeza, diciendo: «Maestro, ¿cómo es que los escribas dicen que Elías debe venir primero antes de que aparezca el Mesías?»\footnote{\textit{Elías ha de venir}: Mt 17:10-13; Mc 9:11-13.} Sabiendo que Pedro intentaba evitar mencionar su muerte y resurrección, Jesús respondió: «Es cierto que Elías viene primero para preparar el camino del Hijo del Hombre, el cual debe sufrir muchas cosas y al final ser rechazado. Pero te hago saber que Elías ya ha venido, y que no le recibieron, sino que le hicieron todo lo que quisieron». Los tres apóstoles se dieron cuenta entonces de que se refería a Juan el Bautista como si fuera Elías. Jesús sabía que, si insistían en considerarlo como el Mesías, entonces Juan debía ser el Elías de la profecía.\footnote{\textit{Profecía}: Mal 4:5-6.}

\par
%\textsuperscript{(1754.3)}
\textsuperscript{158:2.3} Jesús les recomendó que guardaran silencio sobre la visión anticipada que habían tenido de la gloria que le esperaba después de su resurrección porque no quería fomentar la idea, ahora que era recibido como el Mesías, de que iba a cumplir en alguna medida sus conceptos erróneos de un libertador que realizaba prodigios. Aunque Pedro, Santiago y Juan reflexionaron sobre todas estas cosas, no hablaron de ellas a nadie hasta después de la resurrección del Maestro.\footnote{\textit{Los apóstoles guardaron silencio}: Mc 9:9-10; Lc 9:36b.}

\par
%\textsuperscript{(1754.4)}
\textsuperscript{158:2.4} Mientras continuaban descendiendo de la montaña, Jesús les dijo: «No habéis querido recibirme como Hijo del Hombre; por eso he permitido que me recibáis de acuerdo con vuestra resolución establecida; pero no os equivoquéis, la voluntad de mi Padre debe prevalecer. Si escogéis seguir así la tendencia de vuestra propia voluntad, debéis prepararos para sufrir muchas desilusiones y experimentar muchas pruebas; pero el entrenamiento que os he dado debería bastar para que atraveséis triunfalmente estas penas que vosotros mismos habréis escogido».

\par
%\textsuperscript{(1754.5)}
\textsuperscript{158:2.5} Jesús no se llevó a Pedro, Santiago y Juan a la montaña de la transfiguración porque estuvieran, de alguna manera, mejor preparados que los otros apóstoles para presenciar lo que sucedió, o porque fueran espiritualmente más capaces de disfrutar de este raro privilegio. De ninguna manera. Sabía muy bien que ninguno de los doce estaba cualificado espiritualmente para esta experiencia; por eso se llevó solamente a los tres apóstoles que estaban asignados para acompañarlo en los momentos en que deseaba estar solo para disfrutar de una comunión solitaria.

\section*{3. El significado de la transfiguración}
\par
%\textsuperscript{(1755.1)}
\textsuperscript{158:3.1} Lo que Pedro, Santiago y Juan presenciaron en la montaña de la transfiguración fue un vislumbre fugaz del espectáculo celestial que tuvo lugar aquel día memorable en el Monte Hermón. La transfiguración fue un acto para:

\par
%\textsuperscript{(1755.2)}
\textsuperscript{158:3.2} 1. La aceptación, por parte del Hijo-Madre Eterno del Paraíso, de la plenitud de la donación de la vida encarnada de Miguel en Urantia. Jesús había recibido ahora la seguridad de que había cumplido las exigencias del Hijo Eterno. Fue Gabriel quien le aportó a Jesús esta seguridad.

\par
%\textsuperscript{(1755.3)}
\textsuperscript{158:3.3} 2. El testimonio de la satisfacción del Espíritu Infinito en cuanto a la plenitud de la donación en Urantia en la similitud de la carne mortal. La representante del Espíritu Infinito en el universo local, la asociada inmediata y la colaboradora siempre presente de Miguel en Salvington, habló en esta ocasión a través del Padre Melquisedek.

\par
%\textsuperscript{(1755.4)}
\textsuperscript{158:3.4} Jesús recibió con agrado estos testimonios del éxito de su misión terrestre, presentados por los mensajeros del Hijo Eterno y del Espíritu Infinito, pero observó que su Padre no indicaba que la donación en Urantia hubiera terminado; la presencia invisible del Padre sólo dio testimonio a través del Ajustador Personalizado de Jesús, diciendo: «Éste es mi hijo amado; prestadle atención»\footnote{\textit{«Éste es mi hijo amado»}: Mt 17:5; Mc 9:7; Lc 9:35.}. Y esto fue expresado en palabras para que los tres apóstoles también pudieran escucharlas.

\par
%\textsuperscript{(1755.5)}
\textsuperscript{158:3.5} Después de esta visita celestial, Jesús intentó conocer la voluntad de su Padre y decidió continuar la donación mortal hasta su fin natural. Éste fue el significado de la transfiguración para Jesús. Para los tres apóstoles, se trató de un acontecimiento que marcó la entrada del Maestro en la fase final de su carrera terrestre como Hijo de Dios e Hijo del Hombre.

\par
%\textsuperscript{(1755.6)}
\textsuperscript{158:3.6} Después de la visita oficial de Gabriel y del Padre Melquisedek, Jesús mantuvo una conversación familiar con estos Hijos ayudantes suyos, y habló con ellos sobre los asuntos del universo.

\section*{4. El muchacho epiléptico}
\par
%\textsuperscript{(1755.7)}
\textsuperscript{158:4.1} Jesús y sus compañeros llegaron al campamento apostólico este martes por la mañana un poco antes de la hora del desayuno. A medida que se acercaban, observaron una multitud considerable reunida alrededor de los apóstoles\footnote{\textit{Una multitud reunida}: Mc 9:14; Lc 9:37.}, y pronto empezaron a oír las ruidosas discusiones y controversias de este grupo de unas cincuenta personas, que incluía a los nueve apóstoles y a una asamblea dividida por igual entre los escribas de Jerusalén y los discípulos creyentes, que habían seguido a Jesús y a sus asociados en su viaje desde Magadán.

\par
%\textsuperscript{(1755.8)}
\textsuperscript{158:4.2} Aunque la muchedumbre sostenía discusiones diversas, la controversia principal estaba centrada en cierto ciudadano de Tiberiades que había llegado el día anterior en busca de Jesús. Este hombre, Santiago de Safed, tenía un hijo único de unos catorce años de edad, que estaba gravemente afligido de epilepsia. Además de esta enfermedad nerviosa, el muchacho había sido poseído por uno de esos intermedios errantes, malévolos y rebeldes, que entonces estaban presentes y sin control en la Tierra, de manera que el joven era epiléptico y a la vez estaba poseído por un demonio.

\par
%\textsuperscript{(1755.9)}
\textsuperscript{158:4.3} Durante cerca de dos semanas, este padre ansioso, oficial subalterno de Herodes Antipas, había vagado por las fronteras occidentales de los dominios de Felipe buscando a Jesús para suplicarle que curara a su hijo afligido. Y no alcanzó al grupo apostólico hasta alrededor del mediodía de este día, mientras Jesús estaba arriba en la montaña con los tres apóstoles.

\par
%\textsuperscript{(1756.1)}
\textsuperscript{158:4.4} Los nueve apóstoles se quedaron muy sorprendidos y bastante inquietos cuando este hombre, acompañado de casi cuarenta personas más que venían buscando a Jesús, se encontró repentinamente con ellos. En el momento de llegar este grupo, los nueve apóstoles, o al menos la mayoría de ellos, habían sucumbido a su antigua tentación ---la de discutir quién sería el más grande en el reino venidero; estaban atareados discurriendo sobre los puestos probables que serían asignados a cada apóstol. No podían simplemente liberarse por completo de la idea, tanto tiempo acariciada, de la misión material del Mesías. Ahora que el mismo Jesús había aceptado la confesión de los apóstoles de que era realmente el Libertador ---al menos había admitido el hecho de su divinidad--- qué cosa más natural que, durante este período en que estaban separados del Maestro, se pusieran a hablar de las esperanzas y ambiciones que predominaban en sus corazones. Estaban ocupados en estas discusiones cuando Santiago de Safed y sus compañeros, que buscaban a Jesús, dieron con ellos.

\par
%\textsuperscript{(1756.2)}
\textsuperscript{158:4.5} Andrés se levantó para saludar a este padre y a su hijo, diciendo: «¿A quién buscáis?» Santiago dijo: «Mi buen hombre, busco a tu Maestro. Busco la curación para mi hijo afligido. Quisiera que Jesús echara a ese diablo que posee a mi hijo». El padre se puso entonces a contar a los apóstoles que su hijo estaba tan afligido, que muchas veces casi había perdido la vida a consecuencia de estos ataques malignos.

\par
%\textsuperscript{(1756.3)}
\textsuperscript{158:4.6} Mientras los apóstoles escuchaban, Simón Celotes y Judas Iscariote se acercaron al padre, diciendo: «Nosotros podemos curarlo; no necesitas esperar a que regrese el Maestro. Somos los embajadores del reino, y ya no mantenemos estas cosas en secreto. Jesús es el Libertador, y nos han sido entregadas las llaves del reino»\footnote{\textit{«Las llaves del reino»}: Mt 16:19.}. Para entonces, Andrés y Tomás estaban consultándose a un lado, mientras que Natanael y los demás observaban la escena, asombrados; todos estaban horrorizados por la súbita audacia, si no presunción, de Simón y de Judas. El padre dijo entonces: «Si os ha sido dado el hacer estas obras, os ruego que pronunciéis las palabras que liberarán a mi hijo de esta esclavitud». Entonces Simón se adelantó, colocó su mano sobre la cabeza del niño, lo miró fijamente a los ojos y ordenó: «Sal de él, espíritu impuro; en nombre de Jesús, obedéceme». Pero el muchacho tuvo simplemente un ataque más violento, mientras los escribas se mofaban de los apóstoles y los creyentes decepcionados sufrían las burlas de estos críticos hostiles.\footnote{\textit{Los apóstoles fracasan en su exorcismo}: Mt 17:16; Mc 9:18b; Lc 9:40.}

\par
%\textsuperscript{(1756.4)}
\textsuperscript{158:4.7} Andrés estaba profundamente disgustado por este esfuerzo descaminado y su lamentable fracaso. Reunió aparte a los apóstoles para conversar y orar. Después de este período de meditación, sintiendo la aguda punzada de la derrota y la humillación que caía sobre todos ellos, Andrés hizo una segunda tentativa por echar al demonio, pero el fracaso coronó de nuevo sus esfuerzos. Andrés confesó francamente su derrota y le rogó al padre que permaneciera con ellos durante la noche o hasta que Jesús regresara, diciendo: «Quizás esta clase de demonios no se va, a menos que se lo ordene personalmente el Maestro».

\par
%\textsuperscript{(1756.5)}
\textsuperscript{158:4.8} Y así, mientras Jesús descendía de la montaña con Pedro, Santiago y Juan, exuberantes y extasiados, sus nueve hermanos estaban también desvelados pero a causa de su confusión, abatimiento y humillación. Formaban un grupo desanimado y escarmentado. Pero Santiago de Safed no quiso darse por vencido. Aunque no podían darle una idea de cuándo volvería Jesús, decidió quedarse allí hasta que regresara el Maestro.

\section*{5. Jesús cura al muchacho}
\par
%\textsuperscript{(1757.1)}
\textsuperscript{158:5.1} Mientras Jesús se acercaba, los nueve apóstoles se sintieron más que aliviados de recibirlo y muy animados al contemplar la alegría y el entusiasmo poco común que se reflejaba en los rostros de Pedro, Santiago y Juan. Todos se abalanzaron para saludar a Jesús y a sus tres hermanos. Mientras intercambiaban los saludos, el gentío se acercó, y Jesús preguntó: «¿Sobre qué estabais discutiendo cuando nos acercábamos?» Pero antes de que los apóstoles desconcertados y humillados pudieran contestar a la pregunta del Maestro, el ansioso padre del joven afligido se adelantó y, arrodillándose a los pies de Jesús, dijo: «Maestro, tengo un hijo, un hijo único, que está poseído por un espíritu maligno. Cuando tiene un ataque, no solamente grita de terror, echa espuma por la boca y cae como muerto, sino que con mucha frecuencia este espíritu maligno que lo posee lo destroza con convulsiones y a veces lo ha arrojado al agua e incluso al fuego\footnote{\textit{Lo arroja al agua y al fuego}: Mt 17:15; Mc 9:22.}. Mi hijo se está consumiendo con un gran rechinar de dientes y a consecuencia de sus numerosas magulladuras. Su vida es peor que la muerte; su madre y yo tenemos el corazón triste y el espíritu destrozado. Ayer, hacia el mediodía, buscándote a ti encontré a tus discípulos, y mientras te esperábamos, tus apóstoles intentaron echar a este demonio, pero no pudieron hacerlo. Y ahora, Maestro, ¿harás esto por nosotros, curarás a mi hijo?»\footnote{\textit{El epiléptico endemoniado}: Mt 17:14-16; Mc 9:15-18; Lc 9:38-40.}

\par
%\textsuperscript{(1757.2)}
\textsuperscript{158:5.2} Cuando Jesús escuchó este relato, tocó al padre arrodillado y le rogó que se levantara, mientras echaba una mirada penetrante a los apóstoles cercanos. Jesús dijo entonces a todos los que estaban delante de él: «Oh generación incrédula y perversa\footnote{\textit{Oh generación incrédula y perversa}: Mt 17:17; Mc 9:19; Lc 9:41.}, ¿cuánto tiempo seré indulgente con vosotros? ¿Cuánto tiempo estaré con vosotros? ¿Cuánto tiempo necesitaréis para aprender que las obras de la fe no aparecen a petición de la incredulidad escéptica?» Luego, señalando al padre desconcertado, Jesús dijo: «Trae aquí a tu hijo». Cuando Santiago hubo traído al muchacho, Jesús preguntó: «¿Cuánto tiempo hace que el niño está afligido de esta manera?» El padre respondió: «Desde que era muy pequeño». Mientras hablaban, el joven sufrió un ataque violento y cayó en medio de ellos, rechinando los dientes y echando espuma por la boca158:05.02 \footnote{\textit{El joven sufre convulsiones}: Lc 9:42a.}. Después de una serie de convulsiones violentas, se quedó tendido como muerto delante de ellos. El padre se arrodilló de nuevo a los pies de Jesús mientras imploraba al Maestro, diciendo: «Si puedes curarlo, te suplico que tengas compasión de nosotros y nos liberes de esta aflicción». Cuando Jesús escuchó estas palabras, bajó la mirada hacia el rostro ansioso del padre, y dijo: «No pongas en duda el poder del amor de mi Padre, sino solamente la sinceridad y el alcance de tu fe158:05.02 \footnote{\textit{Fe y duda}: Mc 9:20-24.}. Todas las cosas son posibles para aquel que cree realmente»\footnote{\textit{Todas las cosas son posibles para el que cree}: Gn 18:14; Jer 32:27; Mt 19:26; Mc 10:27; 14:36; Lc 1:37; 18:27.}. Entonces, Santiago de Safed pronunció aquellas palabras inolvidables mezcladas de fe y de duda: «Señor, yo creo. Te ruego que me ayudes en mi incredulidad»\footnote{\textit{Yo creo, ayuda a mi incredulidad}: Mc 9:24.}.

\par
%\textsuperscript{(1757.3)}
\textsuperscript{158:5.3} Cuando Jesús escuchó estas palabras, se adelantó, cogió al niño de la mano y dijo: «Voy a hacer esto de acuerdo con la voluntad de mi Padre y en honor de la fe viviente. Hijo mío, ¡levántate! Espíritu desobediente, sal de él y no vuelvas»\footnote{\textit{El niño es curado}: Mt 17:18; Mc 9:25-27; Lc 9:42b-43a.}. Luego, Jesús puso la mano del joven en la de su padre, y dijo: «Sigue tu camino. El Padre ha concedido el deseo de tu alma». Todos los que estaban presentes, incluídos los enemigos de Jesús, se quedaron asombrados por lo que habían visto.

\par
%\textsuperscript{(1757.4)}
\textsuperscript{158:5.4} Para los tres apóstoles que habían disfrutado tan recientemente del éxtasis espiritual de las escenas y experiencias de la transfiguración, fue realmente una desilusión volver tan pronto a la escena de la derrota y la frustración de sus compañeros apóstoles. Pero siempre fue así con estos doce embajadores del reino. Alternaban constantemente entre la exaltación y la humillación en las experiencias de su vida.

\par
%\textsuperscript{(1758.1)}
\textsuperscript{158:5.5} Ésta fue una verdadera curación de una doble aflicción, una dolencia física y una enfermedad de espíritu. La curación del muchacho fue permanente a partir de aquel momento. Cuando Santiago hubo partido con su hijo restablecido, Jesús dijo: «Vamos ahora a Cesarea de Filipo; preparaos enseguida». Y formaban un grupo silencioso a medida que viajaban hacia el sur, mientras la multitud iba detrás.

\section*{6. En el jardín de Celsus}
\par
%\textsuperscript{(1758.2)}
\textsuperscript{158:6.1} Pasaron la noche con Celsus, y aquella tarde en el jardín, después de que hubieran comido y descansado, los doce se reunieron alrededor de Jesús, y Tomás dijo: «Maestro, como los que nos quedamos atrás continuamos ignorando todavía lo que sucedió arriba en la montaña, que en tan gran medida animó a nuestros hermanos que te acompañaban, deseamos ardientemente que nos hables de nuestra derrota y nos instruyas en estas cuestiones, puesto que las cosas que sucedieron en la montaña no se pueden revelar en este momento»\footnote{\textit{Los discípulos preguntan por su fracaso}: Mt 17:19; Mc 9:28.}.

\par
%\textsuperscript{(1758.3)}
\textsuperscript{158:6.2} Jesús le contestó a Tomás, diciendo: «Todo lo que tus hermanos escucharon en la montaña os será revelado a su debido tiempo. Pero ahora quiero mostraros la causa de vuestra derrota en aquello que intentasteis tan imprudentemente. Ayer, mientras vuestro Maestro y sus compañeros, vuestros hermanos, subían a aquella montaña para buscar un conocimiento más amplio de la voluntad del Padre y pedir una dotación más rica de sabiduría para hacer eficazmente esa voluntad divina, vosotros que permanecíais aquí de vigilancia, con la instrucción de procurar adquirir una mente con perspicacia espiritual y de orar con nosotros para obtener una revelación más completa de la voluntad del Padre, en lugar de ejercer la fe que está a vuestra disposición, cedisteis a la tentación y caísteis en vuestras viejas tendencias nocivas de buscar para vosotros mismos unos puestos de preferencia en el reino de los cielos ---en ese reino material y temporal que persistís en imaginar. Y os aferráis a estos conceptos erróneos a pesar de la declaración reiterativa de que mi reino no es de este mundo»\footnote{\textit{Mi reino no es de este mundo}: Lc 17:20-21; Jn 18:36; Ro 14:17.}.

\par
%\textsuperscript{(1758.4)}
\textsuperscript{158:6.3} «Apenas capta vuestra fe la identidad del Hijo del Hombre, vuestro deseo egoísta por los ascensos mundanos os arrastra de nuevo, y empezáis a discutir entre vosotros quién será el más grande en el reino de los cielos, un reino que no existe ni existirá nunca tal como os empeñáis en concebirlo. ¿No os he dicho que el que quiera ser el más grande en el reino de la fraternidad espiritual de mi Padre debe volverse pequeño a sus propios ojos, y convertirse así en el servidor de sus hermanos?\footnote{\textit{El más grande debe ser el más servicial}: Mt 20:26; Mt 23:11-12; Mc 9:35; 10:43-44; Lc 22:26.} La grandeza espiritual consiste en un amor comprensivo semejante al amor de Dios, y no en el placer de ejercer el poder material para la exaltación del yo. En aquello que intentasteis y fracasasteis de manera tan completa, vuestra intención no era pura. Vuestro móvil no era divino. Vuestro ideal no era espiritual. Vuestra ambición no era altruista. Vuestra manera de obrar no estaba basada en el amor, y la meta que queríais alcanzar no era la voluntad del Padre que está en los cielos».

\par
%\textsuperscript{(1758.5)}
\textsuperscript{158:6.4} «¿Cuánto tiempo os llevará aprender que no podéis abreviar el curso de los fenómenos naturales establecidos, salvo cuando estas cosas están de acuerdo con la voluntad del Padre? Tampoco podéis realizar una obra espiritual en ausencia de poder espiritual. Y no podéis hacer ninguna de estas cosas, aunque su potencial esté presente, sin la existencia de un tercer factor humano esencial, la experiencia personal de poseer una fe viviente. ¿Necesitáis siempre las manifestaciones materiales para sentiros atraídos hacia las realidades espirituales del reino? ¿No podéis captar el significado espiritual de mi misión sin la manifestación visible de obras excepcionales? ¿Cuándo se podrá contar con vosotros para que os adhiráis a las realidades espirituales superiores del reino, sin hacer caso de la apariencia exterior de todas las manifestaciones materiales?»

\par
%\textsuperscript{(1759.1)}
\textsuperscript{158:6.5} Después de haber hablado así a los doce, Jesús añadió: «Ahora, id a descansar, porque mañana volveremos a Magadán y allí deliberaremos sobre nuestra misión en las ciudades y pueblos de la Decápolis. Como conclusión de la experiencia de este día, dejadme repetir a cada uno de vosotros lo que dije a vuestros hermanos en la montaña, y que estas palabras se graben profundamente en vuestro corazón: El Hijo del Hombre empieza ahora la última fase de su donación. Estamos a punto de comenzar los trabajos que luego conducirán a la gran prueba final de vuestra fe y devoción, cuando yo sea entregado entre las manos de los hombres que buscan mi destrucción. Y recordad lo que os digo: Al Hijo del Hombre le darán muerte, pero resucitará»\footnote{\textit{Anticipación de la pasión}: Mt 16:21; 17:22-23a; 20:17-19; 27:63; Mc 8:31; 9:31; 10:32-34; Lc 9:22,31,43b-44; 18:31-33; 24:7,46; Jn 14:28a; 20:9.}.

\par
%\textsuperscript{(1759.2)}
\textsuperscript{158:6.6} Se retiraron para pasar la noche, llenos de tristeza\footnote{\textit{Los discípulos apenados}: Mt 17:23b.}. Estaban desconcertados; no podían comprender estas palabras. Aunque temían hacer alguna pregunta sobre lo que Jesús había dicho, recordaron todo esto después de su resurrección.\footnote{\textit{Los discípulos no comprenden el significado}: Mc 9:32; Lc 9:45.}

\section*{7. La protesta de Pedro}
\par
%\textsuperscript{(1759.3)}
\textsuperscript{158:7.1} Aquel miércoles por la mañana temprano, Jesús y los doce salieron de Cesarea de Filipo hacia el parque de Magadán, cerca de Betsaida-Julias. Los apóstoles habían dormido muy poco aquella noche; así pues, se levantaron temprano y se prepararon para partir. Incluso a los imperturbables gemelos Alfeo les había conmocionado esta conversación sobre la muerte de Jesús. A medida que viajaban hacia el sur, un poco más allá de las Aguas de Merom llegaron a la carretera de Damasco, y como deseaban evitar a los escribas y a otras personas que Jesús sabía que pronto vendrían caminando detrás de ellos, ordenó continuar hasta Cafarnaúm por la carretera de Damasco que atraviesa Galilea. Hizo esto porque sabía que aquellos que lo seguían continuarían por la carretera al este del Jordán, pues suponían que Jesús y los apóstoles tendrían miedo de cruzar por el territorio de Herodes Antipas. Jesús intentaba eludir a sus críticos y a la multitud que lo seguía, para poder estar a solas con sus apóstoles ese día.

\par
%\textsuperscript{(1759.4)}
\textsuperscript{158:7.2} Habían caminado a través de Galilea hasta bien pasada la hora del almuerzo, cuando se detuvieron a la sombra para descansar. Después de que hubieron compartido la comida, Andrés, dirigiéndose a Jesús, dijo: «Maestro, mis hermanos no comprenden tus palabras profundas. Hemos llegado a creer plenamente que eres el Hijo de Dios, y ahora escuchamos esas extrañas palabras acerca de dejarnos, acerca de morir. No comprendemos tu enseñanza. ¿Es que nos hablas en parábolas? Te rogamos que nos hables claramente y de una manera no velada».

\par
%\textsuperscript{(1759.5)}
\textsuperscript{158:7.3} En respuesta a la petición de Andrés, Jesús dijo: «Hermanos míos, debido a que habéis confesado que soy el Hijo de Dios, me veo obligado a empezar a desvelaros la verdad sobre el final de la donación del Hijo del Hombre en la Tierra. Insistís en aferraros a la creencia de que soy el Mesías, y no queréis abandonar la idea de que el Mesías debe sentarse en un trono en Jerusalén; por eso insisto en deciros que el Hijo del Hombre deberá pronto ir a Jerusalén, sufrir muchas cosas, ser rechazado por los escribas, los ancianos y los principales sacerdotes, y después de todo eso, ser ejecutado y resucitar de entre los muertos. Y no os estoy diciendo una parábola. Os digo la verdad a fin de que estéis preparados para cuando esos acontecimientos caigan repentinamente sobre nosotros»\footnote{\textit{Los apóstoles oyen otra vez de la pasión}: Mt 16:21; 17:22-23a; 20:17-19; 27:63; Mc 8:31; 9:31; 10:32-34; Lc 9:22,31,43b-44; 18:31-33; 24:7,46; Jn 14:28a; 20:9.}. Mientras estaba hablando todavía, Simón Pedro se precipitó impetuosamente hacia él, puso su mano en el hombro del Maestro y dijo: «Maestro, está lejos de nuestra intención discutir contigo, pero declaro que estas cosas no te sucederán nunca»\footnote{\textit{La protesta de Pedro}: Mt 16:22; Mc 8:32.}.

\par
%\textsuperscript{(1760.1)}
\textsuperscript{158:7.4} Pedro habló así porque amaba a Jesús; pero la naturaleza humana del Maestro reconoció en estas palabras de afecto bien intencionado la sugerencia sutil de una tentación, la de cambiar su política de continuar hasta el fin su donación terrestre de acuerdo con la voluntad de su Padre Paradisiaco. Precisamente porque detectó el peligro de permitir que las sugerencias de sus mismos amigos afectuosos y leales le disuadieran, Jesús se volvió hacia Pedro y los otros apóstoles, diciendo: «Quédate detrás de mí. Hueles al espíritu del adversario, al tentador. Cuando habláis de esta manera, no estáis de mi parte, sino más bien de parte de nuestro enemigo. De esta forma vuestro amor por mí lo convertís en un obstáculo para yo hacer la voluntad del Padre. No prestéis atención a los caminos de los hombres, sino más bien a la voluntad de Dios»\footnote{\textit{Jesús reprende a Pedro}: Mt 16:23; Mc 8:33.}.

\par
%\textsuperscript{(1760.2)}
\textsuperscript{158:7.5} Cuando se hubieron recobrado del primer impacto de la punzante reprimenda de Jesús, y antes de reanudar su viaje, el Maestro dijo además: «Si alguien quiere seguirme, que no haga caso de sí mismo, que se encargue diariamente de sus responsabilidades y que me siga\footnote{\textit{Que tome sus responsabilidades y me siga}: Mt 10:38; 16:24; Mc 8:34; 10:21; Lc 9:23-26; 14:27.}. Porque el que quiera salvar su vida egoístamente, la perderá, pero el que pierda su vida por mi causa y por el evangelio, la salvará\footnote{\textit{El que salve su vida egoístamente, la perderá}: Mt 10:39; 16:25; Mc 8:35; Lc 9:24; 17:33.}. ¿De qué le sirve a un hombre ganar el mundo entero si pierde su propia alma? ¿Qué podría dar un hombre a cambio de la vida eterna?\footnote{\textit{El coste de ser dscípulo}: Mt 16:24-28; Mc 8:34-38; Lc 9:23-27.} No os avergoncéis de mí y de mis palabras en esta generación pecaminosa e hipócrita, como yo no me avergonzaré de reconoceros cuando aparezca con gloria delante de mi Padre en presencia de todas las huestes celestiales. Sin embargo, muchos de vosotros que estáis ahora delante de mí no experimentaréis la muerte hasta que hayáis visto llegar con poder este reino de Dios»\footnote{\textit{Muchos verán el reino llegar}: Mc 9:1.}.

\par
%\textsuperscript{(1760.3)}
\textsuperscript{158:7.6} Jesús indicó así claramente a los doce el camino doloroso y conflictivo que debían pisar si querían seguirlo. ¡Qué impacto causaron estas palabras en estos pescadores galileos que se empeñaban en soñar con un reino terrenal con puestos de honor para sí mismos! Pero sus corazones leales se conmovieron ante este llamamiento valiente, y ninguno de ellos sintió deseos de abandonarlo. Jesús no los enviaba solos al combate; él los conducía. Sólo les pedía que lo siguieran valientemente.

\par
%\textsuperscript{(1760.4)}
\textsuperscript{158:7.7} Los doce captaban lentamente la idea de que Jesús les estaba diciendo algo sobre la posibilidad de su muerte. Sólo comprendían vagamente lo que les decía sobre su muerte, mientras que su declaración acerca de resucitar de entre los muertos no consiguió en absoluto grabarse en sus mentes. A medida que pasaban los días, y recordaban su experiencia en la montaña de la transfiguración, Pedro, Santiago y Juan llegaron a comprender mejor algunas de estas cuestiones.

\par
%\textsuperscript{(1760.5)}
\textsuperscript{158:7.8} En toda la asociación de los doce con su Maestro, sólo unas pocas veces vieron la mirada centellante y escucharon las vivas palabras de reproche que Pedro y el resto de los apóstoles recibieron en esta ocasión. Jesús siempre había sido paciente con los defectos humanos de sus apóstoles, pero no fue así cuando se enfrentó a una amenaza inminente contra su programa de hacer implícitamente la voluntad de su Padre durante el resto de su carrera terrestre. Los apóstoles se quedaron literalmente anonadados; estaban asombrados y horrorizados. No encontraban palabras para expresar su tristeza. Empezaron a darse cuenta lentamente de lo que el Maestro tendría que soportar y de que deberían atravesar estas experiencias con él, pero no despertaron a la realidad de estos acontecimientos venideros hasta mucho tiempo después de estas primeras alusiones a la tragedia que amenazaba los últimos días de su vida.

\par
%\textsuperscript{(1761.1)}
\textsuperscript{158:7.9} Jesús y los doce partieron en silencio hacia su campamento del parque de Magadán, pasando por Cafarnaúm. A medida que transcurría la tarde, aunque no conversaron con Jesús, hablaron mucho entre ellos mientras Andrés charlaba con el Maestro.

\section*{8. En la casa de Pedro}
\par
%\textsuperscript{(1761.2)}
\textsuperscript{158:8.1} Entraron en Cafarnaúm al anochecer, pasaron por calles poco frecuentadas, y fueron directamente a la casa de Simón Pedro para cenar. Mientras David Zebedeo se preparaba para llevarlos al otro lado del lago, se demoraron en la casa de Simón, y entonces Jesús, mirando a Pedro y a los demás apóstoles, preguntó: «Cuando caminabais juntos esta tarde, ¿de qué hablabais tan seriamente entre vosotros?» Los apóstoles guardaron silencio, porque muchos de ellos habían continuado la discusión que empezaron en el Monte Hermón sobre los puestos que iban a tener en el reino venidero, sobre quién sería el más grande, y así sucesivamente. Conociendo las cosas que habían ocupado sus pensamientos durante aquel día, Jesús hizo señas a uno de los hijos pequeños de Pedro, sentó al niño entre ellos, y dijo: «En verdad, en verdad os digo que a menos que cambiéis de opinión y os parezcáis más a este niño, poco progreso haréis en el reino de los cielos\footnote{\textit{Entrar como un niño}: Mt 18:2-4; 19:13-14; Mc 9:36-37; 10:13-15; Lc 9:47-48; 18:16-17.}. Quienquiera que se humille\footnote{\textit{El que se humille a sí mismo}: Mt 18:1-4; 23:11-12; Lc 14:11; 18:14.} y se vuelva como este pequeño, se convertirá en el más grande en el reino de los cielos\footnote{\textit{Quién es el más grande}: Mt 18:2-5; 20:26-27; 23:11-12; Mc 10:43-44; Lc 9:46-47; 22:26.}. Quienquiera que recibe a un pequeño como éste, me recibe a mí\footnote{\textit{El que recibe a un niño me recibe a mí}: Mt 10:40; Lc 9:48.}. Y aquellos que me reciben, reciben también a Aquél que me ha enviado. Si queréis ser los primeros en el reino, procurad aportar estas buenas verdades a vuestros hermanos en la carne\footnote{\textit{Para ser el primero, ayudad a los demás}: Mt 20:26-27; 23:11-12; Mc 9:35; 10:43-44; Lc 22:26-27.}. Pero si alguien hace tropezar a uno de estos pequeños, sería mejor para él que le ataran una piedra de molino al cuello y lo arrojaran al mar\footnote{\textit{Atarse una piedra de molino al cuello}: Mt 18:6; Mc 9:42; Lc 17:2.}. Si las cosas que hacéis con vuestras manos, o las cosas que veis con vuestros ojos, ofenden en el progreso del reino, sacrificad esos ídolos queridos, porque es mejor entrar en el reino desprovistos de muchas de las cosas que se aman en la vida, que aferrarse a esos ídolos y encontrarse excluido del reino\footnote{\textit{Sacrificad las cosas que ofenden al progreso}: Mc 9:43-47.}. Pero por encima de todo, procurad no despreciar a uno solo de estos pequeños, porque sus ángeles están siempre contemplando el rostro de las huestes celestiales».

\par
%\textsuperscript{(1761.3)}
\textsuperscript{158:8.2} Cuando Jesús hubo terminado de hablar, subieron a la barca y navegaron hacia el otro lado en dirección a Magadán.