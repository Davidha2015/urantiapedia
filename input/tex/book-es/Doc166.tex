\chapter{Documento 166. La última visita a Perea del norte}
\par
%\textsuperscript{(1825.1)}
\textsuperscript{166:0.1} DEL 11 al 20 de febrero, Jesús y los doce hicieron una gira por todas las ciudades y pueblos del norte de Perea donde trabajaban los asociados de Abner y los miembros del cuerpo de mujeres. Se encontraron con que estos mensajeros del evangelio tenían éxito, y Jesús llamó repetidas veces la atención de sus apóstoles sobre el hecho de que el evangelio del reino se podía difundir sin necesidad de milagros y prodigios.

\par
%\textsuperscript{(1825.2)}
\textsuperscript{166:0.2} Toda esta misión de tres meses en Perea fue llevada a cabo con éxito y con poca ayuda por parte de los doce apóstoles; desde aquel momento en adelante, el evangelio reflejó más las \textit{enseñanzas} de Jesús que su personalidad. Pero sus discípulos no siguieron durante mucho tiempo sus instrucciones, pues poco después de la muerte y resurrección de Jesús, se desviaron de sus enseñanzas y empezaron a construir la iglesia primitiva alrededor de los conceptos milagrosos y de los recuerdos glorificados de su personalidad divina y humana.

\section*{1. Los fariseos en Ragaba}
\par
%\textsuperscript{(1825.3)}
\textsuperscript{166:1.1} El sábado 18 de febrero, Jesús se encontraba en Ragaba, donde vivía un rico fariseo llamado Natanael; puesto que un número considerable de sus compañeros fariseos seguían a Jesús y a los doce por todo el país, este sábado por la mañana Natanael preparó un desayuno para todos ellos, unas veinte personas, e invitó a Jesús como huésped de honor\footnote{\textit{Los fariseos invitan a Jesús}: Lc 11:37.}.

\par
%\textsuperscript{(1825.4)}
\textsuperscript{166:1.2} Cuando Jesús llegó a este desayuno, la mayoría de los fariseos, junto con dos o tres juristas, ya se encontraban allí sentados a la mesa. El Maestro tomó asiento de inmediato a la izquierda de Natanael, sin lavarse las manos en las palanganas. Muchos fariseos, especialmente los que estaban a favor de las enseñanzas de Jesús, sabían que sólo se lavaba las manos con fines higiénicos, y que detestaba estas prácticas puramente ceremoniales; por eso no se sorprendieron de que se dirigiera directamente a la mesa sin haberse lavado las manos dos veces. Pero Natanael se escandalizó porque el Maestro olvidó cumplir con las estrictas exigencias de las prácticas fariseas. Jesús tampoco se lavaba las manos, como hacían los fariseos, después del servicio de cada plato, ni al final de la comida\footnote{\textit{Jesús no se lava las manos}: Lc 11:38.}.

\par
%\textsuperscript{(1825.5)}
\textsuperscript{166:1.3} Después de mucho cuchicheo entre Natanael y un fariseo poco amistoso que estaba a su derecha, y después de muchos levantamientos de cejas y muecas burlonas de desprecio por parte de los que estaban sentados enfrente del Maestro, Jesús finalmente dijo: «Creí que me habíais invitado a esta casa para partir el pan con vosotros y quizás para hacerme preguntas sobre la proclamación del nuevo evangelio del reino de Dios; pero percibo que me habéis traído aquí para presenciar una exhibición de devoción ceremonial a vuestra propia presunción. Ese servicio ya me lo habéis hecho; ¿con qué nueva cosa vais a honrarme como invitado vuestro en esta ocasión?»

\par
%\textsuperscript{(1826.1)}
\textsuperscript{166:1.4} Cuando el Maestro hubo hablado así, bajaron la mirada hacia la mesa y permanecieron en silencio. Como nadie hablaba, Jesús continuó: «Muchos de vosotros, fariseos, estáis aquí conmigo como amigos, y algunos son incluso mis discípulos, pero la mayoría de los fariseos persisten en negarse a ver la luz y en reconocer la verdad, aunque la obra del evangelio se les presente con un gran poder. ¡Con cuánto cuidado limpiáis el exterior de las copas y de los platos, mientras que los recipientes del alimento espiritual están sucios y contaminados!\footnote{\textit{Limpiáis el exterior pero el interior está sucio}: Mt 23:25; Lc 11:39.} Os aseguráis en mostrarle al pueblo una apariencia piadosa y santa, pero vuestra alma interior está llena de presunción, de codicia, de extorsión y de todo tipo de maldad espiritual\footnote{\textit{Malvados, presuntuosos}: Mt 23:28.}. Vuestros dirigentes se atreven incluso a conspirar y a planear el asesinato del Hijo del Hombre. ¿No comprendéis, insensatos, que el Dios del cielo mira los móviles internos del alma\footnote{\textit{Dios mira los móviles}: Lc 16:15.}, así como vuestros fingimientos exteriores y vuestras ostentaciones de piedad?\footnote{\textit{Dios mira tanto el exterior como el interior}: Lc 11:40-41.} No creáis que el hecho de dar limosnas\footnote{\textit{Dar limosnas no es suficente}: Mt 23:23; Lc 11:42.} y de pagar los diezmos os limpiará de vuestra injusticia y os permitirá aparecer puros ante el Juez de todos los hombres. ¡Ay de vosotros, fariseos, que habéis persistido en rechazar la luz de la vida! Sois meticulosos en el pago del diezmo y dais limosnas con ostentación, pero despreciáis a sabiendas la visita de Dios y rechazáis la revelación de su amor. Aunque hacéis bien en prestar atención a esos deberes menores, no deberíais haber dejado sin hacer otras exigencias más importantes. ¡Ay de todos los que rehuyen la justicia, desdeñan la misericordia y rechazan la verdad! ¡Ay de todos los que desprecian la revelación del Padre, mientras aspiran a conseguir los principales asientos en la sinagoga y anhelan los saludos halagadores en las plazas de los mercados!»\footnote{\textit{Lamento por los que buscan los honores}: Mt 23:6-7; Mc 12:36-39; Lc 11:43.}

\par
%\textsuperscript{(1826.2)}
\textsuperscript{166:1.5} Cuando Jesús se levantó para marcharse, uno de los juristas sentados a la mesa le dirigió la palabra, diciendo: «Pero, Maestro, en algunas de tus declaraciones también nos haces reproches. ¿No hay nada bueno en los escribas, los fariseos o los juristas?» Jesús, que permanecía de pie, respondió al jurista: «Vosotros, al igual que los fariseos, os deleitáis en ocupar los mejores lugares en las fiestas y en lucir largas túnicas, mientras que colocáis unas cargas pesadas, difíciles de llevar, sobre los hombros de la gente. Y cuando las almas de los hombres se tambalean debajo de esas pesadas cargas, no levantáis ni uno solo de vuestros dedos. ¡Ay de vosotros, que encontráis vuestra mayor satisfacción en construir tumbas para los profetas\footnote{\textit{Tumbas para los profetas}: Mt 23:29-31.} que vuestros padres mataron!\footnote{\textit{Más «lamentaciones»}: Lc 11:45-50.} Que vosotros aprobáis lo que hicieron vuestros padres se pone de manifiesto en el hecho de que ahora planeáis matar a los que vienen a hacer, en el día de hoy, lo que hicieron los profetas en su día ---proclamar la justicia de Dios y revelar la misericordia del Padre celestial. Pero de todas las generaciones pasadas, la sangre de los profetas y de los apóstoles será exigida a esta generación perversa y presuntuosa. ¡Ay de todos vosotros, juristas, que le habéis quitado la llave del conocimiento a la gente común!\footnote{\textit{Juristas que esconden el conocimiento}: Mt 23:13; Lc 11:52.} Vosotros mismos os negáis a entrar en el camino de la verdad, y al mismo tiempo quisierais impedir la entrada a todos los que la buscan. Pero no podéis cerrar así las puertas del reino de los cielos; las hemos abierto a todos los que tienen fe para entrar; y esos portales de misericordia no serán cerrados por los prejuicios y la arrogancia de los falsos educadores y de los pastores engañosos que se parecen a los sepulcros blanqueados\footnote{\textit{Sepulcros blanqueados}: Mt 23:27-28.}, los cuales aparecen hermosos por fuera, pero por dentro están llenos de huesos de muertos y de todo tipo de impurezas espirituales».

\par
%\textsuperscript{(1826.3)}
\textsuperscript{166:1.6} Cuando Jesús hubo terminado de hablar en la mesa de Natanael, salió de la casa sin haber participado en la comida. De todos los fariseos que habían escuchado estas palabras, algunos creyeron en su enseñanza y entraron en el reino, pero más numerosos fueron los que persistieron en el camino de las tinieblas, estando cada vez más decididos a espiarlo para poder atrapar algunas de sus palabras y utilizarlas para procesarlo y juzgarlo ante el sanedrín de Jerusalén.\footnote{\textit{Los fariseos esperan para «atrapar» a Jesús}: Lc 11:53-54.}

\par
%\textsuperscript{(1827.1)}
\textsuperscript{166:1.7} Había únicamente tres cosas a las que los fariseos prestaban una atención particular:

\par
%\textsuperscript{(1827.2)}
\textsuperscript{166:1.8} 1. Practicar estrictamente el diezmo.

\par
%\textsuperscript{(1827.3)}
\textsuperscript{166:1.9} 2. Cumplir escrupulosamente las reglas de purificación.

\par
%\textsuperscript{(1827.4)}
\textsuperscript{166:1.10} 3. Evitar asociarse con todos los que no fueran fariseos.

\par
%\textsuperscript{(1827.5)}
\textsuperscript{166:1.11} En aquel momento, Jesús trataba de poner al descubierto la esterilidad espiritual de las dos primeras prácticas; en cuanto a sus observaciones destinadas a reprender a los fariseos por su rechazo a mantener relaciones sociales con los no fariseos, las reservó para una ocasión posterior en la que cenaría de nuevo con muchos de estos mismos hombres.

\section*{2. Los diez leprosos}
\par
%\textsuperscript{(1827.6)}
\textsuperscript{166:2.1} Al día siguiente, Jesús fue con los doce a Amatus, cerca de la frontera de Samaria. Al acercarse a la ciudad, se encontraron con un grupo de diez leprosos que residían por algún tiempo cerca de aquel lugar. Nueve de ellos eran judíos y uno samaritano. Normalmente, estos judíos habrían evitado toda asociación o todo contacto con este samaritano, pero la aflicción que tenían en común era más que suficiente para superar todos los prejuicios religiosos. Habían oído hablar mucho de Jesús y de sus primeras curaciones milagrosas, y como los setenta tenían la costumbre de anunciar la hora aproximada en que Jesús llegaría cuando el Maestro estaba de gira con los doce, los diez leprosos se habían enterado de que aparecería por estas inmediaciones hacia esta hora; en consecuencia, estaban apostados aquí en las afueras de la ciudad, con la esperanza de atraer su atención y pedirle la curación. Cuando los leprosos vieron llegar a Jesús, no se atrevieron a acercarse a él y se mantuvieron a distancia, gritándole: «Maestro, ten piedad de nosotros; límpianos de nuestra aflicción. Cúranos como has curado a otros».\footnote{\textit{Diez leprosos curados}: Lc 17:11-13.}

\par
%\textsuperscript{(1827.7)}
\textsuperscript{166:2.2} Jesús acababa de explicar a los doce por qué los gentiles de Perea, junto con los judíos menos ortodoxos, estaban más dispuestos que los judíos de Judea, más ortodoxos y atados a la tradición, a creer en el evangelio predicado por los setenta. Había llamado su atención sobre el hecho de que su mensaje también había sido recibido más fácilmente por los galileos, e incluso por los samaritanos. Pero los doce apóstoles aún no estaban dispuestos a mantener sentimientos amistosos hacia los samaritanos, despreciados durante tanto tiempo.

\par
%\textsuperscript{(1827.8)}
\textsuperscript{166:2.3} En consecuencia, cuando Simón Celotes observó al samaritano entre los leprosos, intentó persuadir al Maestro para que siguiera andando hasta la ciudad sin detenerse siquiera para intercambiar saludos con ellos. Jesús le dijo a Simón: «Pero, ¿y si el samaritano ama a Dios tanto como los judíos? ¿Vamos a juzgar a nuestros semejantes? ¿Quién puede decirlo? Si curamos a estos diez hombres, quizás el samaritano resulte ser más agradecido incluso que los judíos. ¿Te sientes seguro de tus opiniones, Simón?» Simón replicó de inmediato: «Si los purificas, lo averiguarás enseguida». Y Jesús contestó: «Así será, Simón, y pronto conocerás la verdad sobre la gratitud de los hombres y la misericordia amorosa de Dios».

\par
%\textsuperscript{(1827.9)}
\textsuperscript{166:2.4} Jesús se acercó a los leprosos, y dijo: «Si queréis recuperar la salud, id inmediatamente a mostraros a los sacerdotes, como lo exige la ley de Moisés». Y mientras iban de camino, recuperaron la salud. Cuando el samaritano vio que había sido curado, volvió sobre sus pasos buscando a Jesús, y empezó a glorificar a Dios en voz alta. Cuando hubo encontrado al Maestro, cayó de rodillas a sus pies y dio gracias por su purificación\footnote{\textit{El samaritano da gracias}: Lc 17:14-16.}. Los otros nueve, los judíos, también habían descubierto que habían sido curados, y aunque también estaban agradecidos por su purificación, continuaron su camino para mostrarse a los sacerdotes.

\par
%\textsuperscript{(1828.1)}
\textsuperscript{166:2.5} Mientras el samaritano permanecía arrodillado a los pies de Jesús, el Maestro miró sucesivamente a los doce, especialmente a Simón Celotes, y dijo: «¿No han sido purificados los diez? ¿Dónde están entonces los otros nueve, los judíos? Solamente uno, este extranjero, ha regresado para dar gloria a Dios». Luego dijo al samaritano: «Levántate y sigue tu camino; tu fe te ha curado».\footnote{\textit{¿Dónde están los otros nueve?}: Lc 17:17-19.}

\par
%\textsuperscript{(1828.2)}
\textsuperscript{166:2.6} Jesús miró de nuevo a sus apóstoles mientras el extranjero se alejaba. Y todos los apóstoles miraron a Jesús, excepto Simón Celotes, que tenía la mirada baja. Los doce no dijeron ni una palabra. Y Jesús tampoco habló; no era necesario hacerlo.

\par
%\textsuperscript{(1828.3)}
\textsuperscript{166:2.7} Aunque estos diez hombres creían realmente que tenían la lepra, solamente cuatro sufrían de ella. Los otros seis fueron curados de una enfermedad de la piel que había sido confundida con la lepra. Pero el samaritano tenía realmente la lepra.

\par
%\textsuperscript{(1828.4)}
\textsuperscript{166:2.8} Jesús ordenó a los doce que no dijeran nada sobre la purificación de los leprosos, y cuando entraban en Amatus, comentó: «Ya veis cómo los hijos de la casa, incluso cuando son desobedientes a la voluntad de su Padre, dan por sentadas sus bendiciones. Creen que es de poca importancia el dejar de dar las gracias cuando el Padre les concede la curación, pero cuando los extranjeros reciben los dones del dueño de la casa, se llenan de asombro y se sienten obligados a dar las gracias en reconocimiento por las buenas cosas que les han sido concedidas». Y los apóstoles continuaron sin decir nada en respuesta a las palabras del Maestro.

\section*{3. El sermón en Gerasa}
\par
%\textsuperscript{(1828.5)}
\textsuperscript{166:3.1} Mientras Jesús y los doce conversaban con los mensajeros del reino en Gerasa, uno de los fariseos que creían en él hizo la pregunta siguiente: «Señor, ¿en realidad se salvarán pocas o muchas personas?»\footnote{\textit{¿Se salvarán muchos?}: Lc 13:22-23.} Y Jesús contestó:

\par
%\textsuperscript{(1828.6)}
\textsuperscript{166:3.2} «Os han enseñado que sólo los hijos de Abraham serán salvados, que sólo los gentiles de adopción pueden esperar la salvación. Como las Escrituras indican que de todas las multitudes que salieron de Egipto, sólo Caleb y Josué\footnote{\textit{Caleb y Josué}: Nm 26:65.} vivieron para entrar en la tierra prometida, algunos de vosotros habéis deducido que sólo un número relativamente pequeño de aquellos que buscan el reino de los cielos conseguirá entrar en él».

\par
%\textsuperscript{(1828.7)}
\textsuperscript{166:3.3} «También tenéis otro dicho entre vosotros, y es un dicho que contiene mucha verdad: El camino que conduce a la vida eterna es recto y estrecho, y la puerta de acceso es igualmente estrecha, de manera que, de aquellos que buscan la salvación, pocos son los que logran entrar por esa puerta. También tenéis una enseñanza que dice que el camino que conduce a la destrucción es amplio, que su entrada es ancha, y que muchos escogen seguir ese camino\footnote{\textit{El camino ancho y el estrecho}: Dt 30:15-19; Mt 7:13-14.}. Este proverbio no está desprovisto de significado. Pero yo declaro que la salvación es, en primer lugar, una cuestión de elección personal. Aunque la puerta que conduce al camino de la vida sea estrecha, es lo suficientemente ancha como para recibir a todos los que intentan entrar sinceramente\footnote{\textit{Todos podrán entrar por la puerta}: Lc 13:24.}, porque yo soy esa puerta. Y el Hijo nunca le negará la entrada a ningún hijo del universo que aspira, por la fe, a encontrar al Padre a través del Hijo».

\par
%\textsuperscript{(1829.1)}
\textsuperscript{166:3.4} «Pero he aquí el peligro para todos los que quisieran aplazar su entrada en el reino, a fin de continuar buscando los placeres de la inmadurez y permitirse las satisfacciones del egoísmo: Al haberse negado a entrar en el reino como experiencia espiritual, quizás intenten más tarde entrar en él cuando la gloria del mejor camino sea revelada en la era por venir. Por consiguiente, aquellos que despreciaron el reino\footnote{\textit{Aquellos que despreciaron el reino}: Lc 13:25-27.} cuando yo vine en la similitud de la humanidad, tratarán de encontrar una entrada cuando sea revelado en la similitud de la divinidad; pero entonces diré a todos esos egoístas: No sé de dónde venís. Tuvisteis la oportunidad de prepararos para esta ciudadanía celestial, pero rehusasteis todas estas ofertas de misericordia; rechazasteis todas las invitaciones para venir mientras que la puerta estaba abierta. Ahora, para vosotros que habéis rechazado la salvación, la puerta está cerrada\footnote{\textit{No hay entrada para los descreídos}: Mt 7:21-23; Mt 25:11-13.}. Esta puerta no está abierta para aquellos que quieren entrar en el reino para glorificarse egoístamente. La salvación no es para los que no están dispuestos a pagar el precio de una dedicación entusiasta a hacer la voluntad de mi Padre. Cuando en vuestro espíritu y en vuestra alma le habéis dado la espalda al reino del Padre, es inútil permanecer mental y corporalmente delante de esta puerta, y llamar diciendo\footnote{\textit{Permanecer en la puerta y llamar}: Ap 3:20.}: `Señor, ábrenos; nosotros también queremos ser grandes en el reino.' Entonces declararé que no pertenecéis a mi redil. No os recibiré para que estéis con los que han librado el buen combate de la fe\footnote{\textit{Librar el buen combate de la fe}: 1 Ti 6:12; 2 Ti 4:7.} y han ganado la recompensa del servicio desinteresado en el reino en la Tierra. Y cuando digáis: `¿No comimos y bebimos contigo, y no enseñaste en nuestras calles?', entonces declararé de nuevo que sois unos extranjeros espirituales; que no servimos juntos en el ministerio de misericordia del Padre en la Tierra; que no os conozco; y entonces, el Juez de toda la Tierra os dirá: `Apartaos de nosotros, todos los que habéis disfrutado con las obras de la iniquidad.'»

\par
%\textsuperscript{(1829.2)}
\textsuperscript{166:3.5} «Pero no temáis; todo el que desee sinceramente encontrar la vida eterna entrando en el reino de Dios, hallará con seguridad esa salvación eterna. Pero vosotros, que rechazáis esta salvación, algún día veréis a los profetas de la semilla de Abraham sentarse en este reino glorificado con los creyentes de las naciones gentiles, para compartir el pan de la vida y refrescarse con el agua de la vida\footnote{\textit{Los perdedores verán a otros comer juntos}: Mt 8:11-12; Lc 13:28-30.}. Aquellos que se apoderen así del reino mediante el poder espiritual y los asaltos perseverantes de la fe viviente, vendrán del norte y del sur, del este y del oeste. Y mirad, muchos que son los primeros serán los últimos, y aquellos que son los últimos serán muchas veces los primeros»\footnote{\textit{Muchos primeros serán últimos, y últimos los primeros}: Mt 19:30; 20:16; Mc 9:35; 10:31; Lc 13:30.}.

\par
%\textsuperscript{(1829.3)}
\textsuperscript{166:3.6} Ésta fue, en verdad, una versión nueva e insólita del viejo proverbio bien conocido sobre el camino recto y estrecho.

\par
%\textsuperscript{(1829.4)}
\textsuperscript{166:3.7} Lentamente, los apóstoles y muchos discípulos aprendían el significado de la declaración inicial de Jesús: «A menos que nazcáis de nuevo, que nazcáis del espíritu\footnote{\textit{Debéis nacer de nuevo del espíritu}: Jn 3:3-7; 1 P 1:23.}, no podréis entrar en el reino de Dios». Sin embargo, para todos los que son honrados de corazón y tienen una fe sincera, es eternamente cierto que: «Mirad, permanezco en la puerta del corazón de los hombres y llamo\footnote{\textit{Jesús llama}: Ap 3:20-21.}; si alguien me abre, entraré, cenaré con él y lo alimentaré con el pan de la vida; seremos uno solo en espíritu y en propósito, y así seremos siempre hermanos en el largo y fructífero servicio de buscar al Padre Paradisiaco». Así pues, si los que se van a salvar son muchos o pocos, eso depende enteramente de que muchos o pocos hagan caso de la invitación: «Yo soy la puerta\footnote{\textit{Yo soy la puerta}: Jn 10:7,9.}, yo soy el camino nuevo y viviente\footnote{\textit{Yo soy el camino nuevo y viviente}: Jn 14:6.}, y cualquiera que lo desee puede entrar\footnote{\textit{Cualquiera que lo desee puede entrar}: Sal 50:15; Jl 2:32; Zac 13:9; Mt 7:24; 10:32-33; 12:50; 16:24-25; Mc 3:35; 8:34-35; Lc 6:47; 9:23-24; 12:8; Jn 3:15-16; 4:13-14; 11:25-26; 12:46; Hch 2:21; 10:43; 13:26; Ro 9:33; 10:13; 1 Jn 2:23; 4:15; 5:1; Ap 22:17b.} para emprender la búsqueda interminable de la verdad, que durará la vida eterna».

\par
%\textsuperscript{(1829.5)}
\textsuperscript{166:3.8} Los mismos apóstoles eran incapaces de comprender plenamente su enseñanza sobre la necesidad de utilizar la fuerza espiritual a fin de vencer todas las resistencias materiales, y para superar todos los obstáculos terrenales que casualmente pudieran impedir la comprensión de los valores espirituales, sumamente importantes, de la nueva vida en el espíritu, como hijos liberados de Dios.

\section*{4. La enseñanza sobre los accidentes}
\par
%\textsuperscript{(1830.1)}
\textsuperscript{166:4.1} Aunque la mayoría de los palestinos sólo hacían dos comidas al día, Jesús y los apóstoles tenían la costumbre, cuando iban de viaje, de detenerse al mediodía para descansar y tomar un refrigerio. En una de estas detenciones del mediodía, en el camino de Filadelfia, fue cuando Tomás le preguntó a Jesús: «Maestro, después de haber escuchado tus comentarios mientras viajábamos esta mañana, me gustaría averiguar si los seres espirituales están implicados en la producción de acontecimientos extraños y extraordinarios en el mundo material, y preguntar además si los ángeles y otros seres espirituales son capaces de impedir los accidentes».

\par
%\textsuperscript{(1830.2)}
\textsuperscript{166:4.2} En respuesta a la pregunta de Tomás, Jesús dijo: «¿He estado tanto tiempo con vosotros, y sin embargo continuáis haciéndome estas preguntas? ¿No habéis observado que el Hijo del Hombre vive como uno de vosotros, y que se niega firmemente a emplear las fuerzas del cielo para su sostenimiento personal? ¿No vivimos todos con los mismos recursos que emplean todos los hombres para existir? ¿Acaso veis que el poder del mundo espiritual se manifieste en la vida material de este mundo, salvo en la revelación del Padre y en la curación esporádica de sus hijos afligidos?»

\par
%\textsuperscript{(1830.3)}
\textsuperscript{166:4.3} «Vuestros antepasados han creído durante demasiado tiempo que la prosperidad era el signo de la aprobación divina, y que la adversidad era la prueba del desagrado de Dios. Afirmo que esas creencias son supersticiones. ¿No observáis que un número mucho mayor de pobres reciben el evangelio con regocijo y entran inmediatamente en el reino? Si las riquezas prueban el favor divino, ¿por qué los ricos se niegan tantas veces a creer en esta buena nueva que procede del cielo?»

\par
%\textsuperscript{(1830.4)}
\textsuperscript{166:4.4} «El Padre hace caer su lluvia sobre los justos y los injustos; el Sol brilla de igual manera sobre los virtuosos y los perversos\footnote{\textit{El sol brilla sobre todos}: Mt 5:45b.}. Habéis oído hablar de aquellos galileos cuya sangre mezcló Pilatos con la de los sacrificios, pero yo os digo que esos galileos no eran de ninguna manera más pecadores que todos sus semejantes, simplemente porque esto les sucedió a ellos. También conocéis la historia de los dieciocho hombres que perecieron por la caída de la torre de Siloé. No creáis que esos hombres que fueron aniquilados así eran más pecadores que todos sus hermanos de Jerusalén. Esas personas fueron simplemente las víctimas inocentes de uno de los accidentes del tiempo»\footnote{\textit{Víctimas inocentes del tiempo}: Lc 13:1-5.}.

\par
%\textsuperscript{(1830.5)}
\textsuperscript{166:4.5} «Existen tres tipos de acontecimientos que se pueden producir en vuestras vidas:»

\par
%\textsuperscript{(1830.6)}
\textsuperscript{166:4.6} «1. Podéis participar en aquellos acontecimientos normales que forman parte de la vida que vosotros y vuestros compañeros vivís sobre la faz de la Tierra».

\par
%\textsuperscript{(1830.7)}
\textsuperscript{166:4.7} «2. Podéis ser víctimas por casualidad de uno de los accidentes de la naturaleza, de una de las desgracias humanas, sabiendo muy bien que esos sucesos no están de ninguna manera preparados de antemano ni son producidos de otro modo por las fuerzas espirituales del planeta».

\par
%\textsuperscript{(1830.8)}
\textsuperscript{166:4.8} «3. Podéis recoger la cosecha de vuestros esfuerzos directos por acatar las leyes naturales que gobiernan el mundo».

\par
%\textsuperscript{(1830.9)}
\textsuperscript{166:4.9} «Había un hombre que plantó una higuera en su patio, y después de ir muchas veces a buscar los frutos sin encontrar ninguno, llamó a los viñadores y les dijo: `He venido aquí durante tres temporadas para buscar los frutos de esta higuera y no he encontrado ninguno. Derribad este árbol estéril; ¿para qué tiene que estar estorbando en el suelo?' Pero el jardinero en jefe respondió a su señor: `Déjalo tranquilo durante un año más para que yo pueda cavar a su alrededor y echarle abono; si el año que viene no produce frutos, entonces lo cortaremos.' Y cuando se hubieron sometido así a las leyes de la fertilidad, fueron recompensados con una cosecha abundante, ya que el árbol estaba vivo y en buen estado».\footnote{\textit{La párabola de la higuera estéril}: Lc 13:6-9.}

\par
%\textsuperscript{(1831.1)}
\textsuperscript{166:4.10} «En las cosas de la enfermedad y de la salud, deberíais saber que esos estados físicos son el resultado de causas materiales; la salud no es la sonrisa del cielo, ni la aflicción el enojo de Dios».

\par
%\textsuperscript{(1831.2)}
\textsuperscript{166:4.11} «Los hijos humanos del Padre tienen la misma capacidad para recibir las bendiciones materiales; por eso, concede las cosas físicas a los hijos de los hombres sin discriminación. Cuando se trata de atribuir los dones espirituales, el Padre está limitado por la capacidad del hombre para recibir estos dones divinos. Aunque el Padre no hace acepción de personas\footnote{\textit{El Padre no hace acepción de personas}: 2 Cr 19:7; Job 34:19; Eclo 35:12; Hch 10:34; Ro 2:11; Gl 2:6; 3:28; Ef 6:9; Col 3:11.}, en la atribución de los dones espirituales está limitado por la fe del hombre y por su buena disposición para atenerse siempre a la voluntad del Padre».

\par
%\textsuperscript{(1831.3)}
\textsuperscript{166:4.12} Mientras viajaban hacia Filadelfia, Jesús continuó enseñándoles y respondiendo a sus preguntas sobre los accidentes, las enfermedades y los milagros, pero no fueron capaces de comprender plenamente esta enseñanza. Una hora de enseñanza no es suficiente para cambiar por completo las creencias de toda una vida, y por eso Jesús creyó necesario reiterar su mensaje, decirles una y otra vez lo que deseaba hacerles comprender; y aún así, no lograron captar el significado de su misión terrenal hasta después de su muerte y resurrección.

\section*{5. La congregación de Filadelfia}
\par
%\textsuperscript{(1831.4)}
\textsuperscript{166:5.1} Jesús y los doce iban de camino para visitar a Abner y a sus asociados, que predicaban y enseñaban en Filadelfia. De todas las ciudades de Perea, es en Filadelfia donde el grupo más numeroso de judíos y gentiles, ricos y pobres, eruditos e ignorantes, aceptó las enseñanzas de los setenta y entró así en el reino de los cielos. La sinagoga de Filadelfia nunca había estado sometida a la supervisión del sanedrín de Jerusalén, por lo que nunca había estado cerrada a las enseñanzas de Jesús y sus compañeros. En ese mismo momento, Abner enseñaba tres veces al día en la sinagoga de Filadelfia.

\par
%\textsuperscript{(1831.5)}
\textsuperscript{166:5.2} Esta misma sinagoga se convirtió más tarde en una iglesia cristiana y fue el cuartel general de los misioneros que promulgaron el evangelio en las regiones situadas al este. Fue mucho tiempo la plaza fuerte de las enseñanzas del Maestro, y durante siglos se mantuvo sola en esta región como centro del conocimiento cristiano.

\par
%\textsuperscript{(1831.6)}
\textsuperscript{166:5.3} Los judíos de Jerusalén siempre habían tenido problemas con los judíos de Filadelfia. Después de la muerte y resurrección de Jesús, la iglesia de Jerusalén, cuyo jefe era Santiago, el hermano del Señor, empezó a tener graves dificultades con la congregación de creyentes de Filadelfia. Abner se convirtió en el jefe de la iglesia de Filadelfia, y continuó siéndolo hasta su muerte. Este distanciamiento de Jerusalén explica por qué los relatos evangélicos del Nuevo Testamento no mencionan nada sobre Abner y su obra. Esta enemistad entre Jerusalén y Filadelfia permaneció durante toda la vida de Santiago y Abner, y continuó hasta algún tiempo después de la destrucción de Jerusalén. Filadelfia fue realmente el centro de la iglesia primitiva en el sur y en el este, como Antioquía lo fue en el norte y el oeste.

\par
%\textsuperscript{(1831.7)}
\textsuperscript{166:5.4} Fue una aparente desgracia para Abner estar en desacuerdo con todos los jefes de la iglesia cristiana primitiva. Riñó con Pedro y Santiago (el hermano de Jesús) sobre cuestiones relacionadas con la administración y la jurisdicción de la iglesia de Jerusalén; se separó de Pablo por divergencias sobre filosofía y teología. Abner tenía una filosofía más babilónica que helenista, y se opuso obstinadamente a todos los intentos de Pablo por rehacer las enseñanzas de Jesús para que ocasionaran menos objeciones, primero entre los judíos, y luego entre los grecorromanos que creían en los misterios.

\par
%\textsuperscript{(1832.1)}
\textsuperscript{166:5.5} Abner se vio obligado así a vivir una vida de aislamiento. Era el jefe de una iglesia que no gozaba de ninguna reputación en Jerusalén. Se había atrevido a desafiar a Santiago, el hermano del Señor, que posteriormente fue apoyado por Pedro. Esta conducta lo separó efectivamente de todos sus antiguos asociados. Luego se atrevió a oponerse a Pablo. Aunque estaba totalmente de acuerdo con la misión de Pablo entre los gentiles, y aunque lo apoyaba en sus disputas con la iglesia de Jerusalén, se opuso encarnizadamente a la versión de las enseñanzas de Jesús que Pablo había elegido predicar. En los últimos años de su vida, Abner denunció a Pablo como el «hábil corruptor de las enseñanzas de la vida de Jesús de Nazaret, el Hijo del Dios viviente».

\par
%\textsuperscript{(1832.2)}
\textsuperscript{166:5.6} Durante los últimos años de Abner y hasta algún tiempo después de su muerte, los creyentes de Filadelfia se atuvieron a la religión de Jesús, tal como éste la había vivido y enseñado, más estrictamente que cualquier otro grupo en la Tierra.

\par
%\textsuperscript{(1832.3)}
\textsuperscript{166:5.7} Abner vivió hasta los 89 años de edad, y murió en Filadelfia el día 21 de noviembre del año 74. Hasta el final de su vida, fue un creyente fiel en el evangelio del reino celestial y un instructor del mismo.