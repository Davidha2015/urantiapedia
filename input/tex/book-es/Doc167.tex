\chapter{Documento 167. La visita a Filadelfia}
\par 
%\textsuperscript{(1833.1)}
\textsuperscript{167:0.1} A LO largo de todo este período de ministerio en Perea, cuando se menciona que Jesús y los apóstoles visitaban las diversas localidades donde trabajaban los setenta, es bueno recordar que, por regla general, sólo lo acompañaban diez apóstoles, pues tenía la costumbre de dejar en Pella al menos a dos apóstoles para instruir a la multitud. Mientras Jesús se preparaba para ir a Filadelfia, Simón Pedro y su hermano Andrés regresaron al campamento de Pella para enseñar a la muchedumbre allí reunida. Cuando el Maestro dejaba el campamento de Pella para recorrer Perea, no era raro que lo siguieran entre trescientas y quinientas personas que residían en el campamento. Cuando llegó a Filadelfia, iba acompañado por más de seiscientos seguidores.

\par 
%\textsuperscript{(1833.2)}
\textsuperscript{167:0.2} Ningún milagro se había producido durante la reciente gira de predicación a través de la Decápolis y, a excepción de la purificación de los diez leprosos, hasta ese momento no había habido ningún milagro en esta misión en Perea. Era un período en el que el evangelio se proclamaba con poder, sin milagros, y la mayor parte del tiempo sin la presencia personal de Jesús ni tampoco de sus apóstoles.

\par 
%\textsuperscript{(1833.3)}
\textsuperscript{167:0.3} Jesús y los diez apóstoles llegaron a Filadelfia el miércoles 22 de febrero, y el jueves y el viernes los pasaron descansando de sus viajes y trabajos recientes. Santiago habló en la sinagoga aquel viernes por la noche, y se convocó un consejo general para el día siguiente al anochecer. Estaban muy contentos por el progreso del evangelio en Filadelfia y en los pueblos cercanos. Los mensajeros de David también trajeron noticias de los nuevos progresos del reino por toda Palestina, así como buenas nuevas de Alejandría y Damasco.

\section*{1. El desayuno con los fariseos}
\par 
%\textsuperscript{(1833.4)}
\textsuperscript{167:1.1} En Filadelfia vivía un fariseo muy rico e influyente que había aceptado las enseñanzas de Abner, y que invitó a Jesús a desayunar en su casa el sábado por la mañana\footnote{\textit{El desayuno con el fariseo}: Lc 14:1a.}. Se sabía que a Jesús se le esperaba en Filadelfia a esa hora, por lo que un gran número de visitantes, entre ellos muchos fariseos, habían venido de Jerusalén y de otros lugares. En consecuencia, unos cuarenta de estos dirigentes y algunos juristas, fueron invitados a este desayuno que se había organizado en honor del Maestro.

\par 
%\textsuperscript{(1833.5)}
\textsuperscript{167:1.2} Mientras Jesús permanecía cerca de la puerta hablando con Abner, y después de que el mismo anfitrión se hubiera sentado, uno de los fariseos principales de Jerusalén, miembro del sanedrín, entró en la sala y, siguiendo su costumbre, se dirigió directamente al asiento de honor, a la izquierda del anfitrión\footnote{\textit{El asiento de honor}: Lc 14:7a.}. Pero como este lugar había sido reservado para el Maestro, y el de la derecha para Abner, el anfitrión señaló al fariseo de Jerusalén que se sentara en el cuarto asiento a la izquierda, y este dignatario se ofendió mucho por no haber recibido el asiento de honor.

\par 
%\textsuperscript{(1834.1)}
\textsuperscript{167:1.3} Pronto, todos estuvieron sentados y disfrutando de la conversación entre ellos, ya que la mayoría de los presentes eran discípulos de Jesús o bien eran favorables al evangelio. Sólo sus enemigos notaron el hecho de que no había cumplido con la ceremonia de lavarse las manos antes de sentarse para comer\footnote{\textit{Los fariseos vigilan}: Lc 14:1b.}. Abner se lavó las manos al principio de la comida, pero no durante el servicio.

\par 
%\textsuperscript{(1834.2)}
\textsuperscript{167:1.4} Hacia el final de la comida, un hombre procedente de la calle entró en la sala; había estado afligido durante mucho tiempo con una enfermedad crónica, y ahora se encontraba en un estado hidrópico. Este hombre era creyente, y había sido bautizado recientemente por los compañeros de Abner. No le pidió a Jesús que lo curara, pero el Maestro sabía muy bien que este enfermo había acudido al desayuno con la esperanza de evitar las multitudes que se agolpaban a su alrededor, y así tener más posibilidades de atraer su atención. Este hombre sabía que por aquella época se realizaban pocos milagros; sin embargo, había razonado en su fuero interno que su estado lastimoso quizás atraería la compasión del Maestro. Y no estaba equivocado porque, en cuanto entró en la sala, tanto Jesús como el presuntuoso fariseo de Jerusalén advirtieron su presencia. El fariseo no tardó en expresar su indignación porque se permitiera a un individuo así entrar en la sala. Pero Jesús miró al enfermo y le sonrió con tanta bondad que éste se acercó y se sentó en el suelo. Cuando la comida estaba finalizando, el Maestro paseó su mirada sobre los demás invitados y luego, después de lanzar una mirada significativa al hombre con hidropesía\footnote{\textit{El hombre con hidropesía}: Lc 14:2.}, dijo: «Amigos míos, educadores de Israel y expertos juristas, me gustaría haceros una pregunta: ¿Es lícito, o no, curar a los enfermos y a los afligidos en el día del sábado?» Pero los que estaban allí presentes conocían muy bien a Jesús; guardaron silencio, y no contestaron a su pregunta\footnote{\textit{Jesús pregunta, no hay respuesta}: Lc 14:3-4a.}.

\par 
%\textsuperscript{(1834.3)}
\textsuperscript{167:1.5} Jesús se dirigió entonces al lugar donde estaba sentado el enfermo, lo cogió de la mano, y le dijo: «Levántate y sigue tu camino. No has pedido la curación, pero conozco el deseo de tu corazón y la fe de tu alma»\footnote{\textit{Jesús cura al hombre con hidropesía}: Lc 14:4b.}. Antes de que el hombre abandonara la sala, Jesús volvió a su sitio y dirigió la palabra a los que estaban en la mesa\footnote{\textit{Habla con los fariseos}: Lc 14:5-14.}, diciendo: «Mi Padre hace estas obras, no para induciros a entrar en el reino, sino para revelarse a los que ya están en el reino. Podéis percibir que sería muy propio del Padre hacer precisamente estas cosas, porque ¿quién de vosotros, si su animal favorito se cayera en el pozo el día del sábado, no saldría inmediatamente para sacarlo de allí?» Como nadie quería contestarle, y ya que su anfitrión aprobaba evidentemente lo que estaba sucediendo, Jesús se levantó y dijo a todos los presentes: «Hermanos míos, cuando os inviten a un banquete nupcial, no os sentéis en el asiento principal, no sea que un hombre más ilustre que vosotros haya sido invitado, y el anfitrión tenga que venir a rogaros que dejéis vuestro sitio al otro huésped de honor. En ese caso se os pedirá, para vuestra verg\"uenza, que ocupéis un sitio inferior en la mesa. Cuando os inviten a una fiesta, es una demostración de sabiduría, al llegar a la mesa del festín, buscar el asiento más humilde y sentaros allí, de tal manera que, cuando el anfitrión examine a los convidados, pueda deciros: `Amigo mío, ¿por qué te has sentado en el asiento más humilde? Ven más arriba'; y así ese hombre será glorificado en presencia de los demás invitados. No lo olvidéis: el que se ensalza a sí mismo será humillado, mientras que el que se humilla sinceramente será ensalzado\footnote{\textit{El exaltado será humillado y el humillado exaltado}: Mt 18:1-4; Mt 23:12; Lc 1:52; 14:11; 18:14.}. Por eso, cuando convidéis a almorzar o deis una cena, no invitéis siempre a vuestros amigos, vuestros hermanos, vuestros parientes o a vuestros vecinos ricos, para que ellos puedan invitaros a cambio a sus fiestas, y seáis así recompensados. Cuando ofrezcáis un banquete, invitad de vez en cuando a los pobres, a los mutilados y a los ciegos. De esa manera os sentiréis dichosos en vuestro corazón, porque sabéis muy bien que los cojos y los lisiados no pueden devolveros vuestra ayuda amorosa».

\section*{2. La parábola de la gran cena}
\par 
%\textsuperscript{(1835.1)}
\textsuperscript{167:2.1} Cuando Jesús terminó de hablar en la mesa del desayuno del fariseo, uno de los juristas presentes, deseando romper el silencio, dijo sin reflexionar: «Bendito sea el que coma pan en el reino de Dios»\footnote{\textit{Un dicho judío}: Lc 14:15.} ---lo cual era un dicho corriente en aquella época. Jesús contó entonces una parábola, que incluso su amistoso anfitrión se vio obligado a tomar en serio. Dijo:

\par 
%\textsuperscript{(1835.2)}
\textsuperscript{167:2.2} «Cierto dirigente ofreció una gran cena, y como había invitado a muchos huéspedes, a la hora de la cena envió a sus criados para que dijeran a los invitados:
`Venid, pues ya está todo preparado.' Pero todos empezaron a excusarse unánimemente. El primero dijo: `Acabo de comprar una finca, y es absolutamente necesario que vaya a examinarla; te ruego que me excuses.' Otro dijo: `He comprado cinco yuntas de bueyes, y tengo que ir a recibirlas; te ruego que me excuses.' Y otro dijo: `Acabo de contraer matrimonio, y por esta razón no puedo ir.' Así pues, los criados regresaron e informaron de esto a su señor. Cuando el dueño de la casa escuchó esto, se irritó mucho, y volviéndose hacia sus criados, les dijo: `He preparado este banquete de boda; los animales cebados han sido matados, y todo está preparado para mis huéspedes, pero han desdeñado mi invitación; cada cual se ha ido a sus tierras y a sus mercancías, e incluso han mostrado una falta de respeto a mis criados que les pedían que vinieran a mi festín. Salid pues rápidamente a las calles y callejuelas de la ciudad, a las carreteras y a los caminos, y traed aquí a los pobres y a los parias, a los ciegos y a los cojos, para que haya invitados en el banquete de boda.' Los criados hicieron lo que su señor les había ordenado, y aún así quedaba sitio para más invitados. Entonces el señor dijo a sus criados: `Salid ahora a los caminos y a los campos, y obligad a los que estén allí a que vengan, para que se llene mi casa. Os aseguro que ninguno de los que fueron invitados en primer lugar probará mi cena.' Los criados hicieron lo que les había ordenado su señor, y la casa se llenó».\footnote{\textit{Parábola de la gran cena}: Mt 22:1-10; Lc 14:16-24.}

\par 
%\textsuperscript{(1835.3)}
\textsuperscript{167:2.3} Cuando escucharon estas palabras, todos se marcharon, y cada uno se fue a su propia casa. De todos los fariseos despectivos que estaban presentes aquella mañana, al menos uno comprendió el significado de esta parábola, porque fue bautizado aquel mismo día y confesó públicamente su fe en el evangelio del reino. Aquella noche, Abner predicó sobre esta parábola en el consejo general de los creyentes.

\par 
%\textsuperscript{(1835.4)}
\textsuperscript{167:2.4} Al día siguiente, todos los apóstoles emprendieron el ejercicio filosófico de tratar de interpretar el significado de esta parábola de la gran cena. Aunque Jesús escuchó con interés todas las diversas interpretaciones, se negó firmemente a ofrecerles una ayuda adicional para comprender la parábola. Solamente dijo: «Que cada uno encuentre el significado por sí mismo y en su propia alma».

\section*{3. La mujer de carácter débil}
\par 
%\textsuperscript{(1835.5)}
\textsuperscript{167:3.1} Abner se había encargado de que el Maestro enseñara en la sinagoga este sábado; era la primera vez que Jesús aparecía en una sinagoga desde que todas habían sido cerradas a sus enseñanzas por orden del sanedrín. Al final del servicio, Jesús observó delante de él a una mujer anciana que tenía una expresión abatida y el cuerpo muy encorvado. Esta mujer había estado durante mucho tiempo tiranizada por el miedo, y toda alegría había desaparecido de su vida. Cuando Jesús bajó del púlpito, se acercó a ella, tocó el hombro de su figura encorvada, y le dijo: «Mujer, si tan sólo quisieras creer, te liberarías por completo de tu debilidad de carácter». Y esta mujer, que había estado encorvada y atada por las depresiones del miedo durante más de dieciocho años, creyó en las palabras del Maestro y, gracias a la fe, se puso derecha inmediatamente. Cuando esta mujer se dio cuenta de que estaba erguida, elevó la voz y glorificó a Dios.\footnote{\textit{Curación en sábado}: Lc 13:10-13.}

\par 
%\textsuperscript{(1836.1)}
\textsuperscript{167:3.2} Aunque la aflicción de esta mujer era completamente mental, ya que su forma encorvada se debía a su mente deprimida, la gente creyó que Jesús había curado una verdadera enfermedad física. La asamblea de la sinagoga de Filadelfia era favorable a las enseñanzas de Jesús, pero el jefe principal de la sinagoga era un fariseo hostil. Como compartía la opinión de la asamblea de que Jesús había curado una enfermedad física, se indignó porque Jesús se hubiera atrevido a hacer una cosa así el día del sábado, y poniéndose de pie delante del auditorio, dijo: «¿No hay seis días para que los hombres puedan hacer todo su trabajo? Venid pues para ser curados durante esos días laborables, pero no en el día del sábado».\footnote{\textit{El jefe principal critica la curación}: Lc 13:14.}

\par 
%\textsuperscript{(1836.2)}
\textsuperscript{167:3.3} Cuando el jefe hostil hubo dicho esto, Jesús regresó a la tribuna de los oradores y dijo\footnote{\textit{La respuesta de Jesús}: Lc 13:15-17.}: «¿Por qué jugar a ser hipócritas? Cada uno de vosotros, ¿no saca a su buey del establo el día del sábado, para llevarlo al abrevadero? Si esa buena acción es permisible el día del sábado, esta mujer, una hija de Abraham, que ha estado encogida por el mal durante estos dieciocho años, ¿no debería ser liberada de esa esclavitud y llevada a compartir las aguas de la libertad y de la vida, incluso en este día de sábado?» Mientras la mujer continuaba glorificando a Dios, el detractor quedó puesto en evidencia, y la asamblea se regocijó con ella de que hubiera sido curada.

\par 
%\textsuperscript{(1836.3)}
\textsuperscript{167:3.4} Como consecuencia de haber criticado públicamente a Jesús este sábado, el jefe principal de la sinagoga fue destituido y reemplazado por un seguidor de Jesús.

\par 
%\textsuperscript{(1836.4)}
\textsuperscript{167:3.5} Jesús liberaba con frecuencia a estas víctimas del miedo de su debilidad de carácter, de su depresión mental y de su esclavitud al temor. Pero la gente creía que todas estas aflicciones eran, o bien enfermedades físicas, o posesiones por los espíritus malignos.

\par 
%\textsuperscript{(1836.5)}
\textsuperscript{167:3.6} El domingo, Jesús enseñó de nuevo en la sinagoga, y aquel día a mediodía Abner bautizó a muchas personas en el río que corría al sur de la ciudad. Al día siguiente, Jesús y los diez apóstoles habrían emprendido el viaje de vuelta al campamento de Pella si no hubiera sido por la llegada de uno de los mensajeros de David, que traía un mensaje urgente para Jesús de parte de sus amigos de Betania, cerca de Jerusalén.

\section*{4. El mensaje de Betania}
\par 
%\textsuperscript{(1836.6)}
\textsuperscript{167:4.1} El domingo 26 de febrero, a una hora muy tardía de la noche, un corredor procedente de Betania llegó a Filadelfia, trayendo un mensaje de Marta y María que decía: «Señor, aquel que amas está muy enfermo»\footnote{\textit{La noticia de que Lázaro está enfermo}: Jn 11:1-3.}. Este mensaje le llegó a Jesús al final de la conferencia de la tarde, justo en el momento en que se despedía de los apóstoles para pasar la noche. Al principio, Jesús no respondió nada. Entonces se produjo uno de esos extraños intervalos, un período de tiempo en el que parecía estar en comunicación con algo exterior a él y más allá de él. Luego levantó los ojos y se dirigió al mensajero, de manera que los apóstoles le oyeron decir: «Esta enfermedad no le llevará realmente a la muerte. No dudéis de que será empleada para glorificar a Dios y exaltar al Hijo»\footnote{\textit{El comentario de Jesús}: Jn 11:4.}.

\par 
%\textsuperscript{(1837.1)}
\textsuperscript{167:4.2} Jesús estaba muy encariñado con Marta, María y su hermano Lázaro; los amaba con un afecto ferviente\footnote{\textit{Jesús apreciaba a Lázaro}: Jn 11:5.}. Su primer pensamiento humano fue acudir inmediatamente en su ayuda, pero otra idea apareció en su mente combinada. Casi había perdido la esperanza de que los dirigentes judíos de Jerusalén aceptaran alguna vez el reino, pero seguía amando a su pueblo, y ahora se le había ocurrido un plan para que los escribas y fariseos de Jerusalén tuvieran una nueva oportunidad de aceptar sus enseñanzas; de este último llamamiento a Jerusalén decidió hacer, si su Padre quería, la obra exterior más profunda y asombrosa de toda su carrera terrenal. Los judíos estaban aferrados a la idea de un libertador que hiciera prodigios. Y aunque rehusaba rebajarse a realizar maravillas materiales o a llevar a cabo exhibiciones temporales de poder político, ahora buscó el consentimiento del Padre para manifestar su poder todavía no demostrado sobre la vida y la muerte.

\par 
%\textsuperscript{(1837.2)}
\textsuperscript{167:4.3} Los judíos tenían la costumbre de enterrar a sus muertos el día de su fallecimiento; era una práctica necesaria en un clima tan caluroso. A menudo sucedía que metían en la tumba a alguien que estaba simplemente en coma, de manera que al segundo, o incluso al tercer día, aquella persona salía de la tumba. Pero los judíos tenían la creencia de que, aunque el espíritu o el alma podía quedarse cerca del cuerpo durante dos o tres días, nunca permanecía allí después del tercer día; que la putrefacción ya estaba avanzada al cuarto día, y que nadie regresaba nunca de la tumba después de transcurrido ese tiempo. Debido a estas razones, Jesús permaneció dos días más en Filadelfia antes de prepararse para salir hacia Betania\footnote{\textit{Permanece dos días en Filadelfia}: Jn 11:6.}.

\par 
%\textsuperscript{(1837.3)}
\textsuperscript{167:4.4} En consecuencia, el miércoles por la mañana temprano dijo a sus apóstoles: «Preparémonos inmediatamente para ir otra vez a Judea»\footnote{\textit{Jesús dice que regresan a Judea}: Jn 11:7.}. Cuando los apóstoles escucharon estas palabras de su Maestro, se retiraron aparte durante un rato para consultarse entre ellos. Santiago tomó la dirección de la conversación, y todos estuvieron de acuerdo en que era una auténtica locura permitir a Jesús que regresara a Judea, por lo que volvieron como un solo hombre para comunicarselo. Santiago dijo: «Maestro, has estado en Jerusalén hace unas semanas, y los dirigentes intentaron matarte, mientras que el pueblo estaba dispuesto a lapidarte. En aquel momento ya diste a esos hombres su oportunidad de recibir la verdad, y no te permitiremos que regreses a Judea»\footnote{\textit{La objección de los apóstoles}: Jn 11:8.}.

\par 
%\textsuperscript{(1837.4)}
\textsuperscript{167:4.5} Jesús dijo entonces: «Pero, ¿no comprendéis que el día tiene doce horas, durante las cuales se pueden hacer las tareas sin peligro?\footnote{\textit{Hay que trabajar durante el día}: Jn 11:9-10.} Si un hombre camina de día, no tropieza puesto que tiene luz. Si camina de noche, está expuesto a tropezar ya que no tiene luz. Mientras dure mi día, no tengo miedo a entrar en Judea. Quisiera realizar otra obra poderosa para esos judíos; quisiera darles otra oportunidad para creer, y en los términos que ellos prefieren ---con las condiciones de una gloria exterior y la manifestación visible del poder del Padre y del amor del Hijo. Además, ¡no os dais cuenta de que nuestro amigo Lázaro se ha dormido\footnote{\textit{Lázaro está dormido}: Jn 11:11.}, y que quiero ir a despertarlo de ese sueño!»

\par 
%\textsuperscript{(1837.5)}
\textsuperscript{167:4.6} A continuación, uno de los apóstoles dijo: «Maestro, si Lázaro se ha dormido, entonces se restablecerá con más seguridad». En aquel tiempo, los judíos tenían la costumbre de hablar de la muerte como de una forma de sueño, pero como los apóstoles no habían comprendido que Jesús quería decir que Lázaro había partido de este mundo, ahora les dijo con toda claridad: «Lázaro ha muerto. Pero por vuestro bien, y aunque esto no salve a los demás, me alegro de no haber estado allí\footnote{\textit{Jesús se alegra de que vayan a ser testigos}: Jn 11:12-15a.}, a fin de que ahora podáis tener una nueva razón para creer en mí; todos os sentiréis fortalecidos por lo que vais a presenciar, y os servirá de preparación para el día en que me despediré de vosotros para ir hacia el Padre».

\par 
%\textsuperscript{(1838.1)}
\textsuperscript{167:4.7} Como no pudieron persuadirlo para que se abstuviera de ir a Judea, y como algunos apóstoles eran incluso reacios a acompañarlo, Tomás se dirigió a sus compañeros, diciendo: «Ya le hemos expresado nuestros temores al Maestro, pero él está decidido a ir a Betania. Estoy convencido de que esto significa el fin; lo matarán con toda seguridad, pero si ésta es la elección del Maestro, entonces comportémonos como unos hombres valientes; vamos también para poder morir con él»\footnote{\textit{El parecer de Tomás}: Jn 11:15b-16.}. Siempre fue así; en las cuestiones que requerían un coraje deliberado y sostenido, Tomás fue siempre el sostén principal de los doce apóstoles.

\section*{5. En el camino de Betania}
\par 
%\textsuperscript{(1838.2)}
\textsuperscript{167:5.1} Un grupo de casi cincuenta amigos y enemigos siguió a Jesús en el camino hacia Judea. El miércoles, a la hora de la comida del mediodía, habló a sus apóstoles y a este grupo de seguidores sobre las «Condiciones de la salvación», y al final de esta lección, contó la parábola del fariseo y el publicano (un recaudador de impuestos). Jesús dijo: «Ya veis que el Padre concede la salvación a los hijos de los hombres, y esta salvación es un don gratuito para todos los que tienen la fe necesaria para recibir la filiación en la familia divina. El hombre no puede hacer nada para ganar esta salvación. Las obras presuntuosas no pueden comprar el favor de Dios, y una gran cantidad de oraciones en público no compensarán la falta de fe viviente en el corazón. Podéis engañar a los hombres con vuestro servicio aparente, pero Dios examina vuestra alma. Lo que os digo está bien ilustrado en el ejemplo de dos hombres, uno fariseo y el otro publicano, que entraron a orar en el templo\footnote{\textit{Parábola de los dos orantes}: Lc 18:9-14.}. El fariseo permanecía de pie y oraba para sus adentros: `Oh Dios, te doy las gracias por no ser como los demás hombres, que son opresores, ignorantes, injustos, adúlteros, o incluso como ese publicano. Ayuno dos veces por semana, y doy el diezmo de todo lo que gano.' En cambio el publicano permanecía a lo lejos, sin atreverse siquiera a levantar los ojos al cielo, y se golpeaba el pecho, diciendo: `Dios, sé misericordioso con un pecador como yo.' Os digo que el publicano regresó a su casa con la aprobación de Dios más bien que el fariseo, porque todo aquel que se ensalza a sí mismo será humillado, pero aquel que se humilla será ensalzado».\footnote{\textit{Quien se humilla será ensalzado}: Mt 18:1-4; Mt 23:12; Lc 14:11; 18:14.}

\par 
%\textsuperscript{(1838.3)}
\textsuperscript{167:5.2} Aquella noche en Jericó, los fariseos hostiles trataron de coger al Maestro en una trampa, incitándolo a discutir sobre el matrimonio y el divorcio\footnote{\textit{Pregunta sobre el divorcio}: Mt 19:3; Mc 10:2.}, como sus colegas habían hecho anteriormente en Galilea, pero Jesús evitó hábilmente sus esfuerzos por ponerlo en un conflicto con sus leyes relacionadas con el divorcio. Así como el publicano y el fariseo ilustraban la buena y la mala religión, sus prácticas del divorcio servían para establecer un contraste entre las mejores leyes matrimoniales del código judío y la relajación vergonzosa con que los fariseos interpretaban estas reglas mosaicas sobre el divorcio. El fariseo se juzgaba a sí mismo utilizando el criterio más bajo; el publicano se medía por el ideal más elevado. Para el fariseo, la devoción era un medio de inducirle a la inactividad presuntuosa y a la certeza de una falsa seguridad espiritual; para el publicano, la devoción era un medio de despertar su alma para que comprendiera la necesidad de arrepentirse, de confesarse y de aceptar, por la fe, un perdón misericordioso. El fariseo buscaba la justicia; el publicano buscaba la misericordia. La ley del universo es: pedid y recibiréis\footnote{\textit{Pedid y recibiréis}: Mt 21:22; Mc 11:24; Jn 14:13-14; Jn 16:24.}; buscad y encontraréis\footnote{\textit{Pedid y recibiréis; buscad y encontraréis}: Mt 7:7-8; Lc 11:9-10.}.

\par 
%\textsuperscript{(1838.4)}
\textsuperscript{167:5.3} Aunque Jesús se negó a dejarse arrastrar a una controversia con los fariseos sobre el divorcio, proclamó una enseñanza positiva de los ideales más elevados relativos al matrimonio\footnote{\textit{Sobre el matrimonio y el divorcio}: Mt 19:3-12; Mc 10:2-12.}. Exaltó el matrimonio como la relación más ideal y más elevada de todas las relaciones humanas. Asimismo, insinuó su enérgica desaprobación por las prácticas de divorcio relajadas e injustas de los judíos de Jerusalén, que en aquella época permitían que un hombre se divorciara de su esposa por las razones más insignificantes, tales como ser una mala cocinera, una ama de casa deficiente, o simplemente porque se había enamorado de una mujer más bonita.

\par 
%\textsuperscript{(1839.1)}
\textsuperscript{167:5.4} Los fariseos habían llegado incluso a enseñar que este tipo de divorcio fácil era una dispensa especial concedida al pueblo judío, y a los fariseos en particular. Y así, aunque Jesús se negó a hacer declaraciones sobre el matrimonio y el divorcio, censuró muy severamente estas burlas vergonzosas de la relación matrimonial, y señaló su injusticia para con las mujeres y los niños. Nunca aprobó una práctica de divorcio que proporcionara al hombre alguna ventaja sobre la mujer; el Maestro sólo apoyaba aquellas enseñanzas que concedían a las mujeres la igualdad con los hombres.

\par 
%\textsuperscript{(1839.2)}
\textsuperscript{167:5.5} Aunque Jesús no ofreció unos nuevos preceptos que rigieran el matrimonio y el divorcio, instó a los judíos a que cumplieran con sus propias leyes y con sus enseñanzas más elevadas. Recurrió constantemente a las Escrituras en su esfuerzo por mejorar las prácticas de todas estas conductas sociales. A la vez que defendía así los conceptos elevados e ideales del matrimonio, Jesús evitó hábilmente contradecir a sus interrogadores respecto a las prácticas sociales representadas en sus leyes escritas, o en sus privilegios de divorcio, tan apreciados por ellos.

\par 
%\textsuperscript{(1839.3)}
\textsuperscript{167:5.6} A los apóstoles les resultaba muy difícil comprender la desgana que mostraba el Maestro en hacer declaraciones positivas sobre los problemas científicos, sociales, económicos y políticos. No se daban plenamente cuenta de que su misión terrenal estaba exclusivamente interesada en las revelaciones de las verdades espirituales y religiosas.

\par 
%\textsuperscript{(1839.4)}
\textsuperscript{167:5.7} Después de que Jesús hubiera hablado sobre el matrimonio y el divorcio, más tarde aquella misma noche sus apóstoles le hicieron muchas preguntas adicionales en privado, y sus respuestas a estas preguntas liberaron sus mentes de muchos conceptos equivocados\footnote{\textit{Conceptos equivocados}: Mc 10:10.}. Al final de esta conferencia, Jesús dijo: «El matrimonio es honorable y todos los hombres deberían desearlo. El hecho de que el Hijo del Hombre continúe solo su misión terrenal, no es de ninguna manera un rechazo a la deseabilidad del matrimonio. Es voluntad del Padre que yo actúe de esta manera, pero el mismo Padre ha ordenado la creación del hombre y de la mujer\footnote{\textit{Creación del hombre y la mujer}: Gn 1:27; Mt 19:4; Mc 10:6.}, y es voluntad divina que los hombres y las mujeres encuentren su servicio más elevado, y la alegría consiguiente, estableciendo un hogar para recibir y criar a los hijos, en cuya creación estos padres se convierten en asociados de los Hacedores del cielo y de la Tierra. Por esta razón, el hombre dejará a su padre y a su madre para unirse a su mujer, y los dos se volverán como uno solo»\footnote{\textit{Se volverán como uno solo}: Gn 2:24; Mt 19:5; Mc 10:7-8; Ef 5:31.}.

\par 
%\textsuperscript{(1839.5)}
\textsuperscript{167:5.8} De esta manera, Jesús liberó la mente de los apóstoles de un gran número de preocupaciones acerca del matrimonio, y aclaró muchos malentendidos relacionados con el divorcio; al mismo tiempo, contribuyó mucho a realzar sus ideales de unión social y a acrecentar su respeto por las mujeres, los niños y el hogar.

\section*{6. La bendición de los niños}
\par 
%\textsuperscript{(1839.6)}
\textsuperscript{167:6.1} El mensaje vespertino de Jesús sobre el matrimonio y la bendición que suponen los niños se difundió por todo Jericó, de manera que, a la mañana siguiente, mucho antes de que Jesús y los apóstoles se prepararan para partir, e incluso antes de la hora del desayuno, decenas de madres llegaron al lugar donde Jesús estaba alojado, trayendo a sus hijos en brazos o llevándolos de la mano, con el deseo de que bendijera a los pequeños. Cuando los apóstoles salieron y vieron esta multitud de madres con sus hijos\footnote{\textit{Las madres llevan a su hijos}: Mt 19:13; Mc 10:13; Lc 18:15.}, intentaron despedirlas, pero estas mujeres se negaron a marcharse hasta que el Maestro hubiera impuesto sus manos sobre sus hijos y los hubiera bendecido. Cuando los apóstoles reprendieron ruidosamente a estas madres, Jesús escuchó el alboroto, salió y los amonestó con indignación, diciendo: «Dejad que los niños se acerquen a mí; no se lo impidáis, porque de ellos es el reino de los cielos. En verdad, en verdad os digo que el que no reciba el reino de Dios como un niño pequeño, difícilmente entrará en él para crecer hasta la plena estatura de la madurez espiritual»\footnote{\textit{Jesús atiende a los niños}: Mt 19:14; Mc 10:14-15; Lc 18:16-17.}.

\par 
%\textsuperscript{(1840.1)}
\textsuperscript{167:6.2} Cuando el Maestro hubo hablado a sus apóstoles, recibió a todos los niños y les impuso sus manos mientras decía a sus madres palabras de ánimo y de esperanza\footnote{\textit{Jesús bendice a los niños}: Mt 19:15; Mc 10:16.}.

\par 
%\textsuperscript{(1840.2)}
\textsuperscript{167:6.3} Jesús habló con frecuencia a sus apóstoles de las mansiones celestiales y les enseñó que los hijos de Dios que progresan allí deben crecer espiritualmente, como los niños crecen físicamente en este mundo. De hecho, las cosas sagradas tienen muchas veces una apariencia corriente, y aquel día, aquellos niños y sus madres no se dieron cuenta de que las inteligencias espectadoras de Nebadon contemplaban a los niños de Jericó jugando con el Creador de un universo.

\par 
%\textsuperscript{(1840.3)}
\textsuperscript{167:6.4} La condición de la mujer en Palestina mejoró mucho gracias a las enseñanzas de Jesús; y lo mismo hubiera sucedido en todo el mundo si sus seguidores no se hubieran alejado tanto de lo que el Maestro se había esmerado en enseñarles.

\par 
%\textsuperscript{(1840.4)}
\textsuperscript{167:6.5} Fue también en Jericó, en conexión con una discusión sobre la temprana formación religiosa de los niños en los hábitos de la adoración divina, donde Jesús inculcó a sus apóstoles el gran valor de la belleza como influencia que conduce al impulso de adorar, especialmente entre los niños. Mediante sus preceptos y su ejemplo, el Maestro enseñó el valor de adorar al Creador en medio de los contornos naturales de la creación. Prefería comunicarse con el Padre celestial en medio de los árboles y entre las humildes criaturas del mundo natural. Sentía el regocijo de contemplar al Padre a través del espectáculo inspirador de los reinos estrellados de los Hijos Creadores.

\par 
%\textsuperscript{(1840.5)}
\textsuperscript{167:6.6} Cuando no es posible adorar a Dios en los tabernáculos de la naturaleza, los hombres deberían hacer todo lo posible por tener unas casas llenas de belleza, unos santuarios con una sencillez atrayente y una decoración artística, para que puedan despertarse las emociones humanas más elevadas en asociación con un acercamiento intelectual a la comunión espiritual con Dios. La verdad, la belleza y la santidad son unas ayudas poderosas y eficaces para la verdadera adoración. Pero la comunión espiritual no se fomenta con unos simples adornos masivos y el embellecimiento exagerado del arte esmerado y ostentoso del hombre. La belleza es más religiosa cuando es más sencilla y semejante a la naturaleza. ¡Es una pena que los niños pequeños tengan su primer contacto con los conceptos de la adoración en público en unas salas frías y estériles, tan desprovistas del atractivo de la belleza y tan vacías de toda insinuación a la alegría y a la santidad inspiradora! El niño debería ser iniciado a la adoración en el mundo de la naturaleza, y después acompañar a sus padres a los edificios públicos de las asambleas religiosas, que posean al menos tanto atractivo material y belleza artística como el hogar donde vive cada día.

\section*{7. La conversación sobre los ángeles}
\par 
%\textsuperscript{(1840.6)}
\textsuperscript{167:7.1} Mientras viajaban por las colinas desde Jericó a Betania, Natanael caminó casi todo el tiempo al lado de Jesús, y su discusión sobre la relación de los niños con el reino de los cielos les llevó indirectamente a reflexionar sobre el ministerio de los ángeles. Natanael le hizo finalmente al Maestro la pregunta siguiente: «Puesto que el sumo sacerdote es un saduceo, y en vista de que los saduceos no creen en los ángeles, ¿qué vamos a enseñarle al pueblo sobre los ministros celestiales?» Entonces Jesús, entre otras cosas, dijo:

\par 
%\textsuperscript{(1841.1)}
\textsuperscript{167:7.2} «Las huestes angélicas son una orden distinta de seres creados; son enteramente diferentes a la orden material de criaturas mortales, y funcionan como un grupo distinto de inteligencias del universo. Los ángeles no pertenecen al grupo de criaturas llamadas en las Escrituras «los Hijos de Dios»\footnote{\textit{Los ángeles no son los «Hijos de Dios»}: Gn 6:2,4; Job 1:6; Job 2:1; Job 38:7.}; no son tampoco los espíritus glorificados de los mortales que han continuado progresando a través de las mansiones de las alturas\footnote{\textit{Las mansiones de las alturas}: Jn 14:2.}. Los ángeles son una creación directa, y no se reproducen. Las huestes angélicas solamente tienen un parentesco espiritual con la raza humana. A medida que el hombre progresa en el viaje hacia el Padre que está en el Paraíso, pasa en un momento dado por un estado de existencia semejante al de los ángeles, pero el hombre mortal nunca se convierte en un ángel».

\par 
%\textsuperscript{(1841.2)}
\textsuperscript{167:7.3} «Los ángeles no mueren nunca, como mueren los hombres. Los ángeles son inmortales, a menos que se impliquen en el pecado, como hicieron algunos de ellos con los engaños de Lucifer. Los ángeles son los servidores espirituales en el cielo, y no son infinitamente sabios ni todopoderosos. Pero todos los ángeles leales son realmente puros y santos».

\par 
%\textsuperscript{(1841.3)}
\textsuperscript{167:7.4} «¿No recuerdas que ya os he dicho en otra ocasión que si vuestros ojos espirituales fueran ungidos, entonces veríais los cielos abiertos y contemplaríais a los ángeles de Dios subiendo y bajando?\footnote{\textit{Los ángeles subiendo y bajando}: Gn 28:12; Jn 1:51.} El ministerio de los ángeles es el que hace posible que un mundo pueda mantenerse en contacto con otros mundos, porque ¿no os he dicho repetidas veces que tengo otras ovejas que no pertenecen a este redil?\footnote{\textit{Otras ovejas en otro redil}: Jn 10:16.} Estos ángeles no son los espías del mundo espiritual, que os vigilan, y luego van a contarle al Padre los pensamientos de vuestro corazón y a informarle de las acciones de la carne. El Padre no tiene necesidad de ese servicio, ya que su propio espíritu vive dentro de vosotros\footnote{\textit{El espíritu de Dios vive en vosotros}: Job 32:8,18; Is 63:10-11; Ez 37:14; Mt 10:20; Lc 17:21; Jn 17:21-23; Ro 8:9-11; 1 Co 3:16-17; 6:19; 2 Co 6:16; Gl 2:20; 1 Jn 3:24; 4:12-15; Ap 21:3.}. Pero estos espíritus angélicos se ocupan de mantener informada a una parte de la creación celestial acerca de los acontecimientos que se producen en otras partes lejanas del universo. Muchos ángeles están asignados al servicio de las razas humanas, mientras ejercen su actividad en el gobierno del Padre y en los universos de los Hijos. Cuando os enseñé que muchos de estos serafines eran espíritus ministrantes\footnote{\textit{Espíritus ministrantes}: Mt 4:6,11; 26:53; 28:2; Mc 1:13; Lc 22:43.}, no os lo decía en un lenguaje figurado ni en términos poéticos. Todo esto es verdad, independientemente de vuestra dificultad para comprender estas cosas».

\par 
%\textsuperscript{(1841.4)}
\textsuperscript{167:7.5} «Muchos de estos ángeles están ocupados en la tarea de salvar a los hombres, porque, ¿no os he hablado de la alegría seráfica\footnote{\textit{La alegría seráfica}: Lc 15:7,10.}, cuando un alma escoge abandonar el pecado y empezar la búsqueda de Dios? Os he hablado incluso de la alegría en la \textit{presencia de los ángeles} del cielo cuando un pecador se arrepiente, señalando de este modo la existencia de otras órdenes más elevadas de seres celestiales, que se ocupan igualmente del bienestar espiritual y del progreso divino del hombre mortal».

\par 
%\textsuperscript{(1841.5)}
\textsuperscript{167:7.6} «Estos ángeles también están muy relacionados con los medios a través de los cuales el espíritu del hombre es liberado de los tabernáculos de la carne y su alma acompañada hasta las mansiones del cielo\footnote{\textit{Mansiones del Cielo}: Jn 14:2.}. Los ángeles son los guías seguros y celestiales del alma del hombre durante ese período de tiempo desconocido e indeterminado que transcurre entre la muerte física y la nueva vida en las moradas espirituales».

\par 
%\textsuperscript{(1841.6)}
\textsuperscript{167:7.7} Jesús hubiera continuado hablando con Natanael sobre el ministerio de los ángeles, pero fue interrumpido por la llegada de Marta, a quien unos amigos le habían informado que el Maestro se acercaba a Betania, pues lo habían visto subir las colinas del este. Y ahora Marta se apresuraba a recibirlo.