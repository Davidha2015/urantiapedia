\chapter{Documento 168. La resurrección de Lázaro}
\par 
%\textsuperscript{(1842.1)}
\textsuperscript{168:0.1} POCO después del mediodía, Marta salió al encuentro de Jesús\footnote{\textit{Marta se encuentra con Jesús}: Jn 11:20.} cuando éste atravesaba la cima de la colina cerca de Betania. Su hermano Lázaro había muerto hacía cuatro días\footnote{\textit{Lázaro muerto durante cuatro días}: Jn 11:17.} y el domingo al anochecer había sido colocado en el sepulcro de la familia, situado en un extremo del jardín. Este mismo jueves por la mañana, habían hecho rodar la piedra a la entrada de la tumba.

\par 
%\textsuperscript{(1842.2)}
\textsuperscript{168:0.2} Cuando Marta y María enviaron a Jesús el aviso de la enfermedad de Lázaro, confiaban en que el Maestro haría algo al respecto. Sabían que su hermano estaba irremediablemente enfermo, y aunque apenas se atrevían a esperar que Jesús dejara su trabajo de enseñanza y predicación para venir a ayudarlos, tenían tanta confianza en su poder de curar las enfermedades, que pensaron que le bastaría con pronunciar las palabras curativas y Lázaro recuperaría inmediatamente la salud. Cuando Lázaro murió pocas horas después de que el mensajero saliera de Betania hacia Filadelfia, dedujeron que el Maestro no se había enterado de la enfermedad de su hermano hasta que fue demasiado tarde, hasta que ya estaba muerto desde hacía varias horas.

\par 
%\textsuperscript{(1842.3)}
\textsuperscript{168:0.3} Sin embargo, se sintieron muy desconcertadas, al igual que todos sus amigos creyentes, por el mensaje que trajo el corredor\footnote{\textit{El mensaje del corredor}: Jn 11:4.} cuando llegó a Betania el martes por la mañana. El mensajero insistió en que había oído decir a Jesús: «... esta enfermedad no le llevará realmente a la muerte». Tampoco podían comprender por qué Jesús no les había enviado ningún mensaje, ni les había ofrecido su ayuda de alguna otra manera.

\par 
%\textsuperscript{(1842.4)}
\textsuperscript{168:0.4} Muchos amigos de las aldeas vecinas, y otros de Jerusalén, vinieron a consolar a las hermanas que estaban muy afligidas\footnote{\textit{Los amigos se reúnen para consolar}: Jn 11:18-19.}. Lázaro y sus hermanas eran los hijos de un judío ilustre y acaudalado, que había sido el vecino principal del pueblecito de Betania. A pesar de que los tres habían sido, desde hacía tiempo, unos discípulos apasionados de Jesús, eran sumamente respetados por todos los que los conocían. Habían heredado unos grandes viñedos y huertos de olivos en aquellas proximidades, y el hecho de que pudieran permitirse un sepulcro privado en sus propias tierras era una prueba más de su riqueza. Sus padres ya habían sido enterrados en este sepulcro.

\par 
%\textsuperscript{(1842.5)}
\textsuperscript{168:0.5} María había renunciado a la idea de la venida de Jesús y se había entregado a su aflicción, pero Marta se aferró a la esperanza de que Jesús vendría, y la conservó hasta el momento en que hicieron rodar la piedra delante de la tumba, aquella misma mañana, y sellaron la entrada. E incluso entonces, encargó a un joven vecino que vigilara la carretera de Jericó desde la cima de la colina al este de Betania; éste fue el muchacho que le llevó a Marta la noticia de que Jesús y sus amigos se acercaban.

\par 
%\textsuperscript{(1842.6)}
\textsuperscript{168:0.6} Cuando Marta se encontró con Jesús, cayó a sus pies, exclamando: «Maestro, ¡si hubieras estado aquí, mi hermano no habría muerto!» Muchos temores atravesaban la mente de Marta, pero no expresó ninguna duda ni se atrevió a criticar o a poner en tela de juicio la conducta del Maestro en relación con la muerte de Lázaro. Cuando hubo terminado de hablar, Jesús se inclinó para levantarla, y le dijo: «Ten fe únicamente, Marta, y tu hermano resucitará»\footnote{\textit{«Lázaro resucitará»}: Jn 11:20-24.}. Entonces Marta contestó: «Sé que resucitará en la resurrección del último día; e incluso ahora creo que nuestro Padre te concederá todo lo que le pidas a Dios».

\par 
%\textsuperscript{(1843.1)}
\textsuperscript{168:0.7} Entonces Jesús miró a Marta fijamente a los ojos, y le dijo: «Yo soy la resurrección y la vida\footnote{\textit{Yo soy la resurrección y la vida}: Jn 11:25-27.}; el que cree en mí, aunque muera, vivirá. En verdad, cualquiera que vive y cree en mí no morirá nunca realmente\footnote{\textit{Quien crea en mí nunca morirá}: Dn 12:2; Mt 19:16,29; 25:46; Mc 10:17,30; Lc 10:25; 18:18,30; Jn 3:15-16,36; 4:14,36; 5:24,39; 6:27,40,47; 6:54,68; 8:51-52; 10:28; 11:25-26; 12:25,50; 17:2-3; Hch 13:46-48; Ro 2:7; 5:21; 6:22-23; Gl 6:8; 1 Ti 1:16; 6:12,19; Tit 1:2; 3:7; 1 Jn 1:2; 2:25; 3:15; 5:11,13,20; Jud 1:21; Ap 22:5.}. Marta, ¿crees esto?» Y Marta respondió al Maestro: «Sí, creo desde hace mucho tiempo que tú eres el Libertador, el Hijo del Dios vivo, aquel que debía venir a este mundo».

\par 
%\textsuperscript{(1843.2)}
\textsuperscript{168:0.8} Cuando Jesús preguntó por María, Marta se dirigió inmediatamente a la casa y le dijo a su hermana en voz baja: «El Maestro está aquí y ha preguntado por ti»\footnote{\textit{Jesús envía a por María}: Jn 11:28-31.}. Cuando María escuchó esto, se levantó en seguida y salió apresuradamente para ir a recibir a Jesús, que permanecía en el mismo lugar donde Marta lo había encontrado primero, a cierta distancia de la casa. Cuando los amigos que estaban con María, tratando de consolarla, vieron que se levantaba rápidamente y salía, la siguieron suponiendo que iba a la tumba para llorar.

\par 
%\textsuperscript{(1843.3)}
\textsuperscript{168:0.9} Muchos de los presentes eran enemigos encarnizados de Jesús. Por eso Marta había salido para encontrarse con él a solas, y por eso también había entrado para informar en secreto a María de que el Maestro había preguntado por ella. Aunque Marta anhelaba ver a Jesús, deseaba evitar que su llegada repentina en medio de un grupo numeroso de sus enemigos de Jerusalén pudiera ocasionar alguna posible situación desagradable. Marta había tenido la intención de permanecer en la casa con sus amigos mientras María iba a saludar a Jesús, pero no lo consiguió, porque todos siguieron a María, y se encontraron así de manera inesperada en presencia del Maestro.

\par 
%\textsuperscript{(1843.4)}
\textsuperscript{168:0.10} Marta llevó a María ante Jesús, y cuando ésta lo vio, cayó a sus pies, exclamando: «¡Si tan sólo hubieras estado aquí, mi hermano no hubiera muerto!» Cuando Jesús vio hasta qué punto estaban todos afligidos por la muerte de Lázaro, su alma se llenó de compasión\footnote{\textit{Jesús y María se lamentan}: Jn 11:32-33.}.

\par 
%\textsuperscript{(1843.5)}
\textsuperscript{168:0.11} Cuando los acompañantes vieron que María había ido a saludar a Jesús, se apartaron a corta distancia, mientras Marta y María hablaban con el Maestro; recibieron palabras adicionales de consuelo y una exhortación a que conservaran una fe firme en el Padre y se conformaran por completo a la voluntad divina.

\par 
%\textsuperscript{(1843.6)}
\textsuperscript{168:0.12} La mente humana de Jesús se conmovió poderosamente debido al conflicto entre su amor por Lázaro y las desoladas hermanas, y su desprecio y desdén por las muestras exteriores de afecto que manifestaban algunos de estos judíos incrédulos y con intenciones asesinas. A Jesús le causaba indignación que algunos de estos supuestos amigos mostraran una aflicción forzada y externa por Lázaro, cuando esa falsa pena estaba acompañada en sus corazones por una enemistad tan implacable contra él. Sin embargo, algunos de estos judíos eran sinceros en su luto, pues eran verdaderos amigos de la familia.

\section*{1. En la tumba de Lázaro}
\par 
%\textsuperscript{(1843.7)}
\textsuperscript{168:1.1} Después de que Jesús hubiera pasado unos momentos consolando a Marta y María, apartados de los acompañantes, les preguntó: «¿Dónde lo habéis puesto?»\footnote{\textit{¿Dónde lo habéis puesto?}: Jn 11:34.} Entonces Marta dijo: «Ven a ver». Mientras el Maestro seguía en silencio a las dos hermanas afligidas, lloró\footnote{\textit{Jesús llora, acude a la tumba}: Jn 11:35-38.}. Cuando los judíos amistosos que los seguían vieron sus lágrimas, uno de ellos dijo: «Mirad cómo lo amaba. El que abrió los ojos del ciego, ¿no podría haber impedido la muerte de este hombre?» Para entonces ya se encontraban delante del sepulcro familiar, que era una pequeña cueva natural, o declive, en el saliente de una roca de unos diez metros de altura, situada en el extremo más alejado del jardín.

\par 
%\textsuperscript{(1844.1)}
\textsuperscript{168:1.2} Es difícil explicar a la mente humana por qué exactamente lloró Jesús\footnote{\textit{¿Por qué lloró Jesús?}: Jn 11:35.}. Aunque tenemos acceso al registro de las emociones humanas y de los pensamientos divinos conjuntos de Jesús, tal como constan en la mente del Ajustador Personalizado, no estamos totalmente seguros de la causa real de estas manifestaciones emocionales. Tendemos a creer que Jesús lloró debido a una cantidad de pensamientos y sentimientos que atravesaban su mente en aquel momento, tales como:

\par 
%\textsuperscript{(1844.2)}
\textsuperscript{168:1.3} 1. Sentía una compasión sincera y dolorosa por Marta y María; tenía un afecto humano real y profundo por estas hermanas que habían perdido a su hermano.

\par 
%\textsuperscript{(1844.3)}
\textsuperscript{168:1.4} 2. Se sentía mentalmente agitado por la presencia de la multitud de acompañantes, algunos sinceros y otros simplemente hipócritas. Siempre le molestaban estas manifestaciones exteriores de duelo. Sabía que las hermanas amaban a su hermano y que tenían fe en la supervivencia de los creyentes. Estas emociones contradictorias quizás explican por qué lloró cuando se acercaban a la tumba.

\par 
%\textsuperscript{(1844.4)}
\textsuperscript{168:1.5} 3. Dudaba sinceramente en devolverle a Lázaro la vida mortal. Sus hermanas lo necesitaban realmente, pero Jesús lamentaba tener que llamar a su amigo para que luego tuviera que experimentar una cruel persecución; sabía muy bien que Lázaro tendría que sufrirla por haber sido el objeto de la demostración más grande de poder divino del Hijo del Hombre.

\par 
%\textsuperscript{(1844.5)}
\textsuperscript{168:1.6} Y ahora podemos contar un hecho interesante e instructivo: Aunque este relato se desarrolla como un acontecimiento aparentemente natural y normal de los asuntos humanos, tiene algunos aspectos indirectos muy interesantes. Aunque el mensajero fue a ver a Jesús el domingo para informarle de la enfermedad de Lázaro, y aunque Jesús envió un mensaje indicando que «no le llevaría a la muerte»\footnote{\textit{No le llevaría a la muerte}: Jn 11:4.}, sin embargo fue personalmente hasta Betania, e incluso preguntó a las hermanas: «¿Dónde lo habéis puesto?»\footnote{\textit{¿Dónde lo habéis puesto?}: Jn 11:34.} Todo esto parece indicar que el Maestro actuaba a la manera de esta vida y de acuerdo con los conocimientos limitados de la mente humana. Sin embargo, los archivos del universo revelan que el Ajustador Personalizado de Jesús emitió unas órdenes para que se retuviera indefinidamente en el planeta al Ajustador del Pensamiento de Lázaro, después de su muerte, y que esta orden se registró apenas quince minutos antes de que Lázaro exhalara su último suspiro.

\par 
%\textsuperscript{(1844.6)}
\textsuperscript{168:1.7} ¿Sabía la mente divina de Jesús, incluso antes de que Lázaro muriera, que lo resucitaría de entre los muertos? No lo sabemos. Sólo sabemos lo que indicamos aquí.

\par 
%\textsuperscript{(1844.7)}
\textsuperscript{168:1.8} Muchos enemigos de Jesús tendían a mofarse de sus manifestaciones de afecto, y decían entre ellos: «Si tanto apreciaba a este hombre, ¿por qué esperó tanto para venir a Betania? Si él es lo que ellos pretenden, ¿por qué no ha salvado a su querido amigo? ¿Para qué sirve curar a los desconocidos en Galilea, si no puede salvar a los que ama?» Y de otras muchas maneras se burlaron y le restaron importancia a las obras y enseñanzas de Jesús.

\par 
%\textsuperscript{(1844.8)}
\textsuperscript{168:1.9} Y así, hacia las dos y media de la tarde de este jueves, todo el escenario estaba preparado en esta pequeña aldea de Betania para la representación de la obra más grande de todas las relacionadas con el ministerio terrenal de Miguel de Nebadon, para la manifestación más grande de poder divino que se produjo durante su encarnación, puesto que su propia resurrección tuvo lugar después de que hubiera sido liberado de las cadenas de la morada mortal.

\par 
%\textsuperscript{(1845.1)}
\textsuperscript{168:1.10} El pequeño grupo reunido delante de la tumba de Lázaro poco podía imaginar que allí cerca se encontraba presente una enorme multitud de todas las órdenes de seres celestiales, congregados bajo la dirección de Gabriel y ahora en espera por mandato del Ajustador Personalizado de Jesús, vibrando de expectación y preparados para ejecutar las órdenes de su amado Soberano.

\par 
%\textsuperscript{(1845.2)}
\textsuperscript{168:1.11} Cuando Jesús pronunció aquellas palabras, ordenando: «Quitad la piedra»\footnote{\textit{Quitad la piedra}: Jn 11:39a.}, las huestes celestiales reunidas se prepararon para representar el drama de la resurrección de Lázaro en la similitud de su carne mortal. Esta forma de resurrección implica unas dificultades de ejecución que trascienden de lejos la técnica habitual de la resurrección de las criaturas mortales en el estado morontial, y necesita muchas más personalidades celestiales y una organización mucho mayor de recursos universales.

\par 
%\textsuperscript{(1845.3)}
\textsuperscript{168:1.12} Cuando Marta y María escucharon este mandato de Jesús ordenando que se quitara la piedra que estaba delante de la tumba, se llenaron de emociones contradictorias. María esperaba que Lázaro fuera resucitado de entre los muertos, pero Marta, aunque compartía hasta cierto punto la fe de su hermana, estaba más preocupada por el temor de que la apariencia de Lázaro no fuera presentable para Jesús, los apóstoles y sus amigos. Marta dijo: «¿Tenemos que quitar la piedra? Mi hermano ya lleva muerto cuatro días, de manera que la descomposición del cuerpo ya ha empezado»\footnote{\textit{¿Por qué? El cuerpo está descompuesto}: Jn 11:39b.}. Marta dijo esto también porque no estaba segura de la razón por la que el Maestro había pedido que se apartara la piedra; pensaba que Jesús quizás sólo quería echarle una última mirada a Lázaro. La actitud de Marta no era firme ni constante. Como dudaban en quitar la piedra, Jesús dijo: «¿No os he dicho desde el principio que esta enfermedad no le llevaría a la muerte? ¿No he venido para cumplir mi promesa? Y después de llegar hasta vosotras, ¿no he dicho que, si tan sólo creyerais, veríais la gloria de Dios?\footnote{\textit{Si creéis veréis la gloria de Dios}: Jn 11:40.} ¿Por qué dudáis? ¿Cuánto tiempo necesitaréis para creer y obedecer?»

\par 
%\textsuperscript{(1845.4)}
\textsuperscript{168:1.13} Cuando Jesús hubo terminado de hablar, sus apóstoles, con la ayuda de unos vecinos voluntarios, agarraron la piedra y la hicieron rodar hasta quitarla de la entrada de la tumba\footnote{\textit{La piedra es hecha rodar}: Jn 11:41a.}.

\par 
%\textsuperscript{(1845.5)}
\textsuperscript{168:1.14} Los judíos tenían la creencia común de que la gota de hiel situada en la punta de la espada del ángel de la muerte empezaba a actuar al final del tercer día, de manera que la totalidad de su efecto se producía al cuarto día. Admitían que el alma del hombre podía demorarse cerca de la tumba hasta el final del tercer día, tratando de reanimar el cadáver; pero creían firmemente que antes del amanecer del cuarto día, ese alma se había ido a la morada de los espíritus difuntos.

\par 
%\textsuperscript{(1845.6)}
\textsuperscript{168:1.15} Estas creencias y opiniones acerca de los muertos y de la partida de los espíritus de los muertos, sirvieron para asegurar en la mente de todos los que ahora estaban presentes en la tumba de Lázaro, y de todos los que pudieran enterarse posteriormente de lo que estaba a punto de suceder, que éste era un caso real y verdadero de resurrección de entre los muertos, debido a un acto personal de aquel que había declarado ser «la resurrección y la vida»\footnote{\textit{La resurrección y la vida}: Jn 11:25.}.

\section*{2. La resurrección de Lázaro}
\par 
%\textsuperscript{(1845.7)}
\textsuperscript{168:2.1} Mientras este grupo de unos cuarenta y cinco mortales permanecía delante de la tumba, pudieron ver vagamente la forma de Lázaro, envuelta en unos vendajes de lino, descansando en el nicho inferior derecho de la cueva fúnebre. Mientras estas criaturas terrenales se hallaban allí en silencio, casi sin aliento, una enorme hueste de seres celestiales se había situado en sus puestos preliminares, para responder a la señal de actuar en cuanto la diera su comandante Gabriel.

\par 
%\textsuperscript{(1846.1)}
\textsuperscript{168:2.2} Jesús levantó los ojos y dijo: «Padre, te doy las gracias por haber escuchado y concedido mi petición. Sé que me escuchas siempre, pero te hablo así a causa de aquellos que están aquí conmigo, para que puedan creer que me has enviado al mundo, y para que sepan que actúas conmigo en esto que estamos a punto de realizar». Cuando hubo terminado de orar, dijo en voz alta: «Lázaro, ¡sal fuera!»\footnote{\textit{La oración y el «Lázaro, ¡sal fuera!»}: Jn 11:41b-43.}

\par 
%\textsuperscript{(1846.2)}
\textsuperscript{168:2.3} Los espectadores humanos permanecieron inmóviles, pero toda la inmensa hueste celestial bullía en una acción unificada, obedeciendo la palabra del Creador. En sólo doce segundos del tiempo terrestre, la forma hasta entonces inanimada de Lázaro empezó a moverse, y pronto se sentó en el borde de la plataforma de piedra donde había descansado. Su cuerpo estaba envuelto en las mortajas y su rostro cubierto con un paño. Mientras permanecía de pie delante de ellos ---vivo--- Jesús dijo: «Desatadlo y dejadlo salir»\footnote{\textit{Desatadlo y dejadlo salir}: Jn 11:44.}.

\par 
%\textsuperscript{(1846.3)}
\textsuperscript{168:2.4} Todos los espectadores, salvo los apóstoles así como Marta y María, huyeron hacia la casa. Estaban pálidos de terror y abrumados por el asombro. Aunque algunos permanecieron allí, muchos regresaron apresuradamente a sus hogares.

\par 
%\textsuperscript{(1846.4)}
\textsuperscript{168:2.5} Lázaro saludó a Jesús y a los apóstoles, preguntó por el significado de las mortajas y por qué se había despertado en el jardín. Jesús y los apóstoles se apartaron, mientras Marta le contaba a Lázaro su muerte, entierro y resurrección. Tuvo que explicarle que había muerto el domingo y que ahora había sido devuelto a la vida el jueves, ya que Lázaro no había tenido conciencia del tiempo desde que había caído en el sueño de la muerte.

\par 
%\textsuperscript{(1846.5)}
\textsuperscript{168:2.6} Mientras Lázaro salía de la tumba, el Ajustador Personalizado de Jesús, ahora jefe de su orden en este universo local, ordenó al antiguo Ajustador de Lázaro, entonces en espera, que volviera a residir en la mente y el alma del resucitado.

\par 
%\textsuperscript{(1846.6)}
\textsuperscript{168:2.7} Luego Lázaro se acercó a Jesús y, junto con sus hermanas, se arrodilló a los pies del Maestro para dar gracias y alabar a Dios. Jesús cogió a Lázaro de la mano, y lo levantó diciendo: «Hijo mío, lo que te ha sucedido será experimentado también por todos los que creen en este evangelio, excepto que serán resucitados con una forma más gloriosa. Serás un testigo viviente de la verdad que he proclamado ---yo soy la resurrección y la vida\footnote{\textit{La resurrección y la vida}: Jn 11:25.}. Pero ahora entremos todos en la casa y tomemos algún alimento para estos cuerpos físicos».

\par 
%\textsuperscript{(1846.7)}
\textsuperscript{168:2.8} Mientras caminaban hacia la casa, Gabriel disolvió los grupos adicionales de las huestes celestiales reunidas, y procedió a registrar el primer y último caso, sucedido en Urantia, en el que una criatura mortal había sido resucitada en la similitud de su cuerpo físico mortal.

\par 
%\textsuperscript{(1846.8)}
\textsuperscript{168:2.9} Lázaro apenas podía comprender lo que había sucedido. Sabía que había estado muy enfermo, pero sólo podía recordar que se había dormido y que había sido despertado. Nunca pudo decir nada sobre aquellos cuatro días en la tumba, porque había estado totalmente inconsciente. El tiempo no existe para aquellos que duermen el sueño de la muerte.

\par 
%\textsuperscript{(1846.9)}
\textsuperscript{168:2.10} Muchos creyeron en Jesús a consecuencia de esta obra poderosa, pero otros sólo endurecieron su corazón para rechazarlo aún más\footnote{\textit{Creyentes y no creyentes}: Jn 11:45-47.}. Al día siguiente al mediodía, esta historia se había difundido por todo Jerusalén. Decenas de hombres y mujeres fueron a Betania para contemplar a Lázaro y hablar con él, y los fariseos, alarmados y desconcertados, convocaron apresuradamente una reunión del sanedrín para determinar lo que había que hacer con respecto a estos nuevos acontecimientos.

\section*{3. La reunión del sanedrín}
\par 
%\textsuperscript{(1847.1)}
\textsuperscript{168:3.1} Aunque el testimonio de este hombre resucitado de entre los muertos contribuyó mucho a consolidar la fe de la masa de creyentes en el evangelio del reino, tuvo poca o ninguna influencia sobre la actitud de los jefes y dirigentes religiosos de Jerusalén, excepto que apresuró su decisión de destruir a Jesús y de poner fin a su obra.

\par 
%\textsuperscript{(1847.2)}
\textsuperscript{168:3.2} Al día siguiente, viernes, el sanedrín se reunió a la una para deliberar de nuevo sobre la cuestión: «¿Qué vamos a hacer con Jesús de Nazaret?»\footnote{\textit{¿Qué hacemos con Jesús?}: Jn 11:47b.} Después de más de dos horas de discusiones y debates enconados, cierto fariseo propuso una resolución pidiendo la muerte inmediata de Jesús, proclamando que era una amenaza para todo Israel y comprometiendo formalmente al sanedrín para que decidiera su muerte, sin juicio y haciendo caso omiso de todo precedente.

\par 
%\textsuperscript{(1847.3)}
\textsuperscript{168:3.3} Este augusto cuerpo de dirigentes judíos había decretado una y otra vez que Jesús debía ser apresado y sometido a juicio, inculpado de blasfemia y de otras muchas acusaciones de desacato a la ley sagrada judía. En una ocasión anterior habían llegado incluso a declarar que debía morir, pero ésta era la primera vez que el sanedrín indicaba el deseo de decretar su muerte con antelación a todo juicio. Pero esta resolución no fue puesta a votación, ya que catorce miembros del sanedrín dimitieron en masa cuando se propuso esta acción inaudita. Aunque estas dimisiones no tuvieron efecto oficial durante casi dos semanas, este grupo de catorce se separó del sanedrín aquel día y no volvió a sentarse nunca más en el consejo. Cuando estas dimisiones fueron aceptadas posteriormente, cinco miembros más fueron expulsados porque sus colegas opinaban que albergaban sentimientos amistosos hacia Jesús. Con la expulsión de estos diecinueve hombres, el sanedrín estaba en disposiciones de juzgar y condenar a Jesús con una solidaridad que rozaba la unanimidad.

\par 
%\textsuperscript{(1847.4)}
\textsuperscript{168:3.4} A la semana siguiente, Lázaro y sus hermanas fueron convocados ante el sanedrín. Después de haberse escuchado el testimonio de los tres, no se podía albergar ninguna duda de que Lázaro había sido resucitado de entre los muertos. Aunque los anales del sanedrín admitían prácticamente la resurrección de Lázaro, el registro contenía una resolución que atribuía este prodigio, y todos los demás realizados por Jesús, al poder del príncipe de los demonios, declarándose que Jesús estaba aliado con él.

\par 
%\textsuperscript{(1847.5)}
\textsuperscript{168:3.5} Sea cual fuere el origen de su poder para realizar prodigios, estos dirigentes judíos estaban persuadidos de que si no lo paraban de inmediato, muy pronto toda la gente corriente creería en él, y que además surgirían graves complicaciones con las autoridades romanas, puesto que muchos de sus creyentes lo consideraban como el Mesías, el libertador de Israel\footnote{\textit{Jesús es visto como una amenaza}: Jn 11:48.}.

\par 
%\textsuperscript{(1847.6)}
\textsuperscript{168:3.6} En esta misma reunión del sanedrín fue donde el sumo sacerdote Caifás expresó por primera vez el viejo dicho judío, que luego repitió tantas veces: «Es mejor que muera un solo hombre, a que perezca la comunidad»\footnote{\textit{Es mejor que sólo muera un hombre}: Jn 11:49-50; Jn 18:14.}.

\par 
%\textsuperscript{(1847.7)}
\textsuperscript{168:3.7} Aunque Jesús había recibido aviso de las acciones del sanedrín durante este sombrío viernes por la tarde, no se inquietó en lo más mínimo y continuó descansando todo el sábado con unos amigos en Betfagé, una aldea cercana a Betania. El domingo por la mañana temprano, Jesús y los apóstoles se reunieron, como habían convenido, en la casa de Lázaro, se despidieron de la familia de Betania, y emprendieron su viaje de vuelta al campamento de Pella\footnote{\textit{Jesús regresa}: Jn 11:53-54.}.

\section*{4. La respuesta a la oración}
\par 
%\textsuperscript{(1848.1)}
\textsuperscript{168:4.1} En el camino desde Betania a Pella, los apóstoles hicieron muchas preguntas a Jesús y el Maestro contestó sin reparos a todas ellas, excepto a las relacionadas con los detalles de la resurrección de los muertos. Estos problemas sobrepasaban la capacidad de comprensión de sus apóstoles, y por eso el Maestro rehusó discutir estas cuestiones con ellos. Como habían partido de Betania en secreto, nadie los acompañaba. Por consiguiente, Jesús aprovechó la ocasión para decirle muchas cosas a los diez que, en su opinión, los prepararía para los días difíciles que se avecinaban.

\par 
%\textsuperscript{(1848.2)}
\textsuperscript{168:4.2} Los apóstoles tenían la mente muy excitada y pasaron bastante tiempo discutiendo de sus experiencias recientes relacionadas con la oración y la respuesta a la oración. Todos recordaban la declaración que Jesús había hecho en Filadelfia al mensajero de Betania, cuando dijo claramente: «Esta enfermedad no le llevará realmente a la muerte»\footnote{\textit{La enfermedad no es para la muerte}: Jn 11:4.}. Sin embargo, a pesar de esta promesa, Lázaro había muerto realmente. Durante todo aquel día, volvieron a hablar una y otra vez de este problema de la respuesta a la oración.

\par 
%\textsuperscript{(1848.3)}
\textsuperscript{168:4.3} Las respuestas de Jesús a sus numerosas preguntas se pueden resumir como sigue:

\par 
%\textsuperscript{(1848.4)}
\textsuperscript{168:4.4} 1. La oración es una expresión de la mente finita en su esfuerzo por acercarse al Infinito. Por consiguiente, la formulación de una oración está necesariamente limitada por el conocimiento, la sabiduría y los atributos de lo finito; del mismo modo, la respuesta ha de estar condicionada por la visión, los objetivos, los ideales y las prerrogativas del Infinito. Nunca se puede observar una continuidad ininterrumpida de fenómenos materiales entre la formulación de una oración y la recepción de la plena respuesta espiritual a la misma.

\par 
%\textsuperscript{(1848.5)}
\textsuperscript{168:4.5} 2. Cuando una oración se queda aparentemente sin respuesta, el retraso es a menudo el presagio de una respuesta mejor, aunque esa respuesta se demore considerablemente por alguna buena razón. Cuando Jesús dijo que la enfermedad de Lázaro no le llevaría realmente hasta la muerte, éste ya había muerto hacía once horas. Ninguna oración sincera se queda sin respuesta, salvo cuando el punto de vista superior del mundo espiritual ha concebido una respuesta mejor, una respuesta que satisface la petición del espíritu del hombre en contraposición con la oración de la simple mente humana.

\par 
%\textsuperscript{(1848.6)}
\textsuperscript{168:4.6} 3. Cuando las oraciones temporales son compuestas por el espíritu y expresadas con fe, a menudo son tan amplias y abarcan tantas cosas que sólo se pueden contestar en la eternidad; a veces, la súplica finita está tan llena del deseo de captar lo Infinito, que la respuesta debe ser aplazada durante mucho tiempo a fin de esperar la creación de la capacidad adecuada para recibirla; la oración de la fe puede abarcar tanto, que la respuesta sólo se puede recibir en el Paraíso.

\par 
%\textsuperscript{(1848.7)}
\textsuperscript{168:4.7} 4. Las respuestas a la oración de la mente mortal son a menudo de tal naturaleza, que sólo se pueden recibir y reconocer después de que esa misma mente que ora ha alcanzado el estado inmortal. Muchas veces, la oración de un ser material sólo se puede contestar cuando ese individuo ha progresado hasta el nivel del espíritu.

\par 
%\textsuperscript{(1848.8)}
\textsuperscript{168:4.8} 5. La oración de una persona que conoce a Dios puede estar tan distorsionada por la ignorancia y tan deformada por la superstición, que responder a la misma sería muy poco deseable. Los seres espirituales intermedios tienen entonces que traducir de tal manera esa oración que, cuando llega la respuesta, el peticionario no logra reconocer en absoluto que se trata de la respuesta a su oración.

\par 
%\textsuperscript{(1848.9)}
\textsuperscript{168:4.9} 6. Todas las oraciones verdaderas son dirigidas a los seres espirituales, y todas esas peticiones deben ser contestadas en términos espirituales, y todas esas respuestas deben consistir en realidades espirituales. Los seres espirituales no pueden ofrecer respuestas materiales ni siquiera a las súplicas espirituales de los seres materiales. Los seres materiales sólo pueden orar eficazmente cuando «oran en espíritu»\footnote{\textit{Orar en el espíritu}: Jn 4:23-24; Ro 8:26-27; 1 Co 14:14-19; Ef 6:18.}.

\par 
%\textsuperscript{(1849.1)}
\textsuperscript{168:4.10} 7. Ninguna oración puede esperar una respuesta a menos que haya nacido del espíritu y haya sido alimentada por la fe\footnote{\textit{Alimentada por la fe}: Mt 21:21-22; Mc 11:24.}. Vuestra fe sincera implica que habéis concedido prácticamente de antemano, a los que escuchan vuestra oración, el pleno derecho de contestar a vuestras súplicas de acuerdo con esa sabiduría suprema y ese amor divino que, según describe vuestra fe, impulsan siempre a esos seres a quienes dirigís vuestras oraciones.

\par 
%\textsuperscript{(1849.2)}
\textsuperscript{168:4.11} 8. El niño siempre está en su derecho cuando se atreve a dirigir una petición al padre; y el padre siempre cumple con sus obligaciones paternales hacia el niño inmaduro cuando su sabiduría superior le dicta que retrase la respuesta a la súplica del niño, la modifique, la divida, la trascienda o la aplace hasta otra fase de su ascensión espiritual.

\par 
%\textsuperscript{(1849.3)}
\textsuperscript{168:4.12} 9. No vaciléis en formular las oraciones que expresan los anhelos del espíritu\footnote{\textit{Anhelos del espíritu}: Stg 5:15-16.}; no dudéis de que vuestras súplicas recibirán una respuesta. Esas respuestas permanecerán en depósito, esperando a que hayáis alcanzado, en este mundo o en otros, esos niveles espirituales futuros de verdadera consecución cósmica, en los que os será posible reconocer y apropiaros de las respuestas tanto tiempo esperadas a vuestras peticiones anteriores pero inoportunas.

\par 
%\textsuperscript{(1849.4)}
\textsuperscript{168:4.13} 10. Todas las súplicas sinceras nacidas del espíritu recibirán, con certeza, una respuesta. Pedid y recibiréis\footnote{\textit{Pedid y recibiréis}: Mt 7:7-8; 21:22; Mc 11:24; Lc 11:9-10; Jn 14:13-14; 16:24.}. Pero debéis recordar que sois unas criaturas que progresan en el tiempo y el espacio; por eso tenéis que contar constantemente con el factor espacio-temporal en vuestra experiencia de recibir personalmente las respuestas completas a vuestras diversas oraciones y peticiones.

\section*{5. ¿Qué fue de Lázaro?}
\par 
%\textsuperscript{(1849.5)}
\textsuperscript{168:5.1} Lázaro permaneció en su casa de Betania, donde fue un centro de gran interés para muchos creyentes sinceros y numerosos curiosos, hasta la semana de la crucifixión de Jesús, momento en que recibió la advertencia de que el sanedrín había decretado su muerte. Los dirigentes de los judíos estaban decididos a poner fin a la difusión ulterior de las enseñanzas de Jesús, y estimaron acertadamente que sería inútil hacer morir a Jesús si permitían que Lázaro, el cual representaba el apogeo mismo de sus obras prodigiosas, viviera y diera testimonio del hecho de que Jesús lo había resucitado de entre los muertos. Lázaro ya había sufrido crueles persecuciones por parte de ellos.

\par 
%\textsuperscript{(1849.6)}
\textsuperscript{168:5.2} Así pues, Lázaro se despidió apresuradamente de sus hermanas en Betania, huyó hacia Jericó, atravesó el Jordán, y no se permitió ningún largo descanso hasta haber llegado a Filadelfia. Lázaro conocía bien a Abner, y aquí se sentía a salvo de las intrigas asesinas del malvado sanedrín.

\par 
%\textsuperscript{(1849.7)}
\textsuperscript{168:5.3} Poco después de esto, Marta y María vendieron sus tierras de Betania y se reunieron con su hermano en Perea. Entretanto, Lázaro se había convertido en el tesorero de la iglesia de Filadelfia. Apoyó firmemente a Abner en su controversia con Pablo y la iglesia de Jerusalén, y murió finalmente, a los 67 años de edad, de la misma enfermedad que se lo había llevado en Betania cuando era más joven.