\chapter{Documento 172. La entrada en Jerusalén}
\par 
%\textsuperscript{(1878.1)}
\textsuperscript{172:0.1} JESÚS y los apóstoles llegaron a Betania poco después de las cuatro de la tarde del viernes 31 de marzo del año 30. Lázaro, sus hermanas y sus amigos los estaban esperando; en vista de que un gran número de personas venía diariamente para hablar con Lázaro sobre su resurrección, Jesús fue informado de que se había preparado todo para que se alojara con un creyente vecino, un tal Simón, el ciudadano principal de aquel pueblecito desde la muerte del padre de Lázaro.

\par 
%\textsuperscript{(1878.2)}
\textsuperscript{172:0.2} Aquella tarde, Jesús recibió a muchos visitantes, y la gente común de Betania y Betfagé hizo todo lo posible para que se sintiera bienvenido. Muchos creían que Jesús iba ahora a Jerusalén, desafiando por completo el decreto de muerte del sanedrín, para proclamarse rey de los judíos, pero la familia de Betania ---Lázaro, Marta y María--- comprendía más plenamente que el Maestro no era un rey de ese tipo; sentían vagamente que ésta podía ser su última visita a Jerusalén y Betania.

\par 
%\textsuperscript{(1878.3)}
\textsuperscript{172:0.3} Los jefes de los sacerdotes fueron informados de que Jesús estaba alojado en Betania, pero pensaron que sería mejor no intentar capturarlo entre sus amigos\footnote{\textit{La trama}: Jn 11:57.}; decidieron esperar a que entrara en Jerusalén. Jesús sabía todo esto, pero conservaba una calma majestuosa; sus amigos nunca lo habían visto más tranquilo y agradable; incluso los apóstoles estaban sorprendidos de que estuviera tan indiferente, cuando el sanedrín había pedido a todos los judíos que se lo entregaran. Mientras el Maestro dormía aquella noche, los apóstoles estuvieron vigilando de dos en dos, y muchos de ellos se habían ceñido la espada. A la mañana siguiente temprano, fueron despertados por cientos de peregrinos que venían de Jerusalén, aunque fuera sábado, para ver a Jesús\footnote{\textit{Mucha gente para ver a Jesús}: Jn 12:9.} y a Lázaro, a quien había resucitado de entre los muertos.

\section*{1. El sábado en Betania}
\par 
%\textsuperscript{(1878.4)}
\textsuperscript{172:1.1} Los peregrinos que venían de fuera de Judea, así como las autoridades judías, se habían preguntado: «¿Qué pensáis? ¿Vendrá Jesús a la fiesta?» Por ello, la gente se alegró cuando escuchó que Jesús estaba en Betania, pero los jefes de los sacerdotes y de los fariseos estaban un poco perplejos\footnote{\textit{Los jefes especulan sobre su visita}: Jn 11:55-56.}. Se sentían contentos de tenerlo bajo su jurisdicción, pero estaban algo desconcertados por su audacia; recordaban que en su visita anterior a Betania, Lázaro había sido resucitado de entre los muertos, y Lázaro se estaba convirtiendo en un gran problema para los enemigos de Jesús.

\par 
%\textsuperscript{(1878.5)}
\textsuperscript{172:1.2} Seis días antes de la Pascua, la tarde después del sábado, todo Betania y todo Betfagé se reunió para celebrar la llegada de Jesús con un banquete\footnote{\textit{El banquete}: Mt 26:6; Mc 14:3a; Jn 12:1-2.} público en la casa de Simón. Esta cena era en honor de Jesús y de Lázaro, y fue ofrecida desafiando al sanedrín. Marta dirigía el servicio de la comida; su hermana María se encontraba entre las espectadoras, porque era contrario a la costumbre de los judíos que una mujer se sentara en un banquete público. Los agentes del sanedrín estaban presentes, pero temían arrestar a Jesús en medio de sus amigos.

\par 
%\textsuperscript{(1879.1)}
\textsuperscript{172:1.3} Jesús conversó con Simón sobre el Josué de antaño, cuyo nombre era homónimo del suyo, y contó cómo Josué y los israelitas habían llegado a Jerusalén a través de Jericó. Al comentar la leyenda del derrumbamiento de las murallas de Jericó\footnote{\textit{Las murallas de Jericó}: Jos 6:20.}, Jesús dijo: «No me ocupo de esas murallas de ladrillo y de piedra; pero quisiera que las murallas del prejuicio, de la presunción y del odio se desmoronaran delante de esta predicación del amor del Padre por todos los hombres».

\par 
%\textsuperscript{(1879.2)}
\textsuperscript{172:1.4} El banquete continuó de una manera muy alegre y normal, salvo que todos los apóstoles estaban más serios que de costumbre. Jesús estaba excepcionalmente alegre y había jugado con los niños hasta el momento de sentarse a la mesa.

\par 
%\textsuperscript{(1879.3)}
\textsuperscript{172:1.5} No sucedió nada extraordinario hasta cerca del final del festín, cuando María, la hermana de Lázaro, se salió del grupo de espectadoras, avanzó hasta el lugar donde Jesús estaba reclinado como huésped de honor, y se puso a abrir un gran frasco de alabastro que contenía un ung\"uento muy raro y costoso. Después de ungir la cabeza del Maestro, empezó a verterlo sobre sus pies, y luego se soltó los cabellos para secárselos con ellos\footnote{\textit{La unción de Jesús por una mujer}: Mt 26:7-9; Mc 14:3b-5; Jn 11:2; Jn 12:3-5.}. El olor del ung\"uento impregnó toda la casa, y todos los presentes se asombraron por lo que María había hecho. Lázaro no dijo nada, pero cuando alguna gente murmuró manifestando su indignación porque un ung\"uento tan caro se utilizara de esta manera, Judas Iscariote se dirigió al lugar donde Andrés estaba reclinado y dijo: «¿Por qué no se ha vendido ese ung\"uento y se ha dado el dinero para alimentar a los pobres? Deberías decirle al Maestro que censure este derroche».

\par 
%\textsuperscript{(1879.4)}
\textsuperscript{172:1.6} Sabiendo lo que pensaban y escuchando lo que decían, Jesús puso su mano sobre la cabeza de María, que estaba arrodillada a su lado, y con una expresión de bondad en su rostro, dijo: «Que cada uno de vosotros la deje en paz. ¿Por qué la molestáis con esto, ya que ha hecho una buena cosa según su corazón? A vosotros que murmuráis y decís que este ung\"uento debería haberse vendido y el dinero entregado a los pobres, dejad que os diga que a los pobres los tendréis siempre con vosotros, de manera que podréis ayudarlos en cualquier momento que os parezca bien. Pero yo no estaré siempre con vosotros; pronto iré hacia mi Padre. Esta mujer ha guardado este ung\"uento durante mucho tiempo para cuando entierren mi cuerpo; y puesto que le ha parecido bien efectuar esta unción anticipándose a mi muerte, esa satisfacción no le será denegada. Al hacer esto, María os ha reprendido a todos, en el sentido de que con este acto manifiesta su fe en lo que he dicho sobre mi muerte y ascensión hacia mi Padre que está en los cielos. Esta mujer no será recriminada por lo que ha hecho esta noche; os digo más bien que en las eras por venir, en cualquier parte del mundo que se predique este evangelio, lo que ella ha hecho se contará en memoria suya».\footnote{\textit{Jesús defiende a María}: Mt 26:10-13; Mc 14:6-9; Jn 12:7-8.}

\par 
%\textsuperscript{(1879.5)}
\textsuperscript{172:1.7} A causa de esta reprimenda, tomada por una recriminación personal, Judas Iscariote se decidió finalmente a buscar venganza para sus sentimientos heridos\footnote{\textit{Judas se decide a la traición}: Mt 26:14-16; Mc 14:10-11; Lc 22:3-6.}. Muchas veces había albergado estas ideas de manera subconsciente, pero ahora se atrevía a considerar estos pensamientos perversos en su mente clara y consciente. Otras muchas personas lo animaron en esta actitud, pues el precio de este ung\"uento equivalía al salario de un hombre durante un año ---suficiente para abastecer de pan a cinco mil personas\footnote{\textit{El precio del ung\"uento}: Mt 26:7; Mc 14:3; Jn 12:3.}. Pero María amaba a Jesús; había adquirido este precioso ung\"uento para embalsamar su cuerpo después de muerto, pues creía en sus palabras cuando les advertía que tenía que morir; y no se le iba a privar de ello si había cambiado de idea y escogido otorgar esta ofrenda al Maestro mientras aún estaba vivo.

\par 
%\textsuperscript{(1879.6)}
\textsuperscript{172:1.8} Tanto Lázaro como Marta sabían que María había tardado mucho tiempo en ahorrar el dinero destinado a comprar este frasco de nardo, y aprobaban por completo que actuara en este asunto según los deseos de su corazón, pues eran ricos y podían permitirse fácilmente hacer esta ofrenda.

\par 
%\textsuperscript{(1880.1)}
\textsuperscript{172:1.9} Cuando los jefes de los sacerdotes tuvieron noticia de esta cena en Betania en honor de Jesús y Lázaro, empezaron a consultarse para ver lo que debían hacer con Lázaro. Decidieron enseguida que Lázaro también tenía que morir. Concluyeron, con toda la razón, que sería inútil ejecutar a Jesús si dejaban vivir a Lázaro, a quien Jesús había resucitado de entre los muertos.

\section*{2. El domingo por la mañana con los apóstoles}
\par 
%\textsuperscript{(1880.2)}
\textsuperscript{172:2.1} Aquel domingo por la mañana, en el hermoso jardín de Simón, el Maestro convocó a sus doce apóstoles a su alrededor y les dio sus instrucciones finales antes de entrar en Jerusalén. Les dijo que probablemente pronunciaría muchos discursos y enseñaría numerosas lecciones antes de volver hacia el Padre, pero aconsejó a los apóstoles que se abstuvieran de hacer cualquier trabajo público durante esta estancia para pasar la Pascua en Jerusalén. Les indicó que permanecieran cerca de él y que «vigilaran y oraran». Jesús sabía que muchos de sus apóstoles y seguidores inmediatos llevaban sus espadas escondidas en aquel mismo momento, pero no hizo ninguna alusión a este hecho.

\par 
%\textsuperscript{(1880.3)}
\textsuperscript{172:2.2} Estas instrucciones matutinas abarcaron un breve repaso del ministerio de los apóstoles desde el día de su ordenación, cerca de Cafarnaúm, hasta este día en que se preparaban para entrar en Jerusalén. Los apóstoles escucharon en silencio, y no hicieron ninguna pregunta.

\par 
%\textsuperscript{(1880.4)}
\textsuperscript{172:2.3} Aquella mañana temprano, David Zebedeo había entregado a Judas los fondos obtenidos con la venta del equipo del campamento de Pella, y Judas a su vez había puesto la mayor parte de este dinero en manos de Simón, su anfitrión, para que lo guardara en lugar seguro en previsión de las necesidades de su entrada en Jerusalén.

\par 
%\textsuperscript{(1880.5)}
\textsuperscript{172:2.4} Después de la conferencia con los apóstoles, Jesús mantuvo una conversación con Lázaro y le indicó que evitara sacrificar su vida al espíritu vengativo del sanedrín. Obedeciendo esta recomendación, Lázaro huyó unos días después a Filadelfia, cuando los agentes del sanedrín enviaron a unos hombres para que lo arrestaran.

\par 
%\textsuperscript{(1880.6)}
\textsuperscript{172:2.5} En cierto modo, todos los seguidores de Jesús sentían la crisis inminente, pero la jovialidad inhabitual y el buen humor excepcional del Maestro impidieron que se dieran plenamente cuenta de la gravedad de la situación.

\section*{3. La partida hacia Jerusalén}
\par 
%\textsuperscript{(1880.7)}
\textsuperscript{172:3.1} Betania estaba a unos tres kilómetros del templo, y era la una y media de aquel domingo por la tarde cuando Jesús se preparó para salir hacia Jerusalén. Sentía un profundo afecto por Betania y su gente sencilla. Nazaret, Cafarnaúm y Jerusalén lo habían rechazado, pero Betania lo había aceptado, había creído en él. Fue en este pueblecito, en el que casi todos los hombres, mujeres y niños eran creyentes, donde Jesús escogió realizar la obra más poderosa de su donación terrenal: la resurrección de Lázaro. No resucitó a Lázaro para que los habitantes pudieran creer, sino más bien porque ya creían.

\par 
%\textsuperscript{(1880.8)}
\textsuperscript{172:3.2} Jesús había reflexionado toda la mañana sobre su entrada en Jerusalén. Hasta ese momento, siempre se había esforzado por impedir que el público lo aclamara como el Mesías, pero ahora la situación era diferente. Se estaba acercando al final de su carrera en la carne, el sanedrín había decretado su muerte, y no iba a pasar nada porque permitiera a sus discípulos que expresaran libremente sus sentimientos, tal como hubiera ocurrido si hubiera elegido hacer una entrada oficial y pública en la ciudad.

\par 
%\textsuperscript{(1881.1)}
\textsuperscript{172:3.3} Jesús no decidió efectuar esta entrada pública en Jerusalén como un último intento por hacerse con el favor popular, ni como una tentativa final para obtener el poder. Tampoco lo hizo del todo para satisfacer los anhelos humanos de sus discípulos y apóstoles. Jesús no albergaba ninguna de las ilusiones de un soñador fantasioso; sabía muy bien cuál iba a ser el desenlace de esta visita.

\par 
%\textsuperscript{(1881.2)}
\textsuperscript{172:3.4} Después de haber decidido hacer una entrada pública en Jerusalén, el Maestro se vio enfrentado a la necesidad de escoger un método apropiado para ejecutar esta resolución. Jesús reflexionó sobre las numerosas profecías, más o menos contradictorias, llamadas mesiánicas, pero sólo parecía haber una que pudiera seguir de manera apropiada\footnote{\textit{Planificando la entrada en Jerusalén}: Mt 21:4-5; Jn 12:14b-15.}. La mayoría de estas declaraciones proféticas describían a un rey, el hijo y sucesor de David, un hombre audaz y enérgico que liberaría temporalmente a todo Israel del yugo de la dominación extranjera. Pero había un pasaje en las Escrituras que a veces había sido asociado con el Mesías por parte de aquellos que más defendían el concepto espiritual de su misión; Jesús consideró que podría utilizar coherentemente este pasaje como guía para la entrada que proyectaba hacer en Jerusalén. Este escrito se encontraba en Zacarías y decía: «Regocíjate mucho, oh hija de Sión; da gritos de júbilo, oh hija de Jerusalén. He aquí que tu rey viene hacia ti. Es justo y trae la salvación. Viene como alguien humilde, montado en un asno, en un pollino, el hijo de una burra»\footnote{\textit{Entrada sobre un asno}: Zac 9:9.}.

\par 
%\textsuperscript{(1881.3)}
\textsuperscript{172:3.5} Un rey guerrero siempre entraba en una ciudad montado a caballo; un rey en misión de paz y de amistad siempre entraba montado en un asno. Jesús no quería entrar en Jerusalén a lomos de un caballo, pero estaba dispuesto a entrar pacíficamente y con buena voluntad, subido en un burro, como el Hijo del Hombre.

\par 
%\textsuperscript{(1881.4)}
\textsuperscript{172:3.6} Jesús había intentado durante mucho tiempo, mediante una enseñanza directa, inculcar a sus apóstoles y a sus discípulos que su reino no era de este mundo, que se trataba de un asunto puramente espiritual; pero no había tenido éxito en este esfuerzo. Ahora quería intentar realizar, mediante un gesto simbólico, aquello que no había conseguido hacer por medio de una enseñanza clara y personal. En consecuencia, inmediatamente después del almuerzo, Jesús llamó a Pedro y a Juan y les ordenó que fueran a Betfagé, un pueblo vecino un poco retirado de la carretera principal, a corta distancia al noroeste de Betania. Les dijo además: «Id a Betfagé, y cuando lleguéis al cruce de los caminos, encontraréis el pollino de una burra atado allí. Desatad el pollino y traedlo con vosotros. Si alguien os pregunta por qué hacéis esto, decid simplemente: `El Maestro lo necesita.'»\footnote{\textit{Los apóstoles son enviados a por el asno}: Mt 21:1-3; Mc 11:1-3; Lc 19:29-31.} Cuando los dos apóstoles fueron a Betfagé tal como el Maestro les había ordenado, encontraron al pollino atado en la calle al lado de su madre y cerca de una casa de esquina. Mientras Pedro empezó a desatar el pollino, llegó el dueño y preguntó por qué hacían eso. Cuando Pedro le contestó lo que Jesús les había ordenado, el hombre dijo: «Si vuestro Maestro es Jesús de Galilea, el pollino está a su disposición». Y así regresaron llevando al pollino con ellos\footnote{\textit{Los apóstoles se hacen con el asno}: Mt 21:6-7a; Mc 11:6-7a; Lc 19:32-35a.}.

\par 
%\textsuperscript{(1881.5)}
\textsuperscript{172:3.7} Entretanto, varios cientos de peregrinos se habían reunido alrededor de Jesús y de sus apóstoles. Desde media mañana, los visitantes que pasaban camino de la Pascua se habían detenido allí. Mientras tanto, David Zebedeo y algunos de sus antiguos mensajeros decidieron dirigirse apresuradamente a Jerusalén, donde difundieron eficazmente la noticia, entre las multitudes de peregrinos que visitaban el templo, de que Jesús de Nazaret iba a hacer una entrada triunfal en la ciudad. En consecuencia, varios miles de estos visitantes acudieron en masa para saludar a este profeta, autor de prodigios, del que tanto se hablaba, y que algunos creían que era el Mesías. Esta multitud que salía de Jerusalén encontró a Jesús y al gentío que se dirigía hacia la ciudad poco después de que hubieran pasado la cima del Olivete, y hubieran empezado a descender hacia la ciudad\footnote{\textit{Se reúne la multitud}: Jn 12:17-18.}.

\par 
%\textsuperscript{(1882.1)}
\textsuperscript{172:3.8} Cuando la procesión partió de Betania, había un gran entusiasmo en la alegre multitud de discípulos, creyentes y peregrinos visitantes, muchos de ellos procedentes de Galilea y Perea. Justo antes de partir, las doce mujeres del cuerpo femenino original, acompañadas por algunas de sus asociadas, llegaron al lugar y se unieron a esta procesión excepcional que se dirigía alegremente hacia la ciudad.

\par 
%\textsuperscript{(1882.2)}
\textsuperscript{172:3.9} Antes de partir, los gemelos Alfeo colocaron sus mantos encima del asno y lo sujetaron mientras se subía el Maestro. A medida que la procesión avanzaba hacia la cima del Olivete, la alegre multitud echaba al suelo sus prendas de vestir y traía ramas de los árboles cercanos para hacerle una alfombra de honor al asno que llevaba al Hijo real, al Mesías prometido. Mientras la multitud jubilosa continuaba avanzando hacia Jerusalén\footnote{\textit{Entrada triunfal en Jerusalén}: Mt 21:7b-9; Mc 11:7b-10; Lc 19:35b-38; Jn 12:12-14a.}, empezaron a cantar, o más bien a gritar al unísono, el salmo: «Hosanna al hijo de David; bendito es el que viene en nombre del Señor\footnote{\textit{Bendito es el que viene en nombre del Señor}: Sal 118:26.}. Hosanna en las alturas. Bendito sea el reino que desciende del cielo».

\par 
%\textsuperscript{(1882.3)}
\textsuperscript{172:3.10} Jesús se mostró alegre y jovial durante el trayecto hasta que llegó a la cumbre del Olivete, desde donde se tenía una vista panorámica sobre la ciudad y las torres del templo; el Maestro detuvo allí la procesión, y un gran silencio se apoderó de todos mientras lo veían llorar. Bajando la mirada sobre la inmensa multitud que salía de la ciudad para recibirlo, el Maestro, con mucha emoción y una voz llorosa, dijo: «¡Oh Jerusalén, si tan sólo hubieras conocido, tú también, al menos en este día tuyo, las cosas que pertenecen a tu paz, y que podrías haber tenido con tanta profusión! Pero ahora estas glorias están a punto de ocultarse a tus ojos. Estás a punto de rechazar al Hijo de la Paz y de volverle la espalda al evangelio de la salvación. Pronto vendrán los días en que tus enemigos abrirán una trinchera a tu alrededor, y te asediarán por todas partes; te destruirán por completo, de manera que no quedará piedra sobre piedra. Y todo esto te sucederá porque no has reconocido la hora de tu visita divina. Estás a punto de rechazar el regalo de Dios, y todos los hombres te rechazarán»\footnote{\textit{Lamento de Jesús por Jerusalén}: Lc 19:41-44.}.

\par 
%\textsuperscript{(1882.4)}
\textsuperscript{172:3.11} Cuando hubo terminado de hablar, empezaron a descender del Olivete y pronto se reunieron con la multitud de visitantes que venía de Jerusalén ondeando ramas de palmera, gritando hosannas y expresando de otras maneras su regocijo y su buena hermandad. El Maestro no había planeado que estas multitudes salieran de Jerusalén para encontrarse con ellos; fue obra de otras personas. Nunca premeditó nada que fuera teatral.

\par 
%\textsuperscript{(1882.5)}
\textsuperscript{172:3.12} Junto con la multitud que afluía para dar la bienvenida al Maestro, también venían muchos fariseos y otros enemigos suyos. Estaban tan perturbados por esta explosión repentina e inesperada de aclamación popular, que tuvieron miedo de arrestarlo, por temor a que esta acción precipitara una revuelta abierta del pueblo. Temían enormemente la actitud de la gran cantidad de visitantes, que habían oído hablar mucho de Jesús, y gran número de los cuales creían en él.

\par 
%\textsuperscript{(1882.6)}
\textsuperscript{172:3.13} Al acercarse a Jerusalén, la multitud se volvió más expresiva, tanto que algunos fariseos se abrieron paso hasta Jesús y dijeron: «Instructor, deberías reprender a tus discípulos y exhortarlos a que se comporten de una manera más correcta». Jesús respondió: «Es muy adecuado que estos hijos den la bienvenida al Hijo de la Paz, a quien los jefes de los sacerdotes han rechazado. Sería inútil detenerlos, no sea que estas piedras al borde del camino se pongan a gritar en su lugar»\footnote{\textit{Jesús defiende a sus acompañantes}: Lc 19:39-40.}.

\par 
%\textsuperscript{(1882.7)}
\textsuperscript{172:3.14} Los fariseos se apresuraron a adelantarse a la procesión para volver al sanedrín, que entonces estaba reunido en el templo, e informaron a sus colegas: «Mirad, todo lo que hacemos no sirve para nada; estamos confundidos por ese galileo. La gente se ha vuelto loca por él; si no detenemos a esos ignorantes, todo el mundo le seguirá»\footnote{\textit{Los fariseos llevan sus informes al Sanedrín}: Lc 12:19.}.

\par 
%\textsuperscript{(1883.1)}
\textsuperscript{172:3.15} En realidad, no había que atribuir ningún significado profundo a esta explosión superficial y espontánea de entusiasmo popular. Esta bienvenida, aunque alegre y sincera, no representaba ninguna convicción real o profunda en el corazón de esta multitud jubilosa. Esta misma muchedumbre estuvo igualmente dispuesta a rechazar rápidamente a Jesús, más tarde aquella misma semana, en cuanto el sanedrín hubo tomado una posición firme y decidida contra él, cuando perdieron sus ilusiones ---cuando se dieron cuenta de que Jesús no iba a establecer el reino de acuerdo con sus esperanzas albergadas durante mucho tiempo.

\par 
%\textsuperscript{(1883.2)}
\textsuperscript{172:3.16} Pero toda la ciudad estaba extraordinariamente agitada, de manera que todo el mundo preguntaba: «¿Quién es ese hombre?» Y la multitud contestaba: «Es Jesús de Nazaret, el profeta de Galilea»\footnote{\textit{Muchos preguntan «¿quién es Jesús?»}: Mt 21:10-11a.}.

\section*{4. La visita al templo}
\par 
%\textsuperscript{(1883.3)}
\textsuperscript{172:4.1} Mientras los gemelos Alfeo devolvían el asno a su dueño, Jesús y los diez apóstoles se separaron de sus asociados inmediatos y se pasearon por el templo\footnote{\textit{Jesús visita el templo}: Mc 11:11a.}, observando los preparativos para la Pascua. No se hizo ningún intento por molestar a Jesús, ya que el sanedrín temía mucho al pueblo, y después de todo, ésa era una de las razones por las que Jesús había permitido que la multitud lo aclamara de aquella manera. Los apóstoles apenas comprendían que éste era el único procedimiento humano que podía impedir, de manera eficaz, que Jesús fuera arrestado inmediatamente en cuanto entrara en la ciudad. El Maestro deseaba dar a los habitantes de Jerusalén, destacados y humildes, así como a las decenas de miles de visitantes para la Pascua, esta última oportunidad adicional de escuchar el evangelio y de recibir, si querían, al Hijo de la Paz.

\par 
%\textsuperscript{(1883.4)}
\textsuperscript{172:4.2} Ahora, mientras avanzaba la tarde y las multitudes iban en busca de alimento, Jesús y sus seguidores inmediatos se quedaron solos. ¡Qué día tan extraño había sido! Los apóstoles estaban pensativos, pero mudos. En todos sus años de asociación con Jesús, nunca habían visto un día como éste. Se sentaron un rato cerca del tesoro del templo, observando cómo la gente dejaba caer sus contribuciones: los ricos ponían mayores cantidades en la caja de las ofrendas, y todos daban algo según sus posibilidades. Al final llegó una pobre viuda, vestida miserablemente, y observaron que echaba dos ébolos
(pequeñas monedas de cobre) en el embudo. Entonces Jesús llamó la atención de los apóstoles sobre la viuda, diciendo: «Retened bien lo que acabáis de ver. Esa pobre viuda ha echado más que todos los demás, porque todos los demás han echado, como don, una pequeña parte de lo que les sobraba, pero esa pobre mujer, aunque está necesitada, ha dado todo lo que tenía, incluso su sustento»\footnote{\textit{Los ébolos de la viuda}: Mc 12:41-44; Lc 21:1-4.}.

\par 
%\textsuperscript{(1883.5)}
\textsuperscript{172:4.3} A medida que avanzaba la tarde, caminaron en silencio por los patios del templo, y después de haber observado una vez más estas escenas familiares, Jesús recordó las emociones asociadas a sus visitas anteriores, sin excluir las primeras, y dijo: «Subamos a Betania para descansar»\footnote{\textit{Regreso a Betania}: Mc 11:11b.}. Jesús, con Pedro y Juan, fueron a la casa de Simón, mientras que los demás apóstoles se alojaron con sus amigos de Betania y Betfagé.

\section*{5. La actitud de los apóstoles}
\par 
%\textsuperscript{(1883.6)}
\textsuperscript{172:5.1} Este domingo por la tarde, mientras regresaban a Betania, Jesús caminó delante de los apóstoles. No se dijo ni una palabra hasta que se separaron después de llegar a la casa de Simón. Nunca hubo doce seres humanos que experimentaran unas emociones tan diversas e inexplicables como las que surgían ahora en la mente y en el alma de estos embajadores del reino. Estos robustos galileos estaban confusos y desconcertados; no sabían qué esperar inmediatamente después; estaban demasiado sorprendidos como para sentirse muy asustados. No sabían nada de los planes del Maestro para el día siguiente, y no hicieron ninguna pregunta. Se fueron a sus alojamientos, aunque no durmieron mucho, a excepción de los gemelos. Pero no mantuvieron una vigilia armada alrededor de Jesús en la casa de Simón\footnote{\textit{La actitud de los apóstoles}: Jn 12:16a.}.

\par 
%\textsuperscript{(1884.1)}
\textsuperscript{172:5.2} Andrés estaba totalmente desconcertado, casi desorientado. Fue el único apóstol que no intentó evaluar seriamente la explosión popular de aclamaciones. Estaba demasiado preocupado por la idea de su responsabilidad como jefe del cuerpo apostólico, como para analizar seriamente el sentido o el significado de los ruidosos hosannas de la multitud. Andrés estaba atareado vigilando a algunos de sus compañeros, pues temía que se dejaran llevar por sus emociones durante la agitación popular, especialmente Pedro, Santiago, Juan y Simón Celotes. Durante todo este día y los que siguieron inmediatamente después, Andrés estuvo preocupado con serias dudas, pero nunca expresó ninguno de estos recelos a sus compañeros apostólicos. Le inquietaba la actitud de algunos de los doce, pues sabía que estaban armados con espadas; pero ignoraba que su propio hermano Pedro llevaba una de aquellas armas. Así pues, la procesión hacia Jerusalén sólo causó en Andrés una impresión relativamente superficial; estaba demasiado atareado con las responsabilidades de su cargo como para sentirse afectado por otras cosas.

\par 
%\textsuperscript{(1884.2)}
\textsuperscript{172:5.3} Simón Pedro se sintió al principio casi arrebatado por esta manifestación popular de entusiasmo; pero se había serenado notablemente en el momento de regresar aquella noche a Betania. Pedro simplemente no podía imaginar qué es lo que pretendía hacer el Maestro. Estaba terriblemente desilusionado porque Jesús no había aprovechado esta oleada de favor popular para hacer algún tipo de declaración. Pedro no podía comprender por qué Jesús no había hablado a la multitud cuando llegaron al templo, o al menos permitido que uno de los apóstoles se dirigiera al gentío. Pedro era un gran predicador, y le disgustaba ver cómo se desaprovechaba un auditorio tan amplio, tan receptivo y tan entusiasta. Le hubiera gustado tanto predicar el evangelio del reino a este gentío allí mismo en el templo; pero el Maestro les había encargado expresamente que no debían enseñar ni predicar en Jerusalén durante esta semana de la Pascua. La reacción a la espectacular procesión hacia la ciudad fue desastrosa para Simón Pedro; cuando llegó la noche, estaba pensativo y con una tristeza indecible.

\par 
%\textsuperscript{(1884.3)}
\textsuperscript{172:5.4} Para Santiago Zebedeo, este domingo fue un día de perplejidad y de profunda confusión; no conseguía captar el significado de lo que estaba ocurriendo; no podía comprender la intención del Maestro, que permitía estas aclamaciones desenfrenadas, y luego se negaba a decir una palabra a la gente cuando llegaron al templo. Mientras la procesión descendía del Olivete hacia Jerusalén, y más particularmente cuando se encontraron con los miles de peregrinos que salían para acoger al Maestro, Santiago se sintió cruelmente desgarrado entre sus emociones contradictorias de exaltación y satisfacción por lo que veía, y su profundo sentimiento de temor por lo que podía ocurrir cuando llegaran al templo. Luego se sintió abatido y abrumado por la decepción cuando Jesús se bajó del asno y se puso a caminar tranquilamente por los patios del templo. Santiago no podía comprender por qué se desperdiciaba una oportunidad tan magnífica para proclamar el reino. Por la noche, una angustiosa y terrible incertidumbre dominaba su mente.

\par 
%\textsuperscript{(1884.4)}
\textsuperscript{172:5.5} Juan Zebedeo estuvo a punto de comprender por qué Jesús había actuado así; al menos captó parcialmente el significado espiritual de esta supuesta entrada triunfal en Jerusalén. Mientras la multitud se dirigía hacia el templo y Juan observaba a su Maestro sentado a horcajadas en el pollino, recordó que anteriormente había escuchado a Jesús citar el pasaje de las Escrituras, la declaración de Zacarías, que describía la llegada del Mesías como un hombre de paz que entraba en Jerusalén montado en un asno\footnote{\textit{La profecía de Zacarías}: Zac 9:9.}. Mientras Juan le daba vueltas a esta Escritura en su cabeza, empezó a comprender el significado simbólico del espectáculo de este domingo por la tarde\footnote{\textit{La profecía de Zacarías, cumplida}: Zac 9:9.}. Al menos captó el suficiente significado de esta Escritura como para permitirle disfrutar un poco del episodio e impedir deprimirse con exceso por el final aparentemente sin sentido de la procesión triunfal. Juan tenía un tipo de mente que tendía de manera natural a pensar y a sentir en símbolos.

\par 
%\textsuperscript{(1885.1)}
\textsuperscript{172:5.6} Felipe estaba completamente trastornado por lo inesperado y la espontaneidad de la explosión. Mientras descendían del Olivete, no pudo ordenar suficientemente sus pensamientos como para llegar a una opinión determinada sobre el significado de toda esta manifestación. En cierto modo, disfrutó del espectáculo porque su Maestro estaba siendo honrado. Cuando llegaron al templo, le inquietó la idea de que Jesús quizás pudiera pedirle que alimentara a la multitud, de manera que el comportamiento de Jesús de apartarse deliberadamente del gentío, que tan amargamente había desilusionado a la mayoría de los apóstoles, fue un gran alivio para Felipe. Las multitudes habían sido a veces una gran prueba para el administrador de los doce. Después de haberse liberado de estos temores personales referentes a las necesidades materiales del gentío, Felipe se unió a Pedro para expresar su desilusión porque no se había hecho nada por enseñar a la multitud. Aquella noche, Felipe se puso a reflexionar sobre estas experiencias, y estuvo tentado de poner en duda toda la idea del reino; se preguntaba honradamente qué podían significar todas estas cosas, pero no expresó sus dudas a nadie; amaba demasiado a Jesús como para hacer una cosa así. Tenía una gran fe personal en el Maestro.

\par 
%\textsuperscript{(1885.2)}
\textsuperscript{172:5.7} Natanael, aparte de apreciar los aspectos simbólicos y proféticos, fue el que estuvo más cerca de comprender las razones que tenía el Maestro para ganarse el apoyo popular de los peregrinos de la Pascua. Antes de llegar al templo, estuvo razonando que, sin esta entrada espectacular en Jerusalén, Jesús hubiera sido arrestado por los agentes del sanedrín y arrojado en un calabozo en cuanto se hubiera atrevido a entrar en la ciudad. Así pues, no le sorprendió en absoluto que el Maestro dejara de utilizar a la alegre multitud en cuanto se encontró dentro de los muros de la ciudad, después de haber impresionado tan poderosamente a los dirigentes judíos como para que éstos se abstuvieran de proceder a su arresto inmediato. Al comprender la verdadera razón que tenía el Maestro para entrar en la ciudad de esta manera, Natanael siguió adelante con naturalidad y con más equilibrio, y se sintió menos perturbado y desilusionado que los otros apóstoles por la conducta posterior de Jesús. Natanael tenía una gran confianza en la aptitud de Jesús para comprender a los hombres, así como en su sagacidad y destreza para manejar las situaciones difíciles.

\par 
%\textsuperscript{(1885.3)}
\textsuperscript{172:5.8} Mateo se sintió al principio confundido por esta manifestación espectacular. No captó el significado de lo que veían sus ojos hasta que se acordó también del escrito de Zacarías, en el que el profeta aludía al regocijo de Jerusalén porque había llegado su rey trayendo la salvación y montado en el pollino de una burra. Mientras la procesión avanzaba en dirección a la ciudad y luego se dirigía hacia el templo, Mateo se quedó extasiado; estaba seguro de que algo extraordinario iba a suceder cuando el Maestro llegara al templo a la cabeza de esta multitud que lo aclamaba. Cuando uno de los fariseos se mofó de Jesús, diciendo: «¡Mirad todos, mirad quién viene aquí: el rey de los judíos montado en un asno!», Mateo tuvo que hacer un gran esfuerzo para no ponerle las manos encima. Aquel atardecer, ninguno de los doce estaba más deprimido que él durante el camino de vuelta a Betania. Después de Simón Pedro y Simón Celotes, Mateo fue quien experimentó la mayor tensión nerviosa y por la noche estaba agotado. Pero por la mañana ya estaba mucho más animado; después de todo, era un buen perdedor.

\par 
%\textsuperscript{(1886.1)}
\textsuperscript{172:5.9} Tomás fue el hombre más desconcertado y confundido de los doce. La mayor parte del tiempo se limitó a seguir a los demás, contemplando el espectáculo y preguntándose honradamente cuál podía ser el motivo del Maestro para participar en una manifestación tan peculiar. En lo más profundo de su corazón, consideraba toda esta representación como un poco infantil, si no absolutamente disparatada. Nunca había visto a Jesús hacer una cosa semejante, y no sabía cómo explicar su extraña conducta de este domingo por la tarde. Cuando llegaron al templo, Tomás había deducido que la finalidad de esta demostración popular era asustar de tal manera al sanedrín que no se atrevieran a arrestar inmediatamente al Maestro. En el camino de vuelta a Betania, Tomás reflexionó mucho, pero no dijo nada. En el momento de acostarse, la habilidad del Maestro para organizar esta entrada tumultuosa en Jerusalén había empezado a despertar un poco su sentido del humor, y se sintió muy animado por esta reacción.

\par 
%\textsuperscript{(1886.2)}
\textsuperscript{172:5.10} Este domingo empezó siendo un gran día para Simón Celotes. Imaginaba las cosas maravillosas que se harían en Jerusalén los próximos días, y en esto tenía razón, pero Simón soñaba con el establecimiento de la nueva soberanía nacional de los judíos, con Jesús sentado en el trono de David. Simón veía a los nacionalistas entrar en acción en cuanto se anunciara el reino, y se veía a sí mismo al mando supremo de las fuerzas militares, en vías de congregarse, del nuevo reino. Durante el descenso del Olivete, llegó incluso a imaginar que el sanedrín y todos sus partidarios estarían muertos antes de que el Sol se pusiera aquel día. Creía realmente que algo extraordinario iba a suceder. Era el hombre más ruidoso de toda la multitud. Pero a las cinco de la tarde, era un apóstol silencioso, abatido y desilusionado. Nunca se recuperó por completo de la depresión que se apoderó de él a consecuencia de la conmoción de este día; al menos, no hasta mucho tiempo después de la resurrección del Maestro.

\par 
%\textsuperscript{(1886.3)}
\textsuperscript{172:5.11} Para los gemelos Alfeo, éste fue un día perfecto. Lo disfrutaron realmente hasta el fin, y como no estuvieron presentes durante la tranquila visita al templo, se libraron en gran parte de la decepción que siguió a la agitación popular. No podían comprender de ninguna manera el comportamiento abatido de los apóstoles cuando regresaban a Betania aquella noche. En la memoria de los gemelos, éste fue siempre el día en que se sintieron más cerca del cielo en la Tierra. Este día fue la culminación satisfactoria de toda su carrera como apóstoles. El recuerdo de la euforia de este domingo por la tarde los sostuvo durante toda la tragedia de esta semana memorable, hasta el mismo momento de la crucifixión. Fue la entrada real más apropiada que los gemelos podían imaginar; disfrutaron cada momento del espectáculo. Aprobaron plenamente todo lo que vieron y conservaron el recuerdo durante mucho tiempo.

\par 
%\textsuperscript{(1886.4)}
\textsuperscript{172:5.12} De todos los apóstoles, Judas Iscariote fue el que estuvo más desfavorablemente afectado por esta entrada procesional en Jerusalén. Su mente estaba desagradablemente agitada porque el Maestro le había reprendido el día anterior a causa de la unción de María durante la fiesta en casa de Simón. Judas estaba disgustado con todo el espectáculo. Le parecía infantil, si no francamente ridículo. Mientras este apóstol vengativo contemplaba los acontecimientos de este domingo por la tarde, le daba la impresión de que Jesús se parecía más a un payaso que a un rey. Le molestaba enormemente todo el espectáculo. Compartía el punto de vista de los griegos y de los romanos, que despreciaban a todo el que consintiera en montarse en un asno o en el pollino de una burra. Cuando la procesión triunfal hubo entrado en la ciudad, Judas casi había decidido abandonar toda idea de un reino semejante; estaba casi resuelto a renunciar a todas estas tentativas absurdas para establecer el reino de los cielos. Luego se acordó de la resurrección de Lázaro y de otras muchas cosas, y decidió permanecer con los doce, al menos un día más. Además, llevaba la bolsa, y no quería desertar con los fondos apostólicos en su poder. Aquella noche, durante el camino de vuelta a Betania, su conducta no pareció extraña puesto que todos los apóstoles estaban igualmente deprimidos y silenciosos.

\par 
%\textsuperscript{(1887.1)}
\textsuperscript{172:5.13} Judas se dejó influir enormemente por las burlas de sus amigos saduceos. En su determinación final de abandonar a Jesús y a sus compañeros apóstoles, ningún otro factor ejerció una influencia tan poderosa sobre él como cierto episodio que se produjo en el preciso momento en que Jesús llegaba a la puerta de la ciudad: Un distinguido saduceo (amigo de la familia de Judas) se precipitó hacia éste con el ánimo de burlarse jovialmente de él, le dio una palmada en la espalda, y le dijo: «¿Por qué tienes tan mala cara, mi buen amigo? Anímate y únete a todos nosotros para aclamar a ese Jesús de Nazaret, el rey de los judíos, mientras atraviesa las puertas de Jerusalén montado en un burro». Judas nunca había retrocedido ante las persecuciones, pero no podía soportar este tipo de burlas. A su sentimiento de venganza, alimentado durante largo tiempo, se sumaba ahora este miedo mortal al ridículo, este sentimiento terrible y espantoso de sentir verg\"uenza de su Maestro y de sus compañeros apóstoles. En su corazón, este embajador ordenado del reino ya era un desertor; sólo le quedaba encontrar una excusa plausible para romper abiertamente con el Maestro.