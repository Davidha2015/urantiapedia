\chapter{Documento 173. El lunes en Jerusalén}
\par 
%\textsuperscript{(1888.1)}
\textsuperscript{173:0.1} TAL COMO habían planeado de antemano, este lunes por la mañana temprano Jesús y los apóstoles se reunieron en la casa de Simón en Betania y, después de una breve conferencia, partieron para Jerusalén\footnote{\textit{Jesús va a Jerusalén}: Mc 11:12a; Jn 2:13.}. Los doce estaban extrañamente silenciosos mientras se dirigían hacia el templo; no se habían recuperado de la experiencia del día anterior. Estaban expectantes, temerosos y profundamente afectados por cierto sentimiento de distanciamiento que tenía su origen en el repentino cambio de táctica del Maestro, unido a sus instrucciones de que no debían efectuar ningún tipo de enseñanza pública durante toda esta semana de la Pascua.

\par 
%\textsuperscript{(1888.2)}
\textsuperscript{173:0.2} Mientras este grupo descendía del Monte de los Olivos, Jesús iba delante y los apóstoles le seguían de cerca en un silencio meditativo. Un sólo pensamiento predominaba en la mente de todos, salvo en Judas Iscariote, y era el siguiente: ¿Qué hará hoy el Maestro? El único pensamiento que absorbía a Judas era: ¿Qué voy a hacer? ¿Voy a continuar con Jesús y mis compañeros, o voy a retirarme? Y si los dejo, ¿cómo voy a romper?

\par 
%\textsuperscript{(1888.3)}
\textsuperscript{173:0.3} Eran cerca de las nueve de esta hermosa mañana cuando estos hombres llegaron al templo\footnote{\textit{Llegada al templo}: Mt 21:12a; Mc 11:15a; Lc 19:45a.}. Se dirigieron enseguida al gran patio donde Jesús enseñaba con tanta frecuencia, y después de saludar a los creyentes que lo estaban esperando, Jesús se subió a uno de los estrados para educadores y empezó a hablarle a la multitud que se estaba congregando. Los apóstoles se apartaron a corta distancia y esperaron los acontecimientos.

\section*{1. La depuración del templo}
\par 
%\textsuperscript{(1888.4)}
\textsuperscript{173:1.1} Un inmenso tráfico comercial se había desarrollado en asociación con los servicios y las ceremonias de culto en el templo. Existía el comercio de suministrar los animales apropiados para los diversos sacrificios\footnote{\textit{Venta de animales sin defectos}: Mt 21:12a; Mc 11:15a; Lc 19:45b; Jn 2:14a.}. Aunque estaba permitido que los fieles aportaran sus propias ofrendas, persistía el hecho de que los animales debían estar libres de todo «defecto» en el sentido de la ley levítica, según la interpretaban los inspectores oficiales del templo. Muchos fieles habían sufrido la humillación de ver cómo los examinadores del templo rechazaban su animal supuestamente perfecto\footnote{\textit{Sacrificio de animales «sin defectos»}: Lv 22:18-25.}. Por esta razón, se había generalizado la práctica de adquirir los animales propiciatorios en el mismo templo, y aunque había diversos lugares cerca del Olivete donde se podían comprar, se había puesto de moda comprar estos animales directamente en los corrales del templo. Esta costumbre de vender todo tipo de animales propiciatorios en los patios del templo se había desarrollado gradualmente. Así había surgido a la existencia un importante comercio que reportaba unos beneficios enormes. Una parte de estas ganancias estaba reservada para el tesoro del templo, pero la mayoría iba a parar indirectamente a las manos de las familias de los altos sacerdotes en el poder\footnote{\textit{Comercio en los atrios del templo}: Jn 2:14a.}.

\par 
%\textsuperscript{(1888.5)}
\textsuperscript{173:1.2} Esta venta de animales en el templo prosperó porque cuando un fiel compraba un animal, aunque el precio fuera un poco alto, ya no tenía que pagar ningún tributo más, y podía estar seguro de que el sacrificio propuesto no sería rechazado con el pretexto de que el animal tenía defectos reales o imaginarios. De vez en cuando, los precios se recargaban de una manera exorbitante a la gente del pueblo, en particular durante las grandes fiestas nacionales. En un momento dado, los codiciosos sacerdotes llegaron a exigir el equivalente de una semana de trabajo por un par de palomas que deberían haberse vendido a los pobres por unos pocos céntimos. Los «hijos de Anás» ya habían empezado a instalar sus bazares en los recintos del templo, unos mercados de géneros que sobrevivieron hasta que fueron finalmente derribados por una muchedumbre tres años antes de la destrucción del templo mismo.

\par 
%\textsuperscript{(1889.1)}
\textsuperscript{173:1.3} Pero el tráfico de animales propiciatorios y de otras mercancías no era la única manera en que se profanaban los patios del templo. En esta época se había fomentado un amplio sistema de intercambio bancario y comercial, que se realizaba directamente dentro de los recintos del templo\footnote{\textit{Los cambistas}: Mt 21:12b; Mc 11:15b; Lc 19:45b; Jn 2:14.}. Todo esto había sucedido de la manera siguiente: Durante la dinastía de los Asmoneos, los judíos acuñaron su propia moneda de plata, y se había establecido la práctica de exigir que el tributo de medio siclo, y todos los demás derechos del templo, se pagaran con esta moneda judía. Esta reglamentación hacía necesario autorizar a unos cambistas para que intercambiaran este siclo ortodoxo de acuñación judía por los numerosos tipos de monedas que circulaban en toda Palestina y en otras provincias del imperio romano. El impuesto del templo por persona, pagadero por todo el mundo a excepción de las mujeres, los esclavos y los menores, era de medio siclo, una moneda de casi dos centímetros de diámetro, pero bastante gruesa. En los tiempos de Jesús, los sacerdotes también estaban exentos de pagar los impuestos del templo. En consecuencia, entre los días 15 y 25 del mes anterior a la Pascua, los cambistas acreditados instalaban sus puestos en las principales ciudades de Palestina, con el fin de proporcionar a los judíos la moneda apropiada para pagar los impuestos del templo cuando llegaran a Jerusalén. Después de este período de diez días, estos cambistas se trasladaban a Jerusalén y montaban sus mostradores de cambio en los patios del templo. Estaban autorizados a cobrar una comisión equivalente a tres o cuatro céntimos por el cambio de una moneda valorada en unos diez céntimos, y en el caso de que se deseara cambiar una moneda de mayor valor, tenían permiso para cobrar el doble. Estos banqueros del templo también se lucraban cambiando todo el dinero destinado a comprar los animales propiciatorios y a pagar los votos y las ofrendas.

\par 
%\textsuperscript{(1889.2)}
\textsuperscript{173:1.4} Estos cambistas del templo no sólo dirigían un negocio regular de banca para obtener beneficios con el intercambio de más de veinte tipos de monedas que los peregrinos visitantes traían periódicamente a Jerusalén, sino que también se dedicaban a todas las otras clases de operaciones relacionadas con el oficio de banquero. Tanto el tesoro del templo como los jefes del mismo obtenían unos beneficios enormes con estas actividades comerciales. No era raro que el tesoro del templo contuviera más de diez millones de dólares (de 1935), mientras que la gente común y corriente languidecía en la miseria y continuaba pagando estas recaudaciones injustas.

\par 
%\textsuperscript{(1889.3)}
\textsuperscript{173:1.5} Este lunes por la mañana, Jesús intentó enseñar el evangelio del reino celestial en medio de esta multitud ruidosa de cambistas, mercaderes y vendedores de ganado. No era el único que se sentía molesto por esta profanación del templo; la gente corriente, y en especial los visitantes judíos de las provincias extranjeras, también se sentían completamente contrariados por esta profanación especulativa de su templo nacional de culto. En esta época, el mismo sanedrín celebraba sus reuniones regulares en una sala que estaba rodeada por todo este murmullo y confusión del comercio y del trueque.

\par 
%\textsuperscript{(1890.1)}
\textsuperscript{173:1.6} Cuando Jesús estaba a punto de empezar su alocución, se produjeron dos incidentes que atrajeron su atención. En el mostrador de un cambista cercano había surgido una discusión violenta y acalorada porque al parecer se le había cobrado con exceso a un judío de Alejandría, y en el mismo momento, el aire se desgarró con los mugidos de una manada de unos cien bueyes que estaban siendo conducidos de una sección de los corrales a otra. Mientras Jesús se detenía, contemplando de manera silenciosa pero meditativa esta escena de comercio y de confusión, observó cerca de él a un galileo sencillo, un hombre con quien había hablado una vez en Irón, que estaba siendo ridiculizado y empujado por unos judeos arrogantes que se consideraban superiores. Todo esto se combinó para que se produjera en el alma de Jesús uno de esos extraños arrebatos periódicos de indignada emoción.

\par 
%\textsuperscript{(1890.2)}
\textsuperscript{173:1.7} Ante el asombro de sus apóstoles, que estaban allí cerca y que se abstuvieron de participar en lo que siguió a continuación, Jesús bajó del estrado de los instructores, se dirigió hacia el muchacho que conducía el ganado a través del patio, le quitó el látigo de cuerdas y sacó rápidamente a los animales del templo. Pero esto no fue todo. Ante la mirada asombrada de las miles de personas reunidas en el patio del templo, se dirigió a grandes zancadas majestuosas hacia el corral más alejado, y se puso a abrir las puertas de cada establo y a expulsar a los animales encerrados. Para entonces los peregrinos reunidos se habían entusiasmado, y con un griterío tumultuoso se dirigieron a los bazares y empezaron a volcar las mesas de los cambistas. En menos de cinco minutos, todo comercio había sido barrido del templo\footnote{\textit{Jesús limpia el templo}: Mt 21:12-13; Mc 11:15-17; Lc 19:45-46; Jn 2:15-16.}. En el momento en que los guardias romanos cercanos aparecieron en escena, todo estaba tranquilo y las multitudes habían recuperado la calma. Jesús regresó a la tribuna de los oradores, y dijo a la multitud: «Hoy habéis presenciado lo que está escrito en las Escrituras: `Mi casa será llamada una casa de oración\footnote{\textit{La casa de mi Padre es casa de oración}: Is 56:7.} para todas las naciones, pero habéis hecho de ella una cueva de ladrones\footnote{\textit{La casa de mi Padre: cueva de ladrones}: Jer 7:11.}.'»

\par 
%\textsuperscript{(1890.3)}
\textsuperscript{173:1.8} Antes de que pudiera decir una palabra más, la gran asamblea estalló en hosannas de alabanza, y un gran grupo de jóvenes salió enseguida de la multitud para cantar himnos de gratitud porque los mercaderes profanos y usureros habían sido echados del templo sagrado. Mientras tanto, algunos sacerdotes habían llegado al lugar, y uno de ellos dijo a Jesús: «¿No oyes lo que dicen los hijos de los levitas?» Y el Maestro respondió: «¿No has leído nunca que `la alabanza\footnote{\textit{Himnos de alabanza}: Mt 21:15-16.} ha salido perfecta de la boca de los niños y de los lactantes?'»\footnote{\textit{De las bocas de los lactantes}: Sal 8:2.} Durante todo el resto del día, mientras Jesús estuvo enseñando, unos guardianes establecidos por el pueblo estuvieron vigilando todos los arcos de entrada, y no permitieron que nadie transportara ni siquiera una vasija vacía a través de los patios del templo.

\par 
%\textsuperscript{(1890.4)}
\textsuperscript{173:1.9} Cuando los principales sacerdotes y los escribas se enteraron de estos acontecimientos, se quedaron sin habla. Tenían mucho más miedo del Maestro, y estaban aún más decididos a destruirlo. Pero estaban confundidos. No sabían cómo disponer su muerte\footnote{\textit{Los principales furiosos y decididos a dar muerte a Jesús}: Mc 11:18; Lc 19:47b-48.}, porque tenían mucho miedo de las multitudes, que ahora expresaban tan abiertamente su aprobación por la expulsión de los especuladores profanos. Durante todo este día, un día tranquilo y pacífico en los patios del templo, el pueblo escuchó la enseñanza de Jesús y estuvo literalmente pendiente de sus palabras.

\par 
%\textsuperscript{(1890.5)}
\textsuperscript{173:1.10} Este acto sorprendente de Jesús sobrepasaba la comprensión de sus apóstoles. Estaban tan desconcertados por esta acción repentina e inesperada de su Maestro, que durante todo el episodio permanecieron agrupados cerca de la tribuna de los oradores; no levantaron ni un dedo para ayudar a esta depuración del templo. Si este acontecimiento espectacular hubiera ocurrido el día anterior, cuando Jesús llegó triunfalmente al templo al final de la tumultuosa procesión a través de las puertas de la ciudad, todo el tiempo aclamado ruidosamente por la multitud, hubieran estado dispuestos a actuar; pero dada la manera en que se desarrollaron las cosas, no estaban preparados en absoluto para participar.

\par 
%\textsuperscript{(1891.1)}
\textsuperscript{173:1.11} Esta depuración del templo revela la actitud del Maestro hacia la comercialización de las prácticas religiosas, así como su abominación por todas las formas de injusticia y de especulación a expensas de los pobres y de los ignorantes. Este episodio demuestra también que Jesús no aprobaba que se rehusara emplear la fuerza para proteger a la mayoría de un grupo humano determinado contra las prácticas desleales y esclavizantes de unas minorías injustas que pudieran parapetarse detrás del poder político, financiero o eclesiástico. No se debe permitir que los hombres astutos, perversos e insidiosos se organicen para explotar y oprimir a aquellos que, a causa de su idealismo, no están dispuestos a recurrir a la violencia para protegerse o para promover sus proyectos de vida dignos de alabanza.

\section*{2. El desafío a la autoridad del Maestro}
\par 
%\textsuperscript{(1891.2)}
\textsuperscript{173:2.1} El domingo, la entrada triunfal de Jesús en Jerusalén intimidó tanto a los dirigentes judíos que se abstuvieron de arrestarlo. Hoy lunes, esta depuración espectacular del templo también retrasó eficazmente la captura del Maestro. Día tras día, los jefes de los judíos estaban más decididos a destruirlo, pero se sentían aturdidos por dos temores, que se conjugaban para demorar la hora de asestar el golpe. Los principales sacerdotes y los escribas eran reacios a arrestar a Jesús en público, por miedo a que la multitud se revolviera contra ellos con un furioso resentimiento; también temían la posibilidad de tener que llamar a los guardias romanos para sofocar una revuelta popular.

\par 
%\textsuperscript{(1891.3)}
\textsuperscript{173:2.2} En su sesión del mediodía, el sanedrín acordó por unanimidad, ya que ningún amigo del Maestro había asistido a esta reunión, que Jesús debía ser destruido rápidamente. Pero no pudieron ponerse de acuerdo en cuanto al momento y a la manera en que debía ser arrestado. Finalmente acordaron designar a cinco grupos para que salieran a mezclarse entre la gente, e intentaran enredarlo en sus enseñanzas o desacreditarlo de otras maneras a los ojos de los que escuchaban su instrucción. En consecuencia, a eso de las dos, cuando Jesús acababa de empezar su discurso sobre «La libertad de la filiación», un grupo de estos ancianos de Israel se abrió paso hasta llegar cerca de él, lo interrumpieron como de costumbre, y le hicieron esta pregunta: ¿«Con qué autoridad haces estas cosas? ¿Quién te ha dado esa autoridad?»\footnote{\textit{«¿Con qué autoridad haces estas cosas?»}: Mt 21:23; Mc 11:27-28; Lc 20:1-2. \textit{Muéstranos un signo}: Jn 2:18.}

\par 
%\textsuperscript{(1891.4)}
\textsuperscript{173:2.3} Era completamente correcto que los dirigentes del templo y los funcionarios del sanedrín judío hicieran esta pregunta a cualquiera que se atreviera a enseñar y a actuar de la manera extraordinaria característica de Jesús, especialmente en lo referente a su reciente conducta de eliminar todo comercio del templo. Todos estos mercaderes y cambistas operaban con una licencia otorgada directamente por los dirigentes más elevados, y se suponía que un porcentaje de sus ganancias iba directamente al tesoro del templo. No olvidéis que la \textit{autoridad} era la contraseña de toda la sociedad judía. Los profetas siempre provocaban problemas porque tenían la audacia de atreverse a enseñar sin autoridad, sin haber sido debidamente instruidos en las academias rabínicas, ni haber recibido después la ordenación regular del sanedrín. La carencia de esta autoridad para enseñar pretenciosamente en público se consideraba como indicación de una arrogancia ignorante o de una rebelión abierta. En esta época, sólo el sanedrín podía ordenar a un anciano o a un instructor, y la ceremonia debía tener lugar en presencia de al menos tres personas que hubieran sido previamente ordenadas de la misma manera. Esta ordenación confería al educador el título de «rabino», y también lo capacitaba para actuar como juez, «atando y desatando aquellas cuestiones que le fueran presentadas para que emitiera su fallo».

\par 
%\textsuperscript{(1892.1)}
\textsuperscript{173:2.4} Los dirigentes del templo se presentaron ante Jesús a esta hora de la tarde desafiando no solamente su enseñanza, sino sus actos. Jesús sabía muy bien que estos mismos hombres habían afirmado públicamente durante mucho tiempo que su autoridad para enseñar era satánica, y que todas sus obras poderosas habían sido realizadas por el poder del príncipe de los demonios. Por consiguiente, el Maestro empezó su respuesta a aquella pregunta haciéndoles otra pregunta. Jesús dijo: «Me gustaría también haceros una pregunta, y si me la contestáis, os diré igualmente con qué autoridad hago estas obras. ¿De dónde venía el bautismo de Juan? ¿Recibió Juan su autoridad del cielo o de los hombres?»\footnote{\textit{Jesús pregunta a su vez}: Mt 21:24-25a; Mc 11:29-30; Lc 20:3-4.}

\par 
%\textsuperscript{(1892.2)}
\textsuperscript{173:2.5} Cuando sus interrogadores escucharon esto, se apartaron a un lado para consultarse entre ellos acerca de la respuesta que debían dar. Habían pensado en desconcertar a Jesús delante de la multitud, pero ahora eran ellos los que se encontraban bastante confundidos ante todos los que estaban congregados en ese momento en el patio del templo. Y su desconcierto fue aun más evidente cuando regresaron ante Jesús, diciendo: «Respecto al bautismo de Juan, no podemos responder; no sabemos». Contestaron de esta manera al Maestro porque habían razonado entre ellos: Si decimos que viene del cielo, entonces Jesús dirá: `¿Por qué no creísteis en él?', y quizás añada que su autoridad la ha recibido de Juan. Y si decimos que viene de los hombres, entonces la multitud podría revolverse contra nosotros, porque la mayoría piensa que Juan era un profeta. Y así se vieron obligados a presentarse ante Jesús y la gente para confesar que ellos, los educadores y dirigentes religiosos de Israel, no podían (o no querían) expresar una opinión sobre la misión de Juan. Cuando terminaron de hablar, Jesús bajó la mirada hacia ellos, y dijo: «Yo tampoco os diré con qué autoridad hago estas cosas»\footnote{\textit{Sin respuesta}: Mt 21:25b-27; Mc 11:31-33; Lc 20:5-8.}.

\par 
%\textsuperscript{(1892.3)}
\textsuperscript{173:2.6} Jesús nunca tuvo la intención de recurrir a Juan para respaldar su autoridad. El sanedrín nunca había ordenado a Juan. La autoridad de Jesús residía en él mismo y en la supremacía eterna de su Padre.

\par 
%\textsuperscript{(1892.4)}
\textsuperscript{173:2.7} Al emplear esta manera de comportarse con sus adversarios, Jesús no pretendía eludir la pregunta. A primera vista podría parecer que era culpable de responder con una evasiva magistral, pero no era así. Jesús nunca estaba dispuesto a aprovecharse injustamente de nadie, ni siquiera de sus enemigos. Con esta aparente evasiva, en realidad proporcionó a todos sus oyentes la respuesta a la pregunta de los fariseos sobre la autoridad que había detrás de su misión. Ellos habían afirmado que él actuaba con la autoridad del príncipe de los demonios. Jesús había repetido muchas veces que todas sus enseñanzas y obras las realizaba con el poder y la autoridad de su Padre que está en los cielos. Los dirigentes judíos se negaban a aceptar esto, y trataban de acorralarlo para que admitiera que era un educador irregular, puesto que nunca había sido autorizado por el sanedrín. Al contestarles como lo hizo, sin pretender que su autoridad viniera de Juan, satisfizo también a la gente con la conclusión de que el esfuerzo de sus enemigos por hacerlo caer en una trampa recayó eficazmente sobre ellos y los desacreditó considerablemente a los ojos de todos los presentes.

\par 
%\textsuperscript{(1892.5)}
\textsuperscript{173:2.8} Este talento que tenía el Maestro para tratar a sus adversarios era lo que tanto les asustaba de él. Aquel día ya no intentaron hacer más preguntas, y se retiraron para consultarse de nuevo entre ellos. Pero la gente no tardó en discernir la falta de honradez y de sinceridad que había en estas preguntas realizadas por los dirigentes judíos. Incluso la gente común no podía dejar de diferenciar entre la majestad moral del Maestro y la hipocresía insidiosa de sus enemigos. Pero la depuración del templo había llevado a los saduceos a unirse con los fariseos para perfeccionar los planes destinados a destruir a Jesús. Y los saduceos representaban ahora la mayoría del sanedrín.

\section*{3. La parábola de los dos hijos}
\par 
%\textsuperscript{(1893.1)}
\textsuperscript{173:3.1} Mientras los críticos fariseos permanecían allí en silencio delante de Jesús, éste bajó la mirada hacia ellos y dijo: «Puesto que dudáis de la misión de Juan y sois hostiles a la enseñanza y a las obras del Hijo del Hombre, prestad oído a la parábola que os voy a contar: Un gran terrateniente respetado tenía dos hijos, y como deseaba la ayuda de sus hijos para administrar sus grandes posesiones, fue a ver a uno de ellos, diciendo: `Hijo, ve hoy a trabajar a mi viñedo.' Este hijo irreflexivo le contestó a su padre: `No voy a ir', pero luego se arrepintió, y fue. Cuando encontró a su hijo mayor, le dijo igualmente: `Hijo, ve a trabajar a mi viñedo.' Y este hijo hipócrita e infiel le contestó: `Sí, padre mío, voy a ir.' Pero cuando su padre se marchó, no fue. Permitidme que os pregunte, ¿cuál de estos hijos hizo realmente la voluntad de su padre?»\footnote{\textit{Parábola de los dos hijos (parte 1)}: Mt 21:28-31a.}

\par 
%\textsuperscript{(1893.2)}
\textsuperscript{173:3.2} Y la gente respondió al unísono, diciendo: «El primer hijo». Entonces Jesús dijo: «Así es; y ahora os afirmo que los publicanos y las prostitutas, aunque parezcan rechazar la llamada al arrepentimiento, verán el error de su estilo de vida y entrarán en el reino de Dios antes que vosotros, que hacéis grandes ostentaciones de servir al Padre que está en los cielos, mientras os negáis a hacer las obras del Padre. No habéis sido vosotros, los fariseos y los escribas, los que habéis creído en Juan, sino más bien los publicanos y los pecadores; tampoco creéis en mi enseñanza, pero la gente corriente escucha mis palabras con mucho gusto».\footnote{\textit{Parábola de los dos hijos (parte 2)}: Mt 21:31b-32.}

\par 
%\textsuperscript{(1893.3)}
\textsuperscript{173:3.3} Jesús no despreciaba personalmente a los fariseos ni a los saduceos. Lo que trataba de desacreditar era sus sistemas de enseñanza y de prácticas. No sentía hostilidad hacia nadie, pero aquí se estaba produciendo la colisión inevitable entre una religión del espíritu nueva y viviente, y la antigua religión de las ceremonias, la tradición y la autoridad.

\par 
%\textsuperscript{(1893.4)}
\textsuperscript{173:3.4} Los doce apóstoles permanecieron todo este tiempo cerca del Maestro, pero no participaron en absoluto en estas acciones. Cada uno de los doce reaccionaba según su propia manera particular ante los acontecimientos de estos últimos días del ministerio de Jesús en la carne, y cada uno obedecía igualmente el mandato del Maestro de abstenerse de toda enseñanza y de toda predicación en público durante esta semana de la Pascua.

\section*{4. La parábola del propietario ausente}
\par 
%\textsuperscript{(1893.5)}
\textsuperscript{173:4.1} Cuando los principales fariseos y los escribas que habían intentado enredar a Jesús con sus preguntas hubieron terminado de escuchar la historia de los dos hijos, se retiraron para consultarse de nuevo. El Maestro volvió su atención hacia la atenta multitud, y contó otra parábola:

\par 
%\textsuperscript{(1893.6)}
\textsuperscript{173:4.2} «Había un hombre de bien que poseía una propiedad, y plantó una viña. La rodeó de un seto, cavó un hoyo para el lagar y construyó una torre para los guardas. Luego alquiló esta viña a unos arrendatarios y partió para un largo viaje a otro país. Cuando se acercó la temporada de los frutos, envió a unos servidores a los arrendatarios para que cobraran su alquiler. Pero los arrendatarios se consultaron entre ellos y se negaron a entregar a estos servidores los frutos que le debían al señor; en lugar de eso, atacaron a los sirvientes, golpearon a uno, lapidaron a otro, y despidieron a los demás con las manos vacías. Cuando el propietario se enteró de todo esto, envió a otros servidores de más confianza para que trataran con estos malvados arrendatarios, pero éstos hirieron a los nuevos sirvientes y los trataron de una manera vergonzosa. Entonces el señor envió a su servidor favorito, a su administrador, y los arrendatarios lo mataron. Sin embargo, con paciencia e indulgencia, el propietario envió a otros muchos servidores, pero no quisieron recibir a ninguno. A unos los golpearon y a otros los mataron. Cuando el propietario se sintió tratado de esta manera, decidió enviar a su hijo para que tratara con aquellos arrendatarios ingratos, diciéndose: `Pueden maltratar a mis servidores, pero seguramente mostrarán respeto por mi amado hijo.' Pero cuando aquellos arrendatarios malvados e impenitentes vieron venir al hijo, razonaron entre ellos: `Éste es el heredero; vamos a matarlo y entonces la herencia será nuestra.' Así pues lo agarraron, y después de echarlo fuera de la viña, lo mataron. Cuando el dueño de esta viña se entere de que han rechazado y matado a su hijo, ¿qué hará con aquellos arrendatarios ingratos y perversos?»\footnote{\textit{Parábola del propietario ausente (parte 1)}: Mt 21:33-40; Mc 12:1-9a; Lc 20:9-16a.}

\par 
%\textsuperscript{(1894.1)}
\textsuperscript{173:4.3} Cuando la gente escuchó esta parábola y la pregunta que Jesús había hecho, contestaron: «Destruirá a esos miserables y alquilará su viña a otros arrendatarios honrados, que le entregarán los frutos a su debido tiempo»\footnote{\textit{Parábola del propietario ausente (parte 2)}: Mt 21:41; Mc 12:9b.}. Algunos de los oyentes percibieron que esta parábola se refería a la nación judía, a la manera en que había tratado a los profetas y al rechazo inminente de Jesús y del evangelio del reino; entonces dijeron con tristeza: «Quiera Dios que no sigamos haciendo estas cosas».\footnote{\textit{Parábola del propietario ausente (parte 3)}: Lc 20:16b.}

\par 
%\textsuperscript{(1894.2)}
\textsuperscript{173:4.4} Jesús vio que un grupo de saduceos y fariseos se abría paso a través del gentío, y se calló un momento hasta que se acercaron a él; entonces dijo: «Sabéis cómo vuestros padres rechazaron a los profetas, y sabéis muy bien que habéis decidido en vuestro corazón rechazar al Hijo del Hombre». Luego, mirando con una mirada escrutadora a los sacerdotes y a los ancianos que estaban cerca de él, Jesús dijo: «¿No habéis leído nunca en las Escrituras acerca de la piedra que rechazaron los constructores, y que se convirtió en la piedra angular cuando el pueblo la descubrió?\footnote{\textit{Parábola de la piedra angular}: Mt 21:42-44; Mc 12:10; Lc 20:17-18.} Por eso, os advierto una vez más que si continuáis rechazando este evangelio, el reino de Dios será pronto apartado de vosotros, y se entregará a un pueblo dispuesto a recibir la buena nueva y a producir los frutos del espíritu. Esta piedra contiene un misterio, pues el que cae sobre ella, aunque se rompa en pedazos por su causa, se salvará\footnote{\textit{El que caiga sobre ella, se salvará}: Dn 2:35.}; pero aquel sobre quien caiga esta piedra se convertirá en polvo, y sus cenizas se dispersarán a los cuatro vientos»\footnote{\textit{La piedra convertida en piedra angular}: Sal 118:22.}.

\par 
%\textsuperscript{(1894.3)}
\textsuperscript{173:4.5} Cuando los fariseos escucharon estas palabras, comprendieron que Jesús se refería a ellos y a los demás dirigentes judíos. Tenían enormes deseos de agarrarlo en aquel mismo momento, pero tenían miedo de la multitud\footnote{\textit{Los fariseos temían a la multitud}: Mt 21:45-46; Mc 12:12; Lc 21:19.}. Sin embargo, estaban tan irritados por las palabras del Maestro que se retiraron para consultarse de nuevo entre ellos acerca de cómo provocar su muerte. Aquella noche, tanto los saduceos como los fariseos se unieron para planear la manera de hacerlo caer en una trampa al día siguiente.

\section*{5. La parábola del banquete de boda}
\par 
%\textsuperscript{(1894.4)}
\textsuperscript{173:5.1} Después de que los escribas y los dirigentes se hubieron retirado, Jesús se dirigió de nuevo a la multitud reunida y contó la parábola del banquete de bodafootnote{\textit{Jesús habla en parábolas}: Mt 22:1.}. Dijo:

\par 
%\textsuperscript{(1894.5)}
\textsuperscript{173:5.2} «El reino de los cielos se puede comparar con un rey que preparó un banquete de boda para su hijo, y envió a unos mensajeros para que llamaran a los que habían sido previamente invitados a venir a la fiesta, diciendo: `Todo está preparado para la cena nupcial en el palacio del rey.' Sin embargo, muchos de los que habían prometido asistir se negaron a venir en aquel momento. Cuando el rey escuchó que rechazaban su invitación, envió a otros servidores y mensajeros, diciendo: `Decid que vengan todos los que estaban invitados, porque mirad, mi cena está preparada. Mis bueyes y mis cebones han sido matados, y todo está preparado para celebrar la boda inminente de mi hijo.'\footnote{\textit{Parábola del festín de bodas (parte 1)}: Mt 22:2-4; Lc 14:16-20.} Pero de nuevo, aquellos invitados desconsiderados no le dieron importancia a la llamada de su rey, y se fueron por su camino, uno a su granja, otro a su cerámica y otros a sus negocios. Y otros además no se contentaron con menospreciar así la llamada del rey, sino que se rebelaron abiertamente, pegaron a los mensajeros del rey, los maltrataron vergonzosamente, e incluso mataron a algunos de ellos. Cuando el rey observó que sus convidados elegidos, incluídos aquellos que habían aceptado su invitación preliminar y habían prometido asistir al banquete de boda, rechazaban finalmente su llamada, y en rebeldía habían atacado y matado a sus mensajeros elegidos, se encolerizó extremadamente. Entonces, este rey ultrajado mandó salir a sus ejércitos y a los ejércitos de sus aliados, y les ordenó que destruyeran a aquellos asesinos rebeldes y que incendiaran su ciudad».\footnote{\textit{Parábola del festín de bodas (parte 2)}: Mt 22:5-7.}

\par 
%\textsuperscript{(1895.1)}
\textsuperscript{173:5.3} «Después de haber castigado a los que habían despreciado su invitación, fijó un nuevo día para el banquete de bodas y dijo a sus mensajeros: `Los primeros invitados a la boda no eran dignos; id pues ahora a los cruces de los caminos y a las carreteras, e incluso más allá de los límites de la ciudad, e invitad a todos los que encontréis, incluídos los extranjeros, para que vengan y asistan a este banquete de bodas.' Los servidores salieron entonces a las carreteras y a los lugares apartados, y reunieron a todos los que encontraron, buenos y malos, ricos y pobres, de manera que por fin la sala nupcial se llenó de convidados de buena voluntad. Cuando todo estuvo dispuesto, el rey entró para examinar a sus huéspedes, y se sorprendió mucho al ver allí a un hombre sin vestido nupcial\footnote{\textit{Sin vestido nupcial}: Mt 22:11-14.}. Puesto que el rey había proporcionado generosamente vestidos nupciales a todos sus huéspedes, se dirigió a este hombre y le dijo: `Amigo, ¿cómo puede ser que entres en la sala de mis invitados, en esta ocasión, sin el vestido nupcial'? Aquel hombre descuidado se quedó callado. Entonces, el rey dijo a sus servidores: `Echad a este invitado desconsiderado de mi casa, y que comparta la misma suerte que todos los demás que despreciaron mi hospitalidad y rechazaron mi llamada. Sólo quiero tener aquí a los que se regocijan de aceptar mi invitación, y que me hacen el honor de llevar los vestidos nupciales que tan generosamente se han proporcionado a todos.'»\footnote{\textit{Parábola del festín de bodas (parte 3)}: Mt 22:8-10; Lc 14:21-24.}

\par 
%\textsuperscript{(1895.2)}
\textsuperscript{173:5.4} Después de contar esta parábola, Jesús estaba a punto de despedir a la multitud cuando un creyente simpatizante se abrió paso hacia él a través del gentío, y preguntó: «Pero, Maestro, ¿cómo nos enteraremos de esas cosas? ¿Cómo estaremos preparados para la invitación del rey? ¿Qué signo nos darás para que sepamos que eres el Hijo de Dios?» Cuando el Maestro escuchó estas palabras, dijo: «Sólo se os dará un signo». Luego, señalando a su propio cuerpo, continuó: «Destruid este templo, y en tres días lo levantaré»\footnote{\textit{El único signo: la resurrección al tercer día}: Jn 2:18-22.}. Pero no lo comprendieron, y se dispersaron diciéndose entre ellos: «Este templo ha estado en construcción casi cincuenta años, y sin embargo dice que lo destruirá y lo levantará en tres días». Ni siquiera sus propios apóstoles comprendieron el significado de esta declaración, pero posteriormente, después de su resurrección, recordaron lo que el Maestro había dicho.

\par 
%\textsuperscript{(1895.3)}
\textsuperscript{173:5.5} Hacia las cuatro de esta tarde, Jesús hizo señas a sus apóstoles y les indicó que deseaba dejar el templo e ir a Betania para cenar y descansar durante la noche\footnote{\textit{Jesús regresa a Betania por la noche}: Mc 11:19; Lc 21:37b.}. Mientras subían el Olivete, Jesús indicó a Andrés, Felipe y Tomás que al día siguiente debían establecer un campamento más cerca de la ciudad, para poder ocuparlo durante el resto de la semana pascual. Siguiendo estas instrucciones, a la mañana siguiente montaron sus tiendas de campaña en una hondonada de la ladera que dominaba el parque de acampamiento público de Getsemaní, en un pequeño terreno que pertenecía a Simón de Betania.

\par 
%\textsuperscript{(1896.1)}
\textsuperscript{173:5.6} De nuevo, un grupo silencioso de judíos ascendió la pendiente occidental del Olivete este lunes por la noche. Estos doce hombres empezaban a sentir, como nunca lo habían sentido antes, que algo trágico estaba a punto de suceder. La espectacular depuración del templo, durante las primeras horas de la mañana, había despertado sus esperanzas de ver cómo el Maestro se imponía y manifestaba sus grandes poderes, pero los acontecimientos de toda la tarde estuvieron caracterizados por un descenso de la tensión, en el sentido de que todos apuntaban a un rechazo seguro de las enseñanzas de Jesús por parte de las autoridades judías. Los apóstoles se sentían oprimidos por la duda y prisioneros de una terrible incertidumbre. Se daban cuenta de que podían transcurrir sólo unos breves días entre los acontecimientos del día que acababa de terminar y el estallido de una fatalidad inminente. Todos sentían que algo temible estaba a punto de suceder, pero no sabían qué esperar. Cada uno se fue a su sitio para descansar, pero durmieron muy poco. Incluso los gemelos Alfeo empezaron por fin a comprender que los acontecimientos de la vida del Maestro se dirigían velozmente hacia su culminación final.