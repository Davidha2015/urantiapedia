\chapter{Documento 176. El martes por la noche en el Monte de los Olivos}
\par
%\textsuperscript{(1912.1)}
\textsuperscript{176:0.1} ESTE martes por la tarde, cuando Jesús y los apóstoles salían del templo para ir al campamento de Getsemaní, Mateo llamó la atención sobre la estructura del templo, y dijo: «Maestro, observa el aspecto de estos edificios. Mira las piedras macizas y los hermosos adornos; ¿es posible que estos edificios vayan a ser destruidos?» Mientras continuaban hacia el Olivete, Jesús dijo: «Estáis viendo estas piedras y este templo macizo; en verdad, en verdad os digo que en los días que pronto llegarán, no quedará piedra sobre piedra. Todas serán derribadas»\footnote{\textit{Predicción de la destrucción del templo}: Mt 24:1-2; Mc 13:1-2; Lc 21:5-6.}. Estas observaciones que describían la destrucción del templo sagrado despertaron la curiosidad de los apóstoles mientras caminaban detrás del Maestro; no podían concebir ningún acontecimiento, como no fuera el fin del mundo, que pudiera ocasionar la destrucción del templo\footnote{\textit{Confusión con el fin del mundo}: Mt 24:3b.}.

\par
%\textsuperscript{(1912.2)}
\textsuperscript{176:0.2} Para evitar las multitudes que pasaban por el valle de Cedrón hacia Getsemaní, Jesús y sus compañeros tenían la intención de subir la pendiente occidental del Olivete durante una corta distancia, y luego seguir un sendero que conducía a su campamento privado, situado cerca de Getsemaní, a corta distancia por encima del campamento público. Cuando se desviaban para abandonar el camino que conducía a Betania, contemplaron el templo, glorificado por los rayos del Sol poniente; y mientras se detenían en el monte, vieron aparecer las luces de la ciudad y contemplaron la belleza del templo iluminado; y allí, bajo la suave luz de la Luna llena, Jesús y los doce se sentaron. El Maestro conversó con ellos\footnote{\textit{Jesús y los apóstoles conversan}: Mt 24:3a; Mc 13:3-4; Lc 21:7.}, y Natanael hizo enseguida la pregunta siguiente: «Dinos Maestro, ¿cómo sabremos que esos acontecimientos están a punto de suceder?»

\section*{1. La destrucción de Jerusalén}
\par
%\textsuperscript{(1912.3)}
\textsuperscript{176:1.1} En respuesta a la pregunta de Natanael, Jesús dijo: «Sí, voy a hablaros de la época en que este pueblo habrá llenado la copa de su iniquidad, cuando la justicia caerá rápidamente sobre esta ciudad de nuestros padres. Estoy a punto de dejaros; voy hacia el Padre. Después de que me haya ido, tened cuidado de que nadie os engañe, porque muchos vendrán como liberadores y conducirán a mucha gente por el camino equivocado. Cuando escuchéis hablar de guerras y de rumores de guerras, no os preocupéis, porque aunque todas esas cosas sucederán, el fin de Jerusalén aún no está cerca\footnote{\textit{Destrucción de Jerusalén}: Mt 24:4-14; Mc 13:5-13; Lc 21:8-17.}. No os inquietéis por la hambruna o los terremotos; tampoco debéis preocuparos cuando seáis entregados a las autoridades civiles y seáis perseguidos a causa del evangelio\footnote{\textit{Persecuciones y peligros}: Mt 10:17-20; Lc 12:11-12.}. Seréis expulsados de la sinagoga e iréis a la cárcel por mi causa, y algunos de vosotros seréis ejecutados. Cuando seáis llevados ante los gobernadores y los dirigentes, será para dar testimonio de vuestra fe y para mostrar vuestra firmeza en el evangelio del reino\footnote{\textit{Evangelio del reino}: Mt 4:23; 9:35; 24:14; Mc 1:14-15.}. Cuando estéis en presencia de los jueces, no os inquietéis de antemano por lo que vais a decir, porque el espíritu os enseñará en esa misma hora lo que deberéis contestar a vuestros adversarios. En esos días de dolor, incluso vuestros propios parientes\footnote{\textit{Traicionados por los parientes}: Mt 10:21-22.}, bajo la dirección de los que han rechazado al Hijo del Hombre, os entregarán a la cárcel y a la muerte. Durante un tiempo, puede ser que todos los hombres os odien por mi causa, pero incluso durante esas persecuciones, no os abandonaré; mi espíritu no os dejará. ¡Tened paciencia! No dudéis de que este evangelio del reino triunfará sobre todos sus enemigos, y será proclamado finalmente a todas las naciones».

\par
%\textsuperscript{(1913.1)}
\textsuperscript{176:1.2} Jesús hizo una pausa mientras contemplaba la ciudad. El Maestro se daba cuenta de que el rechazo del concepto espiritual del Mesías, la determinación de aferrarse de manera ciega y perseverante a la misión material del libertador esperado, pronto llevaría a los judíos a un conflicto directo con los poderosos ejércitos romanos, y que esa lucha sólo podía terminar con la destrucción final y completa de la nación judía. Cuando el pueblo de Jesús rechazó su donación espiritual y se negó a recibir la luz del cielo que brillaba de manera tan misericordiosa sobre ellos, sellaron así su perdición como pueblo independiente con una misión espiritual especial en la Tierra. Los mismos dirigentes judíos reconocieron posteriormente que esta idea laica del Mesías fue la que condujo directamente al alboroto que provocó finalmente su destrucción.

\par
%\textsuperscript{(1913.2)}
\textsuperscript{176:1.3} Puesto que Jerusalén iba a ser la cuna del movimiento evangélico primitivo, Jesús no quería que sus instructores y predicadores perecieran en la terrible derrota del pueblo judío, asociada a la destrucción de Jerusalén; por eso dio estas instrucciones a sus seguidores. A Jesús le preocupaba mucho que algunos de sus discípulos se implicaran en estas revueltas venideras, y perecieran así en la caída de Jerusalén.

\par
%\textsuperscript{(1913.3)}
\textsuperscript{176:1.4} Andrés preguntó entonces: «Pero, Maestro, si la ciudad santa y el templo van a ser destruidos, y si tú no estás aquí para dirigirnos, ¿cuándo deberemos abandonar Jerusalén?» Jesús dijo: «Podéis permanecer en la ciudad después de mi partida, e incluso durante esos tiempos de dolor y de crueles persecuciones, pero cuando veáis finalmente que Jerusalén está siendo rodeada por los ejércitos romanos, después de la revuelta de los falsos profetas, entonces sabréis que su desolación está próxima; entonces deberéis huir a las montañas. Que nadie que esté en la ciudad y en sus alrededores se detenga para salvar nada\footnote{\textit{¡Salvad vuestras vidas!}: Mt 24:15-18; Mc 13:14-16; Lc 21:20-21.}, y que los que estén fuera no se atrevan a entrar. Habrá una gran tribulación\footnote{\textit{Gran tribulación}: Mt 24:21a; Mc 13:19a; Lc 21:22-24.}, porque serán los días de la venganza de los gentiles. Después de que hayáis abandonado la ciudad, este pueblo desobediente caerá derribado por el filo de la espada y será llevado cautivo por todas las naciones; Jerusalén será así pisoteada por los gentiles. Mientras tanto, os lo advierto, no os dejéis engañar. Si alguien viene hasta vosotros diciendo: `Mirad, aquí está el Libertador', o `Mirad, está allí', no le creáis, porque surgirán muchos falsos educadores y descarriarán a mucha gente\footnote{\textit{Cuidado con los falsos mesías}: Mt 24:23-25; Mc 13:21-23.}; pero vosotros no deberíais dejaros engañar, porque os he dicho todo esto por anticipado».

\par
%\textsuperscript{(1913.4)}
\textsuperscript{176:1.5} Los apóstoles permanecieron mucho tiempo sentados en silencio a la luz de la Luna, mientras estas sorprendentes predicciones del Maestro se grababan en sus mentes confusas. Y fue en conformidad con esta advertencia como prácticamente todo el grupo de creyentes y discípulos huyó de Jerusalén en cuanto aparecieron las tropas romanas, encontrando un refugio seguro al norte de Pella.

\par
%\textsuperscript{(1913.5)}
\textsuperscript{176:1.6} Incluso después de esta advertencia explícita, muchos seguidores de Jesús interpretaron estas predicciones como alusivas a los cambios que ocurrirían evidentemente en Jerusalén, cuando a la reaparición del Mesías le siguiera el establecimiento de la Nueva Jerusalén y la ampliación de la ciudad para que se convirtiera en la capital del mundo\footnote{\textit{Los apóstoles confunden los acontecimientos}: Mt 24:3.}. En su mente, estos judíos estaban decididos a relacionar la destrucción del templo con el «fin del mundo». Creían que esta Nueva Jerusalén\footnote{\textit{Nueva Jerusalén}: Ap 3:12; Ap 21:2.} ocuparía toda Palestina; que después del fin del mundo vendría la aparición inmediata de los «nuevos cielos y de la nueva tierra»\footnote{\textit{Nuevos cielos y nueva tierra}: Is 65:17; 66:22; 2 P 3:13; Ap 21:1.}. Por eso no es de extrañar que Pedro dijera: «Maestro, sabemos que todas las cosas se desvanecerán\footnote{\textit{Todas las cosas se desvanecerán}: Mt 24:35; Mc 13:31; Lc 21:33.} cuando aparezcan los nuevos cielos y la nueva tierra, pero, ¿cómo sabremos cuándo regresarás para efectuar todo esto?»

\par
%\textsuperscript{(1914.1)}
\textsuperscript{176:1.7} Cuando Jesús escuchó esto, se quedó pensativo durante unos momentos y luego dijo: «Os equivocáis continuamente porque siempre tratáis de conectar la nueva enseñanza con la antigua; estáis decididos a tergiversar toda mi enseñanza; insistís en interpretar el evangelio de acuerdo con vuestras creencias establecidas. Sin embargo, trataré de iluminaros».

\section*{2. La segunda venida del Maestro}
\par
%\textsuperscript{(1914.2)}
\textsuperscript{176:2.1} En diversas ocasiones, Jesús había hecho declaraciones que condujeron a sus oyentes a deducir que, aunque se proponía dejar este mundo dentro de poco, regresaría con toda seguridad para consumar la obra del reino celestial. A medida que sus seguidores estaban más convencidos de que los iba a dejar, y después de haber partido de este mundo, era muy natural que todos los creyentes se aferraran firmemente a estas promesas de regresar. Y así, la doctrina de la segunda venida de Cristo se incorporó pronto en las enseñanzas de los cristianos, y casi todas las generaciones posteriores de discípulos han creído devotamente en esta verdad y han esperado con confianza que regresaría algún día.

\par
%\textsuperscript{(1914.3)}
\textsuperscript{176:2.2} Puesto que debían separarse de su Maestro e Instructor, estos primeros discípulos y los apóstoles se aferraron mucho más a esta promesa de regresar, y no tardaron en asociar la vaticinada destrucción de Jerusalén con esta segunda venida prometida. Y continuaron interpretando de esta manera sus palabras, a pesar de que el Maestro, durante todo este anochecer de enseñanza en el Monte de los Olivos, se tomó el enorme trabajo de impedir precisamente este error.

\par
%\textsuperscript{(1914.4)}
\textsuperscript{176:2.3} En su contestación adicional a la pregunta de Pedro, Jesús dijo: «¿Por qué continuáis creyendo que el Hijo del Hombre se sentará en el trono de David, y esperáis que se cumplan los sueños materiales de los judíos? ¿No os he dicho todos estos años que mi reino no es de este mundo?\footnote{\textit{Mi reino no es de este mundo}: Jn 18:36.} Las cosas que ahora contempláis a vuestros pies están llegando a su fin, pero éste será un nuevo comienzo, a partir del cual el evangelio del reino se extenderá por todo el mundo\footnote{\textit{El evangelio por todo el mundo}: Mt 24:14; 28:19-20a; Mc 13:10; 16:15; Lc 24:47; Jn 17:18; Hch 1:8b.}, y esta salvación se difundirá a todos los pueblos. Cuando el reino haya llegado a su plena madurez, estad seguros de que el Padre que está en los cielos no dejará de visitaros con una revelación ampliada de la verdad y con una demostración realzada de la rectitud, tal como ya ha otorgado a este mundo a aquel que se convirtió en el príncipe de las tinieblas, y luego a Adán, que fue seguido por Melquisedek, y en nuestros días, al Hijo del Hombre. Mi Padre continuará así manifestando su misericordia y mostrando su amor, incluso a este mundo oscuro y malvado. Después de que mi Padre me haya investido con todo el poder y la autoridad, yo también continuaré siguiendo vuestra suerte y guiándoos en los asuntos del reino mediante la presencia de mi espíritu, que pronto será derramado sobre todo el género humano. Aunque así estaré presente con vosotros en espíritu, también prometo que regresaré algún día a este mundo donde he vivido esta vida en la carne y he logrado la experiencia simultánea de revelar a Dios a los hombres y de conducir los hombres hacia Dios. Tengo que dejaros muy pronto y reemprender el trabajo que el Padre me ha confiado, pero tened buen ánimo, porque volveré algún día. Mientras tanto, mi Espíritu de la Verdad de un universo os confortará y os guiará».

\par
%\textsuperscript{(1915.1)}
\textsuperscript{176:2.4} «Ahora me veis débil y en la carne, pero cuando regrese será con poder y en el espíritu\footnote{\textit{Segunda venida con poder}: Mt 16:27; 24:30b; 25:31; Mc 13:26; 14:62; Lc 21:27.}. Los ojos de la carne contemplan al Hijo del Hombre en la carne, pero sólo los ojos del espíritu contemplarán al Hijo del Hombre glorificado por el Padre y apareciendo en la Tierra en su propio nombre».

\par
%\textsuperscript{(1915.2)}
\textsuperscript{176:2.5} «Pero la época de la reaparición del Hijo del Hombre sólo se conoce en los consejos del Paraíso; ni siquiera los ángeles del cielo saben cuándo sucederá esto\footnote{\textit{Sólo Dios sabe cuándo}: Mt 24:36; Mc 13:32.}. Sin embargo, deberíais comprender que cuando este evangelio del reino haya sido proclamado en el mundo entero para la salvación de todos los pueblos, y cuando la era haya alcanzado su plenitud, el Padre os enviará otra donación dispensacional, o si no, el Hijo del Hombre regresará para juzgar la era».

\par
%\textsuperscript{(1915.3)}
\textsuperscript{176:2.6} «Y ahora, en lo que se refiere a las tribulaciones de Jerusalén\footnote{\textit{Las tribulaciones de Jerusalén}: Mt 24:34; Mc 13:30; Lc 21:32.}, de las cuales os he hablado, esta generación no pasará hasta que se cumplan mis palabras; pero en lo que respecta a la época de la nueva venida del Hijo del Hombre, nadie en el cielo o en la Tierra puede atreverse a hablar de ello. Pero deberíais ser sabios en lo que se refiere a la maduración de una era\footnote{\textit{Ser sabios en la maduración de una era}: Mt 24:32-33; Mc 13:28-29; Lc 21:29-31.}; deberíais estar alertas para discernir los signos de los tiempos. Cuando la higuera muestra sus ramas tiernas y brotan sus hojas, sabéis que el verano está cerca. De la misma manera, cuando el mundo haya pasado por el largo invierno de la mentalidad materialista y discernáis la venida de la primavera espiritual de una nueva dispensación, deberíais saber que se acerca el verano de una nueva visita».

\par
%\textsuperscript{(1915.4)}
\textsuperscript{176:2.7} «Pero, ¿cuál es el significado de esta enseñanza relacionada con la venida de los Hijos de Dios? ¿No os dais cuenta de que cuando cada uno de vosotros sea llamado a abandonar la lucha de la vida y a traspasar la puerta de la muerte estará en la presencia inmediata del juicio, frente a frente con los hechos de una nueva dispensación de servicio en el plan eterno del Padre infinito?\footnote{\textit{¿Qué diferencia supone para vosotros?}: Mt 24:42-44.} Aquello a lo que el mundo entero debe de hecho enfrentarse literalmente al final de una era, cada uno de vosotros, como individuo, tiene que enfrentarse con toda seguridad, como experiencia personal, cuando llegue al final de su vida física, y con ello pase a enfrentarse a las condiciones y a las exigencias inherentes a la revelación siguiente de la evolución eterna del reino del Padre».

\par
%\textsuperscript{(1915.5)}
\textsuperscript{176:2.8} De todos los discursos que el Maestro dio a sus apóstoles, ninguno causó nunca tanta confusión en sus mentes como éste, pronunciado este martes por la noche en el Monte de los Olivos, sobre el doble tema de la destrucción de Jerusalén y de su propia segunda venida\footnote{\textit{Confusión sobre las advertencias}: Mt 24:15-44; Mc 13:14-37; Lc 21:20-36.}. Por consiguiente, las narraciones escritas posteriormente, basadas en los recuerdos de lo que el Maestro había dicho en esta ocasión extraordinaria, concordaron poco entre sí. En consecuencia, como los relatos dejaron en blanco muchas cosas que se dijeron este martes por la noche, surgieron muchas tradiciones. A principios del siglo segundo, un apocalipsis judío sobre el Mesías, escrito por un tal Selta\footnote{\textit{Adiciones de Selta}: Mc 13:14-37; Lc 21:20-36.}, que estaba ligado a la corte del emperador Calígula, fue íntegramente copiado en el Evangelio según Mateo, y posteriormente añadido (en parte) a los relatos de Marcos y de Lucas. En estos escritos de Selta fue donde apareció la parábola de las diez vírgenes\footnote{\textit{Parábola de las diez vírgenes}: Mt 25:1-13.}. Ninguna parte de los escritos evangélicos sufrió nunca una interpretación errónea tan confusa como la enseñanza de esta noche. Pero el apóstol Juan nunca se dejó confundir de esta manera.

\par
%\textsuperscript{(1915.6)}
\textsuperscript{176:2.9} Mientras estos trece hombres reanudaban su camino hacia el campamento, permanecían callados y bajo los efectos de una gran tensión emocional. Judas había ratificado finalmente su decisión de abandonar a sus compañeros. Ya era tarde cuando David Zebedeo, Juan Marcos y cierto número de discípulos principales recibieron a Jesús y a los doce en el nuevo campamento, pero los apóstoles no querían dormir; querían saber más cosas sobre la destrucción de Jerusalén, la partida del Maestro y el fin del mundo.

\section*{3. La conversación posterior en el campamento}
\par
%\textsuperscript{(1916.1)}
\textsuperscript{176:3.1} Mientras unos veinte de ellos se reunían alrededor del fuego del campamento, Tomás preguntó: «Puesto que tienes que volver para terminar la obra del reino, ¿cuál ha de ser nuestra actitud mientras estás lejos, ocupado en los asuntos del Padre?» Jesús los miró a la luz del fuego y respondió:

\par
%\textsuperscript{(1916.2)}
\textsuperscript{176:3.2} «Tomás, tú tampoco logras comprender lo que he estado diciendo. ¿No te he enseñado todo este tiempo que tu relación con el reino es espiritual e individual, que es totalmente un asunto de experiencia personal en el espíritu mediante la comprensión, por la fe, de que eres un hijo de Dios? ¿Qué puedo decir más? La caída de las naciones, el desplome de los imperios, la destrucción de los judíos incrédulos, el final de una era e incluso el fin del mundo, ¿qué tienen que ver estas cosas con alguien que cree en este evangelio, y que ha refugiado su vida en la seguridad del reino eterno? Vosotros que conocéis a Dios y que creéis en el evangelio, ya habéis recibido las seguridades de la vida eterna. Puesto que vuestra vida ha sido vivida en el espíritu y para el Padre, nada os puede preocupar seriamente. Los constructores del reino, los ciudadanos acreditados de los mundos celestiales, no deben inquietarse por los trastornos temporales o perturbarse por los cataclismos terrestres. A vosotros que creéis en este evangelio del reino, ¿qué os importa que se derrumben las naciones, que se termine la era o que estallen todas las cosas visibles, puesto que sabéis que vuestra vida es el don del Hijo, y que está eternamente segura en el Padre? Como habéis vivido la vida temporal por la fe, y habéis producido los frutos del espíritu con la rectitud del servicio amoroso hacia vuestros semejantes, podéis contemplar con confianza el siguiente paso de la carrera eterna, con la misma fe en la supervivencia que os ha hecho atravesar vuestra primera aventura terrenal de filiación con Dios».

\par
%\textsuperscript{(1916.3)}
\textsuperscript{176:3.3} «Cada generación de creyentes debería continuar su trabajo con vistas al posible regreso del Hijo del Hombre\footnote{\textit{Regreso del Hijo del Hombre}: Mt 16:27; 24:30b; 25:31; Mc 13:26; 14:62; Lc 21:27.}, exactamente como cada creyente individual lleva adelante el trabajo de su vida con vistas a la inevitable muerte natural siempre amenazante. Una vez que os habéis establecido por la fe como hijos de Dios, no importa ninguna otra cosa en lo que respecta a la seguridad de la supervivencia. ¡Pero no os engañéis! Esta fe en la supervivencia es una fe viva, y manifiesta cada vez más los frutos de ese espíritu divino que al principio la inspiró en el corazón humano. El hecho de que hayáis aceptado anteriormente la filiación en el reino celestial, no os salvará si rechazáis a sabiendas y de manera persistente las verdades relacionadas con la producción progresiva de los frutos espirituales de los hijos de Dios en la carne. Vosotros, que habéis estado conmigo en los asuntos terrestres del Padre, incluso ahora podéis abandonar el reino si descubrís que no amáis el camino del servicio del Padre para la humanidad».

\par
%\textsuperscript{(1916.4)}
\textsuperscript{176:3.4} «Como individuos y como generación de creyentes, escuchadme mientras os cuento una parábola: Había un hombre importante que, antes de partir para un largo viaje a otro país, convocó a todos sus servidores de confianza y les entregó todos sus bienes. A uno le dio cinco talentos, a otro dos y a otro uno, y así sucesivamente a todo el grupo de fieles administradores. A cada uno le confió sus bienes según sus capacidades variadas, y luego salió de viaje. Cuando este señor hubo partido, sus servidores se pusieron a trabajar para sacarle provecho a las riquezas que les habían confiado. El que había recibido cinco talentos empezó inmediatamente a negociar con ellos, y muy pronto obtuvo un beneficio de otros cinco talentos. De la misma manera, el que había recibido dos talentos pronto había ganado dos más. Y así, todos aquellos servidores consiguieron beneficios para su señor, excepto aquel que sólo había recibido un talento. Se marchó solo y cavó un hoyo en la tierra, donde escondió el dinero de su señor.\footnote{\textit{Parábola de los talentos}: Mt 25:14-18.} Pronto, el señor de aquellos servidores regresó inesperadamente y llamó a sus administradores para que le rindieran cuentas. Cuando todos se encontraron delante de su amo, el que había recibido los cinco talentos se adelantó con el dinero que se le había confiado y aportó cinco talentos adicionales, diciendo: `Señor, me diste cinco talentos para invertirlos, y me alegra entregarte otros cinco talentos que he ganado.' Entonces su señor le dijo: `Bien hecho, mi buen y fiel servidor, has sido fiel en las pocas cosas; ahora te estableceré como administrador de muchas cosas; comparte inmediatamente la alegría de tu señor.' Luego, el que había recibido los dos talentos se adelantó diciendo: `Señor, me entregaste dos talentos; mira, he ganado estos otros dos talentos.' Y su señor le dijo entonces: `Bien hecho, mi buen y fiel administrador; tú también has sido fiel en las pocas cosas y ahora te pondré a cargo de muchas; comparte la alegría de tu señor.' Entonces se presentó para rendir cuentas el que había recibido un solo talento. Este servidor se adelantó, diciendo:`Señor, yo te conocía y me daba cuenta de que eras un hombre astuto, en el sentido de que esperabas unos beneficios allí donde no habías trabajado personalmente; por eso tenía miedo de arriesgar algo de lo que se me había confiado. Escondí tu talento en un lugar seguro en la tierra; aquí está; ahora tienes lo que es tuyo.' Pero su señor respondió: `Eres un administrador indolente y perezoso. Confiesas con tus propias palabras que sabías que yo te exigiría una rendición de cuentas con unos beneficios razonables, como las que me han rendido hoy tus diligentes compañeros. Por lo tanto, sabiendo esto, al menos deberías haber entregado mi dinero a los banqueros para que, a mi regreso, pudiera recibir lo que es mío más los intereses.' Entonces este señor dijo al administrador principal: `Quítale ese único talento a este servidor inútil y dáselo al que tiene diez talentos.'»\footnote{\textit{Parábola de los talentos, cont.}: Mt 25:19-28.}

\par
%\textsuperscript{(1917.1)}
\textsuperscript{176:3.5} «A todo el que tiene, se le dará más y poseerá en abundancia; pero a aquel que no tiene, incluso lo que tiene se le quitará\footnote{\textit{Parábola de los talentos, final}: Mt 25:29. \textit{Al que tiene se le dará; al que no se le quitará}: Mt 13:12; 25:29; Mc 4:25; Lc 8:18; 19:26.}. No podéis permanecer inmóviles en los asuntos del reino eterno. Mi Padre exige que todos sus hijos crezcan en la gracia\footnote{\textit{Crecer en la gracia}: 2 P 3:18.} y en el conocimiento de la verdad. Vosotros, que conocéis estas verdades, debéis producir cada vez más frutos del espíritu\footnote{\textit{Frutos del espíritu}: Gl 5:22-23; Ef 5:9.} y manifestar una devoción creciente al servicio desinteresado de vuestros compañeros servidores. Y recordad que, en la medida en que ayudáis al más humilde de mis hermanos, ese servicio me lo habréis hecho a mí»\footnote{\textit{Lo que ayudéis a los demás, me ayudáis a mi}: Mt 25:40.}.

\par
%\textsuperscript{(1917.2)}
\textsuperscript{176:3.6} «Así es como deberíais ocuparos de los asuntos del Padre, ahora y en el futuro, e incluso para siempre. Continuad hasta que yo regrese. Haced fielmente lo que se os ha confiado, y así estaréis preparados para la rendición de cuentas que acompaña a la muerte. Habiendo vivido así para la gloria del Padre y la satisfacción del Hijo, entraréis con alegría y un placer extremo al servicio eterno del reino perpetuo».

\par
%\textsuperscript{(1917.3)}
\textsuperscript{176:3.7} La verdad es viviente; el Espíritu de la Verdad siempre está conduciendo a los hijos de la luz a unos nuevos dominios de realidad espiritual y de servicio divino. La verdad no se os da para que la cristalicéis en unas formas establecidas, seguras y veneradas. Vuestra revelación de la verdad debe ser tan realzada al pasar por vuestra experiencia personal, que ha de descubrir una nueva belleza y unos beneficios espirituales reales a todos aquellos que contemplan vuestros frutos espirituales, viéndose inducidos en consecuencia a glorificar al Padre que está en los cielos. Únicamente aquellos fieles servidores que crecen así en el conocimiento de la verdad, y que gracias a ello desarrollan la capacidad de apreciar divinamente las realidades espirituales, pueden esperar «compartir plenamente la alegría de su Señor»\footnote{\textit{Entrar en la alegría del Señor}: Mt 25:21b, 23b.}. Es triste ver a las generaciones sucesivas de seguidores declarados de Jesús, decir a propósito de su administración de la verdad divina: «Maestro, he aquí la verdad que nos confiaste\footnote{\textit{Aquí está la verdad que nos diste}: Mt 25:24-26.} hace cien o mil años. No hemos perdido nada; hemos conservado fielmente todo lo que nos diste; no hemos permitido que se haga ningún cambio en lo que nos enseñaste; aquí está la verdad que nos diste». Pero este pretexto relativo a la indolencia espiritual no justificará, en presencia del Maestro, al administrador estéril de la verdad. El Maestro de la verdad os exigirá una rendición de cuentas de acuerdo con la verdad que os ha sido confiada.

\par
%\textsuperscript{(1918.1)}
\textsuperscript{176:3.8} En el mundo siguiente se os pedirá que deis cuenta de vuestros dones y de vuestras gestiones en este mundo. Que vuestros talentos inherentes sean pocos o muchos, será necesario enfrentarse a una rendición de cuentas justa y misericordiosa. Si los dones sólo se utilizan con fines egoístas y no se presta ninguna atención al deber superior de obtener una producción creciente de los frutos del espíritu, tal como éstos se manifiestan en el servicio a los hombres y en la adoración a Dios en constante expansión, esos administradores egoístas deben aceptar las consecuencias de su elección deliberada.

\par
%\textsuperscript{(1918.2)}
\textsuperscript{176:3.9} Cuánto se parece este servidor infiel provisto de un solo talento a todos los mortales egoístas, en el sentido de que acusó directamente a su señor de su propia pereza. Cuando un hombre se enfrenta con sus propios fracasos, ¡cuánta tendencia tiene a inculpar a los demás, con mucha frecuencia a quienes menos se lo merecen!

\par
%\textsuperscript{(1918.3)}
\textsuperscript{176:3.10} Aquella noche, cuando se retiraban para descansar, Jesús les dijo: «Habéis recibido gratuitamente; por eso deberíais dar gratuitamente la verdad del cielo, y al darla, esta verdad se multiplicará y mostrará la luz creciente de la gracia salvadora a medida que la prodiguéis»\footnote{\textit{Habéis recibido gratuitamente, dad gratuitamente}: Mt 10:8b.}.

\section*{4. El regreso de Miguel}
\par
%\textsuperscript{(1918.4)}
\textsuperscript{176:4.1} De todas las enseñanzas del Maestro, ninguna fase ha sido tan mal comprendida como su promesa de regresar algún día en persona a este mundo. No es de extrañar que Miguel estuviera interesado en regresar algún día al planeta donde había experimentado su séptima y última donación como mortal del reino. Es muy natural creer que Jesús de Nazaret, ahora gobernante soberano de un inmenso universo, esté interesado en regresar, no solamente una vez sino muchas veces, al mundo en el que vivió una vida tan excepcional y donde ganó finalmente para sí mismo el poder y la autoridad universales que el Padre le había otorgado de manera ilimitada. Urantia será eternamente una de las siete esferas de nacimiento de Miguel, en su proceso de ganar la soberanía de un universo.

\par
%\textsuperscript{(1918.5)}
\textsuperscript{176:4.2} Jesús declaró en numerosas ocasiones y a muchas personas su intención de regresar a este mundo. A medida que sus seguidores despertaban al hecho de que su Maestro no iba a ejercer su actividad como libertador temporal, y a medida que escuchaban sus predicciones sobre la destrucción de Jerusalén y la ruina de la nación judía, empezaron a asociar de la manera más natural su regreso prometido con estos acontecimientos catastróficos. Pero cuando los ejércitos romanos arrasaron los muros de Jerusalén, destruyeron el templo y dispersaron a los judíos de Judea, y el Maestro seguía sin revelarse con poder y gloria, sus seguidores empezaron a formular la creencia que acabó por asociar la segunda venida de Cristo con el final de la era, e incluso con el fin del mundo.

\par
%\textsuperscript{(1918.6)}
\textsuperscript{176:4.3} Jesús prometió hacer dos cosas después de haber ascendido hacia el Padre, y una vez que todos los poderes en el cielo y en la Tierra hubieran sido puestos entre sus manos\footnote{\textit{Todo el poder en el Cielo y la Tierra}: Mt 28:18.}. Primero, prometió enviar al mundo, en su lugar, a otro instructor, al Espíritu de la Verdad\footnote{\textit{Otro instructor, el Espíritu de la Verdad}: Ez 11:19; 18:31; 36:26-27; Jl 2:28-29; Lc 24:49; Jn 7:39; 14:16-18,23,26; 15:4,26; 16:7-8,13-14; 17:21-23; Hch 1:5,8a; 2:1-4,16-18; 2:33; 2 Co 13:5; Gl 2:20; 4:6; Ef 1:13; 4:30; 1 Jn 4:12-15.}, y lo hizo el día de Pentecostés\footnote{\textit{El día de Pentecostés}: Hch 2:1-40.}. Y segundo, prometió con toda seguridad a sus seguidores que algún día regresaría\footnote{\textit{El regreso personal de Jesús}: Jn 14:3,28.} personalmente a este mundo. Pero no dijo cómo, dónde ni cuándo volvería a visitar este planeta donde había vivido su experiencia donadora en la carne. En una ocasión insinuó que, como los ojos de la carne lo habían contemplado mientras vivía aquí, a su regreso (al menos en una de sus posibles visitas) sólo sería percibido por el ojo de la fe espiritual.

\par
%\textsuperscript{(1919.1)}
\textsuperscript{176:4.4} Muchos de nosotros tienden a creer que Jesús regresará muchas veces a Urantia durante las eras por venir. No tenemos su promesa expresa de que hará estas múltiples visitas, pero parece muy probable que aquel que lleva entre sus títulos universales el de Príncipe Planetario de Urantia, visitará muchas veces el mundo cuya conquista le ha conferido este título tan excepcional.

\par
%\textsuperscript{(1919.2)}
\textsuperscript{176:4.5} Creemos firmemente que Miguel volverá en persona a Urantia, pero no tenemos la menor idea de cuándo o de qué manera elegirá hacerlo. Su segunda venida a la Tierra ¿se calculará para que ocurra en conexión con el juicio final de la era presente, con o sin la aparición concomitante de un Hijo Magistral? ¿Vendrá en conexión con el final de alguna era urantiana posterior? ¿Vendrá sin anunciarse y como un acontecimiento aislado? No lo sabemos. Sólo estamos seguros de una cosa, y es que cuando regrese, probablemente todo el mundo lo sabrá, porque deberá venir como jefe supremo de un universo, y no como el oscuro recién nacido de Belén. Pero si todos los ojos han de contemplarlo, y si sólo los ojos espirituales podrán discernir su presencia, entonces su venida deberá retrasarse durante mucho tiempo.

\par
%\textsuperscript{(1919.3)}
\textsuperscript{176:4.6} Por lo tanto, haríais bien en no asociar el regreso personal del Maestro a la Tierra con ningún acontecimiento previsto y con ninguna época determinada. Sólo estamos seguros de una cosa: Ha prometido que volverá. No tenemos ninguna idea de cuándo cumplirá esta promesa ni en relación con qué acontecimiento. Que nosotros sepamos, puede aparecer en la Tierra en cualquier momento, y puede no venir hasta que hayan pasado unas eras tras otras y hayan sido debidamente juzgadas por sus Hijos asociados del cuerpo Paradisiaco.

\par
%\textsuperscript{(1919.4)}
\textsuperscript{176:4.7} La segunda venida de Miguel a la Tierra es un acontecimiento con un enorme valor sentimental, tanto para los intermedios como para los humanos; pero por otra parte, no tiene una importancia inmediata para los intermedios ni más importancia práctica para los seres humanos que el acontecimiento común de la muerte natural, la cual precipita repentinamente al hombre mortal en la influencia inmediata de esa sucesión de acontecimientos universales que le conducen directamente a la presencia de este mismo Jesús, el gobernante soberano de nuestro universo. Todos los hijos de la luz están destinados a verlo, y no tiene ninguna importancia que nosotros vayamos hacia él o que se dé la circunstancia de que él venga primero hacia nosotros. Estad pues siempre dispuestos a acogerlo en la Tierra, tal como él está dispuesto a acogeros en el cielo. Esperamos con confianza su gloriosa aparición, e incluso sus repetidas visitas, pero ignoramos por completo cuándo, cómo, o en relación con qué acontecimiento está destinado a aparecer.