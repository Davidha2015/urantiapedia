\chapter{Documento 179. La Última cena}
\par
%\textsuperscript{(1936.1)}
\textsuperscript{179:0.1} DURANTE la tarde de este jueves, cuando Felipe le recordó al Maestro que se acercaba la Pascua y le preguntó sobre sus planes para celebrarla, estaba pensando en la cena pascual que debía tener lugar al día siguiente, viernes, por la noche. Era costumbre empezar los preparativos para la celebración de la Pascua, como muy tarde, al mediodía del día anterior. Como los judíos consideraban que el día comenzaba con la puesta del Sol, esto significaba que la cena pascual del sábado se celebraba el viernes por la noche, poco antes de la medianoche.

\par
%\textsuperscript{(1936.2)}
\textsuperscript{179:0.2} Por esta razón, los apóstoles no lograban comprender en absoluto el anuncio del Maestro de que celebrarían la Pascua un día antes. Pensaban, al menos algunos de ellos, que Jesús sabía que sería arrestado antes de la hora de la cena pascual del viernes por la noche y que, por consiguiente, los reunía para una cena especial este jueves por la noche. Otros pensaban que se trataba simplemente de una ocasión especial, que precedería la celebración regular de la Pascua.

\par
%\textsuperscript{(1936.3)}
\textsuperscript{179:0.3} Los apóstoles sabían que Jesús había celebrado otras Pascuas sin cordero; sabían que no participaba personalmente en ningún oficio del sistema judío que incluyera sacrificios. Había compartido muchas veces el cordero pascual como invitado, pero siempre que él era el anfitrión no se servía cordero. Para los apóstoles no habría sido una gran sorpresa que se hubiera suprimido el cordero incluso la noche de la Pascua, y puesto que esta cena tenía lugar un día antes, la falta de cordero pasó desapercibida.

\par
%\textsuperscript{(1936.4)}
\textsuperscript{179:0.4} Después de que el padre y la madre de Juan Marcos les ofrecieron sus saludos de bienvenida, los apóstoles subieron inmediatamente a la sala de arriba, mientras Jesús se quedaba atrás charlando con la familia Marcos.

\par
%\textsuperscript{(1936.5)}
\textsuperscript{179:0.5} Se había acordado de antemano que el Maestro celebraría este acontecimiento a solas con sus doce apóstoles; por lo tanto, no se había previsto que hubiera ningún criado para servirles.

\section*{1. El deseo de ser preferido}
\par
%\textsuperscript{(1936.6)}
\textsuperscript{179:1.1} Cuando los apóstoles fueron conducidos al piso superior por Juan Marcos, contemplaron una sala amplia y cómoda que estaba completamente preparada la cena, y observaron que el pan, el vino, el agua y las hierbas estaban dispuestos en un extremo de la mesa. Salvo en este extremo donde se encontraban el pan y el vino, esta larga mesa estaba rodeada por trece triclinios, tal como hubiera estado preparada para la celebración de la Pascua en una familia judía adinerada.

\par
%\textsuperscript{(1936.7)}
\textsuperscript{179:1.2} Mientras los doce entraban en esta habitación de arriba, observaron justo por dentro de la puerta los cántaros de agua, las palanganas y las toallas para lavar sus pies polvorientos; y puesto que no estaba previsto que ningún criado hiciera este servicio, los apóstoles empezaron a mirarse entre sí en cuanto Juan Marcos los hubo dejado, y cada uno empezó a pensar para sus adentros: ¿Quién va a lavarnos los pies? Y cada cual también pensó que él no sería el que actuaría así como servidor de los demás.

\par
%\textsuperscript{(1937.1)}
\textsuperscript{179:1.3} Mientras permanecían allí de pie con este dilema en el corazón, examinaron la disposición de los asientos en la mesa, y observaron el diván más elevado del anfitrión, con un lecho a la derecha y los otros once dispuestos alrededor de la mesa hasta llegar al asiento opuesto a este segundo asiento de honor situado a la derecha del anfitrión.

\par
%\textsuperscript{(1937.2)}
\textsuperscript{179:1.4} Esperaban la llegada del Maestro en cualquier momento, pero tenían la incertidumbre de si debían sentarse o esperar a que viniera para que les asignara sus sitios. Mientras titubeaban, Judas se dirigió al asiento de honor, a la izquierda del anfitrión, y manifestó que tenía la intención de recostarse allí como convidado preferido. Este acto de Judas provocó inmediatamente una violenta disputa entre los demás apóstoles. Apenas acababa Judas de ocupar el asiento de honor cuando Juan Zebedeo reclamó para sí el siguiente asiento preferido, el que se encontraba a la derecha del anfitrión. Simón Pedro se enfureció tanto con esta presunción de Judas y de Juan por ocupar los lugares de preferencia que, mientras los demás apóstoles observaban irritados, caminó alrededor de la mesa y se situó en el lecho más bajo, al final de la fila de asientos, exactamente enfrente del que había elegido Juan Zebedeo. Puesto que otros apóstoles habían ocupado los asientos elevados, Pedro pensó en elegir el más bajo, y lo hizo no solamente para protestar contra el orgullo indecente de sus hermanos, sino con la esperanza de que Jesús, cuando entrara y lo viera en el lugar menos honorífico, lo hiciera subir a uno más elevado, desplazando así a otro que se había atrevido a honrarse a sí mismo.\footnote{\textit{Los apóstoles se disputan los asientos}: Lc 9:46; Lc 22:24.}

\par
%\textsuperscript{(1937.3)}
\textsuperscript{179:1.5} Con las posiciones más elevadas y más bajas ya ocupadas, los demás apóstoles escogieron sus sitios, algunos cerca de Judas y otros cerca de Pedro, hasta que todos estuvieron instalados. Estaban sentados alrededor de la mesa en forma de U, en estos divanes reclinados, en el orden siguiente: a la derecha del Maestro, Juan; a la izquierda, Judas, Simón Celotes, Mateo, Santiago Zebedeo, Andrés, los gemelos Alfeo, Felipe, Natanael, Tomás y Simón Pedro.

\par
%\textsuperscript{(1937.4)}
\textsuperscript{179:1.6} Están reunidos para celebrar, al menos en espíritu, una institución que databa incluso de un período anterior a Moisés y que se refería a la época en que sus antepasados eran esclavos en Egipto. Esta cena es su último encuentro con Jesús, e incluso en esta ocasión solemne, bajo la dirección de Judas, los apóstoles se dejan llevar una vez más por su vieja predilección por el honor, la preferencia y la exaltación personal.

\par
%\textsuperscript{(1937.5)}
\textsuperscript{179:1.7} Aún estaban diciéndose recriminaciones irritadas cuando el Maestro apareció en la puerta, donde vaciló un instante mientras una expresión de desencanto se deslizaba lentamente por su rostro. Sin hacer ningún comentario se dirigió a su sitio, y no cambió la distribución de los asientos\footnote{\textit{Jesús entra y se sienta}: Mt 26:20; Mc 14:17; Lc 22:14.}.

\par
%\textsuperscript{(1937.6)}
\textsuperscript{179:1.8} Ahora estaban preparados para empezar la cena, salvo que aún no se habían lavado los pies, y que su estado de ánimo era de todo menos agradable. Cuando el Maestro llegó, aún se estaban haciendo comentarios desfavorables unos a otros, por no decir nada de los pensamientos de algunos de ellos, que tenían el suficiente control emocional como para abstenerse de expresar públicamente sus sentimientos.

\section*{2. El comienzo de la cena}
\par
%\textsuperscript{(1937.7)}
\textsuperscript{179:2.1} Después de que el Maestro hubiera ocupado su lugar, no se dijo ni una palabra durante unos momentos. Jesús los examinó a todos y suavizó la tensión con una sonrisa, diciendo: «He deseado mucho comer esta Pascua con vosotros. Quería comer una vez más con vosotros antes de mi sufrimiento, y sabiendo que mi hora ha llegado, he organizado esta cena con vosotros para esta noche porque, en cuanto al mañana, todos estamos en las manos del Padre, cuya voluntad he venido a hacer. No volveré a comer con vosotros hasta que os sentéis conmigo en el reino que mi Padre me dará cuando haya terminado aquello para lo que me envió a este mundo»\footnote{\textit{Jesús les habla sobre el fin}: Lc 22:15-16.}.

\par
%\textsuperscript{(1938.1)}
\textsuperscript{179:2.2} Después de haber mezclado el agua y el vino, trajeron la copa a Jesús, y cuando la hubo recibido de las manos de Tadeo, la sostuvo mientras daba gracias. Cuando hubo terminado de dar gracias, dijo: «Tomad esta copa y compartidla entre vosotros, y cuando la bebáis, sabed que no volveré a beber con vosotros el fruto de la vid puesto que ésta es nuestra última cena. Cuando nos sentemos de nuevo de esta manera, será en el reino venidero»\footnote{\textit{La primera copa}: Lc 22:17-18. \textit{La segunda copa}: Mt 26:29; Mc 14:25; Lc 22:20.}.

\par
%\textsuperscript{(1938.2)}
\textsuperscript{179:2.3} Jesús empezó a hablar así a sus apóstoles porque sabía que su hora había llegado. Comprendía que había llegado el momento en que debía regresar al Padre, y que su obra en la Tierra estaba casi terminada\footnote{\textit{Jesús sabía que su obra estaba terminada}: Jn 13:1-3.}. El Maestro sabía que había revelado el amor del Padre en la Tierra y había mostrado su misericordia a la humanidad, y que había completado aquello para lo que había venido al mundo, incluido el recibir todo el poder y la autoridad en el cielo y en la Tierra. Asimismo, sabía que Judas Iscariote había decidido plenamente entregarlo esta noche en manos de sus enemigos. Se daba completamente cuenta de que esta pérfida traición era obra de Judas, pero que también agradaba a Lucifer, Satanás y Caligastia, el príncipe de las tinieblas\footnote{\textit{La traición agradaba a los malignos}: Ef 6:12.}. Pero no le temía a ninguno de los que perseguían su derrota espiritual, así como tampoco a los que buscaban su muerte física. El Maestro sólo tenía una inquietud, y era la seguridad y la salvación de sus seguidores escogidos. Y así, sabiendo por completo que el Padre había puesto todas las cosas bajo su autoridad, el Maestro se preparó ahora para poner en práctica la parábola del amor fraterno.

\section*{3. El lavado de pies de los apóstoles}
\par
%\textsuperscript{(1938.3)}
\textsuperscript{179:3.1} Después de beber la primera copa de la Pascua, era costumbre judía que el anfitrión se levantara de la mesa y se lavara las manos. En el transcurso de la comida y después de la segunda copa, todos los invitados se levantaban igualmente y se lavaban las manos. Puesto que los apóstoles sabían que su Maestro nunca guardaba estos ritos de lavado ceremonial de las manos, tenían mucha curiosidad por saber qué se proponía hacer después de que hubieran compartido esta primera copa. Jesús se levantó de la mesa y se dirigió silenciosamente hacia el lado de la puerta donde habían sido colocados los cántaros de agua, las palanganas y las toallas. Y su curiosidad se transformó en asombro cuando vieron que el Maestro se quitaba su manto, se ceñía una toalla y empezaba a echar agua en una de las palanganas para los pies. Imaginad la sorpresa de estos doce hombres, que se habían negado tan recientemente a lavarse los pies los unos a los otros, y que se habían enredado en disputas indecentes acerca de los lugares de honor en la mesa, cuando le vieron rodear el extremo libre de la mesa hasta llegar al asiento más bajo del festín, donde Simón Pedro estaba recostado, y arrodillándose como si fuera un criado, se preparó para lavarle los pies a Simón. Cuando el Maestro se arrodilló, los doce se levantaron como un solo hombre; incluso el traidor Judas olvidó por un momento su infamia hasta el punto de que se levantó con sus compañeros apóstoles en esta expresión de sorpresa, de respeto y de asombro total\footnote{\textit{Jesús se prepara para lavarles los pies}: Jn 13:4-5a.}.

\par
%\textsuperscript{(1938.4)}
\textsuperscript{179:3.2} Allí estaba de pie Simón Pedro, bajando la mirada hacia el rostro alzado de su Maestro. Jesús no dijo nada; no era necesario que hablara. Su actitud revelaba claramente que tenía la intención de lavar los pies de Simón Pedro. A pesar de sus debilidades humanas, Pedro amaba al Maestro. Este pescador galileo fue el primer ser humano que creyó de todo corazón en la divinidad de Jesús \textit{y} que confesó plena y públicamente esta creencia\footnote{\textit{Pedro el primer «creyente»}: Mt 16:15-16; Mc 8:29; Lc 9:20.}. Y desde entonces, Pedro nunca había dudado realmente de la naturaleza divina del Maestro. Puesto que Pedro veneraba y honraba así a Jesús en su corazón, no es de extrañar que a su alma le molestara la idea de que Jesús estuviera arrodillado allí delante de él como un vulgar criado, con el propósito de lavarle los pies como lo hubiera hecho un esclavo. Cuando Pedro recuperó las suficientes facultades como para dirigirse al Maestro, expresó los sentimientos internos de todos sus compañeros apóstoles.

\par
%\textsuperscript{(1939.1)}
\textsuperscript{179:3.3} Después de unos momentos de gran desconcierto, Pedro dijo: «Maestro, ¿tienes realmente la intención de lavarme los pies?» Entonces, levantando la mirada hacia la cara de Pedro, Jesús dijo: «Quizás no comprendes plenamente lo que estoy a punto de hacer, pero más adelante conocerás el significado de todas estas cosas». Entonces, Simón Pedro respiró profundamente y dijo: «Maestro, ¡nunca me lavarás los pies!» Y cada uno de los apóstoles aprobó con la cabeza la firme declaración de Pedro de negarse a permitir que Jesús se humillara de esta manera delante de ellos\footnote{\textit{Pedro se niega a ser lavado}: Jn 13:6-8a.}.

\par
%\textsuperscript{(1939.2)}
\textsuperscript{179:3.4} El atractivo dramático de esta escena insólita conmovió al principio el corazón incluso de Judas Iscariote; pero cuando su intelecto vanidoso juzgó el espectáculo, concluyó que este gesto de humildad era simplemente un episodio más que probaba de manera concluyente que Jesús nunca estaría capacitado para ser el libertador de Israel, y que él, Judas, no había cometido un error al decidir abandonar la causa del Maestro.

\par
%\textsuperscript{(1939.3)}
\textsuperscript{179:3.5} Mientras todos permanecían allí de pie sin aliento por el asombro, Jesús dijo: «Pedro, te aseguro que si no te lavo los pies, no participarás conmigo en lo que estoy a punto de realizar». Cuando Pedro escuchó esta declaración, unida al hecho de que Jesús continuaba arrodillado allí a sus pies, tomó una de esas decisiones de sumisión ciega consistente en obedecer el deseo de aquel a quien respetaba y amaba. Cuando Simón Pedro empezó a darse cuenta de que este acto de servicio propuesto comportaba algún significado que determinaría la unión futura del interesado con la obra del Maestro, no solamente admitió la idea de permitir que Jesús le lavara los pies, sino que con su manera de ser característica e impetuosa, dijo: «Entonces, Maestro, no me laves solamente los pies, sino también las manos y la cabeza»\footnote{\textit{Pedro accede}: Jn 13:8b-9.}.

\par
%\textsuperscript{(1939.4)}
\textsuperscript{179:3.6} Mientras el Maestro se preparaba para empezar a lavar los pies de Pedro, dijo: «El que ya está limpio, sólo necesita que le laven los pies. Vosotros que estáis sentados conmigo esta noche, estáis limpios ---pero no todos. Pero el polvo de vuestros pies debería haberse lavado antes de sentaros a comer conmigo. Además, quisiera hacer este servicio por vosotros como una parábola, para ilustrar el significado de un nuevo mandamiento que pronto os daré»\footnote{\textit{Lavando los pies de Pedro}: Jn 13:10-11.}.

\par
%\textsuperscript{(1939.5)}
\textsuperscript{179:3.7} De la misma manera, el Maestro se desplazó alrededor de la mesa, en silencio, lavando los pies de sus doce apóstoles, sin excluir siquiera a Judas. Cuando Jesús hubo terminado de lavar los pies de los doce\footnote{\textit{Lavando los pies de los apóstoles}: Jn 13:12a.}, se puso su manto, volvió a su asiento de anfitrión, y después de examinar a sus apóstoles desconcertados, dijo:

\par
%\textsuperscript{(1939.6)}
\textsuperscript{179:3.8} «¿Comprendéis realmente lo que os he hecho? Me llamáis Maestro, y decís bien, porque lo soy. Así pues, si el Maestro os ha lavado los pies, ¿por qué no estabais dispuestos a lavaros los pies los unos a los otros? ¿Qué lección deberíais aprender de esta parábola en la que el Maestro hace tan gustosamente el servicio que sus hermanos eran reacios a hacerse los unos a los otros?\footnote{\textit{Servirse unos a otros}: Jn 13:12b-17.} En verdad, en verdad os lo digo: Un servidor no es más grande que su señor; ni el enviado es más grande que aquel que lo envía. Habéis visto en mi vida entre vosotros cómo se ha de servir, y benditos sean los que tengan el coraje misericordioso de servir así. Pero, ¿por qué sois tan lentos en aprender que el secreto de la grandeza en el reino espiritual no se parece a los métodos de poder del mundo material?»\footnote{\textit{Discurso del maestro y el servidor}: Mt 10:24; Lc 22:27a; Jn 15:20.}

\par
%\textsuperscript{(1940.1)}
\textsuperscript{179:3.9} «Cuando entré esta noche en esta sala, no os contentabais con negaros orgullosamente a lavaros los pies los unos a los otros, sino que también teníais que discutir entre vosotros sobre quiénes ocuparían los lugares de honor en mi mesa\footnote{\textit{Jesus sabía de la disputa sobre los honores}: Lc 22:24.}. Esos honores los buscan los fariseos y los hijos de este mundo, pero no debería ser así entre los embajadores del reino celestial. ¿No sabéis que en mi mesa no puede haber ningún lugar de preferencia? ¿No comprendéis que amo a cada uno de vosotros como a los demás? ¿No sabéis que el asiento más cercano a mí, considerado como un honor por los hombres, no significa nada en lo que respecta a vuestra posición en el reino de los cielos? Sabéis que los reyes de los gentiles tienen el dominio sobre sus súbditos, y que a veces se les llama benefactores a los que ejercen esta autoridad. Pero no será así en el reino de los cielos. El que quiera ser grande entre vosotros, que se vuelva como el más joven; y el que quiera ser el jefe, que se convierta en el que sirve\footnote{\textit{Importancia del servicio}: Mc 10:42-45; Lc 22:25-27.}. ¿Quién es más grande, el que se sienta a comer, o el que sirve? ¿No se considera generalmente que el que se sienta a comer es el más grande? Pero observaréis que estoy entre vosotros como alguien que sirve. Si estáis dispuestos a ser compañeros míos en el servicio para hacer la voluntad del Padre, os sentaréis conmigo con poder en el reino venidero\footnote{\textit{Recompensa por el servicio}: Lc 22:28-30.}, haciendo sin cesar la voluntad del Padre en la gloria futura».

\par
%\textsuperscript{(1940.2)}
\textsuperscript{179:3.10} Cuando Jesús hubo terminado de hablar, los gemelos Alfeo trajeron el pan y el vino, con las hierbas amargas y la pasta de frutos secos, que componían el plato siguiente de la Última Cena.

\section*{4. Las últimas palabras al traidor}
\par
%\textsuperscript{(1940.3)}
\textsuperscript{179:4.1} Los apóstoles comieron en silencio durante algunos minutos, pero debido a la influencia de la conducta jovial del Maestro, pronto se sintieron incitados a la conversación, y en muy poco rato la cena continuó como si no hubiera ocurrido nada fuera de lo común que alterara el buen humor y la armonía social de esta extraordinaria ocasión. Después de haber transcurrido cierto tiempo, hacia la mitad de este segundo servicio de la comida, Jesús los miró a todos diciendo: «Os he dicho cuánto deseaba compartir esta cena con vosotros, y sabiendo de qué manera las fuerzas malignas de las tinieblas han conspirado para provocar la muerte del Hijo del Hombre, he decidido tomar esta cena con vosotros en esta sala secreta, un día antes de la Pascua, porque mañana por la noche a esta hora ya no estaré con vosotros. Os he repetido muchas veces que debo regresar al Padre. Ahora ha llegado mi hora, pero no era necesario que uno de vosotros me traicionara entregándome a mis enemigos»\footnote{\textit{Anuncio de la inminente traición}: Mt 26:21; Mc 14:18; Jn 13:21.}.

\par
%\textsuperscript{(1940.4)}
\textsuperscript{179:4.2} La parábola del lavado de los pies y el discurso posterior del Maestro ya habían hecho perder a los doce una buena parte de su presunción y de su confianza en sí mismos. Cuando escucharon esto, empezaron a mirarse unos a otros y a preguntarse vacilantes con tono desconcertado: «¿Soy yo?»\footnote{\textit{«¿Soy yo?»}: Mt 26:22; Mc 14:19; Lc 22:23; Jn 13:22.} Cuando todos hubieron preguntado esto, Jesús dijo: «Aunque es necesario que regrese al Padre, no hacía falta que uno de vosotros se convirtiera en un traidor para cumplir la voluntad del Padre\footnote{\textit{No era necesario un traidor}: Mt 26:24; Mc 14:21; Lc 22:21-22.}. Esto es la maduración del mal escondido en el corazón de uno que no ha logrado amar la verdad con toda su alma. ¡Cuán engañoso es el orgullo intelectual que precede a la caída espiritual! Mi amigo de muchos años, que ahora mismo come mi pan, está dispuesto a traicionarme, incluso ahora que mete su mano conmigo en el mismo plato»\footnote{\textit{Traicionado por un amigo}: Mt 26:23; Mc 14:18b,20.}.

\par
%\textsuperscript{(1940.5)}
\textsuperscript{179:4.3} Cuando Jesús hubo hablado así, todos empezaron de nuevo a preguntar: «¿Soy yo?». Cuando Judas, que estaba sentado a la izquierda de su Maestro, preguntó de nuevo: «¿Soy yo?», Jesús mojó el pan en el plato de las hierbas y se lo dio a Judas diciendo: «Tú lo has dicho»\footnote{\textit{Jesús da pan untado a Judas}: Mt 26:25; Jn 13:25-26.}. Pero los demás no escucharon a Jesús hablarle a Judas. Juan, que estaba recostado a la derecha de Jesús, se inclinó y le preguntó al Maestro: «¿Quién es? Deberíamos saber quién se ha mostrado infiel a su deber». Jesús respondió: «Ya os he dicho que es aquel a quien le he dado el pan mojado». Pero era tan natural que el anfitrión diera el pan mojado al que estaba sentado a su izquierda, que ninguno le prestó atención a este hecho, aunque el Maestro se hubiera expresado con toda claridad. Pero Judas era dolorosamente consciente del significado de las palabras del Maestro unidas a su acción, y empezó a temer que sus hermanos también se dieran cuenta ahora de que él era el traidor.

\par
%\textsuperscript{(1941.1)}
\textsuperscript{179:4.4} Pedro estaba bastante excitado por lo que se había dicho; se inclinó sobre la mesa y se dirigió a Juan: «Pregúntale quién es, o si te lo ha dicho, dime quién es el traidor»\footnote{\textit{Pedro pregunta a Juan quién es el traidor}: Jn 13:23-24.}.

\par
%\textsuperscript{(1941.2)}
\textsuperscript{179:4.5} Jesús puso fin a sus cuchicheos diciendo: «Me apena que este mal haya tenido que ocurrir y he esperado hasta este mismo momento que el poder de la verdad pudiera triunfar sobre los engaños del mal, pero esas victorias no se ganan sin la fe del amor sincero a la verdad. No hubiera querido deciros estas cosas en nuestra última cena, pero deseo advertiros de estas penas y prepararos así para lo que nos espera dentro de poco. Os he dicho esto porque deseo que recordéis, después de mi partida, que conocía todos estos perversos complots, y que os avisé de que iba a ser traicionado. Hago todo esto únicamente para que os sintáis fortalecidos en las tentaciones y pruebas que os esperan»\footnote{\textit{Jesús revela el futuro}: Jn 13:19.}.

\par
%\textsuperscript{(1941.3)}
\textsuperscript{179:4.6} Después de haber hablado así, Jesús se inclinó hacia Judas y le dijo: «Lo que has decidido hacer, hazlo enseguida»\footnote{\textit{Jesús envía a Judas}: Jn 13: 27b-30.}. Cuando Judas escuchó estas palabras, se levantó de la mesa y abandonó apresuradamente la habitación, saliendo a la noche para hacer lo que había decidido llevar a cabo. Cuando los otros apóstoles vieron que Judas salía precipitadamente después de que Jesús le hubiera hablado, creyeron que había ido a buscar algo más para la cena o a hacer algún otro recado para el Maestro, pues suponían que aún tenía la bolsa.

\par
%\textsuperscript{(1941.4)}
\textsuperscript{179:4.7} Jesús sabía ahora que no se podía hacer nada para impedir que Judas se convirtiera en un traidor. Había empezado con doce hombres ---ahora tenía once. Había elegido a seis de estos apóstoles, y aunque Judas se encontraba entre los que habían sido nombrados por sus primeros apóstoles escogidos, el Maestro lo había aceptado, y hasta este mismo momento había hecho todo lo posible por santificarlo y salvarlo, tal como había trabajado por la paz y la salvación de los demás.

\par
%\textsuperscript{(1941.5)}
\textsuperscript{179:4.8} Esta cena, con sus tiernos episodios y sus detalles suaves, fue el último llamamiento de Jesús al desertor Judas, pero fue en vano. Una vez que el amor está realmente muerto, aunque las advertencias se hagan con el máximo de tacto y se transmitan con el espíritu más cariñoso, por regla general sólo intensifican el odio y encienden la malvada resolución de llevar a cabo íntegramente nuestros propios proyectos egoístas.

\section*{5. El establecimiento de la cena del recuerdo}
\par
%\textsuperscript{(1941.6)}
\textsuperscript{179:5.1} Cuando trajeron a Jesús la tercera copa de vino, la «copa de la bendición»\footnote{\textit{La copa de la bendición}: Mt 26:27-29; Mc 14:23-25; Lc 22:20; 1 Co 11:25-26.}, se levantó del diván, tomó la copa en sus manos y la bendijo, diciendo: «Tomad todos esta copa, y bebed de ella. Ésta será la copa de mi recuerdo. Ésta es la copa de la bendición de una nueva dispensación de gracia y de verdad. Será para vosotros el emblema de la donación y del ministerio del Espíritu divino de la Verdad. No volveré a beber esta copa con vosotros hasta que beba de una nueva forma con vosotros en el reino eterno del Padre».

\par
%\textsuperscript{(1942.1)}
\textsuperscript{179:5.2} Mientras bebían esta copa de la bendición con un profundo respeto y en un silencio perfecto, todos los apóstoles sintieron que estaba teniendo lugar algo fuera de lo común. La vieja Pascua conmemoraba la salida de sus padres de un estado de esclavitud racial a otro de libertad individual; ahora, el Maestro instituía una nueva cena de conmemoración como símbolo de la nueva dispensación en la que el individuo esclavizado emerge del cautiverio del ceremonialismo y del egoísmo, y pasa a la alegría espiritual de la fraternidad y la comunidad de los hijos por la fe, liberados, que pertenecen al Dios vivo.

\par
%\textsuperscript{(1942.2)}
\textsuperscript{179:5.3} Cuando terminaron de beber esta nueva copa del recuerdo, el Maestro cogió el pan y, después de dar gracias, lo rompió en pedazos y les pidió que lo pasaran, diciendo: «Tomad este pan del recuerdo y comedlo. Os he dicho que yo soy el pan de la vida\footnote{\textit{Jesús es el pan de la vida}: Jn 6:35,48.}. Y este pan de la vida es la vida unida del Padre y del Hijo en un solo don. La palabra del Padre, tal como es revelada en el Hijo, es en verdad el pan de la vida»\footnote{\textit{El pan de la vida}: Mt 26:26; Mc 14:22; Lc 22:19; 1 Co 11:22b-24.}. Cuando hubieron compartido el pan de la conmemoración, símbolo de la palabra viviente de la verdad encarnada en la similitud de la carne mortal, todos se sentaron.

\par
%\textsuperscript{(1942.3)}
\textsuperscript{179:5.4} Al instituir esta cena del recuerdo, el Maestro recurrió, como siempre tenía costumbre, a las parábolas y a los símbolos. Empleó símbolos porque quería enseñar ciertas grandes verdades espirituales de tal manera que a sus sucesores les resultara difícil atribuir a sus palabras interpretaciones precisas y significados definidos. De esta manera, trataba de impedir que las generaciones siguientes cristalizaran su enseñanza y vincularan sus significados espirituales con las cadenas muertas de la tradición y de los dogmas. Al establecer la única ceremonia, o sacramento, asociada a la totalidad de la misión de su vida, Jesús se esmeró mucho en \textit{sugerir} sus significados, en lugar de recurrir a \textit{definicionesprecisas}. No quería destruir el concepto individual de la comunión divina, estableciendo una práctica precisa; tampoco deseaba limitar la imaginación espiritual del creyente, restringiéndola de manera formalista. Trataba más bien de liberar el alma renacida del hombre para que emprendiera el vuelo con las alas gozosas de una libertad espiritual nueva y viviente.

\par
%\textsuperscript{(1942.4)}
\textsuperscript{179:5.5} A pesar del esfuerzo del Maestro por establecer así este nuevo sacramento de conmemoración, aquellos que le siguieron en los siglos posteriores se encargaron de frustrar eficazmente su deseo expreso, en el sentido de que este simple simbolismo espiritual de aquella última noche en la carne ha sido reducido a interpretaciones precisas y sometido a la precisión casi matemática de una fórmula fija. De todas las enseñanzas de Jesús, ninguna ha sido más reglamentada por la tradición.

\par
%\textsuperscript{(1942.5)}
\textsuperscript{179:5.6} Cuando la cena del recuerdo es compartida por aquellos que creen en el Hijo y conocen a Dios, su simbolismo no necesita estar asociado a ninguna de las falsas interpretaciones pueriles del hombre sobre el significado de la presencia divina, porque en todas esas ocasiones, el Maestro está \textit{realmente presente}. La cena del recuerdo es el encuentro simbólico del creyente con Miguel. Cuando os volvéis así conscientes del espíritu, el Hijo está realmente presente, y su espíritu fraterniza con el fragmento interior de su Padre.

\par
%\textsuperscript{(1942.6)}
\textsuperscript{179:5.7} Después de que hubieron meditado unos momentos, Jesús continuó hablando: «Cuando hagáis estas cosas, recordad la vida que he vivido en la Tierra entre vosotros, y regocijaos con el hecho de que voy a continuar viviendo en la Tierra con vosotros y sirviendo a través de vosotros. Como individuos, no discutáis entre vosotros sobre quién será el más grande. Sed todos como hermanos. Cuando el reino crezca hasta abarcar grandes grupos de creyentes, deberíais absteneros también de luchar por la grandeza o de buscar la preferencia entre esos grupos»\footnote{\textit{Recuerdos de la vida de Jesús}: Lc 22:19c; 1 Co 11:24c,25c.}.

\par
%\textsuperscript{(1943.1)}
\textsuperscript{179:5.8} Este importante acontecimiento tuvo lugar en la habitación superior de un amigo. Ni la cena ni el edificio contenían ninguna forma sagrada o consagración ceremonial. La cena del recuerdo fue establecida sin aprobación eclesiástica.

\par
%\textsuperscript{(1943.2)}
\textsuperscript{179:5.9} Cuando Jesús hubo establecido así la cena del recuerdo, dijo a sus apóstoles: «Cada vez que hagáis esto, hacedlo en memoria mía. Y cuando os acordéis de mí, reflexionad primero sobre mi vida en la carne, recordad que en otro tiempo estuve con vosotros, y luego discernid por la fe que todos cenaréis conmigo algún día en el reino eterno del Padre. Ésta es la nueva Pascua que os dejo, el recuerdo mismo de mi vida de donación, la palabra de la verdad eterna; y de mi amor por vosotros, os dejo la efusión de mi Espíritu de la Verdad sobre todo el género humano»\footnote{\textit{La cena del recuerdo}: Lc 22:19c; 1 Co 11:24-26.}.

\par
%\textsuperscript{(1943.3)}
\textsuperscript{179:5.10} Y terminaron la celebración de esta antigua pero incruenta Pascua en conexión con la inauguración de la nueva cena del recuerdo, cantando todos juntos el salmo ciento dieciocho\footnote{\textit{Final con un canto (Salmo 118)}: Sal 118:1-29; Mt 26:30a; Mc 14:26a.}.