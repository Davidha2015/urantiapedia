\chapter{Documento 181. Las últimas recomendaciones y advertencias}
\par 
%\textsuperscript{(1953.1)}
\textsuperscript{181:0.1} DESPUÉS de terminar el discurso de despedida a los once, Jesús conversó familiarmente con ellos y recordó muchas experiencias que les concernía como grupo y como individuos. Estos galileos empezaban por fin a darse cuenta de que su amigo e instructor iba a dejarlos, y su esperanza se aferraba a la promesa de que después de poco tiempo estaría de nuevo con ellos, pero eran propensos a olvidar que este regreso también sería por poco tiempo. Muchos apóstoles y discípulos principales creían realmente que esta promesa de volver durante una corta temporada (el corto intervalo entre la resurrección y la ascensión) indicaba que Jesús sólo se iba para conversar brevemente con su Padre, después de lo cual volvería para establecer el reino. Esta interpretación de su enseñanza concordaba tanto con sus creencias preconcebidas como con sus ardientes esperanzas. Puesto que sus creencias de toda la vida y sus esperanzas de ver realizados sus anhelos se armonizaban de esta manera, no les fue difícil encontrar una interpretación de las palabras del Maestro que justificara sus intensos deseos.

\par 
%\textsuperscript{(1953.2)}
\textsuperscript{181:0.2} Después de haber debatido el discurso de despedida y de haber empezado a asimilarlo, Jesús llamó de nuevo a los apóstoles al orden y empezó a impartirles sus últimas recomendaciones y advertencias.

\section*{1. Las últimas palabras de consuelo}
\par 
%\textsuperscript{(1953.3)}
\textsuperscript{181:1.1} Cuando los once se hubieron sentado, Jesús se levantó y les dirigió la palabra: «Mientras que esté con vosotros en la carne, sólo puedo ser una persona en medio de vosotros o en el mundo entero. Pero cuando haya sido liberado de esta envoltura de naturaleza mortal, podré regresar como habitante espiritual a cada uno de vosotros y de todos los demás creyentes en este evangelio del reino. De esta manera, el Hijo del Hombre se volverá una encarnación espiritual\footnote{\textit{Volver como una encarnación espiritual}: Jn 16:7.} en el alma de todos los verdaderos creyentes».

\par 
%\textsuperscript{(1953.4)}
\textsuperscript{181:1.2} «Cuando haya regresado para vivir en vosotros y trabajar a través de vosotros, podré continuar conduciéndoos mejor por esta vida y guiaros a través de las muchas moradas\footnote{\textit{El Cielo: muchas moradas}: Jn 14:2-3.} en la vida futura en el cielo de los cielos\footnote{\textit{Cielo de los cielos}: 1 Re 8:27; 2 Cr 2:6; 6:18; Neh 9:6; Sal 148:4; Dt 10:14.}. La vida en la creación eterna del Padre no es un descanso sin fin en la ociosidad ni un reposo egoísta, sino más bien una progresión continua en la gracia, la verdad y la gloria. Cada una de las muchísimas estaciones en la casa de mi Padre es una parada, una vida destinada a prepararos para la siguiente. Los hijos de la luz\footnote{\textit{Hijos de la luz}: Lc 16:8; Jn 12:36; Ef 5:8; 1 Ts 5:5.} continuarán así de gloria en gloria\footnote{\textit{De gloria en gloria}: 2 Co 3:18.} hasta que alcancen el estado divino en el que estarán perfeccionados espiritualmente\footnote{\textit{Convertirse en perfectos espiritualmente}: Gn 17:1; 1 Re 8:61; Lv 19:2; Dt 18:13; Mt 5:48; 2 Co 13:11; Stg 1:4; 1 P 1:16.} como el Padre es perfecto en todas las cosas».

\par 
%\textsuperscript{(1953.5)}
\textsuperscript{181:1.3} «Si queréis seguir mis pasos cuando os haya dejado, esforzaos seriamente por vivir de acuerdo con el espíritu de mis enseñanzas\footnote{\textit{Vivir con el espíritu de mis enseñanzas}: Jn 14:23-24.} y el ideal de mi vida ---hacer la voluntad de mi Padre. Haced esto, en lugar de intentar imitar mi vida sencilla en la carne tal como me he visto obligado a vivirla, necesariamente, en este mundo».

\par 
%\textsuperscript{(1954.1)}
\textsuperscript{181:1.4} «El Padre me ha enviado a este mundo, pero sólo unos pocos de vosotros habéis elegido recibirme plenamente\footnote{\textit{Pocos me habéis recibido}: Jn 1:11-12; Jn 7:43.}. Derramaré mi espíritu sobre todo el género humano\footnote{\textit{Derramaré mi espíritu sobre toda la carne}: Ez 11:19; 18:31; 36:26-27; Jl 2:28-29; Lc 24:49; Jn 7:39; 14:16-18,23,26; 15:4,26; 16:7-8,13-14; 17:21-23; Hch 1:5,8a; 2:1-4,16-18; 2:33; 2 Co 13:5; Gl 2:20; 4:6; Ef 1:13; 4:30; 1 Jn 4:12-15.}, pero no todos los hombres escogerán recibir\footnote{\textit{A todos cuantos le recibieron}: Jn 1:12; 17:2; Hch 2:38-39; Ro 8:14; Gl 6:16.} a este nuevo instructor como guía y consejero del alma. Pero todos los que lo reciban serán iluminados, purificados y confortados. Y este Espíritu de la Verdad se transformará en ellos en una fuente de agua viva que brotará hasta en la vida eterna»\footnote{\textit{El Espíritu de la Verdad, fuente del agua viva}: Jn 4:10-14; Jn 7:38; Ap 7:17.}.

\par 
%\textsuperscript{(1954.2)}
\textsuperscript{181:1.5} «Y ahora que estoy a punto de dejaros, quisiera decir unas palabras de consuelo. Os dejo la paz; mi paz os doy. Os concedo estos dones, no como los ofrece el mundo ---por medidas--- sino que doy a cada uno de vosotros todo lo que quiera recibir. Que vuestro corazón no se perturbe ni sienta temor\footnote{\textit{Que vuestro corazón no se perturbe}: Jn 14:1a,27b.}. Yo he vencido al mundo\footnote{\textit{Yo he vencido al mundo}: Jn 16:33.}, y en mí todos triunfaréis por la fe. Os he advertido que el Hijo del Hombre será ejecutado, pero os aseguro que volveré\footnote{\textit{Moriré pero resucitaré}: Mt 16:21; 17:22-23a; 20:17-19; 27:63; Mc 8:31; 9:31; 10:32-34; Lc 9:22,31,43b-44; 18:31-33; 24:7,46; Jn 14:28a; 20:9.} antes de ir hacia el Padre, aunque sólo sea por poco tiempo. Y después de haber ascendido hasta el Padre, enviaré con seguridad al nuevo instructor para que esté con vosotros y resida en vuestro propio corazón\footnote{\textit{El Espíritu de la Verdad reside en los corazones}: Ez 11:19; 18:31; 36:26-27.}. Cuando veáis que sucede todo esto, no os desalentéis, sino más bien creed\footnote{\textit{No os desalentéis, sino creed}: Jn 14:29.}, puesto que lo sabíais todo de antemano. Os he amado con un gran afecto y no quisiera dejaros, pero esa es la voluntad del Padre\footnote{\textit{Os he amado y hecho la voluntad del Padre}: Jn 14:28b,31; 15:9.}. Mi hora ha llegado».

\par 
%\textsuperscript{(1954.3)}
\textsuperscript{181:1.6} «No dudéis de ninguna de estas verdades, incluso cuando estéis dispersos por las persecuciones y abatidos por numerosas tristezas. Cuando os sintáis solos en el mundo, yo conoceré vuestra soledad, al igual que vosotros conoceréis la mía cuando estéis dispersos cada uno por su lado, dejando al Hijo del Hombre en manos de sus enemigos. Pero nunca estoy solo\footnote{\textit{Nunca estamos solos}: Jn 16:32-33.}; el Padre siempre está conmigo. Incluso en esos momentos rezaré por vosotros. Os he contado todas estas cosas para que podáis tener paz y tenerla más abundantemente. Tendréis tribulaciones en este mundo, pero tened buen ánimo; he triunfado en el mundo y os he mostrado el camino de la alegría eterna y del servicio perpetuo».

\par 
%\textsuperscript{(1954.4)}
\textsuperscript{181:1.7} Jesús da la paz a los que hacen con él la voluntad de Dios\footnote{\textit{Jesús da paz a los que le siguen}: Jn 14:27a.}, pero esta paz no es semejante a las alegrías y satisfacciones de este mundo material. Los materialistas y los fatalistas incrédulos sólo pueden esperar disfrutar de dos tipos de paz y de consuelo del alma: o bien deben ser estoicos, decididos a enfrentarse a lo inevitable y a soportar lo peor con una resolución firme; o bien deben ser optimistas, contentándose siempre con esa esperanza que brota perpetuamente en el seno del hombre, anhelando en vano una paz que nunca llega realmente.

\par 
%\textsuperscript{(1954.5)}
\textsuperscript{181:1.8} Cierta cantidad de estoicismo y de optimismo son útiles para vivir la vida en la Tierra, pero ninguno de los dos tiene nada que ver con esa paz espléndida que el Hijo de Dios confiere a sus hermanos en la carne. La paz que Miguel da a sus hijos de la Tierra es la misma paz que llenaba su propia alma cuando él mismo vivía la vida mortal en la carne y en este mismo mundo. La paz de Jesús es la alegría y la satisfacción de una persona que conoce a Dios, y que ha logrado el triunfo de aprender plenamente a hacer la voluntad de Dios mientras vive la vida mortal en la carne. La paz mental de Jesús estaba fundada en una fe humana absoluta en la realidad de los cuidados sabios y compasivos del Padre divino. Jesús tuvo dificultades en la Tierra, incluso se le había llamado falsamente el «hombre de dolores»\footnote{\textit{Jesús no fue un «hombre de dolores»}: Is 53:3.}, pero en todas estas experiencias y a través de ellas, disfrutó del consuelo de esa confianza que siempre le dio fuerzas para seguir adelante con el objetivo de su vida, con la plena seguridad de que estaba realizando la voluntad del Padre.

\par 
%\textsuperscript{(1954.6)}
\textsuperscript{181:1.9} Jesús era decidido, perseverante y estaba completamente dedicado a realizar su misión, pero no era un estoico insensible y endurecido; siempre buscaba los aspectos alegres en las experiencias de su vida, pero no era un optimista ciego que se engañara a sí mismo. El Maestro sabía todo lo que le sucedería, y no tenía miedo. Después de haber otorgado esta paz a cada uno de sus seguidores, podía decir de manera coherente: «Que vuestro corazón no se perturbe\footnote{\textit{Que vuestro corazón no se perturbe}: Jn 14:1a,27b.} ni sienta temor\footnote{\textit{Ni sienta temor}: Jn 14:27c.}».

\par 
%\textsuperscript{(1955.1)}
\textsuperscript{181:1.10} La paz de Jesús es pues la paz y la seguridad de un hijo que cree plenamente que su carrera en el tiempo y en la eternidad está totalmente a salvo bajo el cuidado y la vigilancia de un Padre espíritu infinitamente sabio, amoroso y poderoso. Ésta es, en verdad, una paz que sobrepasa el entendimiento\footnote{\textit{Paz que sobrepasa el entendimiento}: Flp 4:7.} de la mente mortal, pero que el corazón humano creyente puede disfrutar plenamente.

\section*{2. Las recomendaciones personales de despedida}
\par 
%\textsuperscript{(1955.2)}
\textsuperscript{181:2.1} El Maestro había terminado de dar sus instrucciones de despedida y de impartir sus exhortaciones finales a los apóstoles como grupo. Luego se dirigió a ellos para decirles adiós individualmente y para darle a cada uno sus consejos personales así como su bendición de despedida. Los apóstoles continuaban sentados alrededor de la mesa tal como se habían instalado al principio para compartir la Última Cena. A medida que el Maestro rodeaba la mesa y hablaba con ellos, cada uno se ponía de pie cuando Jesús se dirigía a él.

\par 
%\textsuperscript{(1955.3)}
\textsuperscript{181:2.2} A Juan, Jesús le dijo: «Tú, Juan, eres el más joven de mis hermanos. Has estado muy cerca de mí, y aunque os amo a todos con el mismo amor que un padre tiene por sus hijos, Andrés te designó como uno de los tres que siempre debían estar cerca de mí. Además de esto, te has ocupado en mi nombre de muchos asuntos relacionados con mi familia terrenal, y debes continuar haciéndolo\footnote{\textit{Juan asignado a ayudar a la familia}: Jn 19:26-27.}. Y voy hacia el Padre, Juan, teniendo la plena confianza de que continuarás cuidando de los que son míos en la carne. Procura que la confusión que sufren actualmente sobre mi misión no te impida en absoluto concederles toda la simpatía, los consejos y la ayuda necesarios, como sabes que yo lo haría si tuviera que permanecer en la carne. Y cuando todos lleguen a ver la luz y entren plenamente en el reino, aunque todos vosotros los recibiréis con regocijo, cuento contigo Juan para darles la bienvenida en mi nombre».

\par 
%\textsuperscript{(1955.4)}
\textsuperscript{181:2.3} «Y ahora que comienzo las últimas horas de mi carrera terrenal, permanece cerca de mí para que pueda dejarte cualquier mensaje relacionado con mi familia. En lo que concierne a la obra que el Padre me confió\footnote{\textit{El trabajo confiado por el Padre}: Jn 4:34; 5:17,36; 9:4.}, ahora está terminada\footnote{\textit{El trabajo está terminado}: Jn 17:4; 19:30.} a excepción de mi muerte en la carne, y estoy listo para beber esta última copa. Pero en cuanto a las responsabilidades que me dejó mi padre terrenal José, las he atendido durante mi vida, pero ahora debo contar contigo para que actúes en mi nombre en todos esos asuntos. Te he elegido para que hagas esto por mí, Juan, porque eres el más joven, y por consiguiente es muy probable que vivas más tiempo que los otros apóstoles».

\par 
%\textsuperscript{(1955.5)}
\textsuperscript{181:2.4} «En otro tiempo os llamamos a ti y a tu hermano los hijos del trueno\footnote{\textit{Hijos del trueno}: Mc 3:17.}. Empezaste con nosotros siendo resuelto e intolerante, pero has cambiado mucho desde el día en que querías que hiciera bajar el fuego\footnote{\textit{Querías que bajara fuego}: Lc 9:54.} sobre la cabeza de los incrédulos ignorantes e irreflexivos. Y debes cambiar aún más. Deberías convertirte en el apóstol del nuevo mandamiento que os he dado esta noche. Dedica tu vida a enseñar a tus hermanos a amarse los unos a los otros como yo os he amado»\footnote{\textit{Amarse uno a otro}: Lv 19:18,34; Mt 5:43-44:; 19:19b; 22:39; Mc 12:31,33; Lc 10:27; Jn 13:34-35; 15:12,17; Ro 13:8-10; Gl 5:13-14; 1 Ts 4:9; Stg 2:8; 1 P 1:22; 1 Jn 3:11,23; 4:7,11-12,21; 2 Jn 1:5.}.

\par 
%\textsuperscript{(1955.6)}
\textsuperscript{181:2.5} Mientras Juan Zebedeo permanecía allí de pie en la habitación de arriba con las lágrimas corriendo por sus mejillas, miró de frente al Maestro y dijo: «Así lo haré, Maestro mío, pero, ¿cómo puedo aprender a amar más a mis hermanos?» Entonces Jesús respondió: «Aprenderás a amar más a tus hermanos cuando primero aprendas a amar más a su Padre que está en los cielos, y después de que te intereses realmente más por su bienestar en el tiempo y en la eternidad. Todo interés humano de este tipo se fomenta mediante la simpatía comprensiva, el servicio desinteresado y el perdón sin límites. Nadie debería menospreciar tu juventud, pero te exhorto a que siempre consideres debidamente el hecho de que la edad representa muchas veces la experiencia, y que en los asuntos humanos nada puede reemplazar a la experiencia real. Esfuérzate por vivir en paz\footnote{\textit{Esfuérzate por vivir en paz}: Ro 12:18.} con todos los hombres, especialmente con tus amigos en la fraternidad del reino celestial. Y recuerda siempre, Juan, no luches con las almas\footnote{\textit{No luches con las almas}: Pr 3:30; 2 Ti 2:14,24-25.} que quisieras ganar para el reino».

\par 
%\textsuperscript{(1956.1)}
\textsuperscript{181:2.6} Luego el Maestro rodeó su propio asiento y se detuvo un momento al lado del sitio de Judas Iscariote. Los apóstoles estaban un poco sorprendidos de que Judas aún no hubiera regresado, y tenían mucha curiosidad por conocer el significado de la expresión de tristeza en el rostro de Jesús, mientras éste permanecía al lado del asiento vacío del traidor. Pero ninguno de ellos, a excepción quizás de Andrés, albergaba la más leve sospecha de que su tesorero había salido para traicionar a su Maestro, tal como Jesús les había dado a entender anteriormente por la tarde y durante la cena. Habían sucedido tantas cosas que, por el momento, habían olvidado por completo la declaración del Maestro de que uno de ellos lo traicionaría.

\par 
%\textsuperscript{(1956.2)}
\textsuperscript{181:2.7} Jesús se acercó entonces a Simón Celotes, que se levantó para escuchar la siguiente exhortación: «Eres un verdadero hijo de Abraham, pero cuánto tiempo he estado intentando hacer de ti un hijo de este reino celestial. Te amo y todos tus hermanos también te aman. Sé que me amas, Simón, y que también amas al reino, pero aún tienes la idea fija de hacer venir este reino según tus preferencias. Sé muy bien que acabarás por captar la naturaleza y el significado espirituales de mi evangelio, y que trabajarás valientemente para proclamarlo, pero me preocupa lo que pueda sucederte cuando yo me vaya. Me alegraría saber que no vacilarás; sería feliz si pudiera saber que después de que me vaya hacia el Padre no dejarás de ser mi apóstol, y que te comportarás aceptablemente como embajador del reino celestial».

\par 
%\textsuperscript{(1956.3)}
\textsuperscript{181:2.8} Apenas había terminado Jesús de hablar a Simón Celotes, cuando el fogoso patriota, secándose los ojos, respondió: «Maestro, no temas por mi lealtad. Le he dado la espalda a todo para poder dedicar mi vida al establecimiento de tu reino en la Tierra, y no titubearé. Hasta ahora he sobrevivido a todas las decepciones, y no te abandonaré».

\par 
%\textsuperscript{(1956.4)}
\textsuperscript{181:2.9} Entonces, poniendo su mano en el hombro de Simón, Jesús dijo: «En verdad, es confortante oírte hablar así, especialmente en un momento como éste, pero mi buen amigo, aún no sabes de qué estás hablando. No dudo ni un instante de tu lealtad, de tu devoción. Sé que no dudarías en salir a luchar y morir por mí, como lo harían todos estos otros» (y todos asintieron enérgicamente con la cabeza), «pero no se te pedirá eso. Te he dicho repetidas veces que mi reino no es de este mundo\footnote{\textit{Mi reino no es de este mundo}: Jn 8:23; 18:36.}, y que mis discípulos no lucharán para establecerlo. Te he dicho esto muchas veces, Simón, pero te niegas a enfrentarte a la verdad. No me preocupa tu lealtad hacia mí y hacia el reino, sino ¿qué harás cuando me vaya y caigas por fin en la cuenta de que no has sabido captar el significado de mi enseñanza, y que debes ajustar tus ideas erróneas a la realidad de una clase de asuntos, diferente y espiritual, en el reino?»

\par 
%\textsuperscript{(1956.5)}
\textsuperscript{181:2.10} Simón quería hablar de nuevo, pero Jesús levantó la mano para detenerlo y continuó diciendo: «Ninguno de mis apóstoles tiene un corazón más sincero y honrado que tú, pero después de mi partida, ninguno de ellos se sentirá tan trastornado y tan desanimado como tú. Durante todo tu desánimo mi espíritu permanecerá contigo, y éstos, tus hermanos, no te abandonarán. No olvides lo que te he enseñado en cuanto a la relación entre la ciudadanía en la Tierra y la filiación en el reino espiritual del Padre. Reflexiona bien sobre todo lo que te he dicho acerca de dar al César las cosas que son del César y a Dios las que son de Dios\footnote{\textit{Al César lo del César y a Dios lo de Dios}: Mt 22:21; Mc 12:17; Lc 20:25.}. Dedica tu vida, Simón, a mostrar que el hombre mortal puede cumplir aceptablemente mi mandato de reconocer simultáneamente el deber temporal hacia los poderes civiles y el servicio espiritual en la fraternidad del reino. Si te dejas enseñar por el Espíritu de la Verdad, nunca habrá conflicto entre las exigencias de la ciudadanía en la Tierra y las de la filiación en el cielo, a menos que los gobernantes temporales se atrevan a exigirte el homenaje y la adoración que sólo pertenecen a Dios».

\par 
%\textsuperscript{(1957.1)}
\textsuperscript{181:2.11} «Y ahora, Simón, cuando veas finalmente todo esto, una vez que te hayas liberado de tu depresión y hayas salido a proclamar este evangelio con una gran energía\footnote{\textit{Predicar el evangelio con gran energía}: Hch 4:33.}, no olvides nunca que yo estaba contigo durante todo tu período de desánimo, y que continuaré contigo hasta el fin\footnote{\textit{Estaré contigo siempre}: Mt 28:20.}. Siempre serás mi apóstol, y una vez que estés dispuesto a ver con los ojos del espíritu y a someter más plenamente tu voluntad a la voluntad del Padre que está en los cielos, volverás a trabajar como embajador mío, y nadie te quitará la autoridad que te he conferido porque hayas sido lento en comprender las verdades que te he enseñado. Así pues, Simón, te advierto una vez más que los que combaten con la espada perecen por la espada\footnote{\textit{Los que combaten con la espada perecen por la espada}: Mt 26:52.}, mientras que los que trabajan en el espíritu consiguen la vida eterna en el reino venidero\footnote{\textit{La vida eterna en el reino venidero}: Dn 12:2; Mt 19:16,29; 25:46; Mc 10:17,30; Lc 10:25; 18:18,30; Jn 3:15-16,36; 4:14,36; 5:24,39; 6:27,40,47; 6:54.68; 8:51-52; 10:28; 11:25-26; 12:25,50; 17:2-3; Hch 13:46-48; Ro 2:7; 5:21; 6:22-23; Gl 6:8; 1 Ti 1:16; 6:12,19; Tit 1:2; 3:7; 1 Jn 1:2; 2:25; 3:15; 5:11,13,20; Jud 1:21; Ap 22:5.}, y la alegría y la paz en el reino presente. Cuando la obra que se te ha confiado haya terminado en la Tierra, tú, Simón, te sentarás conmigo en mi reino del más allá. Verás realmente el reino que has anhelado, pero no en esta vida. Continúa creyendo en mí y en lo que te he revelado, y recibirás el don de la vida eterna».

\par 
%\textsuperscript{(1957.2)}
\textsuperscript{181:2.12} Cuando Jesús hubo terminado de hablar a Simón Celotes, se acercó a Mateo Leví y dijo: «Ya no tendrás la responsabilidad de abastecer la tesorería del grupo apostólico. Pronto, muy pronto, todos estaréis dispersos; ni siquiera te permitirán disfrutar de la asociación consoladora y confortante con uno solo de tus hermanos. A medida que continuéis predicando este evangelio del reino, tendréis que encontrar nuevos asociados. Os he enviado de dos en dos durante la época de vuestra preparación, pero ahora que os dejo, cuando os hayáis recuperado de la conmoción, saldréis solos hasta los confines de la Tierra\footnote{\textit{Proclamar el evangelio al mundo}: Mt 24:14; 28:19-20a; Mc 13:10; 16:15; Lc 24:47; Jn 17:18; Hch 1:8b.}, proclamando esta buena nueva: Que los mortales vivificados por la fe son hijos de Dios»\footnote{\textit{Los creyentes, hijos de Dios}: 1 Cr 22:10; Sal 2:7; Is 56:5; Mt 5:9,16,45; Lc 20:36; Jn 1:12-13; 11:52; Hch 17:28-29; Ro 8:14-17,19,21; 9:26; 2 Co 6:18; Gl 3:26; 4:5-7; Ef 1:5; Flp 2:15; Heb 12:5-8; 1 Jn 3:1-2,10; 5:2; Ap 21:7; 2 Sam 7:14.}.

\par 
%\textsuperscript{(1957.3)}
\textsuperscript{181:2.13} Entonces Mateo dijo: «Pero, Maestro, ¿quién nos va a enviar y cómo sabremos adónde ir? ¿Nos mostrará Andrés el camino?» Y Jesús respondió: «No, Leví, Andrés ya no os dirigirá para proclamar el evangelio. Continuará por supuesto siendo vuestro amigo y consejero hasta el día en que llegue el nuevo instructor, y entonces el Espíritu de la Verdad os conducirá por ahí a cada uno de vosotros en el trabajo de expansión del reino. Se han producido en ti muchos cambios desde aquel día en la aduana en que empezaste a seguirme por primera vez\footnote{\textit{Muchos cambios desde que respondiste a la «llamada»}: Mt 9:9; Mc 2:14; Lc 5:27-28.}; pero deberán producirse muchos más antes de que puedas tener la visión de una fraternidad en la cual los gentiles se sentarán con los judíos en una asociación fraternal\footnote{\textit{Muchos más para la armonía}: Ro 10:12; 1 Co 12:13; Gl 3:28; Col 3:11.}. Pero continúa con tu impulso de atraer a tus hermanos judíos hasta que estés plenamente satisfecho, y luego dirígete con energía hacia los gentiles. Leví, puedes estar seguro de una cosa: Te has ganado la confianza y el afecto de tus hermanos; todos te aman». (Y los diez indicaron su conformidad a las palabras del Maestro.)

\par 
%\textsuperscript{(1958.1)}
\textsuperscript{181:2.14} «Leví, sé muchas cosas que tus hermanos ignoran sobre tus ansiedades, sacrificios y esfuerzos para mantener repleta la tesorería, y aunque el que llevaba la bolsa esté ausente, me alegra que el embajador publicano esté aquí en mi reunión de despedida con los mensajeros del reino. Ruego para que puedas discernir el significado de mi enseñanza con los ojos del espíritu. Cuando el nuevo instructor llegue a tu corazón\footnote{\textit{Seguid al nuevo instructor en vuestros corazones}: Ez 11:19; 18:31; 36:26-27; Jl 2:28-29; Lc 24:49; Jn 7:39; 14:16-18,23,26; 15:4,26; 16:7-8,13-14; 17:21-23; Hch 1:5,8a; 2:1-4,16-18; 2:33; 2 Co 13:5; Gl 2:20; 4:6; Ef 1:13; 4:30; 1 Jn 4:12-15.}, síguelo allá donde te conduzca y que tus hermanos vean ---e incluso el mundo entero--- lo que puede hacer el Padre por un detestado recaudador de impuestos que se ha atrevido a seguir al Hijo del Hombre y a creer en el evangelio del reino. Desde el principio, Leví, te he amado como he amado a estos otros galileos. Sabiendo pues muy bien que ni el Padre ni el Hijo hacen acepción de personas\footnote{\textit{El Padre y el Hijo no hacen acepción de personas}: 2 Cr 19:7; Job 34:19; Eclo 35:12; Mt 22:16; Mc 12:14; Lc 20:21; Hch 10:34; Ro 2:11; Gl 2:6; 3:28; Ef 6:9; Col 3:11.}, procura no hacer este tipo de distinciones entre los que se hagan creyentes en el evangelio gracias a tu ministerio. Así pues, Mateo, dedica toda tu futura vida de servicio a mostrar a todos los hombres que Dios no hace acepción de personas; que a los ojos de Dios y en la hermandad del reino, todos los hombres son iguales, todos los creyentes son hijos de Dios»\footnote{\textit{Los creyentes son hijos de Dios}: 1 Cr 22:10; Sal 2:7; Is 56:5; Mt 5:9,16,45; Lc 20:36; Jn 1:12-13; 11:52; Hch 17:28-29; Ro 8:14-17,19,21; 9:26; 2 Co 6:18; Gl 3:26; 4:5-7; Ef 1:5; 2:15; Heb 12:5-8; 1 Jn 3:1-2,10; 5:2; Ap 21:7; 2 Sam 7:14.}.

\par 
%\textsuperscript{(1958.2)}
\textsuperscript{181:2.15} Jesús se dirigió entonces a Santiago Zebedeo, que permaneció de pie en silencio mientras el Maestro le decía: «Santiago, cuando tú y tu hermano menor vinisteis a verme un día buscando preferencias en los honores del reino, os dije que esos honores sólo los podía otorgar el Padre, y os pregunté si erais capaces de beber mi copa\footnote{\textit{¿Serás capaz de beber mi copa?}: Mt 20:20-22; Mc 10:35-39a.}, y los dos me contestasteis que sí. Aunque entonces no hubierais sido capaces de hacerlo, y aunque ahora tampoco lo seáis, pronto estaréis preparados para ese servicio gracias a la experiencia que estáis a punto de atravesar. En aquella ocasión enfadaste a tus hermanos con tu conducta\footnote{\textit{Enfadaste a tus hermanos}: Mt 20:24; Mc 10:41.}. Si aún no te han perdonado del todo, lo harán cuando te vean beber mi copa\footnote{\textit{Santiago «beberá la copa»}: Hch 12:1-2.}. Que tu ministerio sea largo o breve, domina tu alma con paciencia\footnote{\textit{Domina tu alma con paciencia}: Lc 21:19.}. Cuando llegue el nuevo instructor, deja que te enseñe el equilibrio de la compasión y esa tolerancia comprensiva que nace de la confianza sublime en mí y de la sumisión perfecta a la voluntad del Padre. Dedica tu vida a demostrar que el afecto humano y la dignidad divina se pueden combinar en el discípulo que conoce a Dios y cree en el Hijo. Todos los que viven así revelarán el evangelio incluso por su manera de morir. Tú y tu hermano Juan seguiréis caminos diferentes, y es posible que uno de vosotros se siente conmigo en el reino eterno mucho antes que el otro\footnote{\textit{Diferentes duraciones de la vida}: Jn 21:22-23.}. Te ayudaría mucho si pudieras aprender que la verdadera sabiduría abarca la prudencia así como la valentía. Debes aprender que tu agresividad ha de ir acompañada de sagacidad. Llegarán esos momentos supremos en los que mis discípulos no dudarán en dar su vida por este evangelio, pero en todas las circunstancias ordinarias sería mucho mejor aplacar la ira de los incrédulos para que puedas seguir viviendo y continuar predicando la buena nueva. En la medida en que dependa de ti, vive mucho tiempo en la Tierra para que tu larga vida pueda ser fecunda en almas ganadas para el reino celestial».

\par 
%\textsuperscript{(1958.3)}
\textsuperscript{181:2.16} Cuando el Maestro hubo terminado de hablar a Santiago Zebedeo, dio la vuelta hasta el extremo de la mesa donde estaba sentado Andrés, miró a su fiel asistente a los ojos, y dijo: «Andrés, me has representado fielmente como jefe en funciones de los embajadores del reino celestial. Aunque a veces has dudado y en otras ocasiones has manifestado una timidez peligrosa, sin embargo siempre has sido sinceramente justo y eminentemente equitativo en tu trato con tus compañeros. Desde tu ordenación y la de tus hermanos como mensajeros del reino, habéis sido autónomos en todos los asuntos administrativos del grupo, salvo que te designé como jefe en funciones de estos escogidos. En ninguna otra cuestión temporal he actuado para dirigir o influir en tus decisiones. Y lo he hecho así a fin de asegurar la existencia de un jefe que dirija todas vuestras deliberaciones colectivas posteriores. En mi universo y en el universo de universos de mi Padre, nuestros hijos-hermanos son tratados como individuos en todas sus relaciones espirituales, pero en todas las relaciones colectivas, procuramos invariablemente que exista una persona determinada que dirija. Nuestro reino es un reino de orden, y cuando dos o más criaturas volitivas actúan en cooperación, siempre se prevé la autoridad de un jefe».

\par 
%\textsuperscript{(1959.1)}
\textsuperscript{181:2.17} «Y ahora Andrés, puesto que eres el jefe de tus hermanos en virtud de la autoridad que te he conferido, puesto que has servido así como mi representante personal y como estoy a punto de dejaros para ir hacia mi Padre, te libero de toda responsabilidad relacionada con estos asuntos temporales y administrativos. De ahora en adelante ya no tendrás ninguna jurisdicción sobre tus hermanos, excepto la que te has ganado como jefe espiritual y que tus hermanos reconocen por tanto libremente. A partir de este momento ya no puedes ejercer ninguna autoridad sobre tus hermanos, a menos que ellos te restituyan esa potestad mediante un acto legislativo preciso, después de que me haya ido hacia el Padre. Pero el hecho de liberarte de tus responsabilidades como jefe administrativo de este grupo no disminuye de ninguna manera tu responsabilidad moral de hacer todo lo que esté en tu poder para mantener juntos a tus hermanos, con mano firme y afectuosa, durante el período difícil que se avecina, esos días que transcurrirán entre mi partida de la carne y el envío del nuevo instructor que vivirá en vuestro corazón\footnote{\textit{El nuevo instructor en el corazón}: Ez 11:19; 18:31; Ez 36:26-27.} y que os conducirá finalmente a toda la verdad. Mientras me preparo para dejarte, quiero liberarte de toda la responsabilidad administrativa que tuvo su comienzo y su autoridad en mi presencia entre vosotros como uno de vosotros. De ahora en adelante, sólo ejerceré una autoridad espiritual sobre ti y entre vosotros».

\par 
%\textsuperscript{(1959.2)}
\textsuperscript{181:2.18} «Si tus hermanos desean conservarte como consejero, te ordeno que hagas todo lo posible, en todas las cuestiones temporales y espirituales, por promover la paz y la armonía entre los diversos grupos de creyentes sinceros en el evangelio. Dedica el resto de tu vida a fomentar los aspectos prácticos del amor fraternal entre tus hermanos. Sé amable con mis hermanos carnales cuando lleguen a creer plenamente en este evangelio; manifiesta una dedicación afectuosa e imparcial a los griegos en el oeste y a Abner en el este. Aunque estos apóstoles míos pronto se van a dispersar por todos los rincones de la Tierra para proclamar la buena nueva de la salvación mediante la filiación con Dios, debes mantenerlos unidos durante las horas difíciles que se avecinan, ese período de intensa prueba durante el cual deberéis aprender a creer en este evangelio sin mi presencia personal, mientras esperáis pacientemente la llegada del nuevo instructor, el Espíritu de la Verdad. Así pues, Andrés, aunque quizás no te corresponda realizar grandes obras a los ojos de los hombres, conténtate con ser el educador y el consejero de aquellos que las hacen. Continúa hasta el fin tu trabajo en la Tierra, y luego continuarás este ministerio en el reino eterno, porque ¿no te he dicho muchas veces que tengo otras ovejas que no son de este rebaño?»\footnote{\textit{Otras ovejas que no son de este rebaño}: Jn 10:16.}

\par 
%\textsuperscript{(1959.3)}
\textsuperscript{181:2.19} Jesús se dirigió entonces hacia los gemelos Alfeo, se colocó entre ellos, y dijo: «Queridos hijos míos, sois uno de los tres pares de hermanos que escogieron seguirme. Los seis habéis hecho bien en trabajar en paz con los de vuestra propia sangre, pero ninguno lo ha hecho mejor que vosotros. Se avecinan duros tiempos. Quizás no comprendáis todo lo que os sucederá a vosotros y a vuestros hermanos, pero no dudéis nunca de que un día fuisteis llamados para la obra del reino. Durante algún tiempo no habrá multitudes que dirigir, pero no os desaniméis; cuando el trabajo de vuestra vida haya terminado, os recibiré en el cielo, donde contaréis con gloria vuestra salvación a las huestes seráficas y a las multitudes de Hijos elevados de Dios. Dedicad vuestra vida a realzar las faenas vulgares. Mostrad a todos los hombres de la Tierra y a los ángeles del cielo cómo un hombre mortal puede volver con alegría y coraje a sus tareas de años atrás, después de haber sido llamado para trabajar durante una temporada en el servicio especial de Dios. Si vuestro trabajo en los asuntos exteriores del reino ha terminado por ahora, deberíais regresar a vuestros quehaceres anteriores con la iluminación nueva de la experiencia de la filiación con Dios, y con la elevada comprensión de que para aquel que conoce a Dios no existen trabajos vulgares ni faenas laicas. Para vosotros que habéis trabajado conmigo, todas las cosas se han vuelto sagradas, y toda labor terrestre se ha convertido también en un servicio para Dios Padre. Cuando escuchéis hablar de las actividades de vuestros antiguos asociados apostólicos, regocijaos con ellos y continuad vuestro trabajo diario como aquellos que esperan a Dios y sirven mientras esperan. Habéis sido mis apóstoles y lo seréis siempre, y me acordaré de vosotros en el reino venidero».

\par 
%\textsuperscript{(1960.1)}
\textsuperscript{181:2.20} Después, Jesús fue hacia Felipe, que se levantó para escuchar el siguiente mensaje de su Maestro: «Felipe, me has hecho muchas preguntas tontas, y he hecho todo lo posible por contestar a cada una de ellas, y ahora quisiera contestar a la última que ha surgido en tu mente sumamente honrada, pero poco espiritual. Todo el tiempo que he tardado en rodear la mesa hasta llegar a ti te has estado diciendo a ti mismo: `¿Qué voy a hacer si el Maestro se va y nos deja solos en el mundo?' ¡Oh, hombre de poca fe! Y sin embargo tienes casi tanta como muchos de tus hermanos. Has sido un buen administrador, Felipe. Sólo nos has fallado algunas veces\footnote{\textit{Los fallos de Felipe usados por Jesús}: Jn 6:5,7.}, y uno de esos fallos lo utilizamos para manifestar la gloria del Padre. Tu función como administrador está a punto de terminar. Pronto deberás dedicarte más plenamente al trabajo para el que fuiste llamado: la predicación de este evangelio del reino. Felipe, siempre has querido demostraciones, y muy pronto vas a ver grandes cosas. Habría sido mucho mejor que hubieras visto todo esto por la fe, pero como eras sincero incluso en tu visión material, vivirás para ver cómo se cumplen mis palabras. Después, cuando tengas la bendición de la visión espiritual, sal a hacer tu trabajo dedicando tu vida a la causa de guiar a la humanidad en la búsqueda de Dios, y a perseguir las realidades eternas con el ojo de la fe espiritual y no con los ojos de la mente material. Recuerda Felipe que tienes una gran misión en la Tierra, porque el mundo está lleno de gente que tiene la tendencia de ver la vida exactamente como tú. Tienes una gran tarea que hacer, y cuando haya sido terminada en la fe, vendrás hacia mí en mi reino, y tendré el gran placer de mostrarte lo que el ojo no ha visto, lo que el oído no ha escuchado y lo que la mente mortal no ha concebido\footnote{\textit{Lo que el ojo no ha visto, el oído no ha escuchado, etc.}: Job 28:7; Is 64:4; 1 Co 2:9.}. Mientras tanto, sé como un niño pequeño en el reino del espíritufootnote{\textit{Ser como los niños en el espíritu}: Mt 18:3; 19:13-14; Mc 9:36-37; 10:15; Lc 9:48; 18:17.} y permíteme, como espíritu del nuevo instructor, conducirte hacia adelante en el reino espiritual. De esta manera podré hacer por ti muchas cosas que no he podido realizar mientras vivía contigo como un mortal del reino. Y recuerda siempre, Felipe, que el que me ha visto ha visto al Padre»\footnote{\textit{Quien me ha visto, ha visto al Padre}: Jn 12:45; 14:9.}.

\par 
%\textsuperscript{(1960.2)}
\textsuperscript{181:2.21} Entonces el Maestro se acercó a Natanael. Cuando Natanael se levantó, Jesús le pidió que se sentara y sentándose a su lado, le dijo: «Natanael, has aprendido a vivir por encima de los prejuicios y a practicar una tolerancia creciente desde que te convertiste en mi apóstol. Pero tienes que aprender muchas más cosas. Has sido una bendición para tus compañeros, porque tu constante sinceridad siempre les ha servido de aviso. Cuando me haya ido, es posible que tu franqueza te impida llevarte bien con tus hermanos, tanto antiguos como nuevos. Deberías aprender que incluso la expresión de un pensamiento bueno debe ser modulada de acuerdo con el estado intelectual y el desarrollo espiritual del oyente. La sinceridad es extremadamente útil en el trabajo del reino cuando está unida a la discreción».

\par 
%\textsuperscript{(1961.1)}
\textsuperscript{181:2.22} «Si quisieras aprender a trabajar con tus hermanos, podrías realizar cosas más duraderas, pero si sales en busca de aquellos que piensan como tú, en ese caso dedica tu vida a probar que el discípulo que conoce a Dios puede convertirse en un constructor del reino, aunque esté solo en el mundo y completamente aislado de sus compañeros creyentes. Sé que serás fiel hasta el fin, y algún día te daré la bienvenida en el servicio más amplio de mi reino del cielo».

\par 
%\textsuperscript{(1961.2)}
\textsuperscript{181:2.23} Entonces habló Natanael, haciéndole a Jesús la pregunta siguiente: «He escuchado tu enseñanza desde que me llamaste por primera vez al servicio de este reino, pero honradamente no puedo comprender el significado completo de todo lo que nos dices. No sé qué es lo próximo que va a suceder, y creo que la mayoría de mis hermanos están igualmente perplejos, aunque dudan en confesar su confusión. ¿Puedes ayudarme?» Jesús puso su mano en el hombro de Natanael, y dijo: «Amigo mío, no es raro que te sientas perplejo al intentar captar el significado de mis enseñanzas espirituales, puesto que estás muy trabado por tus ideas preconcebidas que tienen su origen en la tradición judía, y muy confundido por tu tendencia persistente a interpretar mi evangelio de acuerdo con las enseñanzas de los escribas y de los fariseos».

\par 
%\textsuperscript{(1961.3)}
\textsuperscript{181:2.24} «Os he enseñado muchas cosas por medio de la palabra, y he vivido mi vida entre vosotros. He hecho todo lo que se puede hacer por iluminar vuestra mente y liberar vuestra alma, y lo que no habéis sido capaces de obtener de mis enseñanzas y de mi vida, ahora tenéis que prepararos para adquirirlo de la mano del maestro de todos los instructores: la experiencia real. En todas esas nuevas experiencias que ahora te esperan, iré delante de ti y el Espíritu de la Verdad estará contigo\footnote{\textit{El Espíritu de la Verdad estará contigo}: Ez 11:19; 18:31; 36:26-27; Jl 2:28-29; Lc 24:49; Jn 7:39; 14:16-18,23,26; 15:4,26; 16:7-8,13-14; 17:21-23; Hch 1:5,8a; 2:1-4,16-18; 2:33; 2 Co 13:5; Gl 2:20; 4:6; Ef 1:13; 4:30; 1 Jn 4:12-15.}. No temas; cuando el nuevo instructor haya llegado, lo que ahora no logras comprender te lo revelará durante el resto de tu vida en la Tierra y a lo largo de toda tu formación durante las eras eternas».

\par 
%\textsuperscript{(1961.4)}
\textsuperscript{181:2.25} Luego el Maestro se volvió hacia todos ellos, y dijo: «No os desaniméis si no lográis captar el pleno significado del evangelio. Sólo sois seres finitos, hombres mortales, y lo que os he enseñado es infinito, divino y eterno. Sed pacientes y tened buen ánimo, porque tenéis ante vosotros las eras eternas para continuar haciendo realidad progresivamente la experiencia de volveros perfectos\footnote{\textit{Perfecionarse espiritualmente}: Gn 17:1; 1 Re 8:61; Lv 19:2; Dt 18:13; Mt 5:48; 2 Co 13:11; Stg 1:4; 1 P 1:16.}, como vuestro Padre en el Paraíso es perfecto».

\par 
%\textsuperscript{(1961.5)}
\textsuperscript{181:2.26} Entonces Jesús se dirigió hacia Tomás, que se puso de pie para escucharle decir: «Tomás, a menudo te ha faltado fe; sin embargo, cuando has tenido tus períodos de duda, nunca te ha faltado el coraje. Sé muy bien que los falsos profetas y los educadores impostores no te engañarán. Después de mi partida, tus hermanos apreciarán mucho más tu manera crítica de ver las nuevas enseñanzas. Cuando todos estéis dispersos hasta los confines de la Tierra en los tiempos venideros, recuerda que sigues siendo mi embajador. Dedica tu vida a la gran tarea de mostrar que la mente material crítica del hombre puede triunfar sobre la inercia de la duda intelectual cuando se enfrenta a la demostración de la manifestación de la verdad viviente, tal como ésta opera en la experiencia de los hombres y mujeres nacidos del espíritu, que producen en sus vidas los frutos del espíritu\footnote{\textit{Los frutos del espíritu}: Gl 5:22-23; Ef 5:9.}, y que se aman los unos a los otros como yo os he amado\footnote{\textit{Amaos los unos a los otros como yo os he amado}: Jn 13:34-35; 15:12.}. Tomás, me alegro de que te unieras a nosotros, y sé que después de un corto período de perplejidad, continuarás al servicio del reino. Tus dudas han confundido a tus hermanos, pero nunca me han preocupado. Tengo confianza en ti, y te precederé hasta los rincones más alejados de la Tierra»\footnote{\textit{Los rincones más alejados de la Tierra}: Hch 1:8.}.

\par 
%\textsuperscript{(1962.1)}
\textsuperscript{181:2.27} Luego el Maestro fue hacia Simón Pedro, el cual se puso de pie mientras Jesús le dirigía la palabra: «Pedro, sé que me amas, y que dedicarás tu vida a proclamar públicamente este evangelio del reino a los judíos y a los gentiles, pero me apena que tus años de asociación tan estrecha conmigo no hayan servido más para ayudarte a reflexionar antes de hablar. ¿Por qué experiencias tendrás que pasar para aprender a contener tu lengua? ¡Cuántas dificultades nos has causado con tus palabras irreflexivas, con tu presuntuosa confianza en ti mismo! Y estás destinado a crearte muchos más problemas si no dominas esta flaqueza. Sabes que tus hermanos te aman a pesar de esta debilidad, y también deberías comprender que este defecto no disminuye de ninguna manera mi afecto por ti, pero sí disminuye tu utilidad y no deja de crearte problemas. Pero la experiencia por la que vas a pasar esta misma noche será sin duda de gran ayuda para ti\footnote{\textit{Última advertencia de una lección}: Mt 26:31-32; Mc 14:27-28.}. Y lo que ahora voy a decirte, Simón Pedro, lo digo igualmente a todos tus hermanos aquí reunidos: Esta noche, todos vais a correr el gran peligro de tropezar por mi causa. Sabéis que está escrito: `Golpearán al pastor y las ovejas serán dispersadas.'\footnote{\textit{Se dispersarán las ovejas}: Zac 13:7.} Cuando ya no esté presente, existirá el gran peligro de que algunos de vosotros sucumban a las dudas y tropiecen a causa de lo que me suceda a mí. Pero os prometo ahora que regresaré por un corto período de tiempo, y que entonces os precederé en Galilea»\footnote{\textit{Iré a Galilea}: Mt 28:7,10; Mc 16:7; Jn 21:1.}.

\par 
%\textsuperscript{(1962.2)}
\textsuperscript{181:2.28} Entonces Pedro, poniendo su mano en el hombro de Jesús, dijo: «No importa que todos mis hermanos sucumban a las dudas por tu causa; prometo que no tropezaré por nada de lo que puedas hacer. Iré contigo y, si es preciso, moriré por ti»\footnote{\textit{La presunción de Pedro}: Mt 26:33,35; Mc 14:29,31; Lc 22:33; Jn 13:37.}.

\par 
%\textsuperscript{(1962.3)}
\textsuperscript{181:2.29} Mientras Pedro permanecía allí delante de su Maestro, temblando de intensa emoción y rebosante de amor sincero por él, Jesús miró directamente a sus ojos humedecidos mientras le decía: «Pedro, en verdad, en verdad te digo que el gallo no cantará esta noche hasta que me hayas negado tres o cuatro veces\footnote{\textit{Predicción de la negación de Pedro}: Mt 26:34; Mc 14:30; Lc 22:34; Jn 13:38.}. Y así, lo que no has logrado aprender mediante tu asociación pacífica conmigo, lo aprenderás a través de muchas dificultades y de grandes tristezas. Después de que hayas aprendido realmente esta lección indispensable, deberías fortalecer a tus hermanos\footnote{\textit{Fortalecer a tus hermanos}: Lc 22:32b.} y continuar viviendo una vida dedicada a la predicación de este evangelio, aunque puedas terminar en la cárcel y quizás sigas mis pasos, pagando el precio supremo del servicio amoroso en la construcción del reino del Padre».

\par 
%\textsuperscript{(1962.4)}
\textsuperscript{181:2.30} «Pero recuerda mi promesa: Cuando haya resucitado\footnote{\textit{Promesa: resucitaré al tercer día}: Mt 16:21; 17:23a; 20:19; 27:63; Mc 8:31; 9:31; 10:34; Lc 9:22; 18:33; 24:7,36; Jn 20:9.}, permaneceré algún tiempo\footnote{\textit{Me demoraré un tiempo}: Jn 16:16-22.} con vosotros antes de ir hacia el Padre. Esta misma noche le suplicaré al Padre que fortalezca\footnote{\textit{Suplicaré que os fortalezca}: Lc 22:32a.} a cada uno de vosotros para la prueba que muy pronto tendréis que atravesar. Os amo a todos con el mismo amor que el Padre me ama, y por eso, de ahora en adelante, deberíais amaros los unos a los otros como yo os he amado»\footnote{\textit{Amaros como yo os he amado}: Jn 13:34-35; 15:12,17.}.

\par 
%\textsuperscript{(1962.5)}
\textsuperscript{181:2.31} Luego, después de haber cantado un himno, partieron hacia el campamento del Monte de los Olivos\footnote{\textit{Canto del himno y salida}: Mc 14:26.}.