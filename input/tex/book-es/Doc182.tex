\chapter{Documento 182. En Getsemaní}
\par 
%\textsuperscript{(1963.1)}
\textsuperscript{182:0.1} ERAN aproximadamente las diez de este jueves por la noche cuando Jesús llevó de regreso a los once apóstoles desde la casa de Elías y María Marcos hasta el campamento de Getsemaní. Desde el día que estuvo con el Maestro en las colinas, Juan Marcos se había ocupado de vigilar constantemente a Jesús. Como tenía necesidad de dormir, Juan había descansado varias horas mientras el Maestro estaba con sus apóstoles en la sala de arriba, pero al escuchar que bajaban las escaleras, se levantó y se puso rápidamente un manto de lino; luego los siguió a través de la ciudad, cruzó el arroyo Cedrón y continuó hasta su campamento privado que lindaba con el parque de Getsemaní. A lo largo de esta noche y del día siguiente, Juan Marcos permaneció tan cerca del Maestro que lo presenció todo y escuchó muchas cosas que dijo el Maestro desde este instante hasta el momento de la crucifixión.

\par 
%\textsuperscript{(1963.2)}
\textsuperscript{182:0.2} Mientras Jesús y los once regresaban al campamento, los apóstoles empezaron a preguntarse por el significado de la prolongada ausencia de Judas; hablaron entre sí acerca de la predicción del Maestro de que uno de ellos lo traicionaría, y sospecharon por primera vez que las cosas no iban bien con Judas Iscariote. Pero no se dedicaron abiertamente a hacer comentarios sobre Judas hasta que llegaron al campamento y observaron que no estaba allí esperándolos para recibirlos. Cuando todos acosaron a Andrés para saber qué le había pasado a Judas, su jefe se limitó a comentar: «No sé dónde está Judas, pero me temo que nos ha abandonado».

\section*{1. La última oración en grupo}
\par 
%\textsuperscript{(1963.3)}
\textsuperscript{182:1.1} Poco después de llegar al campamento, Jesús les dijo: «Amigos y hermanos míos, me queda muy poco tiempo que estar con vosotros, y deseo que nos aislemos mientras le rogamos a nuestro Padre que está en los cielos que nos dé fuerzas para sostenernos en esta hora y de aquí en adelante en todo el trabajo que tenemos que hacer en su nombre».

\par 
%\textsuperscript{(1963.4)}
\textsuperscript{182:1.2} Después de haber hablado así, Jesús los llevó un poco más arriba por el Olivete hasta una gran roca plana desde donde se veía todo Jerusalén, y les pidió que se arrodillaran en círculo a su alrededor como lo habían hecho el día de su ordenación; luego, mientras permanecía allí en medio de ellos, glorificado en la suave luz de la Luna, levantó los ojos al cielo y oró\footnote{\textit{Levantó sus ojos y oró}: Jn 17:1a.}:

\par 
%\textsuperscript{(1963.5)}
\textsuperscript{182:1.3} «Padre, mi hora ha llegado; glorifica ahora a tu Hijo para que el Hijo pueda glorificarte. Sé que me has dado plena autoridad sobre todas las criaturas vivientes de mi reino, y daré la vida eterna a todos los que se vuelvan hijos de Dios por la fe. Y la vida eterna consiste en que mis criaturas te conozcan como el único verdadero Dios y Padre de todos, y que crean en aquel que has enviado a este mundo. Padre, te he exaltado en la Tierra y he realizado la obra que me encargaste. Casi he terminado mi donación a los hijos de nuestra propia creación; sólo me queda abandonar mi vida en la carne. Ahora, oh Padre mío, glorifícame con la gloria que tenía contigo antes de que existiera este mundo y recíbeme una vez más a tu diestra»\footnote{\textit{Oración: parte 1}: Jn 17:1b-5.}.

\par 
%\textsuperscript{(1964.1)}
\textsuperscript{182:1.4} «Te he manifestado a los hombres que escogiste en el mundo para dármelos. Son tuyos ---como toda vida está en tus manos--- tú me los diste y yo he vivido entre ellos enseñándoles el camino de la vida, y ellos han creído. Estos hombres están aprendiendo que todo lo que tengo procede de ti, y que la vida que vivo en la carne es para hacer que los mundos conozcan a mi Padre. La verdad que me has dado se la he revelado a ellos. Estos amigos y embajadores míos han querido recibir sinceramente tu palabra. Les he dicho que he salido de ti, que tú me has enviado a este mundo, y que estoy a punto de volver a ti. Padre, ruego de hecho por estos hombres escogidos. Y ruego por ellos, no como rogaría por el mundo, sino como por aquellos a quienes he elegido en el mundo para que me representen en el mundo después de que haya regresado a tu tarea, al igual que te he representado en este mundo durante mi estancia en la carne. Estos hombres son míos; tú me los has dado; pero todas las cosas que son mías son siempre tuyas, y has hecho que todo lo que era tuyo ahora sea mío. Has sido exaltado en mí, y ahora ruego para que yo pueda ser honrado en estos hombres. No puedo estar más tiempo en este mundo; estoy a punto de volver a la tarea que me has encargado. Tengo que dejar atrás a estos hombres para que nos representen y representen a nuestro reino entre los hombres. Padre, mantén fieles a estos hombres mientras me preparo para abandonar mi vida en la carne. Ayuda a estos amigos míos para que sean uno en espíritu, como nosotros también somos uno. Mientras podía estar con ellos, podía velar por ellos y guiarlos, pero ahora estoy a punto de irme. Permanece cerca de ellos, Padre, hasta que podamos enviar al nuevo instructor para que los consuele y los fortalezca»\footnote{\textit{Oración: parte 2}: Jn 17:6-12a.}.

\par 
%\textsuperscript{(1964.2)}
\textsuperscript{182:1.5} «Me diste doce hombres, y los he conservado a todos salvo a uno, el hijo de la venganza, que no ha querido seguir asociado con nosotros. Estos hombres son débiles y frágiles, pero sé que podemos confiar en ellos; los he puesto a prueba; me aman al igual que te veneran a ti\footnote{\textit{Oración: parte 3}: Jn 17:12c.}. Aunque deberán sufrir mucho por mí, deseo que también estén llenos de alegría ante la seguridad de la filiación en el reino celestial. He dado a estos hombres tu palabra y les he enseñado la verdad. El mundo puede odiarlos como me ha odiado a mí, pero no pido que los saques del mundo, sino que los protejas del mal que hay en el mundo. Santifícalos en la verdad; tu palabra es la verdad. Del mismo modo que me enviaste a este mundo, yo estoy a punto de enviar a estos hombres al mundo. Por el bien de ellos, he vivido entre los hombres y he consagrado mi vida a tu servicio, a fin de poder inspirarlos para que se purifiquen por medio de la verdad que les he enseñado y el amor que les he revelado. Sé muy bien, Padre mío, que no necesito pedirte que veles por estos hermanos después de que me haya ido; sé que los amas como yo, pero hago esto para que puedan darse cuenta mejor de que el Padre ama a los hombres mortales como el Hijo los ama»\footnote{\textit{Oración: parte 4}: Jn 17:13b-19.}.

\par 
%\textsuperscript{(1964.3)}
\textsuperscript{182:1.6} «Y ahora, Padre mío, quisiera rogar no solamente por estos once hombres, sino también por todos los demás que ahora creen en el evangelio del reino, o que puedan creer más adelante gracias a la palabra del ministerio futuro de mis apóstoles. Quiero que todos sean uno solo, como tú y yo somos uno. Tú estás en mí y yo estoy en ti, y deseo que estos creyentes estén igualmente en nosotros; que nuestros dos espíritus residan en ellos. Si mis hijos son uno solo como nosotros somos uno, y si se aman los unos a los otros como yo los he amado\footnote{\textit{Amaos los unos a los otros como yo os he amado}: Jn 13:34-35; 15:12,17.}, entonces todos los hombres creerán que he salido de ti y estarán dispuestos a recibir la revelación que he efectuado de la verdad y la gloria. He revelado a estos creyentes la gloria que tú me has dado. Así como tú has vivido conmigo en espíritu, yo he vivido con ellos en la carne. Así como tú has sido uno conmigo, yo he sido uno con ellos, y el nuevo instructor será siempre uno con ellos y en ellos. He hecho todo esto para que mis hermanos en la carne puedan saber que el Padre los ama como el Hijo los ama, y que tú los amas como me amas a mí. Padre, trabaja conmigo para salvar a estos creyentes a fin de que dentro de poco puedan estar conmigo en la gloria, y luego continúen hasta unirse contigo en el abrazo del Paraíso. A los que sirven conmigo en la humillación, quisiera tenerlos conmigo en la gloria para que puedan ver todo lo que has puesto entre mis manos como cosecha eterna de la siembra del tiempo en la similitud de la carne mortal. Anhelo mostrar a mis hermanos terrestres la gloria que tenía contigo antes de la fundación de este mundo. Este mundo sabe muy poco de ti, Padre justo, pero yo te conozco y te he hecho conocer a estos creyentes, y ellos harán conocer tu nombre a otras generaciones. Y ahora les prometo que estarás con ellos en el mundo al igual que has estado conmigo ---que así sea»\footnote{\textit{Oración: parte 5}: Jn 17:20-26.}.

\par 
%\textsuperscript{(1965.1)}
\textsuperscript{182:1.7} Los once permanecieron arrodillados en círculo alrededor de Jesús durante varios minutos, antes de levantarse y regresar en silencio al campamento cercano.

\par 
%\textsuperscript{(1965.2)}
\textsuperscript{182:1.8} Jesús oró por la \textit{unidad} entre sus seguidores, pero no deseaba la uniformidad. El pecado crea un nivel muerto de inercia maligna, pero la rectitud alimenta el espíritu creativo de la experiencia individual en las realidades vivientes de la verdad eterna y en la comunión progresiva de los espíritus divinos del Padre y del Hijo. En la comunión espiritual de un hijo creyente con el Padre divino, nunca puede haber una finalidad doctrinal ni una superioridad sectaria de conciencia de grupo.

\par 
%\textsuperscript{(1965.3)}
\textsuperscript{182:1.9} En el transcurso de esta oración final con sus apóstoles, el Maestro aludió al hecho de que había manifestado al mundo el \textit{nombre} del Padre. Y esto es realmente lo que hizo al revelar a Dios mediante su vida perfeccionada en la carne. El Padre que está en los cielos había intentado revelarse a Moisés, pero no pudo ir más allá de hacer que se dijera: «YO SOY»\footnote{\textit{YO SOY}: Ex 3:6.}. Y cuando se le instó a que revelara más cosas de sí mismo, sólo se reveló: «YO SOY el que SOY»\footnote{\textit{YO SOY el que SOY}: Ex 3:14.}. Pero cuando Jesús hubo terminado su vida terrenal, el nombre del Padre se había revelado de tal manera que el Maestro, que era el Padre encarnado, podía decir en verdad:

\par 
%\textsuperscript{(1965.4)}
\textsuperscript{182:1.10} Yo soy el pan de la vida.\footnote{\textit{Yo soy el pan de la vida}: Jn 6:35,48,51.}

\par 
%\textsuperscript{(1965.5)}
\textsuperscript{182:1.11} Yo soy el agua viva.\footnote{\textit{Yo soy el agua viva}: Jn 4:10-14; Jn 7:37-38.}

\par 
%\textsuperscript{(1965.6)}
\textsuperscript{182:1.12} Yo soy la luz del mundo.\footnote{\textit{Yo soy la luz del mundo}: Jn 1:9; Jn 8:12; 9:5; 12:46.}

\par 
%\textsuperscript{(1965.7)}
\textsuperscript{182:1.13} Yo soy el deseo de todos los tiempos.\footnote{\textit{Yo soy el deseo de todos los tiempos}: Hag 2:7.}

\par 
%\textsuperscript{(1965.8)}
\textsuperscript{182:1.14} Yo soy la puerta abierta a la salvación eterna.\footnote{\textit{Yo soy la puerta abierta a la salvación eterna}: Jn 10:1-3,7.9.}

\par 
%\textsuperscript{(1965.9)}
\textsuperscript{182:1.15} Yo soy la realidad de la vida sin fin.\footnote{\textit{Yo soy la realidad de la vida sin fin}: Jn 3:16-17,36; 6:27,40; 10:28-29; 17:2-3.}

\par 
%\textsuperscript{(1965.10)}
\textsuperscript{182:1.16} Yo soy el buen pastor.\footnote{\textit{Yo soy el buen pastor}: Jn 10:11-16; Heb 13:20.}

\par 
%\textsuperscript{(1965.11)}
\textsuperscript{182:1.17} Yo soy el sendero de la perfección infinita.\footnote{\textit{Yo soy el sendero de la perfección infinita}: Heb 7:9-17.}

\par 
%\textsuperscript{(1965.12)}
\textsuperscript{182:1.18} Yo soy la resurrección y la vida.\footnote{\textit{Yo soy la resurrección y la vida}: Jn 11:25.}

\par 
%\textsuperscript{(1965.13)}
\textsuperscript{182:1.19} Yo soy el secreto de la supervivencia eterna.\footnote{\textit{Yo soy el secreto de la supervivencia eterna}: Mc 10:29-30; Jn 6:68; Ro 6:23.}

\par 
%\textsuperscript{(1965.14)}
\textsuperscript{182:1.20} Yo soy el camino, la verdad y la vida.\footnote{\textit{Yo soy el camino, la verdad y la vida}: Jn 14:6.}

\par 
%\textsuperscript{(1965.15)}
\textsuperscript{182:1.21} Yo soy el Padre infinito de mis hijos finitos.\footnote{\textit{Yo soy el Padre infinito de mis hijos finitos}: Jn 3:35-36; 5:19-24; 10:30; 17:20-21.}

\par 
%\textsuperscript{(1965.16)}
\textsuperscript{182:1.22} Yo soy la verdadera vid; vosotros sois los sarmientos.\footnote{\textit{Yo soy la verdadera vid; vosotros sois los sarmientos}: Jn 15:1,5.}

\par 
%\textsuperscript{(1965.17)}
\textsuperscript{182:1.23} Yo soy la esperanza de todos los que conocen la verdad viviente.\footnote{\textit{Yo soy la esperanza de los que conocen la verdad viviente}: Col 1:27; 1 Ts 2:19; 1 Ti 1:1; Tit 2:13.}

\par 
%\textsuperscript{(1965.18)}
\textsuperscript{182:1.24} Yo soy el puente viviente que va de un mundo a otro.

\par 
%\textsuperscript{(1965.19)}
\textsuperscript{182:1.25} Yo soy el enlace viviente entre el tiempo y la eternidad.

\par 
%\textsuperscript{(1965.20)}
\textsuperscript{182:1.26} Jesús amplió así la revelación viviente del nombre de Dios para todas las generaciones. De la misma manera que el amor divino revela la naturaleza de Dios, la verdad eterna revela su nombre en unas proporciones siempre crecientes.

\section*{2. Las últimas horas antes de la traición}
\par 
%\textsuperscript{(1966.1)}
\textsuperscript{182:2.1} Los apóstoles se quedaron profundamente anonadados cuando regresaron a su campamento y comprobaron que Judas no estaba allí. Mientras los once emprendían una viva discusión sobre el asunto de su compañero apóstol traidor, David Zebedeo y Juan Marcos llevaron a Jesús a un lado y le revelaron que habían estado observando a Judas durante varios días, y que sabían que tenía la intención de traicionarlo poniéndolo en manos de sus enemigos. Jesús los escuchó pero se limitó a decir: «Amigos míos, al Hijo del Hombre no puede sucederle nada a menos que lo quiera el Padre que está en los cielos. Que no se inquiete vuestro corazón; todas las cosas concurrirán para la gloria de Dios y la salvación de los hombres».

\par 
%\textsuperscript{(1966.2)}
\textsuperscript{182:2.2} La actitud jovial de Jesús iba decayendo. A medida que pasaba el tiempo se volvía cada vez más serio e incluso triste. Los apóstoles, que estaban muy agitados, eran reacios a regresar a sus tiendas aunque se lo pidiera el mismo Maestro. Al volver de su conversación con David y Juan, Jesús dirigió sus últimas palabras a los once, diciendo: «Amigos míos, id a descansar. Preparaos para el trabajo de mañana. Recordad que todos deberíamos someternos a la voluntad del Padre que está en los cielos. Os dejo mi paz». Después de hablar así, les indicó que regresaran a sus tiendas, pero mientras se iban, llamó a Pedro, Santiago y Juan\footnote{\textit{Tres discípulos escogidos}: Mt 26:37; Mc 14:33.}, diciendo: «Deseo que permanezcáis un rato conmigo».

\par 
%\textsuperscript{(1966.3)}
\textsuperscript{182:2.3} Los apóstoles se durmieron únicamente porque estaban literalmente agotados. Habían estado escasos de sueño desde que llegaron a Jerusalén. Antes de ir a sus diferentes tiendas para dormir, Simón Celotes los condujo a todos a su tienda, donde estaban guardadas las espadas y otras armas, y entregó a cada uno su equipo de combate. Todos recibieron estas armas y se las ciñeron allí mismo, excepto Natanael. Al rehusar el arma, Natanael dijo: «Hermanos míos, el Maestro nos ha dicho muchas veces que su reino no es de este mundo, y que sus discípulos no deberían luchar con la espada para establecerlo. Yo creo en esto, y no pienso que el Maestro necesite que utilicemos la espada para defenderlo. Todos hemos visto su enorme poder y sabemos que podría defenderse de sus enemigos si lo deseara. Si no quiere resistirse a sus enemigos, debe ser porque esa conducta representa su intento por realizar la voluntad de su Padre. Rezaré, pero no empuñaré la espada». Cuando Andrés escuchó el discurso de Natanael, devolvió su espada a Simón Celotes. Así pues, nueve de ellos estaban armados cuando se separaron para irse a dormir.

\par 
%\textsuperscript{(1966.4)}
\textsuperscript{182:2.4} El resentimiento que tenían porque Judas era un traidor eclipsó por el momento todo lo demás en la mente de los apóstoles. El comentario del Maestro alusivo a Judas\footnote{\textit{La oración de Jesús y la mención a Judas}: Jn 17:12b.}, expresado en el transcurso de la última oración, había abierto sus ojos al hecho de que los había abandonado.

\par 
%\textsuperscript{(1966.5)}
\textsuperscript{182:2.5} Después de que los ocho apóstoles se hubieron retirado finalmente a sus tiendas, y mientras Pedro, Santiago y Juan estaban esperando recibir las órdenes del Maestro, Jesús le dijo a David Zebedeo: «Envíame a tu mensajero más rápido y fiable». Cuando David trajo ante el Maestro a un tal Jacobo, en otro tiempo corredor al servicio de los mensajes nocturnos entre Jerusalén y Betsaida, Jesús se dirigió a él y le dijo: «Ve a toda prisa hasta Abner en Filadelfia y dile: `El Maestro te envía sus saludos de paz y dice que ha llegado la hora en que será entregado en manos de sus enemigos, que le darán muerte, pero que resucitará de entre los muertos y pronto aparecerá ante ti antes de ir hacia el Padre, y que entonces te dará unas directrices hasta el momento en que el nuevo instructor venga a vivir en vuestro corazón.'» Cuando Jacobo hubo repetido este mensaje a la satisfacción del Maestro, Jesús lo envió a su misión, diciendo: «No temas por lo que alguien pueda hacerte, Jacobo, porque esta noche un mensajero invisible correrá a tu lado».

\par 
%\textsuperscript{(1967.1)}
\textsuperscript{182:2.6} Luego Jesús se volvió hacia el jefe de los visitantes griegos que estaban acampados con ellos y le dijo: «Hermano mío, no te inquietes por lo que está a punto de suceder, puesto que te he avisado de antemano. El Hijo del Hombre será ejecutado a instigación de sus enemigos, los jefes de los sacerdotes y los dirigentes de los judíos, pero resucitaré para estar con vosotros un poco de tiempo antes de ir hacia el Padre. Cuando hayas visto que sucede todo esto, glorifica a Dios y fortalece a tus hermanos».

\par 
%\textsuperscript{(1967.2)}
\textsuperscript{182:2.7} En circunstancias normales, los apóstoles hubieran dado personalmente las buenas noches al Maestro, pero esta noche estaban tan preocupados por la conciencia repentina de la deserción de Judas y tan aturdidos por la naturaleza insólita de la oración de despedida del Maestro, que escucharon su saludo de adiós y se alejaron en silencio.

\par 
%\textsuperscript{(1967.3)}
\textsuperscript{182:2.8} Aquella noche, cuando Andrés se alejaba de su lado, Jesús le dijo lo siguiente: «Andrés, haz lo que puedas para mantener juntos a tus hermanos hasta que yo regrese con vosotros después de haber bebido esta copa. Fortalece a tus hermanos, puesto que ya te lo he dicho todo. Que la paz sea contigo».

\par 
%\textsuperscript{(1967.4)}
\textsuperscript{182:2.9} Ninguno de los apóstoles esperaba que sucediera nada fuera de lo común aquella noche, puesto que ya era muy tarde. Trataron de dormirse para poder levantarse temprano por la mañana y estar preparados para lo peor. Pensaban que los jefes de los sacerdotes intentarían capturar a su Maestro por la mañana temprano, porque nunca se hacía ningún trabajo secular después del mediodía del día de la preparación de la Pascua. Sólo David Zebedeo y Juan Marcos comprendieron que los enemigos de Jesús vendrían con Judas aquella misma noche.

\par 
%\textsuperscript{(1967.5)}
\textsuperscript{182:2.10} David había acordado permanecer de guardia aquella noche en el sendero más elevado que conducía a la carretera de Betania a Jerusalén, mientras que Juan Marcos debía vigilar la carretera que subía del Cedrón a Getsemaní. Antes de que David se dirigiera a su tarea autoimpuesta de centinela en un puesto avanzado, se despidió de Jesús diciendo: «Maestro, he tenido la gran alegría de servir contigo. Mis hermanos son tus apóstoles, pero yo he disfrutado haciendo las cosas menores tal como debían hacerse, y te echaré de menos con todo mi corazón cuando te hayas ido». Jesús le dijo entonces a David: «David, hijo mío, los demás han hecho lo que se les ordenaba que hicieran, pero tú has hecho este servicio por tu propia voluntad, y he sido consciente de tu dedicación. Tú también servirás algún día conmigo en el reino eterno».

\par 
%\textsuperscript{(1967.6)}
\textsuperscript{182:2.11} Entonces, mientras se preparaba para ir a vigilar en el sendero de arriba, David le dijo a Jesús: «Sabes, Maestro, he enviado a buscar a tu familia, y un mensajero me ha dado la noticia de que esta noche están en Jericó. Mañana por la mañana temprano estarán aquí, pues sería peligroso para ellos subir de noche por este maldito camino». Bajando la mirada hacia David, Jesús dijo solamente: «Que así sea, David».

\par 
%\textsuperscript{(1967.7)}
\textsuperscript{182:2.12} Cuando David se marchó hacia la parte alta del Olivete, Juan Marcos empezó a vigilar cerca de la carretera que descendía a lo largo del arroyo hacia Jerusalén. Juan habría permanecido en su puesto si no hubiera sido por su gran deseo de estar cerca de Jesús y de saber qué estaba sucediendo. Poco después de que David lo dejara, y al observar que Jesús se retiraba con Pedro, Santiago y Juan hacia una hondonada cercana, Juan Marcos se sintió tan dominado por una mezcla de devoción y de curiosidad, que abandonó su puesto de centinela y los siguió, ocultándose entre los arbustos. Desde allí observó y escuchó todo lo que sucedió durante estos últimos momentos en el jardín, poco antes de que Judas y los guardias armados aparecieran para arrestar a Jesús.

\par 
%\textsuperscript{(1968.1)}
\textsuperscript{182:2.13} Mientras todo esto se desarrollaba en el campamento del Maestro, Judas Iscariote conversaba con el capitán de los guardias del templo, el cual había reunido a sus hombres antes de ponerse en camino, bajo la dirección del traidor, para arrestar a Jesús.

\section*{3. A solas en Getsemaní}
\par 
%\textsuperscript{(1968.2)}
\textsuperscript{182:3.1} Cuando todo estuvo silencioso y tranquilo en el campamento, Jesús se llevó a Pedro, Santiago y Juan, y subieron un corto trecho hasta una hondonada cercana donde había ido anteriormente con frecuencia para orar y comulgar. Los tres apóstoles no podían dejar de reconocer que el Maestro estaba dolorosamente abrumado. Nunca antes lo habían observado tan triste y agobiado. Cuando llegaron al lugar de sus devociones, pidió a los tres que se sentaran y velaran con él mientras se alejaba a casi un tiro de piedra para orar. Cuando se hubo postrado en el suelo, oró: «Padre mío, he venido a este mundo para hacer tu voluntad, y la he hecho. Sé que ha llegado la hora de abandonar esta vida en la carne, y no rehuyo hacerlo, pero quisiera saber si es tu voluntad que yo beba esta copa. Envíame la seguridad de que te complaceré en mi muerte tal como lo he hecho en mi vida»\footnote{\textit{La primera oración de Getsemaní}: Mt 26:36-39; Mc 14:32-36; Lc 22:41-42.}.

\par 
%\textsuperscript{(1968.3)}
\textsuperscript{182:3.2} El Maestro permaneció unos momentos en actitud de oración, y luego se acercó a los tres apóstoles; los encontró profundamente dormidos, pues tenían los párpados pesados y no podían permanecer despiertos\footnote{\textit{Primera vez que los apóstoles duermen}: Mt 26:40; Mc 14:37.}. Cuando Jesús los despertó, dijo: «¡Cómo! ¿No podéis velar conmigo ni siquiera una hora? ¿No podéis ver que mi alma está extremadamente afligida, afligida de muerte, y que anhelo vuestra compañía?» Cuando los tres se despertaron de su sueño, el Maestro se alejó de nuevo a solas y, cayendo al suelo, oró otra vez: «Padre, sé que es posible evitar esta copa ---todas las cosas son posibles para ti--- pero he venido para hacer tu voluntad, y aunque esta copa sea amarga, la beberé si es tu voluntad»\footnote{\textit{La segunda oración en Getsemaní}: Mt 26:42; Mc 14:39.}. Después de haber orado así, un ángel poderoso descendió a su lado, le habló, lo tocó y lo fortaleció\footnote{\textit{Un ángel poderoso lo confortó}: Lc 22:43.}.

\par 
%\textsuperscript{(1968.4)}
\textsuperscript{182:3.3} Cuando Jesús regresó para hablar con los tres apóstoles, los encontró de nuevo profundamente dormidos\footnote{\textit{La segunda vez que los apóstoles duermen}: Mt 26:43; Mc 14:40.}. Los despertó diciendo: «En esta hora necesito que veléis y oréis conmigo ---necesitáis orar aún más para no caer en la tentación--- ¿por qué os dormís cuando os dejo?»\footnote{\textit{Los apóstoles duermen}: Lc 22:45-46.}

\par 
%\textsuperscript{(1968.5)}
\textsuperscript{182:3.4} Entonces, el Maestro se retiró por tercera vez para orar: «Padre, ves a mis apóstoles dormidos; ten misericordia de ellos. En verdad, el espíritu está dispuesto, pero la carne es débil\footnote{\textit{El espíritu está dispuesto, pero la carne es débil}: Mt 26:41b.}. Y ahora, oh Padre, si esta copa no puede ser apartada, entonces la beberé. Que no se haga mi voluntad, sino la tuya». Cuando hubo terminado de orar, permaneció unos momentos postrado en el suelo. Cuando se levantó y regresó donde estaban sus apóstoles, los encontró dormidos una vez más\footnote{\textit{Tercera oración y tercer sueño de los apóstoles}: Mt 26:44-46; Mc 14:41-42; Lc 22:44-46.}. Los observó y, con un gesto de piedad, dijo tiernamente: «Dormid ahora y descansad; el momento de la decisión ha pasado. Ha llegado la hora en que el Hijo del Hombre será traicionado y entregado a sus enemigos». Mientras se inclinaba y los sacudía para poder despertarlos, dijo: «Levantaos, volvamos al campamento, porque he aquí que el que me traiciona está cerca, y ha llegado la hora en que mi rebaño va a ser dispersado. Pero ya os he hablado de estas cosas».

\par 
%\textsuperscript{(1968.6)}
\textsuperscript{182:3.5} Durante los años que Jesús vivió entre sus discípulos, éstos tuvieron en verdad muchas pruebas de su naturaleza divina, pero en este momento están a punto de presenciar nuevas evidencias de su humanidad. Justo antes de la más grande de todas las revelaciones de su divinidad, su resurrección, deben producirse las pruebas más grandes de su naturaleza mortal: su humillación y su crucifixión.

\par 
%\textsuperscript{(1969.1)}
\textsuperscript{182:3.6} Cada vez que había orado en el jardín, su humanidad se había aferrado más firmemente, por la fe, a su divinidad; su voluntad humana se había unificado más completamente con la voluntad divina de su Padre. Entre otras palabras que le había dicho el ángel poderoso, se encontraba el mensaje de que el Padre deseaba que su Hijo terminara su donación terrenal pasando por la experiencia de la muerte que atraviesan las criaturas, exactamente como todas las criaturas mortales deben experimentar la disolución material cuando pasan de la existencia en el tiempo a la progresión en la eternidad\footnote{\textit{Palabras del ángel}: Lc 22:43.}.

\par 
%\textsuperscript{(1969.2)}
\textsuperscript{182:3.7} Anteriormente aquella noche, no había parecido tan difícil beber la copa, pero cuando el Jesús humano se despidió de sus apóstoles y los envió a descansar, la prueba se volvió más espantosa. Jesús experimentaba esos sentimientos naturales de flujo y de reflujo que toda experiencia humana tiene en común, y en aquel momento estaba cansado de trabajar, agotado por las largas horas de esfuerzo tenaz y de penosa ansiedad a causa de la seguridad de sus apóstoles. Aunque ningún mortal puede atreverse a comprender los pensamientos y sentimientos del Hijo encarnado de Dios en un momento como éste, sabemos que soportó una gran angustia y sufrió una tristeza indecible\footnote{\textit{Agonía de Jesús}: Lc 22:44.}, porque grandes gotas de sudor corrían por su rostro. Por fin estaba convencido de que el Padre tenía la intención de dejar que los acontecimientos naturales siguieran su curso; estaba plenamente decidido a no emplear, para salvarse, ninguno de sus poderes soberanos como jefe supremo de un universo.

\par 
%\textsuperscript{(1969.3)}
\textsuperscript{182:3.8} Las huestes reunidas de una inmensa creación se cernían ahora sobre esta escena\footnote{\textit{Los ejércitos de los cielos a mano}: Mt 26:53.}, bajo el mando temporal conjunto de Gabriel y del Ajustador Personalizado de Jesús. Los jefes de división de estos ejércitos del cielo habían sido advertidos repetidas veces que no interfirieran en estas actividades terrenales, a menos que el mismo Jesús les ordenara que intervinieran.

\par 
%\textsuperscript{(1969.4)}
\textsuperscript{182:3.9} La experiencia de separarse de los apóstoles suponía una gran tensión para el corazón humano de Jesús; esta tristeza de amor pesaba sobre él y le hacía más difícil enfrentarse a una muerte como la que sabía muy bien que le esperaba. Se daba cuenta de cuán débiles e ignorantes eran sus apóstoles, y temía abandonarlos. Sabía muy bien que había llegado la hora de su partida, pero su corazón humano anhelaba descubrir si no existía la posibilidad de que hubiera alguna vía legítima para escapar de este trance terrible de sufrimiento y de pena. Cuando su corazón hubo buscado así una escapatoria, sin conseguirla, estuvo dispuesto a beber la copa. La mente divina de Miguel sabía que había hecho todo lo posible por los doce apóstoles; pero el corazón humano de Jesús deseaba haber hecho más por ellos antes de dejarlos solos en el mundo. El corazón de Jesús estaba destrozado; amaba sinceramente a sus hermanos. Estaba aislado de su familia carnal; uno de sus asociados escogidos lo estaba traicionando. El pueblo de su padre José lo había rechazado y había sellado así su destino como pueblo con una misión especial en la Tierra. Su alma estaba atormentada por el amor frustrado y la misericordia rechazada. Se trataba de uno de esos momentos terribles en la vida de un hombre en que todo parece aplastarlo con una crueldad demoledora y una agonía terrible.

\par 
%\textsuperscript{(1969.5)}
\textsuperscript{182:3.10} La naturaleza humana de Jesús no era insensible a esta situación de soledad personal, de oprobio público y de fracaso aparente de su causa. Todos estos sentimientos pesaban sobre él con una fuerza indescriptible. En medio de esta gran tristeza, su mente volvió a los tiempos de su infancia en Nazaret y de sus primeros trabajos en Galilea. En el momento de esta gran prueba, muchas escenas agradables de su ministerio terrenal surgieron en su mente. Gracias a estos antiguos recuerdos de Nazaret, Cafarnaúm, el Monte Hermón y las salidas y puestas de Sol en el resplandeciente mar de Galilea, logró calmarse mientras fortalecía y preparaba su corazón humano para salir al encuentro del traidor que tan pronto iba a traicionarlo.

\par 
%\textsuperscript{(1970.1)}
\textsuperscript{182:3.11} Antes de que Judas y los soldados llegaran, el Maestro había recuperado por completo su equilibrio habitual; el espíritu había triunfado sobre la carne; la fe se había afirmado sobre todas las tendencias humanas al temor y a albergar dudas. La prueba suprema del desarrollo completo de la naturaleza humana había sido afrontada y superada de manera aceptable. Una vez más, el Hijo del Hombre estaba preparado para enfrentarse a sus enemigos con serenidad y con la plena seguridad de que era invencible como hombre mortal dedicado sin reservas a hacer la voluntad de su Padre.