\chapter{Documento 190. Las apariciones morontiales de Jesús}
\par 
%\textsuperscript{(2029.1)}
\textsuperscript{190:0.1} EL JESÚS resucitado se prepara ahora para pasar un corto período en Urantia con el fin de experimentar la carrera morontial ascendente de un mortal de los reinos. Aunque este período de vida morontial deberá pasarlo en el mundo de su encarnación como mortal, sin embargo será equivalente en todos los sentidos a la experiencia de los mortales de Satania que pasan por la vida morontial progresiva de los siete mundos de las mansiones de Jerusem.

\par 
%\textsuperscript{(2029.2)}
\textsuperscript{190:0.2} Todo este poder inherente a Jesús ---el don de la vida--- \footnote{\textit{Jesús controla toda la «vida»}: Jn 10:17-18; Jn 11:25-26; Jn 17:2-3.}que le permitió resucitar de entre los muertos, es el mismo don de la vida eterna que él concede a los creyentes en el reino, y que incluso ahora asegura la resurrección de éstos de las ataduras de la muerte natural.

\par 
%\textsuperscript{(2029.3)}
\textsuperscript{190:0.3} Los mortales de los reinos se levantarán, en la mañana de la resurrección, con el mismo tipo de cuerpo de transición, o morontial, que Jesús tenía cuando se levantó de la tumba este domingo por la mañana. Estos cuerpos no tienen circulación sanguínea, y estos seres no comen los alimentos materiales corrientes; sin embargo, estas formas morontiales son \textit{reales}. Cuando los diversos creyentes vieron a Jesús después de su resurrección, lo vieron realmente, no fueron víctimas del engaño de sus propias visiones o alucinaciones.

\par 
%\textsuperscript{(2029.4)}
\textsuperscript{190:0.4} Una fe permanente en la resurrección de Jesús\footnote{\textit{La fe en la resurrección}: Hch 4:10-12; Ro 10:9-13; 1 Co 15:12-20; 1 P 1:21.} fue la característica esencial de la fe de todas las ramas de la enseñanza primitiva del evangelio. En Jerusalén, Alejandría, Antioquía y Filadelfia, todos los educadores del evangelio se unieron en esta fe implícita en la resurrección del Maestro.

\par 
%\textsuperscript{(2029.5)}
\textsuperscript{190:0.5} Al examinar el papel sobresaliente que jugó María Magdalena en la proclamación de la resurrección del Maestro, hay que indicar que María era la portavoz principal del grupo femenino, tal como Pedro lo era de los apóstoles. María no era la directora de las mujeres que trabajaban para el reino, pero era su educadora principal y su portavoz pública. María se había convertido en una mujer muy prudente, de manera que la audacia que mostró al hablarle a un hombre que había tomado por el jardinero de José, sólo indica el horror que sintió cuando encontró la tumba vacía. La profundidad y la agonía de su amor, la plenitud de su devoción, fueron las que le hicieron olvidar por un momento las prohibiciones convencionales que tenía una mujer judía para dirigirse a un desconocido.

\section*{1. Los anunciadores de la resurrección}
\par 
%\textsuperscript{(2029.6)}
\textsuperscript{190:1.1} Los apóstoles no querían que Jesús los dejara; por eso no habían hecho caso de todas sus declaraciones sobre su muerte, así como de sus promesas de resucitar. No esperaban que la resurrección se produjera tal como ocurrió, y se negaron a creer hasta que tuvieron que hacer frente al apremio de una evidencia indiscutible y de la prueba absoluta de sus propias experiencias.

\par 
%\textsuperscript{(2030.1)}
\textsuperscript{190:1.2} Cuando los apóstoles se negaron a creer en el relato de las cinco mujeres que manifestaban que habían visto a Jesús y hablado con él, María Magdalena regresó al sepulcro, y las demás volvieron a la casa de José, donde relataron sus experiencias a la hija de José y a las otras mujeres. Y las mujeres creyeron en sus declaraciones. Poco después de las seis, la hija de José de Arimatea y las cuatro mujeres que habían visto a Jesús fueron a la casa de Nicodemo, donde contaron todos estos sucesos a José, Nicodemo, David Zebedeo y a los otros hombres que estaban allí reunidos. Nicodemo y los demás dudaron de esta historia, dudaron de que Jesús hubiera resucitado de entre los muertos; supusieron que los judíos habían trasladado el cuerpo. José y David estaban dispuestos a creer en el relato, de tal manera que se apresuraron a ir a inspeccionar la tumba, y lo encontraron todo tal como las mujeres lo habían descrito. Fueron los últimos en ver así el sepulcro, porque a las siete y media el sumo sacerdote envió al capitán de los guardias del templo a la tumba para que se llevara los lienzos fúnebres. El capitán los envolvió en la sábana de lino y los tiró por un barranco cercano.

\par 
%\textsuperscript{(2030.2)}
\textsuperscript{190:1.3} Desde la tumba, David y José fueron inmediatamente a la casa de Elías Marcos, donde mantuvieron una conferencia con los diez apóstoles en la habitación de arriba. Sólo Juan Zebedeo estaba dispuesto a creer, aunque débilmente, que Jesús había resucitado de entre los muertos. Pedro había creído al principio, pero como no logró encontrar al Maestro, empezó a tener grandes dudas. Todos estaban dispuestos a creer que los judíos se habían llevado el cuerpo. David no quiso discutir con ellos, pero en el momento de irse, dijo: «Vosotros sois los apóstoles, y deberíais comprender estas cosas. No discutiré con vosotros; no obstante, ahora regreso a la casa de Nicodemo, donde he indicado a los mensajeros que nos reuniremos esta mañana. Cuando se hayan reunido, los enviaré a realizar su última misión, la de anunciar la resurrección del Maestro. Escuché decir al Maestro que, después de su muerte, resucitaría al tercer día, y yo le creo». Después de hablar así a los abatidos y desamparados embajadores del reino, este joven que se había nombrado a sí mismo jefe de las comunicaciones y de la información se despidió de los apóstoles. Al salir de la habitación de arriba, dejó caer la bolsa de Judas, que contenía todos los fondos apostólicos, en el regazo de Mateo Leví.

\par 
%\textsuperscript{(2030.3)}
\textsuperscript{190:1.4} Eran aproximadamente las nueve y media cuando el último de los veintiséis mensajeros de David llegó a la casa de Nicodemo. David los reunió enseguida en el espacioso patio y se dirigió a ellos, diciendo:

\par 
%\textsuperscript{(2030.4)}
\textsuperscript{190:1.5} «Amigos y hermanos, me habéis servido todo este tiempo de acuerdo con vuestro juramento hacia mí y entre vosotros mismos, y os tomo por testigos de que hasta ahora nunca he enviado una falsa información por medio de vosotros. Estoy a punto de enviaros a vuestra última misión como mensajeros voluntarios del reino, y al hacer esto os libero de vuestro juramento, y con ello disuelvo este cuerpo de mensajeros. Amigos, os manifiesto que hemos terminado nuestra tarea. El Maestro ya no tiene necesidad de mensajeros mortales; ha resucitado de entre los muertos. Antes de que lo arrestaran nos dijo que moriría y que resucitaría al tercer día. Yo he visto la tumba ---está vacía. He hablado con María Magdalena y con otras cuatro mujeres que han conversado con Jesús. Ahora disuelvo este grupo, me despido de vosotros y os envío a vuestras misiones respectivas con el siguiente mensaje que llevaréis a los creyentes: `Jesús ha resucitado de entre los muertos; la tumba está vacía.'»

\par 
%\textsuperscript{(2030.5)}
\textsuperscript{190:1.6} La mayoría de los que estaban presentes trataron de persuadir a David para que no hiciera esto. Pero no pudieron influir sobre él. Entonces intentaron disuadir a los mensajeros, pero éstos no quisieron prestar atención a sus palabras de duda. Y así, poco antes de las diez de este domingo por la mañana, estos veintiséis corredores salieron como los primeros anunciadores del hecho y de la verdad poderosos de la resurrección de Jesús. Y partieron para esta misión como lo habían hecho para tantas otras, para cumplir el juramento realizado a David Zebedeo y entre ellos mismos. Estos hombres tenían una gran confianza en David. Partieron para efectuar esta tarea sin detenerse siquiera para hablar con las mujeres que habían visto a Jesús; aceptaron la palabra de David. La mayoría de ellos creía en lo que David les había dicho, e incluso aquellos que dudaban un poco, llevaron el mensaje con la misma certeza y la misma rapidez que los demás.

\par 
%\textsuperscript{(2031.1)}
\textsuperscript{190:1.7} Este día, los apóstoles ---el cuerpo espiritual del reino--- están reunidos en la sala de arriba donde manifiestan su temor y expresan sus dudas, mientras que estos mensajeros laicos, que representan el primer intento de socialización del evangelio de la fraternidad de los hombres del Maestro, bajo las órdenes de su jefe audaz y eficiente, salen para proclamar que el Salvador de un mundo y de un universo ha resucitado. Y emprenden este servicio extraordinario antes de que los representantes escogidos del Maestro estén dispuestos a creer en su palabra o a aceptar el testimonio de los testigos oculares.

\par 
%\textsuperscript{(2031.2)}
\textsuperscript{190:1.8} Estos veintiséis fueron enviados a la casa de Lázaro en Betania y a todos los centros de creyentes, desde Beerseba en el sur hasta Damasco y Sidón en el norte, y desde Filadelfia en el este hasta Alejandría en el oeste.

\par 
%\textsuperscript{(2031.3)}
\textsuperscript{190:1.9} Cuando David se hubo despedido de sus hermanos, fue a buscar a su madre a la casa de José, y partieron entonces para Betania a fin de reunirse con la familia de Jesús que les estaba esperando. David permaneció en Betania con Marta y María hasta que éstas vendieron sus bienes terrenales, y luego las acompañó en su viaje para reunirse con su hermano Lázaro en Filadelfia.

\par 
%\textsuperscript{(2031.4)}
\textsuperscript{190:1.10} Cerca de una semana más tarde, Juan Zebedeo llevó a María la madre de Jesús a la casa que él tenía en Betsaida. Santiago, el hermano mayor de Jesús, permaneció con su familia en Jerusalén. Rut se quedó en Betania con las hermanas de Lázaro. El resto de la familia de Jesús regresó a Galilea. David Zebedeo salió de Betania con Marta y María hacia Filadelfia a primeros de junio, al día siguiente de casarse con Rut, la hermana menor de Jesús.

\section*{2. La aparición de Jesús en Betania}
\par 
%\textsuperscript{(2031.5)}
\textsuperscript{190:2.1} Desde el momento de su resurrección morontial hasta el instante de su ascensión espiritual a las alturas, Jesús efectuó diecinueve apariciones distintas de forma visible a sus creyentes en la Tierra\footnote{\textit{Jesús se apareció a creyentes}: Hch 10:41.}. No se apareció a sus enemigos ni a aquellos que no podían hacer un uso espiritual de su manifestación en forma visible. Su primera aparición fue a las cinco mujeres cerca de la tumba; la segunda, a María Magdalena, también cerca de la tumba.

\par 
%\textsuperscript{(2031.6)}
\textsuperscript{190:2.2} La tercera aparición tuvo lugar alrededor del mediodía de este domingo en Betania. Poco después del mediodía, Santiago, el hermano mayor de Jesús, se encontraba en el jardín de Lázaro delante de la tumba vacía del hermano resucitado de Marta y María, dándole vueltas en su cabeza a las noticias que el mensajero de David les había traído una hora antes. Santiago siempre había tendido a creer en la misión de su hermano mayor en la Tierra, pero hacía mucho tiempo que había perdido el contacto con el trabajo de Jesús, y se había puesto a dudar seriamente de las afirmaciones posteriores de los apóstoles de que Jesús era el Mesías. Toda la familia estaba alarmada y casi confundida por la noticia que había traído el mensajero. Mientras Santiago permanecía delante de la tumba vacía de Lázaro, María Magdalena llegó a la casa y empezó a contar emocionadamente a la familia sus experiencias de las primeras horas de la mañana en la tumba de José. Antes de que terminara, David Zebedeo llegó con su madre. Rut creía, por supuesto, en el relato, y lo mismo le sucedió a Judá después de hablar con David y Salomé.

\par 
%\textsuperscript{(2032.1)}
\textsuperscript{190:2.3} Entretanto, mientras buscaban a Santiago y antes de que llegaran a encontrarlo, éste permanecía allí en el jardín cerca de la tumba, y se dio cuenta de una presencia cercana, como si alguien le hubiera tocado en el hombro. Cuando se volvió para mirar, contempló la aparición gradual de una forma extraña a su lado. Estaba demasiado asombrado para hablar y demasiado asustado para huir. Entonces, la extraña forma habló y dijo: «Santiago, vengo para llamarte al servicio del reino. Únete sinceramente a tus hermanos y sígueme». Cuando Santiago escuchó su nombre, supo que era su hermano mayor, Jesús, el que le había dirigido la palabra. Todos tenían más o menos dificultades para reconocer la forma morontial del Maestro, pero pocos de ellos tenían el menor problema para reconocer su voz o identificar de otra manera su encantadora personalidad en cuanto empezaba a comunicarse con ellos\footnote{\textit{Aparición a Santiago}: 1 Co 15:7a.}.

\par 
%\textsuperscript{(2032.2)}
\textsuperscript{190:2.4} Cuando Santiago se dio cuenta de que Jesús le estaba hablando, empezó a ponerse de rodillas, exclamando: «Padre y hermano mío», pero Jesús le pidió que permaneciera de pie mientras hablaba con él. Caminaron por el jardín y conversaron casi tres minutos; hablaron de las experiencias del pasado e hicieron planes para el futuro cercano. Mientras se acercaban a la casa, Jesús dijo: «Adiós, Santiago, hasta que os salude a todos juntos».

\par 
%\textsuperscript{(2032.3)}
\textsuperscript{190:2.5} Santiago entró corriendo en la casa, mientras lo buscaban en Betfagé, exclamando: «Acabo de ver a Jesús y he hablado con él, he charlado con él. No está muerto; ¡ha resucitado! Ha desaparecido delante de mí, diciendo: `Adiós, hasta que os salude a todos juntos'». Apenas había acabado de hablar cuando Judá regresó, y volvió a contar la experiencia del encuentro con Jesús en el jardín para que Judá la escuchara. Todos empezaron a creer en la resurrección de Jesús. Santiago anunció entonces que no volvería a Galilea, y David exclamó: «No solamente lo ven las mujeres emocionadas; incluso los hombres valerosos han empezado a verlo. Espero verlo yo mismo».

\par 
%\textsuperscript{(2032.4)}
\textsuperscript{190:2.6} David no tuvo que esperar mucho tiempo, porque la cuarta aparición de Jesús en la que fue reconocido por los mortales, tuvo lugar poco antes de las dos de la tarde en esta misma casa de Marta y María, cuando apareció de manera visible delante de su familia terrenal y de los amigos de ésta, veinte personas en total. El Maestro apareció en la puerta de atrás, que estaba abierta, diciendo: «Que la paz sea con vosotros. Saludos para aquellos que estuvieron cerca de mí en la carne, y fraternidad para mis hermanos y hermanas en el reino de los cielos. ¿Cómo habéis podido dudar? ¿Por qué habéis esperado tanto tiempo antes de escoger seguir de todo corazón la luz de la verdad? Entrad pues todos en la comunión del Espíritu de la Verdad en el reino del Padre». Cuando empezaron a recuperarse del primer impacto de su asombro y a acercarse a él como para abrazarlo, desapareció de su vista.

\par 
%\textsuperscript{(2032.5)}
\textsuperscript{190:2.7} Todos querían precipitarse hacia la ciudad para contarle a los incrédulos apóstoles lo que había sucedido, pero Santiago los detuvo. Sólo María Magdalena recibió permiso para regresar a la casa de José. Santiago les prohibió que anunciaran públicamente el hecho de esta visita morontial, debido a ciertas cosas que Jesús le había dicho mientras conversaban en el jardín. Pero Santiago nunca reveló más cosas sobre su conversación de este día con el Maestro resucitado en la casa de Lázaro en Betania.

\section*{3. En la casa de José}
\par 
%\textsuperscript{(2033.1)}
\textsuperscript{190:3.1} La quinta manifestación morontial de Jesús, reconocida por los ojos mortales, se produjo en presencia de unas veinticinco mujeres creyentes reunidas en la casa de José de Arimatea, hacia las cuatro y quince minutos de este mismo domingo por la tarde. María Magdalena había vuelto a la casa de José unos minutos antes de esta aparición. Santiago, el hermano de Jesús, había rogado que no se dijera nada a los apóstoles acerca de la aparición del Maestro en Betania, pero no le había pedido a María que se abstuviera de informar a sus hermanas creyentes sobre este acontecimiento. En consecuencia, después de que María hiciera prometer a todas las mujeres que guardarían el secreto, procedió a contarles lo que acababa de suceder mientras estaba con la familia de Jesús en Betania. Estaba precisamente en medio de este relato apasionante, cuando un silencio repentino y solemne se hizo entre ellas; vieron en medio de su grupo la forma enteramente visible de Jesús resucitado. Éste las saludó diciendo: «Que la paz sea con vosotras. En la hermandad del reino no habrá ni judíos ni gentiles, ni ricos ni pobres, ni libres ni esclavos, ni hombres ni mujeres\footnote{\textit{En el reino no habrá diferencias}: 2 Cr 19:7; Job 34:19; Eclo 35:12; Hch 10:34; Ro 2:11; Gl 2:6; 3:28; Ef 6:9; Col 3:11.}. Vosotras también estáis llamadas a divulgar la buena nueva\footnote{\textit{La gran asignación}: Mt 24:14; Mt 28:19-20a; Mc 13:10; Mc 16:15; Lc 24:47; Jn 17:18; Hch 1:8b.} de la liberación de la humanidad a través del evangelio de la filiación\footnote{\textit{El evangelio de la filiación}: 1 Cr 22:10; Sal 2:7; Is 56:5; Mt 5:9,16,45; Lc 20:36; Jn 1:12-13; 11:52; Hch 17:28-29; Ro 8:14-17,19,21; 9:26; 2 Co 6:18; Gl 3:26; 4:5-7; Ef 1:5; Flp 2:15; Heb 12:5-8; 1 Jn 3:1-2,10; 5:2; Ap 21:7; 2 Sam 7:14.} con Dios en el reino de los cielos. Id por el mundo entero proclamando este evangelio y confirmando a los creyentes en la fe del mismo. Y mientras lo hacéis, no olvidéis cuidar a los enfermos y fortalecer a los tímidos y a los que están dominados por el temor. Siempre estaré con vosotras, incluso hasta los confines de la Tierra»\footnote{\textit{La gran promesa}: Mt 28:20.}. Cuando hubo hablado así, desapareció de su vista, mientras las mujeres caían de bruces y adoraban en silencio.

\par 
%\textsuperscript{(2033.2)}
\textsuperscript{190:3.2} De las cinco apariciones morontiales de Jesús acontecidas hasta este momento, María Magdalena había presenciado cuatro.

\par 
%\textsuperscript{(2033.3)}
\textsuperscript{190:3.3} A consecuencia de haber enviado a los mensajeros a media mañana, y debido a la filtración inconsciente de indicios relacionados con esta aparición de Jesús en la casa de José, los dirigentes de los judíos empezaron a recibir noticias al principio del anochecer de que se decía por la ciudad que Jesús había resucitado, y que muchas personas pretendían haberlo visto. Estos rumores excitaron enormemente a los sanedristas. Después de consultar apresuradamente con Anás, Caifás convocó una reunión del sanedrín para las ocho de aquella noche. En esta reunión se tomó la decisión de echar de las sinagogas a toda persona que mencionara la resurrección de Jesús. Se sugirió incluso que cualquiera que afirmara haberlo visto debía ser ejecutado; sin embargo, esta proposición no se sometió a votación ya que la reunión se disolvió en una confusión que rayaba en verdadero pánico. Se habían atrevido a pensar que habían acabado con Jesús. Estaban a punto de descubrir que sus verdaderas dificultades con el hombre de Nazaret sólo acababan de empezar.

\section*{4. La aparición a los griegos}
\par 
%\textsuperscript{(2033.4)}
\textsuperscript{190:4.1} Alrededor de las cuatro y media, el Maestro hizo su sexta aparición morontial a unos cuarenta creyentes griegos que estaban reunidos en la casa de un tal Flavio. Mientras estaban discutiendo las noticias sobre la resurrección del Maestro, éste se manifestó en medio de ellos, a pesar de que las puertas estaban bien cerradas, y les habló diciendo: «Que la paz sea con vosotros. Aunque el Hijo del Hombre apareció en la Tierra entre los judíos, vino para aportar su ministerio a todos los hombres. En el reino de mi Padre no habrá ni judíos ni gentiles\footnote{\textit{Ni judíos ni gentiles}: 2 Cr 19:7; Job 34:19; Eclo 35:12; Hch 10:34; Ro 2:11; Gl 2:6; 3:28; Ef 6:9; Col 3:11.}; todos seréis hermanos ---los hijos de Dios. Id pues a proclamar al mundo entero\footnote{\textit{La gran asignación}: Mt 24:14; 28:19-20a; Mc 13:10; 16:15; Lc 24:47; Jn 17:18; Hch 1:8b.} este evangelio de salvación\footnote{\textit{Evangelio de salvación}: Lc 1:69,77; 3:6; Jn 4:22-23; Hch 4:12; 13:47; 16:17; 28:28; Ro 1:16-17; 10:10-13; Ef 1:13-14; Flp 1:19,28; 1 Ts 5:9; Tit 2:11.} tal como lo habéis recibido de los embajadores del reino, y yo os recibiré en la comunión de la fraternidad\footnote{\textit{Hermandad, fraternidad}: 1 Co 1:9-10; 1 Jn 1:3-7.} de los hijos de la fe\footnote{\textit{Los hijos de la fe}: Gl 3:26.} y de la verdad del Padre». Cuando les hubo encargado esta misión, se despidió y no lo volvieron a ver. Permanecieron dentro de la casa toda la noche; estaban demasiado dominados por el pavor y el miedo como para atreverse a salir. Ninguno de estos griegos tampoco durmió aquella noche; se quedaron despiertos discutiendo estas cosas y esperando que el Maestro los visitara de nuevo. En este grupo había muchos griegos que estaban en Getsemaní cuando los soldados arrestaron a Jesús y Judas lo traicionó con un beso.

\par 
%\textsuperscript{(2034.1)}
\textsuperscript{190:4.2} Los rumores de la resurrección de Jesús y las noticias sobre las numerosas apariciones a sus seguidores se están difundiendo rápidamente, y toda la ciudad está alcanzando un alto grado de agitación. El Maestro ya se ha aparecido a su familia, a las mujeres y a los griegos, y dentro de poco se va a manifestar en medio de los apóstoles. El sanedrín pronto va a empezar a examinar estos nuevos problemas que se han impuesto tan repentinamente a los dirigentes judíos. Jesús piensa mucho en sus apóstoles, pero desea que sigan estando solos algunas horas más para que reflexionen seriamente y mediten cuidadosamente antes de visitarlos.

\section*{5. El paseo con los dos hermanos}
\par 
%\textsuperscript{(2034.2)}
\textsuperscript{190:5.1} En Emaús, a unos once kilómetros al oeste de Jerusalén, vivían dos hermanos, pastores, que habían pasado la semana de la Pascua en Jerusalén asistiendo a los sacrificios, las ceremonias y las fiestas. Cleofás, el mayor, creía parcialmente en Jesús; al menos había sido expulsado de la sinagoga. Su hermano, Jacobo, no era creyente, aunque estaba muy intrigado por las cosas que había escuchado acerca de las enseñanzas y las obras del Maestro.

\par 
%\textsuperscript{(2034.3)}
\textsuperscript{190:5.2} Este domingo por la tarde, a unos cinco kilómetros de Jerusalén y pocos minutos antes de las cinco, mientras estos dos hermanos caminaban por la carretera de Emaús\footnote{\textit{La carretera a Emaús}: Mc 16:12; Lc 24:13-16.}, iban hablando con mucha seriedad de Jesús, de sus enseñanzas, de sus obras, y muy en particular de los rumores de que su tumba estaba vacía, y de que algunas mujeres habían hablado con él. Cleofás tenía una ligera inclinación a creer en estas noticias, pero Jacobo insistía en que todo el asunto era probablemente un engaño. Mientras razonaban y discutían así a medida que se dirigían hacia su casa, la manifestación morontial de Jesús, su séptima aparición, caminó con ellos mientras continuaban el viaje. Cleofás había escuchado a Jesús enseñar con frecuencia y había comido con él en diversas ocasiones en las casas de los creyentes de Jerusalén. Pero no reconoció al Maestro, ni siquiera cuando éste les habló con toda libertad.

\par 
%\textsuperscript{(2034.4)}
\textsuperscript{190:5.3} Después de acompañarlos durante un corto trayecto, Jesús dijo: «¿De qué hablabais con tanta seriedad cuando me acerqué a vosotros?» Cuando Jesús dijo esto, se detuvieron y le miraron con una sorpresa entristecida. Cleofás dijo: «¿Es posible que vivas en Jerusalén y no conozcas las cosas que han sucedido recientemente?» Entonces preguntó el Maestro: «¿Qué cosas?» Cleofás respondió: «Si no sabes estas cosas, eres el único en Jerusalén que no ha escuchado los rumores sobre Jesús de Nazaret, que era un profeta poderoso en palabras y en acciones delante de Dios y de todo el pueblo. Los jefes de los sacerdotes y nuestros dirigentes lo entregaron a los romanos y les pidieron que lo crucificaran. Ahora bien, muchos de nosotros habíamos esperado que él fuera el que liberara a Israel del yugo de los gentiles. Pero esto no es todo. Ahora hace tres días que fue crucificado, y unas mujeres nos han sorprendido hoy declarando que esta mañana muy temprano fueron a su tumba y la encontraron vacía. Y estas mismas mujeres insisten en que han hablado con ese hombre; sostienen que ha resucitado de entre los muertos. Cuando las mujeres informaron de esto a los hombres, dos de sus apóstoles corrieron hasta la tumba y la encontraron igualmente vacía» ---y aquí Jacobo interrumpió a su hermano para decir: «pero no vieron a Jesús»\footnote{\textit{Hablaban sobre Jesús}: Lc 24:17-24.}.

\par 
%\textsuperscript{(2035.1)}
\textsuperscript{190:5.4} Mientras seguían caminando, Jesús les dijo\footnote{\textit{El discurso de Jesús}: Lc 24:27.}: «¡Qué lentos sois en comprender la verdad!\footnote{\textit{Los hombres son lentos en comprender la verdad}: Lc 24:25-27.} Puesto que me decís que estabais discutiendo de las enseñanzas y de las obras de este hombre, quizás yo pueda iluminaros, puesto que estoy más que familiarizado con esas enseñanzas. ¿No recordáis que ese Jesús siempre enseñó que su reino no era de este mundo\footnote{\textit{Su reino no es de este mundo}: Jn 8:23; 18:36.}, y que como todos los hombres son hijos de Dios\footnote{\textit{Todos los hombres son hijos de Dios}: 1 Cr 22:10; Sal 2:7; Is 56:5; Mt 5:9,16,45; Lc 20:36; Jn 1:12-13; 11:52; Hch 17:28-29; Ro 8:14-17,19,21; 9:26; 2 Co 6:18; Gl 3:26; 4:5-7; Ef 1:5; Flp 2:15; Heb 12:5-8; 1 Jn 3:1-2,10; 5:2; Ap 21:7; 2 Sam 7:14.}, deberían encontrar la libertad y la independencia en la alegría espiritual de la comunión de la fraternidad del servicio amoroso en este nuevo reino de la verdad del amor del Padre celestial? ¿No recordáis cómo este Hijo del Hombre proclamó la salvación de Dios para todos los hombres, cuidando a los enfermos y a los afligidos, y liberando a los que estaban encadenados por el miedo y esclavizados por el mal? ¿No sabéis que este hombre de Nazaret dijo a sus discípulos que debía ir a Jerusalén, ser entregado a sus enemigos, que lo ejecutarían, y que resucitaría al tercer día?\footnote{\textit{Predicción de la muerte y resurrección}: Mt 16:21; Mt 17:22-23a; Mt 20:17-19; Mt 27:63; Mc 8:31; Mc 9:31; Mc 10:32-34; Lc 9:22,31,43b-44; Lc 18:31-33; Lc 24:7,46; Jn 14:28a; Jn 20:9.} ¿No os han dicho todo esto? ¿Y no habéis leído nunca en las Escrituras acerca de este día de salvación para los judíos y los gentiles, donde dice que en él todas las familias de la Tierra serán benditas\footnote{\textit{Todas las familias serán benditas}: Gn 12:3; Gn 28:14.}; que él escuchará el lamento de los necesitados\footnote{\textit{Escuchará el lamento de los necesitados}: Sal 72:4,12-13; 113:5-8; Is 41:17.} y salvará el alma de los pobres que lo buscan; que todas las naciones lo llamarán bendito\footnote{\textit{Las naciones lo llamarán bendito}: Sal 72:17; Mal 3:12.}? Que este Libertador será como la sombra de una gran roca\footnote{\textit{La sombra de una gran roca}: Is 32:2.} en una tierra agotada. Que alimentará al rebaño como un verdadero pastor\footnote{\textit{Alimentará a su rebaño como un verdadero pastor}: Is 40:11.}, reuniendo a las ovejas en sus brazos y llevándolas tiernamente en su seno. Que abrirá los ojos de los ciegos espirituales\footnote{\textit{Abrirá los ojos de los ciegos espirituales}: Is 42:7.} y sacará a los presos de la desesperación a la plena luz y libertad; que todos los que están en las tinieblas verán la gran luz\footnote{\textit{Los de las tinieblas verán una gran luz}: Is 9:2.} de la salvación eterna. Que curará a los que tienen el corazón destrozado\footnote{\textit{Curará a los de corazón destrozado}: Is 61:1.}, proclamará la libertad a los cautivos del pecado y abrirá la prisión a los que están esclavizados por el miedo y encadenados por el mal. Que consolará a los afligidos\footnote{\textit{Él consolará a los afligidos}: Is 57:18; 61:2-3; Jer 31:13; Mt 5:4.} y les otorgará la alegría de la salvación\footnote{\textit{La alegría de la salvación}: Sal 51:12; Is 12:3; Hab 3:18.} en lugar de la pena y la tristeza. Que él será el deseo de todas las naciones\footnote{\textit{Él es el deseo de todas las naciones}: Hag 2:7.} y la alegría perpetua\footnote{\textit{La alegría perpetua}: Is 35:10; Is 51:11; Is 61:7.} de los que buscan la rectitud. Que este Hijo de la verdad y de la rectitud\footnote{\textit{El Hijo, sol de recitud}: Mal 4:2.} se elevará sobre el mundo con una luz curativa y un poder salvador\footnote{\textit{Un poder salvador}: Sal 28:8.}; e incluso salvará a su pueblo de sus pecados\footnote{\textit{Nos salvará de los pecados}: Mt 1:21.}; que buscará y salvará realmente a los que están perdidos\footnote{\textit{Buscará y salvará a los que están perdidos}: Mt 18:11; Lc 19:10.}. Que no destruirá a los débiles, sino que aportará la salvación a todos los que tienen hambre y sed de rectitud\footnote{\textit{Sed de rectitud}: Mt 5:6.}. Que los que creen en él tendrán la vida eterna\footnote{\textit{Los que creen en Él vivirán eternamente}: Dn 12:2; Mt 19:16,29; 25:46; Mc 10:17,30; Lc 10:25; 18:18,30; Jn 3:15-16,36; 4:14,36; 5:24,39; 6:27,40,47; 6:54,68; 8:51-52; 10:28; 11:25-26; 12:25,50; 17:2-3; Hch 13:46-48; Ro 2:7; 5:21; 6:22-23; Gl 6:8; 1 Ti 1:16; 6:12,19; Tit 1:2; 3:7; 1 Jn 1:2; 2:25; 3:15; 5:11,13,20; Jud 1:21; Ap 22:5.}. Que derramará su espíritu sobre todo el género humano\footnote{\textit{Derramará su espíritu sobre toda la carne}: Ez 11:19; 18:31; 36:26-27; Jl 2:28-29; Lc 24:49; Jn 7:39; 14:16-18,23-26; 15:4,26; 16:7-8,13-14; 17:21-23; Hch 1:5,8a; 2:1-4,16-18; 2:33; 2 Co 13:5; Gl 2:20; 4:6; Ef 1:13; 4:30; 1 Jn 4:12-15.}, y que este Espíritu de la Verdad será en cada creyente una fuente de agua\footnote{\textit{El espíritu será una fuente de agua viviente}: Jn 4:14.} que brotará hasta la vida eterna. ¿No habéis comprendido la grandeza del evangelio del reino\footnote{\textit{El evangelio del reino}: Mt 3:2; 4:17,23; 5:3,10,19-20; 6:33; 7:21; 8:11; 9:35; 10:7; 11:11-12; 12:28; 13:11,24,31-52; 16:19; 18:1-4,23; 19:14,23-24; 20:1; 21:31,43; 22:2; 23:13; 24:14; 25:1,14; Mc 1:14-15; 4:11,26,30; 9:1,47; 10:14-15,23-25; 12:34; 14:25; 15:43; Lc 4:43; 6:20; 7:28; 8:1,10; 9:2,11,27; 9:60,62; 10:9-11; 11:20; 12:31-32; 13:18,20,28-29; 14:15; 16:16; 17:20-21; 18:16-17,24-25; 19:11; 21:31; 22:16-18; 23:51; Jn 3:3,5; Ro 14:17; 1 Co 4:20; 6:9-10.} que este hombre os entregó? ¿No percibís la grandeza de la salvación\footnote{\textit{La grandez de la salvación}: Heb 2:3.} que os ha llegado?»

\par 
%\textsuperscript{(2035.2)}
\textsuperscript{190:5.5} Para entonces habían llegado cerca del pueblo donde vivían estos hermanos. Estos dos hombres no habían dicho ni una palabra desde que Jesús empezó a enseñarlos mientras andaban por el camino. Pronto se detuvieron delante de su humilde morada, y Jesús estaba a punto de despedirse de ellos para continuar carretera abajo, pero le obligaron a entrar y a quedarse con ellos. Insistieron en que era casi de noche y que permaneciera con ellos. Jesús consintió finalmente, y poco después de entrar en la casa se sentaron para comer. Dieron el pan a Jesús para que lo bendijera, y cuando empezó a partirlo y a darlo a los hermanos, los ojos de éstos se abrieron, y Cleofás reconoció que su invitado era el Maestro mismo. Y cuando dijo: «Es el Maestro..»., el Jesús morontial desapareció de su vista\footnote{\textit{Llegada y reconocimiento}: Lc 24:28-31.}.

\par 
%\textsuperscript{(2036.1)}
\textsuperscript{190:5.6} Entonces se dijeron el uno al otro: «¡No es de extrañar que nuestro corazón ardiera\footnote{\textit{No extraña que nos ardiera el corazón}: Lc 24:32.} por dentro cuando nos hablaba mientras caminábamos por la carretera, y mientras abría nuestra inteligencia a las enseñanzas de las Escrituras!»

\par 
%\textsuperscript{(2036.2)}
\textsuperscript{190:5.7} Ni siquiera se detuvieron para comer. Habían visto al Maestro morontial y salieron precipitadamente de la casa, regresando rápidamente a Jerusalén\footnote{\textit{Regreso a Jerusalén}: Lc 24:33.} para difundir la buena nueva del Salvador resucitado.

\par 
%\textsuperscript{(2036.3)}
\textsuperscript{190:5.8} Hacia las nueve de aquella noche y poco antes de que el Maestro se apareciera a los diez, estos dos hermanos excitados irrumpieron en la habitación de arriba donde estaban los apóstoles, declarando que habían visto a Jesús y que habían hablado con él\footnote{\textit{Buenas noticias, el Señor ha resucitado}: Mc 16:13a; Lc 24:34-35.}. Contaron todo lo que Jesús les había dicho, y que no habían descubierto quién era hasta el momento en que partió el pan.