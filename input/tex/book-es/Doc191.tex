\chapter{Documento 191. Las apariciones a los apóstoles y a otros discípulos principales}
\par 
%\textsuperscript{(2037.1)}
\textsuperscript{191:0.1} EL DOMINGO de la resurrección fue un día terrible en la vida de los apóstoles; diez de ellos pasaron la mayor parte del día en la habitación de arriba detrás de las puertas atrancadas. Podían haber huido de Jerusalén, pero tenían miedo de ser arrestados por los agentes del sanedrín si los encontraban en la calle. Tomás rumiaba a solas sus problemas en Betfagé. Hubiera hecho mejor permaneciendo con sus compañeros apóstoles, y los hubiera ayudado a dirigir sus discusiones por unas vías más provechosas.

\par 
%\textsuperscript{(2037.2)}
\textsuperscript{191:0.2} A lo largo de todo el día, Juan sostuvo la idea de que Jesús había resucitado de entre los muertos. Recordó que en no menos de cinco ocasiones diferentes el Maestro había afirmado que resucitaría de nuevo, y que al menos tres veces había aludido al tercer día. La actitud de Juan tenía una influencia considerable sobre ellos, especialmente sobre su hermano Santiago y sobre Natanael. Juan los habría influido aún más si no hubiera sido el miembro más joven del grupo.

\par 
%\textsuperscript{(2037.3)}
\textsuperscript{191:0.3} Los problemas de los apóstoles estaban muy relacionados con su aislamiento. Juan Marcos los mantenía al corriente de lo que sucedía alrededor del templo y les informaba de los numerosos rumores que se difundían por la ciudad, pero no se le ocurrió recoger las noticias de los diferentes grupos de creyentes a los que Jesús ya se había aparecido. Este era el tipo de servicio que habían prestado hasta ahora los mensajeros de David, pero todos estaban ausentes realizando su última misión como anunciadores de la resurrección a los grupos de creyentes que vivían lejos de Jerusalén. Por primera vez en todos estos años, los apóstoles se dieron cuenta de cuánto habían dependido de los mensajeros de David para recibir su información diaria sobre los asuntos del reino.

\par 
%\textsuperscript{(2037.4)}
\textsuperscript{191:0.4} Como ya era típico en él, Pedro vaciló emocionalmente todo el día entre la fe y la duda con respecto a la resurrección del Maestro. Pedro no podía olvidar la visión de los lienzos fúnebres que yacían allí en la tumba como si el cuerpo de Jesús se hubiera evaporado desde dentro. «Pero», razonaba Pedro, «si ha resucitado y puede mostrarse a las mujeres, ¿por qué no se muestra a nosotros, sus apóstoles?» Pedro se entristecía cuando pensaba que Jesús quizás no venía hacia ellos a causa de su presencia entre los apóstoles, porque lo había negado aquella noche en el patio de Anás. Luego se animaba con el mensaje que habían traído las mujeres: «Id a decir a mis apóstoles ---y a Pedro». Pero estimularse con este mensaje implicaba que tenía que creer que las mujeres habían visto y oído realmente al Maestro resucitado. Pedro alternó así entre la fe y la duda durante todo el día, hasta poco después de las ocho, en que se atrevió a salir al patio. Pedro pensaba alejarse de los apóstoles para no impedir que Jesús viniera hasta ellos porque él había negado al Maestro.

\par 
%\textsuperscript{(2037.5)}
\textsuperscript{191:0.5} Santiago Zebedeo defendió al principio que todos debían ir a la tumba; estaba firmemente a favor de hacer algo para llegar hasta el fondo del misterio. Fue Natanael el que les impidió que se mostraran en público a consecuencia de los argumentos de Santiago, y lo hizo recordándoles la advertencia de Jesús de que no arriesgaran indebidamente sus vidas en estos momentos. Hacia el mediodía, Santiago se había calmado como los demás y permanecieron en una espera vigilante. Habló poco; estaba enormemente desilusionado porque Jesús no se les aparecía, y no sabía nada de las numerosas apariciones del Maestro a otros grupos y a otras personas.

\par 
%\textsuperscript{(2038.1)}
\textsuperscript{191:0.6} Andrés escuchó mucho este día. Estaba extremadamente perplejo por la situación y tenía más dudas de las que le correspondían, pero al menos disfrutaba de cierta sensación de libertad al no tener la responsabilidad de dirigir a los demás apóstoles. En verdad estaba agradecido al Maestro por haberle liberado de las cargas de la jefatura antes de que empezaran a vivir estas horas de confusión.

\par 
%\textsuperscript{(2038.2)}
\textsuperscript{191:0.7} Más de una vez durante las largas horas agotadoras de este día trágico, la única influencia que sostuvo al grupo fue la frecuente contribución de los consejos filosóficos característicos de Natanael. Él fue realmente la influencia que controló a los diez durante todo el día. Ni una sola vez expresó si creía o no en la resurrección del Maestro. Pero a medida que pasaba el día, se sintió cada vez más inclinado a creer que Jesús había cumplido su promesa de resucitar.

\par 
%\textsuperscript{(2038.3)}
\textsuperscript{191:0.8} Simón Celotes estaba demasiado abrumado como para participar en las discusiones. La mayor parte del tiempo permaneció recostado en un diván en un rincón de la habitación, mirando a la pared; no llegó a hablar media docena de veces en todo el día. Su concepto del reino se había derrumbado, y no lograba discernir que la resurrección del Maestro podía cambiar materialmente la situación. Su decepción era muy personal y demasiado aguda como para que pudiera reponerse a corto plazo, ni siquiera ante un hecho tan prodigioso como la resurrección.

\par 
%\textsuperscript{(2038.4)}
\textsuperscript{191:0.9} Aunque parezca extraño, Felipe, que habitualmente se expresaba poco, habló mucho durante toda la tarde de este día. Por la mañana tuvo poco que decir, pero se pasó toda la tarde haciendo preguntas a los demás apóstoles. Pedro se irritó a menudo con las preguntas de Felipe, pero los demás se las tomaron con buena disposición. Felipe deseaba saber en particular, en el caso de que Jesús hubiera resucitado realmente de la tumba, si su cuerpo tendría las marcas físicas de la crucifixión.

\par 
%\textsuperscript{(2038.5)}
\textsuperscript{191:0.10} Mateo estaba sumamente confundido; escuchó las discusiones de sus compañeros, pero pasó la mayor parte del tiempo dándole vueltas en la cabeza al problema de las finanzas futuras del grupo. Independientemente de la supuesta resurrección de Jesús, Judas ya no estaba, David le había entregado los fondos sin ceremonias, y no tenían un jefe con autoridad. Antes de que Mateo llegara a considerar seriamente los argumentos de los demás sobre la resurrección, ya había visto al Maestro cara a cara.

\par 
%\textsuperscript{(2038.6)}
\textsuperscript{191:0.11} Los gemelos Alfeo participaron poco en estos importantes debates; estaban plenamente ocupados en sus trabajos habituales. Uno de ellos expresó la actitud de los dos cuando dijo, en respuesta a una pregunta de Felipe: «No comprendemos esto de la resurrección, pero nuestra madre dice que ha hablado con el Maestro, y nosotros la creemos».

\par 
%\textsuperscript{(2038.7)}
\textsuperscript{191:0.12} Tomás se encontraba en medio de uno de sus típicos períodos de depresión desesperante. Durmió una parte del día y se paseó por las colinas el resto del tiempo. Sentía el impulso de reunirse con sus compañeros apóstoles, pero el deseo de estar solo era más fuerte.

\par 
%\textsuperscript{(2038.8)}
\textsuperscript{191:0.13} El Maestro aplazó su primera aparición morontial a los apóstoles por varias razones. En primer lugar, después de que oyeran hablar de su resurrección, quería que tuvieran tiempo para reflexionar bien sobre lo que les había dicho acerca de su muerte y de su resurrección cuando aún estaba con ellos en la carne. El Maestro quería que Pedro venciera algunas de sus dificultades particulares antes de manifestarse a todos ellos. En segundo lugar, deseaba que Tomás estuviera con ellos en el momento de su primera aparición. Juan Marcos localizó a Tomás en la casa de Simón en Betfagé este domingo por la mañana temprano, e informó de ello a los apóstoles alrededor de las once. Tomás hubiera regresado con ellos en cualquier momento de este día si Natanael u otros dos apóstoles cualquiera hubieran ido a buscarlo. Tenía realmente el deseo de volver, pero como los había dejado la noche anterior de la manera que lo había hecho, era demasiado orgulloso como para regresar tan pronto por su propia cuenta. Al día siguiente estaba tan deprimido que necesitó casi una semana para decidirse a regresar. Los apóstoles le esperaban, y él esperaba que sus hermanos fueran a buscarlo para pedirle que volviera con ellos. Tomás permaneció así alejado de sus compañeros hasta el sábado siguiente por la noche cuando, después del anochecer, Pedro y Juan fueron a Betfagé y lo trajeron de vuelta con ellos. Ésta es también la razón por la que no partieron inmediatamente para Galilea después de que Jesús se les apareciera por primera vez; no querían irse sin Tomás.

\section*{1. La aparición a Pedro}
\par 
%\textsuperscript{(2039.1)}
\textsuperscript{191:1.1} Eran casi las ocho y media de la noche de este domingo cuando Jesús se apareció a Simón Pedro en el jardín de la casa de Marcos. Ésta era su octava manifestación morontial. Pedro había vivido con una pesada carga de dudas y de culpabilidad desde que había negado al Maestro. Toda la jornada del sábado y este domingo había luchado contra el temor de que quizás ya no era un apóstol. Se había estremecido de horror ante la suerte de Judas, e incluso había pensado que él también había traicionado a su Maestro. Toda esta tarde pensó que quizás su presencia entre los apóstoles era la que impedía que Jesús se les apareciera, a condición, por supuesto, de que hubiera resucitado realmente de entre los muertos. Y fue a Pedro, en estas condiciones mentales y con este estado de ánimo, a quien Jesús se apareció mientras el deprimido apóstol deambulaba entre las flores y los arbustos.

\par 
%\textsuperscript{(2039.2)}
\textsuperscript{191:1.2} Cuando Pedro pensó en la mirada afectuosa del Maestro mientras éste pasaba por el porche de Anás, cuando dio vueltas en su cabeza al maravilloso mensaje «Id a decir a mis apóstoles ---y a Pedro» que le habían traído aquella mañana temprano las mujeres que regresaban de la tumba vacía, cuando contempló estas muestras de misericordia, su fe empezó a vencer sus dudas. Entonces se detuvo, apretando los puños, mientras decía en voz alta: «Creo que ha resucitado de entre los muertos; voy a decírselo a mis hermanos». Cuando pronunció estas palabras, la forma de un hombre apareció repentinamente delante de él y le habló con un tono de voz familiar, diciendo: «Pedro, el enemigo deseaba poseerte, pero no he querido abandonarte. Sabía que no me habías negado con el corazón; por eso te había perdonado incluso antes de que me lo pidieras; pero ahora debes dejar de pensar en ti mismo y en los problemas del momento, y prepararte para llevar la buena nueva del evangelio a los que están en las tinieblas. Ya no debe importarte lo que puedas obtener del reino, sino que debes preocuparte más bien por lo que puedas dar a los que viven en una espantosa miseria espiritual. Cíñete, Simón, para la batalla de un nuevo día, para la lucha contra las tinieblas espirituales y las dudas perjudiciales de la mente común de los hombres».

\par 
%\textsuperscript{(2039.3)}
\textsuperscript{191:1.3} Pedro y el Jesús morontial caminaron por el jardín y hablaron de las cosas del pasado, del presente y del futuro durante cerca de cinco minutos. Luego el Maestro desapareció de su vista, diciendo: «Adiós, Pedro, hasta que te vea con tus hermanos».

\par 
%\textsuperscript{(2039.4)}
\textsuperscript{191:1.4} Pedro se quedó aturdido durante un momento al darse cuenta de que había hablado con el Maestro resucitado, y que podía estar seguro de que continuaba siendo un embajador del reino. Acababa de escuchar al Maestro glorificado que le exhortaba a continuar predicando el evangelio. Con todo esto brotando en su corazón, se precipitó hacia la habitación de arriba donde estaban sus compañeros apóstoles, y jadeando de excitación exclamó: «He visto al Maestro; estaba en el jardín. He hablado con él y me ha perdonado».

\par 
%\textsuperscript{(2040.1)}
\textsuperscript{191:1.5} La declaración de Pedro de que había visto a Jesús en el jardín causó una profunda impresión en sus compañeros apóstoles, y estaban casi dispuestos a abandonar sus dudas cuando Andrés se levantó y les advirtió que no se dejaran influir demasiado por el relato de su hermano. Andrés dio a entender que Pedro había visto cosas irreales anteriormente. Aunque Andrés no aludió directamente a la visión nocturna en el Mar de Galilea, donde Pedro afirmó que había visto al Maestro venir hacia ellos caminando sobre el agua, dijo lo suficiente como para mostrar a todos los presentes que guardaba este incidente en la memoria. Simón Pedro se sintió muy dolido por las insinuaciones de su hermano, y cayó inmediatamente en un silencio alicaído. Los gemelos sintieron mucha compasión por Pedro; los dos se acercaron para expresarle su simpatía y decirle que ellos le creían, y reafirmar que su propia madre también había visto al Maestro.

\section*{2. La primera aparición a los apóstoles}
\par 
%\textsuperscript{(2040.2)}
\textsuperscript{191:2.1} Aquella noche poco después de las nueve, después de la partida de Cleofás y Jacobo, mientras los gemelos Alfeo consolaban a Pedro, y Natanael le hacía reproches a Andrés, y mientras los diez apóstoles estaban reunidos allí en la habitación de arriba con todas las puertas cerradas con cerrojo por temor a ser arrestados, el Maestro apareció de pronto en su forma morontial en medio de ellos, diciendo: «Que la paz sea con vosotros. ¿Por qué os asustáis tanto cuando aparezco, como si vierais a un espíritu? ¿No os he hablado de estas cosas cuando estaba presente con vosotros en la carne? ¿No os dije que los jefes de los sacerdotes y los dirigentes me entregarían para ser ejecutado, que uno de vosotros mismos me traicionaría, y que resucitaría al tercer día? ¿Por qué pues todas vuestras dudas y toda esta discusión acerca de los relatos de las mujeres, de Cleofás y de Jacobo, e incluso de Pedro? ¿Cuánto tiempo dudaréis de mis palabras y os negaréis a creer en mis promesas? Y ahora que me veis realmente, ¿vais a creer? Incluso ahora uno de vosotros está ausente. Cuando todos estéis juntos una vez más, y después de que todos sepáis con certeza que el Hijo del Hombre ha salido de la tumba, partid de aquí para Galilea. Tened fe en Dios; tened fe los unos en los otros; y así entraréis en el nuevo servicio del reino de los cielos. Permaneceré con vosotros en Jerusalén hasta que estéis preparados para ir a Galilea. Mi paz os dejo».

\par 
%\textsuperscript{(2040.3)}
\textsuperscript{191:2.2} Cuando el Jesús morontial les hubo dicho esto, desapareció de su vista en un instante. Todos cayeron de bruces, alabando a Dios y venerando a su desaparecido Maestro. Ésta fue la novena aparición morontial del Maestro.

\section*{3. Con los seres morontiales}
\par 
%\textsuperscript{(2040.4)}
\textsuperscript{191:3.1} Jesús pasó todo el día siguiente, lunes, con las criaturas morontiales entonces presentes en Urantia. Más de un millón de directores morontiales y sus asociados, así como los mortales de transición de diversas órdenes procedentes de los siete mundos de las mansiones de Satania, habían venido a Urantia para participar en la experiencia de transición morontial del Maestro. El Jesús morontial permaneció con estas espléndidas inteligencias durante cuarenta días. Los instruyó y aprendió de sus directores la vida de transición morontial tal como la atraviesan los mortales de los mundos habitados de Satania cuando pasan por las esferas morontiales del sistema.

\par 
%\textsuperscript{(2041.1)}
\textsuperscript{191:3.2} Alrededor de la medianoche de este lunes, la forma morontial del Maestro fue ajustada para la transición a la segunda fase de la evolución morontial. Cuando se apareció de nuevo a sus hijos mortales de la Tierra, era un ser morontial de la segunda fase. A medida que el Maestro progresaba en la carrera morontial, las inteligencias morontiales y sus asociados transformadores tenían cada vez más dificultades técnicas para hacer visible al Maestro a los ojos mortales y materiales.

\par 
%\textsuperscript{(2041.2)}
\textsuperscript{191:3.3} Jesús realizó el tránsito a la tercera fase morontial el viernes 14 de abril; a la cuarta el lunes 17; a la quinta el sábado 22; a la sexta el jueves 27; a la séptima el martes 2 de mayo; a la ciudadanía de Jerusem el domingo 7; y entró en el abrazo de los Altísimos de Edentia el domingo 14 de mayo.

\par 
%\textsuperscript{(2041.3)}
\textsuperscript{191:3.4} Miguel de Nebadon completó de esta manera su servicio de experiencia universal, puesto que en conexión con sus donaciones anteriores ya había experimentado por completo la vida de los mortales ascendentes del tiempo y del espacio, desde la estancia en la sede de la constelación hasta el servicio en la sede del superuniverso, y a través de dicho servicio. Precisamente gracias a estas experiencias morontiales, el Hijo Creador de Nebadon acabó realmente y terminó de manera aceptable su séptima y última donación en el universo.

\section*{4. La décima aparición (en Filadelfia)}
\par 
%\textsuperscript{(2041.4)}
\textsuperscript{191:4.1} La décima manifestación morontial de Jesús reconocida por los mortales tuvo lugar el martes 11 de abril, poco después de las ocho, en Filadelfia, donde se mostró a Abner, Lázaro y a unos ciento cincuenta de sus compañeros, incluídos más de cincuenta miembros del cuerpo evangélico de los setenta. Esta aparición se produjo en la sinagoga, poco después de la apertura de una reunión especial convocada por Abner para discutir la crucifixión de Jesús y la noticia más reciente de la resurrección, aportada por un mensajero de David. Puesto que el Lázaro resucitado ahora era miembro de este grupo de creyentes, no les resultaba difícil creer en la noticia de que Jesús había resucitado de entre los muertos.

\par 
%\textsuperscript{(2041.5)}
\textsuperscript{191:4.2} Abner y Lázaro, que estaban juntos en el púlpito, acababan de abrir la sesión en la sinagoga cuando toda la audiencia de creyentes vio aparecer repentinamente la forma del Maestro. Avanzó unos pasos desde donde había aparecido entre Abner y Lázaro, ninguno de los cuales lo había visto, saludó al grupo y dijo:

\par 
%\textsuperscript{(2041.6)}
\textsuperscript{191:4.3} «Que la paz sea con vosotros. Todos sabéis que tenemos un solo Padre en el cielo y que sólo hay un evangelio del reino ---la buena nueva del don de la vida eterna que los hombres reciben por la fe. Mientras os regocijáis en vuestra lealtad al evangelio, rogad al Padre de la verdad que derrame en vuestro corazón un amor nuevo y más grande por vuestros hermanos. Debéis amar a todos los hombres como yo os he amado; debéis servir a todos los hombres como yo os he servido. Con una simpatía comprensiva y con un afecto fraternal, aceptad como compañeros a todos vuestros hermanos que se dedican a la proclamación de la buena nueva, ya sean judíos o gentiles, griegos o romanos, persas o etíopes. Juan proclamó el reino por adelantado; vosotros habéis predicado el evangelio con autoridad; los griegos enseñan ya la buena nueva; y yo voy a enviar pronto el Espíritu de la Verdad al alma de todos estos hermanos míos, que han dedicado su vida tan generosamente a iluminar a sus semejantes que están en las tinieblas espirituales. Todos sois los hijos de la luz; no tropecéis pues en los enredos de los malentendidos causados por la desconfianza y la intolerancia humana. Si la gracia de la fe os ennoblece para amar a los incrédulos, ¿no deberíais amar igualmente a aquellos que son vuestros compañeros creyentes en la gran familia de la fe? Recordad, en la medida en que os améis los unos a los otros, todos los hombres sabrán que sois mis discípulos».

\par 
%\textsuperscript{(2042.1)}
\textsuperscript{191:4.4} «Id pues a proclamar por todo el mundo, a todas las naciones y razas, este evangelio de la paternidad de Dios y de la fraternidad de los hombres, y sed siempre sabios en la elección de vuestros métodos para presentar la buena nueva a las diferentes razas y tribus de la humanidad. Habéis recibido gratuitamente este evangelio del reino, y aportaréis gratuitamente la buena nueva a todas las naciones. No temáis la resistencia del mal porque siempre estoy con vosotros, incluso hasta el fin de los tiempos. Mi paz os dejo».

\par 
%\textsuperscript{(2042.2)}
\textsuperscript{191:4.5} Después de haber dicho «Mi paz os dejo», desapareció de su vista. A excepción de una de sus apariciones en Galilea, donde más de quinientos creyentes lo vieron al mismo tiempo, este grupo de Filadelfia contenía la mayor cantidad de mortales que lo hubiera visto en una misma ocasión.

\par 
%\textsuperscript{(2042.3)}
\textsuperscript{191:4.6} A la mañana siguiente temprano, mientras los apóstoles permanecían en Jerusalén esperando que Tomás se recuperara emocionalmente, estos creyentes de Filadelfia salieron a proclamar que Jesús de Nazaret había resucitado de entre los muertos.

\par 
%\textsuperscript{(2042.4)}
\textsuperscript{191:4.7} El día siguiente, miércoles, Jesús lo pasó sin interrupción en compañía de sus asociados morontiales, y a media tarde recibió la visita de unos delegados morontiales procedentes de los mundos de las mansiones de todos los sistemas locales de esferas habitadas de toda la constelación de Norlatiadek. Y todos se regocijaron en reconocer a su Creador como miembro de su propia orden de inteligencias universales.

\section*{5. La segunda aparición a los apóstoles}
\par 
%\textsuperscript{(2042.5)}
\textsuperscript{191:5.1} Tomás pasó una triste semana completamente solo en las colinas que rodeaban al Olivete. Durante este tiempo sólo vio a Juan Marcos y a los que vivían en la casa de Simón. Eran alrededor de las nueve del sábado 15 de abril cuando los dos apóstoles lo encontraron y se lo llevaron de vuelta a su refugio en la casa de Marcos. Al día siguiente, Tomás escuchó el relato de las historias de las diversas apariciones del Maestro, pero se negó rotundamente a creer. Sostenía que Pedro, con su entusiasmo, los había convencido de que habían visto al Maestro. Natanael razonó con él, pero no sirvió de nada. Había una obstinación emotiva asociada a sus dudas habituales, y este estado mental, unido a su disgusto por haber huido de ellos, concurría a crear una situación de aislamiento que ni siquiera el mismo Tomás comprendía plenamente. Se había apartado de sus compañeros, había seguido su propio camino, y ahora, incluso estando de vuelta entre ellos, tendía inconscientemente a adoptar una actitud de desacuerdo. Era lento en rendirse; no le gustaba ceder. Aunque no tuviera esa intención, disfrutaba realmente con la atención que le prestaban; los esfuerzos de todos sus compañeros por convencerlo y convertirlo le producían una satisfacción inconsciente. Los había echado de menos durante toda una semana, y sus atenciones permanentes le causaban un gran placer.

\par 
%\textsuperscript{(2042.6)}
\textsuperscript{191:5.2} Poco después de las seis, estaban tomando la cena con Tomás sentado entre Pedro y Natanael, cuando el incrédulo apóstol dijo: «No creeré hasta que haya visto al Maestro con mis propios ojos y haya metido mi dedo en la marca de los clavos». Mientras estaban así sentados cenando, con las puertas fuertemente cerradas y atrancadas, el Maestro morontial apareció repentinamente dentro de la curvatura de la mesa, y permaneciendo directamente delante de Tomás, dijo:

\par 
%\textsuperscript{(2043.1)}
\textsuperscript{191:5.3} «Que la paz sea con vosotros. He esperado toda una semana a fin de poder aparecer de nuevo cuando todos estuvierais presentes para escuchar una vez más el encargo de ir a predicar por todo el mundo este evangelio del reino. Os lo digo de nuevo: Del mismo modo que el Padre me ha enviado al mundo, así os envío yo. Al igual que yo he revelado al Padre, vosotros revelaréis el amor divino, no solamente con las palabras, sino en vuestra vida diaria. Os envío, no para que améis el alma de los hombres, sino más bien para que \textit{améis a los hombres}. No debéis proclamar simplemente las alegrías del cielo, sino que debéis manifestar también en vuestra experiencia diaria estas realidades espirituales de la vida divina, puesto que gracias a la fe ya tenéis la vida eterna como un don de Dios. Cuando tengáis fe, cuando el poder de las alturas, el Espíritu de la Verdad, haya venido a vosotros, no esconderéis vuestra luz aquí detrás de unas puertas cerradas; haréis conocer a toda la humanidad el amor y la misericordia de Dios. Ahora huís, por miedo, de los hechos de una experiencia desagradable, pero cuando hayáis sido bautizados con el Espíritu de la Verdad, saldréis con valentía y alegría al encuentro de las nuevas experiencias que viviréis al proclamar la buena nueva de la vida eterna en el reino de Dios. Podéis permanecer aquí y en Galilea durante un corto período mientras os recobráis del impacto de la transición entre la falsa seguridad de la autoridad del tradicionalismo y el nuevo orden de la autoridad de los hechos, de la verdad y de la fe en las realidades supremas de la experiencia viviente. Vuestra misión en el mundo está basada en el hecho de que he vivido entre vosotros una vida revelando a Dios, está basada en la verdad de que vosotros y todos los demás hombres sois los hijos de Dios; y esta misión consistirá en la vida que viviréis entre los hombres ---en la experiencia real y viviente de amar y servir a los hombres como yo os he amado y servido. Que la fe revele vuestra luz al mundo; que la revelación de la verdad abra los ojos cegados por la tradición; que vuestro servicio amoroso destruya eficazmente los prejuicios engendrados por la ignorancia. Acercándoos así a vuestros semejantes con una simpatía comprensiva y con una dedicación desinteresada, los conduciréis al conocimiento salvador del amor del Padre. Los judíos han ensalzado la bondad; los griegos han exaltado la belleza; los hindúes predican la devoción; los lejanos ascetas enseñan la veneración; los romanos exigen la lealtad; pero yo exijo la vida de mis discípulos, incluso una vida de servicio amoroso para vuestros hermanos en la carne».

\par 
%\textsuperscript{(2043.2)}
\textsuperscript{191:5.4} Después de haber hablado así, el Maestro bajó la mirada hacia el rostro de Tomás y dijo: «Y tú, Tomás, que has dicho que no creerías hasta que pudieras verme y meter tu dedo en las marcas de los clavos de mis manos, ahora me has contemplado y escuchado mis palabras; y aunque no veas ninguna marca de clavos en mis manos, puesto que he resucitado con una forma que tú también tendrás cuando te vayas de este mundo, ¿qué vas a decir a tus hermanos? Reconocerás la verdad, porque ya habías empezado a creer en tu corazón incluso cuando afirmabas tan categóricamente tu incredulidad. Tus dudas, Tomás, siempre se afirman con la mayor obstinación cuando están a punto de desmoronarse. Tomás, te ruego que no seas escéptico sino creyente ---y sé que creerás, incluso de todo corazón».

\par 
%\textsuperscript{(2043.3)}
\textsuperscript{191:5.5} Cuando Tomás escuchó estas palabras, cayó de rodillas delante del Maestro morontial y exclamó: «¡Creo! ¡Señor mío y Maestro mío!» Entonces Jesús le dijo a Tomás: «Has creído, Tomás, porque me has visto y escuchado realmente. Benditos sean, en los siglos venideros, aquellos que creerán sin haber visto siquiera con los ojos de la carne ni haber escuchado con los oídos mortales».

\par 
%\textsuperscript{(2043.4)}
\textsuperscript{191:5.6} Luego, mientras la forma del Maestro se acercaba al extremo de la mesa, se dirigió a todos ellos diciendo: «Y ahora, id todos a Galilea, donde pronto me apareceré a vosotros». Después de decir esto, desapareció de su vista.

\par 
%\textsuperscript{(2044.1)}
\textsuperscript{191:5.7} Los once apóstoles estaban ahora plenamente convencidos de que Jesús había resucitado de entre los muertos, y a la mañana siguiente muy temprano, antes del amanecer, partieron para Galilea.

\section*{6. La aparición en Alejandría}
\par 
%\textsuperscript{(2044.2)}
\textsuperscript{191:6.1} Mientras los once apóstoles iban camino de Galilea acercándose al final de su viaje, el martes 18 de abril hacia las ocho y media de la noche Jesús se apareció a Rodán y a unos ochenta creyentes más en Alejandría. Ésta era la duodécima aparición del Maestro en forma morontial. Jesús apareció ante estos griegos y judíos en el momento en que un mensajero de David terminaba su informe sobre la crucifixión. Este mensajero era el quinto corredor de relevo entre Jerusalén y Alejandría, y había llegado a Alejandría a últimas horas de aquella tarde; cuando hubo entregado su mensaje a Rodán, se decidió convocar a los creyentes para que recibieran esta trágica noticia de los labios mismos del mensajero. Alrededor de las ocho, el mensajero Natán de Busiris se presentó ante este grupo y les contó con detalle todo lo que el corredor anterior le había dicho a él. Natán terminó su conmovedor relato con estas palabras: «Pero David, que nos envía esta noticia, informa que el Maestro, en el momento de predecir su muerte, declaró que resucitaría de nuevo». Mientras Natán hablaba todavía, el Maestro morontial apareció allí a la vista de todos. Y cuando Natán se sentó, Jesús dijo:

\par 
%\textsuperscript{(2044.3)}
\textsuperscript{191:6.2} «Que la paz sea con vosotros. Lo que mi Padre me envió a establecer en el mundo no pertenece ni a una raza, ni a una nación, ni a un grupo especial de educadores o de predicadores. Este evangelio del reino pertenece tanto a los judíos como a los gentiles, a los ricos y a los pobres, a los libres y a los esclavos, a los hombres y a las mujeres, e incluso a los niños pequeños. Todos debéis proclamar este evangelio de amor y de verdad mediante la vida que vivís en la carne. Os amaréis los unos a los otros con un afecto nuevo y sorprendente, tal como yo os he amado. Serviréis a la humanidad con una devoción nueva y extraordinaria, tal como yo os he servido. Cuando los hombres vean que los amáis así, y cuando observen el fervor con que los servís, percibirán que sois hermanos por la fe en el reino de los cielos, y seguirán al Espíritu de la Verdad que verán en vuestra vida, hasta que encuentren la salvación eterna».

\par 
%\textsuperscript{(2044.4)}
\textsuperscript{191:6.3} «Al igual que el Padre me ha enviado a este mundo, yo os envío a vosotros. Todos estáis llamados a llevar la buena nueva a aquellos que están en las tinieblas. Este evangelio del reino pertenece a todos los que crean en él; no será confiado al cuidado exclusivo de los sacerdotes. El Espíritu de la Verdad vendrá pronto a vosotros, y os conducirá a toda la verdad. Id pues por el mundo entero predicando este evangelio, y pensad que siempre estoy con vosotros, incluso hasta el fin de los tiempos».

\par 
%\textsuperscript{(2044.5)}
\textsuperscript{191:6.4} Después de haber hablado así, el Maestro desapareció de su vista. Estos creyentes permanecieron allí juntos toda la noche, contando sus experiencias como creyentes en el reino y escuchando las numerosas palabras de Rodán y de sus asociados. Y todos creyeron que Jesús había resucitado de entre los muertos. Un mensajero de David llegó dos días después para anunciarles la resurrección, e imaginad su sorpresa cuando respondieron a su anuncio: «Sí, ya lo sabemos, porque le hemos visto. Anteayer se apareció a nosotros».