\chapter{Documento 193. Las apariciones finales y la ascensión}
\par
%\textsuperscript{(2052.1)}
\textsuperscript{193:0.1} LA DECIMOSEXTA manifestación morontial de Jesús tuvo lugar el viernes 5 de mayo, hacia las nueve de la noche, en el patio de Nicodemo. Esta noche, los creyentes de Jerusalén habían realizado su primer intento por reunirse después de la resurrección. En este momento se encontraban congregados aquí los once apóstoles, el cuerpo de mujeres y sus asociadas, y aproximadamente otros cincuenta discípulos principales del Maestro, incluyendo a varios griegos. Este grupo de creyentes había estado conversando familiarmente durante más de media hora cuando de pronto, el Maestro morontial apareció plenamente a la vista de todos y empezó de inmediato a instruirlos. Jesús dijo:

\par
%\textsuperscript{(2052.2)}
\textsuperscript{193:0.2} «Que la paz sea con vosotros. Éste es el grupo de creyentes más representativo ---apóstoles y discípulos\footnote{\textit{Discurso de Jesús a los líderes}: Lc 24:44-48.}, tanto hombres como mujeres--- al que me he aparecido desde el momento en que fui liberado de la carne. Ahora os tomo por testigos de que os había dicho de antemano que mi estancia entre vosotros debía llegar a su fin. Os dije que pronto debía regresar hacia el Padre\footnote{\textit{Os predije que debía regresar al Padre}: Jn 7:33-34; 14:12b,28; 16:5,10,16,28; 20:17.}. Y luego os dije claramente de qué manera los jefes de los sacerdotes y los dirigentes de los judíos me entregarían para ser ejecutado, y que saldría de la tumba\footnote{\textit{Predicción de la muerte y resurrección}: Mt 16:21; 17:22-23a; 20:17-19; 27:63; Mc 8:31; 9:31; 10:32-34; Lc 9:22,31,43b-44; 18:31-33; 24:7,46; Jn 14:28a; 20:9.}. ¿Por qué, entonces, os habéis desconcertado tanto por todo esto, cuando ha sucedido? ¿Y por qué estabais tan sorprendidos cuando resucité de la tumba al tercer día? No lograsteis creerme porque escuchabais mis palabras sin comprender su significado».

\par
%\textsuperscript{(2052.3)}
\textsuperscript{193:0.3} «Ahora deberíais prestar oído a mis palabras para no cometer de nuevo el error de escuchar mi enseñanza con la mente, sin comprender su significado en vuestro corazón. Desde el principio de mi estancia aquí como uno de vosotros, os enseñé que mi única finalidad era revelar mi Padre que está en los cielos\footnote{\textit{Revelar a mi Padre que está en los cielos}: Mt 5:16,45,48; 6:1,9,14; 6:26,32; 7:11,21; 10:32-33; 11:25; 12:50; 15:13; 16:17; 18:10,14,19,35; 23:9; Mc 11:25-26; Lc 10:21; 11:2,13.} a sus hijos de la Tierra. He vivido la donación de revelar a Dios para que podáis experimentar la carrera de conocer a Dios. He revelado a Dios como vuestro Padre que está en los cielos; os he revelado que sois los hijos de Dios en la Tierra\footnote{\textit{Sois los hijos de Dios en la Tierra}: 1 Cr 22:10; Sal 2:7; Is 56:5; Mt 5:9,16,45; Lc 20:36; Jn 1:12-13; 11:52; Hch 17:28-29; Ro 8:14-17,19,21; 9:26; 2 Co 6:18; Gl 3:26; 4:5-7; Ef 1:5; Flp 2:15; Heb 12:5-8; 1 Jn 3:1-2,10; 5:2; Ap 21:7; 2 Sam 7:14.}. Es un hecho que Dios os ama a vosotros, sus hijos. Por la fe en mis palabras, este hecho se vuelve una verdad eterna y viviente en vuestro corazón. Cuando, por la fe viviente, os volvéis divinamente conscientes de Dios, entonces nacéis del espíritu\footnote{\textit{Nacidos del espíritu}: Jn 3:3-8.} como hijos de la luz\footnote{\textit{Hijos de la luz}: Lc 16:8; Jn 12:36; Ef 5:8; 1 Ts 5:5.} y de la vida, de la misma vida eterna con la que ascenderéis el universo de universos y lograréis la experiencia de encontrar a Dios Padre en el Paraíso».

\par
%\textsuperscript{(2052.4)}
\textsuperscript{193:0.4} «Os exhorto a que recordéis siempre que vuestra misión entre los hombres consiste en proclamar el evangelio del reino ---la realidad de la paternidad de Dios y la verdad de la filiación de los hombres. Proclamad la verdad total de la buena nueva, y no solamente una parte del evangelio salvador. Vuestro mensaje no ha cambiado debido a la experiencia de mi resurrección. La filiación con Dios, por la fe, sigue siendo la verdad salvadora del evangelio del reino. Debéis salir a predicar el amor de Dios\footnote{\textit{Salir a predicar el amor de Dios}: Mt 24:14; Mt 28:19-20a; Mc 13:10; Mc 16:15; Lc 24:47; Jn 17:18; Hch 1:8b.} y el servicio a los hombres. Lo que el mundo más necesita saber es que los hombres son hijos de Dios, y que pueden comprender realmente por la fe esta verdad ennoblecedora, y experimentarla diariamente. Mi donación debería ayudar a todos los hombres a saber que son hijos de Dios, pero este conocimiento será insuficiente si no logran captar personalmente, por la fe, la verdad salvadora de que son los hijos espirituales vivientes del Padre eterno. El evangelio del reino\footnote{\textit{Evangelio del reino}: Mt 3:2; 4:17,23; 5:3,10,19-20; 6:33; 7:21; 8:11; 9:35; 10:7; 11:11-12; 12:28; 13:11,24,31-52; 16:19; 18:1-4,23; 19:14,23-24; 20:1; 21:31,43; 22:2; 23:13; 24:14; 25:1,14; Mc 1:14-15; 4:11,26,30; 9:1:47; 10:14-15,23-25; 12:34; 14:25; 15:43; Lc 4:43; 6:20; 7:28; 8:1,10; 9:2,11,27; 9:60,62; 10:9-11; 11:20; 12:31-32; 13:18,20,28,29; 14:15; 16:16; 17:20-21; 18:16-17,24-25; 19:11; 21:31; 22:16,18; 23:51; Jn 3:3,5; Ro 14:17; 1 Co 4:20; 6:10.} se ocupa del amor del Padre y del servicio a sus hijos en la Tierra».

\par
%\textsuperscript{(2053.1)}
\textsuperscript{193:0.5} «Aquí compartís entre vosotros el conocimiento de que he resucitado de entre los muertos, pero esto no es algo extraordinario. Tengo el poder de abandonar mi vida y de recuperarla de nuevo; el Padre confiere este poder a sus Hijos Paradisiacos\footnote{\textit{La vida se sobrepone a la muerte}: Jn 10:17-18.}. Vuestro corazón debería conmoverse más bien con el conocimiento de que los muertos de una era\footnote{\textit{Resurrección dispensacional}: Mt 27:52-53; Jn 5:25-29.} han emprendido la ascensión eterna poco después de que yo saliera de la tumba nueva de José. He vivido mi vida en la carne para mostraros cómo podéis ser, a través del servicio amoroso, una revelación de Dios para vuestros semejantes, al igual que yo he sido, amándoos y sirviéndoos\footnote{\textit{Ama como Jesús te ama}: Jn 13:34-35; Jn 15:12.}, una revelación de Dios para vosotros. He vivido entre vosotros como el Hijo del Hombre para que vosotros, y todos los demás hombres, podáis saber que todos sois en verdad los hijos de Dios\footnote{\textit{Somos los hijos de Dios en la Tierra}: 1 Cr 22:10; Sal 2:7; Is 56:5; Mt 5:9,16,45; Lc 20:36; Jn 1:12-13; 11:52; Hch 17:28-29; Ro 8:14-17,19,21; 9:26; 2 Co 6:18; Gl 3:26; 4:5-7; Ef 1:5; Flp 2:15; Heb 12:5-8; 1 Jn 3:1-2,10; 5:2; Ap 21:7; 2 Sam 7:14.}. Por eso, id ahora por el mundo entero predicando este evangelio\footnote{\textit{Id por todo el mundo predicando este evangelio}: Mt 24:14; 28:19-20a; Mc 13:10; 16:15; Lc 24:47; Jn 17:18; Hch 1:8b.} del reino de los cielos a todos los hombres. Amad a todos los hombres como yo os he amado; servid a vuestros compañeros mortales como yo os he servido. Habéis recibido gratuitamente, dad gratuitamente\footnote{\textit{Habéis recibido gratuitamente, dad gratuitamente}: Mt 10:8.}. Permaneced aquí en Jerusalén\footnote{\textit{Permaneced en Jerusalén}: Lc 24:49; Hch 1:4.} solamente mientras voy hacia el Padre y hasta que os envíe el Espíritu de la Verdad. Él os guiará hacia una verdad más amplia, y yo iré con vosotros por todo el mundo. Siempre estoy con vosotros\footnote{\textit{Estaré siempre con vosotros}: Mt 28:20.}, y mi paz os dejo».

\par
%\textsuperscript{(2053.2)}
\textsuperscript{193:0.6} Cuando el Maestro les hubo hablado, desapareció de su vista. Estos creyentes no se dispersaron hasta cerca del alba; permanecieron juntos toda la noche discutiendo seriamente las recomendaciones del Maestro y meditando sobre todo lo que les había sucedido. Santiago Zebedeo y otros apóstoles les contaron también sus experiencias con el Maestro morontial en Galilea, y refirieron cómo se les había aparecido tres veces.

\section*{1. La aparición en Sicar}
\par
%\textsuperscript{(2053.3)}
\textsuperscript{193:1.1} El sábado 13 de mayo hacia las cuatro de la tarde, el Maestro se apareció a Nalda y a unos setenta y cinco creyentes samaritanos cerca del pozo de Jacob, en Sicar. Los creyentes tenían la costumbre de reunirse en este lugar, cerca del cual Jesús le había hablado a Nalda sobre el agua de la vida. Este día, justo en el momento en que habían terminado de discutir la noticia de la resurrección, Jesús apareció repentinamente delante de ellos, diciendo:

\par
%\textsuperscript{(2053.4)}
\textsuperscript{193:1.2} «Que la paz sea con vosotros. Os alegráis de saber que yo soy la resurrección y la vida\footnote{\textit{Soy la resurrección y la vida}: Jn 11:25.}, pero esto no os servirá de nada si no nacéis primero del espíritu eterno\footnote{\textit{Nacer del espíritu}: Jn 3:3-7.}, llegando a poseer así, por la fe, el don de la vida eterna\footnote{\textit{La vida eterna por la fe}: Jn 3:15-16.}. Si sois los hijos de mi Padre por la fe\footnote{\textit{Todos los hombres son hijos de Dios}: 1 Cr 22:10; Sal 2:7; Is 56:5; Mt 5:9,16,45; Lc 20:36; Jn 1:12-13; 11:52; Hch 17:28-29; Ro 8:14-17,19,21; 9:26; 2 Co 6:18; Gl 3:26; 4:5-7; Ef 1:5; Flp 2:15; Heb 12:5-8; 1 Jn 3:1-2,10; 5:2; Ap 21:7; 2 Sam 7:14.}, no moriréis nunca, no pereceréis\footnote{\textit{Los hijos de la fe no morirán nunca}: Dn 12:2; Mt 19:16,29; 25:46; Mc 10:17,30; Lc 10:25; 18:18,30; Jn 3:15-16,36; 4:14,36; 5:24,39; 6:27,40,47; 6:54,68; 8:51-52; 10:28; 11:25-26; 12:25,50; 17:2-3; Hch 13:46-48; Ro 2:7; 5:21; 6:22-23; Gl 6:8; 1 Ti 1:16; 6:12,19; Tit 1:2; 3:7; 1 Jn 1:2; 2:25; 3:15; 5:11,13,20; Jud 1:21; Ap 22:5.}. El evangelio del reino os ha enseñado que todos los hombres son hijos de Dios. Y esta buena nueva relativa al amor del Padre celestial por sus hijos de la Tierra debe ser llevada por el mundo entero\footnote{\textit{Enseñad el amor de Dios al mundo}: Mt 24:14; 28:19-20a; Mc 13:10; 16:15; Lc 24:47; Jn 17:18; Hch 1:8b.}. Ha llegado la hora en que no adoraréis a Dios ni en Gerizim ni en Jerusalén, sino allí donde estéis, tal como estéis, en espíritu y en verdad\footnote{\textit{Adoraréis en espíritu y en verdad}: Jn 4:23-24.}. Vuestra fe es la que salva vuestra alma. La salvación es el don de Dios para todos los que creen que son sus hijos. Pero no os engañéis; aunque la salvación es el don gratuito de Dios y se concede a todos los que la aceptan por la fe\footnote{\textit{La fe salva tu alma}: Mc 16:16.}, a ello le sigue la experiencia de producir los frutos de la vida espiritual\footnote{\textit{Los frutos del espíritu}: Gl 5:22-23; Ef 5:9.} tal como ésta se vive en la carne. La aceptación de la doctrina de la paternidad de Dios implica que también aceptáis libremente la verdad asociada de la fraternidad de los hombres. Si el hombre es vuestro hermano, es aún más que vuestro prójimo, a quien el Padre os pide que améis como a vosotros mismos. Como vuestro hermano pertenece a vuestra propia familia, no solamente lo amaréis con un afecto familiar, sino que también lo serviréis como os servís a vosotros mismos. Y amaréis y serviréis así a vuestro hermano porque vosotros, que sois mis hermanos, habéis sido amados y servidos por mí de esa manera. Id pues por todo el mundo contando esta buena nueva\footnote{\textit{Contad esta buena nueva al mundo}: Mt 24:14; 28:19-20a; Mc 13:10; 16:15; Lc 24:47; Jn 17:18; Hch 1:8b.} a todas las criaturas de todas las razas, tribus y naciones. Mi espíritu os precederá, y yo estaré siempre con vosotros»\footnote{\textit{Estaré siempre con vosotros}: Mt 28:20.}.

\par
%\textsuperscript{(2054.1)}
\textsuperscript{193:1.3} Estos samaritanos se quedaron enormemente asombrados con esta aparición del Maestro, y se apresuraron a ir a las ciudades y pueblos vecinos, donde difundieron la noticia de que habían visto a Jesús y que éste les había hablado. Ésta fue la decimoséptima aparición morontial del Maestro.

\section*{2. La aparición en Fenicia}
\par
%\textsuperscript{(2054.2)}
\textsuperscript{193:2.1} La decimoctava aparición morontial del Maestro tuvo lugar en Tiro, el martes 16 de mayo, poco antes de las nueve de la noche. Apareció, una vez más, al término de una reunión de creyentes, cuando estaban a punto de dispersarse, y dijo:

\par
%\textsuperscript{(2054.3)}
\textsuperscript{193:2.2} «Que la paz sea con vosotros. Os alegráis de saber que el Hijo del Hombre ha resucitado de entre los muertos porque sabéis así que vosotros y vuestros hermanos sobreviviréis también a la muerte física. Pero esta supervivencia depende de que hayáis nacido previamente del espíritu\footnote{\textit{Nacido del espíritu}: Jn 3:3-7.} que busca la verdad y encuentra a Dios. El pan y el agua de la vida\footnote{\textit{El pan de la vida}: Jn 6:33-35,48; 6:51,58. \textit{El agua de la vida}: Jn 7:38.} sólo se conceden a los que tienen hambre de la verdad y sed de rectitud ---de Dios\footnote{\textit{Hambrientos y sedientos (de Dios)}: Mt 5:6.}. El hecho de que los muertos resuciten no es el evangelio del reino. Estas grandes verdades y estos hechos universales están todos relacionados con este evangelio, en el sentido de que son una parte del resultado de creer en la buena nueva, y están contenidos en la experiencia posterior de aquellos que, por la fe, se convierten de hecho y en verdad en los hijos perpetuos del Dios eterno. Mi Padre me envió a este mundo\footnote{\textit{El Padre me envió}: Jn 5:23,30,36-37; 6:44,57; 8:16-18,29,42; 10:36; 12:49; 17:21,25; 20:21.} para proclamar a todos los hombres esta salvación de la filiación. Y yo os envío también en todas direcciones\footnote{\textit{La gran asignación}: Mt 24:14; 28:19-20a; Mc 13:10; 16:15; Lc 24:47; Jn 17:18; Hch 1:8b.} para que prediquéis esta salvación de la filiación. La salvación es un don gratuito de Dios, pero aquellos que nacen del espíritu empiezan a manifestar inmediatamente los frutos del espíritu\footnote{\textit{Los frutos del espíritu}: Gl 5:22-23; Ef 5:9.} en el servicio amoroso a sus semejantes. Y los frutos del espíritu divino, producidos en la vida de los mortales nacidos del espíritu y que conocen a Dios, son: servicio amoroso, consagración desinteresada, lealtad valiente, equidad sincera, honradez iluminada, esperanza imperecedera, confianza fiel, ministerio misericordioso, bondad inagotable, tolerancia indulgente y paz duradera. Si unos creyentes declarados no producen estos frutos del espíritu divino en sus vidas, están muertos\footnote{\textit{La fe sin frutos está muerta}: Stg 2:17.}; el Espíritu de la Verdad no está en ellos; son unas ramas inútiles de la vid viviente, y pronto serán cortadas. Mi Padre pide a los hijos de la fe que produzcan muchos frutos del espíritu\footnote{\textit{Se pide que produzcáis muchos frutos}: Mt 3:10; 7:16-20; 12:33; Lc 3:9; 6:43-44; 13:6-9; Jn 15:7-8,16.}. Por consiguiente, si no sois fecundos, él cavará alrededor de vuestras raíces y cortará vuestras ramas estériles. A medida que progreséis hacia el cielo en el reino de Dios, deberéis producir cada vez más los frutos del espíritu. Podéis entrar en el reino como un niño\footnote{\textit{Entrar en el reino como un niño}: Mt 18:2-5; Mt 19:13-14; Mc 9:36-37; Mc 10:13-15; Lc 9:46-48; Lc 18:16-17.}, pero el Padre exige que crezcáis\footnote{\textit{Crecer en el espíritu}: Ef 4:14-15; 1 P 2:2; 2 P 3:18.}, por la gracia, hasta la plena estatura de un adulto espiritual. Cuando salgáis por ahí a contarle a todas las naciones la buena nueva de este evangelio, iré delante de vosotros, y mi Espíritu de la Verdad residirá en vuestro corazón. Mi paz os dejo»\footnote{\textit{Mi paz os dejo}: Jn 14:27.}.

\par
%\textsuperscript{(2054.4)}
\textsuperscript{193:2.3} Entonces, el Maestro desapareció de su vista. Al día siguiente, los creyentes salieron de Tiro para llevar esta historia hasta Sidón e incluso hasta Antioquía y Damasco. Jesús había estado con estos creyentes cuando vivía en la carne, y lo reconocieron rápidamente en cuanto empezó a enseñarlos. Aunque sus amigos no podían reconocer fácilmente su forma morontial cuando ésta se hacía visible, no tardaban en reconocer su personalidad en cuanto les hablaba.

\section*{3. La última aparición en Jerusalén}
\par
%\textsuperscript{(2055.1)}
\textsuperscript{193:3.1} El jueves 18 de mayo por la mañana temprano, Jesús hizo su última aparición en la Tierra\footnote{\textit{Aparición final de Jesús}: Hch 1:4,6.} como personalidad morontial. Cuando los once apóstoles estaban a punto de sentarse para desayunar en la habitación superior de la casa de María Marcos, Jesús se les apareció y les dijo:

\par
%\textsuperscript{(2055.2)}
\textsuperscript{193:3.2} «Que la paz sea con vosotros. Os he pedido que os quedéis aquí en Jerusalén\footnote{\textit{Permanecer en Jerusalén}: Lc 24:49.} hasta que yo ascienda hacia el Padre\footnote{\textit{Voy al Padre}: Jn 7:33-34; 14:12b-28; 16:5,10,16,28; 20:17.}, e incluso hasta que os envíe el Espíritu de la Verdad\footnote{\textit{Falsa idea sobre el Espíritu de la Verdad}: Hch 1:6-8.}, que pronto será derramado sobre todo el género humano y que os dotará de un poder de las alturas». Simón Celotes interrumpió a Jesús para preguntarle: «Entonces, Maestro, ¿restablecerás el reino y veremos la gloria de Dios manifestada en la Tierra?» Cuando Jesús escuchó la pregunta de Simón, contestó: «Simón, continúas aferrado a tus viejas ideas sobre el Mesías judío y el reino material. Pero recibirás un poder espiritual cuando el espíritu haya descendido sobre vosotros, y pronto iréis por todo el mundo predicando este evangelio del reino. Al igual que el Padre me ha enviado al mundo, yo os envío a vosotros\footnote{\textit{Como el Padre me envió, así os envío}: Jn 17:18; Jn 20:21.}. Y deseo que os améis y confiéis los unos en los otros\footnote{\textit{Amaos unos a otros}: Ro 13:8-10; 1 Ts 4:9; 1 P 1:22; 1 Jn 3:11,23; 4:7,11-12; 2 Jn 1:5. \textit{Amarse unos a otros como hizo Jesús}: Jn 13:34-35; 15:12,17.}. Judas ya no está con vosotros porque su amor se enfrió y porque se negó a confiar en vosotros, sus leales hermanos. ¿No habéis leído en las Escrituras el pasaje que dice: `No es bueno que el hombre esté solo\footnote{\textit{No es bueno que el hombre esté solo}: Gn 2:18a.}. Nadie vive para sí mismo'?\footnote{\textit{Nadie vive para sí mismo}: Ro 14:7.} ¿Y también donde dice: `El que quiera tener amigos debe mostrarse amistoso'?\footnote{\textit{Si quieres amigos sé amistoso}: Pr 18:24.} ¿Y no os envié a enseñar de dos en dos\footnote{\textit{Enviados de dos en dos}: Mc 6:7; Lc 10:1.} para que no os sintierais solos y no cayerais en los perjuicios y las desgracias del aislamiento? También sabéis muy bien que, cuando vivía en la carne, nunca me permití estar solo durante mucho tiempo. Desde el principio mismo de nuestra asociación, siempre tuve a dos o tres de vosotros constantemente a mi lado o muy cerca de mí, incluso cuando comulgaba con el Padre. Confiad, pues, y tened confianza los unos en los otros. Esto es tanto más necesario cuanto que en el día de hoy voy a dejaros solos en el mundo. Ha llegado la hora; estoy a punto de ir hacia el Padre».

\par
%\textsuperscript{(2055.3)}
\textsuperscript{193:3.3} Cuando terminó de hablar, les hizo señas para que lo acompañaran y los condujo hasta el Monte de los Olivos, donde se despidió de ellos antes de partir de Urantia. Este recorrido hasta el Olivete fue solemne. Ninguno dijo ni una palabra desde el momento en que salieron de la habitación de arriba hasta que Jesús se detuvo con ellos en el Monte de los Olivos.

\section*{4. Las causas de la caída de Judas}
\par
%\textsuperscript{(2055.4)}
\textsuperscript{193:4.1} En la primera parte de su mensaje de despedida a sus apóstoles, el Maestro aludió a la pérdida de Judas y resaltó el trágico destino de su compañero de trabajo traidor como una advertencia solemne contra los peligros del aislamiento social y fraternal. Quizás sea útil para los creyentes de este siglo y de los siglos futuros, analizar brevemente las causas de la caída de Judas a la luz de las observaciones del Maestro y en vista de las aclaraciones acumuladas de los siglos posteriores.

\par
%\textsuperscript{(2055.5)}
\textsuperscript{193:4.2} Cuando recordamos esta tragedia, pensamos que Judas se desvió, principalmente, porque era una personalidad solitaria muy notoria, una personalidad cerrada y alejada de los contactos sociales corrientes. Se negó insistentemente a confiar en sus compañeros apóstoles, o a fraternizar libremente con ellos. Pero el hecho de ser una personalidad de tipo solitario, en sí mismo y por sí mismo, no le hubiera causado tanto daño a Judas si no hubiera sido porque tampoco logró acrecentar su amor ni crecer en gracia espiritual. Y además, para empeorar más las cosas, guardó rencores persistentes y alimentó enemigos psicológicos tales como la venganza y el ansia generalizada de «desquitarse» de alguien por todas sus decepciones.

\par
%\textsuperscript{(2056.1)}
\textsuperscript{193:4.3} Esta desdichada combinación de peculiaridades individuales y de tendencias mentales se conjugó para destruir a un hombre bien intencionado que no logró subyugar estos males por medio del amor, la fe y la confianza. El hecho de que Judas no tenía necesidad de ir por mal camino está bien demostrado en los casos de Tomás y de Natanael, los cuales estaban aquejados de este mismo tipo de desconfianza y tenían superdesarrolladas sus tendencias individualistas. Incluso Andrés y Mateo tenían muchas inclinaciones en este sentido; pero todos estos hombres experimentaron por Jesús y sus compañeros apóstoles un amor que iba creciendo con el tiempo, y no disminuyendo. Crecieron en la gracia y en el conocimiento\footnote{\textit{Los discípulos crecieron en gracia y en conocimiento}: 2 P 3:18.} de la verdad. Confiaron cada vez más en sus hermanos y desarrollaron lentamente la capacidad de fiarse de sus compañeros. Judas se negó insistentemente a fiarse de sus hermanos. Cuando la acumulación de sus conflictos emocionales le obligaba a buscar alivio en la expresión personal, buscaba invariablemente el consejo y recibía el consuelo poco sensato de sus parientes no espirituales o de aquellos que conocía por casualidad, que eran indiferentes o realmente hostiles al bienestar y al progreso de las realidades espirituales del reino celestial, del que Judas era uno de los doce embajadores consagrados en la Tierra.

\par
%\textsuperscript{(2056.2)}
\textsuperscript{193:4.4} Judas encontró la derrota en los combates de su lucha terrenal a causa de los factores siguientes relacionados con sus tendencias personales y sus debilidades de carácter:

\par
%\textsuperscript{(2056.3)}
\textsuperscript{193:4.5} 1. Era un ser humano de tipo solitario. Era sumamente individualista y eligió convertirse en una clase de persona firmemente «encerrada en sí misma» e insociable.

\par
%\textsuperscript{(2056.4)}
\textsuperscript{193:4.6} 2. Cuando era niño, le habían hecho la vida demasiado fácil. Se indignaba amargamente cuando le contrariaban. Siempre esperaba ganar; era muy mal perdedor.

\par
%\textsuperscript{(2056.5)}
\textsuperscript{193:4.7} 3. Nunca adquirió una técnica filosófica para enfrentarse con las decepciones. En lugar de aceptar las desilusiones como un aspecto normal y común de la existencia humana, recurría infaliblemente a la práctica de acusar a alguien en particular, o a sus compañeros como grupo, de todas sus dificultades y decepciones personales.

\par
%\textsuperscript{(2056.6)}
\textsuperscript{193:4.8} 4. Tendía a guardar rencor; alimentaba constantemente la idea de venganza.

\par
%\textsuperscript{(2056.7)}
\textsuperscript{193:4.9} 5. No le gustaba enfrentarse francamente a los hechos; era poco honrado en su actitud ante las situaciones de la vida.

\par
%\textsuperscript{(2056.8)}
\textsuperscript{193:4.10} 6. Detestaba discutir sus problemas personales con sus asociados inmediatos; se negaba a hablar de sus dificultades con sus verdaderos amigos y con aquellos que lo amaban realmente. En todos sus años de asociación con el Maestro, ni una sola vez se presentó ante él con un problema puramente personal.

\par
%\textsuperscript{(2056.9)}
\textsuperscript{193:4.11} 7. No aprendió nunca que, después de todo, las verdaderas recompensas de una noble vida consisten en premios espirituales, que no siempre se distribuyen durante esta corta y única vida en la carne.

\par
%\textsuperscript{(2056.10)}
\textsuperscript{193:4.12} A consecuencia del aislamiento persistente de su personalidad, sus penas se multiplicaron, sus aflicciones crecieron, sus ansiedades aumentaron y su desesperación alcanzó una profundidad casi insoportable.

\par
%\textsuperscript{(2057.1)}
\textsuperscript{193:4.13} Aunque este apóstol egocéntrico y ultraindividualista tenía muchos problemas psíquicos, emocionales y espirituales, sus dificultades principales eran las siguientes: Como personalidad, estaba aislado. Mentalmente, era desconfiado y vengativo. Por temperamento, era hosco y rencoroso. Emocionalmente, estaba desprovisto de amor y era incapaz de perdonar. Socialmente, no confiaba en nadie y estaba casi enteramente encerrado en sí mismo. En espíritu, se volvió arrogante y egoístamente ambicioso. En la vida, ignoró a los que le amaban, y en la muerte, no tuvo ningún amigo.

\par
%\textsuperscript{(2057.2)}
\textsuperscript{193:4.14} Éstos son, pues, los factores mentales y las influencias nocivas que, tomados en su conjunto, explican por qué un creyente en Jesús bien intencionado y por otra parte anteriormente sincero, incluso después de varios años de asociación íntima con la personalidad transformadora de Jesús, abandonó a sus compañeros, repudió una causa sagrada, renunció a su santa vocación y traicionó a su divino Maestro.

\section*{5. La ascensión del Maestro}
\par
%\textsuperscript{(2057.3)}
\textsuperscript{193:5.1} Eran casi las siete y media de la mañana de este jueves 18 de mayo cuando Jesús llegó a la ladera occidental del Monte Olivete con sus once apóstoles silenciosos y un poco desconcertados. Desde este lugar, situado a unos dos tercios de la subida hasta la cima\footnote{\textit{Ascensión a la montaña}: Lc 24:50a.}, podían contemplar Jerusalén y, debajo de ellos, Getsemaní. Jesús se preparó ahora para decir su último adiós a los apóstoles antes de despedirse de Urantia. Mientras estaba allí de pie delante de ellos, y sin que él lo pidiera, se arrodillaron en círculo a su alrededor, y el Maestro dijo:

\par
%\textsuperscript{(2057.4)}
\textsuperscript{193:5.2} «Os he pedido que permanezcáis en Jerusalén\footnote{\textit{Permanecer en Jerusalén}: Lc 24:49; Hch 1:4.} hasta que seáis dotados de un poder de las alturas. Ahora estoy a punto de despedirme de vosotros; estoy a punto de ascender hacia mi Padre, y pronto, muy pronto, enviaremos al Espíritu de la Verdad\footnote{\textit{Envío del Espíritu de la Verdad}: Ez 11:19; 18:31; 36:26-27; Jl 2:28-29; Lc 24:49; Jn 7:39; 14:16-18,23,26; 15:4,26; 16:7-8,13-14; 17:21-23; Hch 1:5,8a; 2:1-4,16-18; 2:33; 2 Co 13:5; Gl 2:20; 4:6; Ef 1:13; 4:30; 1 Jn 4:12-15.} a este mundo donde he residido; cuando haya venido, empezaréis la nueva proclamación del evangelio\footnote{\textit{Proclamación del nuevo evangelio}: Mt 24:14; 28:19-20a; Mc 13:10; 16:15; Lc 24:47; Jn 17:18; Hch 1:8b.} del reino, primero en Jerusalén, y luego hasta los lugares más alejados del mundo. Amad a los hombres con el amor con que yo os he amado, y servid a vuestros semejantes mortales como yo os he servido. Mediante los frutos espirituales de vuestra vida, impulsad a las almas a creer en la verdad de que el hombre es un hijo de Dios, y de que todos los hombres son hermanos. Recordad todo lo que os he enseñado y la vida que he vivido entre vosotros. Mi amor os cubre con su sombra, mi espíritu residirá con vosotros y mi paz permanecerá en vosotros. Adiós»\footnote{\textit{Deseos finales}: Lc 24:50b-51a.}.

\par
%\textsuperscript{(2057.5)}
\textsuperscript{193:5.3} Después de hablar así, el Maestro morontial desapareció de su vista\footnote{\textit{La ascensión}: Mc 16:19; Lc 24:51b; Hch 1:9.}. Esta supuesta ascensión de Jesús no se diferenció en nada de sus otras desapariciones de la visión humana durante los cuarenta días de su carrera morontial en Urantia.

\par
%\textsuperscript{(2057.6)}
\textsuperscript{193:5.4} El Maestro pasó por Jerusem para dirigirse a Edentia, donde los Altísimos, bajo la observación del Hijo Paradisiaco, liberaron a Jesús de Nazaret del estado morontial, y a través de los canales espirituales de ascensión, lo restituyeron al estado de filiación paradisiaca y de soberanía suprema en Salvington.

\par
%\textsuperscript{(2057.7)}
\textsuperscript{193:5.5} Eran aproximadamente las siete y cuarenta y cinco de esta mañana cuando el Jesús morontial desapareció del campo de observación de sus once apóstoles para empezar la ascensión hacia la diestra de su Padre, y recibir allí la confirmación oficial de su completa soberanía sobre el universo de Nebadon\footnote{\textit{Confirmación de la soberanía de Jesús}: Mc 16:19.}.

\section*{6. Pedro convoca una reunión}
\par
%\textsuperscript{(2057.8)}
\textsuperscript{193:6.1} Siguiendo las instrucciones de Pedro, Juan Marcos y otras personas salieron para convocar a los discípulos principales a una reunión en la casa de María Marcos\footnote{\textit{Primera reunión de los apóstoles}: Hch 1:12-14.}. A las diez y media, ciento veinte de los discípulos más destacados de Jesús que vivían en Jerusalén se habían congregado para escuchar el relato del mensaje de adiós del Maestro y para enterarse de su ascensión. María, la madre de Jesús, se encontraba en este grupo. Había regresado a Jerusalén con Juan Zebedeo cuando los apóstoles volvieron de su reciente estancia en Galilea. Poco después de Pentecostés, María regresó a la casa de Salomé en Betsaida. Santiago, el hermano de Jesús, también estaba presente en esta reunión, la primera conferencia de discípulos que se convocaba después de finalizar la carrera planetaria del Maestro.

\par
%\textsuperscript{(2058.1)}
\textsuperscript{193:6.2} Simón Pedro se encargó de hablar en nombre de sus compañeros apóstoles, e hizo un relato emocionante de la última reunión de los once con su Maestro; describió de la manera más conmovedora el adiós final del Maestro y su desaparición para emprender la ascensión. Nunca había tenido lugar en este mundo una reunión como ésta\footnote{\textit{La reunión}: Hch 1:15-23.}. Esta parte de la reunión duró poco menos de una hora. Pedro explicó entonces que habían decidido elegir a un sucesor de Judas Iscariote, y que se haría un descanso para permitir que los apóstoles decidieran entre los dos hombres que habían sido propuestos para esta función: Matías y Justo.

\par
%\textsuperscript{(2058.2)}
\textsuperscript{193:6.3} Los once apóstoles descendieron entonces al piso de abajo, donde acordaron echar a suertes a fin de determinar cuál de estos hombres se convertiría en apóstol para servir en el lugar de Judas. La suerte cayó sobre Matías, que fue proclamado nuevo apóstol. Fue debidamente instalado en su cargo, y luego nombrado tesorero. Pero Matías participó poco en las actividades posteriores de los apóstoles\footnote{\textit{La elección de Matías}: Hch 1:24-26.}.

\par
%\textsuperscript{(2058.3)}
\textsuperscript{193:6.4} Poco después de Pentecostés, los gemelos regresaron a sus casas en Galilea. Simón Celotes se retiró durante algún tiempo antes de salir a predicar el evangelio. Tomás estuvo preocupado durante un período de tiempo más corto, y luego reanudó su enseñanza. Natanael discrepó cada vez más con Pedro respecto a la cuestión de predicar acerca de Jesús, en lugar de proclamar el evangelio original del reino. A mediados del mes siguiente, este desacuerdo se volvió tan agudo que Natanael se retiró y se fue a Filadelfia para visitar a Abner y Lázaro. Después de permanecer allí durante más de un año, se dirigió hacia los países situados más allá de Mesopotamia, predicando el evangelio tal como él lo entendía.

\par
%\textsuperscript{(2058.4)}
\textsuperscript{193:6.5} De esta manera sólo quedaron seis apóstoles, de los doce originales, para actuar en el escenario de la proclamación inicial del evangelio en Jerusalén: Pedro, Andrés, Santiago, Juan, Felipe y Mateo.

\par
%\textsuperscript{(2058.5)}
\textsuperscript{193:6.6} Poco antes del mediodía, los apóstoles regresaron junto a sus hermanos en la habitación de arriba, y anunciaron que Matías había sido elegido como nuevo apóstol. Luego, Pedro invitó a todos los creyentes a ponerse en oración, a orar a fin de estar preparados para recibir el don del espíritu que el Maestro había prometido enviar.